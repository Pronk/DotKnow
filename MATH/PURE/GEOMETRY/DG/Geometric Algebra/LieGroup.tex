\documentclass[12pt]{article}
\usepackage{mathtools}
\usepackage{amsmath}
\usepackage{amsfonts}
\usepackage{amssymb}
\usepackage{ wasysym }
\usepackage{ stmaryrd }
\usepackage[dvipsnames]{xcolor}
\usepackage[top=20mm, bottom=20mm, left=30mm, right=10mm]{geometry}
%Markup
\newcommand{\TYPE}[1]{\textcolor{NavyBlue}{\mathtt{#1}}}
\newcommand{\FUNC}[1]{\textcolor{Cerulean}{\mathtt{#1}}}
\newcommand{\LOGIC}[1]{\textcolor{Blue}{\mathtt{#1}}}
\newcommand{\THM}[1]{\textcolor{Maroon}{\mathtt{#1}}}
%META
\renewcommand{\.}{\; . \;}
\newcommand{\de}{: \kern 0.1pc =}
\newcommand{\extract}{\rightarrowtriangle}
\newcommand{\where}{\LOGIC{where}}
\newcommand{\If}{\LOGIC{if} \;}
\newcommand{\Then}{ \; \LOGIC{then} \;}
\newcommand{\Else}{\; \LOGIC{else} \;}
%%STD
\newcommand{\Int}{\mathbb{Z} }
\newcommand{\NNInt}{\mathbb{Z}_{+} }
\newcommand{\Reals}{\mathbb{R} }
\newcommand{\Nat}{\mathbb{N} }
\DeclareMathOperator*{\centr}{center}
\DeclareMathOperator*{\argmin}{arg\,min}
\DeclareMathOperator*{\id}{id}
\DeclareMathOperator*{\im}{Im}
\newcommand{\EqClass}[1]{\TYPE{EqClass}\left( #1 \right)}
\newcommand{\Cate}{\TYPE{Category}}
\newcommand{\Func}[2]{\TYPE{Functor}\left( #1, #2 \right)}
\mathchardef\hyph="2D
\newcommand{\Surj}[2]{\TYPE{Surjective}\left( #1, #2 \right)}
%%ProofWritting
\newcommand{\A}{\LOGIC{Assume} \;} 
\newcommand{\As}{\; \LOGIC{as } \;} 
\newcommand{\E}{ \; \LOGIC{Extract} } 
%LinearAlgebra
%TYPES
\newcommand{\VS}[1]{\TYPE{VectorSpace}\left( #1 \right)}
\newcommand{\Lin}[1]{\mathcal{L}\left( #1 \right)}
\newcommand{\vs}[1]{\mathsf{VS}\left( #1 \right)}
\DeclareMathOperator*{\rank}{rank}
%FUNK
\DeclareMathOperator{\rk}{rank}
%Manifolds
%TYPES
\newcommand{\SManifold}{\TYPE{SmoothManifold}}
\newcommand{\SubMan}{\TYPE{SubManifold}}
\newcommand{\smooth}[1]{C^\infty\left( #1 \right)}
\newcommand{\CCurve}[2]{\TYPE{CentredCurve}\left(  #1, #2 \right)}
\newcommand{\Admis}{\TYPE{Admissible}}
\parindent=0em
%FUNC
\DeclareMathOperator{\cca}{chartCentredAt}
\DeclareMathOperator{\codim}{codim}
%TangentSpaces
%TYPES
\newcommand{\TS}{\TYPE{TangentSpace}}
\newcommand{\Germ}{\TYPE{Germ}}
\newcommand{\germs}[2]{\mathcal{F}_{#2}\left( #1 \right)}
\newcommand{\DirAt}{\TYPE{DerivationAtPoint}}
\newcommand{\dirAt}[2]{\mathrm{Dir}\left( #1 , #2 \right)}
\newcommand{\Phys}{\TYPE{PhysicalView}}
\newcommand{\Kin}{\TYPE{MovementDirection}}
\newcommand{\PManifold}{\TYPE{PointedManifolds}}
\newcommand{\PM}{\mathsf{PM}}
\newcommand{\TF}{\TYPE{TangentFunctor}}
\newcommand{\RegP}{\TYPE{RegularPoint}}
\newcommand{\CritP}{\TYPE{CriticalPoint}}
\newcommand{\RegV}{\TYPE{RegularValue}}
\newcommand{\CritV}{\TYPE{CriticalValue}}
\newcommand{\Zero}{\TYPE{Zero}}
\newcommand{\NonDeg}{\TYPE{NonDegenerate}}
\newcommand{\ConstRk}{\TYPE{ConstantRank}}
\newcommand{\Trans}{\TYPE{Transverse}}
%FUNC
\DeclareMathOperator{\alg}{alg}
\DeclareMathOperator{\fAlg}{fromAlg}
\DeclareMathOperator{\phys}{phys}
\DeclareMathOperator{\fPhys}{fromPhys}
\DeclareMathOperator{\kin}{kin}
\DeclareMathOperator{\fKin}{fromKin}
\newcommand{\tM}{\TYPE{TangentMap}}
\newcommand{\derT}{\TYPE{DerivationTransfer}}
\newcommand{\physS}{\TYPE{PhysicalShift}}
\newcommand{\kinT}{\TYPE{DirectionTransfer}}
\newcommand{\diff}{\TYPE{Differential}}
%THM
\newcommand{\Sard}{\THM{Sard}}
\newcommand{\MorseLemma}{\THM{MorseLemma}}
\newcommand{\LevelSubmanifold}{\THM{LevelSubmanifold}}
\newcommand{\TransThm}{\THM{Transversality}}
\newcommand{\TransPB}{\THM{TransversalPullbacks}}
\author{Uncultured Tramp} 
\title{LieGroups.Know}
\begin{document}
\maketitle
Prereqs: Manifolds, Group theory.
\newpage
\section{Lie Groups}
\subsection{Definitions}
\begin{flalign*} 
&\TYPE{LieGroup} :: \; ?\TYPE{Group} \& \TYPE{SManifold} \\
&G : \TYPE{LieGroup} \iff (\cdot_G) \in C^\infty(G \times G, G) \wedge  (\cdot)^{-1}_G  \in C^\infty(G,G)
\\ \\
&(\Reals^n,+), (\mathbb{C}^n,+), U(1) = (S^1,\cdot_{\mathbb{C}}) : \TYPE{LieGroup}  
\\ \\
&\mathrm{GL}(n,\Reals) :: \TYPE{LieGroup}, \\
& \mathrm{GL}(m,\Reals) \de \{ M \in \mathcal{M}^n(\Reals) : \det M \neq 0 \}
\\ \\
&\THM{DijointCosets} :: \forall G : \TYPE{Group}  \. \forall H : \TYPE{Subgroup}(G) \. G = \bigsqcup GH \\
&\LOGIC{Proof} = \\
& \A G :  \TYPE{Group}, \\
& \A H : \TYPE{Subgroup}(G), \\
&  \TYPE{Subgroup}(G)(H) \to e \in H \As (1), \\
& \A g \in G, \\
& (1) \to g \in gH; \\
& G = \bigcup GH \As (2), \\
& \A a,b \in G, \\
& \A  g \in aH \cap bH, \\
&  g \in aH \to  \exists x \in H : g = ax \E, \\
&  g \in bH \to  \exists x \in H : g = bx \E \As y, \\
& \A c \in aH, \\
& c \in bH \to  \exists x \in H : c = ax \E \As z, \\
&  by = g , ax = g  \to by = ax \to b =  axy^{-1} \in aH \to 
\\ & \kern 1pc \to  
c =  az = axy^{-1}yx^{-1}z = byx^{-1}z \in bH ; \\
& aH \subset bH \As (3),\\
& \LOGIC{SymmetricArgument}(3) \to  bH \subset aH \As (4), \\
& (3,4)  \to aH = bH ;; \\ 
& G = \bigsqcup GH ;;\square
\end{flalign*}
\newpage
\begin{flalign*} 
& \THM{OpenCosets} :: \forall G : \TYPE{LieGroup}\. \forall H : \TYPE{Open} \& \TYPE{Subgroup}(G)  
 \. g \in G \. gH : \TYPE{Open}(G) \\
 & \LOGIC{Proof} = \\
 & \A G : \TYPE{LieGroup}, \\
 & \A H : \TYPE{Open} \& \TYPE{Subgroup}(G),  \\
 & \A g \in G, \\
 &  \mu \de \Lambda x \in G \. g^{-1}x : C^\infty(G,G), \\
 & gH = \mu^{-1}(H) \to gH : \TYPE{Open}(G) ;;; \square
 \\ \\
&\THM{DisconnectedSubgroup} :: \forall G : \TYPE{LieGroup} \.   \forall H : \TYPE{Open} \& \TYPE{Subgroup}(G) \. H : \TYPE{Closed}(H) \\
& \LOGIC{Proof} = \\ 
 & \A G : \TYPE{LieGroup}, \\
 & \A H : \TYPE{Open} \& \TYPE{Subgroup}(G),  \\
 & g \in G , \\
 & \THM{OpenCosets}(G,H,g) \to gH : \TYPE{Open}(G) ; \\
 & \forall g \in G \. gH : \TYPE{Open}(G) \As (1), \\
 &  \THM{DijointCosets}(G,H) \to G = \bigsqcup GH \As (2), \\
 & (1,2) \to G \setminus H  : \TYPE{Open}(G) \to H : \TYPE{Closed}(G) ;; \square
\\ \\
& \THM{NeighbourhoodOfUnity} :: \forall G : \TYPE{LieGroup} \& 
\TYPE{Connected}  
\. \forall U \in \mathcal{U}(e_G) \. 
\FUNC{genGroup}(U) = G  \\
& \LOGIC{Proof} = \\
& \A G : \TYPE{LieGroup} \& \TYPE{Connected}  , \\
& \A U \in  \mathcal{U}(e), \\
&  V \de U \cap U^{-1}  : ?U \& \TYPE{Open}(G) , \\
&  U \in  \mathcal{U}(e_G) \to e \in U  \As (1), \\
& e^{-1} = e   \to_{(1)}  V \ne \emptyset, \\
&  V \subset U \to \FUNC{genGroup}(V) \subset \FUNC{genGroup}(U), \\
&  \A k \in \Nat, \\
&  S_k \de \If k == 1 \Then V \Else VS_{k-1} , \\
&  V : \TYPE{Open}(G), S_{k} : \TYPE{Open}(G) \to S_{k+1} = VS_{k} = \bigcup_{v \in V} vS_k : \TYPE{Open}(G) \\
& H \de \bigcup_{k \in \Nat} S_k  : \TYPE{Subgroup} \& \TYPE{Open} (G), \\
& \THM{DisconnectedSubgroup}(G,H) \to H : \TYPE{Closed}(G) \As (2), \\
& V \ne \emptyset \to H \ne \emptyset \As (3), \\ 
& G : \TYPE{Connected} \to_{(1,3)} H = G ,  \\
& H \subset    \FUNC{genGroup}(U) \subset G  \to   \FUNC{genGroup}(U) = G ;; \square
\end{flalign*}
\newpage
\begin{flalign*} 
& \TYPE{IdComponent} :: \prod G : \TYPE{LieGroup} \. ?\TYPE{CC}(G) \\
&  H : \TYPE{IdComponent} \iff e \in H 
\\ \\
& \TYPE{LieSubgroup} :: \prod G : \TYPE{LieGroup} \. ?\TYPE{Subgroup}(G) \\
& H : \TYPE{LieSubgroup} \iff  i_H : \TYPE{Immersion}(H,G)
\\ \\
& \THM{RegularLieSubgroup} :: \forall G : \TYPE{LieGroup} \. \forall H : \TYPE{Subgroup} \& \TYPE{Regular}(G) 
\. \\ & \kern 1pc \.
H : \TYPE{LieSubgroup} \& \TYPE{Closed}(H)\\
& \LOGIC{Proof} = \\
& \A  G : \TYPE{LieGroup}, \\
& \A  H : \TYPE{Subgroup} \& \TYPE{Regular}(G), \\
&  H : \TYPE{Regular}(G) \to i_H : \TYPE{Immersion} \to H : \TYPE{LieSubgroup}(M) \As (1), \\
&  (U,x) \de \TYPE{Regular}(G,H,e), \\
&  V  \de \TYPE{Separable3}(U,e) \to  e \in V \subset \overline V  \subset U, \\
&  \delta \de \Lambda (a,b) \in  G \times G  \. a^{-1}b : C^\infty(G\times G, G) \\
&  \Theta \de  \delta^{-1}(V) : \TYPE{Open}(G \times G) \to \exists O : \TYPE{Open}(G) : O \times O \subset \Theta : e \in O, \\
& \A X \in \overline H , \\
& X \in \overline H \to \exists x  \in \TYPE{ConvergentFrom}(G,H) : \lim_{n \to \infty} x_n = X \E , \\
& \lim_{n \to \infty}\lim_{k \to \infty}\delta(x_n,x_k) = \lim_{n \to \infty}\lim_{k \to \infty}  x_n^{-1}x_k = \Big(\lim_{n \to \infty}  x_n^{-1}\Big)\Big( \lim_{k \to \infty} x_k  \Big)  = X^{-1} X = e
\to \\
& \kern 1pc \to  \exists N \in \Nat : \forall n,k > N \.  \delta(x_n,x_k) \in V   \E, \\
&  (U,x) : \TYPE{SliceChart}(G,H) \to  H \cap U : \TYPE{Closed}(G), \\
&  H \cap U : \TYPE{Closed}(G) \to H \cap U \cap \overline{V} = H \cap \overline V : \TYPE{Closed}(G), \\
&  \A n,k \in \Nat : n,k > N , \\
& x : \TYPE{ConvergentFrom}(G,H) \to  \delta(x_n,x_k) = x_n^{-1}x_k \in  H, \\
& \LOGIC{def}(N) \to \delta(x_n,x_k) \in V \to  \delta(x_n,x_k) \in H \cap V ; \\
& : \forall n,k \in \Nat : n,k > N \.  \delta(x_n,x_k) \in H \cap V \As (2), \\
& \A n \in \Nat : n > N , \\
&  \lim_{k \to \infty} \delta(x_n,x_k) = x_n^{-1} \left( \lim_{k \to \infty} x_k \right)  = x_n^{-1}X \in_{(2)} \overline{H \cap V}  = H \cap \overline{V} \to \\
& \kern 1pc \to x^{-1}X \in H  \to X \in xH = H ;; \\
& : \forall  X \in \overline{H}  \. X \in H \to H : \TYPE{Closed}(G) \As  (3), \\
& (1,3)  \to      \TYPE{LieSubgroup} \& \TYPE{Closed}(G) ;; \square
\\ \\
& \TYPE{CLieSubgroup}(G) \de \TYPE{LieSubgroup} \& \TYPE{Closed}(G) \\
&  H : \TYPE{CLieSubgroup}(G) \iff  H \subset_{LG} G
\end{flalign*}
\newpage
\subsection{Linear Lie Groups}
\begin{flalign*}
&\FUNC{GeneralLinearGroup} :: \mathsf{VS}(\mathbb{F}) \to \TYPE{Group} \\
&\FUNC{GeneralLinearGroup}(V) = \mathrm{GL}(V) \de \Big( \big\{ T :\mathcal{L}(V,V): T : \TYPE{Invertible} \big\}, \circ \Big)  
\\ \\
&\FUNC{GeneralMatrixGroup} :: \TYPE{Field} \to \mathbb{N} \to \TYPE{LieGroup} \\
&\FUNC{GeneralMatrixGroup}( \mathbb{F},n) = \mathrm{GL}(\mathbb{F},n) \de 
\Big( \big\{ M \in \mathcal{M}^n(\Reals) : \det M \neq 0 \big\}, \cdot \Big) 
\\ \\
& \FUNC{SpecialLinearGroup} ::  \mathsf{FVS}(\mathbb{F}) \to \TYPE{LieGroup} \\
& \FUNC{SpecialLinearGroup}(V)  = \mathrm{SL}(V) \de \{ T \in \mathrm{GL}(V) : \det T = 1 \} 
\\ \\
& \TYPE{Nondegenerate} :: \prod V : \mathsf{VS}(\mathbb{F}) \. ?\mathcal{L}_2(V,V; \mathbb{F}) \\
& \beta : \TYPE{Nondegenerate} \iff \forall \sigma \in S(2) \. \Lambda v \in V \. \Lambda w \in W \. (v,w)\sigma\beta : \TYPE{Iso}_{\mathsf{VS}(\mathbb{F})}(V,V^*) 
\\ \\
& \TYPE{ScalarProduct}(V)  \de \TYPE{Nondegenerate} \& \TYPE{Symmetric}(V)
\\ \\
& \TYPE{ScalarProductSpace} \de \sum V :  \mathsf{VS}(\mathbb{F}) \. \TYPE{ScalarProduct}(V)
\\ \\
& \FUNC{OrthogonalLinearGroup} ::  \mathsf{IPVS}(\mathbb{F}) \to \TYPE{Group} \\
& \FUNC{OrthogonalLinearGroup}(V) = \mathrm{Aut}(V):= \big\{ A \in \mathrm{GL}(V) : \forall v,w \in V \. \langle Av, A w \rangle = \langle v, w \rangle \big\}
\\ \\
& \THM{SLAsClosedLieSubgroup} :: \forall V : \mathsf{FVS}(\mathbb{F}) \. \mathrm{SL}(V) \subset_{LG} \mathrm{GL}(V) \\
& \LOGIC{Proof} = \\ 
& \A  V : \mathsf{FVS}(\mathbb{F}), \\ 
& \A  T \in  \mathrm{SL}(V) \to  T \in \mathrm{GL}(V), \\
&   (U,x) \de \FUNC{chartCentredAt}(\mathrm{GL}(V),T), \\
&  x' \de \Lambda A \in \mathrm{GL}(V) \. (\det A  - 1 ) \oplus \bigoplus_{i = 2}^n x^i(A), \\
& (U.x') : \TYPE{SliceChart}(\mathrm{GL}(V),\mathrm{SL}(V),T); \\
& \mathrm{SL}(V) : \TYPE{Regular}(\mathrm{GL}(V)) , \\
& \THM{RegularLieSubgroup}(\mathrm{GL}(V),\mathrm{SL}(V)) \to \mathrm{SL}(V) \subset_{LG} \mathrm{GL}(V)
; \square
\end{flalign*}
\newpage
\begin{flalign*}
& \THM{AutAsClosedLieSubgroup} :: \forall V : \mathsf{FIPVS}(\mathbb{F}) \. \mathrm{SL}(V) \subset_{LG} \mathrm{GL}(V) \\
& \LOGIC{Proof} = \\ 
& \A  V : \mathsf{FIPVS}(\mathbb{F}), \\ 
&  \A  v, w \in V, \\
& a \de \langle v,w \rangle, \\ 
& F_{v,w}\de \big\{ A \in \mathrm{GL}(V) :  \langle Av, A w \rangle = a \big\}, \\ 
& \A  T \in  F_{v,w} \to  T \in \mathrm{GL}(V), \\
&   (U,x) \de \FUNC{chartCentredAt}(\mathrm{GL}(V),T), \\
&  x' \de \Lambda A \in \mathrm{GL}(V) \. (\langle A v, A w \rangle  - a ) \oplus \bigoplus_{i = 2}^n x^i(A), \\
& (U.x') : \TYPE{SliceChart}(\mathrm{GL}(V),F_{v,w},T); \\
& F_{v,w} : \TYPE{Regular}(\mathrm{GL}(V));  \\
& \mathrm{Aut}(V) = \bigcap_{v,w \in V} F_{v,w} : \TYPE{Regular}(\TYPE{GL}(V)), \\
& \THM{RegularLieSubgroup}(\mathrm{GL}(V),\mathrm{Aut}(V)) \to \mathrm{Aut}(V) \subset_{LG} \mathrm{GL}(V)
; \square
\end{flalign*}
\newpage
\subsection{Simplectic Forms}
\begin{flalign*}
&\FUNC{SimplecticProduct} :: \prod V  : \mathsf{FVS}(K) \. (V \times V) \to (V \times V) \to K \\
&\FUNC{SimplecticProduct}(v,w)(a,b) = ((v,w), (a,b))_{\nabla} \de  \sum^n_{i = 1} v_i b_i - \sum^n_{i = 1} w_i a_i
\\
&\FUNC{SimplecticGroup} :: \mathsf{FVS}(K) \to \TYPE{LieGroup} \\
&\FUNC{SimplecticGroup}(V) = \mathrm{Sp}(V) \de \big\{ T \in \mathrm{GL}(V \oplus V) : 
\forall v,w \in V \oplus V  \. (Tv,Tw)_\nabla = (v,w)_\nabla  \big\} 
\\ \\
&\TYPE{Quaternion} :: \TYPE{DivisionRing} \\
&\TYPE{Quaternion}  = \mathbb{H} \de ( \Reals^4, +, \Lambda a +b\mathrm{i} + c\mathrm{j} + d\mathrm{k} ,
x + y\mathrm{i} + z\mathrm{j} + u\mathrm{k} \in \mathbb{H}  \. \\
& \kern 1pc  \. (ax - by - cz - du) + (ay + bx + cu - dz)\mathrm{i}
+  (az + cx - bu + dy)\mathrm{j}  + (au + dx + bz - cy)]\mathrm{k} \\
& \kern 1pc ) \\
&\LOGIC{where} \\
& (a, b, c, d) \in \mathbb{H} \iff (a,b,c,d) = a + b\mathrm{i} + c\mathrm{j} + d\mathrm{k}
\\ \\
&\TYPE{Real} :: \; ?\mathbb{H} \\
&  a + b\mathrm{i} + c\mathrm{j} + d\mathrm{k} : \TYPE{Real} \iff b = c = d = 0 
\\ \\
&\TYPE{Imagenary} :: \; ?\mathbb{H} \\
& a + b\mathrm{i} + c\mathrm{j} + d\mathrm{k} : \TYPE{Imagenary} \iff a = 0
\\ \\
& \FUNC{conjugate} :: \mathbb{H} \to \mathbb{H} \\
& \FUNC{conjugate}(a + b\mathrm{i} + c\mathrm{j} + d\mathrm{j}) \de a - b\mathrm{i} - c\mathrm{j} - d\mathrm{k} \\ 
& \FUNC{conjugate }(x) \de \overline{x} 
\\ \\
& \FUNC{valuate} :: \mathbb{H} \to \Reals \\
& \FUNC{valuate}(x) = |x| \de \sqrt{ \overline{x}x }
\\ \\
& \FUNC{inverse} :: \mathbb{H} \setminus \{ 0 \} \to \mathbb{H} \\
& \FUNC{inverse}(x) = x^{-1} = \frac{\overline x}{|x|^2}
\\ \\
& \FUNC{QVectors} :: \Nat \to \TYPE{RightModule}(\mathbb{H}) \\
& \FUNC{QVectors}(n) = \mathbb{H}^n \de \Big( \mathbb{H}^n, + , \Lambda v \in \mathbb{H}^n \. \Lambda a \in \mathbb{H}^n \. [v_ia]^n_i \Big)
\\ \\
& \FUNC{quaternify} :: \mathbb{C} \to \mathbb{H} \\
& \FUNC{quaternify}(a + b\mathrm{i}) =  a + b\mathrm{i} +0\mathrm{j} + 0\mathrm{k} 
\\ \\
\end{flalign*}
\newpage 
\begin{flalign*}
& \FUNC{ToQuaternion} :: \TYPE{ISO}_{\mathsf{VS}(\mathbb{R})} ( \mathbb{C}^2. \mathbb{H} ) \\
& \FUNC{ToQuaternion}(x,y) = \nu(x,y) \de x + y\mathrm{j}  
\\ \\
& \THM{ComplexDecomposition} :: \forall M \in \mathcal{M}^{n\times m}(\mathbb{H}) \. \exists A,B \in \mathcal{M}^{n \times m}(\mathbb{C}) : \\
&  \kern 1pc : M = A + B\mathrm{j}  \\
& \LOGIC{Scatch} \\
& M = [q_{i,j}]^{n,m}_{i,j = 1} =  [a_{i,j} + b_{i,j}\mathrm{j}]^{n,m}_{i,j = 1} = [a_{i,j}]^{n,m}_{i,j = 1} 
+ [b_{i,j}]^{n,m}_{i,j = 1}\mathrm{j} = A + B\mathrm{j}
\\ \\
&  \FUNC{ToComplexMatrix} :: \mathcal{M}_{\mathsf{VS}(\Reals)}\Big(\mathcal{M}^{n\times m}(\mathbb{H}),\mathcal{M}^{2n \times 2k} (\mathbb{C})\Big) \\
&  \FUNC{ToComplexMatrix}(M) = \vartheta(M) \de \left[ 
\begin{matrix}
A & B \\
-\overline{B} & \overline{A} \\
\end{matrix}
\right] \\
& \kern 1pc \LOGIC{where} (A,B) \de \THM{ComplexDecomposition}(M)
\\ \\
& \THM{MatrixRepresentation}:: \vartheta : \TYPE{Iso}_{\mathsf{GRP}}(U(1,\mathbb{H}),SU(2)) \\
& \LOGIC{Scatch}: \\
& \vartheta((a +b\mathrm{j})(x + y\mathrm{j})) = \vartheta(ax + ay\mathrm{j} + b\overline{x}\mathrm{j} -
b\overline{y}  ) = \\
& = \left[         
 \begin{matrix}
 ax - b\overline y & ay + b\overline{x} \\
 -\overline{ay} + \overline b x &  \overline{ax} -\overline{b}y  
 \end{matrix}
\right] =
\left[
\begin{matrix}
a & b \\
-\overline{b} & \overline{a} \\
\end{matrix}
\right]
\left[
\begin{matrix}
x & y \\
-\overline{y} & \overline{x}
\end{matrix}
\right]
 = \vartheta(a + b\mathrm{j})\vartheta(x + y\mathrm{j}) \\
 &  |a + b\mathrm{j}| = 1 \to |a|^2 + |b|^2 = 1 \to \det \vartheta(a + b\mathrm{j}) = 1 \\
 &  \vartheta(a + b\mathrm{j})^*\vartheta(a + b\mathrm{j}) = I \\
 &   |a|^2 + |b|^2 = |c|^2 + |d|^2 = 1, \\
 & \overline{a}c + \overline{b}d = a\overline{c} + b\overline{d} = 0, \\
 &  ad - bc = 1 \\    
 & \to c = \overline{a} \\
 &  \to d = - \overline{b}
 \\ \\
& \THM{GenLinMark} :: \forall M \in \mathcal{M}^n(\mathbb{H}) \. M \in \mathrm{GL}(n,\mathbb{H}) \iff \vartheta M \in  \mathrm{GL}(2n,\mathbb{C}) \\ 
& \LOGIC{Scatch} : \\
& \vartheta M = \nu^{-1} M \nu, \mathrm{null} \; \nu = \{ 0 \} \leadsto \square
 \\ \\
 &\FUNC{scalarProductH} :: \mathbb{H}^n \to \mathbb{H}^n \to \mathbb{H} \\
 &\FUNC{scalarproductH}(a,b) = \langle a, b\rangle \de \sum^n_{i=1} \overline{a}b
\end{flalign*}
\end{document}
