\documentclass[12pt]{article}
\usepackage{mathtools}
\usepackage{amsmath}
\usepackage{amsfonts}
\usepackage{amssymb}
\usepackage{ wasysym }
\usepackage{ stmaryrd }
\usepackage[dvipsnames]{xcolor}
\usepackage[top=20mm, bottom=20mm, left=30mm, right=10mm]{geometry}
%Markup
\newcommand{\TYPE}[1]{\textcolor{NavyBlue}{\mathtt{#1}}}
\newcommand{\FUNC}[1]{\textcolor{Cerulean}{\mathtt{#1}}}
\newcommand{\LOGIC}[1]{\textcolor{Blue}{\mathtt{#1}}}
\newcommand{\THM}[1]{\textcolor{Maroon}{\mathtt{#1}}}
%META
\renewcommand{\.}{\; . \;}
\newcommand{\de}{: \kern 0.1pc =}
\newcommand{\extract}{\rightarrowtriangle}
\newcommand{\where}{\LOGIC{where}}
\newcommand{\If}{\LOGIC{if} \;}
\newcommand{\Then}{\LOGIC{then} \;}
%%STD
\newcommand{\Int}{\mathbb{Z} }
\newcommand{\NNInt}{\mathbb{Z}_{+} }
\newcommand{\Reals}{\mathbb{R} }
\newcommand{\Nat}{\mathbb{N} }
\DeclareMathOperator*{\centr}{center}
\DeclareMathOperator*{\argmin}{arg\,min}
\DeclareMathOperator*{\id}{id}
\newcommand{\EqClass}[1]{\TYPE{EqClass}\left( #1 \right)}
\newcommand{\Cate}{\TYPE{Category}}
\newcommand{\Func}[2]{\TYPE{Functor}\left( #1, #2 \right)}
\mathchardef\hyph="2D
\newcommand{\Surj}[2]{\TYPE{Surjective}\left( #1, #2 \right)}
%%ProofWritting
\newcommand{\A}{\LOGIC{Assume} \;} 
%LinearAlgebra
%TYPES
\newcommand{\VS}[1]{\TYPE{VectorSpace}\left( #1 \right)}
\newcommand{\Lin}[1]{\mathcal{L}\left( #1 \right)}
\newcommand{\vs}[1]{\mathsf{VS}\left( #1 \right)}
%FUNK
\DeclareMathOperator{\rk}{rank}
%Manifolds
%TYPES
\newcommand{\SManifold}{\TYPE{SmoothManifold}}
\newcommand{\SubMan}{\TYPE{SubManifold}}
\newcommand{\smooth}[1]{C^\infty\left( #1 \right)}
\newcommand{\CCurve}[2]{\TYPE{CentredCurve}\left(  #1, #2 \right)}
\newcommand{\Admis}{\TYPE{Admissible}}
\parindent=0em
%FUNC
\DeclareMathOperator{\cca}{chartCentredAt}
\DeclareMathOperator{\codim}{codim}
%TangentSpaces
%TYPES
\newcommand{\TS}{\TYPE{TangentSpace}}
\newcommand{\Germ}{\TYPE{Germ}}
\newcommand{\germs}[2]{\mathcal{F}_{#2}\left( #1 \right)}
\newcommand{\DirAt}{\TYPE{DerivationAtPoint}}
\newcommand{\dirAt}[2]{\mathrm{Dir}\left( #1 , #2 \right)}
\newcommand{\Phys}{\TYPE{PhysicalView}}
\newcommand{\Kin}{\TYPE{MovementDirection}}
\newcommand{\PManifold}{\TYPE{PointedManifolds}}
\newcommand{\PM}{\mathsf{PM}}
\newcommand{\TF}{\TYPE{TangentFunctor}}
\newcommand{\RegP}{\TYPE{RegularPoint}}
\newcommand{\CritP}{\TYPE{CriticalPoint}}
\newcommand{\RegV}{\TYPE{RegularValue}}
\newcommand{\CritV}{\TYPE{CriticalValue}}
\newcommand{\Zero}{\TYPE{Zero}}
\newcommand{\NonDeg}{\TYPE{NonDegenerate}}
\newcommand{\ConstRk}{\TYPE{ConstantRank}}
\newcommand{\Trans}{\TYPE{Transverse}}
%FUNC
\DeclareMathOperator{\alg}{alg}
\DeclareMathOperator{\fAlg}{fromAlg}
\DeclareMathOperator{\phys}{phys}
\DeclareMathOperator{\fPhys}{fromPhys}
\DeclareMathOperator{\kin}{kin}
\DeclareMathOperator{\fKin}{fromKin}
\newcommand{\tM}{\TYPE{TangentMap}}
\newcommand{\derT}{\TYPE{DerivationTransfer}}
\newcommand{\physS}{\TYPE{PhysicalShift}}
\newcommand{\kinT}{\TYPE{DirectionTransfer}}
\newcommand{\diff}{\TYPE{Differential}}
%THM
\newcommand{\Sard}{\THM{Sard}}
\newcommand{\MorseLemma}{\THM{MorseLemma}}
\newcommand{\LevelSubmanifold}{\THM{LevelSubmanifold}}
\newcommand{\TransThm}{\THM{Transversality}}
\newcommand{\TransPB}{\THM{TransversalPullbacks}}
\author{Uncultured Tramp} 
\title{TangentSpace.Know}
\begin{document}
\maketitle
\section{Classical Differential Geometry}
\subsection{Basic Definitions}
\begin{flalign*} 
&\TYPE{Hypersurface} :: \prod M : \TYPE{SManifold} \. ?\TYPE{Submanifold}(M) \\
&H : \TYPE{Hypersurface} \iff \codim H = 1 
\\ \\
&\TYPE{Parallel} :: \; ? TM \times TM \\
& ( (p,v), (q,w)) : \TYPE{Parallel}  \iff  (p,v) \parallel (q,w) \iff v = w 
\\ \\
&\TYPE{FieldAlongCurve} ::  \TYPE{Curve}(M) \to \TYPE{Field}(M) \\
&  Y : \TYPE{FieldAlongCurve}(c)  \iff \forall t \in \Reals \. \pi \;  Y(c(t)) = \dot c(t)
\\ \\
&\TYPE{DerivativeOfCurve} :: \TYPE{Curve}(\Reals^n) \to \TYPE{Curve}(T\Reals^n) \\
&\TYPE{DerivativeOfCurve}(c) = c' = \sum^n_{i=1} \dot c(t)^i\hat e_i(c(t))   
\\ \\
&\TYPE{HigherDerivativeCurve} :: \TYPE{Curve}(\Reals^n) \to \Nat \to  \TYPE{Curve}(T\Reals^n) \\
&\TYPE{HigherDerivativeCurve}(c,k) = c^{(k)} = \sum^n_{i=1} 
\frac{\mathrm{d}^{k-1}\dot c(t)^i}{\mathrm{d}t^{k-1}}\hat e_i(c(t)) 
\\ \\
&\FUNC{crossProduct} :: C^\infty((\Reals^n)^{n-1},\Reals^n) \\ &
\FUNC{crossProduct}(v) = \times(v) \de  \FUNC{asBasis}(\Reals^n)(\det[v] \otimes x)(x)  
\end{flalign*}
\newpage
\subsection{Curves}
\begin{flalign*} 
&\TYPE{RegularCurve} :: \; ? \TYPE{Curve}(\Reals^n) \\
&\gamma : \TYPE{RegularCurve} \iff \forall t \in \mathrm{Dom} \; \gamma \. \Vert \dot \gamma(t) \Vert \ \neq 0  
\\ \\
&\FUNC{unitTangentField} :: \TYPE{RegularCurve}(\Reals^n) \to  \TYPE{Curve}(\Reals^n) \\
&\FUNC{unitTangentField}(\gamma) = \mathrm{T}_\gamma = t \mapsto \frac{\dot \gamma (t)}{\Vert \dot \gamma(t) \Vert}
\\ \\
&\FUNC{Length} :: \TYPE{Path}(\Reals^n) \to  \Reals_+ \\
&\FUNC{Length}(\gamma) = L(\gamma) = \int^1_0 \Vert \dot \gamma(t) \Vert dt
\\ \\
&\FUNC{arclength} :: \prod \gamma : \TYPE{Curve}(\Reals^n) \. \mathrm{Dom} \; \gamma \to \mathrm{Dom} \; \gamma \to \Reals \\
&\FUNC{Length}(\gamma) = h(t_0)(t) = \int^t_{t_0} \Vert \dot \gamma(t) \Vert dt
\\ \\
&\FUNC{unitSpeedCurve} :: \TYPE{RegularCurve}(\Reals^n) \to \TYPE{RegularCurve}(\Reals^n) \\
&\FUNC{unitSpeedCurve}(\gamma) = c_\gamma \de t \mapsto \gamma \circ (h_{\gamma}(0))^{-1}(t)
\\ \\
&\FUNC{curvatureVector} :: \TYPE{RegularCurve}(\Reals^n) \to \TYPE{Curve}(\Reals^n) \\
&\FUNC{curvatureVector}(\gamma) = \kappa_{\gamma}  \de t  \mapsto \dot{ \mathrm{T}}_\gamma(t)  
\\ \\
&\FUNC{curvatureFunction} :: \TYPE{RegularCurve}(\Reals^n) \to C^\infty[0,1] \\
&\FUNC{curvatureVector}(\gamma) = K_{\gamma}  \de t  \mapsto \Vert \kappa_\gamma(t)\Vert 
\\ \\
&\TYPE{RealCurve} :: \; ? \TYPE{RegularCurve} \\
& \gamma ::   \TYPE{RealCurve} \iff K_{\gamma} > 0
\end{flalign*}
\end{document}