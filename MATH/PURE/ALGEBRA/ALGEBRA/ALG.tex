\documentclass[12pt]{scrartcl}% European-style article
\usepackage{mathtools}%For basic mathimatical symbols
\usepackage{amsmath}  %For basic mathimatical symbols
\usepackage{amsfonts} %For mathematical fonts
\usepackage{hyperref} %For clickable contents
\usepackage{amssymb}  %For more mathematical symbols
\usepackage{wasysym}  %For astronomical symbols
\usepackage{accents}  %For accents
\usepackage{graphicx} %For images
\usepackage{scalerel} %For resizing operators
\usepackage[dvipsnames]{xcolor} %For colors
\usepackage[a4paper,top=5mm, bottom=5mm, left=10mm, right=2mm]{geometry} %Page laout
%Markup
%To visually distinguish different things
\newcommand{\TYPE}[1]{\textcolor{NavyBlue}{\mathtt{#1}}}% Types are things that have members
\newcommand{\FUNC}[1]{\textcolor{Cerulean}{\mathtt{#1}}}% Func are things that trandform typed values
\newcommand{\LOGIC}[1]{\textcolor{Blue}{\mathtt{#1}}}% Logical elements are beyond the scope of the type theory
\newcommand{\THM}[1]{\textcolor{Maroon}{\mathtt{#1}}}% Theorems are things, which need to be proven
%META
%Basic elements of the language
\renewcommand{\.}{\; . \;} %to separate elements of quantified statements
\newcommand{\de}{: \kern 0.1pc =} %to define values of objects
\newcommand{\extract}{\LOGIC{Extract}} %produces a whitness of the existentionally typed object !Legacy! use E(\exists) instead
\newcommand{\where}{\LOGIC{where}} % used to define values post-usage
\newcommand{\If}{\LOGIC{if} \;} % A part of a famous trenary operator
\newcommand{\Then}{ \; \LOGIC{then} \;} % A part of a famous trenary operator
\newcommand{\Else}{\; \LOGIC{else} \;} % A part of a famous trenary operator
\newcommand{\IsNot}{\; ! \;} % A negation for a compound type (Is not a member of the Type, but of the same essence)
\newcommand{\Is}{ \; : \;}  % Type membership
\newcommand{\DefAs}{\; :: \;} % Defined to be a membere of a Type (essence)
\newcommand{\Act}[1]{\left( #1 \right)} % Func acts on an object  
\newcommand{\Example}{\LOGIC{Example} \; } % Used to identify examples !Legacy! we don't have examples any more
\newcommand{\Theorem}[2]{& \THM{#1} \, :: \, #2 \\ & \Proof = \\ } % An environment for declaring and defining=prooving a theorem
\newcommand{\DeclareType}[2]{& \TYPE{#1} \, :: \, #2 \\}% An environment for declaring a type (name + essence)   
\newcommand{\DefineType}[3]{& #1 : \TYPE{#2} \iff #3 \\}% An environment for defining a type (member + name + defining Type )
\newcommand{\DefineNamedType}[4]{& #1 : \TYPE{#2} \iff #3 \iff #4 \\}%An environment for defining a type (member +  name + symbol + defining Type ) 
\newcommand{\DeclareFunc}[2]{& \FUNC{#1} \, :: \, #2 \\}% An environment for declaring a func (name + type)   
\newcommand{\DefineFunc}[3]{&  \FUNC{#1}\Act{#2} \de #3 \\}% An environment for defining a type (name + argument + value expression) 
\newcommand{\DefineNamedFunc}[4]{&  \FUNC{#1}\Act{#2} = #3 \de #4 \\}% An environment for defining a type (name + argument + symbol + value expression)  
\newcommand{\NewLine}{\\ & \kern 1pc}% A shorthand for breaking a line inside Page environment      
\newcommand{\Page}[1]{ \begin{align*} #1 \end{align*}  }% An environment for writting this shit
\newcommand{ \bd }{ \ByDef }% A shorthand                                                  
\newcommand{\NoProof}{ & \ldots \\ \EndProof}% An omission of the proof of the theorem
\renewcommand{\And}{\; \& \;}% A typological and logical and
\newcommand{\Type}{\TYPE{Type}}% A metatype of Types
\newcommand{\Imply}{\Rightarrow}% An implication
%%STD
%Standard mathematical graphic
\newcommand{\Int}{\mathbb{Z}}% Integers
\newcommand{\NNInt}{\mathbb{Z}_{+}}% Positive Integers
\newcommand{\Reals}{\mathbb{R}}% Real Numbers
\newcommand{\Complex}{\mathbb{C}}% Complex Numbers
\newcommand{\Quat}{\mathbb{H}}% Quaternions
\newcommand{\Rats}{\mathbb{Q}}% Rational Numbres
\newcommand{\Nat}{\mathbb{N}}% Natural Numbers
\newcommand{\EReals}{\stackrel{\mathclap{\infty}}{\mathbb{R}}}% Extended real Numbers
\newcommand{\ERealsn}[1]{\stackrel{\mathclap{\infty}}{\mathbb{R}}^{#1}}% Extended Real Plane
\DeclareMathOperator*{\argmin}{arg\,min}% arg min
\DeclareMathOperator*{\id}{id}% identity map
\DeclareMathOperator*{\im}{Im}% an image of the function
\DeclareMathOperator*{\supp}{supp}% a support of something
\newcommand{\EqClass}[1]{\TYPE{EqClass}\left( #1 \right)}% An Equivalence Classes
\newcommand{\Cat}{\TYPE{Category}}% Type of categories
\newcommand{\Mor}{\mathcal{M}}% morphisms of the category
\newcommand{\Obj}{\mathcal{O}}% objects of the category
\newcommand{\Aut}{\mathrm{Aut}}% automorphisms of the object in the category
\newcommand{\End}{\mathrm{End}}% automorphisms of the object in the category
\mathchardef\hyph="2D % a hyphen for the use in the math mode 
\newcommand{\ToInj}{\hookrightarrow} % An arrow for injective maps
\newcommand{\ToSurj}{\twoheadrightarrow} % An arrow for the surjective maps
\newcommand{\ToBij}{\leftrightarrow} % A arrow for the bijective maps
\newcommand{\Set}{\TYPE{Set}} % Type of sets
\newcommand{\du}{\; \triangle \;} % symmetric difference
\renewcommand{\c}{\complement}% set-theoretic complement
%%ProofWritting
% Commands to write proofs
\newcommand{\Say}[3]{& #1 \de #2 : #3, \\} % A Logical Statements (name + expression + type of the expression)
\newcommand{\Conclude}[3]{& #1 \de #2 : #3; \\}% A conclusion which ends a reflection (name with end pointer + expression + type of the expression )
\newcommand{\Derive}[3]{& \leadsto #1 \de #2 : #3, \\} % A Result produced by conlcuding the reflection, must follow comclusion (name + post-reflection + type)         
\newcommand{\DeriveConclude}[3]{& \leadsto #1 \de #2 : #3 ; \\} % Use to follow a conclusion by an another conclusion imedietely ( name with end pointer + post-reflection + type  )
\newcommand{\Assume}[2]{& \LOGIC{Assume} \; #1 : #2, \\} %Starts a reflection (name + type)
\newcommand{\As}{\; \LOGIC{as} \;} %An ambigous symbol (Legacy)
\newcommand{\QED}{\; \square} %A symbol to end the proof
\newcommand{\EndProof}{& \QED \\} %End of proof
\newcommand{\ByDef}{\rotatebox[origin=c]{-180}{$D$}}%\text{\textthorn}}  %Extracts defining type statement from the type member, may be inverted  (T -> Type)
\newcommand{\ByConstr}{\rotatebox[origin=c]{-180}{$C$}}%\text{\textopeno}} %Extract the defining statement from the defined value, may be inverted (T -> Type) 
\newcommand{\Proof}{\LOGIC{Proof} \; } % Begins a Proof
%FOUND
%Foundations of mathematics
%CAT
%Category Theory
\newcommand{\Arrow}[1]{\xrightarrow{#1}}% an arrow representatition of the morphism
\newcommand{\ToIso}[1]{\xleftrightarrow{#1}}% an arrow representation of the isomprphism
%Types
\newcommand{\Cov}{\TYPE{Covariant}}% A type of Covariant functors
\newcommand{\Contra}{\TYPE{Contravariant}}% A type of the Contravariant Functors
\newcommand{\NT}{\TYPE{NaturalTransform}}% A type of the Natural Transormations
\newcommand{\UMP}{\TYPE{UnversalMappingProperty}}% A type of catgories with the universal mapping property ?
\newcommand{\CMP}{\TYPE{CouniversalMappingProperty}}% A type of categories with the couniversal mapping property ?
\newcommand{\paral}{\rightrightarrows} %?
%functions
\newcommand{\op}{\mathrm{op}} %opposite cotegory
\newcommand{\obj}{\mathrm{obj}} %objects?
\DeclareMathOperator*{\dom}{dom} % domain
\DeclareMathOperator*{\codom}{codom}% codomain
\DeclareMathOperator*{\colim}{colim}% colimit
%variable
% Varianles for denoting categories
\newcommand{\C}{\mathcal{C}}
\newcommand{\A}{\mathcal{A}}
\newcommand{\B}{\mathcal{B}}
\newcommand{\D}{\mathcal{D}}
\newcommand{\I}{\mathcal{I}}
\newcommand{\J}{\mathcal{J}}
\newcommand{\R}{\mathcal{R}}
\newcommand{\G}{\mathsf{G}}
%Cats
\newcommand{\CAT}{\mathsf{CAT}} % 2-Category of all Categories
\newcommand{\SET}{\mathsf{SET}} % Category of Sets
\newcommand{\PARALLEL}{\bullet \paral \bullet} % A parallel category
\newcommand{\WEDGE}{\bullet \to \bullet \leftarrow \bullet} % Wedge category
\newcommand{\VEE}{\bullet \leftarrow \bullet \to \bullet} % Vee Category
%Algebra
%Abstract Algebra
%Group Theory
%Types
\newcommand{\Group}{\TYPE{Group}} % Type of groups
\newcommand{\Abel}{\TYPE{Abelean}} % Type of abelean groups
\newcommand{\Sgrp}{\subset_{\mathsf{GRP}}} % Subgroup as a subset
\newcommand{\Nrml}{\vartriangleleft} % Normal Subgroup as a subset
\newcommand{\FG}{\TYPE{FiniteGroup}} % Finite Groups
\newcommand{\Stab}{\mathrm{Stab}}  % A stabilizer
%\newcommand{\FGA}{\TYPE{FinitelyGeneratedAbelean}} % A Finitely Generated abelean group
\newcommand{\DN}{\TYPE{DirectedNormality}} % A normal complex
%Func
\DeclareMathOperator{\tor}{tor} % torsion
\DeclareMathOperator{\bool}{bool} % boolinization
\DeclareMathOperator{\rank}{rank} % a rank
%Cats
\newcommand{\GRP}{\mathsf{GRP}} % A category of Groups
\newcommand{\ABEL}{\mathsf{ABEL}} % a category of Abelean Groups
%Ops
\newcommand{\SDP}{\rightthreetimes} % A very special norm
%LINEAR
%Linear Algebra
%Types
\newcommand{\Basis}{\TYPE{Basis}} % Basis of the linear space
\newcommand{\submod}[1]{\subset_{\LMOD{#1}}}% submodule as a subset
\newcommand{\subvec}[1]{\subset_{\VS{#1}}}% vector subspace as a subset
\newcommand{\FGM}{\TYPE{FinitelyGeneratedModule}}% Finitely generated module
\newcommand{\LI}{\TYPE{LinearlyIndependent}}
\newcommand{\LIS}{\TYPE{LinearlyIndependentSet}}
\newcommand{\FM}{\TYPE{FreeModule}}
\newcommand{\IBP}{\TYPE{InvariantBasisProperty}}
\newcommand{\UTM}{\TYPE{UpperTriangularMatrix}}
\newcommand{\LTM}{\TYPE{LowerTriangularMatrix}}
\newcommand{\Diag}{\TYPE{DiagonalMatrix}}
\newcommand{\FP }{\TYPE{FinitelyPresented}}
\newcommand{\GL}{\mathbf{GL}}% General Linear Group
\newcommand{\SL}{\mathbf{SL}}% Special Linear Group
\newcommand{\SO}{\mathbf{SO}}% Special Orthogonal Group
\newcommand{\SU}{\mathbf{SU}}% Special Unitary Group
\newcommand{\prsubvec}[1]{\subsetneq_{\VS{#1}}}	% poper vector subspace as a subset
\newcommand{\LC}{\TYPE{LinearComplement}} 
\newcommand{\IS}{\TYPE{InvariantSubspace}}
\newcommand{\RP}{\TYPE{ReducingPair}}
\newcommand{\RCF}{\TYPE{RationalCanonicalForm}}
\newcommand{\JCF}{\TYPE{JordanCanonicalForm}}
\newcommand{\Diagble}{\TYPE{Diagonalizable}}
\newcommand{\UT}{\TYPE{UpperTriangulizable}}
\newcommand{\LT}{\TYPE{LowerTriangulizable}}
\newcommand{\IPS}{\TYPE{InnerProductSpace}}
\newcommand{\OBasis}{\TYPE{OrthonormalBasis}}
\newcommand{\FDIPS}{\TYPE{FiniteDimensionalInnerProductSpace}}
\newcommand{\NO}{\TYPE{NormalOperator}}
\newcommand{\NM}{\TYPE{NormalMatrix}}
\newcommand{\SA}{\TYPE{SelfAdjoint}}
\newcommand{\SSA}{\TYPE{SkewSelfAdjoint}}
\newcommand{\PI}{\TYPE{Pseudoinverse}}
\newcommand{\OVS}{\TYPE{OrthogonalVectorSpace}}
\newcommand{\SVS}{\TYPE{SymplecticVectorSpace}}
\newcommand{\MVS}{\TYPE{MetricVectorSpace}}
\newcommand{\FDMVS}{\TYPE{FiniteDimensionalMetricVectorSpace}}
\newcommand{\Sp}{\mathbf{Sp}}%Symplectic Group 
%Func
\DeclareMathOperator{\Span}{span} % spann by subset
\DeclareMathOperator{\Ann}{Ann}   % annihilator
\DeclareMathOperator{\Ass}{Ass}   % associated primes
\DeclareMathOperator{\diag}{diag} % diagonal
\DeclareMathOperator{\adj}{adj}   % an adjoint matrix
\DeclareMathOperator{\tr}{tr}     % trace
\DeclareMathOperator{\codim}{codim} % codimension
\DeclareMathOperator{\Cell}{\mathbf{C}} % a componion matrix
\DeclareMathOperator{\JC}{\mathbf{J}}  % a Jordan cell
\DeclareMathOperator{\bigboxplus}{\scalerel*{\boxplus}{\sum}} % a direct sum of operators in the sence of the reducing a pair
\DeclareMathOperator{\Spec}{Spec} % Spectre
\DeclareMathOperator{\bigbot}{\scalerel*{\bot}{\sum}} % an othogonal direct sum
\DeclareMathOperator{\GS}{\mathbf{GS}} %Gramm-Smmidt process
\DeclareMathOperator{\NGS}{\mathbf{NGS}} %Normalized Gramm-Smmidt process
\DeclareMathOperator{\WI}{\mathrm{WI}} %Witt Index
%Cats
\newcommand{\VS}[1]{#1\hyph\mathsf{VS}} % a category of vector spaces (Field)
\newcommand{\FDVS}[1]{#1\hyph\mathsf{FDVS}} % a category of finite-dimensional vector spaces (Field)
\newcommand{\LMOD}[1]{#1\hyph\mathsf{MOD}} % a category of the left modules (Ring)
\newcommand{\RMOD}[1]{\mathsf{MOD}\hyph#1} % a category of the right modules (Ring)
\newcommand{\LLMAP}[1]{#1\hyph\mathsf{LMAP}} % a cagory of based linear maps with the left scalar multiplication (Ring)
\newcommand{\LMAT}[1]{#1\hyph\mathsf{MAT}}  % a category of based matrices with the left scalar multiplication (Ring)
\newcommand{\NMAT}[1]{#1\hyph\mathbb{N}} % a category of finite matrices (Field)
%Symbols
\renewcommand{\L}{\mathcal{L}}
\renewcommand{\O}{\mathbf{O}}
\newcommand{\U}{\mathbf{U}}
\renewcommand{\S}{\mathbf{S}}
%FIELDS
\newcommand{\Field}{\TYPE{Field}}
\newcommand{\ACF}{\TYPE{AlgebraicallyClosedField}}
%RINGS
%TYPE
\newcommand{\Ring}{\TYPE{Ring}}
\newcommand{\CR}{\TYPE{CommutativeRing}}
\newcommand{\Ideal}{\TYPE{Ideal}}
\newcommand{\ID}{\TYPE{IntegralDomain}}
\newcommand{\UFD}{\TYPE{UniqueFactorizationDomain}}
\newcommand{\PID}{\TYPE{PrincipleIdealDomain}}
\newcommand{\FGI}{\TYPE{FinitelyGeneratedIdeal}}
\newcommand{\ER}{\TYPE{EuclideanRing}}
\newcommand{\DVR}{\TYPE{DiscreteValuationRing}}
\newcommand{\MoFT}{\TYPE{MonoidOfFiniteType}}
%CATS
\newcommand{\RING}{\mathsf{RING}} % A category of Rings
\newcommand{\ANN}{\mathsf{ANN}} % A category of Commutative Rings
%FUNCS
\DeclareMathOperator{\lcd}{lcd} % least common devided 
\DeclareMathOperator{\lc}{lc} % leading coefficient of the polynomial
\DeclareMathOperator{\cont}{cont} % content of the polynomial
\DeclareMathOperator{\antideg}{antideg} % antidegree of the foramal power series
%Symbols
\newcommand{\F}{\mathcal{F}}
%ALGEBRA
\newcommand{\Algebra}{\TYPE{Algebra}}
\newcommand{\LALG}[1]{#1\hyph\mathsf{ALG}}% Left associative unital algebras (Ring)
\newcommand{\RALG}[1]{\mathsf{ALG}\hyph#1}% Right associative unital  algebras (Rings)
\newcommand{\LALGE}[1]{#1\hyph\mathsf{ALGE}}% Left associative unital algebras (Ring)
\newcommand{\RALGE}[1]{\mathsf{ALGE}\hyph#1}% Right associative unital  algebras (Rings)
\newcommand{\LLGE}[1]{#1\hyph\mathsf{LGE}}% Left associative unital algebras (Ring)
\newcommand{\RLGE}[1]{\mathsf{LGE}\hyph#1}% Right associative unital  algebras (Rings)
\newcommand{\LLG}[1]{#1\hyph\mathsf{LG}}% Left associative unital algebras (Ring)
\newcommand{\RLG}[1]{\mathsf{LG}\hyph#1}% Right associative unital  algebras (Rings)
\newcommand{\LCALG}[1]{#1\hyph\mathsf{CALG}}% Left associative unital algebras (Ring)
\newcommand{\RCALG}[1]{\mathsf{CALG}\hyph#1}% Right associative unital  algebras (Rings)
\newcommand{\LCALGE}[1]{#1\hyph\mathsf{CALGE}}% Left associative unital algebras (Ring)
\newcommand{\RCALGE}[1]{\mathsf{CALGE}\hyph#1}% Right associative unital  algebras (Rings)
\newcommand{\LCLGE}[1]{#1\hyph\mathsf{CLGE}}% Left associative unital algebras (Ring)
\newcommand{\RCLGE}[1]{\mathsf{CLGE}\hyph#1}% Right associative unital  algebras (Rings)
\newcommand{\LCLG}[1]{#1\hyph\mathsf{CLG}}% Left associative unital algebras (Ring)
\newcommand{\RCLG}[1]{\mathsf{CLG}\hyph#1}% Right associative unital  algebras (Rings)
\newcommand{\FGA}{\TYPE{FinitelyGeneratedAlgebra}}
\newcommand{\FGCA}{\TYPE{FinitelyGeneratedCommutativeAlgebra}}
\newcommand{\PGA}{\TYPE{PolynomialGradedAlgebra}}
\newcommand{\COALG}[1]{#1\hyph\mathsf{COALG}}%Coalgebra (Rings) 
\newcommand{\CCOALG}[1]{#1\hyph\mathsf{CCOALG}}%Cocommutative Coalgebra (Rings)
\newcommand{\SCOALG}[1]{#1\hyph\mathsf{SCOALG}}%Skew-cocommutative Coalgebra (Rings)
\newcommand{\hit}{\rightharpoonup}
\newcommand{\hitBy}{\leftharpoonup}
\newcommand{\LAMOD}[1]{{\;}_{#1}\mathsf{MOD}}% Left algebra modules
\newcommand{\RAMOD}[1]{\mathsf{MOD}_{#1}}% Right algebra modules
\newcommand{\LCOMOD}[1]{{\;}^{#1}\mathsf{MOD}}% Left Comodules
\newcommand{\RCOMOD}[1]{\mathsf{MOD}^{#1}}% Right Comodules
\newcommand{\BIALG}[1]{#1\hyph\mathsf{BIALG}}% Bialgebras
\newcommand{\LBALG}[1]{{\;}_{#1}\mathsf{ALGE}}% Left Algebras of Bialgebras
\newcommand{\RBALG}[1]{\mathsf{ALGE}_{#1}}% Right Algebras of Bialgebras
\newcommand{\RBCOALG}[1]{\mathsf{COALG}_{#1}}% Right Coalgebras of Bialgebras
\newcommand{\LBCOALG}[1]{{\;}_{#1}\mathsf{COALG}}% Left Coalebras of Bialgebras
\newcommand{\HOPF}[1]{#1\hyph\mathsf{HOPF}}% Hopf Algebras
\newcommand{\RHMOD}[1]{\mathsf{MOD}^{#1}_{#1}}%Right Hopf Modules
\newcommand{\CLIF}[1]{#1\hyph\mathsf{CLIF}}%Clifford Algebras (Field) 
\newcommand{\PIN}{\mathbf{PIN}}
\newcommand{\SPIN}{\mathbf{SPIN}}
\DeclareMathOperator{\CL}{CL}%Clifford Functor
\DeclareMathOperator{\tad}{\widetilde{ad}}%Twisted Adjoint Representation
%Numbers
%Integers
%FUNCS
\DeclareMathOperator{\divi}{div} % devide withou reminder
\DeclareMathOperator{\remi}{rem} % reminder
\DeclareMathOperator{\Frac}{Frac} % Field of fractions
\title{Algebras}
\author{Uncultured Tramp}
\begin{document}
\maketitle
\normalsize
\newpage
\tableofcontents
\newpage
\section{Associative Algebras over Commutative Rings}
\subsection{Categories Of Algebras}
\Page{
	\Conclude{\Algebra}{ \prod R \in \ANN  \.   \sum X \in \LMOD{R} \. X \otimes X \to X  }{\ANN \to \Type}
	\\
	\DeclareFunc{multiplication}{\prod (A,\odot) : \Algebra(R) \. A \otimes A \to A}
	\DefineNamedFunc{multiplication}{A}{(\cdot_A)}{(\odot)}
	\\	
	\DeclareFunc{AlgebraModule}{\prod (A,\odot) : \Algebra(R) \. \LMOD{L}}
	\DefineNamedFunc{RingGroup}{A}{A}{A}
	\\
	\DeclareType{UnitalAlgebra}{ \prod R \in \ANN  ? \Algebra(R)  }
	\DefineType{A}{UnitalAlgebra}{ \exists e \in A : \TYPE{Identity}(\odot)   }
	\\
	\DeclareFunc{identity}{\prod A : \TYPE{UnitalAlgebra} \. A}
	\DefineNamedFunc{identity}{R}{1_R}{ \bd \TYPE{UnitalAlgebra}(A) }
	\\
	\DeclareType{CommutativeAlgebra}{?\Algebra(R)}
	\DefineType{A}{CommutativeAlgebra}{(\cdot_A) : \TYPE{Commutative}(A)}
	\\
	\DeclareType{DivisionAlgebra}{?\Algebra(R)}
	\DefineType{(R,+,\cdot)}{DivisioniAlgebra}{(\cdot) : \TYPE{Invertible}(A \setminus 0)}
	\\
	\DeclareType{AlgebraHomo}{\prod A,B : \Algebra(R) \.  ?(A \Arrow{\LMOD{R}} B)}
	\DefineType{f}{AlgebraHomo}{\forall x,y \in A \. f[x,y] = \Big[f(x),f(y)\Big]}  
	\\
	\DeclareType{UnitalHomo}{\prod A,B : \TYPE{UnitalAlgebra}(R) \. ?\TYPE{AlgebraHomo}(A,B)}
	\DefineType{f}{UnitalHomo}{f(e) = e}  
	\\
	\Theorem{IdIsHomo}{\forall A : \Algebra(R) \. {\id}_{A} : \TYPE{RingHomo}}
	\Assume{a,b}{A}
	\Conclude{(*)}{ \bd \id   }{  \id[a,b] = [a,b]  = [\id(a),\id(b)] }
	\EndProof
	\\
	\Theorem{IdIsUnital}{\forall A : \TYPE{UnitalAlgebra}(R) \. {\id}_{A} : \TYPE{UnitalHomo}}
	\NoProof
}
\Page{
	\\
	\DeclareFunc{structuralHomomrphism}{\forall A : \TYPE{UnitalAlgebra}(R) \.  R \Arrow{\LMOD{R}} A   }
	\DefineNamedFunc{structuralHomomrphis}{\alpha}{\epsilon(\alpha)}{\alpha e}
	\\
	\Theorem{AlgebraHomoCompos}{\forall A,B,C : \Algebra(R) \. \forall f : \TYPE{AlgebraHomo}(A,B) \. \forall g : \TYPE{AlgebraHomo}(B,C) \. 
	 	\NewLine \. g \circ f : \TYPE{UnitalAlgebraHomo}(A,C)}
	\NoProof
	\\
	\Theorem{UnitalAlgebraHomoCompos}{\forall A,B,C : \TYPE{UnitalAlgebra}(R) \. \forall f : \TYPE{UnitalAlgebraHomo}(A,B) \. \NewLine \. \forall g : \TYPE{UnitalAlgebraHomo}(B,C) \. 
		g \circ f : \TYPE{UnitalAlgebraHomo}(A,C)}
	\NoProof
	\\
	\DeclareFunc{AlgebraCat}{\RING \to \CAT}
	\DefineNamedFunc{AlgebraCat}{R}{\LLG{R}}{\Big( \TYPE{\Algebra}(R), \TYPE{AlgebraHomo},\circ,\id \Big)}
	\\
	\DeclareFunc{CommAlgebraCat}{\ANN \to \CAT}
	\DefineNamedFunc{CommAlgebraCat}{R}{\LCLG{R}}{\Big( \TYPE{CommutativeAlgebra}(R), \TYPE{AlgebraHomo},\circ,\id \Big)}		
	\\
	\DeclareFunc{assAlgebraCat}{\ANN \to \CAT}
	\DefineNamedFunc{assAlgebraCat}{R}{\LALG{R}}{\Big( \TYPE{AssociativeAlgebra}(R), \TYPE{AlgebraHomo},\circ,\id \Big)}
	\\
	\DeclareFunc{commAssAlgebraCat}{\ANN \to \CAT}
	\DefineNamedFunc{commAssAlgebraCat}{R}{\LCALG{R}}{\NewLine \de \Big( \TYPE{AssociativeAlgebra} \And \TYPE{CommutativeAlgebra}(R), \TYPE{AlgebraHomo},\circ,\id \Big)}		
	\\
	\DeclareFunc{unitalAlgebraCat}{\ANN \to \CAT}
	\DefineNamedFunc{unitalAlgebraCat}{R}{\LLGE{R}}{\Big( \TYPE{UnitalAlgebra}(R), \TYPE{UnitalAlgebraHomo},\circ,\id \Big)}
	\\
	\DeclareFunc{unitalCommAlgebraCat}{\ANN \to \CAT}
	\DefineNamedFunc{unitalCommAlgebraCat}{R}{\LCLGE{R}}{\NewLine \de \Big( \TYPE{CommutativeAlgebra} \And \TYPE{UnitalAlgebra}(R), \TYPE{UnitalAlgebraHomo},\circ,\id \Big)}		
	\\
	\DeclareFunc{unitalAssAlgebraCat}{\ANN \to \CAT}
	\DefineNamedFunc{unitalAssAlgebraCat}{R}{\LALGE{R}}{\NewLine \de \Big( \TYPE{UnitalAlgebra} \And \TYPE{AssociativeAlgebra}(R), \TYPE{UnitalAlgebraHomo},\circ,\id \Big)}
	\\
	\DeclareFunc{unitalAssCommAlgebraCat}{\ANN \to \CAT}
	\DefineNamedFunc{unitalAssCommAlgebraCat}{R}{\LCALGE{R}}{\NewLine \de \Big( \TYPE{CommutativeAlgebra} \And \TYPE{UnitalAlgebra} \And \TYPE{AssociativeAlgebra}(R), \TYPE{UnitalAlgebraHomo},\circ,\id \Big)}		
}\Page{
	\DeclareType{Subalgebra}{\prod R \in \RING \. \prod A \in \LLG{R} ??A}
	\DefineNamedType{B}{Subalgebra}{B \subset_{\LALG{R}} A}{\big((B,\odot_{A|B}) : \TYPE{Algebra}(R)\big)}
	\\
	\DeclareType{UnitalSubalgebra}{\prod R \in \RING \. \prod A \in \LLG{R} ??A}
	\DefineNamedType{B}{UnitalSubalgebra}{B \subset_{\LALGE{R}} A}{\big((B,\odot_{A|B}) : \TYPE{UnitalAlgebra}(R)\big)}
	\\
	\DeclareFunc{TrivialRing}{ \ANN  }
	\DefineNamedFunc{TrivialRing}{}{\star}{\Big(\{\star\}, (\star,\star) \mapsto \star, (\star,\star) \mapsto \star \Big)}
	\\
	\Theorem{MultZero}{\forall A \in \Algebra(R) \. \forall a \in A \. [0,a] = [a,0] = 0}
	\Say{[0]}{ \bd \Algebra(R) }{ \Big( R \oplus A ,\Lambda (\alpha,a),(\beta,b) \in (R \oplus A) \otimes (R \oplus A) \. \alpha \beta + \beta a + \alpha b + [a, b]   \Big) : \LLGE{R}  }
	\Say{[1]}{ \bd \TYPE{Identity}(1)\bd \TYPE{Distrivutive}(R,+,\cdot)\bd\TYPE{Identity}(0)\bd\TYPE{Identity}(1)  }{  [0,a] + a = [0 + 1,a] = [1,a] = a   }
	\Say{[2]}{ \bd \TYPE{Identity}(1)\bd \TYPE{Distrivutive}(R,+,\cdot)\bd\TYPE{Identity}(0)\bd\TYPE{Identity}(1)  }{  [a,0] + a = [a,0 + 1] = [a,1] = a  }
	\Conclude{(*)}{\THM{IdentityIsUnique}(1)(2)}{[a,0] = 0 = [0,a]}
	\EndProof
	\\
	\Theorem{MultNeg}{\forall R \in \RING \. \forall A \in \LLGE{R} \. \forall a \in A \. [-e,a] = -a = [a, -e] }
	\Say{[1]}{\bd \TYPE{Identity}\bd \TYPE{Distributive}(R)\bd \TYPE{Inverse}(1)}{ a + [-e,a] = [e - e,a] = [0,a] = 0}
	\Say{[2]}{\bd \TYPE{Identity}\bd \TYPE{Distributive}(R)\bd \TYPE{Inverse}(1)}{ a + [a,e] = [a,e - e] = 0}
	\Conclude{(*)}{\THM{InverseIsUnique}(1)(2)}{ [-1,a] = - a = [a,-1]}
	\EndProof
	\\
	\Theorem{SubalgebraImage}{\forall R \in \RING \. \forall A,B \in \LLG{R} \. \forall S : \TYPE{Subalgebra}(A) \. \forall f : A \Arrow{\LLG{R}} B \. f(S) \subset_{\LLG{R}} B}
	\NoProof
	\\
	\Theorem{SubringPreimage}{ \forall R \in \RING \. \forall A,B \in \RING \. \forall S : \TYPE{Subalgebra}(B) \. \forall f : A \Arrow{\LLG{R}} B \. f^{-1}(S) \subset_{\RING} A}
	\NoProof
	\\
	\Theorem{AlgebraOfFunctions}{\forall X \in \SET \. \forall R \in \ANN \. \Big(\Mor_{\SET}(X,R),+,\cdot\Big) \in \LALG{R}}
	\NoProof
}\Page{
	\DeclareFunc{productAlgebra}{ \prod I \in \SET \. \prod R \in \ANN \.  (I \to  \LLG{R}) \to \LLG{R}}
	\DefineNamedFunc{productAlgebra}{A}{\prod_{i \in I} A_i}{  \left( \prod i \in I \. A_i,  a,b \mapsto \Lambda i \in I \.  a_ib_i \right)}
	\\
	\DeclareFunc{projection}{\prod I \in \SET \. \prod R \in \ANN \.  \prod R : I \to \RING \. \prod i \in I \.  \prod_{i \in I} R_i \Arrow{\LLG{R}} R_i }
	\DefineNamedFunc{projection}{a}{\pi_i(a)}{a_i}
	\\
	\DeclareFunc{rightMultiplication}{ \prod R \in \ANN \. \prod R \in \LLG{R} \. A \Arrow{\LLG{R}} \End_{\LLG{R}}(A)}
	\DefineNamedFunc{rightMultiplication}{a}{\rho_a}{ \Lambda b \in R \. ab}
	\\
	\DeclareFunc{leftMultiplication}{ \prod R \in \ANN \. \prod A \in \LLG{R} \. A \Arrow{\LLG{R}} \End_{\LLG{R}}(A)}
	\DefineNamedFunc{leftMultiplication}{a}{\lambda_a}{ \Lambda b \in A \. ba}
	\\
	\Theorem{AssociativeAlgebrasAreRings}{\forall R \in \ANN \. \forall A \in \LALGE{R} \. \Big(A,[\cdot,\cdot]\Big) \in \RING }
	\NoProof
	\\
	\Theorem{RingsAreAssociativeAlgebras}{ \RING \cong_{\CAT} \LALGE{\Int}  }
	\NoProof
	\\
	\DeclareType{LeftUnit}{\prod R \in \ANN \. \prod A \in \LLGE{R} \. ?A}
	\DefineType{u}{LeftUnit}{\exists a \in A : au = e}
	\\
	\DeclareType{RightUnit}{\prod R \in \ANN \. \prod A \in \LLGE{R} \. ?A}
	\DefineType{u}{RightUnit}{\exists a \in A : ua = e }
	\\
	\DeclareType{LeftZeroDivisor}{ \prod R \in \ANN \. \prod A \in \LLG{R} \. ?A   }
	\DefineType{x}{LeftZeroDivizor}{\exists a \in A \. xa = 0 \And x \neq 0}
	\\
	\DeclareType{RightZeroDivisor}{ \prod R \in \ANN \. \prod A \in  \LLG{R} \. ?A   }
	\DefineType{x}{RightZeroDivizor}{\exists a \in R \. ax = 0 \And x \neq 0}
	\\
	\Conclude{\TYPE{ZeroDivisor}}{\Lambda R \in \RING \. \Lambda A \in \LLG{R} \. \TYPE{RightZeroDivisor} | \TYPE{LeftZeroDivisor}(A)}{\prod R \in \RING \. \LLG{R}  \to \Type}
	\\
	\Conclude{ \TYPE{Regular} }{\Lambda R \in \RING \. \Lambda A \in \LLG{R} \. !\TYPE{ZeroDivisor}(A)}{\prod R \in \RING \. \LLG{R} \to \Type}
	\\
	\Conclude{\TYPE{Unit}}{\Lambda R \in \RING \. \Lambda A \in \LLGE{R} \. \TYPE{LeftUnit} \And \TYPE{RightUnit}(A)}{\prod R \in \RING \. \LLGE{R} \to \Type}
}\Page{
	\Theorem{UnitsAreRegular}{\forall R \in \RING \. \forall A \in \LALGE{R} \.  \forall u : \TYPE{Unit}(A) \. u : \TYPE{Regular}(A)}
	\Assume{a}{R}
	\Assume{(1)}{[u,a] = 0}
	\Assume{(2)}{a \neq 0}
	\Say{(3,v)}{\bd \TYPE{LeftUnit}(u)}{\sum v \in A  \.  [v,u] = e }
	\Say{(4)}{\bd \TYPE{Identity}(1)(a) (3)(vua)(1)\THM{ZeroMult}(v)}{ a = [e,a]  = \Big[[v,u],a\Big] = \Big[v,[u,a]\Big] = 0  }
	\Conclude{()}{  (2)(4)  }{\bot}
	\Derive{(1)}{\bd^{-1}\TYPE{RightZeroDivisor}E(\bot)}{[u \IsNot \TYPE{RightZeroDivisor}(R)]}
	\Assume{a}{R}
	\Assume{(2)}{[a,u] = 0}
	\Assume{(3)}{a \neq 0}
	\Say{(4,v)}{\bd \TYPE{LeftUnit}(u)}{\sum v \in R  \.  [u,v] = e }
	\Say{(4)}{\bd \TYPE{Identity}(1)(a) (3)(auv)(1)\THM{ZeroMult}(v)}{ a = [a,e]  = \Big[a,[u,v]\Big] = [0,v] = 0  }
	\Conclude{()}{  (2)(4)  }{\bot}
	\Derive{(2)}{\bd^{-1}\TYPE{LeftZeroDivisor}E(\bot)}{[u \IsNot \TYPE{LeftZeroDivisor}(R)]}
	\Conclude{(3) }{\bd^{-1}\TYPE{Regualar}(1)(2)}{ [u : \TYPE{Regular}] }
	\EndProof
	\\
	\DeclareFunc{groupOfUnits}{\prod R \in \ANN \. \LALGE{R} \to \GRP}
	\DefineNamedFunc{groupOfUnits}{ R }{R^* }{(\TYPE{Unit}(R),\cdot_R)}
	\\
	\DeclareType{Nillpotent}{ \prod R \in \ANN \. \prod A \in \LLG{R} \.  ?A   }
	\DefineType{a}{Nillpotent}{ \exists n \in \Nat :  a^n = 0}
	\\
	\DeclareType{Unipotent}{\prod R \in \ANN \. \prod A \in \LLGE{R} ?A}
	\DefineType{a}{Unipotent}{ a - e : \TYPE{Nillpotent}(R)  }
	\\
	\DeclareType{Idempotent}{\prod R \in \ANN \. \prod A \in \LLG{R} \. ?A}
	\DefineType{a}{Idempotent}{ a^2 = a}
	\\
	\DeclareType{Involution}{\prod R \in \ANN \. \prod A \in \LLGE{R} \.  ?A}
	\DefineType{a}{Involution}{a^2 = e}
	\\
	\Theorem{NillpotentProduct}{\forall R \in \ANN \. \forall A \in \LALG{R} \.  \forall a : \TYPE{Nillpotent}(A) \. \NewLine \. \forall b : \TYPE{Commutes}(A,\cdot_R)(a) \. [a,b] : \TYPE{Nillpotent}(R) }
	\Say{(1,n)}{\bd \TYPE{Nillpotent}(a)}{\sum n \in \Nat \. a^n = 0}
	\Say{(2)}{\bd \TYPE{Commutes}(b)(ab)^n(1)\THM{ZeroMult}(R)(b^n)}{(ab)^n = a^nb^n =0b^n = 0}
	\Conclude{()}{\bd^{-1} \TYPE{Nillpotent}(2)  }{ [ab : \TYPE{NillPotent}(R)]  }
	\EndProof
}\Page{
	\Theorem{NillpotentSum}{\forall R \in \ANN \. \forall A \in \LALG{R} \. \forall a,b : \TYPE{Nillpotent}(A) \. \TYPE{Commutes}(A,\cdot_A)(a,b) \Rightarrow a + b : \TYPE{Nillpotent}(A) }
	\Say{(1,n)}{\bd \TYPE{Nillpotent}(a)}{ \sum n \in \Nat \. a^n = 0}
	\Say{(2,m)}{\bd \TYPE{Nillpotent}(b)}{ \sum m \in \Nat \. b^m = 0}
	\Say{(3)}{\THM{BinomialSum}(b,m,n + m)(1)(2)}{ (a + b)^{n + m} = \sum^{n + m}_{i = 1} C^i_{n + m} a^{i}b^{n + m - i} = 0    }
	\Conclude{()}{\bd^{-1} \TYPE{Nillpotent}(3) }{ [a + b : \TYPE{NillPotent}(R)] }
	\EndProof
	\\
	\Theorem{UnitDiff}{\forall R \in \ANN \. \forall A \in \LALGE{R} \. \forall a \in A^* \. \forall b : \TYPE{Nillpotent}(A) \. \TYPE{Commutes}(A,\cdot_R)(a,b) \Rightarrow a - b \in A^*}
	\Say{(n,1)}{\bd \TYPE{Nillpotent}(b)}{\sum n \in \Nat \. b^n = 0}
	\Say{(2)}{ \THM{SumOfPowers}(a,b,n)(1)\bd \TYPE{Inverse}  }{  (a - b)\left( \sum^{n-1}_{i=0} a^{i}b^{n-1 - i}  \right)a^{-n} = (a^n - b^n)a^{-n} = a^n a^{-n} = 1   }
	\Conclude{(*)}{ \bd^{-1} A^*(2) }{ a - b \in R^*   }
	\EndProof
	\\
	\DeclareType{LeftIdeal}{\prod R \in \ANN \. \prod A \in \LLG{R} \. ?\TYPE{Subgroup}(A)}
	\DefineType{I}{LeftIdeal}{\forall a \in I \. \forall b \in A \. ba \in I}
	\\
	\DeclareType{RightIdeal}{\prod R \in \ANN \. \prod A \in \LLG{R} \.  ?\TYPE{Subgroup}(A)}
	\DefineType{I}{RightIdeal}{\forall b \in I \. \forall b \in A \. ab \in I}
	\\
	\Conclude{\TYPE{TwoSidedIdeal}}{\prod R \in \ANN \. \prod A \in \LLG{R} \. \TYPE{LeftIdeal}(R) \And \TYPE{RightIdeal}(R)}{\prod R \in \ANN \. \LLG{R} \to \Type}
	\\
	\Theorem{CommutativeIdeal}{\forall R \in \ANN \. \forall A \in \LCLG{R} \. \forall I : \TYPE{LeftIdeal}(R) \. I : \TYPE{TwoSidedIdeal}(R) }
	\NoProof
	\\
	\Conclude{\Ideal}{\prod R \in \ANN \. \prod A \in \LCLG{R} \. \TYPE{LeftIdeal}(R)}{\prod R \in \ANN \. \LCLG{R} \to \Type}
	\\
	\DeclareFunc{quatMult}{\prod R \in \ANN \. \prod A \in \LLG{R} \prod I : \TYPE{TwoSidedIdeal} \. \frac{R}{I} \to \frac{R}{I} \to \frac{R}{I}}
	\DefineNamedFunc{quatMult}{[a],[b]}{[a][b]}{[ab]}
	\Assume{x,y}{I}
	\Say{(1)}{\bd \TYPE{RightIdeal}(a,y)}{ay \in I}
	\Say{(2)}{ \bd \TYPE{LeftIdeal}(b,x) }{xb \in I}
	\Say{(3)}{\bd \TYPE{RightIdeal}(x,y)}{xy \in I}
	\Conclude{(*)}{\ldots}{ [a + x][b + y] = [ab + xb + ay  + xy] = [ab]}
	\EndProof
}\Page{
	\DeclareFunc{quotientAlgebra}{\prod R \in \ANN \. \TYPE{TwoSidedIdeal} \to \GRP}
	\DefineNamedFunc{quotientAlgebra}{I}{\frac{R}{I}}{\left(\frac{R}{I},+,\FUNC{quatMult}\right)}
	\\
	\Theorem{LeftIdealPreimage}{ \forall R \in \ANN \. \forall A,B \in \LLG{R} \.  \forall f : A \Arrow{\LLG{R}} B \.  \forall I : \TYPE{LeftIdeal}(B) \. f^{-1}(I) : \TYPE{LeftIdeal}(A)}
	\NoProof
	\\
	\Theorem{RightIdealPreimage}{ \forall R \in \ANN \. \forall A,B \in \LLG{R} \.  \forall f : A \Arrow{\LLG{R}} B \.  \forall I : \TYPE{RightIdeal}(B) \. f^{-1}(I) : \TYPE{RightIdeal}(A)}
	\NoProof
	\\
	\Theorem{TwoSidedIdealPreimage}{ \forall R \in \ANN \. \forall A,B \in \LLG{R} \.  \forall f : A \Arrow{\LLG{R}} B \.  \forall I : \TYPE{TwoSidedIdeal}(B) \.  \NewLine \. f^{-1}(I) : \TYPE{TwoSidedIdeal}(A)}
	\NoProof
	\\
	\Theorem{IdealPreimage}{ \forall R \in \ANN \. \forall A,B \in \LCLG{R} \.  \forall f : A \Arrow{\RING} B \.  \forall I : \TYPE{Ideal}(B) \. f^{-1}(I) : \TYPE{Ideal}(A)}
	\NoProof
	\\
	\Theorem{LeftIdealIntersection}{\forall R \in \ANN \. \forall A \in \LLG{R} \. \forall \mathcal{A} \in \SET \. \forall I : \mathcal{A} \to \TYPE{LeftIdeal}(A) \. \bigcap_{\alpha \in \mathcal{A}} I_{\alpha} : \TYPE{LeftIdeal}(A) }
	\NoProof
	\\
	\Theorem{RightIdealIntersection}{\forall R \in \ANN \. \forall A \in \LLG{R} \. \forall \mathcal{A} \in \SET \. \forall I : \mathcal{A} \to \TYPE{RightIdeal}(R) \. \bigcap_{\alpha \in \mathcal{A}} I_{\alpha} : \TYPE{RightIdeal}(A) }
	\NoProof
	\\
	\Theorem{TwoSidedtIdealIntersection}{\forall R \in \ANN \. \forall A \in \LLG{R} \. \forall \mathcal{A} \in \SET \. \forall I : \mathcal{A} \to \TYPE{TwoSidedIdeal}(A) \. 
		\NewLine \. \bigcap_{\alpha \in \mathcal{A}} I_{\alpha} : \TYPE{TwoSidedIdeal}(A) }
	\NoProof
}\Page{
	\Theorem{IdealIntersection}{\forall R \in \ANN \. \forall A \in \LCLG{R} \. \forall \mathcal{A} \in \SET \. \forall I : \mathcal{A} \to \TYPE{Ideal}(A) \. \bigcap_{\alpha \in \mathcal{A}} I_{\alpha} : \TYPE{Ideal}(A) }
	\NoProof
	\\
	\Theorem{SumOfLeftIdeals}{ \forall R \in \ANN \. \forall A \in \LCLG{R} \. \forall \mathcal{A} \in \SET \. \forall I : \mathcal{A} \to \TYPE{LeftIdeal}(A) \. \sum_{\alpha \in \mathcal{A}} I_{\alpha} : \TYPE{LeftIdeal}(A)}
	\NoProof
	\\
	\Theorem{SumOfRightIdeals}{ \forall R \in \ANN \. \forall A \in \LLG{R} \. \forall \mathcal{A} \in \SET \. \forall I : \mathcal{A} \to \TYPE{RightIdeal}(A) \. \sum_{\alpha \in \mathcal{A}} I_{\alpha} : \TYPE{RightIdeal}(A)}
	\NoProof
	\\
	\Theorem{SumOfTwoSidedIdeals}{ \forall R \in \ANN \. \forall A \in \LLG{R} \. \forall \mathcal{A} \in \SET \. \forall I : \mathcal{A} \to \TYPE{TwoSidedIdeal}(A) \. 
		\NewLine \sum_{\alpha \in \mathcal{A}} I_{\alpha} : \TYPE{TwoSidedIdeal}(A)}
	\NoProof
	\\
	\Theorem{SumOfIdeals}{ \forall R \in \ANN \. \forall A \in \LCLG{R} \. \forall \mathcal{A} \in \SET \. \forall I : \mathcal{A} \to \TYPE{Ideal}(A) \. \sum_{\alpha \in \mathcal{A}} I_{\alpha} : \TYPE{Ideal}(A)}
	\NoProof
        \\
	\DeclareFunc{compositeIdeal}{\prod R \in \ANN \. \forall A \in \LLG{R} \.  \TYPE{LeftIdeal}(A) \times \TYPE{RightIdeal}(A) \to \TYPE{TwoSidedIdeal}(A)}
	\DefineNamedFunc{compositeIdeal}{I,J}{IJ}{ \left\{  \sum^{n}_{\alpha=1} a_\alpha b_\alpha  | n \in \Nat, a : n \to I, b : n \to J  \right\}  }
	\\
	\DeclareFunc{compositeIdeal2}{\prod R \in \ANN \. \forall A \in \LCALG{R} \. \prod n \in \Nat \. n \to \TYPE{Ideal}(A) \to \TYPE{Ideal}(A)}
	\DefineNamedFunc{compositeIdeal2}{I}{\prod^n_{\alpha = 1} I_\alpha }{ \left\{ \sum^m_{\beta = 1} \prod^n_{\alpha = 1} a_{\alpha,\beta} |  m \in \Nat, a : \prod \alpha \in n \. m \to I_{\alpha}  \right\}  }
}\Page{
	\DeclareFunc{genLeftIdeal}{\prod R \in \ANN \. \prod A \in \LLG{R} \.  ?A \to \TYPE{LeftIdeal}(A) }
	\DefineFunc{genLeftIdeal}{S}{\bigcap \{ I : \TYPE{LeftIdeal}(A) : S \subset A  \}}
	\\
	\DeclareFunc{genRightIdeal}{\prod R \in \ANN \.\prod A \in \LLG{R} \.  ?A \to \TYPE{RightIdeal}(A) }
	\DefineFunc{genRightIdeal}{S}{\bigcap \{ I : \TYPE{RightIdeal}(A) : S \subset A  \}}
	\\
	\DeclareFunc{genTwoSidedIdeal}{\prod R \in \ANN \. \prod A \in \LLG{R} \.  ?A \to \TYPE{TwoSidedIdeal}(A) }
	\DefineFunc{genTwoSidedIdeal}{S}{\bigcap \{ I : \TYPE{TwoSidedIdeal}(A) : S \subset A  \}}
	\\
	\DeclareFunc{genIdeal}{\prod R \in \ANN \. \prod A \in \LCLG{R} \.  ?A \to \TYPE{Ideal}(A) }
	\DefineFunc{genIdeal}{S}{\bigcap \{ I : \TYPE{Ideal}(A) : S \subset A  \}}	
	\\
	\Theorem{kernelIdeal}{\forall R \in \ANN \. \forall A,B \in \LLG{R} \. \forall \varphi : A \Arrow{\LLG{R}} B \.  \ker \varphi : \TYPE{TwoSidedIdeal}(A)}
	\NoProof
	\\
	\Theorem{IdealProjectionIsAlgebraHomo}{\forall R \in \RING \. \forall I : \TYPE{TwoSidedIdeal}(R) \.  \pi_{I} : R \Arrow{\RING} \frac{R}{I} }
	\Say{(1)}{\bd \pi_I(1)}{\pi_I(1) = [1]}
	\Assume{a,b}{R}
	\Conclude{()}{\bd \pi_I(ab)\bd \FUNC{quotMult}([a],[b])\bd^{-1} \pi_I(a)\bd^{-1}\pi_I}{\pi_I(ab) = [ab] =[a][b] = \pi_I(a)\pi_I(b)}
	\EndProof
	\\
	\Theorem{EveryIdealIsRHKernel}{ \forall R \in \ANN \. \forall A \in \LLG{R} \. \forall I : \TYPE{TwoSidedIdeal}(R) \. I = \ker \pi_I}
	\NoProof
	\\
	\DeclareFunc{freeCAlgebra}{\prod R \in \ANN \. \Cov(\SET,\LCALGE{R})}
	\DefineNamedFunc{freeCAlgebra}{X}{F_{\LCALGE{R}}(X)}{R\Big[\Int_+^X\Big]}
	\DefineNamedFunc{freeCAlgebra}{X,Y,f}{F_{\LCALGE{R},X,Y}(f)}{\Lambda \sum_{p : X \to \Int_+ } \alpha_p \prod_{x \in X} x^{p_x} \. \sum_{p : X \to \Int_+} \alpha_p \prod_{x \in X} f(x)^{p_x} }
	\\
	\DeclareType{\FGCA}{\prod R \in \ANN \. ?\LCALGE{R}}
	\DefineType{A}{\FGCA}{\exists X \in \SET \. \exists I : \Ideal\Big( F_{\LCALGE{R}}(X)\Big) \. A = \frac{F_{\LCALGE{R}}(X)}{I}}
	\\
	\DeclareFunc{freeAlgebra}{\prod R \in \ANN \. \Cov(\SET,\LALGE{R})}
	\DefineNamedFunc{freeAlgebra}{X}{F_{\LALGE{R}}(X)}{ R^{\oplus\TYPE{String}(X)} }
	\DefineNamedFunc{freeAlgebra}{X,Y,f}{F_{\LALGE{R},X,Y}(f)}{\Lambda \sum_{ x \in \TYPE{String}(X) \to \Int_+ } \alpha_x \prod_{i=1}^{|x|} x_i \. \sum_{x \in \TYPE{String}(X)} \alpha_x \prod_{i=1}^{|x|} f(x_i) }
	\\
	\DeclareType{\FGA}{\prod R \in \ANN \. ?\LALGE{R}}
	\DefineType{A}{\FGA}{\exists X \in \SET \. \exists I : \TYPE{TwoSidedIdeal}\Big( F_{\LALGE{R}}(X)\Big) \. A = \frac{F_{\LALGE{R}}(X)}{I}}
}
\subsection{Tensor Product Of Algebras}
\Page{
	\DeclareFunc{tensorProductOfAlgebras}{\prod R \in \ANN \.  \prod n \in \Nat \. n \to \LALG{R} \to \LALG{R} }
	\DefineNamedFunc{tensorProductOfAlgebras}{ A }{\bigotimes^n_{i=1} A_i}{\NewLine \de \left( \bigotimes^n_{i=1} A_i, 
		\FUNC{tensorize} \Lambda \sum^{m}_{i=1} \bigotimes^{n}_{j=1} a_{i,j},\sum^{m'}_{i=1} \bigotimes^n_{j=1} b_{i,j}  \. \sum^m_{i=1} \sum^{m'} \bigotimes^n_{j=1} a_{i,j}b_{i',j}  \right)}
	\\
	\Theorem{TensorProductOfUnitalAlgebras}{\forall R \in \ANN \. \forall n \in \Nat \. \forall A : n \to \LALGE{R} \. \bigotimes^n_{i=1} A_i \in \LALGE{R} }
	\NoProof
	\\
	\Theorem{TensorProductOfCommutativeAlgebras}{\forall R \in \ANN \. \forall n \in \Nat \. \forall A : n \to \LCALG{R} \. \bigotimes^n_{i=1} A_i \in \LCALG{R} }
	\NoProof
	\\
	\Theorem{AssociativeTensorProductOfAlgebras}{\forall R \in \ANN \. \forall A,B,C \in \LALG{R} \. \NewLine (A \otimes B) \otimes C \cong_{\LALG{R}} A \otimes (B \otimes C)}
	\NoProof
	\\
	\Theorem{TensorProductOfAlgebrasPermutation}{\forall R \in \ANN \.  \forall n \in \Nat \. \forall A : n \to \LALG{R} \. \forall \sigma \in S_n \. 
		\NewLine \.
		\bigotimes^n_{i=1} A_i \cong_{\LALG{R}} \bigotimes^n_{i=1} A_{\sigma(i)}
	}
	\NoProof
	\\
	\Theorem{TrivialTensorProduct}{\forall R \in \ANN \. \forall A \in \LALG{R} \. R \otimes A \cong A}
	\NoProof
}\Page{
	\Theorem{TensorProductOfFractionAlgebras}{\forall R \in \ANN \. \forall \Sigma_1,\Sigma_2 \in \TYPE{MultiplicativeSet}(R) \. \NewLine \.  \Sigma_1^{-1}R \otimes \Sigma_2^{-1}R \cong_{\LALGE{R}} (\Sigma_1\Sigma_2)^{-1}R}
	\Say{\varphi}{\FUNC{tensorize}\left(\Lambda \frac{a}{\sigma} \in \Sigma_1^{-1} \. \Lambda \frac{b}{\sigma'} \. \frac{ab}{\sigma\sigma'} \right)}
	{
		\Sigma_1^{-1}R \otimes \Sigma_2^{-1}R \Arrow{\LMOD{R}} (\Sigma_1\Sigma_2)^{-1}R
	}
	\Say{[1]}{\bd \varphi }{\varphi(1 \otimes 1) = 1}
	\Say{[2]}{\bd \ANN (R)\ByConstr \varphi [2] }{ \Big( \varphi : \Sigma_1^{-1}R \otimes \Sigma_2^{-1} \Arrow{\LALGE{R}} (\Sigma_1\Sigma_2)^{-1}R \Big) } 
	\Assume{\frac{a}{\sigma}}{(\Sigma_1\Sigma_2)^{-1}R}
	\Say{(\alpha,\beta, [1])}{\bd \Sigma_1\Sigma_2(\sigma)}{\sum \alpha \in \Sigma_1 \sum \beta \in \Sigma_2 \. \sigma = \alpha \beta}
	\Conclude{[\ldots*]}{I(\varphi)}{\varphi\left( a \frac{1}{\alpha} \otimes \frac{1}{\beta}  \right) = \frac{a}{\alpha\beta} = \frac{a}{\sigma}}
	\Derive{[3]}{\bd^{-1}\TYPE{Surjective}}{\Big( \varphi : \Sigma_1^{-1}R \otimes \Sigma_2^{-1} \ToSurj (\Sigma_1\Sigma_2)^{-1}R \Big) }
	\Assume{t}{\Sigma_1^{-1}R \otimes \Sigma_2^{-1}R}
	\Assume{[4]}{\varphi(t) = 0}
	\Say{(r,\alpha,\beta,[5])}{\bd t}{\sum r \in R \. \sum \alpha \in \Sigma_1 \. \sum \beta \in \sigma_2 \. t = r\frac{1}{\alpha}\otimes \frac{1}{\beta}}
	\Say{[6]}{[4][5]\ByConstr\varphi}{ 0 = \varphi(t) = \frac{r}{\alpha\beta} }
	\Say{[7]}{\bd \TYPE{MultiplicativeSet}(\Sigma_1,\Sigma_2)[6]}{  r = 0  }
	\Conclude{[t.4.*]}{[5][7]}{t = 0}
	\Derive{[4]}{\THM{ZeroKernelTHM}[3]}{\Big( \varphi : \Sigma_1^{-1}R \otimes \Sigma_2^{-1} \ToIso{\LALGE{R}} (\Sigma_1\Sigma_2)^{-1}R \Big) }
	\Conclude{[5]}{\bd^{-1}\TYPE{Isomotphic}[4]}{\LOGIC{This}}
	\EndProof
}
\newpage
\subsection{Graded Algebras}
\Page{
	\DeclareType{GradedAlgebra}{\prod R \in \ANN \. ? \sum \Delta : \TYPE{CommutativeMonoid} \. \sum A \in \LALG{R} \. \Delta \to \TYPE{Submodule}(R,A)}
	\DefineType{(\Delta,A,H)}{GradedAlgebra}{A = \bigoplus_{\delta \in \Delta} H_i \And \forall \alpha,\beta \in \Delta \. \forall a \in H_\alpha \. \forall b \in H_\beta \. a + b \in H_{\alpha + \beta} }
	\\
	\DeclareType{Homogeneous}{\prod R \in \ANN \. \prod (\Delta,A,H) : \TYPE{GradedAlgebra}(A) \. ?A}
	\DefineType{a}{Homogeneous}{\exists \delta \in \Delta : a \in H_\delta}
	\\
	\DeclareFunc{homogeneousElement}{\prod R \in \ANN \. \prod (\Delta,A,H) : \TYPE{GradedAlgebra}(R) \. A \to \Delta \to A}
	\DefineNamedFunc{homogeneousElement}{a,\delta}{a_\delta}{b_\delta \NewLine \quad \where \NewLine 
		\quad \quad (b,[\ldots]) = \bd\TYPE{GradedAlgebra}(\Delta,A,H) \. \sum b : \prod_{\delta \in \Delta} H_\delta \. a = \sum_{\delta \in \Delta} b_\delta}
	\\
	\Theorem{ZerothHomogeneousSubalgebra}{\forall R \in \ANN \. \forall (\Delta,A,H) : \TYPE{GradedAlgebra}(R) \. H_0 \subset_{\LALG{R}} A}
	\NoProof
	\\
	\Theorem{ZerothHomogeneousUnitalSubalgebra}{\forall R \in \ANN \. \forall (\Delta,A,H) : \TYPE{GradedAlgebra}(R) \. A \in \LALGE{R} \Rightarrow  H_0 \subset_{\LALGE{R}} A}
	\NoProof
	\\
	\DeclareType{\PGA}{?\TYPE{GradedAlgebra}(R)}
	\DefineType{(\Int,A,H)}{\PGA}{ A = \langle H_1 \rangle_{\LALGE{R}} }
	\\
	\Theorem{FreeCoefficientLemma}{
		\forall R \in \ANN \. 
		\forall (\Int,A,H) : \PGA(R) \. 
		\forall [0] : A \in \LALGE{R} \. \NewLine \.  
		H_0 = \langle e \rangle 
	}
	\Assume{a}{H_0}
	\Say{(b,[1])}{\bd \TYPE{GradedAlgebra}(\Int,A,H)(a)}{\sum b : \prod_{\delta \in \Int} H_\delta \. a = \sum_{\delta \in \Delta} b_\delta} 
	\Say{(c,[2])}{[00](b)}{\sum c : \prod_{n \in \Nat}  n \to H_1 \. b_0 \in Re \And \forall n \in \Nat \. b_n = \prod^n_{i=1} c_{n,i} }
	\Say{[3]}{\bd a [1]}{ \forall n \in \Nat \. b_n = 0}
	\Conclude{[a.*]}{[00][1][3][2]}{  a \in Re}
	\Derive{[*]}{ \THM{ZerothHomogeneousUnitalSubalgebra}  }{H_0 = \langle e \rangle}
	\EndProof
	\\
	\DeclareType{HomogeneousIdeal}{\prod R \in \ANN \. \prod (\Delta,A,H) : \TYPE{GradedSubalgebra}(k) \. ?\TYPE{TwoSidedIdeal}(A) }
	\DefineType{I}{HomogeneousIdeal}{\forall a \in I \. \forall \delta \in \Delta \. a_\delta \in I }
}
\Page{
	\Theorem{HomogeneousIdealLemma}{ 
		\forall R \in \ANN \. 
		\forall (\Delta,A,H) : \TYPE{GradedSubalgebra}(k) \. 
		\forall I : \TYPE{TwoSidedIdeal}(A) \. \NewLine
		I : \TYPE{HomogeneousIdeal}(A) \iff \exists X : ?\TYPE{Homogeneous}(A) : I = \langle X \rangle 
	}
	\NoProof
	\\
	\Assume{R}{\ANN}
	\\
	\Assume{(\Delta,A,H)}{\TYPE{GradedAlgebra}(R)}
	\\
	\Assume{[0]}{A \in \LALGE{R}}
	\\
	\Theorem{HomogeneousIdealAsGradedModule}{\forall I : \TYPE{HomogeneousIdeal}(\Delta,A,H) \. (I,I \cap H) : \TYPE{GradedModule}(\Delta,A,H)}
	\NoProof
	\\
	\Theorem{HomogeneousQuotient}{\forall I : \TYPE{HomogeneousIdeal}(\Delta,A,H) \. \left( \Delta, \frac{A}{I}, \frac{I + H}{I}  \right) : \TYPE{GradedAlgebra}(R)}
	\Assume{[a]}{\prod_{\delta \in \Delta} \frac{I + H_\delta}{I}}
	\Assume{[1]}{\sum_{\delta \in \Delta} [a_\delta] = 0}
	\Say{[2]}{\bd \FUNC{quatientModule}[1]}{\sum_{\delta \in \Delta} a_\delta \in H}
	\Say{[3]}{\bd \TYPE{HomogeneousIdeal}[2]}{ \forall \delta \in \Delta \. a_\delta \in H }
	\Conclude{[a.*]}{\bd \FUNC{quatientModule}[3]}{ \forall \delta \in \Delta \. [a_\delta] = 0}
	\Derive{[1]}{\bd^{-1}\TYPE{DirectSum}}{\frac{I}{H} = \bigoplus_{\delta \in \Delta} \frac{I + H_\delta }{I}}
	\Assume{\alpha,\beta}{\Delta}
	\Assume{[a]}{\frac{I + H_\alpha}{I}}
	\Assume{[b]}{\frac{I + H_\beta}{I}}
	\Say{[2]}{\bd \TYPE{GradedAlgebra}(\Delta,A,H)(a,b)}{ab \in H_{\alpha + \beta}}
	\Conclude{[a.*]}{\bd \FUNC{quotientAlgebra}[1][2]}{[a][b] = [ab] \in \frac{H_{\alpha+\beta} + I }{I}}
	\Derive{[*]}{\bd^{-1}\TYPE{GradedAlgebra}}{\left( \left(\Delta, \frac{A}{I},\frac{I + H}{I} \right) : \TYPE{GradedAlgebra}(R)  \right)}
	\EndProof
}
\Page{
	\Theorem{GradedTensorProduct}{\forall R \in \ANN \. \forall n \in \Nat \. \forall (\Delta,A,H) : n \to \TYPE{GradedAlgebra}(R) \.
		\NewLine \. \left( \prod^n_{i=1} \Delta_i,\bigotimes^n_{i=1} A_i, \bigotimes^n_{i=1} H_i\right) : \TYPE{GradedAlgebra}(R) } 
	\NoProof
	\\
	\Theorem{IntegralGradedTensorProduct}{ \forall n \in \Nat \. \forall (\Int,A,H) : n \to \TYPE{GradedAlgebra}(R) \. \NewLine \.  
		\left( \Int, \bigotimes^n_{i=1} A_i, \Lambda m \in \Int \. \bigoplus \sum i : n \to \Int \. \sum_{k \in n} i_k = m \. \bigotimes^n_{j=1} H_{j,i_j}   \right)
		: \TYPE{GradedAlgebra}(R)
	}
	\NoProof
	\\
	\DeclareType{GradedAlgHomo}{\prod (\Delta,A,H),(\Delta,B,H') : \TYPE{GradedAlgebra}(R) \. ? A \Arrow{\LALGE{R}} B}
	\DefineType{f}{GradedeAlgHomo}{\forall \delta \in \Delta \. f^{-1} H'_\delta = H_\delta}
	\\
	\DeclareFunc{categoryOfGradedAlgebras}{\ANN \to \TYPE{commutativeMonoid} \to \CAT}
	\DefineNamedFunc{categoryOfGradedAlgebras}{R,\Delta}{\LALGE{R}(\Delta)}{(\TYPE{GradedAlgebra}(R), \TYPE{GradedAlgHomo},\circ,\id)}
	\\
	\Theorem{AssociativeTensorProductOfAlgebras}{\forall R \in \ANN \. \forall A,B,C \in \LALG{R}(\Delta) \. \NewLine (A \otimes B) \otimes C \cong_{\LALG{R}(\Delta)} A \otimes (B \otimes C)}
	\NoProof
	\\
	\Theorem{TensorProductOfGradedAlgebrasPermutation}{\forall R \in \ANN \.  \forall n \in \Nat \. \forall A : n \to \LALG{R}(\Delta) \. \forall \sigma \in S_n \. 
		\NewLine \.
		\bigotimes^n_{i=1} A_i \cong_{\LALG{R}(\Delta)} \bigotimes^n_{i=1} A_{\sigma(i)}
	}
	\NoProof
	\\
	\Theorem{TrivialTensorProduct}{\forall R \in \ANN \. \forall A \in \LALG{R}(\Delta) \. R \otimes A \cong_{\LALG{R}(\Delta)} A}
	\NoProof
	\\
	\Theorem{TensorProductOfGradedHomo}{ 
		\forall R \in \ANN \. 
		\forall n \in \Nat \. 
		\forall A,B : n \to \LALG{R}(\Delta) \. \NewLine \.  
		\forall f : \prod^n_{i=1} A_i \Arrow{\LALG{R}(\Delta)} B_i \.  
		\bigotimes^n_{i=1} f : \bigotimes^n_{i=1} A_i \Arrow{\LALG{R}(\Delta)} \bigotimes^n_{i=1} B_i  	
	}
	\NoProof
}
\Page{
	\Theorem{CentaralIdemppotentHasDegreeZero}
	{
		\forall R \in \ANN \.
		\forall (\Int_+,A,H) \in \LALGE{R}(\Int_+) \. \NewLine \.  
		\forall a \in Z(A) \.
		\forall [0] : a^2 = a \.
		a \in H_0
	}
	\Say{b}{a - a_0}{A}
	\Say{[1]}{\ByConstr b}{b_0 = 0}
	\Say{[2]}{[0]\bd a_0}{a_0^2 = a_0}
	\Say{[3]}{[2]\bd Z(A) \THM{BinomialExpansion}(2)}{ (1 - a_0)^2 = 1 - 2a_0 + a_0 = 1 - a_0  }
	\Say{[4]}{ \ByConstr b [0]  }{ (1 - a_0)a = (1 - a_0)(b + a_0) = (1 - a_0)b }
	\Say{[5]}{[4][3]}{\Big( (1 - a_0)b : \TYPE{Idempotent}(Z(A)) \Big)}
	\Say{[6]}{[5][1]}{(1 -a_0)b = 0}
	\Say{[7]}{\bd \LALGE{R}(A)[6]}{a_0b = b}
	\Say{[8]}{\ByConstr b [0] \ByConstr b [2][7]}{  a_0 + b  = a = a^2  = a_0^2 + 2ba_0 + b^2 = a_0 + 2b + b^2  }
	\Say{[9]}{\bd \LALGE{R}(A)}{b^2 = -b}
	\Say{[10]}{[9][1]}{b = 0 }
	\Conclude{[*]}{\ByConstr b \bd a_0}{a \in H_0}
	\EndProof
	\\
	\DeclareFunc{leggedAlgebra}{\prod R \in \ANN \. \LMOD{R}  \to  \LALGE{R}(\Int_+) }
	\DefineFunc{leggedAlgebra}{ M }{
		\Big( \Int,\big(R \times M, \Lambda (\alpha,m),(\beta,n) \in R \times M \. (\alpha, \beta, \beta n + \alpha m\big), 
		\NewLine,
		\Lambda k \in \Int_+ \. \If k == 0 \Then R \times \{0\} \Else \If k == 1 \Then \{0\} \times M  \Else  \{0\}   \Big)
	}
	\\
	\Theorem{LeggedAlgebraIsCommutative}{ \forall R \in \ANN \. \forall (\Int_+,A,H) \in \LALGE{R}(\Int_+) \. \forall M \in \LMOD{R} \. 
		\NewLine  (\Int_+,A,H) = \FUNC{leggedAlgebra}(M) \Rightarrow A \in \LCALGE{R} 
	}
	\NoProof
	\\
	\Theorem{LeggedAlgebraIsCommutative}{ \forall R \in \ANN \. \forall (\Int_+,A,H) \in \LALGE{R}(\Int_+) \. \forall M \in \LMOD{R} \. 
		\NewLine  (\Int_+,A,H) = \FUNC{leggedAlgebra}(M) \Rightarrow A \in \LCALGE{R} 
	}
	\NoProof
	\\
	\Theorem{LeggedAlgebraIsPolynomial}{ \forall R \in \ANN \.  \forall M \in \LMOD{R} \.  M : \PGA(R)
	}
	\NoProof
	\\
}\Page{
	\DeclareType{PoincareGradedAlgebra}{ \prod k : \Field \. ? \LALGE{k}(\Int)}
	\DefineType{(\Int,A,H)}{PoincareGradedAlgebra}{\forall n \in \Int \. \dim H_n < \infty }
	\\
	\DeclareFunc{seriesOfPoincare}{\TYPE{PoincareGradedAlgebra}(k) \to \Int\big[[\Int]\big]}
	\DefineNamedFunc{seriesOfPoincare}{\Int,A,H}{P(\Int,A,H)(x)}{ \sum_{n \in \Int} (\dim H_n) x^n}
	\\
	\DeclareType{LorantGradedAlgebra}{ \prod k : \Field \. ? \TYPE{PoincareGradedAlgebra}}
	\DefineType{(\Int,A,H)}{LorantGradedAlgebra}{\exists N \in \Int \. \forall n : \TYPE{Before}(N) \. \dim H_n = 0 }
	\\
	\Theorem{PoincareSeriesProduct}{ \forall k : \Field \. \forall n \in \Nat \. \forall A : n \to \TYPE{LorantGradedAlgebra}(k) \. 
		 \. \NewLine \. P\left(\bigotimes^n_{i=1} A_i \right)(x) = \prod^n_{i=1} P(A_i)(x)
	}
	\NoProof
	\\
	\DeclareType{PositiveHomogeneous}{\prod R \in \ANN \. \prod A \in \LALGE{R}(\Int) \. ?\TYPE{Homogeneous}(A)}
	\DefineType{a}{PositiveHomogeneous}{\deg a > 0}
	\\
	\DeclareType{HilbertModule}{\prod k : \Field \. \prod A \in \LALGE{k}(\Int) \. ?\LMOD{A}(\Int_+)}
	\DefineType{(M,H)}{HilbertModule}{\forall n \in \Int_+ \. \dim_k H_n < \infty \And  M : \TYPE{Noetherian}(A) }
	\\
	\DeclareFunc{seriesOfHilbert}{\prod k : \Field \. \prod A \in \LALGE{k} \. \TYPE{HilbertModule}(A) \to \Int\big[[\Int_+]\big]}
	\DefineNamedFunc{seriesOfHilbert}{M,O}{H(M,O)(x)}{ \sum^{\infty}_{n=0} (\dim_k O_n) x^n}
	\\
	\Theorem{HilbertSeriesTheorem}{ 
		\forall k : \Field \. 
		\forall A \in \LALGE{k}(\Int) \.
		\forall M  : \TYPE{HilbertModule}(A) \.
		\forall n \in \Int_+ \. \NewLine  \. 
		\forall a : n \to \TYPE{PositiveHomogeneous}(A) \. 
		\forall [0] : A = \Big\langle  \{ a_n| n \in \Nat \} \Big\rangle_{\LALGE{R}} \. 
		\forall [00] : A = Z(A)
		\NewLine 
		\exists! Q \in \Int\big[\Int_+\big] \. 
		 H(A)(x) = \frac{Q(M)}{\prod^n_{i=1}(1 - x^{\deg a_i} )}
	}
	& (!) \\
}
\Page{
	\DeclareFunc{structuralPolynomial}{\TYPE{HilbertModule}(A) \to \sum n \in \Int_+ \. 
		\sum a : n \to \TYPE{PositiveHomogneous}(A) \. 
		A = \langle \{ a_i |i \in n \} \rangle  \to \Int\big[\Int_+\big] \.
		}
	\DefineNamedFunc{structuralPolynomial}{M,(n,a,\star)}{Q(M,n,a)}{\THM{HilbertSeriesTheorem}(M,n,a,\star)}
	\\
	\DeclareType{HilbertAlgebra}{\prod k : \Field \.  ?\PGA(k)}
	\DefineType{(\Int,A,H)}{HilbertAlgebra}{Z(A) = A \And A : \FGA(k) } 
	\\
	\DeclareType{HilbertPolynomial}{\prod k : \Field \. \prod A \in \LALGE{k} \. \prod M :  \TYPE{HilbertModule}(k) \. ?\Rats[\Int_+]}
	\DefineType{h}{HilbertPolynomial}{ \forall n \in \Int_+ \. H(M)(x) = \sum^\infty_{n=0} h(n)x^n }
	\\
	\Theorem{HibertPolynomialTheorem}{\forall k : \Field \. \forall A : \TYPE{HilbertAlgebra}(k) \. \forall M : \TYPE{HilbertModule}(A) \. 
		\NewLine \. \exists ! h : \TYPE{HilbertPolynomial}(M)}
	& (!) \\
	\\
	\DeclareFunc{polynomialOfHilbert}{\prod A : \TYPE{HilbertAlgebra}(k) \. \TYPE{HilbertModule}(k)   \to \Rats\big[\Int_+\big]}
	\DefineNamedFunc{polynomialOfHilbert}{M}{h(M)(x)}{ \THM{HilbertPolynomialTHM} }
}
\newpage
\subsection{Skew Tensor Product and Skew Algebras}
\Page{
	\DeclareFunc{doubleMultiindexSign}{ \prod  n \in \Nat \. \Int^n \times \Int^n \to \{1,-1\}  }
	\DefineNamedFunc{doubleMultiindexSign}{I,J}{(-1)^{I,J}}{\If \FUNC{isEven}\left(\sum^{n}_{k=1}\sum^n_{l=k+1} I_l J_k  \right) \Then 1 \Else -1}
	\\
	\DeclareFunc{skewTensorProduct}{ \prod n \in \Nat \. n \to \LALG{R}(\Int)  \to \LALG{R}(\Int)  }	
	\DefineNamedFunc{skewTensorProduct}{(\Int,A,H)}{ \widetilde{\bigotimes}^n_{i=1} (\Int,A_i,H_i)}
	{
		\NewLine \de
		\Bigg( 
			\Int , 
			\bigg(  
				\bigotimes^n_{i=1} A_i,
				\bd \LALGE{R}(\Delta) 
				\Lambda 
				\sum 
					I,J \in \Int^n \. 
					\left( 
						x : \prod^n_{k=1} H_{I_k} , 
						y : \prod^n_{k=1} H_{J_k}
					\right) :  \bigotimes^n_{i=1} A_i \times \bigotimes^n_{i=1} A_i 
					\. \NewLine  \quad \.  
				(-1)^{I,J}\bigotimes^n_{i=1} x_i y_i		
			\bigg), 
			\Lambda 
			N \in \Int \. 
			\bigoplus 
				\sum I \in \Int^n \. 
				\sum^n_{k=1} I_k = N \. 
			\prod^n_{k=1} H_{I_k} 
		\Bigg)
	}
	\\
	\Theorem{AssociativeSkewTensorProductOfAlgebras}{\forall R \in \ANN \. \forall A,B,C \in \LALG{R}(\Int) \. \NewLine (A \widetilde{\otimes} B) \widetilde{\otimes} C \cong_{\LALG{R}(\Int)} 
		A \widetilde{\otimes} (B \widetilde{\otimes} C)}
	\NoProof
	\\
	\DeclareType{SkewAlgebra}{?\LALG{R}(\Int)}
	\DefineType{(\Int ,A, H)}{SkewAlgebra}{ \forall a,b : \TYPE{Homogeneous}(A) \. ab = (-1)^{ij}ba \quad \where \quad a \in H_i \And b \in H_j  }
	\\
	\Theorem{AlternatingAlgebraTHM}{
		\forall R \in \ANN \.  
		\forall (\Int, A, H) : \PGA(R) \. 
		\NewLine 
		\forall [0] : \forall a \in A \. 2a = 0 \Rightarrow a = 0 \. 
			\Big(\forall a \in H_1 \. a^2 = 0\Big) 
			\iff
			A : \TYPE{SkewAlgebra}(R)
		}
	\Assume{L}{\forall a \in H_1 \. a^2 = 0}
	\Say{[1]}{\THM{AlternateIsSkew}(L)}{ \forall a,b \in H_1 \. ab = - ba   }
	\Assume{n,m}{\Nat}
	\Assume{a}{H_1^n}
	\Assume{b}{H_1^m}
	\Conclude{[\ldots*]}{[1]^{nm}}{ \prod^n_{i=1}a_i \prod^m_{i=1}b = (-1)^{n+m} \prod^m_{i=1}b_i \prod^n_{i=1} a_i}
	\DeriveConclude{[L.*]}{[0]\bd^{-1}\TYPE{SkewAlgebra}}{((\Int,A,H) : \TYPE{SkewAlgebra}(R)))}
	\Derive{[1]}{ I(\rightarrow)  }{\LOGIC{Left} \Rightarrow \LOGIC{Right}}
	\Assume{R}{((\Int,A,H) : \TYPE{SkewAlgebra}(R))}
	\Assume{a}{H_1}
	\Say{[2]}{\bd \TYPE{SkewAlgebra}(R)}{ a^2 = - a^2 }
	\Conclude{[a.*]}{ [00][2]}{ a^2 = 0  }
	\DeriveConclude{[a.*]}{I(\iff)[1]I(\Rightarrow)I(\forall)}{\LOGIC{This}}
	\EndProof
}\Page{
	\Theorem{SkewTensorProductTheorem}{\forall n \in \Nat \. \forall (\Int,A,H) : n \to \TYPE{SkewAlgebra}(R) \. \NewLine \.  \widetilde{\bigotimes}^n_{i=1} (\Int,A_i,H_i) : \TYPE{SkewAlgebra}(R) }
	\Assume{(\Int,A,H),(\Int,B,H')}{\TYPE{SkewAlgebra}(R)}
	\Assume{i,i',j,j'}{\Int}
	\Assume{a}{H_i}
	\Assume{x}{H_j}
	\Assume{b}{H_{i'}'}
	\Assume{y}{H_{j'}'}
	\Say{[1]}{ \bd \FUNC{SkewTensorProduct} }{ (a \widetilde{\otimes} b)(x \widetilde{\otimes} y) = (-1)^{ij'} ax \widetilde{\otimes} by  }
	\Conclude{[\ldots*]}{\bd \FUNC{SkewTensorProduct} \bd \TYPE{SkewAlgevra}(R)(A,B)[1]\bd (-1) \bd \RING(\Int)}{
		\NewLine :
		(x \widetilde{\otimes} y)(a \widetilde{\otimes} b) = 
		(-1)^{ji'} xa \widehat{\otimes} yb =   
		(-1)^{ i'j} \Big( (-1)^{ij}ax \Big) \widetilde{\otimes} \Big( (-1)^{i'j'} by \Big) = 
		(-1)^{ij + i'j' + i'j - ij'} (a \widetilde{\otimes} b) (x \widetilde{\otimes} y) = \NewLine = 
		(-1)^{ij + i'j' + i'j' + ij'} (a \widetilde{\otimes} b) (y \widetilde{\otimes} x) =
		(-1)^{(i + i')(j + j')} (a \widetilde{\otimes} b) (y \widetilde{\otimes} x) 
	}
	\DeriveConclude{[\ldots*]}{\bd \FUNC{SkewTensorProduct} \bd^{-1}\TYPE{SkewAlgebera}}{ \Big( (\Int,A,H) \widetilde{\otimes} (\Int,B,H') : \TYPE{SkewAlgebra}(R) \Big)   }
	\DeriveConclude{[*]}{\THM{AssociateveSkewTensorProductOfAlgebras}}{\LOGIC{This}}
	\EndProof
	\\
	\DeclareFunc{twistingIsomorphism}{\prod A,B \in \LALGE{R}(\Int) \. A \otimes B \Arrow{\LMOD{R}} B \otimes A}
	\DefineNamedFunc{twistingIsmorphism}{}{\tau_{A,B}}{ \bd \LALGE{R}(\Int) \Lambda n,m \in \Int \. \Lambda a \in A_n \. \Lambda b \in B_m \. (-1)^{mn}b \otimes a } 
	\\
	\Theorem{TwistingIsomorphismTheorem}{\forall A,B \in \LALGE{R}(\Int) \. \tau_{A,B} : A \widetilde{\otimes} B \ToIso{\LALGE{R}} B \widetilde{\otimes} A}
	\Say{[1]}{\bd \tau_{A,B}}{\tau_{A,B}(e_A \otimes e_B) = e_B \otimes e_A}
	\Assume{n,n',m,'m'}{\Int}
	\Assume{a}{A_n}
	\Assume{a'}{A_{n'}}
	\Assume{b}{B_{m}}
	\Assume{b'}{B_{m'}}
	\Say{[1]}{\bd \FUNC{SkewTensorProduct} \bd \FUNC{twistingIsomorphism} \bd (-1)}{ 
		\NewLine
		\tau_{A,B}\Big( a \otimes b \cdot a' \otimes  b' \Big) = 
		(-1)^{n'm}\tau_{A,B} \Big( aa' \otimes bb' \Big) = 
		(-1)^{ nm' + nm + n'm + 2n'm }( bb' \otimes aa') =  \NewLine = 
		(-1)^{  nm + n'm + nm' }( bb' \otimes aa')   	
	}
	\Say{[2]}{\bd \FUNC{twistingIsomorphism} \bd \FUNC{SkewTensorProduct}}
	{
		\NewLine 
		\tau_{A,B}(a \otimes b) \cdot \tau_{A,B}(a' \otimes b') =
		(-1)^{nm+ n'm' + nm'} \Big( b\otimes a \cdot b' \otimes a' \Big) = 
		(-1)^{nm + n'm' + nm'} bb' \otimes aa'
	}
	\Conclude{[\ldots*]}{[1][2]}{\tau_{A,B}(a \otimes b)\tau_{A,B}( a' \otimes b') = \tau_{A,B}(a \otimes b \cdot a' \otimes b')}
	\DeriveConclude{[*]}{[1]\bd \FUNC{SkewTensorProduct}}{\LOGIC{This}}
	\EndProof
	\\
	\Theorem{SkewTensorProductPermutation}{ \forall n \in \Nat \. \forall A : n \to \LALGE{R}(\Int) \. \forall \sigma \in S_n \. \widetilde{\bigotimes}^n_{i=1} A_i \cong_{\LALGE{R}} \widetilde{\bigotimes}^n_{i=1} A_{\sigma(i)}  }
	\NoProof
}
\Page{
	\Theorem{SkewMultiplicationMorphism}{\forall A : \TYPE{SkewAlgebra}(R) \. \mu_A : A \widetilde{\otimes} A : \Arrow{\LALGE{R}} A}
	\Say{[1]}{ \bd \mu_A  }{ \mu_A(e \otimes e) = e}
	\Assume{n,n',n,'m'}{\Int}
	\Assume{a}{A_n}
	\Assume{a'}{A_{n'}}
	\Assume{b}{A_{m}}
	\Assume{b'}{A_{m'}}
	\Say{[1]}{\bd \FUNC{skewTensorProduct}\bd\mu}{ 
		\mu(a \otimes b \cdot a' \otimes b') = 
		(-1)^{n'm}\mu(aa' \otimes bb') = 
		(-1)^{n'm} aa'bb'
	}
	\Say{[2]}{\bd\mu\bd \TYPE{SkewAlgebra}}{\mu(a \otimes b)\mu(a' \otimes b') =  aba'b' = (-1)^{n'm}aa'bb'}
	\Conclude{\ldots*}{[1][2]}{\mu\Big( a \otimes b \cdot a' \otimes b'\Big) = \mu(a \otimes b)\mu(a' \otimes b')}
	\DeriveConclude{[*]}{[1]\bd \FUNC{SkewTensorProduct}}{\LOGIC{This}}
	\EndProof
	\\
	\DeclareFunc{doublingDegrees}{\prod R \in \ANN \. \LALGE{R}(\Int) \to \LALGE{R}(\Int)}
	\DefineNamedFunc{doublingDegrees}{\Int,A,H}{(\Int,A,H)^{\mathrm{dd}}}{\Big(\Int,A,\Lambda n \in \Int \. \If \FUNC{isOdd}(n) \Then \{0\} \Else H_{\frac{n}{2}} \Big) } 
	\\
	\Theorem{DoublingDegreesTensorDistributive}{\forall A,B \in \LALGE{R}(\Int) \. A^{\mathrm{dd}} \otimes B^{\mathrm{dd}} = (A \otimes B)^{\mathrm{dd}} }
	\Say{X}{\A^{\mathrm{dd}} \otimes \B^{\mathrm{dd}}}{\LALGE{R}(\Int)}
	\Say{Y}{\Big(A \otimes B)^{\mathrm{dd}}}{\LALGE{R}(\Int)}
	\Assume{n}{\Int}
	\Assume{[1]}{(n : \TYPE{Odd})}
	\Say{[2]}{\bd \FUNC{integralTensorProduct}}{ X_n = \bigoplus \sum k,l \in \Int \. k + l = n \. A^{\mathrm{dd}}_k \otimes B^{\mathrm{dd}}_l }
	\Assume{k,l}{\Int}
	\Assume{[3]}{k + l = n}
	\Say{[4]}{\THM{OddSum}[3]}{\Big( k : \TYPE{Odd} \Big| l : \TYPE{Odd}\Big)}
	\Conclude{[\ldots*]}{\bd \FUNC{doublingDegrees}[4]\bd \FUNC{TensorProduct}}{ A^{\mathrm{dd}}_k \otimes B^\mathrm{dd}_l = \{0\} }
	\Derive{[3]}{\bd \FUNC{directSum}[2]}{X_n = 0}
	\Conclude{[1.*]}{\bd \FUNC{doublingDegrees}[3]}{Y_n = X_n}
	\Derive{[1]}{I(\Rightarrow)}{n : \TYPE{Odd} \Rightarrow Y_n = X_n}
	\Assume{[2]}{ (n : \TYPE{Even}) }
	\Say{[3]}{\bd \FUNC{integralTensorProduct}}{ X_n = \bigoplus \sum k,l \in \Int \. k + l = n \. A^{\mathrm{dd}}_k \otimes B^{\mathrm{dd}}_l }
	\Assume{k,l}{\Int}
	\Assume{[4]}{k + l = n}
	\Assume{[5]}{ A^{\mathrm{dd}}_k \otimes B^{\mathrm{dd}}_l \neq \{0\}}
	\Say{[6]}{[5]\bd \FUNC{doublingDegrees}}{(k,l : \TYPE{Even})}
	\Conclude{[\ldots*]}{\bd \FUNC{doublingDegrees}[6]}{ A^{\mathrm{dd}}_k \otimes B^{\mathrm{dd}}_l = A_{\frac{k}{2}} \otimes B_{\frac{k}{2}}  } 
	\DeriveConclude{[2.*]}{[2]\ByConstr Y}{ X_n = Y_n}
	\Derive{[2]}{I(\Rightarrow)}{n : \TYPE{Even} \. \Rightarrow Y_n = X_n}
	\Conclude{[n.*]}{E(|)\THM{EvenOrOdd}[1][2]}{Y_n = X_n}
	\Derive{[1]}{I(\forall)}{\forall n \in \Int \. Y_n = X_n}
	\Conclude{[*]}{\ByConstr X \ByConstr Y [1]}{X = Y}
	\EndProof
}
\newpage
\subsection{Derivations on Algebras}
\Page
{
	\DeclareType{MapOfDegree}{\prod (\Delta,A,H) : \TYPE{GradedAlgebra}(R) \. ?(A \to A)}
	\DefineType{f}{MapOfDegree}{\exists! \delta \in \Delta : \forall \alpha \in \Delta \. f(H_\alpha) \subset H_{\alpha + \delta}}
	\\
	\DeclareFunc{mapDegree}{\TYPE{MapOfDegree}(\Delta,A,H) \to \Delta }
	\DefineNamedFunc{mapDegree}{f}{\deg f}{\bd \TYPE{MapOfDegree}(\Delta,A,H)(f)}
	\\
	\DeclareType{Derivation}{\prod A \in \LLG{R} \. ?\End_{\LMOD{R}}(A)}
	\DefineType{D}{Derivation}{ \forall a,b \in A \.  D[a,b] = [Da,b] + [a,Db]  }
	\\
	\DeclareType{GradedDerivation}{\prod (\Int,A,H) \. ?\TYPE{Derivation}(A)}
	\DefineNamedType{D}{GradedDerivation}{D \in \mathcal{D}(A,H) }{ D : \TYPE{MapOfDegree}(\Int,A,H) \And \deg D = -1 }
	\\
	\Theorem{DerivationOfProduct}{\forall A \in \LALG{R} \. \forall D : \TYPE{Derivation}(A) \. \forall n \in \Nat \. \forall a : n \to A \. 
		 \NewLine D \prod^n_{i=1} a_i = \sum^n_{k=1} \prod^{k-1}_{i=1} a_i D a_k \prod^{n}_{i=k+1} a_i}
	\NoProof
	\\
	\Theorem{ModuleOfDerivations}{\forall A \in \LLG{R} \. \TYPE{Derivatopn}(A) \in \LMOD{R}}
	\NoProof
	\\
	\Theorem{ModuleOfDerivations2}{\forall (\Delta,A,H) : \TYPE{GradedAlgebra}(A) \. \D(A,H) \in \LMOD{R}  }
	\NoProof
	\\
	\Theorem{NeutralDerivation}{\forall A \in \LLGE{R} \. \forall D : \TYPE{Derivation}(A) \. De = 0}
	\Say{[1]}{\bd \TYPE{Neutral}(e)\bd \TYPE{Derivation}(D)\bd \TYPE{Neutral}(e)}{De = D[e,e] = [De,e] + [e,De] = 2De}
	\Conclude{[*]}{\bd \ABEL(A)[1]}{De = 0 }
	\EndProof
}\Page{
	\Theorem{PolynomialDerivationsCommute}{\forall (A,H) : \PGA(R) \. \forall D,D' \in \D(A,H) \.  \NewLine \. DD' = D'D }
	\Say{[1]}{\THM{FreeCoeffiecientLemma}(A)}{H_0 = Re}
	\Assume{a}{H_1}
	\Say{(\alpha,[2])}{\bd \FUNC{mapDegree}\bd \D(A,H)(D)[1]}{\sum \alpha \in R \. Da = \alpha e}
	\Say{(\beta,[3])}{\bd \FUNC{mapDegree}\bd \D(A,H)(D')[1]}{\sum \beta \in R \. D'a = \beta e}
	\Conclude{[a.*]}{[2] \bd \LMOD{R}(A,A)(D')  [1] \bd \LMOD{R}(A,A)(D') [3]}{ D'Da = D' \alpha e = \alpha D' e = 0 = \beta D e = D \beta e = DD'a }
	\Derive{[2]}{I(\forall)}{\forall a \in H_1 \. D'Da = D'Da}
	\Assume{n}{\Nat}
	\Assume{b}{H_n}
	\Say{\big(m,a,[3]\big)}{\bd \PGA(R)(A)(b)}{\sum m \in \Nat \. \sum  a :  H_1^{m \times n} \. b = \sum^m_{i=1} \prod^n_{j=1} a_{i,j} }
	\Conclude{[n.b.*]}{[3](D'Da)\THM{DerivationProduct}^2(D)(D')[2](a)\THM{DerivationProduct}^2(D')(D)[3]}
	{
		\NewLine
		D'D b = 
		D'D\sum^m_{i=1}\prod^n_{j=1} a_{i,j} =  
		D' \sum^m_{i=1} \sum^n_{k=1} \prod^{k-1}_{j=1} a_{i,j} D a_{i,k} \prod^n_{j=k+1} a_{i,j} = \NewLine = 
		\sum^m_{i=1} \sum^n_{k=1}\sum^n_{k \neq l = 1} \prod^{\min(k-1,l-1)}_{j=1} a_{i,j} \Big( [k < l] D a_{i,k} + [l < k] D' a_{i,l}\Big)
		\prod^{\max(k-1,l-1)}_{j = \min(k+1,l+1)} a_{i,j} \Big( [k > l] D a_{i,k} + [l > k] D' a_{i,l}\Big) \NewLine  \prod^n_{j = \max(k+1,l+1)} a_{i,j} +
		\sum^n_{k=1} \prod^{k-1}_{i=1} a_i D'D a_k \prod^{n}_{i=k+1} a_i = \NewLine =
		\sum^m_{i=1} \sum^n_{l=1}\sum^n_{l \neq k = 1} \prod^{\min(k-1,l-1)}_{j=1} a_{i,j} \Big( [k < l] D a_{i,k} + [l < k] D' a_{i,l}\Big)
		\prod^{\max(k-1,l-1)}_{j = \min(k+1,l+1)} a_{i,j} \Big( [k > l] D a_{i,k} + [l > k] D' a_{i,l} \Big) \NewLine \prod^n_{j = \max(k+1,l+1)} a_{i,j} +
		\sum^n_{k=1} \prod^{k-1}_{i=1} a_i DD' a_k \prod^{n}_{i=k+1} a_i =
		D \sum^m_{i=1} \sum^n_{l=1} \prod^{l-1}_{j=1} a_{i,j} D' a_{i,l} \prod^n_{j=l+1} a_{i,j} =  
		D D'\sum^m_{i=1}\prod^n_{j=1} a_{i,j} = 
		DD'b  
	}
	\Derive{[3]}{I^2(\forall)}{\forall n \in \Nat \. \forall b \in H_n \. D'D b = DD' b}
	\Conclude{[*]}{\bd \TYPE{GradedAlgebra}(\Int,A,H)[3]}{D'D = DD'}
	\EndProof
	\\
	\Theorem{PolynomialDerivationsAgree}{\forall (A,H) : \PGA(R) \. \forall D,D' \in \D(A,H) \.  \NewLine \. 
		\forall [0] : \forall a \in H_1 \. Da = D'a \. D = D'}
	\NoProof
	\\
	\DeclareType{SkewDerivation}{\prod (\Int,A,H) \in \LALGE{R}(\Int) \. ?\TYPE{MapOfDegree}(\Int,A,H)}
	\DefineNamedType{D}{SkewDerivation}{D \in \widetilde{\D}(A,H)}{ \deg D = -1 \And  \NewLine \And \forall n \in \Nat \. \forall a \in A \. \forall h \in H_n  D (ha) = (Dh)a + (-1)^n h Da  }
}\Page{
	\Theorem{SkewDerivationOfProduct}{\forall (\Int,A,H) \in \LALGE{R}(\Int) \. \forall D \in \widetilde{\D}(A,H) \. \forall n \in \Nat \. \forall a : n \to H_1 \. 
		 \NewLine D \prod^n_{i=1} a_i = \sum^n_{k=1} (-1)^{k+1}\prod^{k-1}_{i=1} a_i D a_k \prod^{n}_{i=k+1} a_i}
	\NoProof
	\\
	\Theorem{NeutralSkewDerivation}{\forall (\Int,A,H) \in \LALGE{R}(\Int) \. \forall D : \TYPE{Derivation}(A,H) \. De = 0}
	\Say{[1]}{\THM{UnitDegree}(A,H)}{\deg e = 0}
	\Say{[2]}{\bd \TYPE{Neutral}(e)\bd \TYPE{SkewDerivation}(D)\bd \TYPE{Neutral}(e)}{De = De^2 = (De)e + e(De) = 2De}
	\Conclude{[*]}{\bd \ABEL(A)[1]}{De = 0 }
	\EndProof
	\\
	\Theorem{PolynomialSkewDerivationsAnticommute}{\forall (A,H) : \PGA(R) \. \forall D,D' \in \D(A,H) \.  \NewLine \. DD' + D'D = 0 }
	\Say{[1]}{\THM{FreeCoeffiecientLemma}(A)}{H_0 = Re}
	\Assume{a}{H_1}
	\Say{(\alpha,[2])}{\bd \FUNC{mapDegree}\bd \D(A,H)(D)[1]}{\sum \alpha \in R \. Da = \alpha e}
	\Say{(\beta,[3])}{\bd \FUNC{mapDegree}\bd \D(A,H)(D')[1]}{\sum \beta \in R \. D'a = \beta e}
	\Conclude{[a.*]}{[2][3]\bd \LMOD{R}(A,A)(D)(D')\THM{NeutralSkewDerivation}(D)(D')}{(DD' + D'D) a = \beta D e + \alpha D' e = 0}
	\Derive{[2]}{I(\forall)}{ \forall a \in H_1  \. (DD' + D'D) a = 0}
	\Assume{n}{\Nat}
	\Assume{[3]}{\forall a \in H_n \. (DD' + D'D)(a) = 0}
	\Assume{a}{H_1}
	\Assume{b}{H_n}
	\Conclude{[a.b.*]}{ \bd \TYPE{SkewDerivation}(D)(D')\bd\LALGE{R}(A)\bd\LMOD{R}(A,A)(D)(D')[2][3] }
	{
		(DD' + D'D)(ab) = 
		(DD'a)b + a(DD'b) - Da D'b + D'a Db + (D'Da)b + a(D'Db) -D'a Db + DaD'b = 
		\big((DD' + D'D)a\big)b + a\big((DD' + D'D)b\big) = 0 
	}
	\DeriveConclude{[n.3.*]}{\bd \PGA(A,H)}{\forall a \in H_{n+1} \. (DD' + D'D)a = 0}
	\Derive{[3]}{\bd \Nat[2] }{\forall n \in \Nat \. \forall a \in H_n \. (DD' + D'D)a = 0}
	\Conclude{[*]}{\bd \LALGE(R)(\Int)[3]}{ DD' + D'D = 0}
	\EndProof
	\\
	\Theorem{PolynomialSkewDerivationsZeroSquare}{\forall (A,H) : \PGA(R) \. \forall D \in \D(A,H) \.  \NewLine \. D^2 = 0 }
	\NoProof
	\\
	\Theorem{PolynomialSkewDerivationsAgree}{\forall (A,H) : \PGA(R) \. \forall D,D' \in \widetilde{\D}(A,H) \.  \NewLine \. 
		\forall [0] : \forall a \in H_1 \. Da = D'a \. D = D'}
	\NoProof
}
\Page{
	\DeclareType{DerivationOfDegree}{\prod A : \LALGE{R}(\Int) \. \prod n \in \Int \. ?\TYPE{MapOfDegree}(R)}
	\DefineNamedType{D}{DerivationOfDegree}{D \in \D^n(A)}{\deg D = n \And D : \TYPE{Derivation}(A)}
	\\
	\Theorem{GeneralisedDerivationTHM}{\forall A \in \LALGE{R}(\Int) \forall n,m \in \Int \. \forall D \in \D^n(A) \. \forall D' \in \D^m(A) \. 
		\NewLine \. DD' - D'D \in \D^{n+m}(A)}
	\Assume{a,b}{A}
	\Conclude{[a,b.*]}{\bd \TYPE{Derivation}(D)\bd \TYPE{Derivation}(D') \bd \LMOD{R}(A,A)(DD')(D'D)}
	{
		\NewLine :
		\big( DD' - D'D \big)(ab) =
		D\Big( (D'a)b + a(D'b)  \Big) - D'\Big( (Da)b + a(Db) \Big) = \NewLine
		(DD'a)b + (D'a)(Db)  + (Da)(D'b) + a(DD'b) - (D'Da)b - (Da)(D'b) - (D'a)(Db) - a(D'Db) = \NewLine 
		\Big(\big( DD' - D'D  \big)a\Big) + a\Big( \big( DD' - D'D \big)b \Big) 
	}
	\DeriveConclude{[*]}{\bd \D^{n + m}(A)}{DD' - D'D \in \D^{n+m}(A)}
	\EndProof
	\\
	\DeclareFunc{mainInvolution}{\prod A \in \LALGE{R}(\Int) \. \Aut_{\LALGE{R}(\Int)}(A)}
	\DefineNamedFunc{mainInvolution}{}{J_A}{\bd \LALGE{R}(\Int)(A)\Lambda n \in \Int \. \Lambda a \in A_n \. J_A(a) = (-1)^n a}
	\\
	\DeclareType{SkewDerivationOfDegree}{\prod A : \LALGE{R}(\Int) \. \prod n \in \Int \. ?\TYPE{MapOfDegree}(R)}
	\DefineNamedType{D}{SkewDerivationOfDegree}{D \in \widetilde{\D}^n(A)}{\deg D = n 
		\And  \NewLine \And   \forall a,b \in A \.   
		D(ab) = (Da)b  + J^n_A(a)Db
	}
	\\
	\Theorem{GeneralisedSkewDerivationTHM}{\forall A \in \LALGE{R}(\Int) \forall n,m \in \Int \. \forall D \in \D^n(A) \. \forall D' \in \D^m(A) \. 
		\NewLine \. DD' - (-1)^{nm}D'D \in \D^{n+m}(A)}
	\Assume{a,b}{A}
	\Conclude{[a,b.*]}{\bd \TYPE{SkewDerivationOfDegree}(D)\bd \TYPE{SkewDerivationOfDegree}(D') \bd \LMOD{R}(A,A)(DD')(D'D)}
	{
		\NewLine :
		\big( DD' - (-1)^{nm}D'D \big)(ab) =
		D\Big( (D'a)b + J^m(a)(D'b)  \Big) - (-1)^{nm}D'\Big( (Da)b + J^n{a}(Db) \Big) = \NewLine =
		(DD'a)b + J^n(D'a)(Db)  +  D J^{m}(a)(D'b) + J^{m + n}(a)(DD'b) -  \NewLine -
		(-1)^{nm}\Big( (D'Da)b -  J^{m}(Da)(D'b) -  D' J^n(a)(Db) - J^{m+n}(a)(D'Db) \Big)= \NewLine 
		(DD'a)b + J^n(D'a)(Db)  + (-1)^{nm} J^{m}( D a)(D'b) + J^{m + n}(a)(DD'b) -  \NewLine -
		(-1)^{nm}\Big( (D'Da)b -  J^{m}(Da)(D'b)  - (-1)^{nm}  J^n(D'a)(Db) - J^{m+n}(a)(D'Db) \Big)= \NewLine 
		\Big(\big( DD' - (-1)^{nm}D'D  \big)a\Big) + J^{m + n}(a)\Big( \big( DD' - (-1)^{nm}D'D \big)b \Big) 
	}
	\DeriveConclude{[*]}{\bd \D^{n + m}(A)}{DD' - D'D \in \D^{n+m}(A)}
	\EndProof
}
\newpage
\subsection{Finite-Dimensional Associative Algebras over Fields}
\Page{
	\DeclareType{AlgebraRepresentation}{ \prod R \in \ANN \. \prod A \in \LALGE{R} \. \prod M \in \LMOD{R} \. \NewLine \. ?\Big(A \Arrow{\LALGE{R}} \End_{\LMOD{R}}(M) \Big) } 
	\\
	\DeclareType{Faithful}{?\TYPE{AlgebraRepresentation}(R,A,M)}
	\DefineType{\rho}{Faithful}{\rho : A \ToInj \End_{\LMOD{R}}(M)}
	\\
	\DeclareFunc{lefttRegularRepresentation}{ \prod R \in \ANN \. \prod A \in \LALGE{R} \. \TYPE{Faithul}(R;A,A)}
	\DefineNamedFunc{leftRegularRepresentation}{a}{L_A(a)}{\Lambda b \in A \. ab} 
	\\
	\DeclareFunc{leftRegularMatrixRepresentation}{ \prod k : \Field \. \prod A \in \LALGE{R} \And \FDVS{R} \. \NewLine \. \Basis(A)  \to  \TYPE{Faithul}(R,A,A^{\dim A \times \dim A}) }
	\DefineNamedFunc{leftRegularMatrixRepresentation}{e,a}{L_{A,e}(a)}{ {L_A(a)}^{e,e} }
	\\
	\Theorem{FiniteRankIdealProperty}{\forall k : \Field \. \prod A \in \LALGE{R} \. \Big\{ a \in A | \rank L_{A,e}(a) \le \infty  \Big\} : \TYPE{Ideal}(A)}
	\NoProof
	\\
	\DeclareFunc{finiteRankIdeal}{ \prod k : \Field \. \prod A \in \LALGE{R} \. \Ideal(A)  }
	\DefineNamedFunc{finiteRankIdeal}{}{I_{\rank < \infty}(A)}                                                               
	\\
	\Theorem{FiniteRankIdealTHM}{\forall k : \Field \. \prod V \in \VS{R} \. \forall I : \Ideal(\End_{\VS{k}}(V)) \. \NewLine \.  
		\forall [0] : I \neq 0  \.  I_{\rank < \infty}(\End_{\VS{k}}(V)) \subset I  
	}
	\Say{\Big(B,[3]\Big)}{ \bd I [0]}{\sum B \in I : B \neq 0}
	\Assume{A}{I_{\rank < \infty}(\End_{\VS{k}}(V))}
	\Say{\Big(F,[1]\Big)}{\bd I_{\rank < \infty}(A)\bd \rank}{\sum F : \rank A \to \End_{\VS{k}} \. A = \sum^{\rank A} \forall i \in \rank A \. \rank F = 1 }
	\Assume{i}{\rank f}
	\Say{\Big(v,u,[2]\Big)}{\THM{Rank1Reperesentation}(F_i)[1]}{\sum u,v \in V \. F_1(u) = v \And \ker F_1 \oplus \Span(u) = V}   
	\Say{\Big(x,[4]\Big)}{\bd 0 [3]}{\sum x\in V \. Bx \neq 0}  
	\Say{[4]}{\bd T_{B(x),v},T_{u,x}[2]}{F_i = T_{u,x} B T_{B(x),v}}
	\Conclude{[i.*]}{\bd \TYPE{Ideal}[4]}{F_i \in I}
	\DeriveConclude{[A.*]}{\bd \TYPE{Ideal}[1]}{A \in I} 
	\DeriveConclude{[*]}{\bd \TYPE{Subset}}{I_{\rank < \infty}(\End_{\VS{k}(V)}) \subset I}
	\EndProof
}
\Page{
	\DeclareType{Algebraic}{ \prod k : \Field \. \prod A \in \LALGE{k} \. ?A }
	\DefineType{a}{Algebraic}{ \exists f \in k[x]  \. f(a) = 0  } 
	\\
	\DeclareFunc{minimalPolynomial}{ \prod k : \Field \. \prod A \in \LALGE{k} \. \TYPE{Algebraic}(A) \to k[x] }
	\DefineNamedFunc{minimalPolynomial}{a}{M_a}{ \bd \PID\Big(k[x]\Big) \bd \TYPE{Algebraic}(a) }
	\\
	\Theorem{AlgebraicSubalgeraStructure}{ 
		\forall k : \Field \. 
		\forall A \in \LALGE{k} \. 
		\forall a : \TYPE{Algebraic}(A) \. \NewLine \. 
		k[a] \cong_{\LALGE{k}} \frac{k[x]}{M_a}
	}
	\NoProof
	\\
	\Theorem{FiniteDimensionalIsAlgebraic}
	{
		\forall k : \Field \. 
		\forall A \in \LALGE{k} \. \NewLine \.
		\dim A < \infty \Rightarrow \forall a \in A \. a : \TYPE{Algebraic}
	}
	\NoProof
	\\
	\Theorem{AlgebraicInvertibility}
	{
		\forall k : \Field \.
		\forall A \in \LALGE{k} \. 
		\forall a : \TYPE{Algebraic}(A) \. \NewLine \. 
		a \in A^* \iff a \in A^\times 
	}
	\NoProof
	\\
	\Theorem{MinimalAlgebraicRoots}
	{
		\forall k : \Field \.
		\forall A \in \LALGE{k} \.
		\forall a : \TYPE{Algebra}(A) \.  
		\forall \rho \in k \.\NewLine  \. 
		\rho \in \FUNC{roots}(k,m_a(x)) \iff 
		a - \rho e \not \in A^\times
	}
	\NoProof
	\\
	\DeclareFunc{spectreOfElement}
	{
		\prod k : \Field \.
		\prod A \in \LALGE{k} \.
		\TYPE{Algebraic}(A) \to \TYPE{Measure}(k,2^k) 
	}
	\DefineNamedFunc{spectreOfElement}{a}{\sigma(a)}{ \Lambda K \subset k \. \sum_{\alpha \in K} \max \Big\{ t \in \Int_+ : (x - \alpha)^t | m_a(x)  \Big\}  }
}
\Page{
	\Theorem{CommutativeSpectre}{
			\forall k : \ACF \. 
			\forall A : \LALGE{k} \.
			\forall a,b : \TYPE{Algebraic}(A) \. \NewLine  
			\supp \sigma(ab) = \supp \sigma(ba)
	}
	\Assume{\rho}{A}
	\Assume{[1]}{\rho \neq 0}
	\Assume{[2]}{ \sigma(ab)\{\rho\} = 0  }
	\Say{[3]}{\THM{MinimalAlgebraicRoots}\;\THM{AlgebraicInvertability}[2]}{ ab - \rho e \in A^* }
	\Say{[4]}{ \bd \LALGE{k}(A) \bd A^* [3] \bd \ABEL(A)}
	{
		(ba - \rho e) \Big( b(ab - \rho e )^{-1} a   - e \Big) = 
		b (ab - \rho e) (ab  - \rho e)^{-1} a  - ba + \rho e = 
		ba - ba + \rho e = \rho e
	}
	\Say{[5]}{ [1]\THM{AlgebraicInvertability}[4] }{ ba - \rho e \in A^*}
	\Conclude{[\rho.*]}{ \THM{MinimalAlgebraicRoots}[5] }{ \sigma(ba)\{\rho\} = 0}  
	\Derive{[1]}{\bd \subset}{ \supp \sigma(ba) \subset \supp \sigma(ab)  }
	\Assume{[2]}{\sigma(ab)\{0\} = 0 }
	\Say{[3]}{\THM{AlgebraicInvertability}\;\THM{MinimalAlgebraicRoots}}{ab \in A^* }
	\Say{[4]}{\bd \LALGE{k}(A,\End_{\VS{k}}(A))(L_A)[3]}{  L_A(a)L_A(b) = L_A(ab) \in \Aut_{\VS{k}}(A)}
	\Say{[5]}{ \THM{InvertibleProduct} [4] }{ L_A(a),L_A(b) \in \Aut_{\VS{k}}(A)  }
	\Say{[6]}{\bd \LALGE{k}(A,\End_{\VS{k}}(A))(L_A)[5]}{a,b \in A^\times}
	\Say{[7]}{\THM{AlgebraicInvertability}}{a,b \in A^*}
	\Say{[8]}{\bd R^* [7]}{ba \in A^*}
	\Conclude{[9]}{\THM{AlgebraicInvertability}\;\THM{MinimalAlgebraicRoots}}{\sigma(ba)\{0\} = 0 }
	\Derive{[10]}{\LOGIC{SymmetricArgument}\bd \TYPE{Subset}}{\supp (ab) = \supp (ba)}
	\EndProof
	\\
	\DeclareFunc{quaternions}{ \LALGE{\Reals}} 
	\DefineNamedFunc{quaternions}{ }{\mathbb{H}}{\frac{\mathrm{Free}_{\LALGE{\Reals}}\{ \mathrm{i},\mathrm{j},\mathrm{k}\}}{\Big( \mathrm{i}^2 + 1,\mathrm{j}^2 +1, \mathrm{k}^2 + 1, \mathrm{ij} - \mathrm{k} \Big)}}
	\\
	\Theorem{quaternionicIdentities}{ \mathrm{ik} = -\mathrm{j} \And \mathrm{kj} = -\mathrm{i}  }
	\NoProof
	\\
	\Theorem{tripleQuaternionicIdentity}
	{
		\mathrm{jik} = 1
	}
	\NoProof
	\\
	\Theorem{ReversedQuaternionicIdentities}
	{ \mathrm{ij} = -k \And \mathrm{ki} = \mathrm{j} \And \mathrm{jk} = \mathrm{i} }
	\NoProof
	\\
	\Theorem{DimensionOfQuaternions}{\dim \Quat = 4}
	\NoProof
}
\Page{ 
	\Theorem{QuaternionicBasis}{\{1,\mathrm{i,j,k} : \TYPE{Basis}(\Quat)\}}
	\NoProof
	\\
	\Theorem{InvetibleQuaternions}{ \mathbb{H} : \TYPE{DivisionAlgebra}(\Reals)}
	\NoProof
	\\
	\Theorem{AlgebraiclyClosedDivision}{\forall k : \ACF \. \forall A : \TYPE{DivisionAlgebra}(k) \. A : \Field}
	\\
	\Theorem{WidderburnsTheorem}{ \forall q : \TYPE{PrimePower} \. \forall A : \TYPE{DivisionAlgebra}(\mathbb{F}_q) \. \NewLine \. \forall [0] : |A| < \infty \.  A : \Field  }
	\Say{(p,k,[00])}{\bd \TYPE{PrimePower}}{\sum p : \TYPE{Prime}(\Int) \. \sum k \in \Nat \. q = p^k  }
	\Say{[1]}{\THM{CommutativeByConjugation}\bd \FUNC{centre}[0]}{|A^*| = |Z^*(A)| + \sum_{\gamma \in C(A,*) : |\gamma| \neq 1} |\gamma| } 
	\Say{[2]}{ \bd \Field (Z(A))  }{\Big(Z(A) \in \Field \Big)}
	\Say{[3]}{\bd \LALGE{\mathbb{F}_q}(A)[2]}{A \in \VS{Z(A)}}
	\Say{n}{|Z(A)|}{\Nat}
	\Say{(5,t)}{[4][0]\bd \dim_{Z(A)} A}{ \sum t \in \Nat  |A| = n^t  } 
	\Say{(\alpha,[6])}{\THM{ClassEquation}(A)}{ \sum \alpha \subset A \. \sum_{\gamma \in C(A,*) : |\gamma| \neq 1} |\gamma| = \sum_{a \in \alpha} \frac{|A^*|}{|Z_A^*(a)|}  } 
	\Assume{a}{\alpha}
	\Say{[7]}{\bd \VS{Z(A)}\bd (Z_A(a))}{\Big( Z_A(a) : \VS{Z(A)}  \Big)}
	\Conclude{(s(a),a.*)}{\bd \dim_{Z(A)} Z_A(a)}{ \sum s(a) \in \Nat \. |Z_A(a)| = n^{s(a)}   }
	\Derive{(s,[7])}{I\left(\sum\right)}{\sum s : \alpha \to \Nat \. \forall a \in \alpha \. |Z_A(a)| = n^{s(a)}} 
	\Say{[8]}{[5][6][7]}{ n^t - 1 = n - 1 + \sum_{a \in \alpha} \frac{n^t - 1}{n^{s(a)} - 1}}
	\Say{[9]}{ \bd \Int }{  n( n^{t-1} - 1) = \sum_{a \in \alpha} \frac{n^t - 1}{n^{s(a)} - 1 }   } 
	\Say{[10]}{\THM{SubgroupOrder}(A,Z_A(a))[5][7]}{\forall a \in \alpha \. n^{s(a)} - 1 | n^t - 1  }
	\Say{[11]}{\THM{CyclicDivisibility}[10]}{ s(a) | t}
	\Say{[12]}{\THM{CyclotomicDivision}[11][8]}{  Q_t(n) | n - 1   }
	\Say{[13]}{\THM{ComplexDifferenceEstimates}(n,1,\TYPE{PrimitiveRootsOfUnity}(\Complex,t))\NewLine\THM{IncreasingProduct}}{t > 1 \Rightarrow |Q_t(n)| > n - 1}
	\Say{[14]}{\THM{NaturalDivisorsAreLess}[12][13]}{t = 1}
	\Conclude{[*]}{\bd \Field[14][6][1]}{\left( A : \Field \right)}
	\EndProof
}
\Page{
	\Theorem{FrobeneusTheorem}{ \forall A : \TYPE{DivisionAlgebra}(\Reals) \. \forall [0] : \dim A < \infty \.   A = \Reals | A = \Complex | A = \Quat  }
	\Say{D}{\{a \in A : \exists \alpha \in \Reals_{-} \. a^2 = \alpha e \}}{?A}
	\Assume{a}{A}
	\Assume{[1]}{a \neq 0}
	\Say{[2]}{\bd \TYPE{DivisionAlgebra}(A)(A)\THM{MinimalAlgebraicRoots}\THM{AlgebraicInvertability}}{ \NewLine : \Big( m_a(x) : \TYPE{Irreducible}(\Reals)  \Big)}
	\Assume{[3]}{(\deg m_a(x) = 1)}
	\Conclude{[3.*]}{\bd m_a(x) [3]}{ \exists \alpha \in \Reals \. a = \alpha e}
	\Derive{[3]}{I(\Rightarrow)}{\deg m_a(x) = 1 \Rightarrow a \in \Reals }
	\Assume{[4]}{\deg m_a(x) = 2}
	\Say{\Big( \alpha ,\beta ,[5]\Big)}{\THM{RealIrreducibleQuadric}[2][4]}{  :  \sum \alpha,\beta \in \Reals \. m_a(x) = x^2 + \alpha x + \beta  \And \alpha^2 < 4 \beta}
	\Say{[6]}{\bd m_a(x) \THM{BinomialEquation}}{ : 0 = m_a(a) = a^2 + \alpha a + \beta e = \left(a + \frac{\alpha e}{2}\right)^2 + \beta e - \frac{\alpha^2}{4} e }
	\Conclude{[a.*]}{[6][5]}{ \left(a + \frac{\alpha e}{2} \right)^2 \in \Reals_{--}e   } 
	\Derive{[1]}{\bd D E(|)}{ A = D + \Reals e}
	\Assume{u,v}{D}
	\Say{[2]}{ \THM{RealSquaresPositive}(u,v) }{ u,v \not \in \Reals e    }
	\Say{(\lambda,\mu,[3])}{\bd D(u,v)}{\sum \lambda,\mu \in \Reals^++ \. u^2 = -\lambda e \And v^2 = - \mu e }
	\Assume{[5]}{\Big( \{u,b\} : \LI(\Reals,A) \Big)}
	\Say{(\alpha,x,\beta,y,[4])}{[1](u,v)}{ \sum x,y \in D \. \sum \alpha,\beta \in \Reals \. u + v = x + \alpha e \And u - v = y + \beta e }
	\Assume{[6]}{\Big( \{  y,x,e \} : \TYPE{LinearlyDependent}(A)  \Big)}
	\Say{\Big(\alpha,\beta,[7]\Big)}{ [6]  }{\sum \alpha,\beta \in \Reals \.  x = \alpha y + \beta e}
	\Say{[8]}{ \bd \LALGE{\Reals}(A) }{ x^2 = \alpha^2 y^2 + 2\beta \alpha v + \beta^2 e}
	\Conclude{[6.*]}{[2][8] }{ \bot  }
	\Derive{[6]}{E(\bot)}{ \Big( \{ x,y,e\} : \LI(\Reals,a)  \Big) }
	\Say{[7]}{\bd \LALGE{\Reals}[3][4]}{  -2(\lambda + \mu)e = (u+v)^2 + (u-v)^2 =  \NewLine = (x + \alpha e)^2 + (y + \beta  e)^2 = x^2 + y^2 + 2 \alpha  x +  2  \beta y   + (\alpha^2 + \beta^2)e      }
	\Say{[8]}{[7] - \ldots}{  2 \alpha x + 2 \beta y =  x^2 + y^2 + (2\lambda + 2\mu + \alpha^2 + \beta^2  )e       } 
	\Say{[9]}{[8][6]}{\alpha = 0 \And \beta = 0}
	\Conclude{[(u,v).*]}{ [4][9]}{u+v \in D}
	\Derive{[2]}{\bd \VS{\Reals}\bd \TYPE{InnerSum}}{  A = D \oplus \Reals e  }
	\Say{[1.1]}{[2]\dim D = 0}{  \dim D = 0 \Rightarrow A \cong_{\LALGE{\Reals}} \Reals }
	\Say{[2.2]}{[2]\dim D = 1}{  \dim D = 1 \Rightarrow A \cong_{\LALGE{\Reals}} \Complex }
	\Assume{[3]}{\dim D > 1}
	\Say{(\mathrm{i},[4])}{\bd \dim D \bd D \THM{PositiveRealSquareRoot}}{\sum \mathrm{i} \in D \. \mathrm{i}^2 = -e }
	\Say{ p  } {\Lambda u,v \in D \. -uv - vu}{ \L(D,D;D) }
	\Assume{u,v}{D}
	\Say{[5]}{\bd \LALGE{\Reals}(A)\bd D }{ p^2(u,v) (-uv - vu)^2  = u^2v^2  + vu^2v + uv^2u _ v^2u^2 = 4v^2u^2 \in \Reals_+ e  }
	\Conclude{(u,v).*}{ \THM{MinimalAlgebraicRoots}[5] }{ p(u,v) \in \Reals e}
	\Derive{[5]}{\bd^{-1} \TYPE{InnerProduct}(D)}{\Big( p : \TYPE{InnerProduct}(D) \Big)}
}\Page{
	\Say{(S,[6])}{ [0] \THM{OrthogonalDecompositionExists}(\Reals \mathrm{i})}{ \sum S \subvec{\Reals} D \. D = \Reals \mathrm{i} \bot S }
	\Say{(\mathrm{j},[8])}{ [3][6]\bd D}{\sum \mathrm{j} \in S \. \mathrm{j}^2 = -e } 
	\Say{ \mathrm{k}}{ \mathrm{i}\mathrm{j}  }{A}
	\Say{[9]}{\bd \LALGE{\Reals}(A)\bd \mathrm{k}}{ 0 = (\mathrm{k} - \mathrm{k})^2 = (\mathrm{ij} + \mathrm{ji})^2 =  2\mathrm{k}^2 + 2}
	\Say{[10]}{\bd D[9]}{ \mathrm{k} \in D}
	\Say{[11]}{\bd \mathrm{k}\bd \mathrm{j}}{p(\mathrm{i},\mathrm{k}) = \mathrm{i}p(\mathrm{i},\mathrm{j}) = 0}
	\Say{[12]}{\bd \mathrm{k}\bd \mathrm{j}}{p(\mathrm{j},\mathrm{k}) = p(\mathrm{i},\mathrm{j}) \mathrm{j} = 0}
	\Say{(Z,[13])}{\THM{OrthogonalDecompositioExists}}{ \sum Z \subvec{\Reals} S \.  D = \Reals\mathrm{i} \bot \Reals \mathrm{j} \bot \Reals \mathrm{k} \bot Z}
	\Assume{z}{Z}
	\Assume{[14]}{z^2 = - e}
	\Say{[15]}{\bd \TYPE{Orthogonal}\bd p [13]}{ \mathrm{i}z = - z\mathrm{i}}
	\Say{[16]}{\bd \TYPE{Orthogonal}\bd p [13]}{ \mathrm{j}z = - z\mathrm{j}}
	\Say{[17]}{\bd \TYPE{Orthogonal}\bd p [13]}{ \mathrm{k}z = - z\mathrm{k}}
	\Say{[18]}{[17]\bd \mathrm{k}[15][16]}{  \mathrm{ij}z = -z\mathrm{ij} = \mathrm{i}z\mathrm{j} = -\mathrm{ij}z   }  
	\Say{[19]}{[19]\bd \ABEL(A)}{\mathrm{ij}z = 0}
	\Conclude{[z.*]}{\bd \TYPE{DivisionAlgebra}(A)[14][8][4]}{\bot}
	\Derive{[14]}{E(\bot)}{Z = \{0\}}
	\Conclude{[3.*]}{  \bd \Quat[14]}{ A \cong_{\LALGE{\Reals}} \Quat  }
	\Derive{[3.3]}{I(\Rightarrow)}{\dim D > 1 \Rightarrow A \cong_{\LALGE{\Reals}} \Quat}
	\Conclude{[*]}{\bd \Int_+ E(|)[1.1][2.2][3.3]}{A \cong_{\LALGE{\Reals}} \Reals | A \cong_{\LALGE{\Reals}} \Complex | A \cong_{\LALGE{\Reals}} \Quat }
	\EndProof
}
\newpage
\subsection{Widderburn Representation Theorems}
\Page{
	\DeclareType{RepresentationInvariantMaps}
	{
		\prod R \in \ANN \. 
		\prod A,B : \LMOD{R} \.
		\NewLine \. 
		\TYPE{Representayion}
		(\End_{\LMOD{R}}(A),B)
		\to  ?\LMOD{R}(A,B)
	}
	\DefineNamedType{T}{RepresentationInvariantMap}
	{  
		\Lambda \rho :   
		\TYPE{Representation}(\End_{\LMOD{R}}(A),B) \. 
		T \in \L_\rho(A;B) 
	}
	{  
		\NewLine \iff 
		\Lambda \rho :   
		\TYPE{Representation}(\End_{\LMOD{R}}(A),B) \. 
		\forall f \in \End_{\LMOD{R}}(A) \.
		f T = T\rho(f)
	}
	\\
	\DeclareType{RepresentationInvariantOperators}
	{
		\prod R \in \ANN \. 
		\prod A,B : \LMOD{R} \.
		\NewLine \. 
		\TYPE{Representayion}
		(\End_{\LMOD{R}}(A),B)
		\to  ?\End_{\LMOD{R}}(B)
	}
	\DefineNamedType{T}{RepresentationInvariantMap}
	{  
		\Lambda \rho :   
		\TYPE{Representation}(\End_{\LMOD{R}}(A),B) \. 
		T \in \L_\rho(B) 
	}
	{  
		\NewLine \iff 
		\Lambda \rho :   
		\TYPE{Representation}(\End_{\LMOD{R}}(A),B) \. 
		\forall f \in \End_{\LMOD{R}}(A) \.
		\rho(f) T = T\rho(f)
	}
	\\
	\DeclareFunc{tensorEvaluation}
	{
		\prod R \in \ANN \.
		\prod A,B \in \LMOD{R} \.
		\L(A;B) \otimes A \Arrow{\LMOD{R}} B
	}
	\DefineNamedFunc{tensorEvaluation}{T \otimes a}
	{ \mathcal{E}(T \otimes a) }{ T(a)  }
	\\
	\Theorem{InvariantEvaluation}
	{
		\forall R \in \ANN \.
		\forall A,B \in \LMOD{R} \.
		\forall \rho : \TYPE{Representation}(\End_{\LMOD{R}}(A),B) \. 
		\NewLine \.
		\forall f : A \Arrow{\LMOD{R}} B \.
		\Big(\mathcal{E} \rho(f)\Big)_{|\L_\rho(A,B)\otimes A} = 
		\Big((\id \otimes f) \mathcal{E}\Big)_{|\L_\rho(A,B) \otimes A}
	}
	\Assume{T}{\L_\rho(A,B)}
	\Assume{a}{A}
	\Conclude{[T.*]}
	{
		\bd \mathcal{E}
		\bd L_{\rho}(A,B)(T)
		\bd^{-1} \mathcal{E}
		\bd^{-1} (\id \otimes f)
	}
	{
		\NewLine :
		(T \otimes a)
		\Big(\mathcal{E} \rho(f)\Big)  = 
		a \; T \; \rho(f)  =
		a \; f \; T =
		\Big(T \otimes (a \; f) \Big) \mathcal{E} =
		(T \otimes a)\Big((\id \otimes f) \mathcal{E}\Big)
	}
	\DeriveConclude{[*]}
	{
		\bd \FUNC{tensorProduc}
	}
	{
		\LOGIC{This}
	}
	\EndProof
	\\
	\Theorem{WidderburnEvaluationTheorem}
	{
		\forall k : \Field \.
		\forall V,W : \FDVS{k} \.
		\forall \rho : \TYPE{Representation}(\End_{\VS{k}}(V),W) \.
		\NewLine \.
		\mathcal{E} : \L_\rho(V,W) \otimes V \ToIso{\VS{k}} W 
	}
	\Say{(n,e)}{\bd \FDVS{k}}{ \sum n \in \Nat \. \sum e : \Basis(n,V)}
	\Assume{i,j}{n}
	\Conclude{T(e_i \otimes e^*_j)}
	{ \Lambda v \in V \. \alpha_j e^*_j(v)e_i  }
	{ \End_{\VS{k}}(V) }
	\Derive{T}{
		\THM{TensorProductBasis}(V,V^*)\bd \Basis(n,V)(e)
		\bd e^*
	}
	{  V \otimes V^*  \ToIso{\VS{k}} \End_{\VS{k}}(V)   }
	\Say{F}{
		\Lambda i \in n \. 
		\Lambda v \in V \. 
		\Lambda w \in W \.
		\sum^n_{j=1} e^*_j(x)\rho\Big( T(e_j \otimes e^*_i) \Big)(y)
	}
	{
		n \to \L(V,W;W)		
	}
}\Page{
	\Assume{i}{n}
	\Assume{S}{\L_\rho(V;W)}
	\Assume{v,u}{V}
	\Conclude{[u,*]}{
		\ByConstr F_i 
		\bd \L_\rho(V;W)(Sy)
		\ByConstr T 
		\THM{OperatorByBasis}(S)
	}
	{
		\NewLine :
		F_i(u)(Sv) = 
		\sum^n_{j=1} e^*_j(u) \rho\Big( T(e_j \otimes e^*_i)\Big)(Sv) 
		= \NewLine = 
		\sum^n_{j=1} e^*_j(u)  ( v \; T(e_j \otimes e^*_i) \; S ) =
		\sum^n_{j=1} e^*_j(u) v^i Se_j = 
		v^i S(u)
	}
	\DeriveConclude{[i.*]}{ I(=,\to) }{F_i(Sv) = v^i S } 
	\Derive{[1]}{I^3(\forall)}{
		\forall i \in n \. 
		\forall S \in \L_\rho(V;W) \.
		\forall v \in V \.
		F_i(Sv) = v^i S
	}
	\Assume{w}{W}
	\Conclude{[w.*]}
	{
		\ByConstr F 
		\bd e^* 
		\bd \VS{k}\Big( \End_{\VS{k}}(V),\End_{\VS{k}}(W)\Big)(\rho)
		\ByConstr T 
		\bd \LALGE{k}\Big( \End_{\VS{k}}(V),\End_{\VS{k}}(W)\Big)(\rho)	
	}
	{
		\NewLine : 
		\sum^n_{i=1} F_i( e_i, w ) =
		\sum^n_{i=1} \sum^n_j e^*_j(e_i)  
		\rho\Big( T(e_j \otimes e^*_i)\Big)(w) =
		\sum^n_{i=1} \rho\Big( T(e_i \otimes e^*_i)\Big)(w) =
		\NewLine = 
		\rho\left( \sum^n_{i=1} T(e_i \otimes e^*_i )\right)(w) = 
		\rho({\id}_V)(w) = 
		{\id}_W(w) =
		w 
	}
	\Derive{[2]}{I(\forall)}
	{
		\forall w \in W \. \sum^n_{i=1} F_i(e_i,w) = w
	}
	\Assume{w}{W}
	\Assume{i}{n}
	\Assume{S}{\End_{\VS{k}}(V)}
	\Assume{v}{V}
	\Conclude{[v.*]}
	{
		\ByConstr F_i
		\THM{OperatorInCoordinates}(S)
		\bd \VS{k}\Big( \End_{\VS{k}}(V),\End_{\VS{k}}(W)\Big)(\rho)
		\NewLine
		\bd \VS{k} \Big( V \otimes V^*,  \End_{\VS{k}}(W)\Big)(T) 
		\ByConstr T
		\bd \LALGE{k}\Big( \End_{\VS{k}}(V),\End_{\VS{k}}(W)\Big)(\rho)	
		\ByConstr^{-1} F_i
	}
	{
		\NewLine :
		v \; S \; F_i(w) =
		F_i\big( v \; S, w\big) =
		\sum^n_{j=1} e_j^* ( v \; S) \rho( T(e_j \otimes e_i^*) )(w) =
		\sum^n_{j=1} \sum^n_{t=1} v^t S_{j,t} 
		\rho( T(e_j \otimes e_i^*) )(w)= \NewLine = 
		\sum^n_{t=1} v^t \rho 
		\left( \sum^n_{j=1}  T(S_{j,t}e_j \otimes e_i^*) \right)(w) =
		\sum^n_{t=1} v^t \rho( T(e_t \otimes e_i^*) S )(w)   =
		w \left(\sum^n_{t=1} e_t^*(v) \rho\Big( T(e_t \otimes e_i^*)\Big)
		\right) \rho(S) = \NewLine = 
		F_i(v, w) \rho(S) = 
		v\;  F_i(w) \; \rho(S)
	}
	\DeriveConclude{[S.*]}{I(=,\to)}{SF_i(w) = F_i(w)\rho(S)}
	\DeriveConclude{[w.*]}{I(\forall)\bd \L_\rho(V;W)}{F_i(w) \in \L_\rho(V;W)}
	\Derive{[3]}{I^2(\forall)\bd^{-1}\TYPE{Subset}\bd^{-1}\FUNC{image}}
	{F_i(W) \subset \L_\rho(V;W)}
	\Say{\A}{\Lambda w  \in W \.  \sum^n_{i=1} F_i(w) \otimes e_i}
	{W \Arrow{\VS{k}} \L_\rho(V,W) \otimes V}
	\Assume{S}{\L_\rho(V,W}
	\Assume{v}{V}
	\Conclude{[S.*]}{
		\bd \mathcal{E}
		\ByConstr \A [1] 
		\bd \L\Big( \L_\rho(V;W),V,\L_\rho(V;W)\Big)(\otimes)
		\bd \FUNC{coordinates}(e,v)
	}
	{
		\NewLine :
		(S \otimes v)\mathcal{E}\A = 
		v \; S \A  =
		\sum^n_{i=1}  F_i(v \; S)  \otimes e_i =
		\sum^n_{i=1}  v^i S \otimes e_i =
		S \otimes \left( \sum^n_{i=1} v^ie_i \right) =
		S \otimes v
	}
	\Derive{[4]}{\bd \FUNC{tensorProduct}}
	{\mathcal{E}\A = \id}
}\Page{
	\Assume{w}{W}
	\Conclude{[w.*]}
	{  
		\ByConstr \A
		\bd \mathcal{E}
		[2]
	}
	{
		w \A\mathcal{E}
		\Big( \sum^n_{i=1} F_i(w) \otimes e_i  \Big)\mathcal{E} =
		\sum^n_{i=1} F_i(e_i,w) =
		w
	}
	\Derive{[5]}{I(=,\to)}{\A\mathcal{E} = \id}
	\Say{[6]}{\bd^{-1}\TYPE{Inverse}[4][5]}{ \mathcal{E}^{-1} = \A}
	\Conclude{[*]}{\bd^{-1}\TYPE{Iso}}{ \LOGIC{This} }
	\EndProof
	\\
	\Theorem{RepresentationInvariantDimension}
	{
		\forall k : \Field \.
		\forall V,W \in \FDVS{k} \. \NewLine \. 
		\forall \rho : 
		\End_{\VS{k}}(V) \Arrow{\LALGE{k}} \End_{\VS{k}}(W) \.
		\dim V  \dim \L_\rho(V; W) = \dim W 
	}
	\NoProof
	\\
	\DeclareFunc{tensorComposition}
	{
		\prod R \in \ANN \.
		\prod A,B \in \LMOD{R} \. \NewLine \. 
		\LMOD{R}(A,B) \otimes \End_{\LMOD{R}(B)} 
		\Arrow{\LMOD{R}} 
		\LMOD{R}(A,B)
	}
	\DefineNamedFunc{tensorComposition}{T \otimes S}
	{\C(T \otimes S)}{TS}
	\\
	\Theorem{WidderburnCompositionTheorem}
	{
		\forall k : \Field \.
		\forall V,W : \FDVS{k} \.
		\forall \rho : \TYPE{Representation}(\End_{\VS{k}}(V),W) \.
		\NewLine \.
		\C :  L_\rho(W) \ToIso{\LALGE{k}} 
		\End_{\VS{k}}\Big(\L_\rho(V,W)\Big)
	}
	\Say{\B}{
		\Lambda \Omega \in \End_{\VS{k}}\Big( \L_\rho(V,W) \Big) \.
		\mathcal{E}^{-1}( \Omega \otimes \id)\mathcal{E}
	}
	{
		\End_{\VS{k}}\Big( \L_\rho(V,W)  \Big) 
		\Arrow{\VS{k}}
		\End_{\VS{k}}(W)
	}
	\Assume{\Omega}{\End_{\VS{k}}\Big(\L_\rho(V,W)\Big)}
	\Assume{S}{\End_{\VS{k}}(V)}
	\Assume{w}{W}
	\Conclude{[w.*]}{
		\ByConstr \B
		\ByConstr \mathcal{E}^{-1}
		\bd \FUNC{tensorMap}
		\bd \mathcal{E}
		\ByConstr F_i
		\bd \LALGE{k}\Big(\End_{\VS{k}}(V),\End_{\VS{k}}(W)\Big)(\rho)			     \ByConstr T
		\NewLine
		\bd^{-1} \FUNC{operatorMatrix}(S,e)
		\bd \VS{k}\Big( \End_{\VS{k}}(V),\End_{\VS{k}}(V)  \Big) \rho\Omega
		\bd \FUNC{operatorMatrix}(S,e)
		\bd^{-1} F
		\NewLine
		\bd \L_\rho(V,W) (w F_t \Omega)
		\bd^{-1} \B
	}
	{
		\NewLine :
		w \rho(S) \B(\Omega) =
		w \rho(S) \mathcal{E}^{-1} (\Omega \otimes \id) \mathcal{E} =
		\left(\sum^n_{i=1} F_i\Big( w \rho(S)\Big) \otimes e_i\right)
		(\Omega \otimes \id) \mathcal{E} =  
		\sum^n_{i=1} (w \; \rho(S) \; F_i \Omega)(e_i)
		= \NewLine =  
		\sum^n_{i,j=1}  \Omega\Big( e_j^* 
		 \rho\big( T(e_j \otimes e_i^*) \big) \rho(S)(w) \Big)(e_i) 
		= 
		\sum^n_{i,j=1}  \Omega\Big( e_j^* 
		 \rho\big( ST(e_j \otimes e_i^*) \big)(w) \Big)(e_i) 
		=
		\NewLine
		=
		\sum^n_{i,j=1}  
		\Omega\left( e_j^* 
		 \rho\left( \sum^n_{t=1} S_{i,t}T(e_j \otimes e_t^*) 
		 \right)(w) \right)(e_i) =
		\sum^n_{i,j,t=1}
		\Omega\left( e_j^* 
		 \rho\left( \sum^n_{t=1} T(e_j \otimes e_t^*) 
		 \right)(w) \right)(S_{i,t}e_i) =
		\NewLine = 
		\sum^n_{j,t=1}
		\Omega\left( e_j^* 
		 \rho\left( \sum^n_{t=1} T(e_j \otimes e_t^*) 
		 \right)(w) \right)(Se_t) =
		 \sum^n_{t=1} ( w  F_t \Omega)(Se_t) =
		 \left(\sum^n_{t=1} (w F_t \Omega)(e_t)\right) \rho(S) =
		 w \rho(S) \B(\Omega)
	}
	\DeriveConclude{[S.*]}{I(=,\to)}
	{ \rho(S)\B(\Omega) = \B(\Omega)\rho(S)  }
	\DeriveConclude{[\Omega.*]}{\bd \L_\rho(W)}{\B(\Omega) \in \L_\rho(W)}
	\Derive{[7]}{\bd^{-1}\TYPE{Subset}\bd^{-1}\FUNC{image}}
	{\im \B \subset \L_R(W)}
}\Page{
	\Assume{\Omega}{\End_{\VS{k}}\Big(L_\rho(V,W)\Big)}
	\Assume{S}{\L_\rho(V,W)}
	\Assume{v}{V}
	\Conclude{[v.*]}{   
		\ByConstr \B 
		\bd \C
		\ByConstr \mathcal{E}^{-1}
		![1]
		\bd \L\Big( \L_\rho(V,W), V ; \L_\rho(V,W)\otimes V\Big)(\otimes)
		\bd \FUNC{coordinates}(v)
		\bd \FUNC{tensorMap}(\Omega,\id)
		\bd \mathcal{E}
	}
	{
		\NewLine :
		v \Big( S (\Omega \B \C) \Big) = 
		v \Big( S \big(\mathcal{E}^{-1} (\Omega \otimes \id) \mathcal{E} \C
		\big)\Big) =
		(v \; S) \Big( \mathcal{E}^{-1} (\Omega \otimes \id) 
		\mathcal{E}\Big) =
		\left(\sum^n_{i=1} F_i\Big( v S\Big) \otimes e_i\right)
		(\Omega \otimes \id) \mathcal{E} = \NewLine =  
		\left(\sum^n_{i=1}  v^i S \otimes e_i\right)
		(\Omega \otimes \id) \mathcal{E} =  
		(S \otimes v)(\Omega \otimes \id) \mathcal{E} =
		v \; \Omega(S)
	}
	\DeriveConclude{[S.*]}{  I(=,\to) }{  S (\Omega \B \C) = S \; \Omega }
	\DeriveConclude{[\Omega.*]}{I(=,\to)}{ \Omega \; \B\C = \Omega}
	\Derive{[8]}{I(=,\to)}{\B\C = \id}
	\Assume{S}{\L_\rho(W)}
	\Assume{w}{W}
	\Conclude{[w.*]}
	{
		\ByConstr \B
		\ByConstr \mathcal{E}^{-1}
		\bd \FUNC{tensorMap}
		\bd \C
		\ByConstr F_i
		\bd \mathcal{E}
		![2]
	}
	{
		\NewLine : 
		w ( S\C\B) = 
		w \Big( \mathcal{E}^{-1}(S\C \otimes \id)\mathcal{E} \Big) =
		\sum^n_{i=1} (F_i(w) \otimes e_i)(S\C \otimes \id)\mathcal{E} =
		\sum^n_{i=1} (F_i(w)S \otimes e_i)\mathcal{E} =
		\sum^n_{i=1} F_i(e_i,w) S =
		w \; S
	}
	\DeriveConclude{[S.*]}{I(=,\to)}{ S\; \C\B = S}
	\Derive{[9]}{I(=,\to)}{\C\B = \id}
	\Say{[10]}{\bd^{-1} \FUNC{Inverse}[8][9]}{\C\B = \id}
	\Conclude{[*]}{\bd \FUNC{Iso}[10]}{\LOGIC{This}}
	\EndProof
	\\
	\DeclareType{EquevalentAlgebraRepresentation}
	{
		\prod R \in \ANN \.
		\prod A \in \LALGE{R} \.
		\prod X,Y \in \LMOD{R} \. \NewLine \.
		?\Big(  \TYPE{Representation}(A,X) \times 
		\TYPE{Representation}(A,Y)\Big)
	}
	\DefineNamedType{(\rho,\rho')}{EquivalentAlgebraRepresentation}
	{\rho \cong \rho'}{
		\exists \varphi : 
		X \Arrow{\LMOD{R}} Y :
		\forall a \in A \. 
		\rho(a)\varphi = \rho'(a)\varphi
	}
	\\
	\Theorem{TensorRepresentationEquivalence}
	{
		\forall k : \Field \.
		\forall V,W \in \FDVS{k} \.
		\forall A \in \LALGE{k} \. 
		\NewLine 
		\forall R : A \otimes \End_{\VS{k}}(V) 
		\Arrow{\LALGE{k}} \End_{\VS{k}}(W) \.
		\exists  \rho' : A \Arrow{\LALGE{k}} 
		\End_{\VS{R}}\Big(\L_\rho(V,W) \Big) :
		R \cong \rho' \otimes \id 
		\NewLine \; \where \;
		\rho = \Lambda T \in \End_{\VS{kG}}(V) \. R(e_A \otimes T)
	}
	\Say{p}{\Lambda a \in A \. R(a \otimes \id)}
	{
		A \Arrow{\LALGE{k}} \End_{\VS{k}}(W)
	}
	\Assume{a}{A}
	\Assume{S}{\End_{\VS{k}}(V)}
	\Conclude{[S.*]}
	{
		\ByConstr \rho \ByConstr p 
		\bd^2 \LALGE{k}\Big( A \otimes \End_{\VS{k}}(V),\End_{\VS{k}}(W)\Big)
		R
		\ByConstr^{-1} \rho \ByConstr^{-1} p 	
	}
	{ 
		\NewLine
		\rho(S) p(a) = 
		R(e \otimes S) R(a \otimes \id) =
		R(a \otimes S) = 
		R(a \otimes \id)R(e \otimes S) =
		p(a)\rho(S)
	}
	\DeriveConclude{[a.*]}{\bd \L_\rho(W)}{ p(a) \in \L_\rho(W)  }
	\Derive{[11]}{\bd^{-1} \TYPE{Subset}\bd^{-1}\TYPE{Image}}
	{
		\im p \subset \L_\rho(W)
	}
	\Say{\rho'}{ p\C}{ 
		A \Arrow{\LALGE{k}} \End_{\VS{k}}\Big(\L_\rho(V,W)\Big)}
}\Page{
	\Assume{a}{A}
	\Assume{f}{\End_{\VS{k}}(V)}
	\Assume{S}{\L_\rho(V,W)}
	\Assume{v}{W}
	\Conclude{[S.*]}{
		\bd \FUNC{tensorMap}
		\ByConstr \rho' 
		\bd \mathcal{E}
		\ByConstr p
		\bd \C
		\bd \L_\rho(V,W)(S)
		\ByConstr \rho 
		\bd \LALGE{k}\Big( A \otimes \End_{\VS{k}}(V),\End_{\VS{k}}(W)\Big)
		R
		\bd^{-1} \mathcal{E}
	}
	{
		\NewLine :
		(S \otimes v)(\rho'(a) \otimes f)\mathcal{E} = 
		(S \; \rho'(a)) \otimes (v \; f) \mathcal{E} =
		(v \; f) \Big(S \; (a p\C)\Big) =
		(v \; f) \Big( S  \;  R(a \otimes \id) \C\Big) =
		\NewLine = 
		v \; f \; S \;  R(a \otimes \id)  =
		v \; S \; \rho(f) \; R(a \otimes \id) =
		v \; S \; R(e \otimes f) R(a \otimes \id) =
		v \; S \; R(a \otimes f) =
		(S \otimes v)\mathcal{E}R(a \otimes f)  
	}
	\DeriveConclude{[a.*]}{I(=,\to)}
	{ 
		(\rho'(a) \otimes f)\mathcal{E} = \mathcal{E} R(a \otimes f)  
	}
	\Conclude{[*]}{
		\bd \FUNC{tensorProduct} 
		\bd^{-1} \TYPE{EquivalentAlgebraRepresentation}
	}
	{
		\LOGIC{This}
	}
	\EndProof
}
\newpage
\section{Coalgebras and Comodules}
\subsection{Coalgebras}
\Page{
	\DeclareType{Coalgebra}{ \prod R \in \ANN \. \prod A :\LMOD{R} \. \left(A \otimes A \Arrow{\LMOD{R}} A\right) \times A \LMOD{R} \Arrow{\LMOD{R}} R }
	\DefineType{(A,\Delta,\eta)}{A}{  \Delta(\id \otimes \Delta) = \Delta (\Delta \otimes \id) 
		\And \Delta (\id \otimes \eta) =  \id \otimes 1 \And   \Delta (\eta \otimes \id) = 1 \otimes \id  }
	\\
	\DeclareFunc{comultiplication}{\prod A : \TYPE{Coalgebra} \. A \Arrow{\LMOD{R}} A \otimes A }
	\DefineNamedFunc{comultiplication}{(A,\Delta,\eta,[1],[2],[3]) }{\Delta_A}{ \Delta  }  
	\\
	\DeclareFunc{comultiplication}{\prod A : \TYPE{Coalgebra} \. A \Arrow{\LMOD{R}} A \otimes A }
	\DefineNamedFunc{comultiplication}{(A,\Delta,\eta,[1],[2],[3]) }{\Delta_A}{ \Delta  }  
	\\
	\DeclareFunc{counit}{\prod A : \TYPE{Coalgebra} \. A \Arrow{\LMOD{R}} R }
	\DefineNamedFunc{counit}{(A,\Delta,\eta,[1],[2],[3]) }{\eta_A}{ \eta  }  
	\\
	\DeclareFunc{comultiplicationProperty}{\prod A : \TYPE{Coalgebra} \. \Type }
	\DefineFunc{comultiplicationProperty }{A,\Delta,\eta,[1],[2],[3] }{ [1]  }  
	\\
	\DeclareFunc{rightCounitProperty}{\prod A : \TYPE{Coalgebra} \. \Type }
	\DefineFunc{rightCounitProperty }{A,\Delta,\eta,[1],[2],[3] }{ [2]  }  
	\\
	\DeclareFunc{leftCounitProperty}{\prod A : \TYPE{Coalgebra} \. \Type }
	\DefineFunc{leftCounitProperty }{A,\Delta,\eta,[1],[2],[3] }{ [3]  } 
	\\
	\DeclareType{Cocomutative}{?\TYPE{Coalgebra}(A)}
	\DefineType{A}{Cocomutative}{\FUNC{swap}\;\Delta_A = \Delta_A}
	\\
	\DeclareFunc{SweedlerSum}{\prod A : \TYPE{Coalgebra}(R) \. A \Arrow{\LMOD{R}} A \otimes A }
	\DefineNamedFunc{SweedlerSum}{a}{\sum_{(a)} a_{(1)} \otimes a_{(2)} }{\Delta_A(a)}
	\\
	\DeclareFunc{trivialCoalgebra}{\prod R \in \ANN \. \TYPE{Coalgebra}(R)}
	\DefineFunc{trivialCoalgebra}{}{  \Big(R, \Lambda r \in R \. r ( e \otimes e ), \id\Big)  }
	\\
	\DeclareFunc{dividedPowerCoalgebra}
	{
		\prod R \in \ANN \. \TYPE{Coalebra(R)} 
	}
	\DefineFunc{dividedPowerCoalgebra}{}{ \Big( R[x], \bd \TYPE{Free}(R[x])\Lambda n \in \Int_+ \. \sum^n_{i=0} C^i_n x^{i} \otimes x^{n-i} , \bd \TYPE{Free}(R[x])\Lambda n \in \Int_+ \. \delta^n_0  \Big)  }
}
\Page{
	\DeclareType{Coideal}{ \prod A : \TYPE{Coalgebra}(R) \. \TYPE{Submodule}(A) }
	\DefineType{I}{Coideal}{ \forall a \in I \. \Delta(a) \subset I \otimes A +  A \otimes I \And \eta(a) = 0 }
	\\
	\Theorem{CoidealQuotient}{\forall A : \TYPE{Coalgebra}(R) \. \forall I : \TYPE{Coideal}(A) \.  \left( \frac{A}{I},  \Delta\Big([\cdot]_I\otimes [\cdot]_I\Big),\eta \right) : \TYPE{Coalgebra}(I)}
	\Assume{h}{I}
	\Say{[1]}{\bd \TYPE{Coideal}(I)(h)}{\Delta(h) \subset A \otimes I  + A \otimes I}
	\Conclude{[h.*]}{ \bd \FUNC{SweedlerSum}\bd \TYPE{quotient}[1]}{\sum_{(h)} [h_1] \otimes [h_2] = 0  }
	\Derive{[1]}{\bd \TYPE{Subset}}{I \subset \ker \Delta [\cdot]_I \otimes [\cdot]_I}
	\Say{\Big(\phi,[2]\Big)}{\THM{QuotientMapTHM}[1]}{\sum \phi : \frac{A}{I} \Arrow{\LMOD{R}} \frac{A}{I} \otimes \frac{A}{I} \. \pi_I \phi = \Delta (\pi_I \otimes \pi_I)  }
	\Say{\Big(\eta',[3]\Big)}{\THM{QuotientMapTHM}\bd\TYPE{Coideal}(A)(I)}{\sum \eta' : \frac{A}{I} \Arrow{\LMOD{R}} R \. \pi_I \eta' = \eta  }
	\Conclude{[*]}{\bd^{-1}\TYPE{Coalgebra}(R)[2][3]}{\LOGIC{This}}
	\EndProof
	\\
	\DeclareFunc{quotientCoalgebra}{\prod A : \TYPE{Coalgebra}(R) \. \TYPE{Coideal}(I) \to \TYPE{Coalgebra}(A)}
	\DefineNamedFunc{quotientCoalgebra}{I}{\frac{A}{I}}{\THM{CoidealQuotient}}
	\\
	\DeclareType{Grouplike}{ \prod A : \TYPE{Coalgebra}(R) \. ?A}
	\DefineType{g}{Grouplike}{\Delta(g) = g \otimes g \And g \neq 0}
	\\
	\Theorem{GrouplikeCounit}{\forall A : \TYPE{Coalgebra} \And \TYPE{TorsionFree}(R) \. \forall g : \TYPE{Grouplike}(A) \. \eta(g) = e}
	\NoProof
	\\
	\Theorem{GrouplikeOfDividedPower}{ \forall R \in \ID \. \TYPE{Grouplike}\;\FUNC{dividedPowerCoalgebra}(R) = \{1\}}
	\Assume{a_i x^i}{\TYPE{Grouplike}\;\FUNC{dividedPowerCoalgebra}}
	\Say{n}{\deg a_i x^i}{\Int_+}
	\Say{[1]}{\bd \TYPE{Groupllike}(a_i x^i)\bd \FUNC{dividedPowerCoalgebra}}
	{
		a_ix^i \otimes a_ix^i = \Delta(a_ix^i)  = a_i C^j_i x^j \otimes x^{i-j}            
	}
	\Assume{[0]}{n > 0}
	\Say{[2]}{\THM{TensorProductBasis}[1]}{a_n^2 = 0}
	\Conclude{[0.*]}{\bd \deg [2]}{ \bot  }
	\Derive{[2]}{E(\bot)\bd \deg}{ a_i x^i = a_0}
	\Conclude{[\ldots*]}{\THM{GrouplikeCounit}\bd \FUNC{dividedPowerCoalgebra}[2]}{ a_i x^i = 1  } 
	\DeriveConclude{[*]}{\bd^{-1} \TYPE{Singleton}}{\LOGIC{This}}
	\EndProof
}\Page{
	\Theorem{GrouplikeLinearlyIndependent}{\forall R : \ID \. \NewLine \. \forall A : \TYPE{Coalgebra}(R) \And \TYPE{TorsionFree}(R) \. \TYPE{Grouplike}(A) : \LIS(A)}
	\Say{G}{\TYPE{Grouplike}(A)}{?A}
	\Assume{a,b}{G}
	\Assume{\alpha}{R}
	\Assume{[1]}{\alpha a = b}
	\Say{[2]}{\bd \TYPE{Grouplike}(a,b)}{  \alpha a \otimes a = \Delta(\alpha a) = \Delta(b) = b \otimes b =  \alpha^2 a \otimes a   }
	\Conclude{[(a,b)*]}{ \bd \TYPE{TorsionFree}(R)(A)\bd \ID(R)  }{ \alpha = 1  }
	\Derive{[0]}{I(\forall)}{\forall a,b \in G \. \forall \alpha \in R \. \alpha a = b \Rightarrow a = b }
	\Assume{\alpha}{R^{\oplus G}} 
	\Assume{[1]}{\alpha_g g = 0}
	\Assume{[2]}{\alpha \neq 0}
	\Say{(g,[3])}{ E(\#,\to)[2]}{ \sum g \in G \. \alpha_g \neq 0  }
	\Say{k}{\mathrm{Frak}(R)}{\Field}
	\Say{V}{A \otimes_R k}{\VS{k}} 
	\Say{[4]}{[3][1]}{g =_V \frac{\alpha_h}{\alpha_g} h}
	\Say{I}{\Big\{ h \in G \setminus \{g\} : \alpha_h \neq 0  \Big\}}{\TYPE{Finite}(G)}
	\Assume{[5]}{(I : \LIS(V))}
	\Say{[6]}{ \bd \TYPE{Grouplike}(g)[4] }{  \frac{\alpha_h \alpha_f}{\alpha_f}  h \otimes f  = g \otimes g  = \Delta(g) = \frac{\alpha_h}{\alpha_g} h \otimes h  }  
	\Say{[7]}{\THM{TensorProductBasis} \bd \LIS(V)(I)}{ \forall h,f \in I \. \alpha_h \alpha_f = 0  }
	\Say{[8]}{\bd \ID(R) [7]}{\alpha_I = 0}
	\Conclude{[\alpha.*]}{\ByConstr I [8]}{I = \emptyset}
	\Derive{[1]}{I(\Rightarrow)}{ G : \TYPE{LineatlyDependent}(A) \Rightarrow \forall I \subset G \. |I| > 1  \Rightarrow I : \TYPE{LinearlyDependent}(A) }
	\Conclude{[*]}{[1][0]}{(G : \LIS(A))}
	\EndProof
	\\
	\DeclareType{CoalgebraMorphism}{\prod A ,B : \TYPE{Coalgebra}(R) \. ? (A \Arrow{\LMOD{R}} B)}
	\DefineType{f}{CoalgebraMorphism}{\forall x \in A \. \Delta_A(f \otimes f) = f \Delta_B \And \eta_A = f\eta_B}
	\\
	\DeclareFunc{coalgebraCategory}{\RING \to \CAT}
	\DefineNamedFunc{coalgebraCategory}{R}{ \COALG{R} }{ \Big( \TYPE{Coalgebra}(R), \TYPE{CoalgebraMorphism}(R),\circ,\id  \Big)  }
	\\
	\Theorem{CounitMorphism}{\forall A \in \TYPE{Coalgebra}(R) \.  \eta_A : A \Arrow{\COALG{R}} R }
	\NoProof
	\\
	\Theorem{HomoPreservesGrouplike}{\forall A,B \in \TYPE{Coalgrbra}(R) \. \forall f : A \Arrow{\COALG{R}} B \.  \NewLine \. \forall g : \TYPE{Grouplike}(A) \. f(g) : \TYPE{Grouplike}(B)}
	\NoProof
}
\Page{
	\DeclareFunc{tensorProductOfCoalgebras}{\COALG{R} \times \COALG{R} \to \COALG{R}}
	\DefineNamedFunc{tensorProductOfCoalgebras}{A,B}{A \otimes B}{  \bigg( A \otimes B, \NewLine, 
		\bd \TYPE{TensorProduct} \Lambda a \in A \. \Lambda b \in B \. \sum_{(a),(b)} (a_1 \otimes b_1) \otimes (a_2 \otimes b_2), \NewLine ,
		\bd \TYPE{TensorProduct} \Lambda a \in A \. \Lambda b \in B \.  \eta_A(a)\eta_B(b)                    
	\bigg)}
	\\
	\DeclareFunc{freeCoalgebra}{ \Big[\VS{k}\Big]_e \Arrow{\CAT} \COALG{k}}
	\DefineNamedFunc{freeCoalgebra}{V,E}{F_{\COALG{R}}(V,E)}{(M,\bd \Basis(V,E)\Lambda e \in E \. e \otimes e, \bd \Basis(V,E)\Lambda e \in E \. 1 ) } 
	\\
	\DeclareType{LeftCoideal}{ \prod A : \TYPE{Coideal}(R) \. ?\TYPE{Submodule}(A)    }
	\DefineType{I}{LeftCoideal}{ \eta(A) = 0 \And  \Delta(I) \subset I \otimes A }
	\\
	\DeclareType{RightCoideal}{\prod A : \TYPE{Coideal}(R) \. ?\TYPE{Submodule}(A)  }
	\DefineType{I}{RightCoideal}{\eta(A) = 0 \And \Delta(I) \subset I \otimes A }
	\\
	\Theorem{RightCoidealIsCoideal}{\forall I : \TYPE{RightCoideal}(A) \.  I : \TYPE{Coideal}(A)}
	\NoProof
	\\
	\Theorem{LeftCoidealIsCoideal}{\forall I : \TYPE{RightCoideal}(A) \. I : \TYPE{Coideal}(A)}
	\NoProof
	\\
	\Theorem{SumOfCoideals}{\forall I,J : \TYPE{Coideal}(A) \. I + J : \TYPE{Coideal}(A)}
	\NoProof
}
\newpage
\subsection{Algebra-Coalgebra Duality}
\Page{
	\DeclareFunc{dualAlgebra}{ \prod R \in \ANN \. \COALG{R}^{\mathrm{op}} \Arrow{\CAT} \LALGE{R}}
	\DefineNamedFunc{dualAlgebra}{A}{A^*}{\left( A^*, \Lambda f,g \in A^* \. \Lambda a \in A \. \sum_{(a)} f(a_1)g(a_2), \eta   \right) } 
	\\
	\DeclareType{Cofinite}{\prod R \in \ANN \. \prod A : \LALGE{R} \. ? \TYPE{Ideal}(A) }
	\DefineType{I}{Cofinite}{\exists F : \TYPE{Finite}\left(\frac{A}{I}\right) \. \frac{A}{I} = \langle F \rangle_{\LMOD{E}}}
	\\
	\DeclareFunc{finiteDual}{ \prod R \in \ANN \. \LALGE{R} \to \LALGE{R}}
	\DefineNamedFunc{finiteDual}{A}{A^\circ}{\Big\{ f \in M^* : \exists I : \TYPE{Cofinite}(A)  : f(I) = \{0\} \Big\}}
	\\
	\Theorem{FiniteDualWhitness}{
		\forall R \in \ANN \. 
		\forall A \in \LALGE{R} \. 
		\forall f \in A^\circ \. 
		\forall I : \TYPE{Ideal}(A ) \.
		\forall [0] : f(I) = \{0\} \. \NewLine
		\exists \overline{f} \in \left( \frac{A}{I} \right)^* \.
		\pi^*_I \overline{f} = f
	}
	\NoProof
	\\
	\Theorem{FiniteDualTensorProduct}{ 
		\prod R  : \TYPE{Field} \. 
		\prod A \in \LALGE{R} \.
		\mu_A^*(A^\circ) \subset A^\circ \otimes A^\circ
	}
	\Assume{f}{A^\circ}
	\Say{\Big(I,[1]\Big)}{\bd A^\circ(f)}{\sum I : \TYPE{cofinite}(A) \. f(I) = \{0\}  }
	\Say{\Big(F,[2]\Big)}{\bd \TYPE{Cofinite}(A)(I)}{\sum F : \TYPE{Finite}\left(\frac{A}{I}\right) \. \frac{A}{I} = \langle F \rangle_{\LMOD{R}}  }
	\Say{ \Big( \overline{f},[3] \Big) }{ \THM{FiniteDualWhitness}(f,I) }{ \sum \overline{f} \in \left( \frac{A}{I} \right)^* \. \pi^*_I \overline{f} = f }
	\Say{\Big(\phi,[4]\Big)}{\THM{FGDualTensorBasis}(\Big(\mu_{\frac{A}{I}}^* \Big) \overline{f})}
	{ \sum \phi : F^2 \to R \. \Big(\mu_{\frac{A}{I}}^* \Big) \overline{f} = \sum_{a,b \in F} \phi_{a,b} a^* \otimes b^*}
	\Assume{\sum^n_{i=1} x_i \otimes y_i}{A \otimes A}
	\Conclude{\ldots*}{ \bd \mu_A^* [3] \bd^{-1} \mu_{\frac{A}{I}}^*\bd \Ideal(A)(I) [4] \bd^{-1} \pi^*_I    }
	{ 
	    (\mu_A^* f) \sum^n_{i=1} x_i \otimes y_i = 
	    \sum^n_{i=1} f( x_i y_i ) = 
	    \sum^n_{i=1} \pi^*_I \overline{f}(x_i y_i) = \NewLine =  
	    \Big(\mu_{\frac{A}{I}}^* \Big) \overline{f} \sum^n_{i=1} [x_i] \otimes [y_i] =
	    \sum_{a,b \in F} \phi_{a,b}(a^* \otimes b^*)\sum^n_{i=1}   [x_i] \otimes [y_i] =
	    \sum_{a,b \in F} \phi_{a,b}(\pi^*_I a^* \otimes \pi^*_I b^*) \sum^n_{i=1}  x_i \otimes y_i 
	}
	\Derive{[5]}{I(=,\to)}{ \mu_A^* f = \sum_{a,b \in F} \phi_{a,b} (\pi^*_I a^* \otimes \pi^*_I b^*) }
	\Say{[6]}{\bd \pi^*_I \bd \pi_I}{\forall a \in F \. I \subset \ker \pi^*_I a }
	\Say{[7]}{\bd \TYPE{Cofinite}(I)}{ \forall a \in F \. \pi^*_I a \in A^\circ}
	\Conclude{[*]}{[5][7]}{\mu_A*(A^\circ) \subset A^\circ \otimes A^\circ}
	\EndProof
}
\Page{
	\Theorem{FiniteDualIsCoalgebra}
	{
		\forall k : \Field \.
		\forall A \in \LALGE{k} \.
		(A^\circ, \mu_A^*, e_A^* ) \in \COALG{k}                                                                          
	}
	\Assume{f}{A^*}
	\Assume{a,b,c}{A}
	\Conclude{[(a,b,c).\ldots]}{\bd \mu^*_A}{  \mu_A^* (\mu_A^* \otimes \id)(f)(a \otimes b \otimes c) = f(abc)  = \mu_A^* (\id \otimes \mu_A^*)(f)(a \otimes b \otimes c) }
	\DeriveConclude{f.(*)}{\bd \TYPE{TensorProduct}I(=,\to)}{\mu_A^* (\mu_A^* \otimes \id)(f)  = (\mu_A^* ( \id \otimes \mu_A^*(f) }
	\Derive{[1]}{I(=,\to)}{ \mu_A^* (\mu_A^* \otimes \id) = \mu_A^*(\mu_A^* \otimes \mu_A^*)}
	\Assume{f}{A^*}
	\Assume{a}{A}
	\Conclude{[f.*]}{\THM{TrivialTensorProduct} \bd e_A^* \bd e}{  
		\mu^*_A(\id \otimes e_A^*) (f)(a) = 
		(\id \otimes e_A^*) (f \circ \mu_A^*) (a \otimes 1)  = \NewLine =  
		f( a e ) = f(a) = f(ea) = 
		\mu^*_A(e_A^* \otimes \id)
	}
	\Derive{[2]}{ I(=,\to) }{ \mu^*_A(\id \otimes e_A^*) = \id = \mu^*_A(e_A^* \otimes \id)  }
	\Conclude{[*]}{\bd \COALG{k}}{ A^\circ \in \COALG{k}} 
	\EndProof
	\\
	\DeclareFunc{finiteDualCoalgebra}
	{
		\prod k : \Field \.
		\LALGE{k}^{\mathrm{op}} \Arrow{\CAT} \COALG{k}
	}
	\DefineNamedFunc{finiteDualCoalgebra}{A}{A^\circ}{\THM{FiniteDualIsCoalgebra}}
	\DefineNamedFunc{finiteDualCoalgebra}{A,B,\varphi}{\varphi^\circ}{f^*_{|B^\circ}}
	\Assume{f}{B^\circ}
	\Assume{a \otimes a' }{A \otimes A}
	\Conclude{\Big( a \otimes a'.*\Big)}{\bd \mu^*B \bd \varphi^\circ \bd \LALGE{k}(A,B)(\varphi)\bd^{-1}\varphi^\circ }{ 
			\mu^*_B(\varphi^\circ \otimes \varphi^\circ) (f) (a \otimes a') = 
			f\Big( \varphi(a)\varphi(a') \Big) =
			f\Big( \varphi(aa')  \Big) = \NewLine = 
			\varphi^\circ f (aa') = 
			\varphi^\circ \mu^*_A (f) (a \otimes a') } 
	\DeriveConclude{f.*}{\bd \TYPE{TensorProduct}I(=,\to)}{  \mu^*_B (\varphi^\circ \otimes \varphi^\circ)(f) = \varphi^\circ \mu^*_A(f)  }  
	\Derive{[1]}{  I(=,\to)  }{ \mu^*_B (\varphi^\circ \otimes \varphi^\circ) = \varphi^\circ \mu^*_A  }
	\Assume{f}{B^\circ}
	\Conclude{[f.*]}{\bd \varphi^\circ \bd e^*_A \bd \LALGE{k}(A,B)(\varphi)}{  \varphi^\circ e^*_A (f) = f\Big(\varphi(e_A)\Big) = f(e_B) = e^*_B f  }
	\Derive{[2]}{I(=,\to)}{ \varphi^\circ e^*_A = e^*_B  }   
	\Conclude{[*]}{\bd \COALG{k}(B,A)}{\varphi^\circ \in \COALG{k}(B,A)} 
	\EndProof
	\\
	\Theorem{FiniteMonoidAlgebraDual}{
		\forall M : \TYPE{FiniteMonoid} \. 
		\forall k : \TYPE{Field} \.
		\forall m \in M \. \NewLine \. 
		\Delta_{k^\circ[M]}(\mathrm{d}x_m) = \sum_{a,b \in M : ab = m}  \mathrm{d}x^a \otimes \mathrm{d}x^b
	}
	\Assume{P,Q}{k[M]}
	\Conclude{\Big[(P,Q).*\Big]}{\bd \FUNC{finiteDualCoalgebra}\big(k[M]\big) \bd \mu^*_{k[M]}\bd k[M] \bd \mathrm{d}x^m \bd^{-1} \mathrm{d}x \otimes \mathrm{d}x \bd^{-1} P(x) \otimes Q(x)  }
	{
		\NewLine = 
		\Delta_{k^\circ[M]}(\mathrm{d}x^m)\Big(P(x) \otimes Q(x)\Big) =
		\mu^*_{k[M]}(\mathrm{d}x^m)\Big(P(x) \otimes Q(x)\Big) =
		\mathrm{d}x^m \Big( P(x)Q(x) \Big) = 
		\mathrm{d}x^m  \sum_{a,b \in M} P_aQ_b x^{ab} = \NewLine = 
		\sum_{a,b \in M : ab = m} P_aQ_b =  
		\sum_{a,b \in M : ab = m} \mathrm{d}x^a \otimes \mathrm{d}x^b \sum_{a,b \in M} PaQ_b (x^a \otimes x^b  ) =
		\sum_{a,b \in M : ab = m} \mathrm{d}x^a \otimes \mathrm{d}x^b \Big(P(x) \otimes Q(x) \Big)   
	}
	\DeriveConclude{[*]}{\bd \TYPE{TensorProduct}I(=,\to)}{\LOGIC{This}}
	\EndProof
}\Page{
	\Theorem{CommutativeDualCoalg}{\forall A \in \LCALGE{k} \. A^\circ : \TYPE{Cocommutative}(k)}
	\Assume{f}{A^\circ}
	\Assume{a \otimes a'}{A \otimes A}
	\Conclude{[f.*]}{ \THM{FunctionalSwap} \bd \mu^*_A \bd \LCALGE{k} \bd^{-1} \mu^*_A  }
	{
		\Big(\mu^*_A \FUNC{swap} (f)\Big) (a \otimes b) =
		\mu^*_A(f) ( b \otimes a ) = \NewLine = 
		f(ba) = 
		f(ab) =
		\mu^*_A (f)(a \otimes b)
	}
	\Derive{[1]}{\bd \TYPE{TensorProduct}I^2(=,\to)}{\mu^*_A \FUNC{swap} = \mu^*_A }
	\Conclude{[*]}{\bd^{-1}\TYPE{Cocomutative}[1]}{\LOGIC{This} }
	\EndProof
	\\
	\Theorem{CocommutativeDualAlg}{\forall A \in \COALG{R} \. \forall [0] : (A : \TYPE{Cocommutative}(R)) \.  A^* \in \LCALGE{R}}
	\Assume{f,g}{A^*}
	\Assume{a}{A}
	\Conclude{[(f,g).*]}{ \bd \mu_{A*} \bd \TYPE{Cocomutative}(A) \THM{FunctionalSwap} \bd^{-1} \mu_{A^*}  }
	{
		fg(a) = (f \otimes g) \Delta(a) = 
		(f \otimes g)\Big( \Delta \; \FUNC{swap}(a) \Big) = \NewLine = 
		(g \otimes f) \Delta(a) =
		gf(a)
	}
	\Derive{[1]}{I(\forall)I(=,\to)}{\forall f,g \in At^* \. fg = gf}
	\Conclude{[*]}{\bd \LCALGE{R}}{\LOGIC{This} }
	\EndProof
	\\
	\DeclareFunc{linearlyRecursiveSequances}{\prod k : \Field \. \VS{k}}
	\DefineNamedFunc{linearlyRecursiveSequances}{ }{\mathrm{LR}(k)}{ \left\{ s \in K^{\Int_+} :  \exists! P(x) \in k[x] : \forall n \in \Int_+ \. s_{n + \deg P + 1} = \sum^{\deg P }_{i=0} P_i s_{n +i}  \right\}} 
	\\
	\DeclareFunc{linearlyRecursiveDegree}{\prod k : \Field \. \mathrm{LR}(k) \to \Nat}
	\DefineNamedFunc{linearlyRecursiveDegree}{s}{\deg s}{\deg P + 1 \; \where \; P = \bd \mathrm{LR}(k)(s)}
	\\
	\DeclareFunc{characteristicPolynomial}{\prod k : \Field \. \mathrm{LR}(k) \to \TYPE{Monic}(k) }
	\DefineNamedFunc{characteristicPolynomial}{s}{\chi_s(x)}{x^{\deg s} - P(x) \; \where \; P(x) = \bd \mathrm{LR}(k)(s)}
	\\
	\DeclareFunc{dualLRPolynomialEmbedding}{ \prod k : \Field \. \mathrm{LR}(k) \Arrow{\VS{k}} \Big(k[x] \Big)^*   }
	\DefineNamedFunc{dualLRPolynomialEmbedding}{s}{f_s}{\sum^\infty_{n=0} s_i \mathrm{d}x^n}
	\\
	\DeclareFunc{linearRecursion}{ \prod k : \Field \. \prod n \in \Nat \.  k^n \to k^n \to \mathrm{LR}(k)    }
	\DefineNamedFunc{linearRecursion}{a,v}{s}{\Lambda i \in \Int_+ \. \If i < n \Then v_{i+1} \Else \sum^n_{j=0} a_j s_{i - n + j}}
}\Page{
	\Theorem{LRIsomorphism}{\forall k : \Field \. f : \mathrm{LR}(k) \ToIso{\VS{k}} \Big(k[x]\Big)^\circ}
	\Assume{s}{\mathrm{LR}(k)}
	\Say{n}{\deg s}{\Nat}
	\Say{P}{\bd \mathrm{LR}(k)(s)}{k[x]}
	\Assume{m}{\Int_+}
	\Conclude{[m.*]}{ \bd f_s \bd \chi_s(x) \bd \mathrm{d}x \bd P  }{ 
		f_s \chi_s(x)x^m = 
		\sum^\infty_{i=1} s_i \mathrm{d}x^i  \big( x^{n+m} - P(x)x^m \big) =
		s_{n+m} - \sum^{n-1}_{i=0} s_{i+m} P_i = 
		0
	}
	\Derive{[1]}{\bd k[x]\bd^{-1} \FUNC{proncipleIdeal}(\chi_s(x))\bd \TYPE{Subset}\bd \ker f_s}
	{  \Big( \chi_s(x)  \Big) \subset \ker f_s  }
	\Say{[2]}{\THM{PrincipleQuotientDim}\big(\chi_s(x)\big)\bd \chi_s(x)}{  \dim \frac{k[x]}{\Big( \chi_s(x) \Big)} = \deg s  }
	\Say{[3]}{\bd^{-1} \TYPE{Cofinite}[2]}{\Big(\big(\chi_s(x)\big) : \TYPE{Cofinite}\big( k[x] \big)\Big)}
	\Conclude{(s.*)}{\bd \Big(k[x]\Big)^\circ [3][1]}{f_s \in \Big(k[x]\Big)^\circ}
	\Derive{[1]}{\bd \FUNC{image}}{ \im f \subset \Big( k[x] \Big)^\circ }
	\Assume{g}{\big( k[x]  \big)^\circ} 
	\Say{\Big(I,[2]\Big)}{\bd \big( k[x] \big)^\circ (g) }{ \sum I : \Ideal\big(k[x]\big) \. \dim \frac{k[x]}{I} < \infty \And g(I) = \{0\}}
	\Say{\Big(Q(x),[3]\Big)}{\bd \PID\big(k[x]\big)(I)[2]}{\sum Q(x) : \TYPE{Monic}(k) \. I = \Big( Q(x)\Big) }
	\Say{n}{\deg Q(x)}{\NNInt}
	\Assume{[4]}{n = 0}
	\Say{[5]}{[2][4]}{g = 0}
	\Conclude{[4.*]}{\bd f [5]}{g = f_0}
	\Derive{[4]}{I(\Imply)}{n = 0 \Imply \exists s \in \mathrm{LR}(k) : g = f_s}
	\Assume{[5]}{n \in \Nat }
	\Say{P(x)}{x^n - Q(x)}{k[x]}
	\Say{v}{\lambda i \in n \.  g(x^{i-1}) }{k^n}
	\Say{s}{\FUNC{linearRecursion}(P,v)}{\mathrm{LR}(k)}
	\Conclude{[5.*]}{\ByConstr  s \ByConstr v [2]}{ f_s = g }
	\Derive{[5]}{I(\Imply)}{n \in \Nat \Imply \exists s \in \mathrm{LR}(k) : g = f_s }
	\Conclude{[g.*]}{E(|)\bd (\Int_+)[4][5]}{ \exists s \in \mathrm{LR}(k) : g = f_s }
	\Derive{[2]}{\bd^{-1} \TYPE{Surjective}}{\Big( f : \mathrm{LR}(k) \ToSurj \big(k[x]\big)^\circ  \Big)}
	\Conclude{[*]}{\bd f [2]}{\Big( f : \mathrm{LR}(k) \ToIso{\VS{k}} \big(k[x]\big)^\circ  \Big)}
	\EndProof
	\\
	\DeclareFunc{linearlyRecursiveCoalgebra}{\forall k : \Field \. \COALG{k}}
	\DefineNamedFunc{linearlyRecursiveCoalgebra}{}{\mathrm{LR}(k)}{\Big( \mathrm{LR}(k), f^* \Delta_{\big(k[x]\big)^\circ}(f^{-1} \otimes f^{-1}), f^* \eta_{\big(k[x]\big)^\circ}  \big)}
}\Page{
	\DeclareFunc{hitAction}{ \prod A : \LALGE{R} \. A \Arrow{\LMOD{R}} ( A^* \LMOD{R} A^*)}
	\DefineNamedFunc{hitAction}{a,f}{a \hit f}{\Lambda b \in A \. f(ab) }
	\\
	\DeclareFunc{hitByAction}{\prod A : \LALGE{R} \. A \Arrow{\LMOD{R}} (A^* \LMOD{R} A^*)}
	\DefineNamedFunc{hitByAction}{a,f}{f \hitBy a}{\Lambda b \in A \. f(ba) }
	\\
	\Theorem{FiniteHitAction}{\forall R : \Field \. \forall A : \LALGE{R} \. \forall f \in A^* \. f \in A^\circ \iff \dim(A \hit f) < \infty }
	\Assume{[1]}{f \in A^\circ}
	\Say{\Big(I,[2],[3]\Big) }{ \bd \FUNC{finiteDual}[1] }{  \sum I : \Ideal(A) \. I \subset \ker f  \And \dim \frac{A}{I} < \infty   }
	\Say{\Big(\overline{f},[4]\Big)}{ \THM{IsomorphismTHM}\big(A,I,[2]\big)  }{ \sum \overline{f} : \frac{A}{I} \Arrow{\LMOD{R}} k \.  f = \pi_I \overline{f}   }
	\Assume{a,b}{A}
	\Conclude{[(a,b)*]}{\bd \FUNC{hitAction}(a,f)[4]}{  (a \hit f)(b) = f(ab) = \overline{f}[ab] }
	\DeriveConclude{[1]}{\bd^{-1} \TYPE{Injective} \; \THM{InjectiveDim} \; \THM{ImageDim}(\cdot \hit \overline{f})[3] }
	{  \dim (A \hit f)  \le  \dim \left( \frac{A}{I} \hit \overline{f} \right) < \infty  } 
	\Derive{[1]}{I(\Imply)}{ f \in A^\circ \Imply \dim (A \hit f) < \infty}
	\Assume{[2]}{ \dim( A \hit f) < \infty} 
	\Say{K}{\{ a \in A : \forall b,c \in A  \.  f(bac)  \}}{\TYPE{Submodule}(A)}
	\Say{[3]}{\bd \ker f \bd K}{K \subset \ker f}
	\Say{[4]}{\bd^{-1} \bd \Ideal \bd \ker f}{ \Big(K : \Ideal(A)\Big) }
	\Say{[5]}{ \THM{KerImTHM} \; \THM{SubsetDim} \; \THM{EndDim}[2]  }
	{  \dim \frac{A}{I} = \dim \Big(A \hit (A \hit f)\Big) \le \dim_R \End_{\VS{R}}( A \hit f) < \infty }
	\Conclude{[2.*]}{ \bd A^\circ [5]}{f \in A^\circ }
	\Derive{[2]}{I(\Imply)}{ \dim (A \hit f) < \infty \Imply f \in A^\circ  }
	\Conclude{[*]}{I(\iff)[1][2]}{ f \in A^\circ \iff \dim ( A \hit f) < \infty   }
	\EndProof
	\\
	\Theorem{FiniteHitByAction}{\forall R : \Field \. \forall A : \LALGE{R} \. \forall f \in A^* \.  f \in A^\circ \iff \dim (f \hitBy A) < \infty}
	\NoProof
	\\
	\Theorem{FiniteDualGrouplike}{\forall A : \LALGE{R} \. \forall f \in \A^\circ \. f : \TYPE{Grouplike}(A^\circ) \iff f : A \Arrow{\LALGE{R}} R }
	\NoProof
	\\
	\DeclareFunc{expEvaluation}{ \prod R \in \ANN \. \prod G : \TYPE{Monoid} \. R^G \Arrow{\LMOD{R}} \left( k[G] \right)^\circ  }
	\DefineNamedFunc{expEvaluation}{ f }{\phi(f)}{ \Lambda \alpha_i x^g \. \alpha f(g) }
	\\
	\DeclareFunc{representativeCoalgebra}{ \prod k : \Field \. \TYPE{Monoid} \to \COALG{k} } 
	\DefineNamedFunc{representativeCoalgebra}{G}{\R_k(G)}{\phi^{-1}\Big(\big(k[G]\big)^\circ\Big)}
	\\
}
\Page{
	\Theorem{FiniteCanonicalInjection}{\forall k : \Field \. \forall A : \COALG{k} \. \forall a \in A \. \epsilon(a) \in A^{\star\circ} 
		\NewLine \quad \where \quad \epsilon = \FUNC{canonicalInjection}(A)}
	\Assume{f,g }{A^*}
	\Conclude{[(f,g).*]}{\bd \FUNC{hitAction} \bd \FUNC{canonicalInjection}\bd \FUNC{dualAlgebra}\bd^{-1} \TYPE{CanonicalInjection}}{  
		\NewLine :
		\Big(f \hit \epsilon(a)\Big)(g) = 
		\epsilon(a)(fg) = 
		fg(a) = 
		\sum_{(a)} f(a_1)g(a_2) = 
		\sum_{(a)} f(a_1) \epsilon(a_2)(g)   
	}
	\Derive{[1]}{I(\forall)I(=,\to)}{\forall f \in A^* \.  \big(f \hit \epsilon(a)\big) = \sum_{(a)}  f(a_1)\epsilon(a_2)    } 
	\Say{[2]}{[1]\bd^{-1}\FUNC{span}}{ A^* \hit \epsilon(a) \subset \Span \{ \epsilon(a_2)   \}_(a) }
	\Say{[3]}{\bd \COALG{k}\bd^{-1} \FUNC{dimension}[2]}{\dim (A^* \hit \epsilon(a)) < \infty}
	\Conclude{[*]}{\THM{FiniteHitAction}[3]}{\epsilon(a) \in A^{*\circ}} \EndProof 
	\\ 	
	\Theorem{CanonicalInjectionCoalgHomo}{ \forall k : \Field \. \forall A : \COALG{k} \.  \epsilon : A \Arrow{\COALG{k}} A^{*\circ} \.  \NewLine \quad \where \quad \epsilon = \FUNC{canonicalInjection}(A) } 
	\Assume{a}{A} 
	\Conclude{[a.*]}{\bd  \FUNC{finiteDualCoalgebra} \bd \FUNC{dualAlgebra} \bd \FUNC{canonicalInjection} }{  \NewLine : \eta_{A^{*\circ}}\big(\epsilon(a)\big) = \epsilon(a)(e_{A^*}) = \epsilon(a)(\eta_A) = \eta_A(a) } 
	\Derive{[1]}{I(=,\to) }{\varepsilon \eta_{A^{*\circ}} = \eta_A} \Assume{a}{A} \Assume{f,g}{A^*} 
	\Conclude{[a.*]}{\bd \FUNC{dualFiniteCoalg} \bd \FUNC{canonicalInjection}\bd \FUNC{dualAlgebra} \bd^{-1} \FUNC{canonicalInjection} \bd \TYPE{SweedlerSum}  } 
	{ \NewLine : \Delta\big(\epsilon(a)\big)(f \otimes g) = \epsilon(a)(fg) = fg(a) = \sum_{(a)} f(a_1)g(a_2) = \sum_{(a)}  \Big(\epsilon(a_1) \otimes \epsilon(a_2)\Big)( f \otimes g  ) = (\epsilon \otimes \epsilon)(\Delta\;a)(f \otimes g)  } 
	\Derive{[2]}{I(=,\to)}{  \Delta \epsilon = (\epsilon \otimes \epsilon)\Delta   } 
	\Conclude{[3]}{\bd \COALG{k}(A,A^{*\circ})[1][2] }{  \LOGIC{This}   } 
	\EndProof 
	\\ 
	\DeclareType{Coreflexive}{\prod k : \Field \. ?\COALG{k}} 
	\DefineType{A}{Coreflexive}{ \epsilon : A \ToIso{\COALG{k}} A^{*\circ} \NewLine \quad \where \quad \epsilon = \FUNC{canonicalInjection}(A) } 
}\Page{ 
	\Theorem{TopologicalCoreflexivityCriterion}
	{ \forall k : \Field \.  \forall A \in \COALG{k} \.  \forall A : \TYPE{Coreflexive} \iff 
		\NewLine  
		\iff \forall I : \Ideal(A^*) \.  \dim \frac{A^*}{I} < \infty \Rightarrow I : \TYPE{Closed}\Big(A^*,\F(A,k)\Big) } 
	\Assume{[1]}{A : \TYPE{Coreflexive}} 
	\Assume{I}{\Ideal(A^*)} \Assume{[2]}{\dim \frac{A^*}{I} < \infty} 
	\Say{V}{ \epsilon^{-1}(I^\bot \cap A^{*\circ}) }{ \TYPE{VectorSubspace}(A) } 
	\Say{[3]}{\THM{ComplementDim}[2] }{ \dim I^\bot = \codim I < \infty  } 
	\Assume{h}{I^\bot} \Say{[4]}{\bd \TYPE{Orthogonal}(I,h)}{I \subset \ker h} 
	\Conclude{[h.*]}{\bd \FUNC{finiteComplement}[4]}{ h \in A^{*\circ}} 
	\Derive{[4]}{\bd \TYPE{Subset}\bd V[3]  }{ \dim V = \codim I < \infty   } 
	\Say{[5]}{\THM{OrthogonalIsomorphism}(V)}{  V^{\bot\bot} \cong_{\VS{k}} V  } 
	\Say{[6]}{\bd V \THM{DoubleOrthogonalTheorem}(I) }{ \overline{I} = V^\bot} 
	\Say{[7]}{\THM{ComplementDim}[4][6][2]}{\dim V^{\bot\bot} = \codim \overline{I}  = \dim V = \codim I}
	\Say{[8]}{\bd \FUNC{closure}\THM{EqualByCodimmension} [7]}{I = \overline{I}}
	\Conclude{[I.*]}{\bd \FUNC{closure}[8]}{\Big(I : \TYPE{Closed}\big(A^*,\F(A,k)\big)\Big)}  
	\Derive{LR}{I(\Imply)I(\forall)I(\Imply)}{\LOGIC{Left} \Imply \LOGIC{Right}}
	\Assume{R}{\LOGIC{Right}}
	\Assume{F}{A^{*\circ}\circ}
	\Say{\I}{\{ I : \Ideal(A^*) \. I \subset \ker F \And \codim I < \infty  \}}{?\Ideal(I)}
	\Say{[2]}{  \bd \FUNC{finiteDual}(A^*)(F) }{ \I \neq \emptyset  }
	\Assume{I}{\I}
	\Say{[3]}{\ByConstr \I (I)}{ I \subset \ker F \And \dim \frac{A^*}{I} < \infty}
	\Say{[4]}{R[3]}{\Big( I : \TYPE{Closed}\big(A^*,\F(A,k)\big)\Big)}
	\Say{\Big(V,[5]\Big)}{\THM{ClosedSubspaceIsOrtgogonal}}{\sum V \subvec{k} A \. V^\bot = I}
	\Say{[6]}{\THM{ClosedOrthogonalIsomorphism}}{ \epsilon_{|V} : V \ToIso{\VS{k}} I^\bot}
	\Conclude{[*.I]}{\bd \TYPE{Surjective}}{\exists a \in A : F = \epsilon(a) }
	\DeriveConclude{[2.*]}{I(\forall)[2]}{\exists a \in A : F = \epsilon(a)}
	\DeriveConclude{ [*] }{\bd^{-1}\TYPE{Coreflexive}I(\Imply)I(\iff)}{\LOGIC{Left} \iff \LOGIC{Right}}
	\EndProof
}\Page{
	\Theorem{CoalgebraAsRepresentative}{ 
		\forall k \in \Field \. 
		\forall A \in \COALG{k} \. \NewLine \.  
		\exists M : \TYPE{Monoid} :
		\exists R \subset_{\COALG{k}}  \R_k(M) :
		R \cong_{\COALG{k}} A
	}
	\Say{M}{(A^*,\mu)}{\TYPE{Monoid}}
	\Say{\varphi}{\Lambda a \in A \. \bd k[M] \Lambda f \in A^* \. f(a) }
	{ A \Arrow{\VS{k}} M^k    }
	\Assume{a}{A}
	\Say{[1]}{\bd \ker \ByConstr \varphi}{\ker \varphi(a) = \Big\langle\{a\}^\bot\Big\rangle}
	\Conclude{[a.*]}{\THM{FiniteCanonicalInjection}(A)}{\varphi(a) \in A^*}
	\Derive{[1]}{\bd \R_k(M)}{\varphi : A \Arrow{VS{k}} \R_k(M) }
	\Say{[2]}{\bd M}{ k\big[M\big] \cong_{\LALGE{K}} A^* }
	\Conclude{[*]}{\bd \varphi \bd \FUNC{finiteCanonicalInjection}}{ A \cong_{\COALG{k}} \varphi(A ) }
	\EndProof
	\\
	\Theorem{CanonicalInjectionAlgHomo}{ \forall k : \Field \. \forall A : \LALGE{k} \.  \epsilon : A \Arrow{\LALGE{k}} A^{\circ*} \.  \NewLine \quad \where \quad \epsilon = \FUNC{canonicalInjection}(A) } 
	\Assume{F}{A^{\circ*}}
	\Assume{f}{A^\circ}
	\Say{[a.*.1]}{  \bd \FUNC{dualAlgebra} \FUNC{canonicalInjection}  \bd \FUNC{finiteDualCoalg} \bd \VS{k}(A,k)(f_2) \bd \COALG{k}(A^\circ)  }
	{  
		\NewLine :
		\epsilon(e)F(f)  =  
		\sum_f  \epsilon(e)(f_1) F(f_2) = 
		\sum_f  f_1(e) F(f_2)   = 
		F \left(  \sum_f  \eta(f_1) f_2  \right) = 
		F(f)
	} 
	\Say{[a.*.2]}{  \bd \FUNC{dualAlgebra} \FUNC{canonicalInjection}  \bd \FUNC{finiteDualCoalg} \bd \VS{k}(A,k)(f_1) \bd \COALG{k}(A^\circ)  }
	{  
		\NewLine :
		 F\epsilon(e)(f)  =  
		\sum_f   F(f_1)\epsilon(e_1) = 
		\sum_f  F(f_1)f_2(e)    = 
		F \left(  \sum_f  \eta(f_2) f_1  \right) = 
		F(f)
	} 
	\Derive{[1]}{I(=,\to) }{ \epsilon(e) = e} 
	\Assume{a,b}{A} 
	\Assume{f}{A^\circ} 
	\Conclude{[a.*]}
	{ \bd \FUNC{dualAlgebra} \bd \FUNC{canonicalInjection} \bd \FUNC{finiteDualCoalg} \bd^{-1} \FUNC{canonicalInjection}  } 
	{
		\NewLine:
		\epsilon(a)\epsilon(b)(f) =
		\sum_{f} \epsilon(a)(f_1)\epsilon(b)(f_2) = 
		\sum_{f} f_1(a)f_2(b) = 
		f(ab) =
		\epsilon(ab)(f)
	} 
	\Derive{[2]}{I(=,\to)}{  \mu \epsilon = (\epsilon \otimes \epsilon)\mu   } 
	\Conclude{[3]}{\bd \LALGE{k}(A,A^{\circ*})[1][2] }{  \LOGIC{This}   } 
	\EndProof
	\\
	\DeclareType{Proper}{\prod k : \Field \. ?\LALGE{k}}
	\DefineType{A}{Proper}{\epsilon_{|A^\circ} : \TYPE{Injective}(A,A^{\circ*})} 
	\\
	\DeclareType{WeaklyReflexive}{\prod k : \Field \. ?\LALGE{k}}
	\DefineType{A}{WeaklyReflexive}{\epsilon_{|A^\circ} : \TYPE{Surjective}(A,A^{\circ*})} 
	\\
	\DeclareType{Reflexive}{\prod k : \Field \. ?\LALGE{k}}
	\DefineType{A}{Reflexive}{\epsilon_{|A^\circ} : \TYPE{Bijective}(A,A^{\circ*})} 
}\Page{
	\Theorem{TopologicalPropernesCriterion}
	{
		\forall k : \Field \.  
		\forall A : \LALGE{k} \. 
		A : \TYPE{Proper}(k) \iff
		A^\circ : \TYPE{Dense}\Big(A^*, \F(A,k) \Big)
	}
	\Assume{[1]}{\left( A : \TYPE{Proper}(k) \right)}
	\Assume{f}{A^*}
	\Assume{U}{O \in \mathcal{U}(f)}
	\Assume{[2]}{O \cap A^{\circ} = \emptyset}
	\Say{\Big( n, a, \alpha, [3]\Big)}{  \bd \F(A,k)[2] }{ 
		\sum n \in \Nat \. \sum a : \LI(n,A)  \. 
		\sum \alpha : n \to A \. \NewLine \. 
		\forall f \in A^\circ \. 
		\exists i \in n \. f(a_i) \neq \alpha_i 
	}
	\Say{\Big(i,[4]\Big)}{ [3](0) }{\sum i \in n \. \alpha_i \neq 0}
	\Say{[5]}{ \bd \VS{k}(A^*)[3][4]}{\forall f \in A^* \. f(a_i) = 0}
	\Say{[6]}{ \bd \LI(n,A)(a)(a_i) }{a_i \neq 0}
	\Say{[7]}{\bd \TYPE{Proper}(k)(A)\bd\TYPE{Injectivive}(A,A^{\circ*})[6]}{\epsilon_{|A^\circ}(a)\neq 0}
	\Say{[8]}{\bd^{-1}\epsilon_{|A^\circ}[5]}{\epsilon_{|A^\circ}(a) = 0}
	\Conclude{[1.*]}{[7][8]}{\bot}
	\Derive{LR}{E(\bot)\bd^{-1}I^2(\forall)\TYPE{Dense}\Big(A^*,\F(A,k)\Big)I(\Rightarrow)}
	{  \Big( A : \TYPE{Proper}(k) \Imply A^\circ : \TYPE{Dense}\big(A^*,\F(A,k)\big)\Big) }  
	\Assume{[1]}{\Big( A^\circ : \TYPE{Dense}\big( A^*,\F(A,k)\big) \Big)}
	\Assume{a,b}{A}
	\Assume{[2]}{a \neq b}
	\Say{\Big(f,[3]\Big)}{\bd \TYPE{Injective}(\epsilon)\big(a,b,[2]\big)\bd \epsilon}{\sum f \in A^* \. f(a) \neq f(b)}
	\Say{U}{\{ g \in A^* : g(a) = f(a) \And g(b) = f(b) \}}{\F(A,k)}
	\Say{\Big(g,[4]\Big)}{\bd \TYPE{Dense}\Big(A, \F(A,k)\Big)(A^\circ)(U)}{\sum g \in A^\circ \. g \in U }
	\Say{[5]}{\ByConstr(U)[4][3]}{g(a) = f(a) \neq f(b) = g(b)}
	\Conclude{[1.*]}{\bd^{-1} \varepsilon_{|A^\circ} [5]}{ \varepsilon_{|A^\circ}(a) \neq \varepsilon_{|A^\circ}(b)  }
	\DeriveConclude{[*]}{I(\Imply)I(\forall)\bd^{-1}\TYPE{Injective}\bd^{-1}\TYPE{Reflexive}I(\Imply)I(\iff)(LR)}{\LOGIC{This}}
	\EndProof
}\Page{
	\Theorem{IdealPropernesCriterion}
	{
		\forall k : \Field \.  
		\forall A : \LALGE{k} \. \NewLine \.  
		A : \TYPE{Proper}(k) \iff
		\bigcap \{ I : \Ideal(A) : \codim I < \infty  \} = \{0\} 
	}
	\Assume{[1]}{\left( A : \TYPE{Proper}(k) \right)}
	\Assume{a}{\bigcap\{ I : \Ideal(A) : \codim I < \infty  \} = \{0\} }
	\Say{[2]}{\bd A^\circ \ByConstr(a)}{\forall f \in A^\circ \. f(a) = 0}
	\Say{[3]}{\bd^{-1} \epsilon_{|A^\circ}[2]}{\epsilon_{|A^\circ}(a) = 0}
	\Conclude{[1.*]}{\bd \TYPE{Proper}(k)(A)\bd\TYPE{Injective}[3]}{a = 0}
	\Derive{[LR]}{I(\forall)\bd^{-1}\TYPE{Singleton}I(\Imply)}{
		A : \TYPE{Proper}(k) \Imply
		\bigcap \{ I : \Ideal(A) : \codim I < \infty  \} = \{0\} 
	}
	\Assume{[1]}{\bigcap \{ I : \Ideal(A) : \codim I < \infty  \} = \{0\}}
	\Assume{a}{A}
	\Assume{[2]}{ a \neq 0 }
	\Say{\Big(I,[3]\Big)}{[1][2]}{\sum I :\Ideal(A) \. \codim I < \infty \And a \not \in I}
	\Say{\Big(f,[4])}{\THM{FunctionalConstruction}[3]}{\sum f \in A^* \. I \subset \ker f \And f(a) = 1}
	\Say{[5]}{\bd A^\circ [4]}{f \in A^\circ}
	\Conclude{[6]}{[5][4]}{ \epsilon_{|A^\circ}(a)}
	\DeriveConclude{[*]}{I(\Imply)I(\forall)\bd^{-1}\TYPE{Injective}\bd^{-1}\TYPE{Reflexive}I(\Imply)I(\iff)(LR)}{\LOGIC{This}}
	\EndProof
	\\
	\Theorem{DualAlgebraLeftAdjoint}{ \forall k :  \Field \. \NewLine \.  \Big( \FUNC{finiteDualCoalgebra}(k),\FUNC{dualAlgebra}(k)\Big):\TYPE{LeftAdjoint}( \LALGE{k},\COALG{k})  }
	\NoProof
}
\newpage
\subsection{Main Theorem of Coalgebras}
\Page{
	\DeclareType{Subcoalgebra}{\prod R \in \ANN \. \prod A \in \COALG{R} \. ??A}
	\DefineNamedType{B}{Subcoalgebra}{B \subset_{\COALG{R}} A }{(B,\Delta_A,\eta_A) \in \COALG{R} }
	\\
	\Theorem{IdealsSubcoalgebrasDuality}{\forall k : \Field \. \forall A : \COALG{k} \. \forall I : \Ideal(A^*) \. \epsilon^{-1}(I^\bot) : \TYPE{Subcoalgebra}(A) }
	\Say{B}{\epsilon^{-1} \; I^\bot}{\TYPE{VectorSubspace}(A)}
	\Assume{b}{B}
	\Assume{n}{\Nat}
	\Assume{v,u}{\LI(n,A)}
	\Assume{[5]}{\Delta(b) = \sum^n_{i=1} v_i \otimes u_i}
	\Assume{i}{n}
	\Assume{[6]}{v_i \not \in B}
	\Say{\Big( f ,[7]\Big)}{ \bd B [6]}{\sum f \in I \. f(v_i) \neq 0}
	\Say{\Big(g,[8]\Big)}{\THM{AlgebraicReizReprezentationTHM}(u_i,1,\widehat{u}_i)}
	{
		\sum g \in A^* \. g(u_i) = 1 \And \forall j \in (n-1) \. h(u_j) = 0
	}
	\Say{[9]}{ \bd \FUNC{dualAlgebra}(A)[8][7]}{fg(b) = \sum^n_{i=1} f(v_i)g(u_i) = f(v_i) \neq 0}
	\Say{[10]}{\bd \Ideal(I)(f,g)}{ fg \in I }
	\Say{[11]}{ \bd B[10]  }{fg(b) = 0}
	\Conclude{[6.*]}{[9][11]}{\bot}
	\Derive{[b.*.1]}{E(\bot)}{v_i \in B}
	\Assume{[6]}{u_i \not \in B}
	\Say{\Big( f ,[7]\Big)}{ \bd B [6]}{\sum f \in I \. f(u_i) \neq 0}
	\Say{\Big(g,[8]\Big)}{\THM{AlgebraicReizReprezentationTHM}(v_i,1,\widehat{v}_i)}
	{
		\sum g \in A^* \. g(v_i) = 1 \And \forall j \in (n-1) \. h(v_j) = 0
	}
	\Say{[9]}{ \bd \FUNC{dualAlgebra}(A)[8][7]}{gf(b) = \sum^n_{i=1} g(v_i)f(u_i) = f(u_i) \neq 0}
	\Say{[10]}{\bd \Ideal(I)(f,g)}{ gf \in I }
	\Say{[11]}{ \bd B[10]  }{gf(b) = 0}
	\Conclude{[6.*]}{[9][11]}{\bot}
	\DeriveConclude{[b.*.2]}{E(\bot)}{u_i \in B}
	\DeriveConclude{[5]}{\bd \COALG{k}\bd \TYPE{Subcoalgebra}}{\Big( B : \TYPE{Subcoalgebra}(V) \Big) }
	\EndProof
	\\
	\Theorem{SubcoalgebrasIdealsDuality}{\forall k : \Field \. \forall A \in \COALG{k} \. \forall B : \subset_{\COALG{k}} A \. B^\bot : \Ideal(A^*) }
	\NoProof
	\\
}
\Page{
	\Theorem{QuotientDuality}{\forall k : \Field \. \forall A \in \COALG{k} \. \forall B : \subset_{\COALG{k}} A \. B^* \cong_{\LALGE{k}} \frac{A^*}{B^\bot} }
	\NoProof
	\\
	\Theorem{MainTheoremOfCoalgebras}{ \forall k : \Field \. \forall A \in \COALG{k} \. \forall a \in A \. \exists B \subset_{\COALG{k}} A : a \in B \And \dim B < \infty } 
	\Say{[1]}{\THM{FiniteCanonicalInjection}(a)}{\epsilon(a) \in A^{*\circ}}
	\Say{\Big([2],I)}{\bd \FUNC{finiteDual}[1]}{\sum I : \Ideal(A^*) \. \codim I < \infty \And I \subset \ker \epsilon(a)}
	\Say{B}{\epsilon^{-1} \; I^\bot}{\TYPE{VectorSubspace}(A)}
	\Say{[3]}{\THM{InjectionDim}(\epsilon)  \THM{OrthogonalDim}(I)[2]}{\dim B \le \dim I^\bot = \codim I < \infty}
	\Say{[4]}{\bd \FUNC{preimage}\bd \FUNC{kernel}\ByConstr B [2]}{ a \in B }
	\Say{[5]}{\THM{IdealsSubcoalgebraDuality}(A)(I)\ByConstr(B)}{ \Big(B : \TYPE{Subcoalgebra}(A)\Big)  }
	\Conclude{[*]}{I(\And)[3][4][5]}{\LOGIC{This}}
	\EndProof
	\\
	\Theorem{IdealsSubcoalgebrasDuality2}{\forall k : \Field \. \forall A \in \LALGE{k} \. \forall I : \Ideal(A) \. I^\bot \cap A^\circ : \TYPE{Subcoalgebra}(A^\circ) }
	\NoProof
	\\
	\Theorem{SubcoalgebrasIdealsDuality2}{\forall k : \Field \. \forall A \in \LALGE{k} \. \forall B : \subset_{\COALG{k}} A^\circ \. \epsilon^{-1}\Big(B^\bot\Big) : \Ideal(A) }
	\NoProof
	\\
	\Theorem{CoidealsSubalgebrasDuality}{\forall k : \Field \. \forall A \in \COALG{k} \. \forall I : \TYPE{Coideal}(A) \. I^\bot  : \TYPE{Subalgebra}(A^*) }
	\NoProof
	\\
	\Theorem{SubalgebrasCoidealsDuality}{\forall k : \Field \. \forall A \in \COALG{k} \. \forall B  \subset_{\LALGE{k}}  A^*  \. \epsilon^{-1}\Big(B^\bot\Big) : \TYPE{Coideal}(A) }
	\NoProof
	\\
	\Theorem{SubalgebrasXoidealsDuality2}{\forall k : \Field \. \forall A \in \LALGE{k} \. \forall B \subset_{\LALGE{k}} A  \. B^\bot \cap A^\circ : \TYPE{Coideal}(A^\circ) }
	\NoProof
}\Page{
	\Theorem{CoidealsSubalgebraDuality2}{\forall k : \Field \. \forall A \in \LALGE{k} \. \forall I : \TYPE{Coideal}(A^\circ) \. \epsilon^{-1}\Big(I^\bot\Big) : \TYPE{Subalgebra}(A) }
	\NoProof
	\\
	\Theorem{LeftIdealsCoidealsDuality}{\forall k : \Field \. \forall A \in \LALGE{k} \. \forall I : \TYPE{LeftIdeal}(A) \. I^\bot \cap A^\circ : \TYPE{LeftCoideal}(A^\circ) }
	\NoProof
	\\
	\Theorem{LeftCoidealIdealsDuality}{\forall k : \Field \. \forall A \in \LALGE{k} \. \forall I : \TYPE{LeftCoideal}( A^\circ) \. \epsilon^{-1}\Big(I^\bot\Big) : \TYPE{LeftIdeal}(A) }
	\NoProof
	\\
	\Theorem{LeftCoidealIdealsDuality2}{\forall k : \Field \. \forall A \in \COALG{k} \. \forall I : \TYPE{LeftCoideal}(A) \. I^\bot  : \TYPE{LeftIdeal}(A^*) }
	\NoProof
	\\
	\Theorem{LeftIdealsCoidealsDuality}{\forall k : \Field \. \forall A \in \COALG{k} \. \forall I : \TYPE{LeftIdeal}(A^*)  \. \NewLine \. \epsilon^{-1}\Big(I^\bot\Big) : \TYPE{LeftCoideal}(A) }
	\NoProof
	\\
	\Theorem{RightIdealsCoidealsDuality}{\forall k : \Field \. \forall A \in \LALGE{k} \. \forall I : \TYPE{RightIdeal}(A) \. \NewLine \. I^\bot \cap A^\circ : \TYPE{RightCoideal}(A^\circ) }
	\NoProof
	\\
	\Theorem{RightCoidealIdealsDuality}{\forall k : \Field \. \forall A \in \LALGE{k} \. \forall I : \TYPE{RightCoideal}( A^\circ) \. \NewLine \. \epsilon^{-1}\Big(I^\bot\Big) : \TYPE{RightIdeal}(A) }
	\NoProof
}\Page{
	\Theorem{RightCoidealIdealsDuality2}{\forall k : \Field \. \forall A \in \COALG{k} \. \forall I : \TYPE{RightCoideal}(A) \. I^\bot  : \TYPE{RightIdeal}(A^*) }
	\NoProof
	\\
	\Theorem{RightIdealsCoidealsDuality}{\forall k : \Field \. \forall A \in \COALG{k} \. \forall I : \TYPE{RightIdeal}(A^*)  \. \NewLine \. \epsilon^{-1}\Big(I^\bot\Big) : \TYPE{RightCoideal}(A) }
	\NoProof
	\\
	\Theorem{SubcoalgebraIntersection}{\forall k : \Field \. \forall A \in \COALG{k}\. \forall X \in \SET \. \forall I : X \to \TYPE{Subcoalgebra}(A) \. \NewLine \. \bigcap_{x \in X} I_x : \TYPE{Subcoalgebra}(A)}
	\NoProof
	\\
	\Theorem{LeftCoidealIntersection}{\forall k : \Field \. \forall A \in \COALG{k}\. \forall X \in \SET \. \forall I : X \to \TYPE{LeftCoideal}(A) \. \NewLine\. \bigcap_{x \in X} I_x : \TYPE{LeftCoideal}(A)}
	\NoProof
	\\
	\Theorem{RightCoidealIntersection}{\forall k : \Field \. \forall A \in \COALG{k}\. \forall X \in \SET \. \forall I : X \to \TYPE{RightCoideal}(A) \. \NewLine\. \bigcap_{x \in X} I_x : \TYPE{RightCoideal}(A)}
	\NoProof
}
\newpage
\subsection{Tensor Products of Coalgebras}
\Page{
	\DeclareType{GradedCoalgebra}
	{
		\prod R \in \ANN \. 
		\prod I : \TYPE{Monoid} \. 
		\sum M : \LMOD{R}(I) \. \NewLine \. 
		M \Arrow{\LMOD{R}(I)} M \otimes M \times
		M \Arrow{\LMOD{R}(I)} k \. 
	}
	\DefineType{(M,\Delta,\eta)}{GradedCoalgebra}{(M,\Delta,\eta) \in \COALG{R}}
	\\
	\DeclareType{GradedCoalgebraHomo} 
	{
		\prod R \in \ANN \.
		\prod I : \TYPE{Monoid} \.
		\prod A,B : \TYPE{GradedCoalgebra}(R,I) \.
		A \Arrow{\LMOD{R}(I)} B
	}
	\DefineType{f}{GradedCoalgebraHomo}{ f : A \Arrow{\COALG{R}} B \And f : A \Arrow{\LMOD{R}(\I)} B}
	\\
	\DeclareFunc{categoryOfGradedCoalgebras}
	{
		\ANN \to \TYPE{Monoid} \to \CAT
	}
	\DefineNamedFunc{categoryOfGradedCoalgebra}{R,M}{\COALG{R}(M)}{ \Big( \TYPE{GradedCoalgebra}, \TYPE{GradedCoalgebraHomo}, \id,\circ \Big) } 
	\\
	\Theorem{TensorProductOfCoalgebraHomo}
	{
		\forall R \in \ANN \.
		\forall X, X', Y, Y' : \COALG{R} \. \NewLine \. 
		\forall \varphi : X \Arrow{\COALG{R}} Y \. 
		\forall \psi : X' \Arrow{\COALG{R}} Y' \. 
		\varphi \otimes \psi : X \otimes X' \Arrow{\COALG{R}} Y \otimes Y' 
	}
	\Assume{x}{X}
	\Assume{x'}{X'}
	\Conclude{[x'.*.1]}{\bd \FUNC{homoTensorProduct} \bd \COALG{R}(X,X')(\varphi)\bd \COALG{R}(Y,Y')(\psi)\bd \FUNC{coalgebraTensorProduct}}
	{  
		\NewLine:
		\Delta\Big((\varphi \otimes \psi)(x \otimes x')\Big) = 
		\Delta\Big(\varphi(x) \otimes \psi(x')\Big) =
		\sum_{x,x'} (\varphi(x_1) \otimes \psi(x'_1) \otimes (\varphi(x_1) \otimes \psi(x'_1)) = \NewLine =  
		(\varphi \otimes \psi) \otimes (\varphi \otimes \psi) \Delta(x \otimes x')
	}
	\Conclude{[x'.*.2]}{\bd \FUNC{homoTensorProduct} \bd \COALG{R}(X,X')(\varphi)\bd \COALG{R}(Y,Y')(\psi)\bd \FUNC{coalgebraTensorProduct}}
	{  
		\NewLine :
		\eta\Big((\varphi \otimes \psi)(x \otimes x')\Big) = 
		\eta\Big(\varphi(x) \otimes \psi(x')\Big) =
		\eta\Big(\varphi(x)\Big)\Big( \psi(x)\Big)  = 
		\eta(x)\eta'(x') = \eta(x \otimes x')
	}
	\DeriveConclude{[*]}{I(\forall)\bd \TYPE{TensorProduct}}{ \varphi \otimes \psi : X \otimes X' \Arrow{\COALG{R}} Y \otimes Y'  }
	\EndProof
	\\
	\Theorem{CoalgTensorProductAssociativty}{\forall R \in \ANN \. \forall A,B,C \in \COALG{R} \. 
		\NewLine \. (A \otimes B) \otimes C \cong_{\COALG{R}} A \otimes (B \otimes C)}
	\NoProof
	\\
	\Theorem{CoalgTensorProductPermutation}{
		\forall R \in \ANN \. 
		\forall n \in \Nat \. 
		\forall A : n \to  \COALG{R} \. 
		\forall \sigma \in S_n \. \NewLine \.
		\bigotimes^n_{i=1} A_i \cong_{\COALG{R}} \bigotimes^n_{i=1} A_{\sigma(i)} 
	}
	\NoProof
	\\
	\Theorem{CoalgTrivialTensorProduct}{
		\forall R \in \ANN \. 
		\forall A : n \to \COALG{R} \. 
		R \otimes A \cong_{\COALG{R}} A
	}
	\NoProof
}
\Page{
	\Theorem{TensorProductOfGradedCoalgebraHomo}
	{
		\forall R \in \ANN \.
		\forall M : \TYPE{Monoid}
		\forall X, X', Y, Y' : \COALG{R}(M) \. \NewLine \. 
		\forall \varphi : X \Arrow{\COALG{R}(M)} Y \. 
		\forall \psi : X' \Arrow{\COALG{R}(M)} Y' \. 
		\varphi \otimes \psi : X \otimes X' \Arrow{\COALG{R}(M)} Y \otimes Y' 
	}
	\NoProof
	\\
	\Theorem{CoalgTensorProductAssociativty}{
		\forall R \in \ANN \.
		\forall M : \TYPE{Monoid} \.
		\forall A,B,C \in \COALG{R}(M) \. 
		\NewLine \. (A \otimes B) \otimes C \cong_{\COALG{R}(M)} A \otimes (B \otimes C)}
	\NoProof
	\\
	\Theorem{GradedCoalgTensorProductPermutation}{
		\forall R \in \ANN \. 
		\forall M : \TYPE{Monoid} \.
		\forall n \in \Nat \. 
		\forall A : n \to  \COALG{R}(M) \.  \NewLine \. 
		\forall \sigma \in S_n \. 
		\bigotimes^n_{i=1} A_i \cong_{\COALG{R}(M)} \bigotimes^n_{i=1} A_{\sigma(i)} 
	}
	\NoProof
	\\
	\Theorem{CoalgTrivialTensorProduct}{
		\forall R \in \ANN \. 
		\forall M : \TYPE{Monoid} \.
		\forall A : n \to \COALG{R}(M) \. 
		R \otimes A \cong_{\COALG{R}(M)} A
	}
	\NoProof
	\\
	\DeclareFunc{skewTensorProductOfCoalgebras}
	{
		\prod R \in \ANN \. 
		\prod n \in \Nat \.
		n \to \COALG{R}(\Int) \to \COALG{R}(\Int) 
	}
	\DefineNamedFunc{skewTensorProductOfCoalgebras}
	{
		A
	}
	{
		\widetilde{\bigotimes}^n_{i=1} A_i
	}
	{
		\NewLine \de
		\bigg( A \otimes B ,  
		\bd \TYPE{TensorProduct } \bd \TYPE{GradedAlgebra}   \. \Lambda a \in \prod^n_{i=1}\TYPE{Homogeneous} A_i \. \NewLine \.  
			 \sum_a (-1)^{I,J} \bigotimes^n_{i=1} a_{i,1} \otimes  \bigotimes^n_{i=1} a_{i,2} 
			 \quad \where \quad I = (\deg a_{i,1})^n_{i=1}, J = (\deg a_{i,2})^n_{i=1}
		;\eta_{A \otimes B}  \bigg)
	}
	\\
	\Theorem{SkewTensorProductOfGradedHomo}
	{
		\forall R \in \ANN \.
		\forall n : \Nat \to \COALG{R}(\Int) \. \NewLine \. 
		\forall X, Y, : n \to  \COALG{R}(\Int) \.  
		\forall \varphi : \prod^n_{i=1} X_i \Arrow{\COALG{R}(\Int)} Y_i \. 
		\bigotimes^n_{i=1} \varphi_i :\widetilde{\bigotimes}^n_{i=1} X_i  \Arrow{\COALG{R}(\Int)} \widetilde{\bigotimes}^n_{i=1} Y_i 
	}
	\NoProof
}\Page{
	\Theorem{CoalgSkewTensorProductAssociativty1}{
		\forall R \in \ANN \.
		\forall n \in \Nat \.
		\forall A \in n \to \COALG{R}(\Int) \. 
		\NewLine \. A_1 \widetilde{\otimes} \widetilde{\bigotimes}^{n}_{i=2} A_i \cong_{\COALG{R}(\Int)} \widetilde{\bigotimes}^n_{i=1} A_i}
	\NoProof	
	\\
	\Theorem{CoalgSkewTensorProductAssociativty2}{
		\forall R \in \ANN \.
		\forall n \in \Nat \.
		\forall A \in n \to \COALG{R}(\Int) \. 
		\NewLine \. \left( \widetilde{\bigotimes}^{n-1}_{i=1} A_i \right) \widetilde{\otimes} A_n  \cong_{\COALG{R}(\Int)} \widetilde{\bigotimes}^n_{i=1} A_i}
	\NoProof	
	\\
	\Theorem{CoalgSkewTensorProductAssociativty}{
		\forall R \in \ANN \.
		\forall A,B,C \in \COALG{R}(\Int) \. 
		\NewLine \. (A \widetilde{\otimes} B) \widetilde{\otimes} C  \cong_{\COALG{R}(\Int)} 
		A \widetilde{\otimes } (B \widetilde{\otimes} A)}
	\NoProof
	\\
	\Theorem{TwistingCoalgebraHomomorphism}
	{
		\forall R \in \ANN \.
		\forall A,B \in \COALG{R}(\Int) \.
		\tau_{A,B} : A \widetilde{\otimes} B \ToIso{\COALG{R}(\Int)} B \widetilde{\otimes} A
	}
	\NoProof
	\\
	\DeclareFunc{categoryOfCocommutativeCoalgebras}{\ANN \to \CAT}
	\DefineNamedFunc{categoryOfCocommutativeCoalgebras}{R}{\CCOALG{R}}{ \Big( \TYPE{Cocommutaive},\TYPE{CoalgebraHomo},\circ,\id\Big) }
	\\
	\DeclareType{CoskewCoalgebra}{\prod R \in \ANN \. ?\COALG{R}}
	\DefineType{A}{CoskewCoalgebra}{ \Delta_A \; T_{A,A} =  \Delta_A  }
	\\
	\DeclareFunc{categoryOfCoskewCoalgebras}{\ANN \to \CAT}
	\DefineNamedFunc{categoryOfCoskewCoalgebras}{R}{\SCOALG{R}}{ \Big( \TYPE{CoskewCoalgebra},\TYPE{CoalgebraHomo},\circ,\id\Big) }
	\\
	\Theorem{TensorProductsPreserveCocommutativity}
	{
		\forall R \in \ANN \. \forall A,B \in \CCOALG{R} \. A \otimes B \in \CCOALG{R}
	}
	\NoProof
	\\
	\Theorem{SkewTensorProductsPreserveSkewCocommutativity}
	{
		\forall R \in \ANN \. \forall A,B \in \SCOALG{R} \. \NewLine \. A \widetilde{\otimes} B \in \SCOALG{R}
	}
	\NoProof
}
\Page{
	\DeclareFunc{counitalProjection}{
		\prod R \in \ANN \. 
		\prod n \in \Nat \. 
		\prod A : n \to \COALG{A} \. 
		\prod^n_{i=1} \bigotimes^n_{j=1} A_j \Arrow{\COALG{R}} A_i }
	\DefineNamedFunc{counitalProjection}
	{
		t
	}
	{
		\pi_i(t)
	}
	{
		\bigotimes_{j=1}^{i-1} \eta_{A_j} \otimes \id_{A_i} \otimes \bigotimes_{j=i+1}^n \eta_{A_j} (t)
	}
	\Assume{a}{\prod^n_{i=1} A_i}
	\Say{a.*.1}{
		\bd^{-1} \FUNC{SweedlerNotation} \bd \pi_i \bd 
		\TYPE{TensorFunc} 
		\bd \LMOD{R}(A_j,R)(\eta_j)
		\bd \COALG{R}(A_j)
		\NewLine 
		\bd \FUNC{SweedlerNotation} 
		\bd^{-1} \pi_i
	} 
	{
		\NewLine : 
		\bigotimes^n_{j=1} a_j \Delta \pi_i \otimes \pi_i =
		\sum_a \bigotimes^n_{j=1} a_{j,1} \otimes \bigotimes a_{j,2}
		\bigotimes_{j=1}^{i-1} \eta_{A_j} \otimes \id_{A_i} \otimes \bigotimes_{j=i+1}^n \eta_{A_j} (t)
		\otimes
		\bigotimes_{j=1}^{i-1} \eta_{A_j} \otimes \id_{A_i} \otimes \bigotimes_{j=i+1}^n \eta_{A_j} (t) = \NewLine = 
		\sum_a    \prod^n_{j=1,j\neq i} \eta(a_{j,1}) \eta(a_{j,2})  a_{i,1} \otimes a_{i,2} =
		\prod^n_{j=1,j \neq i} \eta\left( \sum_{a_{j}} \eta (a_{j,2})  a_{j,1} \right) \sum_{a_i} a_{i,1} \otimes a_{i,2}  =
		\prod^n_{j=1,j \neq i} \eta(a_j) \sum_{a_i} a_{i,1} \otimes a_{i,2}  = \NewLine = 
		\prod^n_{j=1,j\neq i} \eta(a_j) a_i \Delta = 
		\bigotimes^n_{j=1} a_j \; \pi_i \;  \Delta 
	}
	\Conclude{a.*.2}
	{
		\bd \pi_i \bd \LMOD{R}(A_i,R)(\eta_i)\bd^{-1} \eta
	}
	{
		\bigotimes^n_{j=1} a_j \pi_i \eta = 
		\prod^n_{j=1, j \neq 1} \eta( a_j) a_i \eta_{A_i} = 
		\prod^n_{j=1} \eta(a_j) = 
		\bigotimes^n_{j=1} a_j \eta
	}
	\DeriveConclude{[*]}{\bd \TYPE{TensorProduct}}{\LOGIC{This}}
	\EndProof
	\\
	\Theorem{TensorProductIsCCOALGProduct}
	{ \forall R \in \ANN \. \Big( \FUNC{tensorProduct},\pi\Big) : \TYPE{FiniteProduct}(\CCOALG{k}) }
	\Assume{n}{\Nat}
	\Assume{A}{n \to \CCOALG{R}}
	\Assume{B}{\CCOALG{R}}
	\Assume{\varphi}{\prod^n_{i=1} B \Arrow{\COALG{R}} A_i }                                 
	\Say{\psi}{ \Delta^n \bigotimes^n_{i=1} \varphi_i }{  B \Arrow{\COALG{R}} \bigotimes^n_{i=1} A_i }
	\Assume{i}{n}
	\Assume{b}{B}
	\Conclude{i.*}{
		\ByConstr \psi \ByConstr \pi_i
		\bd^{-1} \FUNC{SweedlersNotation}
		\bd \FUNC{tensorMap}(\varphi)
		\bd \FUNC{tensorMap}(\eta_A) \NewLine  
		\bd \LMOD{R}(B,A_i)(\varphi_i)\bd \COALG{R}(B,A_j)(\varphi_j)
		\bd \COALG{R}(B)
	}
	{ 
	       \NewLine : 
	       b \; \psi \; \pi_i =  
	       b \; \Delta^n \; \bigotimes^n_{j=1}  \varphi_j \; \bigotimes^{i-1}_{j=1} \eta_{A_j} \otimes \id_{A_i} \otimes \bigotimes^n_{j=i+1} \eta_{A_j} = 
	       \sum_b \bigotimes^n_{j=1} b_j \; \bigotimes^n_{j=1} \varphi_j \; 
	       \bigotimes^{i-1}_{j=1} \eta_{A_j} \otimes \id_{A_i} \otimes \bigotimes^n_{j=i+1} \eta_{A_j} = \NewLine = 
	       \sum_b \bigotimes^n_{j=1} \varphi_j(b_j) \bigotimes^{i-1}_{j=1} \eta_{A_j} \otimes \id_{A_i} \otimes \bigotimes^n_{j=i+1} \eta_{A_j} =
	       \sum_b  \prod^n_{j=1,j\neq i}  \eta\Big(\varphi_j(b_j)\Big) \varphi_i(b_i)  = 
	       \varphi_i\left( \sum_b \prod^n_{j=1, j \neq i}\eta(b_j) b_i   \right) = 
	       \varphi_i(b)
	}
	\Derive{[1]}{I(=,\to)I(\forall)}{\forall i \in n \. \psi \pi_i = \varphi_i}
	\Assume{\psi'}{B \Arrow{\COALG{R}} \bigotimes^n_{i=1} A_i}
	\Assume{[2]}{\forall i \in n \. \psi' \pi_i = \varphi_i}
	\Assume{b}{B}
}\Page{
	\Conclude{n.*}
	{  
		\ByConstr \psi [2]
		\bd^{-1} \TYPE{SweedlerSum} \bd \pi_i 
		\bd \FUNC{tensorMap} (\eta_{A})
		\bd \L\left(A; \bigotimes^n_{i=1} A_i\right)(\otimes)
		\bd \CCOALG{R}(B) \NewLine 
		\bd \FUNC{coalgebraTensorProduct}(A)
		\bd \COALG{R}\left( \bigotimes^n_{i=1} \right)
	}
	{
		b \; \psi = 
		b \; \Delta^n \; \otimes^n_{i=1} \varphi_i  = 
		\sum_b \bigotimes^n_{i=1} b_i   \; \otimes^n_{i=1} \psi' \pi_i = \NewLine = 
		\sum_b \bigotimes^n_{i=1} \left(  \psi'( b_i) \; \bigotimes^{i-1}_{j=1} \eta_{A_j} \otimes \id_{A_i} \otimes \bigotimes^n_{j=i+1} \eta_{A_j} \right) =
		\sum_b \bigotimes^n_{i=1}  \sum_{a_i = \psi'(b_i)}  \prod^n_{j=1,j\neq i} \eta(a_{i}^j)  a_{i}^i = \NewLine 
		\sum_b \sum_{a_i = \psi'(b_i)}  \prod_{i \neq j} \eta(a_{i}^j) \bigotimes^n_{i=1}  a_{i}^i =
		\sum_b \sum_{a_i = \psi'(b_i)}  \prod_{i = 1} \prod_{j = 2} \eta(a_{i}^j) \bigotimes^n_{i=1}  a_{1}^i = 
		\sum_b \prod_{i \neq 1} \eta\Big( \psi'(b_i) \Big) \psi'(b_1) = 
		b \; \psi'
	}
	\DeriveConclude{[*]}{\bd^{-1}\TYPE{FiniteProduct}}{\LOGIC{This}}
	\EndProof
	\Theorem{TensorProductIsSCOALGProduct}
	{ \forall R \in \ANN \. \Big( \FUNC{SkewTensorProduct},\pi\Big) : \TYPE{FiniteProduct}(\SCOALG{k}) }
	\Assume{n}{\Nat}
	\Assume{A}{n \to \SCOALG{R}}
	\Assume{B}{\SCOALG{R}}
	\Assume{\varphi}{\prod^n_{i=1} B \Arrow{\COALG{R}} A_i }                                 
	\Say{\psi}{ \Delta^n \bigotimes^n_{i=1} \varphi_i }{  B \Arrow{\COALG{R}} \bigotimes^n_{i=1} A_i }
	\Assume{i}{n}
	\Assume{b}{B}
	\Conclude{i.*}{
		\ByConstr \psi \ByConstr \pi_i
		\bd^{-1} \FUNC{SweedlersNotation}
		\bd \FUNC{tensorMap}(\varphi)
		\bd \FUNC{tensorMap}(\eta_A) \NewLine  
		\bd \LMOD{R}(B,A_i)(\varphi_i)\bd \COALG{R}(B,A_j)(\varphi_j)
		\bd \COALG{R}(B)
	}
	{ 
	       \NewLine : 
	       b \; \psi \; \pi_i =  
	       b \; \Delta^n \; \bigotimes^n_{j=1}  \varphi_j \; \bigotimes^{i-1}_{j=1} \eta_{A_j} \otimes \id_{A_i} \otimes \bigotimes^n_{j=i+1} \eta_{A_j} = 
	       \sum_b \bigotimes^n_{j=1} b_j \; \bigotimes^n_{j=1} \varphi_j \; 
	       \bigotimes^{i-1}_{j=1} \eta_{A_j} \otimes \id_{A_i} \otimes \bigotimes^n_{j=i+1} \eta_{A_j} = \NewLine = 
	       \sum_b \bigotimes^n_{j=1} \varphi_j(b_j) \bigotimes^{i-1}_{j=1} \eta_{A_j} \otimes \id_{A_i} \otimes \bigotimes^n_{j=i+1} \eta_{A_j} =
	       \sum_b  \prod^n_{j=1,j\neq i}  \eta\Big(\varphi_j(b_j)\Big) \varphi_i(b_i)  = 
	       \varphi_i\left( \sum_b \prod^n_{j=1, j \neq i}\eta(b_j) b_i   \right) = 
	       \varphi_i(b)
	}
	\Derive{[1]}{I(=,\to)I(\forall)}{\forall i \in n \. \psi \pi_i = \varphi_i}
	\Assume{\psi'}{B \Arrow{\COALG{R}} \bigotimes^n_{i=1} A_i}
	\Assume{[2]}{\forall i \in n \. \psi' \pi_i = \varphi_i}
	\Assume{b}{B}
	\Conclude{n.*}
	{  
		\ByConstr \psi [2]
		\bd^{-1} \TYPE{SweedlerSum} \bd \pi_i 
		\bd \FUNC{tensorMap} (\eta_{A})
		\bd \L\left(A; \bigotimes^n_{i=1} A_i\right)(\otimes)
		\bd \SCOALG{R}(B) \NewLine 
		\bd \FUNC{coalgebraTensorProduct}(A)
		\bd \COALG{R}\left( \bigotimes^n_{i=1} \right)
	}
	{
		b \; \psi = 
		b \; \Delta^n \; \otimes^n_{i=1} \varphi_i  = 
		\sum_b \bigotimes^n_{i=1} b_i   \; \otimes^n_{i=1} \psi' \pi_i = \NewLine = 
		\sum_b \bigotimes^n_{i=1} \left(  \psi'( b_i) \; \bigotimes^{i-1}_{j=1} \eta_{A_j} \otimes \id_{A_i} \otimes \bigotimes^n_{j=i+1} \eta_{A_j} \right) =
		\sum_b \bigotimes^n_{i=1}  \sum_{a_i = \psi'(b_i)}  \prod^n_{j=1,j\neq i} \eta(a_{i}^j)  a_{i}^i = \NewLine 
		\sum_b \sum_{a_i = \psi'(b_i)}  \prod_{i \neq j} \eta(a_{i}^j) \bigotimes^n_{i=1}  a_{i}^i =
		\sum_b \sum_{a_i = \psi'(b_i)}  \prod_{i = 1} \prod_{j = 2} \eta(a_{i}^j) \bigotimes^n_{i=1}  a_{1}^i = 
		\sum_b \prod_{i \neq 1} \eta\Big( \psi'(b_i) \Big) \psi'(b_1) = 
		b \; \psi'
	}
	\DeriveConclude{[*]}{\bd^{-1}\TYPE{FiniteProduct}}{\LOGIC{This}}
	\EndProof
}
\newpage
\subsection{Cofreedom}
\Page{
	\DeclareType{CofreeCoalgebra}
	{
		\prod R \in \ANN \. 
		\prod M \in \LMOD{R} \.
		? \sum A \in \COALG{R} \. A  \Arrow{\LMOD{R}} M
	} 
	\DefineType{(A,\pi)}{CofreeCoalgebra}{\forall B : \COALG{R} \. \forall \varphi : B \Arrow{\LMOD{R}} M \. \exists! \psi : B \Arrow{\COALG{R}} A \. \psi \pi = \varphi } 
	\\
	\Theorem{CofreeCoalgebraSurjectivity}
	{
		\forall R \in \ANN \.
		\forall M \in \LMOD{R} \.
		\forall (A,\pi) : \TYPE{CofreeCoalgebra}(M) \.
		\pi : A \ToSurj M
	}
	\Assume{m}{M}
	\Say{\mu}{\Lambda t \in R \. tm }{R \Arrow{\LMOD{R}} M}
	\Say{\Big( \psi,[1]\Big)}{\bd \TYPE{CofreeCoalgebra}(A,\pi)(\nu)}{\sum \psi : R \Arrow{\COALG{R}} A \. \psi \pi = \mu }
	\Say{[2]}{[1]\ByConstr(\mu)}{ \psi \pi(e) = \mu(e) = m}
	\Conclude{[m.*]}{\bd \FUNC{image}[2]}{ m \in \im \pi  }
	\DeriveConclude{[*]}{I(\forall)\bd^{-1}\TYPE{Surjective}}{( \pi : A \ToSurj M )}
	\EndProof
	\\
	\Theorem{IsomorphicCofreeCoalgebra}
	{
		\forall R \in \ANN \.
		\forall M \in \LMOD{R} \.\NewLine \. 
		\forall (A,\pi),(B,\pi') : \TYPE{CofreeCoalgebra}(M) \.
		A \cong_{\COALG{R}} B
	}
	\NoProof
	\\
	\Theorem{DoubleDualCofreeCoalgebra}
	{
		\forall k : \Field \.
		\forall V \in \VS{k} \.
		\exists \TYPE{CofreeCoalgebra}(V^{**})
	}
	\Say{\pi}{\Lambda f \in V^{*\otimes\circ} \. f_{|V^{*\otimes}_1}}{V^{*\otimes\circ} \Arrow{\VS{k}} V^{**}} 
	\Assume{A}{\COALG{R}}
	\Say{\Big( \phi, [1] \Big)}{\THM{DualAdjucntion}(V,A)}{
		\sum \phi : (A \Arrow{\VS{k}} V^{**} ) \ToIso{\VS{k}} (V^* \Arrow{\VS{k}} A^{*} ) \. \NewLine \.
		\forall T : A \Arrow{\VS{k}}  V^** \.  
		\forall  f \in V^* \. 
		\forall a \in A \. 
		\phi(T)(f)(a)  =  T(a)(f)          
	}
	\Say{\phi'}{(\cdot)^\otimes}{ (V^* \Arrow{\VS{k}}  A^*) \ToIso{\VS{k}} (V^{*\otimes} \Arrow{\LALGE{k}} A^*) }
	\Say{\phi''}{\THM{DualAlgebraLeftAdjoint}}{ (V^{*\otimes} \Arrow{\LALGE{k}}  A^*) \ToIso{\SET}  (A \Arrow{\COALG{k}} V^{*\otimes\circ} )  }                       
	\Assume{\varphi}{A \Arrow{\VS{k}} V^{**}}
	\Say{\psi}{\phi\phi'\phi''(\varphi)}{A \Arrow{\COALG{k}} V^{*\otimes\circ}}
	\Assume{a}{A}
	\Conclude{[a.*]}{  \ByConstr \psi \ByConstr \phi \ByConstr \phi' \ByConstr \phi'' \bd \pi I(\to)(\varphi(a))  }
	{
		\NewLine
		\psi \pi(a) =  
		\phi\phi'\phi''(\varphi) \pi(a) =
		\pi\Big( \phi'\phi''\big(  \Lambda f \in V^* \. \Lambda a \in A \. \varphi(a)(f) \big)(a) \Big) = \NewLine = 
		\pi\left( \phi''\left(  \Lambda \sum^n_{i=1}  \bigotimes^{i}_{j=1} f_{i,j} \in V^{*\otimes} \. \Lambda a \in A \.  \sum^n_{i=1} \prod^i_{j=1} \varphi(a_{i,j})(f_{i,j}) \right)(a) \right) = \NewLine =
		\pi\left( \Lambda a \in A \. \Lambda \sum^n_{i=1} \bigotimes^{i}_{j=1} f_{i,j} \in V^{*\otimes} \. \sum^n_{i=1} \prod^i_{j=1} \varphi(a_{i,j})(f_{i,j}) (a) \right) = \NewLine 
		\pi\left( \Lambda \sum^n_{i=1} \bigotimes^{i}_{j=1} f_{i,j} \in V^{*\otimes} \. \sum^n_{i=1} \prod^i_{j=1} \varphi(a_{i,j})(f_{i,j})  \right) = 
		\Lambda  f \in V^* \. \varphi(a)(f) = 
		\varphi(a)
	}
}
\Page{
	\Derive{[1]}{E(=,\to)}{\psi\pi = \varphi}
	\Assume{\psi'}{A \Arrow{\COALG{k}} V^{*\otimes\circ}}
	\Assume{[2]}{\psi'\pi = \varphi}
	\Say{\Big(\varphi',[3]\Big)}{\bd \TYPE{Bijection}(\phi\phi'\phi'')}{\sum \varphi : A \Arrow{\VS{k}} V^{**} \. \phi_1\phi_2\phi_3(\varphi') = \psi'}
	\Say{[4]}{\ldots[3][2]}{ \varphi=\varphi' }
	\Conclude{[A.*]}{ E(=,\to)[4]\ByConstr \psi [3]}{\psi = \psi'} 
	\DeriveConclude{[*]}{I(\exists!)I^2(\forall)\bd^{-1}\TYPE{CofreeCoalgebra}}
	{
		\Big( \big( V^{*\otimes\circ},\pi \big) : \TYPE{CofreeCoalgebra}(V)  \Big)
	}
	\EndProof
	\\
	\Theorem{InheritingCofreeCoalgebra}{
		\forall k : \Field \. 
		\forall V \in \VS{k} \.
		\forall U \subvec{k} V \.
		\forall (A,\pi) : \TYPE{CofreeCoalgebra}(V) \. \NewLine \. 
		\exists \TYPE{CofreeCoalgebra}(U)
	}
	\Say{B}{\sum\{ E \subset_{\COALG{k}} A : \pi(E) \subset U   \}}{\COALG{k}} 
	\Assume{C}{\COALG{k}}
	\Assume{\varphi}{C \Arrow{\VS{k}} U}
	\Say{\Big(\psi,[1]\Big) }{\bd \TYPE{CofreeCoalgebra}(V)(A,\pi)(C,\varphi)}{ \sum \psi : C \Arrow{\COALG{k}} A \. \psi \pi = \varphi}
	\Assume{a}{\im \psi}
	\Say{Z}{\langle a \rangle_{\COALG{k}}}{\TYPE{Subcoalgebra}(A)}
	\Say{\Big(c,[2]\Big)}{ \bd \FUNC{image}(\psi)(a) }{\sum c \in C \. a = \psi(c)}
	\Assume{z}{Z}
	\Say{\Big(n,m,i,j,y,[3]\Big)}{\bd \FUNC{spawnedCoalgebra}(a)(z)}
	{\sum n,m \in \Nat \. \sum i \in n \. \sum j \in m \. \sum  y : m \to n \to A \. \NewLine \.  \Delta^n(a) = \sum^m_{i=1}\bigotimes^n_{j=1} y_{i,j} \And y_{i,j} = z }
	\Say{\Big( m',x, [4]  \Big)}{\bd \COALG{k}(C)(c)(n)}{\sum m' \in \Nat \sum x : m' \to n \to A \. \Delta^n(c) = \sum^{m'}_{i=1} \bigotimes^n_{i=1} x_{i,j}}
	\Say{[5]}{[4]\bd \COALG{k}(C,A)(\psi)[3]}{
		\sum^{m'}_{i=1} \bigotimes^n_{i=1} \psi(x_{i,j}) =
		\psi^{\otimes n}\Big( \Delta^n(c) \Big) =
		\Delta^n \psi(c) = 
		\Delta^n (a) =  
		\sum^m_{i=1} \bigotimes^n_{j=1} y_{i,j}
	}
	\Say{[6]}{[3][5]}{z \in \im \psi}
	\Conclude{[z.*]}{[1][6]}{\pi(z) \in U}
	\Derive{[3]}{I(\forall) \bd^{-1}\TYPE{Subset}}{\pi(Z) \subset U}
	\Conclude{[a.*]}{\ByConstr B \ByConstr Z  [3]}{a \in B}
	\Derive{ [2] }{I(\forall)\bd^{-1}\TYPE{Subset}}{\im \psi \subset B}
	\Say{[3]}{[2][1]}{ \psi^{|B} \pi_{|B} = \varphi }
	\Assume{\psi'}{ C  \Arrow{\COALG{k}} B }
	\Assume{[4]}{\psi' \pi_{|B} = \psi}
	\Conclude{[C.*]}{\bd \TYPE{CofreeCoalgebra}(V)(A,\pi)(\psi')\bd \LOGIC{Unique}}{\psi = \psi'}
	\DeriveConclude{[*]}{I(\exists!)I^2(\forall)\bd^{-1}\TYPE{CofreeCoalgebra}}
	{\Big( (B,\pi_{|B}) : \TYPE{FreeCoalgebra}(U) \Big) }
	\EndProof
}
\Page{
	\Theorem{CofreeCoalgebraExists}{\forall k : \Field \. \forall V : \VS{k} \. \exists \TYPE{CofreeCoalgebra}(V)}
	\Say{(A,\pi)}{\THM{DoubleDualCofreeCoalgebra}(V) }{\TYPE{CofreCoalgebra}(V^{**})}
	\Say{(A',\pi')}{\THM{InheretingCofreeCoalgebra}\Big(V^{**},\epsilon \; V,(A,\pi)\Big)}{\TYPE{CofreeCoalgebra}(\epsilon \; V)}
	\Conclude{[*]}{\bd \TYPE{Isomorphism}(\epsilon) \bd \TYPE{CofreeCoalgebra}(\epsilon \; V) (A',\pi')}
	{
		\Big( (A',\pi'\epsilon^{-1} ) : \TYPE{CofreeCoalgebra}(V) \Big)
	}
	\EndProof
	\\
	\DeclareFunc{cofreeCoalgebraFunctor}{\prod k : \Field \. \VS{k} \Arrow{\CAT} \COALG{k}}
	\DefineNamedFunc{cofreeCoalgebraFunctor}{V}{\mathrm{CF}(V)}{\THM{CofreeCoalgebraExists}(V)}
	\DefineNamedFunc{cofreeCoalgebraFunctor}{V,W,T}{\mathrm{CF}_{V,W}(T)}{ \bd \TYPE{CofreeCoalgebra}(\mathrm{CF}(W),\pi')(\pi T)  
		\NewLine
		\quad \where \quad (\mathrm{CF}(V),\pi) = \THM{CofreeCoalgebraExists}(V) 
		\NewLine
		\quad {\color{white} \mathtt{where}} \quad (\mathrm{CF}(W),\pi') = \THM{CofreeCoalgebraExists}(W)
	}
	\\
	\Theorem{CoalgebrasForgetfulFunctorAdjoint}
	{
		\forall k : \Field \. 
		(\mathrm{CF},U_{\COALG{k},\VS{k}}) : \TYPE{RightAdjoint}(\COALG{k},\VS{k})
	}
	\NoProof
	\\
	\DeclareType{CocommutativeCofreeCoalgebra}{\prod k : \Field \. \prod V : \VS{k} \. ?\sum A :\CCOALG{k} \. A  \Arrow{\VS{k}} V } 
	\DefineType{(A,\pi)}{CocommutiveCofreeCoalgebra}{\forall B : \COALG{R} \. \forall \varphi : B \Arrow{\LMOD{R}} M \. \NewLine \. \exists! \psi : B \Arrow{\COALG{R}} A \. \psi \pi = \varphi  }
	\\
	\Theorem{CofreeCoalgebraSurjectivity}
	{
		\forall R \in \ANN \.
		\forall M \in \LMOD{R} \. \NewLine \. 
		\forall (A,\pi) : \TYPE{CocommutativeCofreeCoalgebra}(M) \.
		\pi : A \ToSurj M
	}
	\Assume{m}{M}
	\Say{\mu}{\Lambda t \in R \. tm }{R \Arrow{\LMOD{R}} M}
	\Say{\Big( \psi,[1]\Big)}{\bd \TYPE{CocommutativeCofreeCoalgebra}(A,\pi)(\nu)}{\sum \psi : R \Arrow{\CCOALG{R}} A \. \psi \pi = \mu }
	\Say{[2]}{[1]\ByConstr(\mu)}{ \psi \pi(e) = \mu(e) = m}
	\Conclude{[m.*]}{\bd \FUNC{image}[2]}{ m \in \im \pi  }
	\DeriveConclude{[*]}{I(\forall)\bd^{-1}\TYPE{Surjective}}{( \pi : A \ToSurj M )}
	\EndProof
	\\
	\Theorem{IsomorphicCofreeCoalgebra}
	{
		\forall R \in \ANN \.
		\forall M \in \LMOD{R} \.\NewLine \. 
		\forall (A,\pi),(B,\pi') : \TYPE{CocommutativeCofreeCoalgebra}(M) \.
		A \cong_{\COALG{R}} B
	}
	\NoProof
}
\Page{
	\Theorem{CocommutativeCofreeCoalgebraExists}
	{
		\forall k : \Field \.
		\forall V : \VS{k} \.
		\exists \TYPE{CocommutativeCofreeCoalgebra}(M) \.
	}
	\Say{(A,\pi)}{\THM{CofreeCoalgebraExists}(V)}{\TYPE{CofreeCoalgebra}(V)}
	\Say{B}{\sum \{ E \subset A : (E,\delta,\eta) \in \CCOALG{k} \}}{\CCOALG{k}}
	\Assume{C}{\CCOALG{k}}
	\Assume{\varphi}{C \Arrow{\VS{k}} V}
	\Say{\Big(\psi,[1]\Big)}{\bd \TYPE{CofreeCoalgebra}(V)(A,\pi)}{ \sum \psi : C \Arrow{\COALG{k}} A \. \psi \pi = \varphi}
	\Assume{a}{\im \psi}
	\Say{\Big(c,[2]\Big)}{\bd \FUNC{image}\bd a}{\sum c \in C \. \psi(c) = a}
	\Say{[2]}{\bd \COALG{k}(C,A)(\psi)[2] }{\langle a \rangle_{\COALG{z}} \in \Im \psi}
	\Say{[3]}{ [2] \bd \COALG{k}(C,A)(\psi) \bd \FUNC{swap} \bd \TYPE{Cocommutative}(C) \bd \COALG{k}(C,A) [2]  }
	{ 
		\NewLine : 
		a \; \Delta_A \; \FUNC{swap} \;   =  
		c \; \psi \; \Delta_A  \;\FUNC{swap}   =
		c \; \Delta_C \; (\psi \otimes \psi) \; \FUNC{swap}  = 
		c \; \Delta_C \; \FUNC{swap} \; (\psi \otimes \psi)  =
		c \; \Delta_C \; (\psi \otimes \psi)  = \NewLine = 
		c \; \psi \; \Delta_A =
		a \; \Delta_A
	}
	\Conclude{[a.*]}{\ByConstr B[2][3]}{a \in B}
	\Derive{[2]}{\bd \TYPE{Subset}}{\im \psi \subset B}
	\Say{[3]}{[1][2]}{  \psi^{|B}\pi_{|B} = \varphi }
	\Assume{\psi'}{ C  \Arrow{\CCOALG{k}} B }
	\Assume{[4]}{\psi' \pi_{|B} = \varphi}
	\Conclude{[C.*]}{\bd \TYPE{CofreeCoalgebra}(V)(A,\pi)(\psi')\bd \LOGIC{Unique}}{\psi = \psi'}
	\DeriveConclude{[*]}{I(\exists!)I^2(\forall)\bd^{-1}\TYPE{CocommutativeCofreeCoalgebra}}
	{\Big( (B,\pi_{|B}) : \TYPE{CocommutativeFreeCoalgebra}(U) \Big) }
	\EndProof
	\\
	\DeclareFunc{CocommutativeCoalgebraFunctor}{\prod k : \Field \. \VS{k} \Arrow{\CAT} \CCOALG{k}}
	\DefineNamedFunc{cofreeCoalgebraFunctor}{V}{\mathrm{CCF}(V)}{\THM{CocommutativeCofreeCoalgebraExists}(V)}
	\DefineNamedFunc{cofreeCoalgebraFunctor}{V,W,T}{\mathrm{CCF}_{V,W}(T)}{ \bd \TYPE{CocommutativeCofreeCoalgebra}(\mathrm{CF}(W),\pi')(\pi T)  
		\NewLine
		\quad \where \quad (\mathrm{CCF}(V),\pi) = \THM{CocommutativeCofreeCoalgebraExists}(V) 
		\NewLine
		\quad {\color{white} \mathtt{where}} \quad (\mathrm{CCF}(W),\pi') = \THM{CocommutativeCofreeCoalgebraExists}(W)
	}
	\\
	\Theorem{CommutativeCoalgebrasForgetfulFunctorAdjoint}
	{
		\NewLine
		::
		\forall k : \Field \. 
		(\mathrm{CCF},U_{\CCOALG{k},\VS{k}}) : \TYPE{RightAdjoint}(\COALG{k},\VS{k})
	}
	\NoProof	
}
\Page{
	\Theorem{CocommutativeCofreeCoalgebraOfSum}
	{
		\forall k : \Field \. 
		\forall V,V' \in \VS{l} \. 
		\NewLine \. 
		\Big(A \otimes A' , \pi\Big) : \TYPE{CocmmutativeCofreeCoalgebra}(V \oplus V') \NewLine 
		\quad \where \quad (A,\nu) = \THM{CocommutativeCofreeCoalgebraExists}(V) 
		\NewLine
		\quad {\color{white} \mathtt{where}} \quad (A',\nu') = \THM{CocommutativeCofreeCoalgebraExists}(V')
		\NewLine
		\quad {\color{white} \mathtt{where}} \quad \pi  = \bd \TYPE{TensorProduct}(A,A') 
		\Lambda a \in A \. a' \in A' \. \Big(\eta'(a')\nu(a), \eta(a)\nu'(a')\Big)
	}
	\Assume{B}{\CCOALG{k}}
	\Assume{\varphi}{B \Arrow{\VS{k}} V \oplus V'}
	\Say{(\psi, [1])}{\bd \TYPE{CocommutativeCofreeCoalgebra}(V)(A,\nu)(\varphi \pi_1)}
	{\sum \psi : B \Arrow{\CCOALG{k}} A \. \psi \nu = \varphi \pi_1 } 
	\Say{(\psi', [2])}{\bd \TYPE{CocommutativeCofreeCoalgebra}(V')(A',\nu')(\varphi \pi_2)}
	{\sum \psi' : B \Arrow{\CCOALG{k}} A' \. \psi' \nu' = \varphi \pi_2} 
	\Assume{b}{B}
	\Conclude{[b.*]}{ \THM{SweedlerNotation}(b) \ByConstr \pi \bd \COALG{k}(B,A)(\psi) \bd \COALG{k}(B,A')(\psi')[1][2]
		\bd\VS{k}(\varphi)(B,V) \NewLine  \bd \COALG{k}(B) \bd \FUNC{directSum}(V,V')    }
	{
		b \; \Delta \; (\psi \otimes \psi') \; \pi  =
		\left( \sum_b  b_1 \otimes b_2 \right) \; (\psi \otimes \psi')  \; \pi = \NewLine = 
		\left( \sum_b \eta'\big(\psi'(b_2)\big) \nu\big( \psi(b_1)\big), \sum_b \eta\big(\psi(b_1)\big) \nu'\big( \psi'(b_2)\big)\right) = 
		\left( \sum_b \eta(b_2) \varphi(b_1), \sum_b \eta (b_1) \varphi(b_2)\right) = \NewLine = 
		\left(  \pi_1 \varphi \left( \sum_b \eta(b_2) b_1)\right), \pi_ 2 \varphi \left( \sum_b \eta(b_1) b_2 \right)\right) = 
		\Big(  \pi_1 \; \varphi(b) , \pi_2 \; \varphi(b) \; \Big) =
		\eta(b) \varphi(b)                                                               
	}
	\Derive{[1]}{I(=,\to)}{ \Delta (\psi \otimes \psi') \; \pi = \id  }
	\Assume{\widehat{\psi}}{ B  \Arrow{\CCOALG{k}} A \otimes A' }
	\Assume{[2]}{\widehat{\psi} \pi = \varphi}
	\Say{\overline{\psi}}{ \widehat{\psi} (\id \otimes \eta')  }{B \Arrow{\CCOALG{k}} A }
	\Say{\overline{\psi}'}{ \widehat{\psi} (\eta \otimes \id )  }{B \Arrow{\CCOALG{k}} A' }
	\Say{[3]}{ \ByConstr \overline{\psi} \bd \VS{k}(B,A)(\nu)\ByConstr^{-1} \pi    }
	{
		 \overline{\psi} \; \nu =
		 \widehat{\psi}  \; (\id \otimes \eta') \; \nu =
		 \widehat{\psi}  \; (\nu \otimes \eta')  = 
		 \widehat{\psi} \pi \pi_1  
	}
	\Say{[4]}{\bd \TYPE{CocommutativeFreeCoalgebra}(V)(A,\nu))[3]}{ \overline{\psi} = \psi } 
	\Say{[5]}{ \ByConstr \overline{\psi}' \bd \VS{k}(B,A')(\nu')\ByConstr^{-1} \pi    }
	{
		 \overline{\psi}' \; \nu' =
		 \widehat{\psi}  \; (\eta \otimes \id ) \; \nu' =
		 \widehat{\psi}  \; (\eta \otimes \nu')  = 
		 \widehat{\psi} \pi \pi_2  
	}
	\Say{[6]}{\bd \TYPE{CocommutativeFreeCoalgebra}(V)(A',\nu')[5]}{ \overline{\psi}' = \psi' } 
	\Assume{b}{B}
	\Conclude{[B.*]}{ [5][6]\ByConstr \widehat{\psi} \ByConstr \widehat{\psi}' \bd \FUNC{sweedlerNotation} \ \bd \FUNC{tensorFunction} \bd 
		\L(A,A';A \otimes A')(\FUNC{tensorproduct}) \NewLine \bd \CCOALG{k}(B,A \otimes A')(\widehat{\psi})  \bd \FUNC{tensorProductCoalgebra} 
		\bd^{-1} \widehat{\psi} \bd \CCOALG{k}(B,A\otimes A')(\psi) \NewLine \bd \COALG{k}(B)
	}
	{   
		b \;\Delta \; (\psi \otimes \psi') =  
		b \; \Delta \; (\overline{\psi} \otimes \overline{\psi}') =  
		\sum_{b} \; \Big( b_1 \; \widehat{\psi} \; (\id \otimes \eta') \otimes  b_2 \; \widehat{\psi} \; (\eta \otimes \id) \Big) = \NewLine =  
		\sum_{b} \; \left( \sum_{ a_1 = \widehat{\psi}(b_1) }  \eta'(a_1^2) a_1^1 \right) \otimes \left( \sum_{a_2 = \widehat{\psi}(b_2)} \eta(a_2^1)a_2^2 \right) = \ 
		\sum_{b} \sum_{a_1=\widehat{\psi}(b_1)} \sum_{a_2=\widehat{\psi}(b_2)}  \eta(a_2^1)\eta'(a_1^2) a_1^1 \otimes a_2^2 =   \NewLine =
		\sum_{b} \sum_{a_1=\widehat{\psi}(b_1)} \sum_{a_2=\widehat{\psi}(b_2)}  \eta(a_2^1)\eta'(a_2^2) a_1^1 \otimes a_1^2 =
		\sum_{b} \eta(\widehat{\psi}(b_2))\widehat{\psi}(b_1) = 
		\sum_{b} \eta(b_2)\widehat{\psi}(b_1) =
		\widehat{\psi}(b)
	}
	\DeriveConclude{[*]}{I(\exists!)I^2(\forall)\bd^{-1}\TYPE{CocommutativeCofreeCoalgebra}}
	{ \NewLine : \Big( (A \otimes A',\pi) : \TYPE{CocommutativeFreeCoalgebra}(U) \Big)  }
	\EndProof
}
\newpage
\subsection{Comodules}
\Page{
	\DeclareType{LeftAlgebraModule}{\prod R \in \ANN \. \prod A \in \LALGE{R} \. \sum M : \LMOD{R} \. A \otimes M \Arrow{\LMOD{R}} M } 
	\DefineType{(M,\mu)}{LeftAlgebraModule}{   (\id \otimes \mu)\mu = (\mu_A \otimes \id)\mu  \And (e_A \otimes \id)\mu = \cdot}
	\\
	\DeclareType{RightAlgebraModule}{\prod R \in \ANN \. \prod A \in \LALGE{R} \. \sum M : \LMOD{R} \. M \otimes A \Arrow{\LMOD{R}} M } 
	\DefineType{(M,\mu)}{RightAlgebraModule}{   (\mu \otimes \id)\mu = (\id \otimes \mu_A)\mu \And (\id \otimes e_A)\mu = \cdot }
	\\
	\DeclareType{LeftComodule}{\prod R \in \ANN \. \prod A \in \COALG{R} \. \sum M : \LMOD{R} \. M \Arrow{\LMOD{R}} A \otimes M } 
	\DefineType{(M,\rho)}{LeftComodule}{   \rho(\id \otimes \rho) = \rho(\Delta \otimes \id) \And  \rho ( \eta_A \otimes \id) = \id   }
	\\
	\DeclareType{RightAlgebraComodule}{\prod R \in \ANN \. \prod A \in \COALG{R} \. \sum M : \LMOD{R} \. M \Arrow{\LMOD{R}} M \otimes A } 
	\DefineType{(M,\rho)}{RightComodule}{ \rho (\rho \otimes \id) = \rho (\id \otimes \Delta) \And \rho (\id \otimes \eta_A) = \id}
	\\
	\DeclareType{LeftAlgebraModuleMorphism}{\prod R \in \ANN \. \prod A \in \LALGE{R} \. \NewLine \. \prod X,Y : \TYPE{LeftAlgebraModule}(A) \. X \Arrow{\LMOD{R}} Y } 
	\DefineType{\varphi}{LeftAlgebraModuleMorphism}{   (\id \otimes \varphi)\mu_Y = \mu_X \varphi }
	\\
	\DeclareType{RightAlgebraModuleMorphism}{\prod R \in \ANN \. \prod A \in \LALGE{R} \. \NewLine \. \prod X,Y : \TYPE{RightAlgebraModule}(A) \. X \Arrow{\LMOD{R}} Y } 
	\DefineType{\varphi}{RightAlgebraModuleMorphism}{   (\varphi \otimes \id)\mu_Y = \mu_X \varphi }
	\\
	\DeclareType{LeftComoduleMorphism}{\prod R \in \ANN \. \prod A \in \COALG{R} \. \prod X,Y : \TYPE{LeftComodule}(A) \. X \Arrow{\LMOD{R}} Y } 
	\DefineType{\varphi}{LeftComoduleMorphism}{  \rho_X(\id \otimes \varphi) =  \varphi \rho_Y }
	\\
	\DeclareType{RightComoduleMorphism}{\prod R \in \ANN \. \prod A \in \COALG{R} \. \prod X,Y : \TYPE{RightComodule}(A) \. X \Arrow{\LMOD{R}} Y } 
	\DefineType{\varphi}{RightComoduleMorphism}{  \rho_X (\varphi \otimes \id) =  \varphi \rho_Y }
	\\
	\DeclareFunc{leftAlgebraModuleCategory}{\prod R \in \ANN \. \LALGE{R} \to \CAT}
	\DefineNamedFunc{leftAlgebraModuleCategory}{A}{\LAMOD{A}}{\NewLine \de \Big( \TYPE{LeftAlgebraModule}(A),\TYPE{LeftAlgebraiModuleMorphism}(A), \circ, \id  \Big) }
	\\
	\DeclareFunc{rightAlgebraModuleCategory}{\prod R \in \ANN \. \LALGE{R} \to \CAT}
	\DefineNamedFunc{rightAlgebraModuleCategory}{A}{\RAMOD{A}}{
		\NewLine \de \Big( \TYPE{RightAlgebraModule}(A),\TYPE{RightAlgebraModuleMorphism}(A), \circ, \id  \Big) }
	\\
	\DeclareFunc{leftComoduleCategory}{\prod R \in \ANN \. \COALG{R} \to \CAT}
	\DefineNamedFunc{leftComoduleCategory}{A}{\LCOMOD{A}}{\Big( \TYPE{LeftAlgebraComodule}(A),\TYPE{LeftComoduleMorphism}(A), \circ, \id  \Big) }
}\Page{
	\DeclareFunc{rightComoduleCategory}{\prod R \in \ANN \. \COALG{R} \to \CAT}
	\DefineNamedFunc{rightComoduleCategory}{A}{\RCOMOD{A}}{\Big( \TYPE{RightComodule}(A),\TYPE{RightComoduleMorphism}(A), \circ, \id  \Big) }
	\\
	\Theorem{CoalgebraAsComodule}{\forall R \in \ANN \. \forall A \in \COALG{R} \. (A,\Delta) \in \RCOMOD{A}}
	\NoProof
	\\
	\Theorem{ConstructedComoduleStructure}{\forall R \in \ANN \. \forall A \in \COALG{R} \. \forall M \in \LMOD{R} \. (M \otimes A,\id \otimes \Delta) \in \RCOMOD{A}}
	\NoProof
	\\
	\DeclareFunc{setComodule}{\prod R \in \ANN  \. \prod X : \SET \. (X \to \LMOD{R}) \to \RCOMOD{\mathrm{F}(X)}   } 
	\DefineFunc{setComodule}{M}{ \left( \bigoplus_{x \in X} M_x, \bd \TYPE{DirectSum}(X,M) \. \Lambda x \in X \. \Lambda m \in M \. m \otimes e_x \right) }                           
	\\
	\Theorem{FundamentalTheoremOfComodules}{ 
		\forall R \in \ANN \. 
		\forall A \in \COALG{R} \. 
		\forall M \in \RCOMOD{A} \.
		\forall m \in M \. \NewLine \.  
		\exists N \subset_{\RCOMOD{A}} M :
		m \in N \And \dim N < \infty
	}
	\NoProof
	\\
}
\newpage
\subsection{Rationality}
\subsection{Bicomodules}
\subsection{Cotensor Products}
\subsection{Simplicity and Injectivity}
\subsection{Torsion Theories}
\subsection{Cosemisimplicity}
\subsection{Semiperfectnes}
\subsection{Duals of Frobeneus Theories}
\newpage
\section{Theory of Hopf Algebras}
\subsection{Bialgebras}
\Page{
	\DeclareType{Bialgebra}{\prod R \in \ANN \. \prod A \in \LMOD{R} \.  
		\NewLine \.
		 (A \otimes A \Arrow{\LMOD{R}} A) \times 
		 (R \Arrow{\LMOD{R}}) \times 
		 (A \Arrow{\LMOD{R}} A \otimes A) \times 
		 (A \Arrow{\LMOD{R}} R)
	}
	\DefineType{(A,\mu, e,\Delta,\eta)}{Bialgebra}{
		(A,\Delta,\eta) \in \COALG{R} \And
		(A,\mu,e) \in \LALGE{R} \And \NewLine \And
		\mu : A \otimes A \Arrow{\COALG{R}} A \And
		e  : R \Arrow{\COALG{R}} A \And
		\Delta : A \Arrow{\LALGE{R}} A \otimes A \And
		\eta : A \Arrow{\LALGE{R}} R
	}
	\\
	\DeclareType{BialgebraMorphism}
	{
		\prod R \in \ANN \. 
		\prod A,B : \TYPE{Bialgebra}(R) \.
		A \Arrow{\LALGE{R}} B
	}
	\DefineType{f}{BialgebraMorphism}
	{f : A \Arrow{\COALG{R}} B}
	\\
	\DeclareFunc{bialgebraCategory}{ \ANN \to \Cat }
	\DefineNamedFunc{bialgebraCategory}{R}{\BIALG{R}}
	{\Big(\TYPE{Bialgebra}, \TYPE{BialgebraMorphism}, \id, \circ \Big) }
	\\
	\DeclareType{Primitive}{\prod R \in \ANN \. \prod A \in \BIALG{R} \.?A }
	\DefineType{a}{Primitive}{\Delta(a) = a \otimes e + e \otimes a}
	\\
	\DeclareFunc{monoidBialgebra}{\prod R \in \ANN \. \TYPE{Monoid} \to \BIALG{R}}
	\DefineNamedFunc{monoidBialgebra}{M}{RM}{ 
			\Big( 
				R^{\oplus M}, 
				\bd \FUNC{directPower}\Lambda a,b \in M \. ab,
				\Lambda \alpha \in R \. \alpha e_M , \NewLine,  
				\bd \FUNC{directPower} \Lambda a \in M \. a \otimes a,  
				\bd \FUNC{directPower} \Lambda a \in M \. e_A
			\Big)
		}
	\\
	\DeclareFunc{grouplike}{\prod R \in \ANN \. \BIALG{R}}
	\DefineFunc{grouplike}{}
	{
		\Big(
			R[x],
			\bd R[x] (x \otimes x),
			\bd R[x] (1)
		\Big)
	}
	\\
	\DeclareFunc{primitive}{\prod R \in \ANN \. \BIALG{R}}
	\DefineFunc{primitive}{}
	{
		\Big(
			R[x],
			\bd R[x] (x \otimes 1 + 1 \otimes x),
			\bd R[x] (0)
		\Big)
	}
	\\
	\DeclareType{Biideal}{
		\prod R \in \ANN \. 
		\prod A \in \BIALG{R} \. 
		?A
	}
	\DefineType{I}{Biideal}{I : \TYPE{Ideal} \And \TYPE{Coideal}(A)}
	\\
	\DeclareFunc{BialgebraQuotient}{
		\prod R \in \ANN \.
		\prod A \in \BIALG{R} \.
		\TYPE{Biideal}(A) \to \BIALG{R}
	} 
	\DefineNamedFunc{BialgebraQuotient}{I}{\frac{A}{I}}{\frac{A}{I}}
}
\Page{
	\Theorem{PolynomialBialgebraClassification}
	{
		\forall R \in \ANN \. 
		\forall \Delta : R[x] \Arrow{\LALGE{R}} R[x] \otimes R[x] \.
		\forall \eta : R[x] \Arrow{\LALGE{R}} R \.\NewLine \.
		\Big( R[x],\Delta,\eta \Big) : \BIALG{R} \Rightarrow
		\Big(R[x],\Delta,\eta\Big) \cong \FUNC{grouplike}(R) \bigg|
		\Big(R[x],\Delta,\eta\Big) \cong \FUNC{primitive}(R)
	}
	\Say{A}{\Big(R[x],\Delta,\eta\Big)}{ \BIALG{R}  }
	\Say{[1]}{\bd \BIALG{k}(A)}{ \Delta(1) = 1 \otimes 1 \And \eta(1) = e_k }
	\Say{\Big( n,m, \beta [2] \Big)}
	{
		\THM{TensorProductBasis}(A,A)(\Delta(x))
	}
	{
		\sum n,m \in \Int_+ \.
		\beta : n \to m \to R \.
		\Delta(x) = \beta_{i,j} x^i \otimes x^j
	}
	\Say{[3]}{\bd A\bd \COALG{A}[2]}{ \{ (i,j) \in n \times m \. \beta_{imj} \neq 0  \} \neq \emptyset }
	\Say{ I}{ \max \{ i \in n : \exists j \in m : \beta_{i,j} \neq 0\}   }{n}
	\Say{ J}{ \max \{ j \in n :  \beta_{I,j} \neq 0\}   }{m}
	\Say{[4]}{\bd \FUNC{multideg}\bd \BIALG{R}(A)\bd \COALG{R}(A)\bd \FUNC{multideg}(\ldots)}
	{
		\NewLine
		(I^2,IJ,J) =
		\mathrm{multideg} \; (\Delta \otimes \id)\Delta(x) =
		\mathrm{multideg} \;(\id \otimes \Delta)\Delta(x) = (I, IJ, J^2)  }
	\Say{[5]}{\THM{IdempotentIntegers}[4]}{ (I,J) \in \{0,1\}^2  }
	\Say{ J'}{ \max \{ j \in m : \exists i \in n : \beta_{i,j} \neq 0\}   }{m}
	\Say{ I'}{ \max \{ j \in n :  \beta_{I,j} \neq 0\}   }{n}
	\Say{[6]}{\bd \FUNC{multideg}\bd \BIALG{R}(A)\bd \COALG{R}(A)\bd \FUNC{multideg}(\ldots)}
	{
		\NewLine
		({I'}^2,I'J',J') =
		\mathrm{multideg} \; (\Delta \otimes \id)\Delta(x) =
		\mathrm{multideg} \;(\id \otimes \Delta)\Delta(x) = (I', I'J', {J'}^2)  }
	\Say{[7]}{\THM{IdempotentIntegers}[4]}{ (I',J') \in \{0,1\}^2  }
	\Assume{[8]}{(I = 0}
	\Say{[9]}{[8][2]}{ \Delta(x) = \beta_{0,0} 1 \otimes 1 + \beta_{0,1} 1 \otimes x  }
	\Say{[10]}{\bd \COALG{R}(A)[9][1]}{ x = (\id \otimes \eta) \circ \Delta(x) = \beta_{0,0} + \beta_{0,1}\eta(x) }
	\Conclude{[8.*]}{\bd R[x]}{\bot}
	\Derive{[8]}{E(\bot)}{ I \neq 0}
	\Assume{[9]}{(I,J) = (1,0)}
	\Say{[10]}{[9][2][7]}{ \Delta(x) = \beta_{1,0} x \otimes 1 + \beta_{0,0} 1 \otimes 1 + \beta_{0,1} 1 \otimes x}
	\Say{[11]}{\bd \COALG{R}(A)[10][1]}{ 
		x = 
		(\eta \otimes \id)\circ \Delta(x) = 
		\beta_{1,0} \eta{x} + \beta_{0,0} + \beta_{0,1}x 
	}
	\Say{[12]}{\bd \COALG{R}(A)[10][1]}{ 
		x = 
		(\id \otimes \eta)\circ \Delta(x) = 
		\beta_{0,1} \eta{x} + \beta_{0,0} + \beta_{1,0}x 
	}
	\Say{[13]}{\bd R[x][10][11]}{\beta_{1,0} = \beta_{0,1} = 0, \beta_{0,0} = -\eta(x)}
	\Say{\varphi}{\bd R[x] (x - \beta_{0,0})}{ A \ToIso{\BIALG{R}} A }
	\Conclude{[9.*]}{[13](A)}{ A \cong_{\BIALG{R}} \FUNC{primitive}(R) }
	\Derive{[9]}{I(\Imply)}{(I,J) = (1,0) \Imply A \cong_{\BIALG{R}} \FUNC{primitive}(R)}
	\Assume{[10]}{(I,J) = (1,1)}
	\Say{[11]}{[10][2][7]}{ 
		\Delta(x) = 
		\beta_{0,0} 1 \otimes 1 + \beta_{1,0} x \otimes 1 + \beta_{0,1} 1 \otimes x + \beta_{1,1} x \otimes x
	}
	\Say{[12]}{\bd \COALG{R}(A)[11]}{   
		x = 
		(\id \otimes \eta)\Delta(x)  =
		(\beta_{0,0} + \beta_{0,1} \eta(x)) + ( \beta_{1,0} + \beta_{1,1}\eta(x))x
	}
	\Say{[13]}{\bd \COALG{R}(A)[11]}{   
		x = 
		(\eta \otimes \id)\Delta(x)  =
		(\beta_{0,0} + \beta_{1,0} \eta(x)) + ( \beta_{0,1} + \beta_{1,1}\eta(x))x
	}
	\Say{y}{\bd R[x] (x - \beta_{0,0})}{ A \ToIso{\BIALG{R}} R[y] }
	\Say{[14]}{\bd y [12][13]}{ \Delta(y) = 1 \otimes y + y \otimes 1 + \beta_{1,1} y \otimes y }
	\Say{[15]}{\ByConstr (I,J)[10]}{\beta_{1,1}\neq 1}
	\Say{z}{\beta_{1,1}y + 1}{R[y] \ToIso{\BIALG{R}} R[z] }
	\Say{[16]}{\ByConstr z \bd \L\Big(R[y],R[y]; R[y] \otimes R[y]\Big)\FUNC{tensorProduct} \ByConstr z}{
		\Delta(z) = 
		\Delta(\beta_{1,1}y + 1) = 
		1 \otimes 1 + \beta_{1,1} (1 \otimes y)  + \beta_{1,1} (y \otimes 1) + \beta^2_{1,1} (y \otimes y) =
		1 \otimes 1  +   \beta_{1,1} (1 \otimes y)  + \beta_{1,1}(z \otimes 1) + 
		( \beta_{1,1} y +1) \otimes (\beta_{1,1} y + 1)  - 
		(\beta_1 y + 1)  \otimes 1 -  1 \otimes (\beta_1 y) =  z \otimes z	
	}
	\Conclude{[10.*]}{\ByConstr z\ByConstr y\bd^{-1}\FUNC{grouplike}[16]}{ A \cong_{\BIALG{R}} \FUNC{grouplike}(A) }
	\Derive{[10]}{I(\Imply)}{(I,J)=(1,1) \Imply A \cong_{\BIALG{R}} \FUNC{grouplike}(A) }
	\Conclude{[*]}{[5][8][9][10]}{A \cong_{\BIALG{R}} \FUNC{primitive}(A) \bigg| A \cong_{\BIALG{R}} \FUNC{grouplike}(A)}
	\EndProof
}
\Page{
	\DeclareType{BialgebraModuleAlgebra}{\prod R \in \ANN \. \prod B : \BIALG{R} \. ?\LAMOD{B} \And \LALGE{R}}
	\DefineType{A}{BialgebraModuleAlgebra}{ 
		(\id \otimes \mu_A) \mu_{B,A} = (\Delta_B \otimes {\id}_A \otimes {\id}_A)\mu_{B,A}^{\otimes 2} \mu_A
		\And 
		(\id \otimes e_A) \mu_{B,A} =   \eta_B  e_A 
	}
	\\
	\DeclareFunc{categoryOfBialgebraModuleAlgebras}
	{
		\prod R \in \ANN \. \BIALG{R} \to \CAT
	}
	\DefineNamedFunc{categoryOfBialgebraModuleAlgebras}
	{B}{\LBALG{B}}{ \NewLine \de \Big( \TYPE{BialgebraModuleAlgebra}, \LAMOD{B} \And \LALGE{R},  \id, \circ\Big) }
	\\
	\DeclareType{BialgebraModuleCoalgebra}{\prod R \in \ANN \. \prod B : \BIALG{R} \. ?\RAMOD{B} \And \COALG{R}}
	\DefineType{A}{BialgebraModuleCoalgebra}{ 
		\mu_{A,B} \Delta_A =  (\Delta_A \otimes \Delta_B)\mu_{A,B}^{\otimes 2}  \And
		\mu_{A,B} \eta_A =  (\eta_A \otimes \eta_B) \mu_R
	}
	\\
	\DeclareFunc{categoryOfBialgebraModuleCoalgebras}
	{
		\prod R \in \ANN \. \BIALG{R} \to \CAT
	}
	\DefineNamedFunc{categoryOfBialgebraModuleCoalgebra}
	{B}{\RBCOALG{B}}{ \NewLine \de \Big( \TYPE{BialgebraModuleAlgebra}, \LAMOD{B} \And \LALGE{R},  \id, \circ\Big) }
	\\
	\DeclareFunc{rightBimoduleDualAlgebra}{\prod R \in \ANN \. \prod B \in \BIALG{R} \. \LBALG{B} }
	\DefineNamedFunc{rightBimoduleDualAlgebra}{}{B^*}{(B^*, \Lambda b \in B \. \Lambda f \in B^* \. a \hit f)}
	\Assume{f,g}{B^\circ}
	\Assume{b,x}{B}
	\Say{\Big[(f,g).*.1\Big]}{ 
		\bd B^* 
		\bd \FUNC{hitAction} 
		\bd \FUNC{dualAlgebra}
		\bd \BIALG{R}(B)  
		\bd^{-1} \FUNC{hitAction}
		\bd^{-1} B^*
	}
	{
		\NewLine
		\Big(b(fg)\Big)(x) =
		(b \hit fg)(x) =
		fg(bx) =
		\sum_{y=bx} f(y_1)g(y_2) =
		\sum_{b} \sum_{x} f(b_1x_1)g(b_2x_2) = \NewLine =  
		\sum_{b} \sum_{x} (b_1 \hit f)(x_1)(b_2 \hit g)(x_2)
		\sum_{b} (b_1 \hit f)(b_2 \hit g)(x)
		\sum_{b} (b_1 f)(b_2 f)(x)
	}
	\Conclude{\Big[(f,g).*.1 \Big]}
	{ \bd B^* \bd \FUNC{hitAction}  \bd \LALGE{R}(B,R)(\eta)\bd^{-1} B^*   }
	{
		\NewLine
		be_{B^*}(x) =
		(b \hit \eta)(x) =
		\eta(bx) =
		\eta(b)\eta(x) =
		\eta(b) e_{B^*}(x)
	}
	\DeriveConclude{[*]}{\bd \LBALG{B}}{B^* \in \LBALG{B}}
	\EndProof
}\Page{
	\Theorem{FiniteDualBialgebra}{\forall R : \Field \. \forall B \in \BIALG{R} \. B^\circ \in \BIALG{R}}
	\Assume{f,g}{B^\circ}
	\Assume{b,x}{B}
	\Conclude{[(b,x).*]}{ \bd \FUNC{hitBy} \bd \FUNC{dualAlgebra} \bd \BIALG{R} \bd^{-1} \FUNC{hitBy} \bd^{-1}\FUNC{dualAlgebra} }{ 
		\NewLine :
		(fg \hitBy b)(x) =
		fg(xb) = 
		\sum_{y = xb} f(y_1)g(y_2) =
		\sum_{x} \sum_a f(x_1b_1)g(x_1b_2) = \NewLine = 
		\sum_{x} \sum_a  (f \hitBy b_1)(x_1)(f \hitBy b_2)(x_2) =
		\sum_a    (f \hitBy b_1)(f \hitBy b_2)(x) 
	}
	\Derive{[1]}{\bd^{-1}\TYPE{Subset}}{ (fg \hitBy B) \subset (f \hitBy B)(g \hitBy B)}
	\Say{[2]}{\THM{FiniteHitByAction}(f)}{ \dim (f \hitBy B) < \infty }
	\Say{[3]}{\THM{FiniteHitByAction}(g)}{ \dim (g \hitBy B) < \infty }
	\Say{[4]}{\THM{SubsetDimension}[1]\THM{ProductDimension}[2][3]}{ \dim (fg \hitBy B)< \dim (f \hitBy B)(g \hitBy B) M \infty}
	\Conclude{[(f,g).*]}{\THM{FiniteHitByAction}[4]}{fg \in B^\circ}
	\Derive{[1]}{I(\forall)}{\forall f,g \in B^\circ \. fg \in B^\circ}
	\Say{[2]}{\bd \FUNC{kernel} \bd \BIALG{R}\bd^{-1}\Ideal}{\Big( \ker e_{B^*} : \Ideal(B) \Big) }
	\Say{[3]}{\bd B^\circ[2]}{e_{B^*} \in B^\circ }
	\Assume{f,g}{B^*}
	\Assume{x,y}{B}
	\Assume{\alpha}{R}
	\Say{\Big[(f,g).*.1\Big]}{ \bd \FUNC{finiteDualCoalgebra}\bd \FUNC{dualAlgebra} \bd \BIALG{R}(B) 
		\NewLine \bd^{-1} \TYPE{finiteDualCoalgebr}  
		\bd^{-1} \FUNC{dualAlgebra}\bd^{-1} \FUNC{tensorProductAlgebra}}{ 
		\Delta(fg)(x \otimes y) =
		fg(xy) = \NewLine =  
		\sum_{z = xy} f(z_1)g(z_2) =
		\sum_x \sum_y f(x_1y_1)g(x_2y_2) = 
		\sum_x \sum_y \Delta(f)(x_1 \otimes y_1)\Delta(g)(x_2 \otimes y_2) =
		\Delta(f)\Delta(g)(x \otimes y)
	}
	\Say{\Big[(f,g).*.2\Big]}{ \bd \FUNC{finiteDualCoalgebra}\bd \FUNC{dualAlgebra} \bd \BIALG{R}(B) \NewLine
		\bd^{-1} \FUNC{finiteDualCoalgebra} \bd \FUNC{tensorProductCoalgebra}  }
	{ 
		\eta(fg) = 
		fg(e_B) = 
		f(e_B)g(e_B) = 
		\eta(f)\eta(g) = \NewLine =  
		\eta(f \otimes g) 
	} 
	\Say{\Big[ (f,g).*.3 \Big]}{ \bd \FUNC{finiteDualCoalgebra}\bd \FUNC{dualAlgebra} \bd \BIALG{R}(B) \NewLine
		\bd^{-1} \FUNC{finiteDualCoalgebra} \bd \FUNC{tensorProductCoalgebra}  }
	{ 
		\Delta\Big(e_{B^\circ}(\alpha)\Big)(x \otimes y) = 
		e_{B^\circ}(\alpha)(xy) = 
		\alpha \eta_B(xy) = \NewLine =  
		\alpha \eta_B(x)\eta_B(y) =
		e_{B^\circ}\otimes e_{B^\circ}\big(\Delta(\alpha)\Big)(x \otimes y) 
	}
	\Conclude{\Big[ (f,g).*.4 \Big]}{ \bd \FUNC{finiteDualCoalgebra}\bd \FUNC{dualAlgebra} \bd \BIALG{R}(B) 
		\NewLine \
		\bd^{-1} \FUNC{finiteDualCoalgebra} \bd \FUNC{tensorProductCoalgebra}  }
	{ 
		\eta_{B^\circ}(e_{B^\circ}(\alpha)) = 
		e_{B^\circ}(\alpha)(e_B) =
		\alpha \eta_B( e_B) = 
		\alpha = \NewLine =  
		\eta_R(\alpha)
	}
	\DeriveConclude{[*]}{\bd \BIALG{R}}{B^\circ \in \BIALG{R}}
	\EndProof
}
\Page{
	\DeclareFunc{productOfHadamard}{\prod k : \Field \. \mathrm{LR}(k) \otimes \mathrm{LR}(k) \Arrow{\COALG{k}} \mathrm{LR}(k)}
	\DefineNamedFunc{productOfHadamard}{}{\odot_H}{\FUNC{multiplication}\Big(\FUNC{grouplike}(A)\Big)^\circ}
	\\
	\Theorem{HadamardProductFormula}
	{
		\forall k : \Field \. 
		\forall s,t \in \mathrm{LR}(k) \.
		s \odot_H t = \Lambda n \in \Int_+ \. s_n t_n
	}
	\Assume{p}{k[x]}
	\Say{n}{\deg p}{\Int_+}
	\Conclude{[p.*]}{ 
		\bd k[x](p) 
		\bd \FUNC{dualAlgera}(\FUNC{grouplike}(k))
		\bd \FUNC{tensorMap}(s,t)
		\bd \mathrm{LR}(k) 
		\bd^{-1} \bd \mathrm{LR}(k)
		\bd k[x] (p)
	}
	{  
		\NewLine :
		p \; (s \odot_H t) =
		\sum^{n}_{i=0} p_i x^i \; (s \odot_H t) =
		\sum^n_{i=1} p_i x^i \otimes x^i \; (s \otimes t)
		\sum^n_{i=0}  p_i  s(x^i) t(x^i) = 
		\sum^n_{i=0} p_i s_i t_i  \NewLine = 
		p \; \Lambda i \in \Int_+ \.  s_it_i
	}
	\DeriveConclude{[*]}{I(=,\to)}{\LOGIC{This}}
	\EndProof
	\\
	\Theorem{HadamrdProductCharacteristicPolynomial}
	{
		\forall k : \TYPE{NumericField} \.
		\forall s,t \in \mathrm{LR}(k) \.
		\forall n,m \in \Nat \. \NewLine \. 
		\forall \alpha :  n \ToInj \widehat{k} \.
		\forall \beta  :  m \ToInj \widehat{k} \.
		\forall (0.1) : \chi_s(x) = \prod^n_{i=1} (x - \alpha_i) \.
		\forall (0.2) : \chi_t(x) = \prod^n_{i=1} (x - \beta_i)  \. \NewLine \.
		\chi_{t \odot_H s}(x)  =  \prod^n_{i=1} \prod^m_{j=1} (x - \alpha_i \beta_j)
	}
	\NoProof
	\\
	\DeclareFunc{productOfHurwitz}{\prod k : \Field \. \mathrm{LR}(k) \otimes \mathrm{LR}(k) \Arrow{\COALG{k}} \mathrm{LR}(k)}
	\DefineNamedFunc{productOfHurwitz}{}{*_H}{\FUNC{multiplication}\Big(\FUNC{primitive}(A)\Big)^\circ}
	\\
	\Theorem{HurwitzProductFormula}
	{
		\forall k : \Field \. 
		\forall s,t \in \mathrm{LR}(k) \.
		s \odot_H t = \Lambda n \in \Int_+ \. \sum^n_{i=0} C^i_n s_{n-i} t_i
	}
	\Assume{p}{k[x]}
	\Say{n}{\deg p}{\Int_+}
	\Conclude{[p.*]}{ 
		\bd k[x](p) 
		\bd \FUNC{dualAlgera}(\FUNC{primitive}(k))
		\bd \FUNC{tensorMap}(s,t)
		\bd \mathrm{LR}(k) 
		\bd^{-1} \bd \mathrm{LR}(k)
		\bd k[x] (p)
	}
	{  
		\NewLine :
		p \; (s *_H t) =
		\sum^{n}_{m=0} p_m x^i \; (s *_H t) =
		\sum^n_{m=0} p_m \sum^m_{i=0} C^i_m x^{m-i} \otimes x^i \; (s \otimes t) = \NewLine = 
		\sum^n_{m=0} p_m \sum^m_{i=0} C^i_m  s(x^{m-i}) t(x^i) = 
		\sum^n_{m=0} p_m \sum^m_{i=0} C^i_m  s_{m-i} t_i  \NewLine = 
		p \; \Lambda m \in \Int_+ \. \sum^m_{i=0}  s_{m-i}t_i
	}
	\DeriveConclude{[*]}{I(=,\to)}{\LOGIC{This}}
	\EndProof
}\Page{
	\Theorem{HurwitzProductCharacteristicPolynomial}
	{
		\forall k : \TYPE{NumericField} \.
		\forall s,t \in \mathrm{LR}(k) \.
		\forall n,m \in \Nat \. \NewLine \. 
		\forall \alpha :  n \ToInj \widehat{k} \.
		\forall \beta  :  m \ToInj \widehat{k} \.
		\forall (0.1) : \chi_s(x) = \prod^n_{i=1} (x - \alpha_i) \.
		\forall (0.2) : \chi_t(x) = \prod^n_{i=1} (x - \beta_i)  \. \NewLine \.
		\chi_{t *_H s}(x)  =  \prod^n_{i=1} \prod^m_{j=1} (x - \alpha_i - \beta_j)
	}
	\NoProof
}
\newpage
\subsection{Algebraic Myhill-Nerode Theorem}
\Page{
	\Theorem{AlgebraicFiniteIndexLemma}
	{
		\forall \Sigma : \TYPE{Finite} \.
		\forall  L \in \L(\Sigma) \.
		\forall  k : \Field \. \NewLine \. 
		L : \TYPE{FiniteIndex}(\Sigma) \iff
		\Big|  \sigma^* \hit_k \chi_L \Big| < \infty
	}
	\Say{\F}{  ( \omega \hit_k \chi_L)  }{?k\Sigma^{**}}
	\Assume{f}{\F}
	\Say{\Big( \alpha,[1] \Big)}{\ByConstr (\F)}{\sum \alpha \in \Sigma^* \. f = \alpha \hit \chi_L }
	\Assume{\beta}{\Sigma^*}
	\Assume{[2]}{ \beta \hit \chi_L \neq f }
	\Say{\Big(\omega,[3]\Big)}{\bd \FUNC{hitAction}}{\sum \omega \in \Sigma^* \. \chi_L(\alpha\omega) \neq \chi_L(\beta\omega)}
	\Conclude{[\beta.*]}{\bd \FUNC{characteristicFunction}[3]}{\alpha \not \sim_L \beta}
	\DeriveConclude{[2]}{I(\forall)I(\Imply)}{\forall \beta \in \Sigma^* \. \beta \hit \chi_L \neq f \Imply \alpha \not \sim_L \beta} 
	\Derive{[1]}{I(\forall)}
	{
	    \forall f \in \F \. 
	    \forall \alpha,\beta \in \Sigma^* \. 
	    \Big( (\alpha \hit \chi_L) = f \And
	    (\beta  \hit \chi_L) \neq f \Big) \Imply
	    \alpha \not \sim_L \beta
	}
	\Conclude{[*]}{\bd \TYPE{FiniteIndex}[1]}{\LOGIC{This}}
	\EndProof
	\\
	\Theorem{AlgebraicMyhillNerodeTheorem}
	{
		\forall M : \TYPE{Monoid} \.
		\forall k : \TYPE{Field} \. 
		\forall f \in kM^* \. 
		\Big| ( M \hit f) \Big| < \infty \iff \NewLine \iff 
		\Big(\exists B : \BIALG{k} :
		\exists \psi : kM \Arrow{\BIALG{k}} B :
		\exists p \in B^\circ :
		\dim B < \infty \And  f = \psi \; p \Big)
	}
	\Assume{[1]}{| M \hit f| < \infty}
	\Say{\F}{M \hit f}{ ?kM^*  }
	\Say{R}{\{ \Lambda g \in \F \. m \hit g | m \in M  \}}{?(\F \to \F)}
	\Say{[2]}{ \THM{PowerSetCardinality}\ByConstr(R)[1] }{ |R| < \infty   }
	\Assume{A,B}{R}
	\Say{(a,b,[3])}{\ByConstr(R) }{\sum a,b \in M \. A = (a \hit \cdot) \And B = (b \hit \cdot) }
	\Conclude{AB}{(a \hit \cdot)}{R}
	\Derive{(\cdot)}{I(\to)}{(R \times R) \to R}
	\Say{[3]}{\ByConstr (R,\cdot)\bd \TYPE{Monoid}(M)}{\Big((R,\cdot):\TYPE{Monoid}\Big)}
	\Say{B}{kR}{\BIALG{k}}
	\Say{[4]}{\bd kR [2]}{\dim B < \infty} 
	\Say{\psi}{ \Lambda p \in kM \. \sum_{m \in M } p_m( m  \hit \cdot ) }{ kM \Arrow{\BIALG{k}} B}
	\Assume{A}{R}
	\Say{(a,[5])}{\ByConstr(R)[5]}{ \sum a \in M \. A = a \hit \cdot  }
	\Say{p(A)}{f(a)}{k}
	\Assume{b}{M}
	\Assume{[6]}{ A = b \hit \cdot}
	\Conclude{[A.*]}{\ByConstr R(A)[6][5]}{  f(b) = (Af)(e) = f(a)   }
	\Derive{p}{\bd \FUNC{monoidBialgebra}}{ B^\circ}
	\Conclude{[*]}{\ByConstr p }{ f = \psi \; p}
}\Page{
	\Derive{[1]}{I(\Imply)}{\LOGIC{Left} \Imply \LOGIC{Right}}
	\Assume{[2]}{\LOGIC{Right}}
	\Say{R}{\TYPE{Grouplike}(B)}{?B}
	\Say{[3]}{\THM{LinearlyIndependentGrouplike}(R)}{\Big( R : \LI(B)\Big)}
	\Say{[4]}{\bd \FUNC{dimension}[2]}{|B|< \infty}
	\Say{\mu_{B,kM}}{\Lambda b \in B \. \Lambda a \in kM \. b\psi(a)}
	{ B \otimes kM \Arrow{\VS{k}} B  }
	\Assume{b}{B}
	\Assume{a}{kM}
	\Conclude{[b.*]}{
		\bd \mu_{B,kM} 
		\bd \BIALG{k}(B) 
		\bd \BIALG{k}(kM,B)(\psi)
		\bd \FUNC{SweedlerNotation}(b,a) \NewLine 
		\bd \FUNC{tensorProductAlgebra}(B,kM)
		\bd^{-1} \mu_{B,kM}
	}{ 
		\Delta(ba) =
		\Delta\Big(b\psi(a)\Big) =
		\Delta(b)\Delta\Big(\psi(a)\Big) = \NewLine = 
		\Delta(b)(\psi \otimes \psi)\Delta(a) = 
		\sum_{b,a} b_1\psi(a_1)  \otimes b_2\psi(a_2) =
		\sum_{b,a} b_1a_1 \otimes b_2a_2
	}
	\Derive{[5]}{\bd \RBCOALG{kM}}{ (B,\mu_{B,kM}) \in \RBCOALG{kM}}
	\Assume{r}{R}
	\Assume{a}{M}
	\Say{[6]}{\bd \RBCOALG{kM}[5] \bd \FUNC{monoidBialgebra}(k,M) }{\Delta(ra)= ra \otimes ra}
	\Conclude{[r.*]}{\bd R [6] }{ra \in R}
	\Derive{[6]}{\bd \TYPE{Subset}}{RM \subset M}
	\Assume{a,x}{M}
	\Say{[7]}{
		\bd \FUNC{hitAction}
		[2]
		\bd \BIALG{k}(kM,B)(\psi)
		\bd \BIALG{k}(B)
		\bd^{-1}\mu_{B,kM}
	}{
		\NewLine :
		(a \hit f)(x) = 
		f(ax) = 
		p\Big( \psi(ax) \Big) =
		p\Big( \psi(a)\psi(x)\Big) =
		p\Big(e\psi(a)\psi(x)\Big) =
		p\Big(eax)
	}
	\Conclude{\Big[(a,x).*\Big]}{[6][7]}{\exists r \in R \. (a \hit f)(x) = (r \hit p)\psi(x)}
	\Derive{[7]}{I(\forall)I(=,\to)}{\forall a \in M \. \exists r \in R \. (a \hit f) = (r \hit p)\psi}
	\Conclude{[2.*]}{[4][7]}{|M \hit f| < \infty}
	\DeriveConclude{[*]}{I(\Imply)I(\iff)[1]}{\LOGIC{Left} \iff \LOGIC{Right}}
	\EndProof
	\\
	\DeclareType{MyhillNerodeAlgebra}{
		\prod k : \Field \. 
		\prod M : \TYPE{Monoid} \.
		\prod f \in kM^* \.
		?\BIALG{k}
	}
	\DefineType{B}{MyhillNerodeAlgebra}{
		\exists \psi : kM \Arrow{\BIALG{k}} B :
		\exists p \in B^\circ :
		\dim B < \infty \And  f = \psi \; p 	
	}
	\\
	\DeclareFunc{algebraicFiniteAutomaton}
	{
		\prod \Sigma : \TYPE{Finite} \.
		\prod L \in \L(\Sigma) \.
		\prod k : \TYPE{Field} \.\NewLine \.
		\TYPE{MyhillNerodeAlgebra}(k,\Sigma^*,\chi_L) \to 
		\sum A : \TYPE{FiniteAutomoton} \. \FUNC{language}(A) = L
	}
	\DefineFunc{algebraicFiniteAutomoton}{ (B,\psi,p,[0])  }
	{
	    \Big(\Sigma,\TYPE{Grouplike}(B),\mu_{B,k\Sigma^*},e_B,e_B L \Big) 
	}
	\Assume{\omega,\omega'}{ \sigma^*   }
	\Assume{[1]}{\omega \in L}
	\Assume{[2]}{\omega \not \in L }
	\Say{[3]}{\omega\bd \chi_L [1][0]\bd \LALGE{k}(B)\bd \RBCOALG{k\Sigma^*}(B)}
	{
	    1 = 
	    \chi_L(\omega) = 
	    p\Big( \psi(\omega)  \Big) = 
	    p\Big( e_B \psi(\omega) \Big) = p( e_B \omega )
	}
	\Say{[4]}{\omega'\bd \chi_L [1][0]\bd \LALGE{k}(B)\bd \RBCOALG{k\Sigma^*}(B)}
	{
	    0 = 
	    \chi_L(\omega') = 
	    p\Big( \psi(\omega')  \Big) = 
	    p\Big( e_B \psi(\omega') \Big) = p( e_B \omega' )	
	}
	\Conclude{\Big[[(\omega,\omega').*]\Big]}{I(\to,\#)}
	{
		e_B \omega \neq e_B \omega'
	}
	\DeriveConclude{[*]}{\ByConstr(A)\bd \FUNC{language}}
	{\FUNC{language}(A) = L}
	\EndProof
}
\newpage
\subsection{Regular Sequences}
\Page{
	\DeclareType{RegularSequence}
	{  
		\prod k : \Field \.   ?(\Nat \to k)
	}
	\DefineType{x}{RegularSequence}
	{
		\exists M : \TYPE{Monoid} :
		\exists m : \Nat \ToBij M :
		\exists f \in kM^* :
		x = f(m) \And | M \hit f | < \infty
	}
	\\
	\Theorem{RegularSequenceCharacterization}{ 
		\forall k : \Field \. 
		\forall M : \TYPE{Monoid} \.
		\forall m : \Nat \ToBij M \.
		\forall x : \Nat \to k \. \NewLine
		x : \TYPE{RegularSequenc}(k) \iff
		\exists f \in kM^* \.
		x = f(m) \And  \dim (kM \hit f) < \infty \And |f(M)| < \infty
	}
	\Assume{f}{kM^*}
	\Assume{[1]}{x = f(M)}
	\Assume{[2]}{\dim (kM \hit f) < \infty}
	\Assume{[3]}{|f(M)| < \infty}
	\Say{n}{\dim (kM \hit f)}{\Int_+}
	\Say{\Big(g,p,[4]\Big)}{\THM{BasisWithSpecialSupportTHM}}{
		\sum g : \TYPE{Basis}(kM \hit f) \. 
		\sum p : n \to M \. \NewLine \.  
		\forall i,j \in n \. g_i(p_j) = \delta^i_j
	}
	\Assume{a}{M}
	\Say{\Big( \alpha, [5] \Big)}{\bd \Basis(kM \hit f)(g)(a \hit f)}
	{\sum \alpha \in  k^n \. a \hit f = \alpha g}
	\Say{[6]}{[4][5]}{  \forall i \in n \.  \alpha_i =  f( a_i ) \in f(M)    }
	\Conclude{[*]}{\bd \TYPE{SetImage}[5][6]}{(a \hit f) \in f(M)\{g_i\}^n_{i=1}}
	\Derive{[5]}{\bd \TYPE{Subset}}{ (M \hit f) \subset f(M)\{g_i\}^n_{i=1}  }
	\Say{[6]}{ \THM{FiniteProduct}[5][3]}{  | M \hit f | < \infty    }
	\Conclude{[*]}{\bd^{-1} \TYPE{RegularSequence}[1][6]}{ \Big(x : \TYPE{RegularSequence}(k) \Big)   }
	\EndProof
	\\
	\Theorem{FiniteFieldLinearlyRecursiveIsRegular}
	{
		\forall p : \TYPE{Prime}(\Int) \.
		\forall n \in \Nat \. 
		\forall x \in \mathrm{LR}(\mathbb{F}_{p^n}) \. \NewLine \. 
		x : \TYPE{RegularSequence}(\mathbb{F}_{p^n}) 
	}
	\Say{q}{p^n}{\Nat}
	\Say{k}{\mathbb{F}_q}{\Field}
	\Say{M}{\Int_+}{\TYPE{Monoid}}
	\Say{[1]}{\ByConstr M \ByConstr}{kM \cong_{\BIALG{k}} k[x]}
	\Say{f}{\bd k[x] \Lambda i \in \Int_+ \. f(x^i) = s_i }{\Big(k[x]\Big)^\circ}
	\Say{[2]}{\THM{FiniteHitAction}\ByConstr \bd s}{\dim \Big(k[x] \hit f\Big) < \infty }
	\Say{[3]}{\bd \mathbb{F}_q}{|f(M)| < \infty}
	\Conclude{[*]}{ \THM{RegularSequenceCharactrization}\Big( [1],[2] \Big)[3]}
	{\Big(s : \TYPE{RegularSequence}(\mathbb{F}_{p^n}) \Big)}
	\EndProof
}
\newpage
\subsection{Hopf Algebras}
\Page{
	\DeclareType{HopfAlgebra}{\prod R \in \ANN \. \prod B : \BIALG{R} \. B \Arrow{\LMOD{R}} B}
	\DefineType{(B,\sigma)}{HopfAlgebra}
	{
		\Delta(\id \otimes \sigma)\mu = \eta e = \Delta(\sigma \otimes \id)\mu
	}
	\\
	\DeclareFunc{antipode}{\prod R \in \ANN \. \prod (B,\sigma) : \TYPE{HopfAlgebra}(R) \. B \Arrow{\LMOD{R}} B }
	\DefineFunc{antipode}{}{\sigma}
	\\
	\DeclareFunc{categoryOfHopfAlgebra}
	{
		\ANN \to \CAT
	}
	\DefineNamedFunc{categoryOfHopfAlgebra}
	{R}{\HOPF{R}}{\Big(\TYPE{HopfAlgebra}(R),\BIALG{R}, \circ, \id  \Big)}
	\\
	\DeclareFunc{groupHopfAlgebra}
	{
		\prod R \in \ANN \.
		\to  \GRP 
		\TYPE{HopfAlgebra}(R)
	}
	\DefineNamedFunc{groupHopfAlgebra}{ G }
	{ RG }{ \Big( RG, \bd RG \Lambda g \in G \. g^{-1} \Big) }
	\\
	\DeclareType{QuantumGroup}{\prod R \in \ANN \. ?\HOPF{R}}
	\DefineType{ A}{QuantumGroup}{A \IsNot \TYPE{Commutative}(R) \And A \IsNot \TYPE{Cocommutative}(R)}
	\\
	\DeclareFunc{convolutionProduct}{
		\prod R \in \ANN \. 
		\prod A \in \COALG{R} \.
		\prod B \in \LALGE{R} \.
		\NewLine \. 
		\LMOD{R}(A,B) \otimes \LMOD{R}(A,B) 
		\Arrow{\LMOD{R}} \LMOD{R}(A,B)
	}
	\DefineNamedFunc{convolutionProduduct}{\varphi \otimes \psi}{\varphi * \psi}
	{ \mu_B (\varphi \otimes \psi)\Delta_A  }
	\\
	\Theorem{ConvolutionMonoid}{
		\forall R \in \ANN \.
		\forall A \in \COALG{R} \.
		\forall B \in \LALGE{R} \.
		\Big( \LMOD{R}(A,B), * \Big) : \TYPE{Monoid}
	}
	\Assume{\phi,\phi',\phi''}{\LMOD{R}(A,B)}
	\Assume{a}{A}
	\Conclude{[*]}{ \bd \LALGE{R}(B) \bd \COALG{R}(A) \bd \FUNC{SweedlerNotation}  }{
		\NewLine :
		(\phi * \phi') *\phi''(a) =
		\sum_{a} \phi(a_1)\phi'(a)\phi''(a) = 
		\phi * (\phi' * \phi'')(a)
	}
	\Derive{[1]}{I(=,\to)\bd^{-1}\TYPE{Associtaive}}{\Big((*) : \TYPE{Associative}\Big)}
	\Assume{\phi}{\LMOD{R}(A,B)}
	\Assume{a}{A}
	\Say{[\phi.*.1]}{\bd \FUNC{convolution} \bd e \bd \GRP(A,B)(\phi) \bd \COALG{R}(A) }{ 
			\NewLine :
			(\eta e * \phi)(a) =  
			\sum_a \eta(a_1)\phi(a_2) = 
			\phi\left( \sum_a \eta(a_1)a_2 \right) =
			\phi(a)
		}
	\Conclude{[\phi.*.2]}{\bd \FUNC{convolution} \bd e \bd \GRP(A,B)(\phi) \bd \COALG{R}(A) }{ 
			\NewLine :
			( \phi * \eta a)(a) =  
			\sum_a \eta(a_2)\phi(a_1) = 
			\phi\left( \sum_a \eta(a_2)a_1 \right) =
			\phi(a)
		}
	\Derive{[2]}{\bd \TYPE{Neutral}}
	{ \eta_A e_B : \TYPE{Neutral}(*)  }
	\Conclude{[*]}{\bd^{-1}\TYPE{Monoid}[1][2]}
	{\LOGIC{This}}
	\EndProof
}
\Page{
	\Theorem{AntipodIsInverseOfIdentity}
	{
		\forall R \in \ANN \. 
		\forall A \in \HOPF{R} \.
		\sigma_A * {\id}_A = \eta_A e_A = {\id}_A * \sigma_A
	}
	\NoProof
	\\
	\Theorem{AntipodeAntihomo}
	{
		\forall R \in \ANN \.
		\forall A \in \HOPF{R} \.
		\forall a,b \in A \. 
		\sigma(ab) = \sigma(b)\sigma(a)
	}
	\Say{\varphi}{ \mu(\sigma \otimes \sigma)\FUNC{swap}  }{A \otimes A \Arrow{\LMOD{R}} A}
	\Assume{a,b}{A}
	\Say{\Big[(a,b).*.1\Big]}
	{
		\bd \FUNC{convolution}
		\bd \FUNC{tensorProductCoalgebra}
		\ByConstr \varphi \bd \FUNC{tensorMap} 
		\bd^2 \HOPF(R)(A)		
	}
	{
		\NewLine : 
		(a \otimes b)(\mu * \varphi) = 
		(a \otimes b)\Delta (\mu \otimes \varphi) \mu =
		\sum_a \sum_b  (a_1 \otimes b_1) \otimes (a_2 \otimes b_2) 
		(\mu \otimes \varphi) \mu  = \NewLine = 
		\sum_a \sum_b a_1 b_1 \sigma(b_2) \sigma(a_2) = 
		\eta(b) \sum_a a_1 \sigma(a_2) =
		\eta(b) \eta(a) e
	}
	\Conclude{\Big[(a,b).*.2\Big]}
	{
		\bd \FUNC{convolution}
		\bd \FUNC{tensorProductCoalgebra}
		\ByConstr \varphi \bd \FUNC{tensorMap} 
		\bd^2 \HOPF(R)(A)		
	}
	{
		\NewLine :
		(a \otimes b)(\varphi * \mu) = 
		(a \otimes b)\Delta (\varphi \otimes \mu) \mu =
		\sum_a \sum_b  (a_1 \otimes b_1) \otimes (a_2 \otimes b_2) 
		(\varphi \otimes \mu) \mu  =  \NewLine =
		\sum_a \sum_b \sigma(b_1) \sigma(a_1) a_2 b_2 = 
		\eta(a) \sum_b  \sigma(b_1) b_2 =
		\eta(b) \eta(a) e
	}
	\Derive{[1]}{\bd^{-1} \TYPE{Inverse}}{ \varphi = \mu^{-1}  }
	\Assume{a,b}{A}
	\Say{\Big[(a,b).*.1\Big]}{
		\bd \FUNC{convolution}
		\bd \FUNC{tensorProductCoalgebra}
		\bd \FUNC{tensorMap}
		\bd \BIALG{R}(A)
		\bd \HOPF{R}(A)
		\NewLine
		\bd \BIALG{R}(A)\bd \ANN(R)
	}
	{
		(a \otimes b)( \mu \sigma  * \mu) =
		(a \otimes b) \Delta ( \mu \sigma * \mu  ) \mu =
		\sum_a \sum_b  (a_1 \otimes b_1) \otimes (a_2 \otimes b_2)
		(\mu \sigma * \mu)  = \NewLine = 
		\sum_a \sum b \sigma(a_1b_1)a_2b2 = 
		\sum_{c = ab} \sigma(c_1)c_2 = 
		\eta(ab)e = 
		\eta(b)\eta(a)e
	}
	\Say{\Big[(a,b).*.2\Big]}{
		\bd \FUNC{convolution}
		\bd \FUNC{tensorProductCoalgebra}
		\bd \FUNC{tensorMap}
		\bd \BIALG{R}(A)
		\bd \HOPF{R}(A)
		\NewLine 
		\bd \BIALG{R}(A)\bd \ANN(R)
	}
	{
		(a \otimes b)( \mu  * \mu\sigma) =
		(a \otimes b) \Delta ( \mu  * \mu\sigma  ) \mu =
		\sum_a \sum_b  (a_1 \otimes b_1) \otimes (a_2 \otimes b_2)
		(\mu  * \mu \sigma)  = \NewLine = 
		\sum_a \sum b  a_1b_1 \sigma(a_2b2) = 
		\sum_{c = ab} c_1\sigma(c_2) = 
		\eta(ab)e = 
		\eta(b)\eta(a)e
	}
	\Derive{[2]}{\bd^{-1} \TYPE{Inverse}(*)}
	{
		\mu\sigma = \mu^{-1}
	}
	\Say{[3]}{[2][1]}{ \mu\sigma = \varphi  }
	\Conclude{[*]}{\bd \varphi [3]}{ \LOGIC{This} }
	\EndProof
	\\
	\Theorem{UnityAntipode}
	{
		\forall R \in \ANN \.
		\forall A \in \HOPF{R} \.
		e \sigma = e
	}
	\NoProof
}
\Page{
	\Theorem{InvolutionAntipode}
	{
		\forall R \in \ANN \.
		\forall A \in \HOPF{R} \And \TYPE{Cocommutative} \.
		\sigma^2 = \id
	}
	\Assume{a}{A}
	\Say{[a.*.1]}
	{
		\bd \FUNC{convolution}
		\bd \FUNC{SweedlerNotation}
		\bd \FUNC{tensorMap}
		\THM{AntipodeAntihomo}(A)\Big(a_1,\sigma(a_2)\Big)
		\NewLine
		\bd \TYPE{Cocommutative}(A)
		\bd \LMOD{R}(A,A)(\sigma)
		\bd \HOPF{R}(A)
		\bd \LMOD{R}(A,A)(\sigma)
		\THM{UnityAntipode}(A)
	}
	{
		\NewLine :  
		a \; (\sigma * \sigma^2) =
		a \Delta \; (\sigma \otimes \sigma^2) \; \mu =
		\sum_{a} a_1 \otimes a_2 \; (\sigma \otimes \sigma^2) \; \mu =
		\sum_{a} \sigma(a_1)\sigma^2(a_2) =
		\sum_{a} \sigma\Big(\sigma(a_2)a_1\Big) = \NewLine = 
		\sum_{a} \sigma\Big( a_1 \sigma(a_2)\Big) =
		\sigma\left( \sum_{a} a_1 \sigma(a_2) \right) =
		\sigma( \eta(a) e ) =
		\eta(a)
	}
	\Conclude{[a.*.2]}
	{
		\bd \FUNC{convolution}
		\bd \FUNC{SweedlerNotation}
		\bd \FUNC{tensorMap}
		\THM{AntipodeAntihomo}(A)\Big(a_1,\sigma(a_2)\Big)
		\NewLine
		\bd \TYPE{Cocommutative}(A)
		\bd \LMOD{R}(A,A)(\sigma)
		\bd \HOPF{R}(A)
		\bd \LMOD{R}(A,A)(\sigma)
		\THM{UnityAntipode}(A)
	}
	{
		\NewLine :  
		a \; (\sigma^2 * \sigma) =
		a \Delta \; (\sigma^2 \otimes \sigma) \; \mu =
		\sum_{a} a_1 \otimes a_2 \; (\sigma^2 \otimes \sigma) \; \mu =
		\sum_{a} \sigma^2(a_1)\sigma(a_2) =
		\sum_{a} \sigma\Big(a_2\sigma(a_1)\Big) = \NewLine = 
		\sum_{a} \sigma\Big( \sigma(a_1)a_2\Big) =
		\sigma\left( \sum_{a} \sigma(a_1)a_2 \right) =
		\sigma( \eta(a) e ) =
		\eta(a)
	}
	\Derive{[1]}{\bd^{-1}\TYPE{Invers}(*)}{\sigma^2 = \sigma^{-1}}
	\Conclude{[*]}{\THM{AntipodeIsInverseOfIdentity}[1]}
	{
		\sigma^2 = \id
	}
	\EndProof
	\\
	\Theorem{ComultiplicationOfAntipode}
	{
		\forall R \in \ANN \.
		\forall A \in \HOPF{R} \.
		\sigma \; \Delta  =  \Delta \; (\sigma \otimes \sigma) \;  \FUNC{swap}
	}
	\Say{\varphi}{\Delta \; (\sigma \otimes \sigma) \; \FUNC{swap} }
	{ A \Arrow{\LMOD{R}} A \otimes A }
	\Assume{a}{A}
	\Say{[a.*.1]}
	{
		\bd \FUNC{convolution}
		\bd \FUNC{SweedlerNotation}
		\bd \FUNC{tensorMap}
		\bd \FUNC{tensorProductAlgebra}
		\NewLine
		\bd \COALG{R}(A) \bd \HOPF{R}(A)
		\bd \FUNC{tensorProduct}
		\bd \COALG{R}(A)
		\bd \HOPF{R}(A)
	}
	{
		\NewLine : 
		a \; \Delta * \varphi =
		a \; \Delta (\Delta \otimes \varphi) \mu =
		\sum_a (a_1 \otimes a_2) \; (\Delta \otimes \varphi) \mu =
		\sum_a  (a_1 \otimes a_2) \otimes \big(\sigma(a_4) \otimes \sigma(a_3)\big) \mu  = \NewLine = 
		\sum_a  a_1 \sigma(a_4) \otimes a_2 \sigma(a_3) =
		\sum_a  a_1 \sigma(a_3) \otimes \eta(a_2) e =
		\left( \sum_a  a_1 \eta(a_2)\sigma(a_3) \right) \otimes e =
		\left( \sum_a  a_1 \sigma(a_2)   \right) \otimes e = \NewLine =  
		\eta(a) e \otimes e
	}
	\Conclude{[a.*.2]}
	{
		\bd \FUNC{convolution}
		\bd \FUNC{SweedlerNotation}
		\bd \FUNC{tensorMap}
		\bd \FUNC{tensorProductAlgebra}
		\NewLine
		\bd \COALG{R}(A) \bd \HOPF{R}(A)
		\bd \FUNC{tensorProduct}
		\bd \COALG{R}(A)
		\bd \HOPF{R}(A)
	}
	{
		\NewLine : 
		a \; \varphi * \Delta =
		a \; \Delta (\varphi \otimes \Delta) \mu =
		\sum_a (a_1 \otimes a_2) \; (\varphi \otimes \Delta) \mu =
		\sum_a \big(\sigma(a_2) \otimes \sigma(a_1)\big) \otimes (a_3 \otimes a_4\big) \mu  = \NewLine = 
		\sum_a   \sigma(a_2) a_3 \otimes \sigma(a_1) a_4 =
		\sum_a  \eta(a_2) e \otimes \sigma(a_1)a_4 =
		e \otimes \left( \sum_a  a_1 \eta(a_2)\sigma(a_3) \right)  =
		e \otimes \left( \sum_a  a_1 \sigma(a_2)   \right)  = \NewLine =  
		\eta(a) e \otimes e
	}
	\Derive{[1]}{\bd^{-1}\TYPE{Inverse}(*)}
	{  \varphi = \Delta^{-1}   }
}
\Page{
	\Assume{a}{A}
	\Say{\Big[a.*.1\Big]}{
		\bd \FUNC{convolution} 
		\bd \FUNC{SweedlerNotation}
		\bd \FUNC{tensorMap}
		\bd \BIALG{R}(A)
		\bd \LMOD{R}(A,A \otimes A)
		\NewLine
		\bd \HOPF{R}(A)
		\bd \BIALG{R}(A)
	}
	{
		a \; \sigma \Delta * \Delta = 
		a \; \Delta \Big( \sigma \Delta \otimes \Delta  \Big) \mu = 
		\sum_a a_1 \otimes a_2 \;  \Big( \sigma \Delta \otimes \Delta \Big) \mu = \NewLine = 
		\sum_a  \left( \sum_{b = \sigma(a_1)} b_1 \otimes b_2    \right) \otimes (a_2 \otimes a_3) \mu =
		\sum_a \sum_{b = \sigma(a_1)} b_1a_2 \otimes b_2a_3 =
		\sum_a \Delta( \sigma(a_1)a_2  ) = 
		\Delta \left( \sum_a \sigma(a_1)a_2 \right) = \NewLine = 
		\Delta ( \eta(a) e ) = 
		\eta(a) e \otimes e
	} 
	\Say{\Big[a.*.1\Big]}{
		\bd \FUNC{convolution} 
		\bd \FUNC{SweedlerNotation}
		\bd \FUNC{tensorMap}
		\bd \BIALG{R}(A)
		\bd \LMOD{R}(A,A \otimes A)
		\NewLine
		\bd \HOPF{R}(A)
		\bd \BIALG{R}(A)
	}
	{
		a \;  \Delta * \sigma \Delta = 
		a \; \Delta \Big( \Delta \otimes \sigma \Delta  \Big) \mu = 
		\sum_a a_1 \otimes a_2 \;  \Big(  \Delta \otimes \sigma \Delta \Big) \mu = \NewLine = 
		\sum_a  (a_1 \otimes a_2) \otimes \left( \sum_{b = \sigma(a_3)} b_1 \otimes b_2    \right)  \mu =
		\sum_a \sum_{b = \sigma(a_3)} a_1b_1 \otimes a_2b_2 =
		\sum_a \Delta( a_1\sigma(a_2)   ) = 
		\Delta \left( \sum_a a_1\sigma(a_2) \right) = \NewLine = 
		\Delta ( \eta(a) e ) = 
		\eta(a) e \otimes e
	} 
	\Derive{[2]}{\bd^{-1}\TYPE{Inverse}(*)}
	{  \sigma \Delta  = \Delta^{-1}   }
	\Conclude{[*]}{[1][2]\ByConstr \varphi}{\LOGIC{This}}
	\EndProof
	\\
	\Theorem{CounitOfAntipode}{
		\forall R \in \ANN \. 
		\forall A \in \HOPF{R} \. 
		\sigma \; \eta = \eta
	}
	\Assume{a}{A}
	\Conclude{[a.*]}
	{
		\bd \BIALG{R}(A)
		\bd \HOPF{R}(A)
		\bd \BIALG{R}(A)
		\bd \LMOD{R}(A,A)(\sigma \eta)
		\bd \COALG{R}(A)
	}
	{
	     \NewLine : 
	     a \; \eta  =
	     a \; \eta \; e \; \eta =
	     a \; \Delta \; (\sigma \otimes \id) \; \mu \; \eta =
	     a \;  \Delta \; (\sigma \otimes \id) \; (\eta \otimes \eta) \; \mu =
	     a \;  \Delta (\eta \otimes \id) \; \mu \; \sigma \; \eta = 
	     a \; \sigma \; \eta
	}
	\DeriveConclude{[*]}{I(=,\to)}
	{
		\eta = \sigma \; \eta
	}
	\EndProof
	\\
	\DeclareType{HopfIdeal}{\prod R \in \ANN \. \prod A \in \HOPF{R} \. ?\TYPE{Biideal}(A)}
	\DefineType{I}{HopfIdeal}{\sigma(I) \subset I}
	\\
	\Theorem{HopfQuotient}
	{
		\forall R \in \ANN \.
		\forall A \in \HOPF{R} \. 
		\forall I : \TYPE{HopfIdeal}(A) \.
		\left( \frac{A}{I}, \widehat{\sigma}_I  \right) \in \HOPF{R}
	}
	\NoProof
}\Page{
	\Theorem{HopfDuality}{  
		\forall R \in \ANN \.
		\forall A \in \HOPF{R} \.
		\forall [0] : \dim A < \infty \.
		A^\circ \in \HOPF{R}
	}
	\Assume{f}{A^\circ}
	\Assume{a}{A}
	\Say{[f.*.1]}
	{
		\bd \FUNC{finiteDualBialgebra}
		\bd \ABEL(A,R)(f)
		\bd \HOPF{R}(A)
	}
	{
		\NewLine :
		a \; f \; \Delta \; (\id \otimes \sigma^*) \; \mu  =
		\sum_a f\big( a_1 \sigma(a_2)  \big)   =
		f \left( \sum_a a_1 \sigma(a_2) \right)  =
		f(a)
	}
	\Conclude{[f.*.1]}
	{
		\bd \FUNC{finiteDualBialgebra}
		\bd \ABEL(A,R)(f)
		\bd \HOPF{R}(A)
	}
	{
		\NewLine :
		a  \;  f \; \Delta \; (\sigma^* \otimes \id) \; \mu  =
		\sum_a f\big( \sigma(a_1) a_2  \big)   =
		f \left( \sum_a \sigma(a_1) a_2 \right)  =
		f(a)
	}
	\DeriveConclude{[*]}{\bd \HOPF{R}}{A^\circ \in \HOPF{R}}
	\EndProof
}
\newpage
\subsection{Integrals of Hopf Algebras}
\Page{
	\DeclareType{LeftIntegral}{\prod R \in \ANN \. \prod A \in \HOPF{R} \. ?A}
	\DefineNamedType{a}{LeftIntegral}{a \in \int_A^l }{\forall x \in A \. xa = \eta(x)a}
	\\
	\DeclareType{RightIntegral}{\prod R \in \ANN \. \prod A \in \HOPF{R} \. ?A}
	\DefineNamedType{a}{RightIntegral}{a \in \int_A^r }{\forall x \in A \. ax = \eta(x)a}
	\\
	\Theorem{IntegralsAreSubMod}{
		\forall R \in \ANN \. 
		\forall A \in \HOPF{R} \. 
		\int^l_A, \int^r_A \submod{R} A
	}
	\NoProof
	\\
	\Theorem{IntegralsAreIdeal}
	{
		\forall R \in \ANN \.
		\forall A \in \HOPF{R} \.
		\int^l_A,\int^r_A : \Ideal(A)
	}
	\Assume{a}{\int^l_A}
	\Assume{x,y}{A}
	\Say{[a.*.1]}{\bd \int^l_A}
	{ 
		yax = \eta(y)ax
	}
	\Conclude{[a.*.2]}{\bd \int^l_A \bd \BIALG{R}(A) \bd \int^l_A}
	{
		yxa = \eta(yx)a = \eta(y)\eta(x)a = \eta(y)xa
	}
	\Derive{[1]}{\bd \int^l_A \bd^{-1} \Ideal(A)}{ \int^l_A \in \mathsf{Spec}(A)  }
	\Assume{a}{\int^r_A}
	\Assume{x,y}{A}
	\Say{[a.*.1]}{\bd \int^r_A \bd \BIALG{R}(A)\bd \ANN(R) \bd \LMOD{R}(A) \bd \int^r_a }
	{ 
		\NewLine :
		axy = 
		\eta(xy)a = 
		\eta(x)\eta(y)a = 
		\eta(y)\eta(x)a =
		\eta(y)ax
	}
	\Conclude{[a.*.2]}{\bd \int^r_A }
	{
		xay = xa \eta(y)
	}
	\Derive{[2]}{\bd \int^r_A \bd^{-1} \Ideal(A)}{ \int^l_A \in \mathsf{Spec}(A)  }
	\Conclude{[*]}{[1][2]}{\LOGIC{This}}
	\EndProof
	\\
	\DeclareType{Unimodular}{\prod_{R \in \ANN} ?\HOPF{R}}
	\DefineType{A}{Unimodular}{\int^l_A = \int^r_A}
}
\Page{
	\Theorem{IntegralsOfFiniteGroupAlgebras}{  
		\forall R \in \ANN \.
		\forall G : \FG \. 
		\int^l_{RG} = \int^r_{RG} = R \sum_{g \in G} g
	}
	\Assume{\alpha,\beta}{R}
	\Assume{h}{G}
	\Say{\Big[ (\alpha,\beta).*.1 \Big]}
	{ 
		\bd \GRP(G) \bd RG
	}
	{
		(\beta h) \alpha \sum_{g \in G} g = 
		(\beta \alpha) \sum_{g \in G} g =  
		\eta(\beta h) \alpha \sum_{g \in G} g
	}
	\Say{\Big[ (\alpha,\beta).*.2 \Big]}
	{ 
		\bd \GRP(G) \bd RG
	}
	{
		\left(\alpha \sum_{g \in G} g \right)  (\beta h) = 
		(\beta \alpha) \sum_{g \in G} g =  
		\eta(\beta h) \alpha \sum_{g \in G} g
	}
	\Derive{[1]}{\bd \int^l_A \bd \int^r_A \bd^{-2} \TYPE{Subset}}
	{  R\sum_{g \in G} g \subset \int^r_A \cap \int^l_A }
	\Assume{v}{RG}
	\Assume{[2]}{v \not \in R\sum_{g \in G} g}
	\Say{\Big( g,h  ,[3] \Big)}{\bd RG [2]} 
	{
		\sum_{g,h \in G} v_g \neq v_h
	}
	\Say{[4]}{\bd RG [3]}{ (gh^{-1} v)_h \neq v_h \And (v g^{-1}h)_g \neq v_g  }
	\Conclude{[v.*]}{\bd RG \bd \int^r_{RG} \bd \int^l_{RG}}{v \not \in \int^r_{RG} \And v \not \in \int^l_{RG}}
	\DeriveConclude{[*]}{\bd \TYPE{SetEq}[1]}{ R \sum_{g} G = \int^r_{RG} = \int^l_{RG}}
	\EndProof
	\\
	\Theorem{IntegralsOfFiniteGroupDualAlgebras}{  
		\forall R \in \ANN \.
		\forall G : \FG \. 
		\int^l_{RG^*} = \int^r_{RG^*} = R  \; \mathrm{d}e
	}
	\Assume{\alpha, \beta}{R}
	\Assume{g,h}{h}
	\Conclude{\Big[(g,h).*  ]}{\bd \FUNC{finiteDualAlgebra} \bd \FUNC{differential}\bd \FUNC{finiteDualAlgebra}}
	{
		(\beta \mathrm{d}g)(\alpha \mathrm{d}e)(h) =
		\alpha \beta  \mathrm{d}g(h)\mathrm{d}e(h) = 
		\alpha \beta * ( \If g == e == h \Then 1 \Else 0 ) =
		\eta(\beta \mathrm{d}g) \alpha \mathrm{d}e(h)
	}
	\Derive{[1]}{\bd \TYPE{Subset}}{  R \mathrm{d}e \subset  \int^l_{GR^*} }
	\Assume{f}{ \int^l_{GR^*}}
	\Say{[2]}{\bd \int^l_{GR^*}\bd RG*}{ (\mathrm{d}e \; f) = \eta(\mathrm{d}e)f = f }
	\Say{[3]}{\bd \FUNC{differential}\bd RG^*}{ (\mathrm{d}e \; f)(v) = f^e \mathrm{d}e  }
	\Conclude{[f.*]}{[2][3]}{f \in R \mathrm{d}e}
	\DeriveConclude{[*]}{\bd \TYPE{Subset}[1]\bd \TYPE{SetEq}}{ \int^l_{RG^*} = \int^r_{RG^*} = R \; \mathrm{d} e}
	\EndProof
	\\
}\Page{
	\DeclareFunc{dualComodule}{\prod k : \Field \. \prod A : \BIALG{k} \. \prod n \in \Nat \. \Basis(n,A) \to \RCOMOD{A}}
	\DefineNamedFunc{dualComodule}{e}{A^*_e}{\left(A^*, \Lambda f \in A^* \. \sum^n_{i=1} e^i f \otimes e_i \right)} 
	\Say{\Big(\alpha,[1]\Big)}
	{ \bd \Basis(n,A)(e)\Delta }
	{ \sum \alpha : n^3 \to k \. \forall i \in n \. \Delta(e_i) =  \alpha_{i,j,l} e_j \otimes e_l  }
	\Say{\Big(\beta,[2]\Big)}
	{ \bd \Basis(n,A)(e)\Delta^2 }
	{ \sum \alpha : n^4 \to k \. \forall i \in n \. \Delta^2(e_i) =  \beta_{i,j,l,t} e_i \otimes e_l \otimes e_t  }
	\Say{[3]}{\bd \COALG{k}(A)[1][2]}{ 
		\forall i,t \in n \.  
		\sum_{j,l=1}^n \beta_{i,j,l,t}e_j \otimes e_l = 
		\sum^n_{j=1} \alpha_{i,j,t} \Delta(e_j)    
	}
	\Assume{f}{A^*}
	\Assume{a}{A}
	\Conclude{[f.*]}
	{
		\bd \rho
		\bd \FUNC{tensorMap} \bd \rho
		\bd \FUNC{SweedlerNotation} \bd \FUNC{dualBasis}
		[1]
		[3]
		[2]
		\bd^{-1} \FUNC{tensorMap}
		\bd^{-1} \rho
	}
	{
		\NewLine : 
		a  (f \; \rho (\rho \otimes \id) ) = 
		a \left( \left( \sum^n_{i=1} e^i f \otimes e_i \right) (\rho \otimes \id)   \right) = 
		a  \; \sum_{i,j=1}^n  e^je^i f \otimes e_j \otimes e_i = 
		\sum_a \sum_{i,j=1}^n a^j_1 a^i_2 f(a_3)e_j \otimes e_i = \NewLine =
	        \sum_{i,j,l,t=1}^n f_l a^t \beta_{t,i,j,l} \alpha e_j \otimes e_i =
		\sum^n_{t,l,i=1} a^t f_l \alpha_{t,i,l}  \Delta(e_i) = 
		a \left(  \sum^n_{i=1} e^i f \otimes \Delta(e_i) \right) =  \NewLine = 
		a \left( \left(  \sum^n_{i=1} e^i f \otimes e_i \right) \id \otimes \Delta \right)  =
		a ( f \; \rho \; \id \otimes \Delta)
	}
	\Derive{[4]}{I(=,\to)}{  \rho (\id \otimes \Delta) = \rho  (\Delta \otimes \id)  }
	\Assume{f}{A^*}
	\Assume{a}{A}
	\Conclude{[f.*]}
	{
		\bd \rho
		\bd \FUNC{tensorMap} 
		\bd \FUNC{SweedlerNotation} 
		[1]
		\bd \LMOD{k}(A,k)(f)
		\bd \COALG{k}(A)
		\bd \Basis(n,A)
	}
	{
		\NewLine : 
		a  (f \; \rho (\id \otimes \eta) ) = 
		a \left( \left( \sum^n_{i=1} e^i f \otimes e_i \right) (\id \otimes \eta)   \right) = 
		\sum_a a^i_1 f(a_2) \eta(e_i) =
		\sum_{i,j,l=1}^n a^i f_l  \alpha_{i,j,l} \eta(e_j) = \NewLine = 
		\sum^n_{i=1} a^i f\left( \sum^n_{j,l=1} \alpha_{i,j,l}\eta(e_j)e_l  \right) =
		\sum^n_{i=1} a^i f(e_i)  =
		\sum^n_{i=1} a^i f_i =
		f(a)
	}
	\Derive{[5]}{I(=,\to)}{\rho(\id \otimes \eta) = \id}
	\Conclude{[*]}{\bd \RCOMOD{A}[5][4]}{A^* \in \RCOMOD{A}}
	\EndProof
	\\
	\DeclareType{HopfModule}{
		\prod R \in \ANN \. 
		\prod A \in \HOPF{R} \.
		?(\RCOMOD{A} \And \RAMOD{A})
	}
	\DefineType{M}{HopfModule}{\rho_M : M \Arrow{\RAMOD{A}} M \otimes A}
	\\
	\DeclareFunc{categoryOfHopfModule}
	{
		\prod R \in \ANN \. \HOPF{R}  \to \CAT
	}
	\DefineNamedFunc{categoryOfHopfModule}
	{ A }{\RHMOD{A}}{\Big( \TYPE{HopfModule}, \RCOMOD{A} \And \RAMOD{A}, \circ, \id  \Big)}
	\\
	\Theorem{MainTheoremOfHopfModules}
	{
		\forall R \in \ANN \. 
		\forall A \in \HOPF{R} \. 
		\forall M \in \RHMOD{A} \.
		M \cong_{\RHMOD{A}} W \otimes A
		\NewLine
		\quad \where \quad
		W = \{ m \in M : \rho(m) = m \otimes e_A  \}
	}
	\NoProof
}
\Page{
	\DeclareFunc{dualModuleOfHopfAlg}{\prod k : \Field \. \prod A : \HOPF{k} \.  \RAMOD{A}}
	\DefineNamedFunc{dualModuleOfHopfAlg}{}{A^*}{\Big(A^*, \Lambda f \in A^* \. \Lambda a \in A \. f \hitBy \sigma(a) \Big)} 
	\\
	\Theorem{AntiRightDualAlgebra}{
		\prod k : \Field \. \prod A : \HOPF{k} \. 
		\forall f,g \in A^* \. \forall a \in A \.
		(fg)a = \sum_a (fa_2)(ga_1) 
	}
	\Assume{x}{A}
	\Conclude{[x.*]}{  
		\bd \FUNC{dualModuleOfHopfAlgebra}
		\bd \FUNC{HitByAction}
		\bd \BIALG{R}(A) \NewLine
		\THM{ComultiplicationOfAntipode} 
		\bd \FUNC{dualAlgebra}
	}
	{
		\NewLine :
		x \; (fg)a = 
		x\big(fg \hitBy \sigma(a)\big) = 
		\sum_{y = x\sigma(a)} f(y_1)g(y_2) = 
		\sum_{x} \sum_{z = \sigma(a)} f(x_1z_1)g(x_2z_2) = \NewLine =
		\sum_{x} \sum_{a}   f\big(x_1\sigma(a_1)\big) g\big(x_2\sigma(a_2)\big)
		=  x\;\sum_a (fa_2)(ga_1)
	}
	\DeriveConclude{[*]}{I(=,\to)}{(fg)a = \sum_a (fa_2)(ga_1)}
	\EndProof
	\\
	\Theorem{dualAsHopfModule}
	{
		\forall k : \Field \.
		\forall A : \HOPF{k} \.
		\forall n \in \Nat \.
		\forall e : \Basis(n,A) \.
		A^*_e \in \RHMOD{A}
	}
	\Assume{f,g}{A^*}
	\Assume{a,x}{A}
	\Conclude{\Big[(f,g).*\Big]}{ 
		\bd  \COALG{k}(A)
		\bd \VS{k}(A^* \otimes A^*; A^*)(\mu_{A^*_e}) \bd \LALGE{k}(A)
		\bd \HOPF{k}(A)
		\bd \FUNC{dualAlgebra}(A) \NewLine 
		\bd \FUNC{dualModuleOfHoplfAlgebra}(A) 
		\THM{AntipodeAntihomo}(A)
		\THM{AntipodeComultiplication}(A)
		\bd \HOPF{k}(A) \NewLine
		\THM{AntipodeCounit}(A)
		\bd \HOPF{k}(A)
		\bd \FUNC{finiteDualAlgebra}(A)
		\NewLine
		\bd^{-1} \FUNC{dualModuleOfHoplfAlgebra}(A)
		\THM{AntiRightDualAlgebra}
		\bd \Basis(n,A)(e)
		\bd \VS{k}(A,A^*)\FUNC{eval} \NewLine
		\bd \FUNC{finiteDualAlgebra}(A)
	}
	{
		\NewLine
		x \; f(g a)  = 
		x \; f\left( g \sum_a  a_1 \eta(a_2) \right) =
		x \; \sum_a \big( f\eta(a_2)e_A \big) (ga_1) =
		x \; \sum_a \big(f \sigma(a_2) a_3 \big) (ga_1)= \NewLine
		\sum_x \sum_a  \Big( x_1 \; f \sigma(a_2) a_3 \Big) \big( x_2 \; ga_1\big)=
		\sum_x \sum_a  f \Big( x_1 \sigma\big( \sigma(a_2) a_3\big)\Big) g\big( x_2 \sigma(a_1) \big) = 
		 \NewLine = 
		\sum_x \sum_a  f \big( x_1 \sigma(a_3)\sigma^2(a_2) \big) g\big( x_2 \sigma(a_1) \big) = 
		\sum_x \sum_a  f \big( x_1 \sigma\eta(a_2) e) g\big( x_2 \sigma(a_1) \big) = \NewLine
		\sum_x \sum_a  f \big( x_1 \eta(a_2) e) g\big( x_2 \sigma(a_1) \big) =  
		\sum_x \sum_a  f \big( x_1 \sigma(a_2)a_3) g\big( x_2 \sigma(a_1) \big) = \NewLine =
		\sum_x \sum_a \sum_f  f_1\big(x_1 \sigma(a_2)\big)f_2(a_3)g\big( x_2 \sigma(a_1)  \big) = 
		\sum_x \sum_a \sum_f  f_2(a_3)  (f_1 a_2)(x_1) (g a_1)(x_2) = \NewLine = 
		 \sum_a \sum_f  f_2(a_3)  \Big( x \;  (f_1a_2)(ga_1) \Big) = 
		 \sum_a \sum_f   f_2(a_2)  \big(  x \; (f_1g) a_1 \big) = \NewLine = 
		 \sum_a \sum_f   f_2(a_2)  \left(  x \; \left( \sum^n_{i=1} f_1(e_i)e^i g \right) a_1 \right) =  
		 \sum_a \sum_f\sum^n_{i=1}    f_1(e_i)f_2(a_2)   (  x \;  e^i g  a_1  ) = \NewLine =  
		 \sum_a \sum^n_{i=1}  f(e_ia_2) ( x \; e^i g a_1 )  
	}
}\Page{
	\Derive{[1]}{I(\forall)I(=,\to)}{\forall f,g \in A^* \. \forall a \in A \.  f(ga) = \sum_a \sum^n_{i=1} f(e_ia_2)(e^i g a_1) }
	\Assume{g}{A^*}
	\Assume{a}{A}
	\Assume{i}{n}
	\Conclude{\Big[.*\Big]}
	{[1](e^i,g,a)\bd \FUNC{tensorProduct}}
	{   
		\NewLine :
		e^i(ga) \otimes e_i  =
		\left( \sum_a \sum^n_{j=1} e^i(e_ja_2)(e^j g a_1) \right) \otimes e_i =
		\sum_a \sum^n_{j=1}  (e^j g a_1)  \otimes e^i(e_ja_2)e_i =             
	}
	\Derive{[2]}{I(\forall)}{
		\forall g \in A^* \. 
		\forall a \in A \. 
		\forall i \in n \. 
		e^i(ga) \otimes e_i = \sum_a \sum^n_{j=1} (e^j g a_1) \otimes e^i(e_ja_2)e_i
	}
	\Assume{f}{A^*}
	\Assume{a}{A}
	\Conclude{[f.*]}
	{
		\bd \rho 
		\Big(\forall i \in n \. [2](f,a,i) \Big)
		\bd \FUNC{dualBasis} \bd \FUNC{tensorProduct}
		\bd \RAMOD{A}(A^* \otimes A)
		\bd^{-1} \rho
	}{
		\NewLine :
		\rho(fa) = 
		\sum^n_{i=1} e^i(fa) \otimes e_i =
		\sum^n_{i,j=1} \sum_a  (e^j f a_1) \otimes e^i(e_ja_2)e_i = 
		\sum^n_{j=1} \sum_a   (e^j  f a_1) \otimes  e_j a_2 = 
		\left( \sum^n_{j=1} e^j f \otimes e_j \right) a = \NewLine = 
		\rho(f)a
	}
	\DeriveConclude{[*]}{\bd \RHMOD{A}}{A^*_e \in \RHMOD{A}}
	\EndProof
	\\
	\Theorem{IntergralOfHopfDual}
	{
		\forall k : \Field \.
		\forall A \in \HOPF{k} \.
		\forall [0] : \dim A < \infty \.
		\int^l_{A^*} = \{ f \in A^* : \rho(f) = f \otimes e  \}
	}
	\Say{n}{\dim A}{\Nat}
	\Say{e}{\THM{BasisExists}}{\TYPE{Basis}(n,A)}
	\Say{W}{\{f \in A^* : \Delta(f) = f \otimes \eta_A \}}{?A^*}
	\Assume{f}{\int^l_{A^*}}
	\Say{[1]}{\bd \int^l_{A^*} \bd \FUNC{dualCoalgebra}(A) \bd \TYPE{Unimodul}(A^*)}{ \forall g \in A^* \. gf = \eta(g)f = g(e)f = fg}
	\Say{[2]}{ \bd \rho [1] \bd \FUNC{dualBasis} \bd \FUNC{tensorProduct}   }{
			\rho(f) = \sum^n_{i=1} e^i f \otimes e_i =
			\sum^n_{i=1} e^i(e) f \otimes e_i =
			\sum^n_{i=1} f \otimes e^i(e)e_i  = 
			 f \otimes e    
		}
	\Conclude{[f.*]}{\ByConstr W[2]}{f \in W}
	\Derive{[1]}{\bd \TYPE{Subset}}{\int^l_{A^*} \subset W}
	\Assume{f}{W}
	\Assume{g}{A^*}
	\Conclude{[g.*]}{ \bd \FUNC{dualBasis} \bd^{-1} \FUNC{tensorMap} \bd^{-1} \rho \ByConstr W \bd \FUNC{tensorMap} 
		\bd \FUNC{finiteDualAlgebra}}
	{   
		\NewLine :
		gf = 
		\sum^n_{i=1} g(e_i)e^i f  =  
		( \sum^n_{i=1} e^if \otimes e_i) (\id \otimes g) \mu =
		\rho(f) (\id \otimes g) \mu = 
		(f \otimes e) (\id \otimes g) \mu =
		g(e)f = 
		\eta(g)f
	}
	\DeriveConclude{[f.*]}{\bd \int^l_{A^*}}{f \in \int^l_{A^*}}
	\DeriveConclude{[*]}{[1]\bd \TYPE{SetEq}}{\int^l_{A^*} = W}
	\EndProof
}\Page{
	\DeclareFunc{antipodalAutoconvolution}{
		\prod R \in \ANN \. 
		\prod A \in \HOPF{k} \. 
		\prod M \in \RHMOD{A} \.
		M \Arrow{\VS{k}} M
	}
	\DefineNamedFunc{antipodalAutoconvolution}{m}{S(m)}{
		\sum_m  m_0 \sigma(a_1)
	}
	\\
	\Theorem{AutoconvolutionMultiplication}
	{
		\forall R \in \ANN \.
		\forall A \in \HOPF{R} \.
		\forall M \in \RHMOD{A} \.
		\forall f \in M \.
		\forall a \in A \. \NewLine \.  
		S(fa) = \eta(a)S(f)
	}
	\Conclude{[*]}{
		\bd S 
		\bd \RHMOD{A}(M)
		\THM{AntipodeAntihomo}(A)
		\bd \HOPF{R}(A)
		\bd \LALGE{R}(A)
		\bd^{-1} S
	}
	{
		 \NewLine :
		 S(fa)  =
		 \sum_{g = fa}  g_0 \sigma(g_1) =
		 \sum_{f,a}  f_0a_1 \sigma(f_1a_2) =
		 \sum_{f,a}  f_0 a_1 \sigma(a_2)\sigma(f_1) =
		 \sum_{f}    f_0 \eta(a)e \sigma(f_1) =
		 \eta(a) \sum_{f} f_0 \sigma(f_1) = \NewLine =
		 \eta(a) S(f)
	}
	\EndProof
	\\
	\Theorem{AutoconvolutionComultiplication}
	{
		\forall R \in \ANN \.
		\forall A \in \HOPF{R} \.
		\forall M \in \RHMOD{A} \.
		\forall f \in M \.
		\rho(S(f)) = S(f) \otimes e
	}
	\Conclude{[*]}{
		\bd S 
		\bd \LMOD{R}(M,M \otimes A)(\rho)
		\bd \RHMOD{A}(M)
		\bd \FUNC{SweedlerNotation}
		\bd \RHMOD{A}(M \otimes A)
		\bd \HOPF{R}(A) \NewLine
		\bd \FUNC{tensorProduct}
		\bd \RCOMOD{A}(M)
		\bd^{-1} S
	}
	{
		\NewLine :
		\rho(S(f)) = 
		\rho\left( \sum_f f_0 \sigma(f_1) \right) =
		\sum_f  \rho\big( f_0 \sigma(f_1) \big) = 
		\sum_f  \rho(f_0)\sigma(f_1) =
		\sum_f  (f_0 \otimes f_1)\sigma(f_2) = \NewLine = 
		\sum_f  f_0 \sigma(f_3) \otimes f_1\sigma(f_2) =
		\sum_f  f_0 \sigma(f_2) \otimes \eta(f_1)e  =
		\left( \sum_f f_0 \eta(f_1)\sigma(f_2) \right) \otimes e =
		\left(  \sum_f f_0 \sigma(f_1)  \right) \otimes e = \NewLine = 
		 S(f) \otimes e
	}
	\EndProof
	\\
	\DeclareFunc{integralHopfModule}{
		\prod k : \Field \. 
		\prod A \in \HOPF{k} \. 
		\prod n \in \Nat \. 
		\prod e : \Basis(n,A) \. 
		\RHMOD{A}
	}
	\DefineNamedFunc{integralHopfModule}{}{\int_{A^*}}{
		\left( 
			\int_{A^*} , 
			\Lambda f \in A^* \. 
			\lambda a \in A \. \eta(a)f
		\right)} 
	\\
	\Theorem{SweedlerLarsonTHM}
	{
		\forall k : \Field \.
		\forall A \in \HOPF{k} \.
		\forall n \in \Nat \.
		\forall e : \Basis(n,A) \.
		A^*_e  \cong_{\RHMOD{A}} \int_{A^*} \otimes A
	}   
	\Say{\varphi}{\bd \FUNC{tensorProduct} \Lambda f \in \int_{A^*} \. \Lambda a \in A \. f\cdot_{A^*}a}
	{ \int_{A^*} \otimes A \Arrow{\RHMOD{A}} A^*   }
	\Say{\psi}{ \rho(S \otimes \id)}{ A^* \Arrow{\RHMOD{A}} \int_{A^*} \otimes A   }
	\Assume{f}{\int_{A^*}}
	\Assume{a}{A}
	\Conclude{[f.*]}{ 
		\ByConstr \varphi
		\ByConstr \psi
		\bd \RHMOD{A}(A^*_e)
		\THM{IntegralOfHopfDual}(f)\bd\FUNC{integralHopfModule}
		\bd \FUNC{tensorMap} \NewLine
		\bd S \THM{IntegralOfHopfDual}
		\THM{AntipodeUnit}
	}
	{
		\NewLine : 
		(f \otimes a) \varphi \psi  = 
		\Big( f \cdot_{A^*} a \Big) \rho (S \otimes \id) = 
		\Big( \rho(f) a \Big) (S \otimes \id)  = 
		(f \otimes a)(S \otimes \id) = 
		S(f)  \otimes a = 
		f\sigma(e) \otimes a =
		f \otimes a
	}
	\Derive{[1]}{I(=,\to))}{\varphi \psi = \id}
}
\Page{
	\Assume{f}{A^*}
	\Conclude{[f.*]}
	{
		\ByConstr \psi \ByConstr \varphi 
		\bd \FUNC{SweedlerNotation}
		\bd \FUNC{tensorMap} \bd S
		\bd \HOPF{k}(A)
		\THM{AntipodeUnit}
		\bd \RCOMOD{A}(A^*_e)
	}
	{
		\NewLine :
		f  \psi \varphi = 
		f \rho (S \otimes \id) \mu  =
		\sum_f f_0 \otimes f_1 (S \otimes \id) \mu =
		\sum_f f_0 \sigma(f_1) f_2 = 
		\sum_f f_0  \eta(f_1) e  =
		\sum_f \eta(f_1) f_0 =
		f
	}
	\Derive{[2]}{I(=,\to)}{\psi \varphi  = \id}
	\Say{[3]}{\bd^{-1} \TYPE{Inverse} [3][2]}{\psi = \varphi^{-1}}
	\Conclude{[4]}{\bd^{-1} \TYPE{Isomorphic}}{ \int_{A^*_e} \otimes A \cong_{\RHMOD{A}} A^*_e }
	\EndProof
	\\
	\Theorem{IntegralsAreLines}{
		\forall k : \Field \.
		\forall A \in \HOPF{k} \.
		\forall n \in \Nat \.
		\forall e : \Basis(n,A) \.
		\dim \int_{A} = 1			
	}
	\Say{[1]}{\THM{DualDimendion}}{\dim A = \dim A^* }
	\Say{[2]}{\THM{SweedlerLarsonTHM}(A^*)\THM{FinitDimReflexive}}{A \cong_{\RHMOD{A}} \int_{A} \otimes A^*}
	\Say{[3]}{\THM{TensorProductDimension}}{ \dim A^* = \dim \int_{A} \dim A^*}
	\Conclude{[4]}{\THM{NeutralNaturalNumberisOne}[3]}{\dim \int_{A} = 1}
	\EndProof
	\\
	\\
	\Theorem{GeneratingIntegralsExists}{
		\forall R : \PID \.
		\forall A \in \HOPF{R} \.
		\forall [0] : \rank A < \infty \. \NewLine \.
		\exists \Lambda \in \int_A :
		\int_A = R\Lambda
	}
	\NoProof
}
\newpage
\subsection{Hopf Orders [!!]} 
\Page{
	\DeclareType{Order}{\prod R : \ID \. \prod G \in \GRP \. \TYPE{UnitalSubalgebra}\Big( \Frac(R)G \Big)}
	\DefineType{O}{Order}{ O : \TYPE{Projective} \And \FG(R) \And \exists E : \Basis\Big(\Frac(R)G\Big) : E \subset O}
	\\
	\Theorem{OrderRoots}{\forall R : \ID \. \forall G \in \GRP \. \forall O : \TYPE{Order}(R,G) \. \NewLine \. 
		\forall o \in O \.  \exists f(x) \in R[x] \.  f(o) = 0
	}
	\NoProof
	\\
	\Theorem{FreeOrder}{\forall R : \ID \And \TYPE{LocalIntegrallyClosed} \. \forall G \in \GRP \. \NewLine \.  
		\forall O  : \TYPE{Order}(R,G) \. O \cong_{\LMOD{R}} R^{|G|}
	}
	\\
	\DeclareType{HopfOrder}{\prod R : \ID \. \prod G \in \GRP \. ?\TYPE{Order}(R,G)}
	\DefineType{O}{HopfOrder}{\Delta(O) \subset O \times O}
}
\newpage
\subsection{Graded Duality }
\Page{
	\DeclareFunc{gradedDual}{\prod R \in \ANN \. \prod G : \TYPE{Monoid} \. \LMOD{R}(G) \Arrow{\CAT} \LMOD{R}^{o}(G) }
	\DefineNamedFunc{gradedDual}{ M  }{\mathfrak{D}(M)}{\bigoplus_{g \in G} M_g^*}
	\DefineNamedFunc{gradedDual}{A,B,\varphi}{\mathfrak{D}_{A,B}(\varphi)}{\bigoplus_{g \in G} \varphi_{|M_g}^*}
	\\
	\DeclareFunc{gradedDualAction}{
		\prod R \in \ANN \. 
		\prod G : \TYPE{Monoid} \. 
		\prod M : \LMOD{R}(G)  \. 
		\mathfrak{D}(M) \Arrow{\LMOD{R}} M^*
	}
	\DefineNamedFunc{gradedDualAction}{f,m}{f(m)}{\sum_{g \in G} f_g(m_g)}
	\\
	\Theorem{GradedDualAlgebra}
	{
		\forall R \in \ANN \. 
		\forall G : \TYPE{Monoid} \. 
		\forall A \in \COALG{R}(G)
		\left(  \mathfrak{D}(M), \mathfrak{D}(\Delta), \mathfrak{D}(\eta )\right) \in \LALGE{R}(G) 
	}
	\NoProof
	\\
	\DeclareFunc{gradedDualAlgebra}{\prod R \in \ANN \. \prod G : \TYPE{Monoid} \. \COALG{R}(G) \Arrow{\CAT} \LALGE{R}^{o}(G) }
	\DefineNamedFunc{gradedDualAlgebra}{ A  }{\mathfrak{D}(A)}{\Big(\mathfrak{D}(A),\mathfrak{D}(\Delta),\mathfrak{D}(\eta) \Big)}
	\DefineNamedFunc{gradedDualAlgebra}{A,B,\varphi}{\mathfrak{D}_{A,B}(\varphi)}{\mathfrak{D}(\varphi)}
	\\
	\Theorem{CoskewDualAlgebra}{\forall R \in \ANN \. \forall A \in \SCOALG{R} \. \mathfrak{D}(A) : \TYPE{SkewAlgebra}(A)}
	\NoProof
	\\
	\DeclareType{GradedHopfAlgebra}{\prod R \in \ANN \. \prod G : \TYPE{Monoid} \. 
		?\Big( \HOPF{R} \And  \big(\LALGE{R} \And \COALG{R}\big)(G) \Big) }
	\DefineType{A}{GradedHopfAlgebra}{\sigma_A : A \Arrow{\LMOD{R}(G)} A}
	\\
	\DeclareFunc{categoryOfGradedHopfAlgebras}{ 
		\ANN \to \TYPE{Monoid} \to \CAT  
	}
	\DefineNamedFunc{categoryOfGradedHopfAlgebras}{ R,G }{\HOPF{R}(G)}
	{ \NewLine \de \Big( \TYPE{GradedHopfAlgebra}, \HOPF{R} \cap \LMOD{R}(G),\circ,\id\Big)  }
	\\
	\DeclareType{TwistedHopfAlgebra}{\prod R \in \ANN \.  
		?\Big( \BIALG{R} \And  \big(\LALGE{R} \And \COALG{R}\big)(\Int) \Big) }
	\DefineType{A}{TwistedHopfAlgebra}{
		\Delta_A : A \Arrow{\LALGE{R}(\Int)} \Big(A \widetilde{\otimes} A\Big)
		\And
		\mu_A  :  \Big(A \widetilde{\otimes} A\Big) \Arrow{\COALG{R}(\Int)} A
        }
	\\
	\DeclareFunc{categoryOfTwistedHopfAlgebras}{ 
		\ANN  \to \CAT  
	}
	\DefineNamedFunc{categoryOfTwistedHopfAlgebras}{ R }{\widetilde{\HOPF{R}}}
	{ \NewLine \de \Big( \TYPE{TwistedHopfAlgebra}, \BIALG{R} \cap \LMOD{R}(G),\circ,\id\Big)  }
	\\
}
\Page{
	\Theorem{TensorProductOfHopfAlgebras}{ 
		\forall R \in \ANN \. 
		\forall n \in \Nat \.
		\forall A : n \to \HOPF{R}
		\left( \bigotimes^n_{i=1} A_i, \bigotimes^n_{i=1} \sigma_i  \right) \in \HOPF{R}
	}
	\Assume{a}{\prod^n_{i=1} A_i}
	\Say{[a.*.1]}{ 
		\bd \FUNC{tensorProductCoalgebra}(n,A) 
		\bd \FUNC{tensorMap}(n,\sigma) \bd \FUNC{tensorProductAlgebra}(n,A)
		\NewLine
		\bd \FUNC{tensorProduct}(n,A) \bd \HOPF{R}(A)
		\bd \FUNC{tensorProductAlgebra}(n,A) \bd \FUNC{tensorProductCoalgebra}(n,A)	
	}
	{
		\NewLine : 
		\bigotimes^n_{i=1} a_i \; \Delta  \left(\id \otimes  \bigotimes^n_{i=1} \sigma_{A_i} \right) \mu = 
		\sum_{a} \bigotimes^n_{i=1} a_{i,1} \otimes \bigotimes^n_{i=1} a_{i,2} 
		\left(\id \otimes \bigotimes^n_{i=1} \sigma_{A_i} \right) \mu =
		\sum_{a} \bigotimes^n_{i=1}  a_{i,1} \sigma_{A_i}(a_{i,2})  = \NewLine = 
		\bigotimes^n_{i=1} \eta_{A_i}(a_i)e_{A_i}  = 
		\prod^n_{i=1} \eta_{A_i}(a_i) \otimes^n_{i=1} e_{A_i} =
		\eta\left( \bigotimes^n_{i=1} a_i\right) e
	}
	\Say{[a.*.2]}{ 
		\bd \FUNC{tensorProductCoalgebra}(n,A) 
		\bd \FUNC{tensorMap}(n,\sigma) \bd \FUNC{tensorProductAlgebra}(n,A)
		\NewLine
		\bd \FUNC{tensorProduct}(n,A) \bd \HOPF{R}(A)
		\bd \FUNC{tensorProductAlgebra}(n,A) \bd \FUNC{tensorProductCoalgebra}(n,A)	
	}
	{
		\NewLine : 
		\bigotimes^n_{i=1} a_i \; \Delta  \left( \bigotimes^n_{i=1} \sigma_{A_i} \otimes \id \right) \mu = 
		\sum_{a} \bigotimes^n_{i=1} a_{i,1} \otimes \bigotimes^n_{i=1} a_{i,2} 
		\left(\bigotimes^n_{i=1} \sigma_{A_i} \otimes \id \right) \mu =
		\sum_{a} \bigotimes^n_{i=1}  \sigma_{A_i}(a_{i,1})  = \NewLine = 
		\bigotimes^n_{i=1} \eta_{A_i}(a_i)e_{A_i}  = 
		\prod^n_{i=1} \eta_{A_i}(a_i) \otimes^n_{i=1} e_{A_i} =
		\eta\left( \bigotimes^n_{i=1} a_i\right) e
	}
	\Derive{[*]}{\bd \FUNC{tensorProduct}\bd \HOPF{R}}{\LOGIC{This}}
	\EndProof
	\\
	\DeclareFunc{tensorProductOfHopfAlgebras}{\prod R \in \ANN \. \prod n \in \Nat \. (n \to \HOPF{R}) \to \HOPF{R}}
	\DefineNamedFunc{tensorProductOfHopfAlgebras}
	{A}{\bigotimes^n_{i=1}A_i}{\left( \bigotimes^n_{i=1} A_i, \bigotimes^n_{i=1} \sigma_i \right)}
	\\
	\Theorem{TwistedTensorProductOfTwistedHopfAlgebras}
	{
		\forall R \in \ANN \. 
		\forall A,B \in \widetilde{\HOPF{R}} \. 
		A \widetilde{\otimes} B \in \widetilde{\HOPF{R}}
	}
	\NoProof
	\\
	\DeclareFunc{twistedTensorProductOfHopfAlgebras}{\prod R \in \ANN \. \prod n \in \Nat \. 
	(n \to \widetilde{\HOPF{R}}) \to \widetilde{\HOPF{R}}}
	\DefineNamedFunc{twistedTensorProductOfHopfAlgebras}
	{A}{\widetilde{\bigotimes}^n_{i=1}A_i}{\widetilde{\bigotimes}^n_{i=1} A_i }
	\\
	\DeclareType{FiniteFreeComponents}{\prod R \in \ANN \. \prod G : \TYPE{Monoid} \. ?\LMOD{R}(G) }
	\DefineType{M}{FiniteFreeComponents}{\forall g \in G \. \exists n \in \Nat \. M_g \cong_{\LMOD{R}} R^n }
}\Page{
	\Theorem{GradedNaturalIsomorphism}
	{
		\forall R \in \ANN \.
		\forall G : \TYPE{Monoid} \.
		\forall M : \TYPE{FiniteFreeComponents}(R,G) \.
		\NewLine \. 
		\epsilon_M : M \ToIso{\LMOD{R}(G)} \mathfrak{D}^2(M) 
	}
	\Assume{m}{M}
	\Assume{g,h}{G}
	\Assume{f}{M^*_h}
	\Assume{[1]}{\epsilon(m)(f) \neq 0 }
	\Say{[2]}{[1] \bd \epsilon}{0 \neq \epsilon(m_g)(f) = f(m_g)}
	\Say{[3]}{\bd \FUNC{gradedDualAction} }{g = h}
	\Conclude{[m.*]}{\bd M^{**}_g [3] }{m_g \in M^{**}_g}
	\Derive{[1]}{\bd \LMOD{R}}{ \left( \epsilon : M \Arrow{\LMOD{R}(G)} \mathfrak{D}^2(M) \right)}
	\Assume{f}{\mathfrak{D}^2(M)}
	\Say{(m,[2])}{\forall\THM{NaturalIsomorphisTheorem}(f_g)}
	{\sum m : \prod_{g \in G} M_g \. \forall g \in G \. f_g = \epsilon(m_g)}
	\Say{[3]}{\bd \mathfrak{D}^2(M)[2]}{m \in M }
	\Conclude{[f.*]}{[2][3]}{f = \epsilon(m)}
	\DeriveConclude{[*]}{\bd^{-1}\TYPE{Surjective}\bd^{-1} \TYPE{Bijective}}{\LOGIC{This}}
	\EndProof
	\\
	\Theorem{TensorProductGradedDuality}
	{     
		\forall R \in \ANN \.
		\forall n \in \Nat \. 
		\forall M : n \to \TYPE{FiniteFreeComponents}(R,\Int) \. \NewLine \.
		\mathfrak{D}\left( \bigotimes^n_{i=1} M_i \right) = \bigotimes^n_{i=1} \mathfrak{D}(M_i)                                                      	}
	\NoProof
	\\
	\Theorem{GradedDualAlgebra}{
		\forall R \in \ANN \.
		\forall A \in \COALG{R}(\Int) \And \TYPE{FiniteFreeComponents}(R) \.
		\Big( \mathfrak{D}(A), \mathfrak{D}(\Delta) , \mathfrak{D}(\eta) \Big) : \LALGE{R}(\Int) 
	}
	\NoProof
	\\
	\DeclareFunc{gradedDualAlgebra}{
		\prod R \in \ANN \.
		\COALG{R}(\Int) \And \TYPE{FiniteFreeComponents}(R) \to
		\LALGE{R}(\Int)
	}
	\DefineFunc{gradedDualCoalgebra}{A}{\Big( \mathfrak{D}(A),\mathfrak{D}(\Delta),\mathfrak{D}(\eta) \Big)}
	\\
	\Theorem{GradedDualCoalgebra}{
		\forall R \in \ANN \.
		\forall A \in \LALGE{R}(\Int) \And \TYPE{FiniteFreeComponents}(R) \.
		\Big( \mathfrak{D}(A), \mathfrak{D}(\mu) , \mathfrak{D}(e) \Big) : \COALG{R}(\Int) 
	}
	\NoProof
	\\
	\DeclareFunc{gradedDualAlgebra}{
		\prod R \in \ANN \.
		\LALGE{R}(\Int) \And \TYPE{FiniteFreeComponents}(R) \to
		\COALG{R}(\Int)
	}
	\DefineFunc{gradedDualCoalgebra}{A}{\Big( \mathfrak{D}(A),\mathfrak{D}(\mu),\mathfrak{D}(e) \Big)}
}
\Page{
	\Theorem{GradedNaturalAlgebraIsomorphism}
	{
		\forall R \in \ANN \.
		\forall A : \LALGE{R}(\Int) \And \TYPE{FiniteFreeComponents} \.
		\NewLine \. 
		\epsilon_A : A \ToIso{\LALGE{R}(G)} \mathfrak{D}^2(A) 
	}
	\NoProof
	\\
	\Theorem{GradedNaturalCoalgebraIsomorphism}
	{
		\forall R \in \ANN \.
		\forall A : \COALG{R}(\Int) \And \TYPE{FiniteFreeComponents} \.
		\NewLine \. 
		\epsilon_A : A \ToIso{\COALG{R}(G)} \mathfrak{D}^2(A) 
	}
	\NoProof
	\\
	\Theorem{GradedDualHopfAlgebra}{
		\forall R \in \ANN \.
		\forall A \in \HOPF{R}(\Int) \And \TYPE{FiniteFreeComponents}(R) \.
		\NewLine \. \mathfrak{D}(A) : \HOPF{R}(\Int) 
	}
	\NoProof
	\\
	\DeclareFunc{gradedDualHopfAlgebra}{
		\prod R \in \ANN \.
		\HOPF{R}(\Int) \And \TYPE{FiniteFreeComponents}(R) \to
		\HOPF{R}(\Int)
	}
	\DefineFunc{gradedDualHopfAlgebra}{A}{ \mathfrak{D}(A) }
	\\
	\Theorem{GradedDualTwistedHopfAlgebra}{
		\forall R \in \ANN \.
		\forall A \in \widetilde{\HOPF{R}}(\Int) \And \TYPE{FiniteFreeComponents}(R) \.
		\NewLine \. \mathfrak{D}(A) : \widetilde{\HOPF{R}}(\Int) 
	}
	\NoProof
	\\
	\DeclareFunc{gradedDualAlgebra}{
		\prod R \in \ANN \.
		\widetilde{\HOPF{R}}(\Int) \And \TYPE{FiniteFreeComponents}(R) \to
		\widetilde{\HOPF{R}}(\Int)
	}
	\DefineFunc{gradedDualCoalgebra}{A}{\mathfrak{D}(A)}
}
\newpage
\section{Classical Clifford Algebras} 
\subsection{Clifford Structure}
\Page{
	\DeclareType{CliffordMap}{
		\prod k : \Field \. 
		\prod A \in \LALGE{k} \.
		\prod V : \OVS(k) \.
		?( V \Arrow{\VS{k}}  A)
	}
	\DefineType{\varphi}{CliffordMap}{ \forall x \in A \. (x \;\varphi)^2 = \langle x, x\rangle e   }
	\\
	\Theorem{CliffordMapProduct}{i
		\forall k : \Field \.
		\forall A \in \LALGE{k} \.
		\forall V : \OVS(k) \. \NewLine \. 
		\forall \varphi : \TYPE{CliffordMap} \.\
		\forall x,y \in A \.
		\varphi(x)\varphi(y) + \varphi(y)\varphi(x) = 2\langle x, y\rangle e
	}
	\Say{[1]}{ \bd \LALGE{R}[1]   }
	{
		\varphi^2(x)  + \varphi(x)\varphi(y) + \varphi(y)\varphi(x) + \varphi^2(y) =
		\varphi^2(x + y) = \langle x + y, x + y  \rangle e    = \NewLine =  
		\langle x, x \rangle e  + 2 \langle x, y \rangle e +   \langle y, y \rangle e = 
		\varphi^2(x)  + 2 \langle x, y \rangle e + \varphi^2(y)
	}
	\Conclude{[*]}{[1] - \varphi^2(x) - \varphi^2(y) }{ \varphi(x)\varphi(y) + \varphi(y)\varphi(x)  =  2 \langle x, y \rangle e } 
	\EndProof
	\\
	\DeclareType{CliffordAlgebra}{ \prod k : \Field \. \sum A \in \LALGE{k} \. \sum V : \OVS(k) \. \TYPE{CliffordMap}(A,V) }
	\DefineType{(A,V,\mathbf{i})}{CliffordAlgebra}{ 
		\Big\langle \mathbf{i}(V) \Big\rangle_{\LALGE{k}} = A 
		\And \forall B \in \LALGE{k} \. \varphi : \TYPE{CliffordMap}(B,V) \.  \NewLine \. 
		\exists f : A \Arrow{\LALGE{k}} B \. \mathbf{i} f = \varphi
	}
	\\
	\DeclareFunc{categoryOfClifford}
	{
		\Field  \to \CAT
	}
	\DefineNamedFunc{categoryOfClifford}{k}{\CLIF{k}}
	{\Big(\TYPE{CliffordAlgebra},(\VS{k},\LALGE{k}),(\circ,\circ^o),(\id,\id)\Big)}
	\\
	\DeclareFunc{complexCliffordAlgebra}{\CLIF{\Reals}}
	\DefineNamedFunc{complexCliffordAlgebra}{}{\Complex_\Reals}{
		\Big( 
			(\Reals ,\Lambda a,b \in \Reals \. -ab) , 
			\Complex, \Lambda a \in \Reals \. a\mathrm{i} 
		\Big)
	}
	\Assume{a}{\Reals}
	\Conclude{[a.*]}{\bd \mathbf{i}\bd \Complex }{ \mathbf{i}^2(a) = (a\mathrm{i})^2 = -a^2 = \langle a, a\rangle 1  }
	\Derive{[1]}{\bd^{-1}\TYPE{CliffordMap}}{\Big( \mathbf{i} : \TYPE{CliffordMap}(\Reals;\Reals,\Complex) \Big)}
	\Assume{A}{\LALGE{\Reals}}
	\Assume{T}{\TYPE{CliffordMap}(\Reals;\Reals;A)}
	\Say{\psi}{\lambda a + b\mathrm{i} \in \Complex \. ae_A + T(b)}{\Complex \Arrow{\VS{k}} A }
	\Assume{ a + b\mathrm{i},a' + b'\mathrm{i}  }{\Complex}
	\Conclude{[T.*]}{\bd \Complex \ByConstr \psi \bd \VS{\Reals}(T)\bd \TYPE{CliffordMap}(\Reals)(\Reals,A)(T)\ByConstr^{-1}\psi}
	{
		\NewLine : 
		\psi\Big( (a + b\mathrm{i})(a' + b'\mathrm{i})  \Big) =
		\psi\Big( aa' - bb' + (a'b + ab' )\mathrm{i}  \Big) = 
		(aa' - bb')e_A  + T(a'b + ab')  = \NewLine = 
		aa'e_A  +  bb'T^2(1)  + a'T(b) + aT(b') =  
		aa'e_A  +  T(b)T(b')  + a'T(b) + aT(b') = 	
		(ae_A + T(b))(a'e_A  + T(b')  ) = \NewLine = 
		\psi\Big( a + b\mathrm{i}  \Big)\psi(a' + b'\mathrm{i})
	}
	\DeriveConclude{[*]}{\bd \CLIF{\Reals}}{\Complex_\Reals \in \CLIF{\Reals}}
	\EndProof
}
\Page{
	\DeclareFunc{realQuaternionCliffordAlgebra}{\CLIF{\Reals}}
	\DefineNamedFunc{realQueternionCliffordAlgebra}{}{\Quat_\Reals}{
		\NewLine \Bigg( 
			\bigg(\Reals^2 , \FUNC{quadraticByMatrix}\Big( (e_1,e_1)\mapsto -1, (e_2,e_2) \mapsto -1, 
			 (e_1,e_2) \mapsto 0, (e_2,e_1) \mapsto 0  \Big) \bigg) , 
			\Quat, \NewLine \quad\quad \bd \Basis(2,\Reals^2)(e)\Big( e_1 \mapsto \mathrm{i}, e_2 \mapsto \mathrm{j}  \Big)
		\Bigg)
	}
	\Assume{a,b}{\Reals}
	\Conclude{\Big[(a,b).*\Big]}{ \bd \mathbf{i}\bd \Quat}
	{
		\mathbf{i}^2(a,b) =
		(a\mathrm{i} + b\mathrm{j})^2 =
		-a^2 - b^2 + ab\mathrm{k} - ab\mathrm{k} = 
		\Big\langle (a,b),(a,b) \Big\rangle 1
	}
	\Derive{ [1] }{\bd \TYPE{CliffordMap}}{\Big( \mathbf{i} : \TYPE{CliffordMap}(\Reals)(\Reals^2,\Quat)\Big)} 
	\Assume{A}{\LALGE{\Reals}}
	\Assume{T}{\TYPE{CliffordMap}(\Reals^2,A)}
	\Say{\psi}{\Lambda a + b\mathrm{i} + c\mathrm{j} + d\mathrm{k} \in \Quat \. a e_A + b T(e_1) + c T(e_2) + d T(e_1)T(e_2)}
	{\Quat \Arrow{\VS{\Reals}} \Reals^2 }
	\Assume{a + b\mathrm{i} + c\mathrm{j} + d\mathrm{k},a' + b'\mathrm{i} + c'\mathrm{j} + d'\mathrm{k}}{\Quat}
	\Say{[1]}{\THM{CliffordMapProduct}(\mathbf{i})(e_1,e_2)}{T(e_1)T(e_2) = - T(e_2)T(e_1)}
	\Conclude{[A.*]}{\bd \Quat \ByConstr \psi [1]\bd \TYPE{CliffoeMap}(T)\bd \LALGE{R}(A)\ByConstr^{-1}\psi }
	{
		\NewLine =
		\psi\Big( (a + b\mathrm{i} + c\mathrm{j} + d\mathrm{k})(a' + b'\mathrm{i} + c'\mathrm{j} +d'\mathrm{k} ) \Big) =
		\NewLine =
		\psi\Big( (aa' -bb' - cc' -dd') +(ab' + ba' +cd' -dc')\mathrm{i}   
			+ (ac' + ca' -bd' + db')\mathrm{j} + (ad' + da' + bc' - cb')\mathrm{k}
		\Big)= \NewLine = 
		(aa' -bb' - cc' -dd')e_A +(ab' + ba' -cd' +dc')T(e_1) + \NewLine \quad 
			+ (ac' + ca' -bd' + db')T(e_2) + (ad' + da' + bc' - cb')T(e_1)T(e_2) = \NewLine =
		aa' + bb'T^2(e_1) + cc' T^2(e_1) + dd'\big(T(e_1)T(e_2)\big)^2 +
		(ab' + ba')T(e_1)  + cd'T(e_2)T(e_1)T(e_2) + \NewLine + dc'T(e_1)T^2(e_2) +  
		(ac' + ca')T(e_2)  + bd' T^2(e_1)T(e_3) + db' T(e_1)T(e_2)T(e_1) + 
		(ad' + da')T(e_1)T(e_2) +  \NewLine + bc' T(e_1)T(e_2) + cb'T(e_1)T(e_2) = \NewLine =  
		\Big(a + bT(e_1) + cT(e_2) + dT(e_1)T(e_2)\Big)\Big(a' + b'T(e_1) + c'T(e_2) + d'T(e_1)T(e_2)\Big) = \NewLine =
		\psi\Big( a + b\mathrm{i} + c\mathrm{j} + d\mathrm{k}\Big)\psi(a' + b'\mathrm{i} +c'\mathrm{j} +d'\mathrm{k})
	}
	\DeriveConclude{[*]}{\bd \CLIF{\Reals}}{\Quat_\Reals \in \CLIF{\Reals}}
	\EndProof
	\\
	\Theorem{CliffordUniversalProperty}
	{
		\forall k : \Field \.
		\forall (V,C,\mathbf{i}) \in \CLIF{k} \.
		\forall A \in \LALGE{k} \. \NewLine \. 
		\forall T : \TYPE{CliffordMap}(k)(V,A) \.
		\exists! f : C \Arrow{\LALGE{k}} A : \mathbf{i}f = T
	}
	\Assume{g}{C \Arrow{\LALGE{k}} A}
	\Assume{[1]}{\mathbf{i}g = T}
	\Assume{y}{C}
	\Say{\Big(n,m,x,[2] \Big)}{\bd \CLIF{k}(V,C,\mathbf{i})}
	{
		\sum n \in \Nat \. 
		\sum m : n \to \Nat \.
		\sum x :  \prod^n_{i=1} V^{m_i} \.
		y = \sum^n_{i=1} \prod^{m_i}_{j=1} \mathbf{i}(x_{i,j})
	}
	\Conclude{[y.*]}{ [2]\bd \LALGE{k}(g)[1]\bd \CLIF{k} \bd \LALGE{k}(f)[1]  }
	{
		\NewLine : 
		g(y) =
		g\left( \sum^n_{i=1} \prod^{m_i}_{j=1} \mathbf{i}(x_{i,j}) \right) =
		\sum^n_{i=1} \prod^{m_i}_{j=1} \mathbf{i}g(x_{i,j}) =
		\sum^n_{i=1} \prod^{m_i}_{j=1}  T(x_{i,j}) =
		\sum^n_{i=1} \prod^{m_i}_{j=1} \mathbf{i}f(x_{i,j}) =
		f\left( \sum^n_{i=1} \prod^{m_i}_{j=1} \mathbf{i}(x_{i,j}) \right) =
		f(y)
	}
	\DeriveConclude{[g.*]}{I(=,\to)}{f=g}
	\DeriveConclude{[*]}{\bd^{-1} \LOGIC{Unique}}{\LOGIC{This}}
	\EndProof
}
\Page{
	\Theorem{CliffordAlgebraIsUnique}
	{
		\forall V : \OVS{k} \. 
		\forall A, B \in \LALGE{k} \.
		\forall f : \TYPE{CliffordMap}(V,A) \. \NewLine \.
		\forall g : \TYPE{CliffordMap}(V,B) \.
		(V,A,g),(V,B,f) \in \CLIF{k} \Rightarrow
		(V,A,g) \cong_{\CLIF{k}} (V,B,F) 
	}
	\NoProof
	\\
	\DeclareFunc{algebraOfClifford}{ 
		\prod k : \Field \. 
		\OVS(k) \to  \CLIF{k}
	}
	\DefineNamedFunc{algebraOfClifford}{V}{\mathrm{CL}(V)}
	{\left( V, \frac{V^\otimes}{I}, \iota_\otimes \pi_i  \right) \quad \where \quad 
		I = \FUNC{ideal}\{ x \otimes x - \langle x, x \rangle | x \in V \}}
	\\
	\Theorem{CliffordAlgebraOfDegenerateSpace}
	{
		\forall k : \Field \. 
		\forall V \in \VS{k} \. 
		\mathrm{CL}(V,0) = V^\wedge
	}
	\NoProof
	\\
	\Theorem{InjectiveCliffordMap}
	{
		\forall k : \Field \.
		\forall (V,C,\mathbf{i}) \in \CLIF{k} \. 
		\mathbf{i} : V \ToInj C
	}
	\Say{U}{\ker \langle \circ,\circ \rangle}{\TYPE{VectorSubspace}(V)}
	\Say{(W,[1])}{\THM{OrthogonalStructure}(V)}{\sum W : \TYPE{VectorSubspace}(V) \. V = W \bot U } 
	\Say{(V,A,\mathbf{i}_W)}{\mathcal{CL}(W)}{\CLIF{k}}
	\Say{(V,U^\wedge,\mathbf{i}_U)}{\mathcal{CL}(U)}{\CLIF{k}}
	\Say{\phi}{ \mathbf{i}_W \otimes 1 + e_A \otimes \mathbf{i}_U }{V \Arrow{\VS{k}} A \widetilde{\otimes} U^\wedge}
	\Assume{v}{V}
	\Say{\Big(u,w,[2]\Big)}{[1](v)}{\sum u \in U \. \sum w \in W \. w + u = v}
	\Conclude{[*]}{   
		[2]\bd \LALGE{A \widetilde{\otimes} U^\wedge \bd \VS{k}(V,A \otimes U^\wedge)
		\ByConstr \phi \bd \OVS{k}[1] [2]
	}
	}{
		\phi^2(v) =
		\phi^2(u + w) = 
		\phi^2(u) + \phi^2(w) + \phi(u)\phi(w) + \phi(w)\phi(u) = 
		-\langle u, u \rangle e_A \otimes 1 + \langle w, w \rangle e_A \otimes 1 +
		\mathbf{i}_U(u) \otimes \mathbf{i}_V(w) - \mathbf{i}_U(u) \otimes \mathbf{i}_V(w) =
		\langle u,u \rangle e_A \otimes 1 +
		2\langle u,w \rangle e_A \otimes 1 +
		\langle w,w \rangle  e_A \otimes 1    =
		\langle u + w, u + w \rangle e_A \otimes 1
		\langle  v, v \rangle e_A \otimes 1
	}
	\Derive{[2]}{\bd^{-1}\TYPE{CliffordMap}}{\Big( \phi : \TYPE{CliffordMap}(k)(V,A \otimes U^\wedge)\Big)}
	\Say{\Big(f,[3] \Big)}{\bd \TYPE{CliffordAlgebra}(k)(V,C,\mathbf{i})}
	{
		\sum f : C \Arrow A \widetilde{\otimes} U^\wedge \.
		\mathbb{i}f = \phi
	}
	\Say{[4]}{\ByConstr \varphi \THM{CliffordAlgebraOfDegenerateSpace}(U)}
	{ \Big( \phi : C \ToInj A \widetilde{\otimes} U^\wedge \Big)  }
	\Conclude{[*]}{\THM{InjectiveByComposition}[4]}
	{
		\LOGIC{This}
	}
	\EndProof
	\\
	\DeclareFunc{functorOfClifford}{\prod k : \Field \. k\hyph\mathsf{OVS} \Arrow{\CAT} \CLIF{k}}
	\DefineNamedFunc{functorOfClifford}{V}{\mathrm{CL}(V) }{\mathrm{CL}(V)}
	\DefineNamedFunc{functorOfClifford}{V,W,T}{\mathrm{CL}_{V,W}(T)}
	{  \Big(T,\bd \CLIF{k}\big(\mathrm{CL}(V)\big)(T\mathbf{i}_W)\Big)}
}
\subsection{Natural Involutions}
\Page{	
	\DeclareFunc{dwgreeInvolution}{
		\prod k : \Field \. \prod V : \OVS{k}
		\mathrm{CL}(V) \Arrow{\CLIF{k}} \mathrm{CL}(V)
	}
	\DefineNamedFunc{degreeInvolution}{}{\omega_V}{\mathrm{CL}_{V,V}(-\id)} 
	\\
	\Theorem{DegreeInvolutionIsInvolution}
	{
		\forall k : \Field \.
		\forall V : \OVS{k} \.
		\omega_V^2 = \id
	}
	\NoProof
	\\
	\DeclareFunc{partZero}{
		\prod k : \Field \. \prod V : \OVS{k} \.
		\TYPE{VectorSubspace}(\mathrm{CL}(V))
	}
	\DefineNamedFunc{partZero}{}{\mathrm{CL}_0(V)}{\ker(\omega_V - \id)}
	\\
	\DeclareFunc{partOne}{
		\prod k : \Field \. \prod V : \OVS{k} \.
		\TYPE{VectorSubspace}(\mathrm{CL}(V))
	}
	\DefineNamedFunc{partOne}{}{\mathrm{CL}_1(V)}{\ker(\omega_V + \id)}
	\\
	\Theorem{InvolutionaryDecomposition}
	{
		\forall k : \Type{NonBinary} \.
		\forall V : \OVS{k} \.
		\mathrm{CL}(V) = \mathrm{CL}_0(V) \oplus \mathrm{CL}_1(V)
	}
	\Assume{y}{\im(\omega_V - \id)} 
	\Say{\Big(x, [2] \Big)}{\bd \FUNC{image}}
	{ \sum x \in V \. y = (\omega_V - \id)x}
	\Conclude{[y.*]}{[2]\bd \omega_V[2]}
	{
		(\omega_V - \id)y = 
		(\omega_V - \id)^2 x =
		2(\id - \omega) x = -2y
	}
	\Derive{[2]}{\bd \ker}{\ker (\omega_V - \id) \cap \im (\omega_V - \id) = \{0\}}
	\Say{[3]}{\THM{DegreeInvolutionIsInvolution}(V)}{(\id + \omega_V)(\id - \omega_V) = \id - \id = 0}
	\Say{[4]}{ \bd^{-1} \ker \bd^{-1} \im [3]}{\im (\omega_V + \id) \subset \ker (\omega_V -\id)}
	\Say{[5]}{[4][2]}{ \im (\omega V + \id) \cap \im (\omega V - \id)}
	\Assume{x}{\ker (\omega_V -\id)}
	\Conclude{[x.*]}{ \bd \VS{k}(\mathrm{CL}(x))\bd x \bd \VS{k}(\omega_V + \id)   }
	{
		x = \frac{1}{2}(\omega_V + \id)x - \frac{1}{2}(\omega_V - \id)x 
		= \frac{1}{2}(\omega_V + \id)x = (\omega_V +\id)\left(\frac{1}{2}x\right)
	}
	\Derive{[6]}{I(\forall)\bd^{-1}\FUNC{image}\bd^{-1}\TYPE{Subset}}{\im (\omega_V + \id) = \ker (\omega_V - \id)} 
	\Assume{x}{\ker (\omega_V + \id)}
	\Conclude{[x.*]}{ \bd \VS{k}(\mathrm{CL}(x))\bd x \bd \VS{k}(\omega_V + \id)   }
	{
		x = \frac{1}{2}(\omega_V + \id)x - \frac{1}{2}(\omega_V - \id)x  
		= \frac{1}{2}(\omega_V + \id)x = \NewLine = (\omega_V +\id)\left(-\frac{1}{2}x\right)
	}
	\Derive{[7]}{2[4] - x}{\im (\omega_V - \id) = \ker (\omega_V - \id)} 
	\Conclude{[*]}{\bd \TYPE{DirectSum}[5][6][7]}
	{ \mathrm{CL}(V) = \mathrm{CL}_0(V) \oplus \mathrm{CL}_1(V)} 
	\EndProof
}\Page{
	\Theorem{ZeroPartProduct}
	{
		\forall k : \Field \. 
		\forall V : \OVS(k) \.
		\mathrm{CL}_0(V)\mathrm{CL}_0(V) \subset \mathrm{CL}_0(V)
	}
	\Assume{x,y}{\mathrm{CL}_0(V)}
	\Say{[1]}{\bd \mathrm{CL}_0(V)\bd(x,y)}
	{
		x,y \in \ker(\omega_V - \id)
	}
	\Say{[2]}{\bd \VS{k}\Big(\mathrm{CL}(V)\Big)\bd \LALGE{k}(\mathrm{CL}(V))[1]}
	{
		\NewLine :
		(\omega_V - \id)(xy) = 
		\omega_V(xy) - xy =
		\omega_V(x)\omega_V(y) - x(\omega_V(y) - y) + (\omega_V(x) - x)y - xy =
		\omega_V(x)\omega_V(y) - x\omega_V(y) + \omega_V(x)y - xy =
		(\omega_V(x) - x)(\omega_V(y) + y) = 0
	}
	\Say{[3]}{\bd \ker [2]}{xy \in \ker (\omega_V - \id)}
	\Conclude{\Big[(x,y).*\Big]}{\bd \mathrm{CL}_0(V)[3]}{xy \in \mathrm{CL}_0(V)}
	\DeriveConclude{[*]}{I(\forall)I\TYPE{Subset}}{ \mathrm{CL}_0(V)\CL_0(V) \subset \CL_0(V)}
	\EndProof
	\\
	\Theorem{ZeroPartOnePartProduct}
	{
		\forall k : \Field \. 
		\forall V : \OVS(k) \.
		\mathrm{CL}_1(V)\mathrm{CL}_0(V) \subset \mathrm{CL}_1(V)
	}
	\NoProof
	\\
	\Theorem{OnePartZeroPartProduct}
	{
		\forall k : \Field \. 
		\forall V : \OVS(k) \.
		\mathrm{CL}_0(V)\mathrm{CL}_1(V) \subset \mathrm{CL}_1(V)
	}
	\NoProof
	\\
	\Theorem{OnePartProduct}
	{
		\forall k : \Field \. 
		\forall V : \OVS(k) \.
		\mathrm{CL}_1(V)\mathrm{CL}_1(V) \subset \mathrm{CL}_0(V)
	}
	\NoProof
	\\
	\Theorem{DegreeGradingOfCliffordAlgebra}
	{
		\forall k : \Field \.
		\forall V : \OVS(k) \. \NewLine \. 
		\Big(\CL(V),\mathbf{F}_2,(0\mapsto\CL_0(V),1\mapsto\CL_0(V)) \Big)
		\in \LALGE{k}(\mathbf{F}_2)
	}
	\NoProof
	\\
	\Theorem{ZeroPartStructure}
	{
		\forall k : \TYPE{NonBinary} \.
		\forall V : \OVS(k) \. \NewLine \.
		\CL_0(V) = \Span\left\{ \prod^{2n}_{i=1} v_i \bigg| n \in \Int_+, v : 2n \to V  \right\}
	}
	\NoProof
	\\
	\Theorem{OnePartStructure}
	{
		\forall k : \TYPE{NonBinary} \.
		\forall V : \OVS(k) \. \NewLine \.
		\CL_0(V) = \Span\left\{ \prod^{2n + 1}_{i=1} v_i \bigg| n \in \Int_+, v : (2n + 1) \to V  \right\}
	}
	\NoProof
}\Page{
	\Theorem{CliffordAlgebraDirectDecomposition}
	{
		\forall k : \TYPE{NonBinary} \.
		\forall A,B : \OVS(k) \. \NewLine \.  
		\CL(A \oplus B) \cong_{\LALGE{k}(\mathbb{F}_2)} \CL(A) \widetilde{\otimes} \CL(B)
	}
	\Say{\varphi}{\bd \FUNC{tensorProduct}\Lambda a \in \CL(A) \. \Lambda b \in \CL(B) \CL_{A,A\oplus B}(\iota_A)(a)\CL_{B,A\oplus B}(b)}
	{ \NewLine : \CL(A)  \widetilde{\otimes} \CL(B) \Arrow{\VS{k}} \CL(A \oplus B) }
	\Assume{a}{A}
	\Assume{b}{B}
	\Say{[1]}{\bd \CLIF{k} \THM{CliffordMapProduct} \bd \FUNC{sumInnerProduct}}
	{ 
		\NewLine :
		\CL_{A,A \oplus B}(\iota_A)(\mathbf{i}_A(a))  
		\CL_{B,A \oplus B}(\iota_B)(\mathbf{i}_B(b))
		+
		\CL_{B,A \oplus B}(\iota_B)(\mathbf{i}_B(b))  
		\CL_{B,A \oplus A}(\iota_A)(\mathbf{i}_A(a)) = \NewLine = 
		(a \iota_A \mathbf{i}_{A \oplus B})
		(b \iota_B \mathbf{i}_{A \oplus B}) +
		(b \iota_B \mathbf{i}_{A \oplus B})
		(a \iota_A \mathbf{i}_{A \oplus B})=
		2 \langle (a,0),(0,b) \rangle e =
		0
	}
	\Conclude{\Big[(a,b).*\Big]}{\bd \LALGE{k}\CL(A \oplus B)[1]}
	{  
		\NewLine :
		\Big(a \mathbf{i}_A \CL_{A,A \oplus B}(\iota_A)\Big)  
		\Big(b \mathbf{i}_B \CL_{B,A \oplus B}(\iota_B)\Big) 
		=   
		-\Big( b \mathbf{i}_B \CL_{B, A \oplus B}(\iota_B)\Big)
		 \Big( a \mathbf{i}_A \CL_{A,A \oplus B}(\iota_A) \Big)
	}
	\Derive{[1]}
	{
		I^2(\forall)
	}
	{
		\forall a \in A \. \forall b \in B \. \NewLine \. 
		\Big(a \mathbf{i}_A \CL_{A,A \oplus B}(\iota_A)\Big)  
		\Big(b \mathbf{i}_B \CL_{B,A \oplus B}(\iota_B)\Big) 
		=   
		-\Big( b \mathbf{i}_B \CL_{B, A \oplus B}(\iota_B)\Big)
		 \Big( a \mathbf{i}_A \CL_{A,A \oplus B}(\iota_A) \Big)
	}
	\Assume{n,n',m,m'}{\Int_+}
	\Assume{a}{n \to A}
	\Assume{b}{m \to B}
	\Assume{a'}{n' \to A}
	\Assume{b'}{m' \to B}
	\Conclude{[\ldots.*]}
	{
		\bd \FUNC{skewTensorProduct}
		\ByConstr \varphi
		\bd \LALGE{k}(\CL(A),\CL(A\oplus B))\CL(\iota_A) \NewLine
		\bd \LALGE{k}(\CL(B),\CL(A\oplus B))\CL(\iota_B)
		[1]
		\ByConstr^{-1}\varphi
	}
	{
		\NewLine :
		\varphi\left( \left(\prod^n_{i=1} a_i \; \mathbf{i}_A \otimes \prod^m_{i=1} b_i \; \mathbf{i}_B \right)
		\left(\prod^{n'}_{i=1} a'_i \; \mathbf{i}_A \otimes \prod^{m'} b'_i \; \mathbf{i}_B  \right) \right) =  
		(-1)^{ mn'}\varphi\left( \prod^n_{i=1} a_i \; \mathbf{i}_A \prod^{n'}_{i=1} a'_i \; \mathbf{i}_A 
			\otimes \prod^m_{i=1} b_i \; \mathbf{i}_B \prod^{m'}b'_i \; \mathbf{i}_B\right)=
		\NewLine = 
		(-1)^{mn'} \CL_{A,A\oplus B}(\iota_A)\left(\prod^n_{i=1} a_i \; \mathbf{i}_A \prod^{n'}_{i=1} a_i' \; \mathbf{i}_A\right)
		\CL_{B,A\oplus B}(\iota_B)\left(\prod^m_{i=1} b_i \; \mathbf{i}_B \prod^{m'}_{i=1} b'_i \; \mathbf{i}_B\right) = \NewLine = 
		(-1)^{n'm} \prod^n_{i=1}  a_i \; \mathbf{i}_A \; \CL_{A,A\oplus B}(\iota_A) 
		\prod^{n'}_{i=1} a'_i \;\mathbf{i}_A \; \CL_{A,A\oplus B}(\iota_A)
		\prod^m_{i=1} b_i \;\mathbf{i}_B \; \CL_{B,A\oplus B}(\iota_B) 
		\prod^{m'}_{i=1} b'_i \; \mathbf{i}_B \; \CL_{B,A\oplus B}(\iota_B) = \NewLine = 	
		\prod^n_{i=1}  a_i \; \mathbf{i}_A \; \CL_{A,A\oplus B}(\iota_A) 
		\prod^{m}_{i=1} b_i \;\mathbf{i}_B \; \CL_{B,A\oplus B}(\iota_B)
		\prod^{n'}_{i=1} a_i' \;\mathbf{i}_A \; \CL_{A,A\oplus B}(\iota_A) 
		\prod^{m'}_{i=1} b'_i \; \mathbf{i}_B \; \CL_{B,A\oplus B}(\iota_B) = \NewLine = 
		\varphi \left( \prod^n_{i=1} a_i \mathbf{i}_A \otimes \prod^m_{i=1} b_i \mathbf{i}_B   \right)
		\varphi \left( \prod^{n'}_{i=1} a'_i \mathbf{i}_A \otimes \prod^{m'}_{i=1} b'_i \mathbf{i}_B \right)
	}
	\Derive{[2]}{\bd \CLIF{k}}{\Big( \varphi : \CL(A) \widetilde{\otimes} \CL(B) \Arrow{\LALGE{k}(\mathbb{F}_2)} \CL(A \oplus B) \Big)  }
	\Say{\psi}{\bd \LALGE{k} \Lambda a \in A \. \Lambda b \in B \. 
		(a \; \mathbf{i}_A) \otimes e_{\CL(B)} + e_{\CL(A)} \otimes (b \; \mathbf{i}_B) }
	{
		\CL(A \oplus B) \Arrow{\LALGE{R}(\mathbb{F}_0)} \CL(A) \widetilde{\otimes} \CL(B)
	}
	\Assume{a}{A}
	\Assume{b}{B}
	\Say{[a.*.1]}{\ByConstr \psi \ByConstr \varphi \bd \FUNC{functorOfClifford} \bd \LALGE{k}(\CL(A \oplus B)) \bd \FUNC{directSum}} 
	{ 
		\NewLine :
		(a,b) \; \mathbf{i}_{A \oplus B} \; \psi \; \varphi =
		\Big( (a \; \mathbf{i}_A) \otimes e_B +  e_A \otimes (b \; \mathbf{i}_B) \Big) \; \varphi = \NewLine =  
		\Big(a \; \mathbf{i}_A \CL_{A,A \oplus B}(\iota_A)( e_{\CL(B)} \CL(B,A \oplus B)(\iota_B) \Big) +
		\Big(e_{\CL(A)} \CL(A,A \oplus B)(\iota_B)(b \; \mathbf{i}_B \CL_{B,A \oplus B}(\iota_B)\Big) = \NewLine =  
		( a \; \iota_A \; \mathbf{i}_{A \oplus B} )e_{\CL(A \oplus B} + e_{\CL(A \oplus B)} (b \; \iota_B \; \mathbf{i}_{A \oplus B}) =
		(a,b) \; \mathbf{i}_{A \oplus B} 
	}
}\Page{
	\Say{[a.*.2]}{\ByConstr \varphi \bd \FUNC{functorOfClifford} \ByConstr \psi}
	{
		\NewLine :
		(a \; \mathbf{i}_A) \otimes e_{\CL(B)} \; \varphi \; \psi = 
		\Big( a \; \mathbf{i}_A \; \CL_{A,A \oplus B }(\iota_A)\Big)e_{\CL(A \oplus B)} \; \varphi =
		(a, 0) \; \mathbf{i}_{A \oplus B} \; \varphi =  
		(a \; \mathbf{i}_A) \otimes e_{\CL(B)}
	}
	\Conclude{[a.*.3]}{\ByConstr \varphi \bd \FUNC{functorOfClifford} \ByConstr \psi}
	{
		\NewLine :
		e_{\CL(A)} \otimes (b \; \mathbf{i}_B)  \; \varphi \; \psi = 
		e_{\CL(A \oplus B)} \Big(b \; \mathbf{i}_A  \; \CL_{A,A \oplus B }(\iota_A)\Big) \; \varphi =
		(0, b) \; \mathbf{i}_{A \oplus B} \; \varphi =  
		e_{\CL(A)} \otimes (b \; \mathbf{i}_B)
	}
	\DeriveConclude{[*]}{\bd \TYPE{Generating}\bd^{-1} \TYPE{Inverse}\bd^{-1} \TYPE{Isomorphic}}{\LOGIC{This}}
	\EndProof
	\\
	\Theorem{CliffordsFunctorPreservesMonomorphisms} 
	{
		\forall k : \TYPE{NonBinary} \.
		\forall V,W : \OVS{k} \. \NewLine \. 
		\forall T : \TYPE{Isometry}(V,W) \.
		T : V \ToInj W \Rightarrow \CL_{V,W}(T) : \CL(V) \ToInj \CL(W)
	}
	\Say{U}{T(V)}{\TYPE{VectorSubspace}}
	\Say{\Big(H, [1] \Big)}{ \THM{OrthogonalDecomposition}(W,U)}
	{
		\sum H \subvec{k} W \. W = U \bot H
	}
	\Say{[2]}{\THM{CliffordAlgebraDirectDecomposition}[1]}
	{
		\CL(W) \cong_{\LALGE{k}} \CL(U) \widetilde{\otimes} \CL(H)
	}
	\Say{\varphi }
	{
		\bd \TYPE{Isomorphic}[2]
	}
	{
		\CL(U) \widetilde{\otimes} \CL(H) \ToIso{\LALGE{k}} \CL(W)
	}
	\Say{[2]}{\ByConstr \varphi}
	{
		T (\mathbf{i}_U \otimes e_H) \varphi = \CL_{V,W}(T)
	}
	\Conclude{[4]}{\THM{InjectiveCompositon}[4]}{\Big(\CL_{V,W}(T) : \CL(V) \ToInj \CL(W) \Big) }
	\EndProof
	\\
	\Theorem{CliffordsFunctorPreservesEpimorphisms} 
	{
		\forall k : \TYPE{NonBinary} \.
		\forall V,W : \OVS(k) \. \NewLine \. 
		\forall T : \TYPE{Isometry}(V,W) \.
		T : V \ToSurj W \Rightarrow \CL_{V,W}(T) : \CL(V) \ToSurj   \CL(W)
	}
	\NoProof
	\\
	\DeclareFunc{semiconjugation}
	{
		\prod k : \Field \. 
		\prod  V : \OVS(k) \.
		\CL(V) \Arrow{\LALGE{k}(\mathbb{F}_0)} \CL^\op(V)
	}
	\DefineNamedFunc{semiconjugation}{}{S_V}{\bd \CLIF{k} \mathbf{i}^{\CL^\op(V)} }
	\\
	\Theorem{SemiconjugationIsInvolution}{
		\forall k : \Field \. 
		\forall V : \OVS{k} \.
		S^2_V = \id
	}
	\NoProof
	\\
	\Theorem{SemiconjugationPreservesCliffordMap}{
		\forall k : \Field \.
		\forall V : \OVS(k) \.
		\mathbf{i}_V \; S_V = \mathbf{i}_V
	}
	\NoProof
	\\
	\Theorem{SemiconjugationPreservesCommutesWithDegreeInvolutiom}{
		\forall k : \Field \. \NewLine \. 
		\forall V : \OVS(k) \.
		\omega_V \; S_V = S_V \; \omega_V
	}
	\NoProof
}\Page{
	\DeclareFunc{conjugation}{ 
		\prod k : \Field \. 
		\prod V : \OVS(k) \.
		\CL(V) \Arrow{\VS{k}} \CL(V)
	}
	\DefineNamedFunc{conjugataion}{x}{\overline{x}}{x \; \omega_V \; S_V}
	\\
	\Theorem{CliffordMapConjugation}
	{
		\forall k : \Field \.
		\forall V : \OVS(k) \. 
		\forall v \in V \.
		\overline{v \; \mathbb{i}_V} = -(v \; \mathbb{i}_V)
	}
	\NoProof
	\\
	\Theorem{ProductConjugation}
	{
		\forall k : \Field \.
		\forall V : \OVS(k) \. 
		\forall a,b \in \CL(V)
		\overline{ ab } = \overline{b}\overline{a}
	}
	\NoProof
}
\subsection{Clifford Algebras over Finite-Dimensional Vector Spaces}
\Page{
	\Theorem{CliffordAlgebraDimension1}
	{
		\forall k : \Field \.
		\forall V : \OVS(k) \.
		\NewLine 
		\dim V = 1 \Imply
		\dim \CL(V) = 2
	}
	\Say{\Big(v,[1]\Big)}{\bd \dim V}{\sum v \in V \. v \neq 0}
	\Say{A}{\Span(e,v\;\mathbf{i})}{  \VS{k}  }
	\Say{[1]}{\bd \TYPE{CliffordMap}(\mathbf{i}) \bd A }{ A \in \LALGE{k}  }
	\Say{[2]}{\ByConstr A \bd \FUNC{span} \ByConstr \FUNC{functorOfClifford}}{\dim A = 2}
	\Say{\Big( \varphi, [3]\Big)}{\bd \CLIF{A}}{
		\sum \varphi : \CL(V) \Arrow{\LALGE{k}} A \. \mathbb{i}_A \varphi = \iota_A
	}
	\Say{[4]}{\bd \TYPE{Monomorphism}[3](\mathbf{i}_A)}{\Big(\varphi : \CL(A) \ToInj A  \Big)}
	\Say{[5]}{\ByConstr A \bd \LALGE{k}(\varphi)}{\Big( \varphi : \CL(A) \ToSurj A  \Big)}
	\Say{[6]}{ \bd \TYPE{Isomorphic}\bd \TYPE{Isomorphism}[4][5]}{ A \cong_{\LALGE{k}} \CL(A)}
	\Conclude{[7]}{[2][6]}{ \dim \CL(V) = 2}
	\EndProof
	\\
	\Theorem{CliffordAlgebraDimension}
	{
		\forall k : \Field  \.
		\forall V : \OVS(k) \.
		\forall n \in \Nat  \.
		\NewLine 
		\dim V = n \Imply
		\dim \CL(V) = 2^n
	}
	\Say{\Big[ e, [1] \Big]}{ \THM{OrthogonalBasisExists} (V)   }
	{ \sum e : \Basis(n,V) \. \forall i,j \in n \. i \neq j \Rightarrow \langle e_i, e_j \rangle = 0 }
	\Say{U}{\Lambda i \in n \. \Span(e_i)}{ \sum^n_{i=1} \TYPE{VectorSubspace}(V)  }
	\Say{[2]}{\THM{CliffordAlgebraDirectDecomposition}(U)}
	{  \CL(V)  \cong_{\LALGE{k}} \widetilde{\bigotimes}^n_{i=1} \CL(U_i)   }
	\Say{[3]}{\bd \FUNC{Span} \bd^{-1} \dim }{\forall i \in n \. \dim U_i = 1}
	\Say{[4]}{\THM{CliffordAlgebraDimension1}[3]}{\forall i \in n \. \dim \CL(U_i) = 2 }
	\Conclude{[5]}{[2][4]}{ \dim \CL(V) = 2^n  }
	\EndProof
	\\
	\Theorem{CliffordAlgebraBasis}
	{
		\forall k : \Field \.
		\forall V : \OVS{k} \.
		\forall n \in \Nat \.
		\forall x : \Basis(n,V) \. \NewLine \.
		\Big( \prod^n_{i=1} v_{i,\alpha_i}  \Big)_{\alpha : n \to \mathbb{B}} 
		\quad \where \quad
		v : 
			\Lambda i \in n \. 
			\Lambda b \in \mathbb{B} \. 
			\If b == 0 \Then x_i \Else e
	}
	\NoProof
	\\
	\DeclareFunc{alternatingIsomorphism}
	{
		\prod k : \TYPE{Numeric} \.
		\prod V : \OVS(k) \And \FDVS{k} \. \NewLine \. 
		V^\wedge \ToIso{\VS{k}} \CL(V) 
	}
	\DefineNamedFunc{alternatingIsomorphism}{}{\xi_V}{
		\bd \FUNC{alternatingAlgebra} 
		\Lambda n \in \Nat \.
		\Lambda v : n \to V \.
		\frac{1}{n!} \sum_{\sigma \in S_n} (-1)^\sigma \prod^n_{i=1} x_{\sigma(i)} \; \mathbf{i}_{V}
	}
	\\
}\Page{
	\DeclareFunc{determinantElement}{
		\prod k : \TYPE{Numeric} \. 
		\prod V : \OVS(k) \And \FDVS{k} \. \NewLine \.
		\prod x : \TYPE{OrthogonalBasis}(V) \.
		\CL(V)
	}
	\DefineNamedFunc{determinantElement}
	{ }
	{x_\Delta}{\prod^{\dim V}_{i=1} x_i \; \mathbf{i}_{V}}
	\\
	\DeclareFunc{determinantScalar}{
		\prod k : \TYPE{Numeric} \. 
		\prod V : \OVS(k) \And \FDVS{k} \.
		\NewLine \. 
		\prod x : \TYPE{OrthogonalBasis}(V) \.
		\CL(V)
	}
	\DefineNamedFunc{determinantScalar}
	{ }
	{\Delta(x)}{\prod^{\dim V}_{i=1} \langle x_i,x_j \rangle}
	\\
	\Theorem{determinantProduct}
	{
		\forall k : \TYPE{Numeric} \.
		\forall V : \OVS(k) \And \FDVS{k} \.
		\NewLine \.
		\forall x : \TYPE{OrthogonalBasis}(V) \.
		x^2_\Delta = (-1)^{\frac{n(n-1)}{2}} \Delta(x) e
		\quad \where \quad n = \dim V
	}
	\NoProof
	\\
	\Theorem{NondegenerateByDeterminantElement}
	{
		\forall k : \TYPE{Numeric} \.
		\forall V : \OVS{k} \And \FDVS{k} \.\NewLine \.
		\forall x : \TYPE{OrthogonalBasis}(V)
		V : \TYPE{Nondegenerate}(k) \iff x_\Delta : \TYPE{Invertible}\Big(\CL(V)\Big)
	}
	\NoProof
	\\
	\Theorem{DegenerateDeterminantElement}
	{
		\forall k : \TYPE{Numeric} \.
		\forall V : \OVS{k} \And \FDVS{k} \.\NewLine \.
		\forall x : \TYPE{OrthogonalBasis}(V) \.
		V : \TYPE{Degenerate}(k) \Imply  x_\Delta^2 = 0
	}
	\NoProof
	\\
	\Theorem{DeterminantElementTransposition1}
	{
		\forall k : \TYPE{Numeric} \.
		\forall V : \OVS{k} \And \FDVS{k} \.\NewLine \.
		\forall x : \TYPE{OrthogonalBasis}(V) \.
		\forall v \in V \.
		x_\Delta (v \; \mathbf{i}_V) = (-1)^{1 - \dim V} (v \; \mathbf{i}_V) x_\Delta
	}
	\NoProof
	\\
	\Theorem{DeterminantElementTransposition2}
	{
		\forall k : \TYPE{Numeric} \.
		\forall V : \OVS{k} \And \FDVS{k} \.\NewLine \.
		\forall x : \TYPE{OrthogonalBasis}(V) \.
		\forall a \in  \CL(V) \.
		x_\Delta a = \omega^{ (\dim V) - 1}(a) x_\Delta
	}
	\NoProof
}\Page{
	\Theorem{CenterIsGradedSubalgebra}
	{
		\forall k : \TYPE{NonBinary} \.
		\forall V : \OVS{k} \And \FDVS{k} \. \NewLine \.
		Z\Big( \CL(V) \Big) \in \LALGE{k}(\mathbb{F}_2)
	}
	\NoProof
	\\
	\Theorem{NondegenerateByDeterminantElement}
	{
		\forall k : \TYPE{Numeric} \.
		\forall V : \OVS{k} \And \FDVS{k} \.\NewLine \.
		\forall x : \TYPE{OrthogonalBasis}(V) \.
		\dim V : \TYPE{Odd} \iff x_\Delta \in Z\Big( \CL(V) \Big)
	}
	\NoProof
	\\
	\Theorem{TrivialAnticentre}
	{
		\forall k : \TYPE{Numeric} \.
		\forall V : \TYPE{Nondegenerate} \And \FDVS{k} \. 
		\mathrm{AZ}_1\Big( \CL(V) \Big) = \{0\} 
	}
	\NoProof
	\\
	\Theorem{LinearCentre}
	{
		\forall k : \TYPE{Numeric} \.
		\forall V : \TYPE{Nondegenerate} \And \FDVS{k} \. 
		\mathrm{Z}_0\Big( \CL(V) \Big) = k e 
	}
	\NoProof
	\\
	\Theorem{OddDimensionalCentreStructure}
	{
		\forall k : \TYPE{Numeric} \.
		\forall V : \TYPE{Nondegenerate}(k) \And \FDVS{k} \.
		\NewLine \.
		\forall x : \TYPE{OrthogonalBasis}(V) \.
		\dim V : \TYPE{Odd} \Imply 
		Z\Big( \CL(V) \Big) = ke + k x_\Delta
	}
	\NoProof
	\\
	\Theorem{OddDimensionalAnticentreStructure}
	{
		\forall k : \TYPE{Numeric} \.
		\forall V : \TYPE{Nondegenerate}(k) \And \FDVS{k} \.
		\NewLine \.
		\dim V : \TYPE{Odd} \Imply 
		AZ\Big( \CL(V) \Big) = 0 
	}
	\NoProof
	\\
	\Theorem{EvenDimensionalCentreStructure}
	{
		\forall k : \TYPE{Numeric} \.
		\forall V : \TYPE{Nondegenerate}(k) \And \FDVS{k} \.
		\NewLine \.
		\dim V : \TYPE{Odd} \Imply 
		Z\Big( \CL(V) \Big) = ke
	}
	\NoProof
}\Page{
	\Theorem{EvenDimensionalAnticentreStructure}
	{
		\forall k : \TYPE{Numeric} \.
		\forall V : \TYPE{Nondegenerate}(k) \And \FDVS{k} \.
		\NewLine \.
		\forall x : \TYPE{OrthogonalBasis}(V) \.
		\dim V : \TYPE{Odd} \Imply 
		AZ\Big( \CL(V) \Big) = k x_\Delta
	}
	\NoProof
	\\
	\DeclareFunc{inverseCliffordAlgebra}
	{
		\prod k : \Field \.
		\OVS(k) \to \LALGE{k} 
	}
	\DefineNamedFunc{inverseCliffordAlgebra}
	{V}{\CL(-V)}{\CL\Big(\big(V,-\langle\cdot,\cdot\rangle_V\big)\Big)}
	\\
	\DeclareFunc{inverseDeterminantScalar}
	{
		\prod k : \Field \.
		\prod V : \OVS{k} \And \FDVS{k} \. \NewLine \. 
		\TYPE{OrthogonalBasis}(V) \to k
	}
	\DefineNamedFunc{inverseDeterminantScalar}
	{x}{\Delta^-(x)}{(-1)^{\dim V} \Delta(x)}
	\\
	\Theorem{InverseCliffordAlgebraIsomorphism}
	{
		\forall k : \TYPE{Numeric} \.
		\forall V : \OVS(k) \. \NewLine \.
		\dim V : \TYPE{Even} \Imply
		\CL(V) \cong_{\LALGE{k}(\mathbb{F}_2)} \CL(-V)
	}
	\NoProof
	\\
	\DeclareFunc{crossDualSpace}
	{
		\prod k : \TYPE{Numeric} \.
		\VS{k} \to \OVS(k)
	}
	\DefineNamedFunc{crossDualSpace}{V}
	{ V^{*,*} }{ \Big( V \oplus V^*, \Lambda (v,f),(w,g) \in V^* \. \frac{1}{2}\big(g(v) + f(w)\big)   \Big) }
	\\
	\Theorem{CrossDualSpaceIsExteriorOperators}
	{
		\forall k : \TYPE{Numeric} \.
		\forall V : \FDVS{k} \.
		V^{*,*} \cong_{\LALGE{k}} \VS{k}(V^\wedge,V^\wedge)  
	}
	\NoProof
	\\
	\Theorem{GeneratorsOfExteriorOperators}
	{
		\forall k : \Field \.
		\forall V : \FDVS{k} \. \NewLine \. 
		\VS{k}(V^\wedge,V^\wedge) = \Big \langle \big\{ \rho_v \big| v \in V  \big\} 
			\And \big\{ \sigma_v \big| f \in V^*  \big\}  \Big\rangle_{\LALGE{k}}
	}
	\NoProof
	\\
	\Theorem{CliffordExteriorOperatorsIsomorphismCriterion}
	{
		\forall k : \Field \.
		\forall V : \FDVS{k} \. \NewLine \.
		\forall n \in \Nat \. 
		\forall [0] : \dim V = 2n \.
		\forall \omega : \TYPE{involution}(V) \.
		\forall [00] \. \omega^\top = - \omega \. \NewLine \. 
		\CL(V) \cong_{\LALGE{k}} \VS{k}\Big({\ker}^\wedge(\omega - \id), {\ker}^\wedge(\omega - \id) \Big)
	}
	\NoProof
}\Page{
	\DeclareFunc{naturalProjection}{ 
		\prod k \in \Field \.
		\prod V \in \OVS(k) \.
		\CL(V) \Arrow{\VS{k}} k
	}        
	\DefineNamedFunc{naturalProjection}{}{\pi_V}{\xi_V^{-1}\pi_0}
	\\
	\DeclareFunc{naturalCliffordForm}{ 
		\prod k \in \Field \.
		\prod V \in \OVS(k) \.
		\L\Big(\CL(V),\CL(V);k) 
	}        
	\DefineNamedFunc{naturalCliffordForm}{a,b}{Q_V(a,b)}{\pi_V(ab)}
	\\
	\DeclareFunc{specialCategoryOfOrthogonalVectorSpaces}{\TYPE{Numperic} \to \CAT }
	\DefineNamedFunc{specialCategoryOfOrthogonalVectorSpaces}
	{k}{k\hyph\mathsf{SOVS}}
	{\NewLine \de \bigg(  
		\sum V : \OVS(k) \. \TYPE{VectorSubspaces}(k), 
		\Lambda (V,A),(W,B) \in k\hyph\mathsf{SOVS} \., \NewLine 
		,\sum f : \TYPE{Isometry}(V,W) \. f(V) \subset B, \circ, \id
		\bigg)
	}
	\\
	\DeclareFunc{forgetfulCliffordFunctor}{\prod k : \TYPE{Numeric} \. \CLIF{k}  \Arrow{\CAT} k\hyph\mathsf{SOVS} } 
	\DefineNamedFunc{forgetfulCliffordFunctor}{V,A,\mathbf{i}}{U^{\CLIF{k}}(V,A,\mathbf{i})}{(A,Q_V)} 
	\DefineNamedFunc{forgetfulCliffordFunctor}{(V,A,\mathbf{i}),(W,B,\mathbf{j}),(T,\varphi)}
	{U_{(V,A,\mathbf{i}),(W,B,\mathbf{j})}^{\CLIF{k}}(T,\varphi)  }{ \varphi \; \xi_W^{-1} \; \pi_1   }
	\Assume{x,a}{A}
	\Say{[1]}{\bd \xi_V \bd \TYPE{Isometry}(T) \bd \varphi}
	{   (xa) \; \xi_V^{-1} \; T^\wedge  =   ( xa ) \; \varphi \; \xi_W^{-1} }
	\Conclude{\Big[(x,a).*\Big]}{ \bd Q_V \bd \pi_V \bd \FUNC{exteriotMap}(T) [1] \bd^{-1} \pi_W \bd^{-1} Q_W  }{ 
			\NewLine :
			Q_V(x,a)
			(xa) \; \pi_V = 0
			(xa) \; \xi_V^{-1} \pi_0 =  
			(xa) \; \xi_V^{-1} \; T^\wedge \pi_0 = 
			(xa) \; \varphi \; \xi_W^{-1} \pi_0 =
			\varphi(x) \varphi(a) \; \pi_W = 
			Q_W\Big(\varphi(x),\varphi(a)\Big)
	}
	\DeriveConclude{[*]}{\bd^{-1}\TYPE{Isometrty}}{\bigg( \varphi : \TYPE{Isometry}\Big( (A,Q_V), (B,Q_W) \Big) \bigg)}
	\EndProof
}\Page{
	\Theorem{CliffordAdjoint}{ 
		\forall k : \TYPE{Numeric} \.
		\left( \CL, U^{\CLIF{k}}  \right) : \TYPE{Adjoint}( \OVS(k), \CLIF{k} )
	}
	\Assume{V}{\OVS(k)}
	\Assume{\Big(W,A,\mathbf{i}\Big)}{\CLIF{k}}
	\Assume{(T,\varphi)}{ \CL(V) \Arrow{\CLIF{k}} (W,A,\mathbf{i}) }
	\Say{F(T,\varphi)}{T \mathbf{i}}{V \Arrow{\VS{k}} A  }
	\Assume{v,v'}{V}
	\Conclude{\Big[(v,v').*\Big]}{ 
		\ByConstr F(T,\varphi)  
		\THM{VectorElementNaturalCliffordMap}
		\bd \TYPE{Isometry}(V,W)(T)
	} 
	{
		Q_W\Big( v \; F(T,\varphi) ,  v' \; F(T,\varphi)    \Big) =
		Q_W\Big( v \; T \; \mathbf{i}, v' \; T \; \mathbf{i}    \Big) =
		\langle v \; T,  v' \; T  \rangle_W =
		\langle v, v' \rangle
	}
	\DeriveConclude{\Big[(T,\varphi).*\Big]}{ \bd^{-1}\TYPE{Isometry}}
	{   F(T,\varphi) : \TYPE{Isometry}\Big( V, (A,Q_W)  \Big)   }
	\Derive{F}{I(\to)}{ \CLIF{k}\Big( \CL(V), (A,Q_W) \Big) \to \TYPE{Isometry}\Big(V, (A,Q_W) \Big)   }
	\Assume{T}{\TYPE{Isometry}\Big(V, (A,Q_W) \Big)}
	\Say{S}{T\xi^{-1}_W \pi_1}{ V \Arrow{\VS{k}} W }                  
	\Assume{x,y}{V}
	\Conclude{     }{}
	{
		\langle Sx, Sy \rangle_W =
		\langle  x \; T \; \xi^{-1}_W \; \pi_1 , y \; T \; \xi^{-1}_W \; \pi_1 \rangle_W =
		Q_V\Big( x \; T \; \xi^{-1}_W \; \pi_1 \; \mathbf{i}_W, y \; T \; \xi^{-1}_W \; \pi_1 \; \mathbf{i}_W \Big) = \NewLine =
		Q_V\Big( x \; T \; \xi^{-1}_W \;  \mathbf{i}_W^\wedge \; \pi_1, y \; T \; \xi^{-1}_W \;  \mathbf{i}_W^\wedge \; \pi_1 \Big) = 
		\bigg(\Big( x \; T \xi^{-1}_W \;  \mathbf{i}_W^\wedge \; \pi_1 \Big) 
		      \Big( y \; T \xi^{-1}_W \;  \mathbf{i}_W^\wedge \; \pi_1 \Big) \bigg) \pi_W =  \NewLine = 
		\bigg(\Big( x \; T \xi^{-1}_V \;  \mathbf{i}_W^\wedge \; \pi_1 \Big) 
		      \Big( y \; T \xi^{-1}_V \;  \mathbf{i}_W^\wedge \; \pi_1 \Big) \bigg) \xi^{-1}_W  \pi_0 =
		 \Big( (w_i \mathbf{i}_W)(  w_j'\mathbf{i}_W) \Big) \xi^{-1}_W \pi_0 =
		\langle e_k, e_k \rangle w_{i,k} w'_{j,k} =
		\langle x, y \rangle
	}
	\NoProof
}
\subsection{Towards Low-Dimensional Classification}
\Page{ 
	\Theorem{AllFDComplexCliffordAlgebrasAreIsomorphic}
	{
		\forall A,B \in \CLIF{\Complex} \.
		\forall [0] : \dim A < \infty \. \NewLine \.
		\forall [00] : \dim A = \dim B \.  
		A \cong_{\CLIF{\Complex}} B
	}
	\NoProof
	\\
	\DeclareFunc{signatureCliffordAlgebra}{ 
		(\Int_+ \times \Int_+) \to \CLIF{\Reals} 
	}
	\DefineNamedFunc{signatureCliffordAlgebra}{p,q}
	{ \CL(p, q) }{ \CL\Big( \Reals^p \oplus \Reals^q, Q_e(I) \oplus Q_e(-I) \Big)   }
	\\
	\DeclareFunc{positiveCliffordAlgebra}{\Int_+ \to \CLIF{\Reals}}
	\DefineNamedFunc{positiveCliffordAlgebra}{n}{\CL_n(+)}{\CL(n,0)}
	\\
	\DeclareFunc{negativeCliffordAlgebra}{\Int_+ \to \CLIF{\Reals}}
	\DefineNamedFunc{negativeCliffordAlgebra}{n}{\CL_n(-)}{\CL(0,n)}
	\\
	\Theorem{PositiveDoubleStepTheorem}
	{
		\forall p,q \in \Int_+ \.
		\CL(p,q) \otimes \CL_2(+) \cong_{\CLIF{\Reals}} \CL(p+2,q)
	}
	\NoProof
	\\
	\Theorem{PositiveDoubleStepTheorem}
	{
		\forall p,q \in \Int_+ \.
		\CL(p,q) \otimes \CL_2(-) \cong_{\CLIF{\Reals}} \CL(p,q+2)
	}
	\NoProof
	\\
	\Theorem{QuarticEquivalence}
	{
		\forall p,q \in \Int_+ \.
		p - q =_{Z_4} 0 \Rightarrow \CL(p,q) \cong_{\CLIF{\Reals}} \CL(q,p)
	}
	\NoProof
	\\
	\Theorem{ZeroSignatureStructure}
	{
		\forall p \in \Int_+ \.
		\CL(p,p) \cong_{\LALGE{\Reals}} \VS{\Reals}\Big( \Reals^{p\wedge}, \Reals^{p\wedge} \Big)
	}
	\NoProof
	\\
	\DeclareFunc{quaternionicIsomorphism4}
	{
		\Quat \otimes \Quat \Arrow{\VS{\Reals}} \VS{\Reals}\Big( \Quat, \Quat \Big)
	}
	\DefineNamedFunc{quaternionicIsomorphism_4}{}{\Lambda t \in \Quat \otimes \Quat \. T_t}
	{ \bd \TYPE{tensorProduct} \Lambda a,b,x \in \Quat \. a x \overline{b}  }
	\\
	\Theorem{QuaternionicIsomorphis}
	{
		T : \CL_4(-) \ToIso{\LALGE{\Reals}} \VS{\Reals}(\Reals^4,\Reals^4)
	}
	\NoProof
}
\subsection{Representation of Clifford Algebras}
\Page{
	\Theorem{NondegenerateRepresentationIsInjective}
	{
		\forall k : \Field \.
		\forall V : \TYPE{Nondegenerate}(k) \. \NewLine \. 
		\forall W \in \FDVS{k} \. 
		\forall \Big( \CL(V), W,\rho \Big) : \mathsf{AR}(k) \.
		\rho : \CL(V) \ToInj \L(W;W)
	}
	\NoProof
	\\
	\DeclareType{Orthogonal}
	{
		\prod k : \Field \.
		\prod V : \OVS(k) \.
		\prod W : \IPS(k) \. \NewLine \. 
		? \TYPE{Representation}\Big(\CL(V),W\Big)
	}
	\DefineType{\rho}{Orthogonal}
	{
		\exists \sigma \in  \{-1,+1\} :
		\forall x \in V \. \forall a,b \in W \.
		\Big\langle \rho(x)a, \rho(x)b  \Big\rangle =
		\sigma \langle x, x \rangle\langle a, b \rangle
	}
	\\
	\DeclareFunc{signOfOrthogonal}
	{
		\prod k : \Field \.
		\prod V : \OVS(k) \.
		\prod W : \IPS(k) \.
		\NewLine
		\TYPE{Orthogonal}(V,W)  \to \{-1,+1\}
	}
	\DefineNamedFunc{signOfOrthogonal}
	{\rho}{\sigma(\rho)}{\bd \TYPE{Orthogonal}}
	\\
	\DeclareType{PositiveOrthogonal}
	{
		\prod k : \Field \.
		\prod V : \OVS(k) \.
		\prod W : \IPS(k) \. \NewLine \. 
		? \TYPE{Orthogonal}(V,W)
	}
	\DefineType{\rho}{PositiveOrthogonal}
	{
		\sigma(\rho) = 1
	}
	\\
	\DeclareType{NegativeOrthogonal}
	{
		\prod k : \Field \.
		\prod V : \OVS(k) \.
		\prod W : \IPS(k) \. \NewLine \. 
		? \TYPE{Orthogonal}(V,W)
	}
	\DefineType{\rho}{NegativeOrthogonal}
	{
		\sigma(\rho) = -1
	}
	\\
	\Theorem{PositivelyOrhognallyRepresentedAreSymmetric}
	{
		\forall k : \Field \.
		\forall V : \TYPE{NonDegenerate}(k) \. \NewLine \. 
		\forall W \in \IPS(k) \. 
		\forall \rho : \TYPE{PositiveOrthogonal}(V,W ) \.
		\forall x \in V \.
		\rho(x) : \TYPE{Symmetric}(W)
	}
	\Assume{v,w}{W}
	\Say{\Big[(v,w).*.1\Big]}{\bd \TYPE{PositiveOrthogonal}(V,W)(\rho)(v,w) \bd \FUNC{AdjointOperator} }
	{
		\NewLine : 
		\langle x, x \rangle \langle v, w \rangle  =
		\langle v \; \rho(x) , w \; \rho(x) \rangle = 
		\langle v \; \rho(x) \; \rho^*(x), w \rangle  
	}
	\Say{\Big[(v,w).*.2\Big]}
	{ 
		\NewLine :
		\bd \LALGE{k}
		\Big( \CL(V), \VS{k}(W,W) \Big)  
		\bd \CLIF{k}
		\Big( \CL(V) \Big)
		\bd \LALGE{k}
		\Big( \CL(V), \VS{k}(W,W) \Big)  
	}
	{
		\NewLine :
		\langle v \; \rho^2(x),  w  \rangle = 
		\langle v \; \rho(x^2),  w  \rangle = 
		\Big\langle v  \; \langle x, x \rangle\rho(e), w  \Big\rangle =
		\langle x, x \rangle \langle v, w \rangle
	}
	\Derive{\Big[ 1 \Big]}{ \THM{NonDegenerateDefines}}{  \rho(x) \; \rho^*(x) = \rho^2(x)}
	\Conclude{[*]}{\bd^{-1} \TYPE{Symmetric}[1]}{ \Big( \rho(x) : \TYPE{Symmetric}(W)  \Big) } 
	\EndProof
}\Page{	
	\Theorem{NegativelyOrhognallyRepresentedAreSkew}
	{
		\forall k : \Field \.
		\forall V : \OVS(k) \. \NewLine \.
		\forall W \in \IPS(k) \. 
		\forall \rho : \TYPE{NegativeOrthogonal}(V,W ) \.
		\forall x \in \CL(V) \.
		\rho(x) : \TYPE{Skew}
	}
	\NoProof
	\\
	\Theorem{PositiveRepresentationClassification}
	{
		\forall n \in \Nat \.
		\forall V : \VS{\Reals} \.
		\forall \rho : \TYPE{Representation}(\Reals, \CL_n(+),V)  \. \NewLine \.
		\exists W : \IPS(\Reals) :
		\exists \rho' : \TYPE{PositiveOrthogonal}(\Reals^n,W) \.
		\rho \sim \rho'
	}
	\Say{[1]}{\bd e \bd \CL_n(+)}{
		\forall i,j \in n \. 
		(e_i \; \mathbf{i})(e_j \; \mathbf{i}) 
		+
		(e_j \; \mathbf{i})(e_i \; \mathbf{i})  =
		2 \delta_{i,j} e
	}
	\Say{[2]}{\bd \TYPE{Automorphism}[1]}{\forall i \in n \. \rho(e_i) \in \Aut_{\VS{\Reals}}(V)}
	\Say{G}{\Big\langle \rho(e_i) \Big\rangle_{\Aut_{\VS{\Reals}}(V)}}{\TYPE{Subgroup}\Big(\Aut_{\VS{\Reals}}(V)\Big)}
	\Say{[3]}{ \bd^{-1} \FG[1]}{\Big( G : \FG  \Big)}
	\Say{Q}{\lambda v,w \in V \. \sum_{g \in G} \langle v \; \rho(g), w \; \rho(g) \rangle}
	{ \TYPE{SymmetricForm}(V)  }
	\Say{W}{(V,Q)}{\OVS(\Reals)}
	\Assume{v,w}{W}
	\Assume{g}{G}
	\Conclude{\Big[(v,w).*\Big]}{ \ByConstr W \bd \LALGE{\Reals}\Big( \CL_n(+),\VS{\Reals}(V,V) \Big)(\rho)
		\THM{GroupCycle}(G)(g) \ByConstr^{-1} W  }
	{
		\NewLine
		\Big\langle v \; \rho(g), w \; \rho(g) \Big\rangle_W = 
		\sum_{f \in G} \Big\langle v ; \rho(g)\rho(f)  , w \; \rho(g)\rho(f) \Big\rangle_V =
		\sum_{f \in G} \Big\langle v ; \rho(gf) , w \; \rho(gf) \Big\rangle_V = \NewLine = 
		\sum_{f \in G} \Big\langle v ; \rho(f)  , w \; \rho(f) \Big\rangle_V =
		\langle v, w \rangle_W
	}
	\Derive{[4]}{I(\forall)}{\forall v,w \in V \. \forall g \in G \. \langle v\;\rho(g) , w \;\rho(g)  \rangle_W = \langle v,w \rangle_W}
	\Say{[5]}{[4]\bd^{-1} \TYPE{PositiveOrthogonal}}
	{
		\Big(	\rho : \TYPE{PositiveOrthogonal}( \Reals^n, W  ) \Big) 
	}
	\Conclude{[*]}{\bd \TYPE{Reflexivity}(\TYPE{EquivalentAlgRepr})(\rho)}{\rho \sim \rho} 
	\EndProof
	\\
	\Theorem{NegativeRepresentationClassification}
	{
		\forall n \in \Nat \. 
		\forall V : \IPS(\Reals) \. \NewLine \.
		\forall \rho : \TYPE{Representation}(\Reals, \CL_n(-),V)  \. 
		\exists W : \IPS(\Reals) : \NewLine : 
		\exists \rho' : \TYPE{NegativeOrthogonal}(\Reals^n,W) \.
		\rho \sim \rho'
	}
	\NoProof
	\\
	\DeclareFunc{twistedAdjointRepresentation}
	{
		\prod k : \TYPE{Field} \.
		\prod V : \OVS{k} \. \NewLine \. 
		\TYPE{Representation}\Big( \CL^*(V),\CL(V) \Big) 
	}
	\DefineNamedFunc{twistedAdjointRepresentation}
	{  x }{ \tad x}{ \Lambda a \in \CL(V) \. \omega_V(x) a x^{-1}}  
}
\Page{
	\Theorem{TwistedAdjointRepresentationKernel}
	{
		\forall k : \TYPE{Numeric} \.
		\forall V : \TYPE{NonDegenerate}(k) \. \NewLine \.
		\ker  {\tad}_V = k e_{\CL(V)}
	}
	\Assume{\lambda}{k}
	\Assume{a}{\CL(V)}
	\Conclude{[a.*]}{
		\bd \tad(\lambda e)
		\bd \LALGE{k}
		\THM{UnitityInverse}
		\bd e_{\CL(V)}
		\bd \FUNC{inverse}
	}   
	{ \NewLine :
		\tad(\lambda e) a =
		\omega_V(\lambda e) a (\lambda e)^{-1} 
		\lambda e  a (\lambda^{-1} e) =
		\lambda \lambda^{-1} a =
		a
	}
	\Derive{[1]}{I(\forall)}{\forall a \in \CL(V) \. \tad(\lambda e) a = a  }
	\Conclude{[\lambda.*]}{\THM{UniqueIdentity}[1]}{\tad(\lambda e) = \id}
	\Derive{[1]}{\bd \ker\bd \TYPE{Subset}}{ ke \subset \ker {\tad}_V}
	\EndProof
}
\newpage
\subsection{Clifford Group}
\Page{
	\DeclareFunc{groupOfClifford}
	{ 
		\prod k : \Field \. 
		\OVS(k) \to \GRP 
	}
	\DefineNamedFunc{groupOfClifford}
	{V}{ \Gamma(V) }{ \mathrm{Stab}\Big( \CL^*(V) ,V \; \mathbf{i}\Big)
		\Big(\tad\Big) } 
	\\
	\Theorem{NondegenerateVectorsInCliffordGroup}
	{
		\forall k : \Field \.
		\forall V : \OVS(k) \.
		\forall v \in V \. \NewLine  \.
		\forall [0] : \langle v, v \rangle \neq 0  \.
		v \; \mathbf{i}_V \in \Gamma(V)	
	}
	\Say{[1]}{\bd \CLIF{k}\Big(\CL(V)\Big)\bd^{-1} \FUNC{inverse} }
	{
		(v \; \mathbf{i}_V)^{-1} = 
		\frac{v \; \mathbf{i}_V}{ \langle v,v \rangle  }
	}
	\Assume{w}{V}
	\Say{[2]}{ \bd \tad [1] \bd \CLIF{k}\Big(\CL(V)\Big)    }
	{
		\NewLine :
		\tad(v \; \mathbf{i}_V)(w \; \mathbf{i}_V) =
		\omega_V(v \; \mathbf{i}_V)(w \; \mathbf{i}_V)(v \; \mathbf{i}_V) =
		-(v \; \mathbf{i}_V)(w \; \mathbf{i}_V)
		\frac{v \; \mathbf{i}_V}{\langle v, v \rangle} =
		\frac{1}{\langle v, v\rangle}
		(w \; \mathbf{i}_V)(v \; \mathbf{i}_V)^2 +
		2\frac{\langle w, v \rangle}{\langle v, v \rangle} 
		(v \; \mathbf{i}_V)   =   \NewLine =       
		(w \; \mathbf{i}_V) +
		2\frac{\langle w, v \rangle}{\langle v, v \rangle} 
		(v \; \mathbf{i}_V)   \in \mathbf{i}(V)          
	}
	\DeriveConclude{[*]}{\bd \Gamma(V)}{ v \; \mathbf{i} \in \Gamma(V)}
	\EndProof
	\\
	\Theorem{CliffordGroupDegreeInvolution}
	{
		\forall k : \TYPE{Numeric} \. 
		\forall V : \TYPE{Nondegenerate}(k) \.
		\forall x \in \Gamma(V) \. \NewLine \.
		\omega_V(x) \in  \Gamma(V)  
	}
	\Assume{v}{V}
	\Conclude{[*]}
	{
		\bd \tad 
		\bd \omega_V 
		\bd \LALGE{k}\Big(\CL(V),\CL(V)\Big)(\omega_V)
		\bd \Gamma(V)(x) 
		\bd \omega_V 
		\bd \Gamma(V)(x)
	}
	{
		\NewLine :
		\tad\Big( x \; \omega_V)(v \; \mathbf{i}) =
		( x \; \omega_V^2)(v \; \mathbf{i})( x \; \omega_V  )^{-1} =
		-( x \; \omega_V)(v \; \mathbf{i})(x^{-1}) \; \omega_V =
		(x \; \omega_V)(v \;  \mathbf{i})(x^{-1}) \in V \; \mathbf{i}
	}
	\DeriveConclude{[*]}{\bd \Gamma(V)}{ x \in \Gamma(V)}
	\EndProof
	\\
	\Theorem{CliffordGroupSemiconjugation}
	{
		\forall k : \TYPE{Numeric} \. 
		\forall V : \TYPE{Nondegenerate}(k) \.
		\forall x \in \Gamma(V) \. \NewLine \.
		S_V(x) \in  \Gamma(V)  
	}
	\Assume{v}{V}
	\Conclude{[*]}
	{
		\bd \tad 
		\bd S_V
		\THM{SemiconjugationPreseresCliffordMap}(v) \NewLine
		\THM{SemiconjugationCommutedWithDegreeInvolutin}
		\bd^{-1} \tad
		\bd \Gamma(V)(x \omega(V))^{-1}
	}
	{
		\NewLine :
		\tad\Big( x \; S_V)(v \; \mathbf{i}) =
		( x \; \omega_V S_V)(v \; \mathbf{i})( x \; S_V  )^{-1} =
		( x^{-1} )(v \; \mathbf{i})(x \; \omega_V) \; S_V =
		\tad\Big( x \omega_V  \Big)^{-1}(v)   \in V \; \mathbf{i}
	}
	\DeriveConclude{[*]}{\bd \Gamma(V)}{ x \in \Gamma(V)}
	\EndProof
	\\
	\Theorem{CliffordGroupConjugation}
	{
		\forall k : \TYPE{Numeric} \. 
		\forall V : \TYPE{Nondegenerate}(k) \.
		\forall x \in \Gamma(V) \. \NewLine \.
		\overline{x} \in  \Gamma(V)  
	}
	\NoProof
}
\Page{
	\Theorem{CliffordGroupConjugateSquare}
	{
		\forall k : \TYPE{Numeric} \.
		\forall V : \TYPE{Nondegenerate}(k)\. \NewLine \.
		\exists \lambda : \Gamma(V) \Arrow{\GRP} k^* :
		\forall x \in \Gamma(V) \.
		x\overline{x} = \lambda(x) e
	}
	\Assume{x}{\Gamma(V)}
	\Assume{v}{V}
	\Say{w}{\tad(\overline{x})(v \mathbf{i})}{V}
	\Say{[1]}{\THM{SemiconjugationPreservesCliffordMap}(w)}
	{
		S_V(w \mathbf{i}) = w \mathbf{i}   
	}
	\Say{[2]}{\bd^{-1} \tad\ByConstr w \bd S_V [1]}
	{
		\omega( \overline{x} ) (v \mathbf{i}) \overline{x}^{-1} =
		\tad(\overline{x})(v \mathbf{i}) = 
		w =
		w \;  S_V =
		(\overline{x} \; S_V)^{-1}( v \mathbf{i})
		( \overline{x} \; \omega_V \; S_V)
	}
	\Say{[3]}{  
		\bd \LALGE{k} \omega_V \Big( \CL(V),\CL(V)  \Big)
		\bd \FUNC{conjugation}
		(\overline{x} \; S_V)[2] \overline{x}  
		\bd^{-1} \FUNC{conjugation}
	} 
	{   
		\NewLine :
		\omega(x\overline{x})(v \mathbf{i})
		(\overline{x} S_V)  \omega( \overline(x) ) (v \mathbf{i}) =  
		(v \mathbf{i})(x \; \omega_V \; S_V)\overline{x} =
		(v \mathbf{i})x \overline{x}
	}
	\Conclude{[v.*]}{\bd \tad [3]}
	{
		\tad\Big(x\overline{x}\Big)(v \; \mathbf{i}_V) =
		\omega(x\overline{x})(v \; \mathbf{i}_V)(x\overline{x})^{-1} =
		(v \; \mathbf{i}_V)x\overline{x}(x\overline{x})^{-1} =
		(v \; \mathbf{i}_V)
	}
	\Derive{ [1]  }
	{ I(=,\to)}{\tad(x\overline{x})_{|V\mathbf{i}} = \id}
	\Say{[2]}{  [1]\bd \tad \bd^{-1} Z_0 \; \CL(V)    }
	{
		(x\overline{x})_0 \in Z_0\;\CL(V) \And 
		(x\overline{x}_1)_1 \in AZ_1\; \CL(V)
	}
	\Say{[3]}{\THM{TrivialAnicentre} \And \THM{LinearCentre}[2]}
	{
		(x\overline{x})_0 \in ke \And (x\overline{x})_1 = 0 
	}
	\Say{\Big(\lambda(x),[1]\Big)}{\bd ke [3]}
	{ \sum \lambda(x) \in k \.  x\overline{x} =  \lambda(x) e  }
	\Conclude{[x.*]}{\bd \GRP\Gamma(V)(x)\bd \LALGE{k}\Big(\CL(V)\Big)}
	{ \lambda(x) \in k^*}
	\Derive{ \lambda }{I\Act{\sum}I\Act{\sum} }
	{
		\prod x \in \Gamma(V) \. 
		\sum \lambda(x) \in k^* \. 
		x\overline{x} = \lambda e
	}
	\Assume{x,y}{\Gamma(V)}
	\Say{[1]}{
		\bd_1^{-1} \lambda (xy) 
		\THM{ConjugationAntihomo}(xy)
		\bd_1 \lambda(y)
		\bd \LALGE{k}\CL(V)
		\bd_1 \lambda(x)
		\bd \ANN(k)
	}
	{ 
		\NewLine :
		\lambda(xy) e =
		xy\overline{x}\overline{y} = 
		x\; y \; \overline{y}\; \overline{x} =
		x\lambda(y)e\overline{x} =
		\lambda(y) x \overline{x} =
		\lambda(y) \lambda(x) e =
		\lambda(x)\lambda(y) e
	}
	\Conclude{\Big[(x,y).*\Big]}{\bd \Field \bd \LALGE{k}\CL(V)}
	{  \lambda(xy) = \lambda(x)\lambda(y)  }
	\DeriveConclude{[*]}{\bd \GRP}{\lambda : \Gamma(V) \Arrow{\VS{k}} k^* }
	\EndProof
	\\
	\DeclareFunc{conjugationSquare}
	{
		\prod k : \TYPE{Numeric}(V) \.
		\prod V : \TYPE{NonDegenerate}(k) \.
		\Gamma(V) \Arrow{\GRP} k^*           
	}
	\DefineNamedFunc{conjugationSquareMap}
	{x}{ \lambda_V(x)}{\THM{CliffordGroupConjugationSquare}}
	\\
	\Theorem{degreeinvolutionpreservesconjugatesquare}
	{
		\forall k : \Type{numeric} \.
		\forall v : \Type{nondegenerate}(k) \. \NewLine \.
		\forall x \in \Gamma(v) \.
		\omega_v(x)\overline{\omega_v(x)} = x\overline{x}
	}
	\NoProof
	\\
	\Theorem{DegreeInvolutionPreservesConjugateSquareMap}
	{
		\forall k : \TYPE{Numeric} \.
		\forall V : \TYPE{Nondegenerate}(k) \. \NewLine \.
		\omega_V \lambda_V = \lambda_V
	}
	\NoProof
}\Page{
	\Theorem{TwistedAdjugationPreservesConjugateSquareMap}
	{
		\forall k : \TYPE{Numeric} \.
		\forall V : \TYPE{Nondegenerate}(k) \. \NewLine \.
		\forall a \in \Gamma(V) \.
		\tad(a) \lambda_V = \lambda_V
	}
	\NoProof
	\\
	\Theorem{twistedAdjugationIsoquadric}
	{
		\forall k : \TYPE{Numeric} \.
		\forall V : \TYPE{Nondegenerate}(k) \. \NewLine \.
		\forall a \in \Gamma(V) \.
		\tad(a)_{|V} : \TYPE{Isoquadric}(V,V)
	}
	\Assume{v}{V}
	\Assume{[0]}{\langle v, v\rangle \neq 0}
	\Say{[1]}
	{
		\bd\FUNC{conjugation}
		\bd \CLIF{k} \; \CL(V)
	}
	{ 
		(v  \mathbf{i})\overline{v  \mathbf{i}} =
		-( v  \mathbf{i})^2 = - \langle v, v \rangle e  
	}
	\Say{[2]}
	{
		\bd^{-1} \lambda_V [1][0]
	}
	{
		\lambda(v\mathbf{i})  = - \langle v,v\rangle          
	}
	\Conclude{[0.*]}{
		[2]
		\THM{TwistedAdjugationPreservesConjugateSquareMap}
		[2] 
	}
	{ 
		\NewLine :
		- \langle \tad(a)v , \tad(a)v
		\lambda_V\Big( \tad(a)v\mathbf{i} \Big)  =
		\lambda_V( v\mathbf{i} ) = - \langle v, v \rangle
	}
	\Derive{[1]}{I(\Imply)}
	{
		\langle v, v \rangle \neq  0 \Imply
		\langle \tad(a)v, \tad(a)v \rangle = \langle v, v \rangle
	}
	\Assume{[0]}{\langle v, v\rangle = 0}
	\Conclude{[0.*]}{\bd \tad{a} \bd \CLIF{k}(V) \bd \GRP \Gamma(V)}
	{
		\langle \tad{a} v, \tad{a} v \rangle 
		=0=
		\langle v, v \rangle
	}
	\Derive{[2]}{I(\Imply)}
	{
		\langle v, v \rangle =  0 \Imply
		\langle \tad(a)v, \tad(a)v \rangle = \langle v, v \rangle
	}
	\Conclude{[v.*]}{\LOGIC{LEM}(\langle v,v \rangle = 0)[1][2]E(|)}
	{
		\langle \tad(a)v, \tad(a)v \rangle = \langle v, v \rangle
	}
	\DeriveConclude{[*]}{\bd^{-1}\TYPE{Isoquadric}}
	{
		\Big( \tad(a) : \TYPE{Isoquadric}(V,V)  \Big)
	}
	\EndProof
	\\
	\DeclareFunc{asOrthogonalTransform}
	{
		\forall k : \TYPE{Numeric} \.
		\forall V : \TYPE{Nondegenerate}(k) \. \NewLine \.
		\Gamma(V) \Arrow{\GRP} \O(V) 
	}
	\DefineNamedFunc{asOrthogonalTransform}{a}{O(a)}{\tad{a}_{|V}}
	\\
	\Theorem{VectorsProduceReflections}
	{
		\forall k : \TYPE{Numeric} \.
		\forall V : \TYPE{Nondegenerate}(k) \. \NewLine \.
		\langle v, v \rangle \neq 0 \Imply O(v \mathbf{i}) = \sigma_v
	}
	\NoProof
}\Page{
	\Theorem{CliffordGroupSpawnsOrthogonalGroup}
	{
		\forall k : \TYPE{Numeric} \.
		\forall V : \TYPE{Nondegenerate}(k) \. 
		O_V : \Gamma(V) \ToSurj \O(V)
	}
	\Assume{T}{\O(V)}
	\Say{\Big(n,S,[1]\Big)}
	{
		\THM{OrthogonalGroupStructure}(T) 
	}
	{
		\sum n \in \Int_+ \.
		\sum S : n \to \TYPE{Symmetry}(V) \.
		T = \prod^n_{i=1} S_i
	}
	\Say{\Big(v,[2]\Big)}{ \bd \TYPE{Symmetry}(S) }
	{
		\sum v : n \to V \. \forall i \in n \. 
		\langle v,v \rangle \neq 0 \And S_i = \sigma_{v_i}
	}
	\Conclude{[T.*]}
	{
		\bd \GRP\Big( \Gamma(V),\O(V) \Big)(O_V)
		\forall i \in n \.\THM{VectorsProduceReflections}(v_i)
		[2][1]
	}
	{
		\NewLine :
		O\left( \prod^n_{i=1} v_i\mathbf{i} \right) = 
		\prod^n_{i=1} O(v_i\mathbf{i}) =
		\prod^n_{i=1} \sigma_{v_i} =
		\prod^n_{i=1} S_i = 
		T	
	}
	\DeriveConclude{[*]}{\bd^{-1}\TYPE{Surjection}}
	{ \Big( O_V : \Gamma(V) \ToSurj \O(V)  \Big)  }
	\EndProof
	\\
	\Theorem{CliffordGroupScalarCriterion}
	{
		\forall k : \TYPE{Numeric} \.
		\forall V : \TYPE{Nondegenerate}(k) \.
		\forall x \in \Gamma(V) \. \NewLine
		\Big( \forall v \in V \. \omega_V(x) (v \mathbf{i}) =
		(v \mathbf{i}) x \Big) \Imply x \in ke_{\CL(V)}
	}
	\\
	\Theorem{CliffordGroupStructure}
	{
		\forall k : \TYPE{Numeric} \.
		\forall V : \TYPE{Nondegenerate}(k) \. \NewLine \.
		\Gamma(V) = \bigg\langle\Big\{ (v \mathbf{i}) 
		\Big| v \in V : \langle v,v \rangle \neq 0 \Big\} \bigg\rangle 
	}
	\Assume{x}{\Gamma(V)}
	\Say{T}{O(x)}{\O(V)}
	\Say{\Big(n,v,[1]\Big)}{\THM{CliffordGroupSpawnsOrthogonalGroup}(T)}
	{ 
		\sum n \in \Nat \.
		\sum v : n \to V \. 
		T = O\left( \prod^n_{i=1} v_i \mathbf{i} \right)
	}
	\Say{a}{x^-1 \prod^n_{i=1} (v_i\mathbf{i})}{\Gamma(V)}
	\Say{[2]}{
		\ByConstr a
		\bd \GRP \Big( \Gamma(V), \O(V) \Big)(O)
		\ByConstr^{-1} T[1]
		\bd \FUNC{inverse}
	}
	{
		O(a) = 
		O\left( x^{-1}\prod^n_{i=1} v_i \mathbf{i}  \right)  =
		O^{-1}(x) O\left( \prod^n_{i=1} v_i \mathbf{i} \right) = \NewLine =
		T^{-1} T = 
		\id
	}
	\Say{[3]}{\bd O [2]}{ 
		\forall v \in V \. 
		\omega(a)(v\mathbf{i}) = (v\mathbf{i})a^{-1}_
	}
	\Say{[4]}{\THM{CliffordGroupScalarCriterion}[3]}
	{
		a \in ke_{\CL(V)}
	}
	\Conclude{\Big(\lambda,[x.*]\Big)}{[4]\ByConstr a}
	{
		\sum \lambda \in k^* \. x = 
			  \left( (\lambda v_1 \mathbf{i}) 
			  	\prod^n_{i=2} (v_i \mathbf{i}) \right)^{-1}
	}
	\DeriveConclude{[*]}{\bd \FUNC{generateGroup}}{\LOGIC{This}}
	\EndProof
}\Page{
	\Theorem{DegreeInvolutionByDeterminant}
	{
		\forall k : \TYPE{Numeric} \.
		\forall V : \TYPE{Nondegenerate}(k) \.
		\forall x \in \Gamma(V) \. \NewLine 
		\omega_V(x) = \big(\det O_V(x)\big) x 
	}
	\Say{\varphi}{\Lambda x \in \Gamma(V) \.\big(\det O_V(x)\big) x}
	{\Gamma(V) \Arrow{\GRP} \Gamma(V)}
	\Assume{v}{V}
	\Assume{[0]}{\langle v, v\rangle \neq 0}
	\Say{[1]}{\THM{VectorsProduceReflections}(v)\THM{ReflectionDeterminant}} 
	{ \det O(v) = -1 }
	\Conclude{[v.*]}{\bd \omega_V [1] \ByConstr \varphi}
	{\omega_V(v) = \varphi(v) }
	\Derive{[1]}{I(\forall)}
	{
		\forall v \in V \.
		\omega_V(v \mathbf{i}) = \varphi(v \mathbf{i})
	}
	\Say{\Big( n, v, [2]  \Big)}
	{ \THM{CliffordGroupStructure}(x)  }
	{
		\sum n \in \Nat \. \sum v : n \to V \. 
		x = \prod^n_{i=1} (v_i \; \mathbf{i}) \. 
	}
	\Conclude{[*]}{[1][2]\bd\GRP\Big(\Gamma(V),\Gamma(V)\Big)(V)}
	{ \omega_V(x)   = \varphi(x)}
	\EndProof
}
\newpage
\subsection{Spin Group and Representation}
\Page{
	\DeclareFunc{pinGroup}{\prod k : 
		\TYPE{Numeric} \. 
		\TYPE{Nondegenerate}(k) \to \GRP  
	}
	\DefineNamedFunc{pinGroup}{V}{\PIN(V)}
	{\Big\{ x \in \Gamma(V) \Big| \lambda_V(x) \in \{-1,+1\}  \Big\}}
	\\
	\DeclareFunc{spinGroup}{\prod k : 
		\TYPE{Numeric} \. 
		\TYPE{Nondegenerate}(k) \to \GRP  
	}
	\DefineNamedFunc{SpinGroup}{V}{\SPIN(V)}
	{\Big\{ x \in \Gamma(V) \Big| \lambda_V(x) = 1  \Big\}}
	\\
	\Theorem{PinGroupSpawnsOrthogonalGroup}
	{
		\forall V : \TYPE{Nondegenerate}(\Reals) \.
		\PIN(V) \; O = \O(V)
	}
	\Assume{T}{\O(V)}
	\Say{\Big(n,v,[1]\Big)}{\THM{CliffordGroupSpawnsOrthogonalGroup}(T)}
	{ 
		\sum n \in \Nat \.
		\sum v : n \to V \.  \NewLine
		T = O\left( \prod^n_{i=1} v_i \mathbf{i} \right) 
		\And \forall i \in n \. \langle v_i, v_i \rangle \neq 0
	}
	\Assume{i}{n}
	\Say{u_i}{\frac{v_i}{\sqrt{|\langle v_i, v_i\rangle|}} }{V}
	\Say{[2]}{\ByConstr u_i \bd \lambda_V \bd \FUNC{absValue}}{
			\lambda_V(u_i) = 
			-\frac{\langle v_i, v_i\rangle}{|\langle v_i,v_i\rangle|}
			\in \{-1,+1\}
	}
	\Conclude{[i.*]}{\bd \PIN(V) }{u_i \in \PIN(V)}
	\Derive{u}{I(\to)}{  n \to V }
	\Conclude{[T.*]}{ \bd O \bd \tad }
	{
		O\left( \prod^n_{i=1} u_i  \right) = T
	}
	\Derive{[*]}{I(\forall)}{\LOGIC{This}}
	\EndProof
	\\
	\Theorem{PinKernel}
	{
		\forall V : \TYPE{Nondegenerate}(\Reals) \.
		\ker O_{V|\PIN(V)} = \mathbb{S}^0
	}
	\NoProof
	\\
	\Theorem{SpinGroupSpawnsSpecialOrthogonalGroup}
	{
		\forall V : \TYPE{Nondegenerate}(\Reals) \.
		\SPIN(V) \; O = \SO(V)
	}
	\NoProof
	\\
	\Theorem{SPinKernel}
	{
		\forall V : \TYPE{Nondegenerate}(\Reals) \.
		\ker O_{V|\SPIN(V)} = \mathbb{S}^0
	}
	\NoProof
}
\Page{
	\DeclareType{MetricComplexStructure}
	{
		\prod V : \TYPE{NonDegenerate}(\Reals) \.
		?\TYPE{ComplexStructure}(V)
	}
	\DefineType{J}{MetricComplexStructure}
	{J : \TYPE{Isoquadric}(V)}
	\\
	\Theorem{MetricComplexStructureAdjoint}
	{
		\forall V : \TYPE{Nondegenerate}(\Reals) \.
		\forall J : \TYPE{MetricComplexStructure}(V) \. 
		J^\star = -J
	}
	\Say{[1]}{\bd \TYPE{Isoquadric}(V)(J)\bd \TYPE{Nondenerate}(\Reals)(V)}
	{J^\star = J^{-1}}
	\Conclude{[*]}{\bd \TYPE{ComplexStructure}(V)(J)[1]}
	{
		J^{\star} = - J
	}
	\EndProof
	\\
	\Theorem{MetricComplexStructureIsSkew}
	{
		\forall V : \TYPE{Nondegenerate}(\Reals) \.
		\forall J : \TYPE{MetricComplexStructure}(V) \.
		\NewLine \.
		J : \TYPE{Skew}(V)
	}
	\\
	\DeclareFunc{complexInvolution}
	{
		\prod V : \TYPE{Nondegenerate}(\Reals) \.
		\TYPE{MetricComplexStructure}(V) \to
		\Complex \otimes V 
		\Arrow{\VS{\Complex}}
		\Complex \otimes V
	}
	\DefineNamedFunc{complexInvolution}{J}{\omega_J}
	{
		\bd \FUNC{tensorProduct} 
		\Lambda z \in \Complex \. 
		\Lambda v \in V \.
		\mathrm{i}z \otimes v\;J
	}
	\\
	\Theorem{ComplexInvolutionIsInvolution}
	{
		\forall V : \TYPE{Nondegenerate}(\Reals) \.
		\forall J : \TYPE{MetricComplexStructure}(V) \. 
		\omega_J^2 = \id
	}
	\Assume{z}{\Complex}
	\Assume{v}{V}
	\Conclude{[z.*]}
	{
		\bd \omega_J 
		\bd \mathrm{i} \bd \TYPE{ComplexStructure}(V)(J)
		\bd \L(\Complex,V;\Complex \otimes V)(\otimes)
		\THM{NegativeSquare}(\Reals)
	}
	{
		\NewLine : 
		\omega^2_J(z \otimes v) = 
		\mathrm{i}^2 z \otimes v \; J^2 = 
		- z \otimes -v =
		(-1)^2 z \otimes v =
		z \otimes v
	}
	\DeriveConclude{[*]}{\bd \FUNC{tensorProduct}I(=,\to)}{\omega_J^2 = \id}
	\EndProof
	\\
	\Theorem{ComplexInvolutionIsSkew}
	{
		\forall V : \TYPE{Nondegenerate}(\Reals) \.
		\forall J : \TYPE{MetricComplexStructure}(V) \. 
		\NewLine \. 
		\omega_J : \TYPE{Skew}(\Complex \otimes V)
	}
	\Assume{t}{\Complex \otimes V}
	\Say{\Big( v,w,[1] \Big)}{\THM{TensorProductBasis}(t)}
	{
		\sum v,w \in V \. t = 1 \otimes v + \mathrm{i} \otimes w
	}
	\Conclude{[t.*]}{
		[1]
		\bd \L(\Complex \otimes V,\Complex \otimes V; \Complex)
		(\Complex \otimes V)
		\bd \omega_J
		\bd \FUNC{innerProductTensorProduct}
		\bd \TYPE{Skew}(V)(J)
		\bd \FUNC{adjOp}(J) \NewLine
		\THM{MetricComplexStructuteAdjoint}(J)
		\bd \L(\Complex \otimes V,\Complex \otimes V; \Complex)V
		\bd \FUNC{inverse}
		}
	{
	\NewLine :
	\langle t \; \omega_J, t  \rangle = 
	\langle 1 \otimes v \; \omega_J, 1 \otimes v \rangle +
	\langle 1 \otimes v \; \omega_J, \mathrm{i} \otimes w \rangle +
	\langle \mathrm{i} \otimes w \; \omega_J, \mathrm{i} \otimes w \rangle +
	\langle \mathrm{i} \otimes w \; \omega_J, 1 \otimes v \rangle = \NewLine
	\langle \mathrm{i} \otimes v \; J, 1 \otimes v \rangle +
	\langle \mathrm{i} \otimes v \; J, \mathrm{i} \otimes w \rangle +
	\langle  -1 \otimes w \; J, \mathrm{i} \otimes w \rangle +
	\langle  -1 \otimes w \; J, 1 \otimes v \rangle = \NewLine =
	 \mathrm{i} \langle v \; J, v \rangle 
	- \langle  v \; J, w \rangle +
	-\mathrm{i}\langle    w \; J,  w \rangle 
	-\langle   w \; J,  v \rangle = 
	-\langle v \; J, w \rangle - \langle v \; J^\star, w \rangle =
	-\langle v \; J, w \rangle + \langle v \; J, w \rangle =
	0
	}
	\DeriveConclude{[*]}{\bd^{-1} \TYPE{Skew}}
	{\Big( \omega_J : \TYPE{Skew}(\Complex \otimes V) \Big)  }
	\EndProof
	\\
	\DeclareFunc{exteriorComplexIso}
	{
		\prod V : \TYPE{Nondegenerate}(\Reals) \.
		\prod J : \TYPE{MetricComplexStructure}(V) \. 
		\NewLine \. 
		\CL(\Complex \otimes V) \ToIso{\LALGE{\Complex}}
		\End_{\VS{\Complex}}\Big(\ker^\wedge(\id - \omega_J)\Big)
	}
	\DefineNamedFunc{exteriorComplexIso}{}{R_J}
	{\THM{CliffordExteriorOperatorsIsomorphismCriterion}(\omega_J)}
}
\Page{
	\DeclareFunc{complexCliffordEmbedding}
	{
		\prod V : \OVS(\Reals) \.
		\CL(V) \ToInj \CL(\Complex \otimes V)
	}
	\DefineNamedFunc{complexCliffordEmbedding}{x}{1 \otimes x}
	{
		\CL(\iota)(x) \; \where \; \iota = \Lambda v \in V \. 1 \otimes v
	}
	\\
	\DeclareFunc{spinRepresentation}
	{
		\prod V : \TYPE{Nondegenerate}(\Reals) \.
		\prod J : \TYPE{MetricComplexStructure}(V) \. 
		\NewLine \. 
		\CL(V) \Arrow{\LALGE{\Reals}}
		\End_{\VS{\Complex}}\Big(\ker^\wedge(\id - \omega_J)\Big)
	}
	\DefineNamedFunc{spinRepresentation}{x}{S_J(x)}
	{R_J(1 \otimes x) }
	\\
	\DeclareType{ComplexIrreducible}
	{
		\prod A : \LALGE{\Reals} \.
		\prod U : \VS{\Complex} \. 
		? A \Arrow{\LALGE{\Reals}} \End_{\VS{\Complex}}(U)
	}
	\DefineType{\rho}{ComplexIrreducible}
	{
		\TYPE{Invariant}(\Complex)(\rho(A)) = \Big\{ \{0\},U \Big\}
	}
	\\
	\Theorem{SpinRepresentationIsIrreducible}{ 
		\forall V : \TYPE{Nondegenerate}(\Reals) \.
		\forall J : \TYPE{MetricComplexStructure}(V) \. 
		\NewLine \.
		S_J : \TYPE{ComplexIrreducible}
		\Big( \CL(V), \ker^\wedge(\id - \omega_J) \Big)
	}
	\Assume{U}{\TYPE{Invariant}\Big(S_J(\CL(V)) \Big)}
	\Say{[1]}{\bd \TYPE{Invariant}(U)}
	{
		\forall x \in \CL(V) \. S_J(x)(U) \subset U
	}
	\Assume{t}{\CL(\Complex \otimes V)}
	\Say{\Big(z,x,[2]\Big)}
	{\THM{CliffordAlgebraScalarExtension}(t)}
	{
		\sum z \in \Complex \.
		\sum x \in \CL(V) \.
		t = z \otimes x
	}
	\Conclude{[t.*] }
	{
		[2]
		\bd \VS{\Complex}\Big(\CL(\Complex \otimes V),
		\End_{\VS{\Complex}}\big(\ker (\id -\omega_J)\big)^\wedge
		(R_J)
		\bd^{-1}(S_J)
		[1](x)
		\bd \TYPE{VectorSubspace}(\Complex)(U)
	}
	{
		\NewLine
		R_J(t)(U) = 
		R_J(z \otimes x)(U) =
		z R_J(1 \otimes x)(U) =
		z S_J(x)(U) \subset U 
	}
	\Derive{[2]}{\bd^{-1} \TYPE{Invariant}}
	{ \bigg( U : \TYPE{Invariant}\Big( R_J\big( \CL(\Complex
		\otimes V ) \big) \Big) \bigg)   }
	\Conclude{[U.*]}{
		\THM{IsomorphismIsIrreducible}(R_J)
		\bd\TYPE{Irreducible}[2]}
	{
		U = \{0\} \Big| U = \ker^\wedge (\id - \omega_J )
	}
	\DeriveConclude{[*]}{\bd^{-1} \TYPE{ComplexIrreducible}}
	{\LOGIC{This}}
	\EndProof
	\\
	\Theorem{HermitianSubstitution}
	{
		\forall V : \TYPE{Nondegenerate}(\Reals) \. 
		\forall J : \TYPE{MetricComplexStructure}(V) \. 
		\forall t \in \Complex \otimes V \. \NewLine \. 
		t    \sigma_H  = 
		\overline{t} \; \sigma
	}
	\Assume{n}{\Nat}
	\Assume{x}{n \to \ker(\id - \omega_J)}
	\Assume{z}{(n-1) \to \ker(\id - \omega_J)}
	\Conclude{[n.*]}
	{
		\bd \FUNC{hermitianSubstitution}
		\bd \FUNC{hermitianProduct}
		\bd^{-1} \FUNC{substitution}
		\bd^{-1} \FUNC{hermitianProduct}
	}
	{
		\NewLine :
		\left\langle  (t  \; \sigma_H)    
		\bigwedge^n_{i=1} x_i, \bigwedge^{n-1}_{i=1} y_i 
		\right\rangle_H =
		\left\langle   
		\bigwedge^n_{i=1} x_i, t \wedge \bigwedge^{n-1}_{i=1} y_i 
		\right\rangle_H =
		\left\langle
		\bigwedge^n_{i=1} x_i, 
		\overline{t} \wedge \bigwedge^{n-1}_{i=1} \overline{y}_i 
		\right\rangle =
		\left\langle
		( \overline{t} \; \sigma)\bigwedge^n_{i=1} x_i, 
		 \bigwedge^{n-1}_{i=1} \overline{y}_i 
		\right\rangle = \NewLine = 
		\left\langle
		( \overline{t} \;  \sigma)\bigwedge^n_{i=1} x_i, 
		 \bigwedge^{n-1}_{i=1} y_i 
		\right\rangle 
	}
	\Derive{[*]}{
		\bd \TYPE{HermitianProduct}
		\bd \TYPE{Nondegenerate}(V)
		\bd \FUNC{exteriorAlgebra}}
	{\LOGIC{This}}
	\EndProof
}\Page{
	\Theorem{SpinRepresentationOfVectorsIsHermitianSymmetric}
	{
		\forall V : \TYPE{Nondegenerate}(\Reals) \. \NewLine \.
		\forall J : \TYPE{MetricComplexStructure}(V) \. 
		\forall v \in  V \.
		S_J(v \; \mathbf{i}) : \TYPE{Symmetric} 
		\Big( \ker^\wedge(\id - \omega_J), 
		\langle \cdot, \cdot \rangle_H \Big)
	}
	\Say{a}{ \frac{1}{2}\Big( 1 \otimes v + \mathrm{i} \otimes (v \; J)  \Big)}
	{  \Complex \otimes V   }
	\Say{b}{ \frac{1}{2}\Big( 1 \otimes v - \mathrm{i} \otimes (v \; J)  \Big)}
	{  \Complex \otimes V   }
	\Say{[1]}{\ByConstr a \ByConstr b}{ \overline{a} = b }
	\Assume{t,s}{\ker^\wedge(\id - \omega_J)}
	\Conclude{\Big[(t,s).*\Big]}
	{
		\ByConstr S_J
		\bd \L\Big((\Complex \otimes V)^\wedge,
			(\Complex \otimes V)^\wedge;\Complex\Big)
		\Big( \langle \cdot, \cdot \rangle_H \Big)
		\ByConstr^{-1} a \ByConstr^{-1} b
		\bd^{-1} \sigma_H(a)
		\NewLine :
		\THM{HermitianSubstitution}^2(b)(a)[1]
		\bd \sigma_H(a)
		\bd \L\Big((\Complex \otimes V)^\wedge,
			(\Complex \otimes V)^\wedge;\Complex\Big)
		\Big( \langle \cdot, \cdot \rangle_H \Big)
		\ByConstr^{-1} S_J
	}
	{
		\NewLine :
		\langle  S_J(v \; \mathbf{i})t, s \rangle_H = 
		\langle a \wedge t, s \rangle_H + 
		\langle \sigma(b)t, s \rangle_H =
		\langle t, \sigma_H(a) s \rangle_H +
		\langle \sigma_H(a)t,  s \rangle_H =
		\langle t, \sigma(b) s \rangle_H +
		\langle t,  a \wedge s \rangle_H = \NewLine =
		\langle  t, S_J(v \; \mathbf{i})s \rangle_H 
	}
	\Derive{[*]}{\bd^{-1}\TYPE{Symmetric}}{\LOGIC{This}}
	\EndProof
	\\
	\Theorem{SpinRepresentationOfVectorsIsQuasiquadric}
	{
		\forall V : \TYPE{Nondegenerate}(\Reals) \. \NewLine \.
		\forall J : \TYPE{MetricComplexStructure}(V) \. 
		\forall v \in V \.
		\forall s,t \in \ker^\wedge(\id - \omega_J) \. \NewLine \.
		\Big\langle S_J(v \mathbf{i})(t), S_J(v\mathbf{i})(s) \Big\rangle_H =
		\langle v, v \rangle \langle t, s\rangle_H
	}
	\Say{[1]}
	{
		\bd \LALGE{\Reals}\Big( \CL(V), 
		\End_{\VS{\Complex}}\big(\ker^{\wedge}(\id - \omega_V)\big) \Big)
		(S_J)
		\bd \CLIF(\Reals)\Big(\CL(V)\Big) \NewLine
		\bd \LALGE{\Reals}\Big( \CL(V), 
		\End_{\VS{\Complex}}\big(\ker^{\wedge}(\id - \omega_V)\big) \Big)
		(S_J)	
	}
	{
		S^2_J(v\mathbf{i}) =
		S_J(v \mathbf{i})^2 = 
		S_J\Big( \langle v, v \rangle e_{\CL(V)} \Big) =
		\langle v, v \rangle \id
	}
	\Conclude{[*]}{
		\THM{SpinRepresentationOfVectorsIsHermitianSymmetric}(V,J,v)
		[1] \NewLine :
		\bd \L\Big((\Complex \otimes V)^\wedge,
			(\Complex \otimes V)^\wedge;\Complex\Big)
		\Big( \langle \cdot, \cdot \rangle_H \Big)	
	}
	{
		\NewLine :
		\Big\langle S_J(v \mathbf{i})(t), S_J(v\mathbf{i})(s) \Big\rangle_H =
		\Big\langle S_J^2(v \mathbf{i})(t), s \Big\rangle_H =
		\Big\langle \langle v, v \rangle t, s \Big\rangle_H =  
		\langle v, v \rangle \langle t, s \rangle_H 
	}
	\EndProof
	\\
	\Theorem{SphereSpinRepresentationIsUnitary}
	{
		\forall V : \TYPE{Nondegenerate}(\Reals) \. 
		\forall J : \TYPE{MetricComplexStructure}(V) \. \NewLine \. 
		S_J\Big(\mathbb{S}_V\Big)\subset\U\Big(\ker^\wedge(\id-\omega_J)\Big)
	}
	\NoProof
	\\
	\DeclareFunc{evenSpinSpace}
	{
		\prod V : \TYPE{Nondegenerate}(\Reals) \. 
		\prod J : \TYPE{MetricComplexStructure}(V) \. \NewLine \. 
		\TYPE{VectorSubspace}\Big(\ker^\wedge(\id - \omega_J)\Big)
	}
	\DefineNamedFunc{evenSpinSpace}{}{V_J^0}
	{ \sum^\infty_{n = 0 } \Big( \ker^\wedge(\id - \omega_J)\Big)_{2n}  }
	\\
	\DeclareFunc{oddSpinSpace}
	{
		\prod V : \TYPE{Nondegenerate}(\Reals) \. 
		\prod J : \TYPE{MetricComplexStructure}(V) \. \NewLine \. 
		\TYPE{VectorSubspace}\Big(\ker^\wedge(\id - \omega_J)\Big)
	}
	\DefineNamedFunc{evenSpinSpace}{}{V_J^1}
	{ \sum^\infty_{n = 0 } \Big( \ker^\wedge(\id - \omega_J)\Big)_{2n+1}  }
}
\Page{
	\Theorem{EvenCliffordElementsPreservesSpinSpaces}
	{
		\forall V : \TYPE{Nondegenerate}(\Reals) \. \NewLine \.
		\forall J : \TYPE{MetricComplexStructure}(V) \. 
		V_J^0, V_J^1 : \TYPE{Invariant}\Big( S_J\big( \CL_0(V)  \big) \Big)
	}
	\NoProof
	\\
	\DeclareFunc{evenHalfSpinRepresentation}
	{
		\prod V : \TYPE{Nondegenerate}(\Reals) \. 
		\prod J : \TYPE{MetricComplexStructure}(V) \. \NewLine \. 
		\CL_0(V) \Arrow{\LALGE{\Reals}} \End_{\VS{\Complex}}(V_J^0)
	}
	\DefineNamedFunc{evenHalfSpinRepesentation}{x}{S^0_J(x)}
	{ \Big(S_J(x)\Big)_{V_J^0} }
	\\
	\DeclareFunc{oddHalfSpinRepresentation}
	{
		\prod V : \TYPE{Nondegenerate}(\Reals) \. 
		\prod J : \TYPE{MetricComplexStructure}(V) \. \NewLine \. 
		\CL_0(V) \Arrow{\LALGE{\Reals}} \End_{\VS{\Complex}}(V_J^1)
	}
	\DefineNamedFunc{oddHalfSpinRepesentation}{x}{S^1_J(x)}
	{ \Big(S_J(x)\Big)_{V_J^1} }
	\\
	\Theorem{EvenHalfSpinRepresentationIsIso}
	{
		\forall V : \TYPE{Nondegenerate}(\Reals) \. \NewLine \.
		\forall J : \TYPE{MetricComplexStructure}(V) \. 
		S_J^0 : \CL_0(V) \ToIso{\LALGE{\Reals}} \End_{\VS{\Complex}}(V_J^0)	     }
	\NoProof
	\\
	\Theorem{OddHalfSpinRepresentationIsIso}
	{
		\forall V : \TYPE{Nondegenerate}(\Reals) \. \NewLine \.
		\forall J : \TYPE{MetricComplexStructure}(V) \. 
		S_J^1 : \CL_0(V) \ToIso{\LALGE{\Reals}} \End_{\VS{\Complex}}(V_J^1)	     }
	\NoProof
}
\newpage
\subsection{Radon-Hurwitz Number}
\Page{
	\Theorem{RadonHurwitzOrthogonalSystem}
	{
		\forall n, k \in \Nat \.
		\forall \rho : 
		\TYPE{OrthogonalRepresentation}
		\Big(\CL_k(-), \End_{\VS{\Reals}}(\Reals^n)\Big) \. 
		\NewLine \. 
		\forall e : \TYPE{OrthogonalBasis}(\Reals^k) \.
		\forall a : \in \mathbb{S}^{n-1} \.
		a \oplus 
		\rho(e \; \mathbf{i})(a) : \TYPE{Orthonormal}(\Reals^n)          
	}
	\Say{\sigma}{\rho(e \; \mathbf{i})}
	{
		k \to \End_{\VS{\Reals}}(\Reals^n) 
	}
	\Say{[1]}{
		\bd \LALGE{\Reals}\Big(\CL_k(-), \End_{\VS{\Reals}}(\Reals^n)\Big) 
		\bd \TYPE{OrhogonalBasis}(\Reals^k)(e)
		\bd \CLIF{\Reals}\Big(\CL_k\Big)
	}
	{
		\NewLine :
		\forall i,j \in k \. 
		\sigma_i\sigma_j + \sigma_j\sigma_i = -2\delta^i_j
	}
	\Say{v}{\sigma(a)}{k \to \Reals^n}
	\Assume{i}{ k }
	\Say{[2]}{ 	
		\ByConstr v_i
		\ByConstr \sigma_i
		\bd \TYPE{OrthogonalRepresentation}(\rho)
		\THM{NegativelyOrthogonalRepresented} 
		\ByConstr v_i
		\ByConstr 
	}
	{
		\NewLine : 
		\langle v_i, a \rangle = 
		\langle \sigma_i(a), a \rangle = 
		\langle \rho(e_i \; \mathbf{i})(a), a \rangle = 
	       -\langle  a, \rho(e_i \; \mathbf{i})(a) \rangle =
	       -\langle a, \sigma_i(a) \rangle = 
	       -\langle v_i, a \rangle 
	}
	\Conclude{[i.*]}{[2]-[2] }{\langle v_i, a\rangle = 0}
	\Derive{[2]}{I(\forall) }
	{
		\forall i \in n \. \langle v_i, a \rangle = 0
	}
	\Assume{i,j}{k}
	\Assume{[3]}{i \neq j}
	\Say{[4]}{	
		\ByConstr v
		\ByConstr \sigma
		\THM{NegativelyOrthogonalRepresented} 
		\bd \LALGE{\Reals}\Big(
			\CL_k(-), 
			\End_{\VS{\Reals}}(\Reals^n)
		\Big)
		[1]
		\NewLine                     
		\THM{NegativelyOrthogonalRepresented} 
		\ByConstr^{-1} v
		\bd \TYPE{Symmetric}
	}
	{
		\NewLine :
		\langle  v_i, v_j \rangle =
		\Big\langle \sigma_i(a),\sigma_j(a) \Big\rangle =
		\Big\langle  \rho(e_i  \mathbf{i})(a), 
			     \rho(e_j  \mathbf{i})(a)  
		\Big\rangle =
		-\Big\langle  
			    \rho(e_j  \mathbf{i})
			    \rho(e_i  \mathbf{i})(a), 
			     a  \rangle
		\Big\rangle =
		-\Big\langle
			\rho\big( 
			(e_i \mathbf{i})(e_j \mathbf{i})a\big), a
		\Big\rangle = \NewLine =
		\Big\langle
			\rho\big( 
			(e_j \mathbf{i})(e_i \mathbf{i})a\big), a	
		\Big\rangle =
		-\Big\langle
			\rho (e_j \mathbf{i}) a,  \rho(e_i \mathbf{i}) a	
		\Big\rangle =
		-\Big\langle
			v_j,  v_i
		\Big\rangle =
		-\Big\langle
			v_i, v_j
		\Big\rangle
	}
	\Conclude{[3.*]}{\frac{[4]-[4]}{2}}{
		\Big\langle v_i, v_j \rangle = 0	
	}
	\Derive{\Big[(i,j).*.1\Big]}
	{I(\Imply)}{(i \neq j) \Imply \langle v_i,v_j\rangle=0  }
	\Conclude{\Big[(i,j).*.1\Big]}
	{
		\ByConstr v
		\ByConstr \sigma_i
		\THM{NegativelyOrthogonalRepresented} 
		\bd \LALGE{\Reals}\Big(
			\CL_k(-), 
			\Reals^n
		\Big)
		[1]
		\bd \mathbb{S}^n(a)
	}
	{
		\NewLine :
		\langle v_i, v_i \rangle = 
		\Big\langle \sigma_i(a), \sigma_i(a) \Big\rangle =
		\Big\langle \rho(e_i \; \mathrm{i})(a),
		\rho(e_i \; \mathrm{i})(a)\Big\rangle =
		-\Big\langle  \rho^2(e_i \; \mathrm{i})(a),
			a \Big\rangle = 
		-\bigg\langle  \rho\Big( (e_i \; \mathrm{i})^2\Big)(a),
			a \bigg\rangle = 
		\langle a,  a  \rangle =
		1
	}
	\DeriveConclude{[*]}{[2]\bd \TYPE{Orthonormal}}
	{\LOGIC{This}  }
	\EndProof
	\\
	\Theorem{RadonHurwitzDimensionBound}
	{
		\forall k \in \Nat \.
		\forall V \in \VS{\Reals} \. \NewLine \.  
		\forall \rho : \TYPE{OrthogonalRepresentation}\Big(\CL_k(-), 
		 V\Big) \.
		\dim V > k
	}
	\NoProof
	\\
	\DeclareFunc{numberOfRadonHurwitz}{\Int_+ \to \Int_+}
	\DefineNamedFunc{numberofRadonHurwitz}
	{n}{K(n)}{ 
		 \NewLine \de
		\max \Big\{ k \in \Nat : \exists 
		\rho : 
		\TYPE{OrthogonalRepresentation}\big(\CL_k(-),
		End_{\VS{\Reals}}(V)\big)  
		\Big\}
	}
	\\
	\Theorem{RadonHurwitzBound}
	{
		\forall n \in \Nat \.
		K(n) < n 
	}
	\NoProof
}
\Page{
	\Theorem{RadonHurwitzRecurrentRelation}
	{
		\forall n \in \Nat \. K(16n) = K(n) + 8 
	}
	\Say{k}{K(n)}{\Nat}
	\Say{\rho}{\bd K(n) \ByConstr k}
	{ 
		\TYPE{OrthogonalRepresentation}   
		\Big( \CL_{16k}(-), \Reals^{n} \Big)
	}
	\Say{[1]}{\THM{BottPeriodicity}(k)}
	{ 
		\CL_{k+8}(-) \cong_{\LALGE{\Reals}} 
		\CL_k(-) \otimes \End_{\VS{\Reals}}(\Reals^{16} )  
	}
	\Say{  \varphi  }{\bd \TYPE{Isomorphic}}
	{
		\CL_{k+8}(-) \Arrow{\LALGE{\Reals}} 
		\CL_k(-) \otimes \End_{\VS{\Reals}}(\Reals^{16})
	}
	\Say{R}{\varphi(\id \otimes \rho)}
	{
		\TYPE{OrthogonalRepresentation}
		\Big( \CL_{k + 8}, \Reals^{16n} \Big) 
	}
	\Say{[2]}{\bd k(16n)(R)}{k(16n) \ge k(n) + 8}
	\Assume{t}{\Nat}
	\Assume{[3]}{t > 8}
	\Assume{R}{
		\TYPE{OrthogonalRepresentation}(\CL_{t}(-),\Reals^{16n})
	}
	\Say{\Big( \rho',[4]\Big)}
	{
		\THM{BottPeriodicity}(t-8)
		\THM{TensorRepresentationEquivalens}(R)
	}
	{
		\NewLine :
		\sum \rho' :
		\TYPE{OrthogonalRepresentation}
		\Big(
			\CL_{t-8}(-), 
			\Reals^{n}
		\Big) \.
		\rho' \otimes \id \cong R
	}
	\Conclude{[t.*]}{\bd k [4]}{ t \le k + 8}
	\DeriveConclude{[*]}{\THM{DoubleIneq}[2]}{k(16n) = k(n) + 8}
	\EndProof
	\\
	\Theorem{RadonHurwitzNumberLittleBound}
	{
		\forall b \in  3 \.
		\forall c : \TYPE{Odd} \.
		k(2^b c) < 8 
	}
	\Assume{[1]}{k(2^b c) \ge 8}
	\Say{[2]}{\bd \Int_+ [1]}{ k(2^b c) - 8 \in \Int_+}
	\Say{\Big(n,[3]\Big)}{\THM{RadonHurwitzRecurrentRelation}[2]} 
	{  n * 16 = 2^b c   }
	\Conclude{[1.*]}{[3]\bd b \bd c \THM{MainTheoremOfArithmetics}}
	{ \bot  }
	\DeriveConclude{[*]}{E(\bot)\bd \TYPE{GreaterOrEqual}}
	{ k(2^b c) < 8  }
	\\
	\Theorem{RadonHurwitzNumberExpression}
	{
		\forall a \in \Nat \.
		\forall b \in \ 3 \.
		\forall c : \TYPE{Odd} \.
		K(16^a 2^b c) = 8a + 2^b - 1
	}
	\NoProof
}
\newpage
\subsection{Towards Enumeration of Orthonormal Frames}
\Page{
	\DeclareType{OrthogonalFamily}
	{
		\prod k : \Field \.
		\prod V : \OVS(k) \.
		\prod n \in \Nat \. 
		?\Big(n \to \TYPE{Skew}(V)\Big)
	}
	\DefineType{\sigma}{OrthogonalFamily}
	{  
		\forall i,j \in n \. \sigma_i \sigma_j + \sigma_j \sigma_i = 
		-2\delta^i_j \id 
	}
	\\
	\Theorem{OrhogonalFamilyInnerProduct}
	{
		\forall k : \TYPE{Numeric} \.
		\forall V : \OVS(k) \.
		\forall n \in \Nat \. \NewLine \. 
		\forall \sigma : \TYPE{OrthogonalFamily}(V,n) \.
		\forall v \in V \.
		\forall i,j \in n \.
		\langle \sigma_i v , \sigma_j v \rangle = 
		\delta^i_j \langle v, v \rangle
	}
	\NoProof
	\\
	\DeclareFunc{orthogonalMultiplication}
	{
		\forall k : \TYPE{Numeric} \.
		\forall V : \OVS(k) \And \FDVS{k}\. \NewLine \. 
		\TYPE{Orthonormal}(V) \to
		 \TYPE{OrthogonalFamily}(\dim V - 1,V) \to
		\L(V,V;V)
	}
	\DefineNamedFunc{orthogonalMultiplication}{e,\sigma,v,u}
	{v \odot_{e,\sigma} u}{v_1 u + \sum^n_{i=2} v_i \sigma_{i-1}(u)}   
	\\
	\DeclareType{OrthogonalMultiplication}
	{
		\prod k : \TYPE{Numeric} \.
		\prod V : \OVS(k) \And \FDVS{k} \NewLine \.
		?\L(V,V;V)
	}
	\DefineType{\mu}{OrthogonalMultiplication}
	{
		\forall v,u \in V \.
		\Big\langle\mu(v,u) ,\mu(v,u)\Big\rangle = 
		\langle v, v \rangle \langle u, u \rangle
		\And \exists v \in \mathbb{S}_V : \mu(v,\cdot) = \id
	}
	\\
	\Theorem{OrthogonalMultiplicationProperty}
	{
		\forall k : \TYPE{Numeric} \. 
		\forall V : \OVS(k) \And \FDVS{k} \. \NewLine \.
		\forall e : \TYPE{Orthonormal}(V) \.  
		\forall \sigma : \TYPE{OrthogonalFamily}(\dim V -1,V) \.
		\odot_{e,\sigma} : \TYPE{OrthogonalMultiplication}
	}
	\NoProof
	\\
	\Theorem{OrthogonalMultiplicationConstruction}
	{
		\forall n \in \Nat \. 
		\forall \mu : \TYPE{OrthogonalMultiplication}(\Reals^{n+1}) \.
		\NewLine \. 
		\exists  e : \TYPE{Orthonormal}(V) \.  
		\exists  \sigma : \TYPE{OrthogonalFamily}(\dim V -1,V) \.
		\odot_{e,\sigma} = \mu
	}
	\NoProof
	\\
	\Theorem{DimensionByOrthogonalMultiplication}
	{
		\forall n \in \Nat \.
		\forall \mu : \TYPE{orthogonalMultiplication}(\Reals^n) \.
		\NewLine \.
		\dim n \in \{ 1,2,4,8\}
	}
	\NoProof
}
\end{document}
