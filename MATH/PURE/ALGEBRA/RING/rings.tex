  \documentclass[12pt]{article}
\usepackage{mathtools}
\usepackage{amsmath}
\usepackage{amsfonts}
\usepackage{amssymb}
\usepackage{ wasysym }
\usepackage{accents}
\usepackage[dvipsnames]{xcolor}
\usepackage[top=20mm, bottom=20mm, left=30mm, right=10mm]{geometry}
%Markup
\newcommand{\TYPE}[1]{\textcolor{NavyBlue}{\mathtt{#1}}}
\newcommand{\FUNC}[1]{\textcolor{Cerulean}{\mathtt{#1}}}
\newcommand{\LOGIC}[1]{\textcolor{Blue}{\mathtt{#1}}}
\newcommand{\THM}[1]{\textcolor{Maroon}{\mathtt{#1}}}
%META
\renewcommand{\.}{\; . \;}
\newcommand{\de}{: \kern 0.1pc =}
\newcommand{\extract}{\LOGIC{Extract}}
\newcommand{\where}{\LOGIC{where}}
\newcommand{\If}{\LOGIC{if} \;}
\newcommand{\Then}{ \; \LOGIC{then} \;}
\newcommand{\Else}{\; \LOGIC{else} \;}
\newcommand{\IsNot}{\; ! \;}
\newcommand{\Is}{ \; : \;}
\newcommand{\DefAs}{\; :: \;}
\newcommand{\Act}[1]{\left( #1 \right)}
\newcommand{\Example}{\LOGIC{Example} \; }
%%STD
\newcommand{\Int}{\mathbb{Z} }
\newcommand{\NNInt}{\mathbb{Z}_{+} }
\newcommand{\Reals}{\mathbb{R} }
\newcommand{\Rats}{\mathbb{Q} }
\newcommand{\Nat}{\mathbb{N} }
\newcommand{\EReals}{\stackrel{\mathclap{\infty}}{\mathbb{R}}}
\newcommand{\ERealsn}[1]{\stackrel{\mathclap{\infty}}{\mathbb{R}}^{#1}}
\DeclareMathOperator*{\centr}{center}
\DeclareMathOperator*{\argmin}{arg\,min}
\DeclareMathOperator*{\argmax}{arg\,max}
\DeclareMathOperator*{\id}{id}
\DeclareMathOperator*{\im}{Im}
\newcommand{\EqClass}[1]{\TYPE{EqClass}\left( #1 \right)}
\newcommand{\Cate}{\TYPE{Category}}
\newcommand{\Func}[2]{\TYPE{Functor}\left( #1, #2 \right)}
\mathchardef\hyph="2D
\newcommand{\Surj}[2]{\TYPE{Surjective}\left( #1, #2 \right)}
\newcommand{\ToInj}{\hookrightarrow}
\newcommand{\ToBij}{\leftrightarrow}
\newcommand{\Set}{\TYPE{Set}}
\newcommand{\du}{\; \triangle \;}
\renewcommand{\c}{\complement}
\renewcommand{\And}{\; \& \;}
%%ProofWritting
\newcommand{\A}{\LOGIC{Assume} \;} 
\newcommand{\As}{\; \LOGIC{as } \;} 
\newcommand{\E}{ \; \LOGIC{Extract} } 
\newcommand{\QED}{\; \square}
\newcommand{\ByDef}{\eth} 
\newcommand{\ByConstr}{\jmath}  
\newcommand{\Alt}{\LOGIC{Alternative} \;}
\newcommand{\CL}{\LOGIC{Close} \;}
\newcommand{\More}{\LOGIC{Another} \;}
\newcommand{\Proof}{\LOGIC{Proof} \; }
%MetricGeometry
\newcommand{\Ball}[3]{ \mathbb{B}^{#1}\left(#2,#3\right) }
\newcommand{\ClBall}[3]{ \overline{ \mathbb{B}}^{#1}\left(#2,#3\right) }
\newcommand{\ToP}{\overset{p}{\to}}
\newcommand{\ToU}{\rightrightarrows}
%LinearAlgebra
%TYPES
\newcommand{\VS}[1]{\TYPE{VectorSpace}\left( #1 \right)}
\newcommand{\Lin}[1]{\mathcal{L}\left( #1 \right)}
\newcommand{\vs}[1]{\mathsf{VS}\left( #1 \right)}
\DeclareMathOperator*{\rank}{rank}
%FUNK
\DeclareMathOperator{\rk}{rank}
\author{Uncultured Tramp} 
\title{Rings.Know}
%Simbpls
\renewcommand{\L}{\mathcal{L}}
%Topology
%TYPES
\newcommand{\TS}{\TYPE{TopologicalSpace}}
%MeasureTheory
%TYPES
\newcommand{\SA}[1]{\TYPE{\sigma \hyph  Algebra}\left( #1 \right) }
\newcommand{\SF}[1]{\TYPE{\sigma \hyph  Finite}\left( #1 \right) }
\newcommand{\CA}[1]{\TYPE{CountablyAdditive}\left( #1 \right) }
\newcommand{\FA}[1]{\TYPE{Charge}\left( #1 \right) }
\newcommand{\LS}{\TYPE{Lebesgue \hyph Stieltjes}}
\newcommand{\DF}{\TYPE{DistributionFunction}}
\renewcommand{\AE}[1]{\quad \mathrm{a\. e\.} \left[#1\right] \,}
\newcommand{\SI}[1]{\TYPE{\sigma \hyph  Ideal}\left( #1 \right) }
%Simbols
\newcommand{\F}{\mathcal{F}}
\renewcommand{\O}{\Omega}
\newcommand{\B}{\mathcal{B}}
\renewcommand{\l}{\lambda}
\renewcommand{\P}{\mathbb{P}}
%Probability
\newcommand{\RA}{\TYPE{RamdomVariable}}
\renewcommand*{\E}{\mathbb{E} \,}
%Ring Theory
%Types
\newcommand{\Ring}{\TYPE{Ring}}
\newcommand{\RING}{\mathsf{Ring}}
\newcommand{\CR}{\TYPE{CommutativeRing}}
\begin{document}
\maketitle
\begin{center}
\end{center}
\tableofcontents
\newpage
\section{Basic Rings}
\subsection{Rings}
\subsection{Ideals}
\subsection{Morphisms}
\subsection{Quotients}
\newpage
\subsection{Polynomials and generating functions}
$$\TYPE{Polynomial} :  \CR \to \CR$$
$$\TYPE{Polynomial}((R) = R[x] \de ( $$
$$ \{ p :  \NNInt \to R : \exists N \in \NNInt : \forall n \in \NNInt : n \ge N \. p_n = 0  \}, $$
$$ p + q \de \Lambda n \in \NNInt \. p_n + q_n,$$
$$  pq \de \Lambda n \in \NNInt \. \sum_{i \in \NNInt} \sum_{j \in \NNInt : i + j = n}  
p_iq_j )$$ 
\\
$$ \FUNC{degree} :: R[x] \to \NNInt | - \infty $$
$$\FUNC{degree}(p) = \deg p \de \If p = 0 \Then -\infty \Else  \max \{ n \in \NNInt : p_n \neq 0  \}  $$
\\
$$ \FUNC{leadingCoeficient} :: R[x] \to R $$
$$\FUNC{leadingCoeficient}(p) = \mathrm{lc}( p) \de \If p = 0 \Then 0 \Else  p_{\deg p} $$
\\
$$\TYPE{Monic} :: ?R[x] $$
$$ p : \TYPE{Monic} \iff \mathrm{lc}(p) = 1 $$
\\
$$ \FUNC{moprhPolyExtension} :: \mathcal{M}_{\RING}(R,S) \to \mathcal{M}_{\RING}(R[x],S[x]) $$
$$
 \FUNC{moprhPolyExtension}(\phi)(p)= p^\phi = \sum^\infty_{i=0} \phi(p_i)x^i 
$$
\\
$$\TYPE{Irreducable} :: ?R[x]$$
$$f : \TYPE{Irreducable}  \iff f \IsNot \TYPE{Unit}([R(x)]) \And \forall p,q \in R[x] :
pq =f \.  p : \TYPE{Unit}([R(x)]) | q :\TYPE{Unit}([R(x)])
$$
\\
\subsubsection{Content and proimitive polinomials}
Assume $R : \TYPE{UFD}$
\\
$$
\TYPE{Content} :R[x] \to ? R
$$
$$
r : \TYPE{Content}(p) \iff r \in C(p) \iff r : \TYPE{GCD}(\{ p_i : i \in \deg p \} )
$$
\\
$$
\TYPE{Primitive} : ?R[x] 
$$
$$
p : \TYPE{Primitive} \iff  C(p) = \{ 1 \}
$$
\newpage
$$
\TYPE{Content} :\mathrm{Frac} \, R[x] \to ? \mathrm{Frac} \, R
$$
\begin{flalign*}
&\TYPE{Content}(f) = C(f) \de \{ u\Pi  | u : \TYPE{Units}(R)\} \\
 &\LOGIC{Where} \\ 
 &\Pi =  \prod_{P : \TYPE{Prime}(R)} P^{e(P)} \\
& e(P) = \min_{i \in \deg f } \{  \exp(f_i,P)\} 
\end{flalign*}
\\
$$
\TYPE{Primitive} :: ? \mathrm{Frac} R[x] 
$$
$$
p : \TYPE{Primitive} \iff  1 \in C(p) 
$$
\\
$
\THM{ContentFact1} :: \forall f \in \mathrm{Frac} \, R \.  \forall a \in C(f) \.  \exists p \in \TYPE{Primitive}( R)  : f = ap
$\\
$Proof \approx $ \\
Let $a = u\Pi$  as in definition of content. Then, by definition of $\Pi$ $$(u\Pi)^{-1}f
 = \sum^{\deg f}_{i=0} r_i x^i = p $$
with each $r_i \in R$ such that $gcd(r) =1$ which means that $p$ is Primitive and in $R[x]\square$. 
\\ \\
$
\THM{ContentFact2} :: \forall p : \TYPE{Primitive}(\mathrm{Frac} \, R) \.   p : \TYPE{Primitive}(\mathrm{Frac} \, R)
$\\
$Proof \approx $\\
We know that $1 \in С(p) $ so by $\THM{ContentFact1}$  $p = 1p$ lies in $R[x]$.
\\ \\
$
\THM{ContentFact3} :: f \in R[x] \iff C(f) \in R
$\\
$Proof \approx $\\
If $f \in R[x]$ when all her coefficients lie in  $R$ so by definition of content in field of fractions $C(f) \in R$. \\
If $C(f) \in R$ when by definition of $C$ no  prime factor of any coefficient of $f$ has strictly negative exponent which is the same as $ f \in R[x]$.
\\ \\
$\THM{GausLemma} :: \forall f, g : \TYPE{Primitive}(R) \. fg : \TYPE{Primitive}(R)  $ \\
$Proof \approx $\\
Assume $f,g$ are primitive polynomials.\\
Assume $fg$ Is not primitive.\\
This means that there exists a prime element $p \in R$ such that $p \in C(fg)$. \\
Let $\pi_p : R \to \frac{R}{(p)}$ denote natural projection. \\
Then $$0 = (fg)^{\pi_p} = f^{\pi_p}g^{\pi_p}$$
but as $f$ and $g$ are primitive $f^{\pi_p}g^{\pi_p} \neq 0$ so we have a contradiction.
\\$\QED$
\newpage
$\THM{ContentProduct} :: \forall   f,g : \mathrm{Frac} \, R [x] \. C(fg) = C(f)C(g)  $ 
\\$\Proof \approx $ \\
For each $a \in C(f)$ and $b \in C(g)$ by $\THM{ContentFact1}$ we write $f = aF$ and $g = aG$ whera $F$ and $G$ are primitive and hence by $\THM{GausLemma} $ $FG$ is primitive. \\
So
$$ C(fg) = C(aFbG) = C(abFG) = ab\TYPE{Unit}(R) = a\TYPE{Unit}(R)b\TYPE{Unit}(R) = C(f)C(g) $$
 $\square$
 \\ 
$\THM{PrimitiveFactorization} :: \forall f \in R[x] \. \forall h \in  \mathrm{Frac} \, R[x] \. \forall p : \TYPE{Primitive}(R) : f = ph \. h \in R[x] $\\
$\Proof \approx$
By $\THM{ContentProduct}$ and using permittivity of $p$ we have
$$
C(f) = C(ph) = C(p)C(h) = C(h)
$$
so by $\THM{contentFact3}$ $h \in R[x]$. 
\\ $\QED$ 
\subsubsection{Irreducibility over field of fractions}
$\TYPE{FactorizationOver} : R[x] \to ?R \to ?(R[x] \times R[x]) $ \\
$(a,b) :\TYPE{FactorizationOver}(f)(S) \iff f = ab \wedge \forall i \in \deg a \. a_i \in S \wedge   \forall i \in \deg b \. b_i \in S  $
\\ \\
$ \TYPE{DegreewiseFactorization} : R[x] \to ?R \to ?(R[x] \times R[x])  $ \\
$(a,b) : \TYPE{DegreewiseFactorization}(f) \iff f = ab \wedge \deg a > 0 \wedge \deg b > 0  $
\\ \\
$ \TYPE{DegreewiseIrrefucable} : ?R[x] $\\
$f : \TYPE{DegreewiseIrreducable} \iff \forall (a,b) \in R[x] \times R[x] \.   (a,b) \IsNot 
\TYPE{DegreewiseFactorization}(f)$
\\ \\
$
\THM{IrreducibilityInFractionsTheorem1} :: \forall f \in R[x] \.   
f : \TYPE{DegreewiseIrreducable}(R) \iff f : \TYPE{DegreewiseIrreducable}( \
\mathrm{Frac} \, R)
$ \\
$ \Proof \approx$ \\
One side is trivial . \\
Now assume that $f$ is degreewise irreducible only in $R$. Assume that $(a,b)$ is a degreewise factorisation of $f$ in  field of fractions. let $\alpha$ be an element from content of $a$ such that $a = \alpha p$ where $p$ is primitive. Then 
$$f = ab = \alpha p b =  (\alpha b) p.$$
As $f \in R[x]$ and $p$ is primitive then by   $\THM{PrimitiveFactorization}$ $(\alpha b) \in R[x]$ and by $\THM{ContentFact2}$ $p \in R[x]$. So $f$ is not irreducible over $R$, a contradiction. \\ $\square$
\\ \\
$
\THM{IrreducibilityInFractionsTheorem1} :: \forall f : \TYPE{Primitive} \, R[x] \.   
f : \TYPE{eIrreducable}(R) \iff f : \TYPE{Irreducable}( \
\mathrm{Frac} \, R)
$ \\
$ \Proof \approx$ \\
If $f$ is not irreducible only in $R$ then one of  her factors must have a degree $0$ which means that $f$ is not primitive. \\
A contradiction. \\
If $f$ is not irreducible only in field of fractions then it must be degreewise reducible and hence reducible in $R$ also. \\
A contradiction. $\QED$ \\

\subsubsection{Division Algorithm}
$
\THM{DivisionAlgorithm} :: \forall f \in R[x] . \forall g : \TYPE{Monic}(R) \.    
\exists q,r \in R[x] : \deg r \le \deg g : f = gq + r
$ \\
$\Proof \approx $ \\
If $\deg g > \deg f$ then just take $q = 0 $ and $r = f$. \\
Otherwise we know that $\deg q =  \deg f - \deg g$ so we can choose   $q_i$ to be unique to be unique solution of linear equation $( (gq)_i = f_i )^{\deg f}_{i = \deg g }$ which always exists as $g$ is monic and $(R,+)$ is a group. \\
So $\deg r \le \deg f $. $\square$. 
\\ \\
$
\THM{RootIsFactor} :: \forall f \in R[x] \. \forall a \in R \. a : \TYPE{Root}(f) \iff x -a : \TYPE{Factor}(f)
$ \\
$\Proof \approx  $ \\    
 Represent $f = q(x - a) + r$. \\
 If $a$ is root of $f$ then $0 = f(a) = q(a - a) + r = r$, so $(x - a)$ is indeed a factor of $f$.  \\
 Another side is trivial $f(a) = q(a - a) = 0$. $\QED$.   
 \\ \\
 $
\THM{MaximalRoots} :: \forall f \in R[x] \.  \#\ker f \le \deg f 
$ \\
$\Proof \approx$
 number of factors of a polynomial cannot exceed her degree and number of roots cannot exceed her degree.
 \\ \\
 
\newpage
\section{Characteristic}
\section{Classes of Commutative rings}
\subsection{Integral Domains}
\subsection{Unique Factorization Domain}
\subsection{PID}
\subsection{Euclidean Domains}
\end{document}
