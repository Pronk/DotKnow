\documentclass[12pt]{article}
\usepackage{mathtools}
\usepackage{amsmath}
\usepackage{amsfonts}
\usepackage{amssymb}
\usepackage{color}
\usepackage{ wasysym }
\usepackage{ stmaryrd }
\renewcommand{\.}{\; . \;}
\newcommand{\de}{: \kern 0.1pc =}
\newcommand{\extract}{\rightarrowtriangle}
\newcommand{\as}{\multimap}
\newcommand{\Module}{\mathrm{Module}}
\newcommand{\problem}{ \textcolor{blue}{\mathtt{problem}}\quad}
\newcommand{\data}{ \textcolor{blue}{\mathtt{data}}\quad}
\newcommand{\pred}{ \textcolor{blue}{\mathtt{predicate}}\quad}
\newcommand{\func}{ \textcolor{blue}{\mathtt{function}}\quad}
\newcommand{\thm}{ \textcolor{blue}{\mathtt{thm}}\quad}
\newcommand{\proof}{ \textcolor{cyan}{\mathtt{proof}}\:}
\newcommand{\Iff}{\mathbf{iff}\:}
\newcommand{\If}{ \mathbf{if}\:}
\newcommand{\cate}{\mathtt{category} \quad}
\newcommand{\Ring}{\mathtt{Ring}}
\newcommand{\Commu}{\mathrm{Commutative}}
\newcommand{\Abel}{\mathtt{Abelean}}
\newcommand{\Ab}{\mathsf{Ab}}
\newcommand{\rmod}{R\hyph\Module}
\newcommand{\Group}{\mathtt{Group} }
\newcommand{\Algebra}{\mathrm{Algebra} }
\newcommand{\ralg}{R\hyph\Algebra }
\newcommand{\Ideal}{\mathrm{Ideal} }
\newcommand{\AlgHomo}{\mathtt{AlgHomo} }
\newcommand{\End}{\mathrm{End} }
\newcommand{\Submod}{\mathrm{Submodule} \; }
\newcommand{\Homo}{\mathtt{Homo} \; }
\newcommand{\Iso}{\mathrm{Isomorphism} \; }
\newcommand{\Lin}{\mathrm{Linear} \;}
\newcommand{\Rmod}{R\hyph\mathsf{Mod} \;}
\newcommand{\Ralg}{R\hyph\mathsf{Alg} \;}
\newcommand{\reals}{\mathbb{R}}
\newcommand{\assume}{\vdash}
\newcommand{\Surj}{\mathrm{Surjection}}
\newcommand{\Inj}{\mathrm{Injection}}
\newcommand{\Bij}{\mathrm{Bijection}}
\newcommand{\proving}{\dashv }
\newcommand{\rationals}{\mathbb{Q}}
\renewcommand{\int}{\mathbb{Z} \;}
\DeclareMathOperator*{\centr}{center}
\DeclareMathOperator*{\argmin}{arg\,min}
\mathchardef\hyph="2D
\title{Modules.Know}
\begin{document}
\maketitle
\parindent=0em 
\begin{flalign*}
&\problem 5.1 \\
&\kern 1pc  \func \mathtt{opposite} :: \Ring \to \Ring \\
&\kern 6pc  \mathtt{opposite} \, R =  R^\circ \de \{R, \bullet \de \Lambda a,b \in R \. b \cdot_R a \}\\
\\
&\kern 1pc \func \mathtt{toOpposite} :: \prod R : \Ring \. R \to R^\circ \\
&\kern 6pc \mathtt{toOpposite} \, r = \pi r \de \: r\\
\\
&\kern 1pc \thm \forall R : \Ring \; \Iff \pi_R : \mathtt{Iso} \. R : \Commu \\
&\kern 2pc \proof =    R : \Ring \vdash \pi_R : \Iso  \vdash \\
&\kern 3pc  a, b \in R \vdash \\
&\kern 4pc  ab = \pi_R \, ab = \pi_R a \pi_R b = a \bullet b = b a \dashv  \\
&\kern 3pc  \to  R : \Commu  \dashv \\
&\kern 2pc  \to  \If \pi_R : \Iso \. R : \Commu  \as (\Rightarrow) \\
&\kern 2pc  R : \Commu  \vdash \\
&\kern 3pc  a, b \in R \vdash \\
&\kern 4pc  \pi_R a \pi_R b = a \bullet b  = b a = a b = \pi_R ab \dashv \\
&\kern 3pc  \to \pi_R : \mathtt{Iso} \vdash \\
&\kern 2pc \to  \If  R : \Commu \. \pi_R : \Iso - (\Rightarrow) \dashv \\
&\kern 1pc  \forall R : \Ring \; \Iff \pi_R : \mathtt{Iso} \. R : \Commu \quad \square
\end{flalign*}
\newpage
\begin{flalign*}
A \mapsto A^\top : \Iso \, \mathcal{M}^n(\reals)\: \mathcal{M}^n(\reals)^\circ
\end{flalign*}
you can define Left $ \rmod $ as $R^\circ \hyph \Module$ \\

\textbf{problem 5.4}
\begin{flalign*}
&\pred \mathtt{Simple} :: ?\rmod \\
&\kern 5.5pc M : \mathtt{Simple} \Leftrightarrow \forall S : \Submod M \. S = \{ 0 \} \vee 
S = M \\
\\ 
&\thm \mathrm{Schur's} \:  \mathrm{Lemma} :: \forall M,N : \mathtt{Simple} 
\.  \forall \phi : \Homo M \: N \. \phi : \Iso \vee \phi = 0  
\end{flalign*}
If $\phi$ is not $\Iso$ or $0$ then either $R \neq \ker \phi \neq \{ 0 \}$, or $R \neq \mathrm{Im} \, \phi \neq \{ 0 \}$. As $\ker \phi$ is also a submodule, first situation contradicts with a simplicity of $M$. And the second situation contradicts the simplicity of $N$ as $\mathrm{Im} \, \phi$  is also a submodule. $\square$ \\


\textbf{problem 5.5 }
\begin{flalign*}
&\thm \mathtt{MorphIsomorph} :: \forall R : \Commu  
\.  \forall M : \rmod \. \mathrm{Hom}_{\Rmod}(R,M) \cong M  
\end{flalign*}
we will define isomorphism map explicitly:
\begin{flalign*}
&\func f :: \mathrm{Hom}_{\Rmod}(R,M) \to M
\\ 
&\kern 5.5pc f \, \phi = \phi(1_R) \\ 
\\
&\func g :: M \to \mathrm{Hom}_{\Rmod}(R,M) 
\\ 
&\kern 5.5pc g \, m \,r = rm 
\end{flalign*}
It is easy to see that both maps are homorphisms:
$$f \; r\phi + s\psi =r\phi(1_R) + s\psi(1_R) = rf \, \phi + sf \, \psi  $$
$$g \, (xm + yn) \, r = r(xm + yn) = rxm + ryn =xrm + yrn = (xg \, m   + yg \, n) \, r $$
Its also easy to prove that maps are inverses
$$f \, (g \,  m) = (g \, m) \, 1_R = 1_R m = m $$
$$\left(g \, (f \, \phi) \right) r = g \, \phi(1_R) \, r = r \phi(1_R) = \phi(r)$$
So this modules are indeed isomorphic. $\square$
\newpage

\textbf{problem 5.6}

Firstly, we will show that a group  with $\rationals$-vector space structure must have all elements all infinite order . Assume that a non-zero element $a$ of finite order $k > 1$ exists. then we know that $ka = 0$, however we also know that $(1 / k  ) (k) a= a $. Which means that \\
$(1 / k  ) 0 \neq 0$ and brings us to a contradiction. 
\\
  
Now assume that $\cdot$ and $*$ are both $\rationals$-vector space structures over abelian group $G$ . Let's  arbitrary select elements $n \in \mathbb{Z}$ and $g \in G$ . Then we can show $b$ multiples of this elements will be equal :
$$n\left((1/n) \cdot g \right) = n \cdot (1 / n) \cdot g = g = n * (1/n) * g = b \left( (1/n) * g \right)$$ 
As our group has infinite order we can factor $b$ out and get an equality $(1/n)\cdot g = (1/n) * g$
from which we can easily derive that \\ $\forall q \in \rationals \. \forall g \in G \. q \cdot g = q * g$ and hence $\cdot = *$. $\square$
\\

\textbf{problem 5.9}\\
$\forall R : \Commu \. \forall  M : \rmod \.  \mathrm{End}_{\Rmod} (M) : \ralg$ \\
Lets repeat definition of $\ralg$.
\begin{flalign*}
&\data\Algebra :: \prod R : \Commu \. \sum S : \Ring \. \Homo  R  \; \; \centr S\\
\end{flalign*}
So we will define this homomorphism in following way:
\begin{flalign*}
&\func f :: R \to \centr \mathrm{End}_{\Rmod} (M) \\
& \kern 5.5pc f(r) \, m = rm
\end{flalign*}
It's easy to sea tht this function i indeed a function into center as \\
$\forall r \in R \. \forall \phi \in  \mathrm{End}_{\Rmod} M \. \forall m \in M$:
$$mf(r)\phi = \phi(rm) = r\phi(m) = m\phi f(r)$$
It's also easy to check that this function is a homomorphism
$\forall a,b,c \in R \. \forall m \in M$:
$$f(ab + c)m = (ab + c)m = abm + cm = f(a)f(b)m + f(c)m = (f(a)f(b) + f(c))m $$ 
So we can claim $ (\mathrm{Aut}_{\Rmod} (M), f) : \ralg$. 
\newpage
$\mathcal{M}^n(R)$ is an $\ralg$ in natural order as $\mathcal{M}^n(R) \cong \mathrm{End}_{\Rmod} R^n$. $\square$ \\

\textbf{problem 5.10} \\
$ \forall R : \Commu \. M : \mathtt{Simple} \, R \. \mathrm{End}_{\Rmod} M : \mathtt{Division}$
As $M$ is Simple we deduce that every its endomorphism is either a zero map or automorphism. And each automorphism is a bijection, hence has an inverse. So, every non-zero element of $\mathrm{End}_{\Rmod} M$ has an inverse which makes it into a division algebra. $\square$ \\
\textbf{problem 5.11} \\
\begin{multline}
\forall R : \Commu \. M : \rmod \. \\
\. \{f : \Homo  R[x]  \; \mathrm{End}_{\mathsf{Abb}} M    \} \cong \mathrm{End}_{\Rmod} M 
\end{multline}
We will begin with constructing bijection math explicitly.
\begin{flalign*}
&\func f:: \Homo  R[x]  \; \mathrm{End}_{\mathsf{Abb}} \, M \to  \mathrm{End}_{\Rmod} M \\
&\kern 5pc  f \, \phi \, m = \phi(x) m \\
&\func g ::  \mathrm{End}_{\Rmod} M  \to \Homo  R[x]  \; \mathrm{End}_{\mathsf{Abb}} \, M  \\
&\kern 5pc  g \, \phi \, p \, m = \sum_{n=0}^\infty p_n \phi^n(m) \\
& m \phi g f = m x \phi g = \phi(m) = m\phi \\
&  m p\phi  f g  = m p \phi(x) g =  m = \sum_{n=0}^\infty p_n (\phi(x))^n(m) = m p \phi
 \\  
&g = f^{-1}  \quad \square\\
\end{flalign*}
\newpage 
\textbf{problem 5.13} ::
\begin{multline*}
\forall R : \mathrm{IntegralDomain} \. \forall I : \mathrm{Principle} \; R \. \If I \neq (0) \. I \cong_{\Rmod} R
\end{multline*}
\begin{flalign*}
&\assume R : \mathrm{IntegralDomain} \as (ID)  \\
&\assume I : \mathrm{Principle} \, R  \to \exists a \in R \. I = (a) \extract a \in R ; I = (a) \\
&\assume I \neq 0 \to a \neq 0 \\
&\func f : R \to I \\
&\kern 5pc f(r) = ra \\
&\assume x,y,z \in R \\
& f(xy + z) = (xy +z)a = xya + za = xf(y) + f(z) \proving \\
& \begin{rcases}
\proving f : \Lin R \, I  \\
\begin{rcases}
I = (a) = f[R] \to f : \Surj  \\
(ID) \wedge a \neq 0 \to f : \Inj
\end{rcases} \to f : \Bij
\end{rcases} \to f : \Iso \to \\
&\to  I \cong_{\Rmod} R \proving_3 \\
&\proving_3  \forall R : \mathrm{IntegralDomain} \. \forall I : \mathrm{Principle} \; R \. \If I \neq (0) \. I \cong_{\Rmod} R \quad \square
\end{flalign*}
\newpage
\textbf{problem 5.14} ::
$$\forall M : \rmod \. \forall N, P :\Submod M \. N + P : \Submod M$$
This is true by  distributivity $r(a+b) = ra + rb$
$$\forall M : \rmod \. \forall N, P :\Submod M \. N \cap P : \Submod M$$
\begin{flalign*}
&\assume r \in R \assume a \in N \cap P \\
&\begin{rcases}
a \in N \to ra \in N \\ 
a \in P \to ra \in P
\end{rcases} \to  ra \in N \cap P\prec
\end{flalign*}
$$\frac{N + P}{N} \cong ' \frac{N}{N} + \frac{P}{P \cap N} = 0 + \frac{P}{P \cap N} = \frac{P}{P \cap N} $$
? $\square$ \\
\textbf{problem 5.15}
$$I(\frac{R}{J}) \cong \frac{I}{J \cap I} \cong \frac{I + J}{J} \quad \square$$
\textbf{problem 5.16} :: \\
$$\forall R : \Commu \. \forall M : \rmod \. \forall a : \mathrm{Nilpotent} \, R
\. \Iff \, M = 0 \. aM = M$$
$(\Leftarrow)$
Simply, $a0 = 0 $, hence $aM = M$ . \\
$(\Rightarrow)$  Assume that $aM = M$. \\
As $a$ is nilpotent $\exists n \in \int_{+} \. a^n=0$. \\
This means that with application of simple induction \\ $0 = a^nM = a^{n-1}M = \ldots = aM = M $ \\
$\square$  \\
\newpage
\textbf{problem 5.17} \\
\begin{flalign*}
&\func \mathrm{Rees} :: \prod R : \Commu \. \Ideal \, R \to \ralg \\
&\kern 5pc \mathrm{Rees} \, I = (\bigoplus^\infty_{i=0} I^i, \Lambda \, r \. [r] \oplus \bigoplus^\infty_{i=0} 0) \\
&\thm \forall R : \Commu \. \forall a : \mathrm{NZD} \. \mathrm{Rees} \, R \, (a) \cong_{\Ralg} R[x]
\end{flalign*}
After problem 5.13 we know that $(a) \cong R$, furthermore applying the same result we can show that $(a)^n \cong (a)^{n-1} \cong \ldots (a) \cong R$  which provides us with the sequence of isomorphisms functions 
$$ f : \prod n \in \int_{+} \. (a)^n \leftrightarrow R $$
Then we construct a map  $\phi : v \mapsto \sum^\infty_{i=0}f_n(v_i)x^i$. Inherently, this map is a bijection.
$$\phi(vw) =\phi \left(\bigoplus^\infty_{n=0} \sum_{i+j=n} v_iw_j \right) = 
\sum_{n=0}^\infty f_n(\sum_{i+j=n} v_iw_j) = \sum_{n=0}^\infty \sum_{i+j=n} f_n(v_iw_j)x^n =$$
$$=\sum_{n=0}^\infty \sum_{i+j=n} f_n(bca^n)x^n=\sum_{n=0}^\infty \sum_{i+j=n}  bcx^n=\sum_{n=0}^\infty \sum_{i+j=n} f_i(v_i)f_j(w_j)x^n = \phi(v)\phi(w)$$
$$\phi(v+w) = \phi \left( \bigoplus^\infty_{n=0} v_n + w_n \right)= \sum^\infty_{n=0} f_n(v_n + w_n)x^n = \sum^\infty_{n=0} f_n(v_n)x^n + \sum^\infty_{n=0} f_n(w_n)x^n = \phi(v) + \phi(w)$$
$$\phi(rw) = \phi\left(\left([r] \oplus \bigoplus^\infty_{i=0} 0\right)w\right) = \phi\left([r] \oplus \bigoplus^\infty_{i=0}\right)\phi(w) = r\phi(w)$$
So we can say that $\phi$ is $\ralg$. As $\phi$ is also a bijection we can claim it to be an isomorphism. $\square$
\end{document}
