\documentclass[12pt]{article}
\usepackage{mathtools}
\usepackage{amsmath}
\usepackage{amsfonts}
\usepackage{amssymb}
\usepackage{wasysym}
\usepackage{accents}
\usepackage[hidelinks]{hyperref}
\usepackage[dvipsnames]{xcolor}
\usepackage[top=20mm, bottom=20mm, left=30mm, right=10mm]{geometry}
\begin{document}
1) $$ [F : \mathbb{Q}] = \phi(9) = 6 $$ as $ F $  is cyclotomic extension, which means that 
 $$ \mathrm{Gal}(F/\mathbb{Q}) =[\mathbb{Z}/9\mathbb{Z}]^* $$ .

2) It is possible to write minimal polynomial $ p \in \mathbb{Q}(\alpha)[ X]$ as :

 $$
   p(X)    =  ( X - e^{2\pi i / 9})( X - e^{ - 2 \pi i / 9 }   ) =  X^2 -  2 \cos( 2 \pi / 9  ) X + 1
 $$ 

as conjugation is a homomorphism of $ \mathbb{Q} $ - algebras and all non rational reals  in $F$  are result of conjugation of elements, which are themselves are $\mathbb{Q}$-linear combinations of basis elements generated by powers of $\zeta$, which already contain conjugates.  
That is 
 $$
    \mathbb{Q}(\alpha)  = F \cap \mathbb{R}
 $$ 

3)    $X^9 - 5$ is the minimal polynomial for $ \gamma $ so $[ L : \mathbb{Q} ] = 9$.   The subfield $K$ must have degree which divides 9 but is not 1 or 9. So degree of $K$ is 3.  $ \mathbb{Q}(\gamma^3)$  has degree 3 and is a subfield of $L$ so it must, indeed, be equal  to  $K$ (consider minimal polynomial $X^3 - \gamma^3$ of $\gamma$ over $K$.) 

4) In case of nontrivial intersection $F \cap L = K$ as it must be nontrivial subextension of $L$. 
 It is also know that extension $F$  is Galois as cyclotomic extension, however $K$ does not split any polynomials, so it is not normal and hence not Galois. And As every subextension of $F$ must be Galois (abelean Galois group) it is clear that $F \cap L = \mathbb{Q}$.

This means that $[M : \mathbb{Q}] = [ L : \mathbb{Q}][F : \mathbb{Q}] = 54$ as $ M = \mathbb{Q}(\zeta, \gamma)$ .

5) $H$ is a subgroup inherited from $L$, the subgroup generated by action $ \sqrt[9]{ 5 } \mapsto \sqrt[9]{25}$ . Another subgroup must be inherited from $F$ . This soubgroup is generated by  action $ \zeta \mapsto \zeta^2 $ . so $H \cong \mathbb{Z}_9^+$ and $S \cong \mathbb{F}_6^+$ . This means that $|G| > 9  $, so the galois group is not commutative. 

6)  All subextensions of degree 2  of $M$ must be also subextensions of $F$. The only  such subextension I know is $\mathbb{Q}(\alpha)$

7) There are two subextensions of $M$ of degree 3 but only one of them is Galois. This subextension must belong to $F$. So, it is $\mathbb{Q}(\xi)$ where $\xi$ is the third primitive root of unity.
\end{document}
