  \documentclass[12pt]{article}
\usepackage{mathtools}
\usepackage{amsmath}
\usepackage{amsfonts}
\usepackage{amssymb}
\usepackage{wasysym}
\usepackage{accents}
\usepackage[hidelinks]{hyperref}
\usepackage[dvipsnames]{xcolor}
\usepackage[top=20mm, bottom=20mm, left=30mm, right=10mm]{geometry}
%Markup
\newcommand{\TYPE}[1]{\textcolor{NavyBlue}{\mathtt{#1}}}
\newcommand{\FUNC}[1]{\textcolor{Cerulean}{\mathtt{#1}}}
\newcommand{\LOGIC}[1]{\textcolor{Blue}{\mathtt{#1}}}
\newcommand{\THM}[1]{\textcolor{Maroon}{\mathtt{#1}}}
%META
\renewcommand{\.}{\; . \;}
\newcommand{\de}{: \kern 0.1pc =}
\newcommand{\extract}{\LOGIC{Extract}}
\newcommand{\where}{\LOGIC{where}}
\newcommand{\If}{\LOGIC{if} \;}
\newcommand{\Then}{ \; \LOGIC{then} \;}
\newcommand{\Else}{\; \LOGIC{else} \;}
\newcommand{\IsNot}{\; ! \;}
\newcommand{\Is}{ \; : \;}
\newcommand{\DefAs}{\; :: \;}
\newcommand{\Act}[1]{\left( #1 \right)}
\newcommand{\Example}{\LOGIC{Example} \; }
%%STD
\newcommand{\Int}{\mathbb{Z} }
\newcommand{\NNInt}{\mathbb{Z}_{+} }
\newcommand{\Reals}{\mathbb{R} }
\newcommand{\Rats}{\mathbb{Q} }
\newcommand{\Nat}{\mathbb{N} }
\newcommand{\EReals}{\stackrel{\mathclap{\infty}}{\mathbb{R}}}
\newcommand{\ERealsn}[1]{\stackrel{\mathclap{\infty}}{\mathbb{R}}^{#1}}
\DeclareMathOperator*{\centr}{center}
\DeclareMathOperator*{\argmin}{arg\,min}
\DeclareMathOperator*{\argmax}{arg\,max}
\DeclareMathOperator*{\id}{id}
\DeclareMathOperator*{\im}{Im}
\newcommand{\EqClass}[1]{\TYPE{EqClass}\left( #1 \right)}
\newcommand{\Cate}{\TYPE{Category}}
\newcommand{\Morph}[3]{\mathcal{M}_{#1}(#2,#3)}
\newcommand{\Func}[2]{\TYPE{Functor}\left( #1, #2 \right)}
\mathchardef\hyph="2D
\newcommand{\Surj}[2]{\TYPE{Surjective}\left( #1, #2 \right)}
\newcommand{\ToInj}{\hookrightarrow}
\newcommand{\ToBij}{\leftrightarrow}
\newcommand{\Set}{\TYPE{Set}}
\newcommand{\du}{\; \triangle \;}
\renewcommand{\c}{\complement}
\renewcommand{\And}{\; \& \;}
%%ProofWritting
\newcommand{\A}{\LOGIC{Assume} \;} 
\newcommand{\As}{\; \LOGIC{as } \;} 
\newcommand{\E}{ \; \LOGIC{Extract} } 
\newcommand{\QED}{\; \square}
\newcommand{\ByDef}{\eth} 
\newcommand{\ByConstr}{\jmath}  
\newcommand{\Alt}{\LOGIC{Alternative} \;}
\newcommand{\CL}{\LOGIC{Close} \;}
\newcommand{\More}{\LOGIC{Another} \;}
\newcommand{\Proof}{\LOGIC{Proof} \; }
%MetricGeometry
\newcommand{\Ball}[3]{ \mathbb{B}^{#1}\left(#2,#3\right) }
\newcommand{\ClBall}[3]{ \overline{ \mathbb{B}}^{#1}\left(#2,#3\right) }
\newcommand{\ToP}{\overset{p}{\to}}
\newcommand{\ToU}{\rightrightarrows}
%LinearAlgebra
%TYPES
\newcommand{\VS}[1]{\TYPE{VectorSpace}\left( #1 \right)}
\newcommand{\Lin}[1]{\mathcal{L}\left( #1 \right)}
\newcommand{\vs}[1]{\mathsf{VS}\left( #1 \right)}
\newcommand{\LInd}{\TYPE{LinearlyIndependant}}
\DeclareMathOperator*{\rank}{rank}
%FUNK
\DeclareMathOperator{\rk}{rank}
\author{Uncultured Tramp} 
\title{Fields.Know}
%Simbpls
\renewcommand{\L}{\mathcal{L}}
%Topology
%TYPES
\newcommand{\TS}{\TYPE{TopologicalSpace}}
%Algebra
%Abstract 
%Groups
%TYPES
\newcommand{\Group}{\TYPE{Group}}
\newcommand{\Abel}{\TYPE{Abelean}}
%Ring Theory
%Types
\newcommand{\Ring}{\TYPE{Ring}}
\newcommand{\RING}{\mathsf{RING}}
\newcommand{\CR}{\TYPE{CommutativeRing}}
\newcommand{\Irr}{\TYPE{Irreducible}}
\newcommand{\PID}{\TYPE{PrincipleIdealDomain}}
\newcommand{\IntD}{\TYPE{IntegralDomain}}
\newcommand{\Prime}{\TYPE{Prime}}
\newcommand{\DWIrr}{\TYPE{DegreewiseIrreducible}}
%FUNC
\newcommand{\cha}{\mathrm{char}}
\newcommand{\lc}{\mathrm{lc}}
%Fields
%Types
\newcommand{\Field}{\hyperref[Field]{\TYPE{Field}}}
\newcommand{\Superfield}{\hyperref[Superfield]{\TYPE{Extension}}}
\newcommand{\Subfield}{\hyperref[Subfield]{\TYPE{Subfield}}}
\newcommand{\RP}{\hyperref[RelativelyPrime]{\TYPE{RelativelyPrime}}}
\newcommand{\Root}{\hyperref[Root]{\TYPE{Root}}}
\newcommand{\Splits}{\hyperref[Splits]{\TYPE{Splits}}}
\newcommand{\SF}{\hyperref[SplittingField]{\TYPE{SplittingField}}}
\newcommand{\Al}{\hyperref[Algebraic]{\mathcal{A}}}
\newcommand{\Tr}{\hyperref[Transcendental]{\mathcal{T}}}
\newcommand{\Minimal}{\hyperref[Minimal]{\TYPE{Minimal}}}
\newcommand{\Conjugate}{\hyperref[Conjugate]{\TYPE{Conjugate}}}
\newcommand{\SR}{\hyperref[SimpleRoot]{\TYPE{SimpleRoot}}}
\newcommand{\MR}{\hyperref[MultipleRoot]{\TYPE{MultipleRoot}}}
\newcommand{\Sep}{\hyperref[Separable]{\TYPE{Separable}}}
\newcommand{\Tower}{\hyperref[Tower]{\TYPE{Tower}}}
\newcommand{\Algebraic}{\hyperref[AlgebraicExtension]{\TYPE{Algebraic}}}
\newcommand{\Trans}{\hyperref[TranscedentalExtension ]{\TYPE{Transcesental}}}
\newcommand{\SE}{\hyperref[SimpleExtension]{\TYPE{SimpleExtension}}}
\newcommand{\SepEl}{\hyperref[SeparableElement]{\TYPE{SeparableElement}}}
\newcommand{\SepEx}{\hyperref[SeparableExtension]{\TYPE{SeparableExtension}}}
\newcommand{\NE}{{\hyperref[NormalExtension]{\TYPE{NormalExtension}}}}
\newcommand{\FGE}{{\hyperref[FGE]{\TYPE{FinitelyGeneratedExtension}}}}
\newcommand{\Dist}{\hyperref[Dist]{\TYPE{Distinguished}}}
\newcommand{\FE}{\hyperref[FiniteExtension]{\TYPE{FiniteExtension}}}
\renewcommand{\AC}{\hyperref[AC]{\TYPE{AlgebraicalyClosed}}}
\newcommand{\EmEx}[3]{\hyperref[EmbedingExtension]{\mathrm{Homm}_{#1}\left (#2,#3 \right )}}
\newcommand{\Char}{\hyperref[Character]{\TYPE{Character}}}
%Funcs
\renewcommand{\gcd}{\hyperref[gcd]{\mathrm{gcd}}}
\newcommand{\minimal}{\hyperref[minimal]{\mathrm{minimal}}}
\newcommand{\mult}{\hyperref[multiplicity]{\mathrm{mult}}}
\newcommand{\EL}{\hyperref[ExtL]{\mathfrak{L}}}
%THM
\newcommand{\ZPTHM}{\hyperref[ZPTHM]{\THM{ZeroPolinomeTHM}}}
\newcommand{\GCDFI}{\hyperref[GCDFieldInvariant]{\THM{GCDFieldInvariant}}}
\newcommand{\RLFI}{\hyperref[RLFieldInvariant]{\THM{RLFieldInvariant}}}
\newcommand{\RE}{\hyperref[RootsExtension]{\THM{RootsExtension}}}
\begin{document}
\maketitle
\tableofcontents
\newpage
\section{Basic Definitions}
$\TYPE{Field} :: ?\sum k : \Set \. ( k \times k \to k) \times (k \times k \to k)$ \\
$(k,\cdot,+) : \Abel \iff (k,+),(k \setminus \{ 0_+ \} ,\cdot) : \Abel \wedge ( k,\cdot,+) : \CR  
\label{Field}$
\\ \\
$\FUNC{implicit} :: \Field \to \Set $\\
$\FUNC{implicit}(k,\cdot,+) \de k $ 
\\ \\
$\FUNC{division} :: \prod k : \Field \. k \to (k \setminus \{0\}) \to k$ \\
$\FUNC{division}(0,a)  =0/a \de 0$ \\
$\FUNC{division}(b,a)  = b/a \de ba^{-1}  $ \\ 
\\ \\
$\TYPE{Subfield} :: \prod K : \Field \. ?\Field $ \\
$ k : \TYPE{Subfield}  \iff \exists \TYPE{Mono}_{\RING}(k,K) 
\label{Subfield}
$
\\ \\
$\TYPE{Extension} :: \prod K : \Field \. ?\Field $ \\
$ k : \TYPE{Extension}  \iff \exists \TYPE{Mono}_{\RING}(K,k) 
\label{Superfield}
$

\newpage
\section{Polynomials over a Field}
\subsection{Polynomials as functions}
\begin{align*}
&\FUNC{implicit} :: \prod k :  \Field \. k[\NNInt] \to  k \to k\\
&\FUNC{implicit}(p)(x) \de \sum^{\deg p}_{i = 0} a_ix^i 
\\ \\
  &\FUNC{implicit} :: \prod k :  \Field \. \prod n \in \Nat \. k[\NNInt^n] \to  k^n \to k\\
&\FUNC{implicit}(p)(x) \de \sum_{\alpha \in \mathrm{multideg} \,p} a_\alpha \prod^n_{i=1}x_i^{\alpha_i} 
\\ \\
\label{ZPTHM}
&\THM{ZeroPolinomial} :: \forall L : \Field \. \forall F : \Subfield(L) : \#F \ge \aleph_0 \. \forall p \in  L[\NNInt^n] \. \\
& \quad \. p_{|F} =_{F^n \to F} 0 \Rightarrow p =_{L[\NNInt^n]} 0 \\
& \Proof - \\
& \A x \in F, \\
& 0 = p(x) = \sum_{\alpha \in \mathrm{multideg} \,p} a_\alpha \prod^n_{i=1}x_i^{\alpha_i} =
 \sum_{\alpha \in \mathrm{multideg} \,p} \sum^d_{k=1} \l_{i,\alpha}e_i \prod^n_{i=1}x_i^{\alpha_i} = 
   \sum^d_{k=1} \left (   \sum_{\alpha \in \mathrm{multideg} \,p}        
      \l_{i,\alpha} \prod^n_{i=1}x_i^{\alpha_i}
   \right )  e_i        \\
   & e_i : \LInd \leadsto \forall i \in d \. 
   \sum_{\alpha \in \mathrm{multideg} \,p}        
      \l_{i,\alpha} \prod^n_{i=1}x_i^{\alpha_i} = 0  ; \\
    & \forall x \in F \. \forall i \in d \. 
   \sum_{\alpha \in \mathrm{multideg} \,p}        
      \l_{i,\alpha} \prod^n_{i=1}x_i^{\alpha_i} = 0   
      \leadsto l = 0
       \leadsto p = 0 \\
      & \QED 
\end{align*}
\newpage
\subsection{Divisibility}
$\TYPE{CommonDivisor} :: \prod k : \Field \. k[\NNInt]\times k[\NNInt] \to ?
\TYPE{Monic}(k)$
\\
$ p :   \TYPE{CommonDivisor}(a,b) \iff p | a \wedge p | b           $
\\ \\
$\TYPE{GCD} :: \prod k : \Field \. \prod a,b \in k[\NNInt] \. ?\TYPE{CommonDivisor}(a,b)$
\\
$ \TYPE{GCD} \de \argmax \deg p $
\\ \\
$\THM{UniqueGCD} :: \forall k : \Field \. \forall a,b \in k[\NNInt] \. \exists ! \TYPE{GCD}(a,b)  $ \\
Proof $\approx$ \\

As $k$ is field $k[x]$ is principle domain, so $(a,b) = ( p )$ for some $p$ which can be taken to be monic without loss of generality as $k$ is a field again. So $\TYPE{CommonDivisor}(a,b)$ exists. Assume we take any other common divisor $q$ but then, 
since $ p = xa + yb  $ and hence $ q | p$,  $q$  has lower degree than $p$. This proofs that $p : \TYPE{GCD}(a,b)$

Now assume $p,q : \TYPE{GCD}(a,b)$. We know  that $\deg p = \deg q$ and that $\mathrm{lc}(p) = \mathrm{lc}(q) = 1$ . If $p \neq q$ then their least common denominator will have higher degree and still be a common divisor, which leads  to contradiction with initial hypothesis. This proves $p = q$ and hence uniqueness. \\
$\QED$
\\ \\
$
\FUNC{gcd} :: \prod k :  \Field \. k[\NNInt] \times k[\NNInt] \to  k[\NNInt]$\\
$\FUNC{gcd}(a,b) \de \gcd(a,b) \de  \THM{UniqueGCD}(k)(a,b) \E
\label{gcd} $
\\ \\
$
\THM{GCDFieldInvariant} :: \forall k : \Field  \.
\forall K : \Subfield(k) \. \forall a,b \in K[\NNInt] \. \gcd_k(a,b) \in K[\NNInt]
\label{GCDFieldInvariant}
$
Proof $\approx$ \\
 $\gcd_K(a,b) \in K[\NNInt]$ exists and  is $\TYPE{GCD}(k)$, as $\gcd_K(a,b) \in (a,b)_k$ and hence divisible by $\gcd_k(a,b)$. Moreover,  $\TYPE{GCD}(k)$ is unique so it must be the case that $\gcd_K(a,b) = \gcd_k(a,b) \in K[\NNInt]$. \\
 $\QED$
\\ \\
$\TYPE{RelativelyPrime} :: \prod k : \Field \. ?(k[\NNInt] \times k[\NNInt]) $ \\
$ (a,b) : \TYPE{RelativelyPrime} \iff \gcd(a,b) = 0$
\label{RelativelyPrime}
\\ \\
$\THM{RLFieldInvariant} :: \forall k : \Field  \. \label{RLFieldInvariant}
\forall K : \Subfield(k) \. \forall (a,b) : \RP(K) \. (a,b) : \RP(k) \.$\\
Proof $\approx$ \\
Application of $\GCDFI$ \\
$\QED$
\newpage
\subsection{Roots}
$\TYPE{Root} :: \prod k : \Field \.  k[\NNInt] \to ?k$ \\
$ x : \TYPE{Root}(p) \iff p(x) = 0  $\label{Root}
\\ \\
$\THM{RootExtension} :: \forall k : \Field \. \forall p \in k[\NNInt]  : \deg p > 0 \. \exists K : \Superfield(k) : \exists \Root(K)(p)
$\label{RootsExtension}\\
Proof $\approx$ \\
$p$ has roots in $k$, otherwise it is irreducible in $k[\NNInt]$. Assume second alternative, as first is trivial. By Ring theory, as $p$ is irreducible, 
$$
 K \de \frac{k[\NNInt]}{(p)} : \Field
$$
with trivial monomorphism $f : k \to K$ such that $f : x \mapsto x \mod p$, hence a superfield of $k$. Now take $a = [0,1] \mod p \in K$, then $p(a) =  p \mod p = 0$ which means that $a$ is a root of $p$ in $K$. \\
$\QED$
\\ \\
$\TYPE{Splits} : \prod k : \Field \. ?k[\NNInt] $ \\
$p : \TYPE{Splits} \iff \exists n \in \Nat : \exists f : n \to k[\NNInt] : \forall i \in n \. \deg f \le 1 : p = \prod^n_{i=1} f_i $\label{Splits}
\\ \\
$\THM{SplittingExtension} :: \forall k : \Field \. \forall p \in k[\NNInt]   \. \exists K : \Superfield(k) : ( f : \Splits(K))$ \\
Proof $\approx$ \\
Corollary of $\RE$. \\
$\QED$
\\ \\
$\THM{RPInExtension} :: \forall k : \Field \. \forall p,q \in k[\NNInt]   \. 
 (p,q) : \RP(k) \iff \forall K : \Superfield(k) \. (p,q) : \RP(K) 
$ 
\\
Proof $\approx$ \\
Corollary of $\RE$. \\
$\QED$
\\ \\
$\THM{RPIrreducable} :: \forall k : \Field \. \forall (p,q) :  \TYPE{Irreducable}  (k[\NNInt]) : p \neq q   \.  \forall K : \Superfield(k) \. (p,q) : \RP (K)
$
\\
Proof $\approx$ \\
Corollary of $\RE$. \\
$\QED$
\\ \\
$
\TYPE{SplittingField} :  \prod  k : \Field \. \TYPE{Finite} (k[\NNInt]) \to ?\Superfield(k) 
$ \\
$
 \TYPE{SplittingField}(P) \de \min \{ K : \Superfield(k) : \forall p \in P \. p : \Splits(K)  \}  \label{SplittingField}
$
\\ \\
$\THM{SplittingFieldExists} :: \forall k : \Field \. \forall P : \TYPE{Finite} (k[\NNInt]) \. \exists  \SF(P) $ \\
Proof $\approx$ \\
Take splitting Field for $\prod P$.\\
$\QED$
\\ \\
$\TYPE{Algebraic} :: \prod k : \Field \. \prod K : \Superfield(k) \. ? K$ \\
$a : \TYPE{Algebraic} \iff a \in \Al(k,K) \iff \exists p \in k[\NNInt] : p(a) = 0  $
\label{Algebraic}
\newpage
$\TYPE{Transcedental} :: \prod k : \Field \. \prod K : \Superfield(k) \. ? K$ \\
$a : \TYPE{Transcedental} \iff a \in \Tr(k,K) \iff a \IsNot \Al(k,K) $
\label{Transcedental}
\\ \\
$\TYPE{Minimal} :: \prod k : \Field \. \prod K : \Superfield(k) \.  \Al(k,K) \to ?\TYPE{Monic}$ \\
$\TYPE{Minimal}(a) \de \argmin \{ \deg p | p \in k[\NNInt] : p(a) = 0 \} $
\label{Minimal}
\\ \\
$\THM{MinimalExists} :: \forall
 k : \Field \. \forall K : \Superfield(k) \.  \forall a \in \Al(k,K) 
 \. \exists ! \Minimal (a)  $\\
 Proof $\approx$ \\
 Set in definition of Minimal  polynomial is  will be an ideal of $k[\NNInt]$, so as
 $k[\NNInt]$ is a principle domain it will have a unique monic generator $p$. $p$ is minimal.\\
$\QED$
\\ \\
$\FUNC{minimal} :: \prod k : \Field \. \prod K : \Superfield(k) \. \prod a \in \Al(k,K) \. \Minimal(a) $ \\
$\FUNC{minimal} = \minimal (a) \de \THM{MinimalExists}(k)(K)(a) \E $
\label{minimal}
\\ \\
$\TYPE{Conjugate} :: \prod k : \Field \. \prod K : \Superfield(k) \. ?(\Al(k,K) \times \Al(k,K))$\\
$(a,b) : \TYPE{Conjugate} \iff \minimal(a) = \minimal(b)$ \label{Conjugate}
\\ \\
$\FUNC{multiplicity} :: \prod k : \Field \. \prod p \in k[\NNInt]\. \Root(p) \to \Nat $\\
$ \FUNC{multiplicity}(a) \de \mult (p, a) \de \max  \big\{ n \in \Nat :[-a,1]^n \; \big| \; p  \big\} \label{multiplicity} $ 
\\ \\
$\TYPE{SimpleRoot} ::  \prod k : \Field \. \prod p \in k[\NNInt]\. ?\Root(p)  $ \\
$a : \TYPE{SimpleRoot} \iff \mult(p,a) = 1  $\label{SimpleRoot}
\\ \\
$\TYPE{MultipleRoot} ::  \prod k : \Field \. \prod p \in k[\NNInt]\. ?\Root(p)  $ \\
$a : \TYPE{MultipleRoot} \iff \mult(p,a) > 1  $\label{MultipleRoot}
\\ \\
$\TYPE{Separable} ::  \prod k : \Field \.  ?\Irr (k[\NNInt]) $ \\
$p : \TYPE{Separable} \iff \forall K : \Superfield(k) \. \forall a : \Root(K)(p) \. a : \SR(K)(p)   $\label{Separable}
\\ \\
$ \THM{SimpleRootsCriterion} :: \forall k \in \Field \. \forall p \in k[\NNInt] \. 
  \big( \forall a : \Root(p) \. a : \SR \iff (p,p') : \RP \big)  $ \\
  Proof $\approx$ \\
  
  Assume all roots are simple. By $\RLFI$ we can work in splitting field of $f$.
  Then
  $$ p(x) = \prod^n_{i=1}(x-a_i) \quad p'(x) = \sum^n_{j=1} \frac{1}{x - a_j}\prod^n_{i=1}(x-a_i) $$
 and from the structure of $p'$ using the fact that $a_i \neq a_j$ for $i \neq j$ we see that it indeed coprime with $p$.
 
 Now assume that $p$ and $p'$ are coprime. Also assume that $\mult(a_1) > 1$, then
 $$
   p'(x) = (x - a_1)\sum^n_{j=1} \frac{1}{(x - a_j)(x -a_1)}\prod^n_{i=1}(x-a_i)
 $$  
 which is not coprime with $p$, hence a contradiction. \\
 $\QED$
 \newpage
 $
 \THM{SeparabilityCriterion} :: \forall k : \Field \. \forall p : \Irr( k[\NNInt] ) \.
  p : \Sep \iff p' \neq_{k \to k} 0   
 $\\
   Proof $\approx$ \\
   
   assume that $p$ is separable which implies that she has only simple roots in her splitting field . As $p$ is irreducable it is not constant, so $p' \neq 0$ . By previous theorem we can see that $p'$ has no common roots with $p$  is not  zero as a function.
   
   Now consider that $p'$ is not a zero . If $p$ and $p'$ is not coprime then $p' \in k[\NNInt]$ and has all repeated roots with multiplicity reduced by one . So $g =\gcd(p,p')$ have $\deg g > 0$. And by $\GCDFI$ $g$ also belongs to $k[\Nat_+]$  which means that $p$is not irreducable, a contradiction.\\
   $\QED$
 \\ \\
  $
 \THM{IrreducableAreSeparableChar0} :: \forall k : \Field : \cha(k) = 0 \. \forall p : \Irr( k[\NNInt] ) \.
  p : \Sep 
 $\\
 Proof $\approx$ \\
 As $p$ is not a constant her derivative is not $0$ and hence is not a zero function. \\
 $\QED$
  \\ \\
 $
 \begin{array}{l}
  \THM{IrreducableInseparableCharN} :: \forall n \in \Nat \. \forall k : 
   \cha(k) =  n  \. \forall p : \Irr(k[\NNInt]) \. \\ \quad \. 
    p \IsNot \Sep \iff \exists f : \Sep \exists d \in \Nat  : p(x) = f\left (x^{p^d} \right )    
    \end{array}
 $  \\
Proof $\approx$  
  
  Assume that  $p$ is inseparable, then $p' = 0$ . This means that all coefficients of $f'$ are divisible by $n$. On the over hand we know that coefficients of $p$ cannot be divisible by $n$  which means that $p(x) =(x^{p^d})$ for some  $f$ and $d$. By taking maximal possible $d$ we must get separable as it will have at least one monomial with exponent coprime with $n$
  
  The inverse proof is obvious \\ 
  $\square$
  \\ \\
 $
  \THM{InseparableCharNMultiplicity} :: \forall n \in \Nat \. \forall k : 
   \cha(k) =  n  \. 
    p \IsNot \Sep  \. \exists d \in \Nat : \forall a : \Root(p) \. \mult(p) = n^d     
 $  \\
 Proof $\approx$  \\
 By previous theorem we know that over splitting field we will have:
 $$
   p(x) = f \left ( x^{p^d}  \right )  = \prod^N_{i =1}\left (x^{p^d} - a_i \right )
   =  \prod^N_{i =1}\left (x^{p^d} - b_i^{p^d} \right ) = 
   \prod^N_{i =1}\left (x - b_i \right )^{p^d}
 $$
 $
  \QED
 $
 \\ \\
$
 \FUNC{radicalExponent} ::  \prod n \in \Nat \. \prod k : \Field  : \cha k = p \. 
 !\Sep \to \Nat  
$\\
$
 \FUNC{radicalExponent}(p)  = d(p) \de    \THM{IrreducableInseparableCharN}(n)(k)_2 \E 
$
\\ \\
 $
  \THM{FFIsSeparable} :: \forall n \in \Nat \. \forall k : 
   \cha(k) =  n : \# k < \aleph_0 \.       
   \forall p : \Irr (k) \.
   p : \Sep  
 $ 
 Assume $p$ is inseparable . Note that $\# k = n^m  $ for some $m$.  So multiplicative group of $k$ has order $n^m - 1$  so for every $a \in k$ $a^{n^m} = a$. So for every $a$,  $a = b^n$ for some $b = a^{n^{m - 1} }$.  so 
 $$p(x) = \sum_{i = 0}^d a_ix^{ip} = \sum_{i = 0}^d b_i^{p}x^{ip} 
 =   \left (  \sum^d_{i = 0} b_i x^i  \right )^p
  $$ 
 which is reducible in $k$ which means contradiction. 
 \newpage
 \subsection{Tests of Irreducibility} 
  $\THM{LoalizationTHM} :: \forall R : \Ring \. \forall  k : \Field \.
  \forall \sigma : \mathcal{M}_{\RING}{(R,F)} \. \forall p \in R[\NNInt] \. 
   \deg(p^\sigma) = \deg(p) \wedge 
    p^\sigma : \Irr(k) \Rightarrow
    p : \DWIrr(R)    
  $  \\
  Proof $\approx$\\
  
  Assume that $p$ is reducible over $R$. Then write $p = fg$. But with this factorization      $ p^\sigma = f^\sigma g^\sigma $ which means that means that $p^\sigma$ is also reducible as $\deg f^\sigma, \deg g^\sigma > 0$, a contradiction. $\QED$
  \\ \\
    $\THM{LoalizationRule} :: \forall R : PID \. 
  \forall a : \Prime(R) \. \forall p \in R[\NNInt] :  a \not | \lc(p) \.
    \pi_a p : \Irr(k) \Rightarrow
    p : \DWIrr(R)    
  $ \\ 
  Proof $\approx$ \\
  
  $\frac{R}{(a)}$ is a field and by hypothesis degree was not reduced.  $\QED$
  \\ \\
  $
  \begin{array}{l}  
  \THM{EisenstinCriterion} :: \forall R : \IntD \. 
   \forall p \in R[\NNInt] \. \\
  \.\forall a : \Prime(R)  : a \not | \lc(p) : a^2 \not| p_0 : 
   \forall i \in \deg p - 1 \. a | p \. p : \DWIrr
  \end{array}
  $
  \\ 
  Proof $\approx$ \\
  Assume that $p$ is degree-wise reducible. Write $p = fg$ and apply projection
  $$
       pi_a(\lc(p))x^{\deg p}    = \pi_a p = (\pi_a f)(\pi_a g )
  $$
  This means that all coefficients of  $f$,$g$ are $0$ expect the leading ones but at least one of them must have non-zero constant part which gives us a contradiction. \\
 $ \QED$
   \newpage
 \subsection{Reciprocal Polynomials}
 \newpage
 \section{Field Extension}
 \subsection{Lattice of Extensions}
  $\FUNC{degree} :: \prod k : \Field \. \Superfield(k) \to \Nat_\infty $ \\
  $ \FUNC{degree}(k) = [K : k] \de \dim_k K$
  \\ \\
  $\TYPE{Tower} :: \,  ? [\:] | \Field \times \TYPE{Tower} $ \\
  $ (A,(B,T)) : \TYPE{Tower} \iff A : \Subfield(B) $\label{Tower} \\
  \\
  $\THM{TowerDegree} :: \forall [A,B,C] : \Tower \. [C : A] = [C : B][B : A]$
  \\
  Proof $\approx$ \\
  Use basis decomposition $\QED$ 
  \\ \\
   $\FUNC{composite} : \prod K : \Field \. \Subfield(K) \times \Subfield(K) \to \Subfield(K) $\\
   $
    \FUNC{composite}(A,B) \de \min \{ k : \Subfield(K)  : A,B \subset k  \}
   $
  \\ \\
   $\FUNC{composite} : \prod K : \Field \. \Set \, \Subfield(K) \to \Subfield(K) $\\
   $
    \FUNC{composite}(\mathcal{E}) = \bigvee \mathcal{E}\de \min \{ k : \Subfield(K) : \forall E \in \mathcal{E} \. E \subset k  \}
   $
   \\ \\
   $
    \FUNC{extensionLattice} :: \Field \to \TYPE{Lattice} \\
    \FUNC{extensionLattice}(K) = \EL(K)\de \big(\Subfield(K),\subset,\cap,\FUNC{composite} \big)   
   $ \label{ExtL}
   \newpage
   \subsection{Types of Extensions}
   $\TYPE{Algebraic} :: \prod k : \Field \. ? \Superfield(k) $ \\
   $ K : \TYPE{Algebraic} \iff \forall a \in K \. a \in \Al(k) $
   \label{AlgebraicExtension} 
   \\ \\
    $\TYPE{NormalExtension} :: \prod k : \Field \. ? \Superfield(k) $ \\
   $ K : \TYPE{NormalExtension} \iff \exists P : \Set \, k[\NNInt] : K = 
   \min\{F : \Superfield(k) : \forall p \in P \. p : \Splits(F)\}$
   \label{NormalExtension} 
   \\ \\
   $\TYPE{Transcendental} \de \IsNot \Algebraic$
   \label{TranscedentalExtension}    
   \\ \\
   $\FUNC{extend} :: \prod K : \Field \. \Subfield(K) \to K \to \Subfield(K)$\\
   $ \FUNC{extend}(k)(a) = k(a) \de \min\{ F : \Superfield(k) : a \in F \}$
   \\ \\
   $\FUNC{extendWithSet} :: \prod K : \Field \. \Subfield(K) \to \Set(K) \to \Subfield(K)$       \\
   $ \FUNC{extend}(k)(A) = k(A) \de \min\{ F : \Superfield(k) : A \subset F \}$
   \\ \\
   $\TYPE{FinitelyGeneratedExtension} :: \prod k : \Field \. ? \Superfield(k) $ \\
   $ K : \TYPE{FinitelyGeneratedExtension} \iff \exists  A : \TYPE{Finite}(K) : K = k(A) $
   \label{FGE}
   \\ \\
   $\TYPE{SimpleExtension} :: \prod k : \Field \. ? \Superfield(k) $ \\
   $ K : \TYPE{SimpleExtension} \iff \exists  a \in K : K = k(a) $ 
   \label{SimpleExtension}
   \\ \\
   $ \TYPE{PrimitiveElement} :: \prod k : \Field \. \prod K  : \SE (k) \. ?K  $ \\
   $   a  :  \TYPE{PrimitiveElement}  \iff  K(a) $
   \\ \\
   $\TYPE{FiniteExtension} :: \prod k : \Field \. ? \Superfield(k) $ \\
   $ K : \TYPE{FiniteExtension} \iff [K : k] < \infty  $ 
   \label{FiniteExtension}
   \\ \\
   $\TYPE{SeparableElement} :: \prod k : \Field \. ?\Al(k) $ \\
   $a :  \TYPE{SeparableElement} \iff  \minimal(a) : \Sep(k)$
   \label{SeparableElement}
   \\ \\
   $ \TYPE{SeparableExtension} :: \prod k : \Field \. ? \Algebraic(k)                       $\\
$   K :   \TYPE{SeparableExtension} \iff \forall a \in K \. a : \SepEl(k) $
\label{SeparableExtension} 
\\ \\
$ \TYPE{GaloisExtension} \de  \SepEx \And \NE $      
\newpage
\subsection{Distinguished Extension} 
$
 \TYPE{2\hyph Tower} :: \, ?\Tower
$\\
$
 T : \TYPE{2\hyph Tower} \iff \mathrm{len}(T) = 3 
$                
\\ \\
$
 \FUNC{lowerStep} ::  \TYPE{2\hyph Tower} \to \Field
$\\
$
 \FUNC{lowerStep}[A,B,C] \de A
$
\\ \\
$
 \FUNC{upperStep} ::  \TYPE{2\hyph Tower} \to \Field
$\\
$
 \FUNC{upperStep}[A,B,C] \de C
$
\\ \\
$
 \FUNC{intermidiateField} ::  \TYPE{2\hyph Tower} \to \Field
 $\\ 
 $
 \FUNC{intermidiateField}[A,B,C] \de   B 
$\\
\\
$
\TYPE{ExtensionSystem} \de ? \, \sum k : \Field \. \Superfield(k)
$
\\ \\
$
 \TYPE{TowerProperty} :: \,  ?\TYPE{ExtensionSystem}  \\
$ 
$
  X : \TYPE{TowerProperty} \iff \forall [A,B,C] : \TYPE{2 \hyph Tower}  \. 
\big(
 (A,B),(B,C) \in X \iff (A,C) \in X \big)              
$
\\ \\
$
 \TYPE{LiftingProperty} :: \,  ?\TYPE{ExtensionSystem}  \\
$ 
$
  X : \TYPE{LiftingProperty} \iff  \forall (A,B) \. \forall K : \Superfield(A) \. (K, B \vee K) \in X              
$
\\ \\
$
 \TYPE{CompositionClosure} :: \,  ?\TYPE{ExtensionSystem}  $ \\
$
  X : \TYPE{CompositionClosure} \iff  \forall (A,B),(A,C)  \. (K, B \vee C) \in X              
$
\\ \\
$
 \TYPE{Distinguished} \de \TYPE{TowerProperty} \And \TYPE{LiftingProperty} \And  \TYPE{CompositionClosure} 
 \label{Dist}
$
\\ \\
$
  \FGE :\Dist
$\\
$
\Proof = 
$\\
$
 \A [A,B,C] : \TYPE{2-Tower},
$ \\
$
\A (A,B),(B,C) : \FGE,
$\\
$
 a \de \ByDef \FGE(A,B) : \TYPE{Finite}(B) : B = A(a),  
$\\
$
 b \de \ByDef \FGE(B,C) : \TYPE{Finite}(C) : C = B(b),  
$ \\
$
 c \de a \cup b : \TYPE{Finite}(C),
$ \\
$
 X \de A(c) : \FGE(A), 
$ \\
$
 (1) \de \ByDef \FGE(A)(X,B,C) : X = C,
$ \\
$
 =\!\!E(1,\ByDef(A,X)) : (  (A,C) : \FGE ) ;
$ \\
$
 \A (A,C) : \FGE,
$\\
$
  c \de  \ByDef \FGE(A,B) : \TYPE{Finite}(B) : C = A(c), 
$ \\
$
 (1) \de \ByDef \FGE(V)(A,B,C,c) B(c) = C, \\
$
\newpage
\subsection{Simple extensions[?]}
$
\THM{SimpleAlgebraicExtensionIso} :: 
\forall k : \Field  \. \forall a \in \Al(k) \.  \frac{k}{(\minimal(a))} \cong_{\RING} k(a)
$\\ 
Proof $\approx$ \\
  $$
    \frac{k}{(\minimal(a))} =_{\Set} \{ p \in k[\NNInt] :  \deg p <  \deg \minimal(a) \}
  $$
 As minimal polynomial $m = \minimal(a)$ is irreducible, and each $p$ in quotient has lower degree $p$ and $m$ will be coprime. So if $p \neq 0$ there are two polynomials $a$,$b$ such that
 $$
          ap  =   ap + bm \mod m = 1                                   
 $$
 So our quotient is a field with $a = p^{-1}$.  To see that our structures are indeed  isomorphic we construct a map $\nu : p(x) \mapsto p(a)$. By results stated above $\nu$
is indeed an isomorphism of rings. $\QED$
 \\ \\
$
\THM{DegreeOfSimpleExtension} :: \forall k : \Field \. \forall a \in \Al(k) \. [k : k(a)] = \deg \minimal( a )
$\\
Proof $\approx$ \\
write minimal polimomial as $x^d - p(x)$ with $d = \deg \minimal(a)$ so there is a relation $ a^d = p(a) $ on $k(a)$.
Which means that  there is a basis in $k(a)$: 

$$1,a, \ldots, a^{d - 1} $$ 
so   $\dim_k k(a) = d$.$\QED$
\\ \\
$
 \THM{ConjugatesAreIso} :: \forall k : \Field \. \forall a,b : \Conjugate(k) \. k(a) \cong_\RING k(b)
$ \\
Proof $\approx$ \\
From $\THM{DegreeOfSimpleExtension}$ it follows that $\dim k(a) = \dim k(b)$ so $ k(a) \cong_{\mathsf{VS}(k)} k(b)$ by properties of finite dimensional vector space. However linear isomorphism $\nu : p(a) \mapsto p(b)$ still will be an isomorphism of rings as it preserves multiplication with same substitution rule arising from minimal polynomial. So
$ k(a) \cong_{\mathsf{RING}} k(b) $ . $\QED$
\\ \\
$
 \THM{SimpleAlgIsAlg} :: \forall k : \Field \. \forall a : \Al(k) \. k(a) : \Algebraic
$ \\
Proof $\approx$ \\
 let $K$ be an algebraic closure of $k$. Then any polynomial $p(a)$ of $a \in K$ is also in $K$, Hence is algebraic. $\QED$.
 \\ \\
$
 \THM{SimpleAlgIsFinite} :: \forall k : \Field \. \forall a : \Al(k) \. k(a) : \FE
$ \\
Proof $\approx$ \\
 It is known that from $\THM{DegreeOfSimpleExtension}$  $[k : k(a)] = \deg \minimal(a) < \infty$. Result follows. $\QED$.
 \\ \\
$
 \THM{SimpleFiniteIsAlg} :: \forall k : \Field \. \forall K : \Superfield(k)  \.  \forall a \in K : (k(a) : \FE) \.  a \in \Al(k)
$ \\
Proof $\approx$ \\
 $k(a)$ will have a basis as a $\VS{k}$ of form $(a^k)_{k = 0}^d$. Finiteness of $a$ implies that for  some $n \in \Nat$ we have relation $a^n = p(a) $ for some $p \in k[\NNInt] : \deg p <  n $. But this means that $a$ is a root of $x^n - p(x)$, hence algebraic. $\QED$
\newpage
$\THM{SimpleAlgChracterization} :: \forall k : \Field \. \forall A : \TYPE{Finite} \, \Al (k) \. \big( \exists a \in \Al (k) : k(A) \cong_{\RING} k(a) \iff 
\#(\{ F : \Field : k <_{\RING}  F <_{\RING} k(A) \} / \cong_{\RING}) < \aleph_0 \big)  
$ \\
Proof $\approx$\\

Assume $k(A) \cong_{\RING} k(\alpha)$ for some $\alpha \in \Al(k)$. Then $ \infty > d =[k(a) : k ] = [k(A) : k]$. So There is a finite basis of $k(A)$ of form $E = (a^{m_i}_{n_i})^d_{i = 1}$ consisting of elements of $A$.  for any distinct $a_j$ we will have distinct extension of $k$, namely $k(a_j)$, we will have different extensions of $k$ such that $ k<_{\RING}k(a_j)<_{\RING} k(A)$ . And so on for all finite combinations $k(a_{i_1},a_{i_2},\ldots,a_{i_j})$. As $d$ is finite where can be only finite amount of differnt combinations up to isomorphism.

Now assume that number of intermediate fields is finite.  Take an arbitrary element $a \in A : a \not \in k$. We will apply finite induction. If $k(a) \cong k(A)$ then we are done. Now assume that we know that $k(\alpha) \cong k(a, \ldots, a_j)$ and $k(\alpha)(\beta) \cong k(A)$. We will need to show that there exists $\gamma$ such that $k(\gamma) = k(A)$. 
 If $k$ is finite. Then multiplicative group $k(A)$ still finite so it is cyclic implying there exists a generating element $\gamma$, hence $k(\gamma) = k(A)$ and we are done.  If $\#k \ge \aleph_0$, let $\gamma = \alpha + a\beta$ with $ 1 \neq a \neq 0 $. As there only finite amount of  fields in-between $k$ and $k(A)$ where mus exist $b \neq a$ such that 
  $k(\alpha + a\beta) \cong k(\alpha + b\beta)$. This implies that  $\alpha,\beta \in k(\gamma)$ so $k(\gamma) \cong k(A)$. Induction implies that there must some $\gamma$ such that $k(\gamma) = k(A)$. $\QED$  \\ \\
$
 \THM{InfSimpleAlgChracterization} :: \forall k : \Field : \# k \ge \aleph_0 \. \forall n \in \Nat \. \forall 
  a \in \Al^n(k) \. \exists v \in k^n : k(\{a \}) = k(\langle  a, v\rangle)    
$ \\
Proof $\approx$\\
Corollary for $\THM{SimpleFiniteIsAlg}$ in infinite case. $\QED$ \\ \\
$ 
\THM{SimpleTransExtension} :: \forall k : \Field \. \forall  r \in \Tr(k) \. k(r) \cong_{\RING} k(\NNInt)
$
\\ \\
$
 \THM{TransExtensionThm} :: \forall k : \Field \. \forall r \in \Tr(k) \. \forall f,g : \RP(k[\NNInt]) \. \\  \. f(r) /g(r) \in  \Tr(k) \wedge k(r) : \Algebraic \left  (
 k(f(t)/g(t)) 
 \right )   
$
\newpage

\subsection{Algebraic Extension and Closure[?]}
$
\THM{FiniteIsAlg} :: \forall k : \Field \. \forall K : \FE(k) \. K : \Algebraic(k)
$
\\  \\
$
\Algebraic : \Dist
$
\\ \\
$
\TYPE{AlgebraiclyClosed} :: \; ? \Field 
$\\
$
 k : \TYPE{AlgebraiclyClosed} \iff \forall  p \in k[\NNInt] \. p : \Splits(k)  
$
\label{AC}
\\ \\
$
\TYPE{AlgebraicClosure}(k) \de \AC \And \Algebraic(k) 
$
\\ \\
$ \THM{ClosureExists} :: \forall k : \Field \. \exists!  K : \Superfield \And \AC $
\\ \\
$
\FUNC{algebraicClosure} :: \prod k : \Field \.  \TYPE{AlgebraicClosure}(k) 
$
\\
$
 \FUNC{algebraicClosure} = \overline{k}  \de  \THM{ClosureExists}(k)
$
\\ \\
$
\THM{AlgebraicAreField} :: \forall k : \Field \. \Al(k) : \Field  
$
\newpage
\subsection{Extensions of Embeddings[?] }
$ 
 \TYPE{EmbedingExtension} :: \prod k,F : \Field  \. 
  \prod K : \Superfield(k) \. (k \ToInj F) \to ?(K \ToInj F) \\
 S :\TYPE{EmbedingExtension}  \iff S \in \mathrm{Homm}_\sigma(K,G) \iff S_{|k} = \sigma
$
 \label{EmbedingExtension}
\\ \\
$
\THM{AutTHM} :: \forall k : \Field \. \forall K : \Superfield(k) \. \EmEx{\mathrm{id_{k,K}}}{K}{K} = \mathrm{Aut}_{\mathsf{ALG}(k)}{K}
$
\\ \\
$ 
\FUNC{simpleExtension} :: \prod k,K : \Field  \. \prod a \in \Al(k)  \.\\ \. \Root(K,\minimal(a)) \to (k \ToInj K) \to k(a) \ToInj K \\
 \FUNC{simpleExtension}(b,\sigma)  =  \sigma_b \de \Lambda f(a) \in k \. f^\sigma(b)
$
\\\\
$
\THM{SimpleEmbeddingExtension} ::   \forall  k,K : \Field  \. \forall a \in \Al(k)
\. \forall s : k \ToInj K \. S \in \EmEx{\sigma}{k(a)}{K} \. \\ 
 \. \exists b \in \Root(K,\minimal(k)(a)) : S = s_b   
$
\\\\
$
\THM{SimpleEmbeddingExtensionSize} ::   \forall  k,K : \Field  \. \forall a \in \Al(k)
\. \forall s : k \ToInj K \.   \\ 
 \.    \#\EmEx{s}{k(a)}{K} = \#  \Root(K,\minimal(k)(a))
$
\\\\\
$
\THM{AlgebraicEmbeddingEx} :: \forall  k : \Field \. \forall K : \AC \. \forall L : \Algebraic(k) \. \\
 \. \forall s : k \ToInj K \. \exists  S \in \EmEx{s}{L}{K} 
$
\\\\
$
 \THM{AlgebraicEmbeddingExSpec} :: \forall  k : \Field \. \forall K : \AC \. \forall L : \Algebraic(k) \. \\
 \. \forall s : k \ToInj K \. \forall a \in L \. \forall b
 \in \Root(L,\minimal(k)(a)) \. 
 \exists  S \in \EmEx{s}{L}{K} : S(a) = b 
$
\\ \\
$
 \THM{AlgebraicClosuresAreIso} :: \forall k : \Field \. 
\\ 
 \forall A,B :\Algebraic(k) \And \AC \. A \cong_{\mathsf{ALG}(k)} B 
$
\\ \\
$
 \TYPE{Character} \de \prod M : \TYPE{Monoid} \. \prod k : \Field \. \Morph{\mathsf{MON}}{M,K_{*}}
$
\\ \\
$
\THM{CharacterIndependance} :: \forall T : \Set \, \Char(M,k) \. T : \LInd(k)
$
\label{Character}
\newpage
\subsection{Splitting Fields and Normal Extension[?]}
$
\THM{SplittinfFieldUnique} :: \forall k : \Field \. \forall P : \Set \, k[\NNInt] \. \exists ! \SF(k,P)
$
\\ \\
$
\THM{NormalExtensnsionProperty} :: \forall K : \NE \And \Algebraic(k) \. \\
\. (\forall s : K \ToInj \overline{k} \. K : \TYPE{Invariant}(s)) \And 
 (\forall p : \Irr(k) : \exists a \in \Root(K,p) \. p : \Splits(k))
$
\\ \\
$
\FUNC{NormalClosure} :: \prod k : \Field \  \Algebraic(k) \to \NE(k) \\
\FUNC{NormalClosure}(K) = \mathrm{nc}(K/k) \de \min \{ L :  \NE \And \Algebraic (k) : K \subset L           \} 
$
\newpage
\subsection{Constructable objects[!]}
\newpage
\section{Separability}
\end{document}
