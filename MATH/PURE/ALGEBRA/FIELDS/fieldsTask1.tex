\documentclass[12pt]{article}
\usepackage{mathtools}
\usepackage{amsmath}
\usepackage{amsfonts}
\usepackage{amssymb}
\usepackage{wasysym}
\usepackage{accents}
\usepackage[hidelinks]{hyperref}
\usepackage[dvipsnames]{xcolor}
\usepackage[top=20mm, bottom=20mm, left=30mm, right=10mm]{geometry}
\begin{document}
\section{Problem about Finite Fields}
\subsection*{Question a)}
 $P$ is irreducible over $\mathbb{F}_2$ 

It can be checked that $P$ has no roots in $\mathbb{F}_2$, so it has no factors of order 1.

The only possible irreducible factor of order 2 is $  X^2 + X + 1  $,  but

$$   (X^2 + X + 1)^4 = X^4 + X^2 + 1 \neq X^4 + X^3 + 1.$$

\subsection*{Degression}

The Polynomial $P$ has degree $4$, which means it has  $4$ 
roots in $\mathbb{F}_{16}$ 
which cicle under Frobenius isomorphism $f(x) = x^2$. Denote them by 

$$ A_1 \to A_2 \to A_3 \to A_4 \to A_1. $$

Moreover, we can express this roots:

Denote by $A_1 =a$ the first root and use it as ptimitive element of  $\mathbb{F}_{16} $ 
over $\mathbb{F}_2$, so

$$
  a^4 = a^3 + 1 .
$$

Then 
$$  A_2 = a^2   $$
$$  A_3 =  a^4 = a^3 + 1 $$
$$  A_4 = a^8 = (a^3 + 1)^2 = a^6 + 1  = a^3 + a^2 + a$$

\subsection*{Question b)}
$P$ has no roots in $\mathbb{F}_4$. 

If it was the case then there would be a root $A_i$ such that $f^2(A_i)=A_i^4 = A_i.$ 
But we can see that this is not true.

\subsection*{Question c)}
$P$ is not irreducible in $\mathbb{F}_{4}$.

Note that  $A_1 + A_3,A_2 + A_4,A_1A_3,A_2A_4 \in \mathbb{F}_4$ as

$$ (A_1 + A_3)^2 = A_1^2 + A_3^2 = A_2 + A_4 $$ 
$$ (A_2 + A_4)^2 = A_2^2 + A_4^2 = A_3 + A_1 $$
$$ (A_1A_2)^2 = A^2_1A^2_3 = A_2A_4$$
$$ (A_2A_4)^2 = A^2_2A^2_4 = A_3A_1 $$
Moreover,
$$
 A_1 + A_3 = a^3 + a + 1 =  a(a^3 + 1) = A_1A_3
$$
$$
 A_2  + A_4 = a^3 + a = a^2(a^3 + a^2 + 1) = A_2A_4.
$$
So we can factor $P$ as 
\begin{multline*}
P(X) = (X + A_1)(X + A_2)(X + A_3)(X + A_4) = (X^2 + (A_1 + A_3)X + A_1A_3)(X^2 + (A_2 + A_4)X + A_2A_4) =
\\ =
( X^2 + bX + b)(X^2 + (b+1)X + b + 1) 
\end{multline*}
where $b$ is a primitive element of $\mathbb{F}_4$ over $\mathbb{F}_2$.
\subsection*{question d)}
$P$ is Irreducible over $\mathbb{F}_8$.

$\mathbb{F}_8 = \mathbb{F}_{2^3}$ does not contain isomorphic copy of 
$\mathbb{F}_{4} = \mathbb{F}_{2^2}$ as $2$ and $3$ are coprime.
This means that it is impossible to properly embed $A_1A_3$ and $A_2A_4$ into $\mathbb{F}_8$ which implies irreducibility. 
\subsection*{question e)} 
As it was shown in  "Degression", $P$ has four roots in $\mathbb{F}_{16}$.
\subsection*{question f)}
 $P$ has no roots in $\mathbb{F}_{32} = \mathbb{F}_{2^5}$ as every root of $P$ will generate subfield isomorphic to $\mathbb{F}_{16} = \mathbb{F}_{2^4}$, however $4$ and $5$ are coprime, so $F_{32}$ has no such subfield.
\subsection*{question g)} 
 $P$ has no roots in $\mathbb{F}_{64} = \mathbb{F}_{2^6}$ as every root of 
$P$ will generate subfield isomorphic to 
$\mathbb{F}_{16} = \mathbb{F}_{2^4}$, however $4$ is not a 
divisor of  $5$, so $F_{64}$ has no such subfield.

\subsection*{question h)}
$P$ is not irreducible in $\mathbb{F}_{64}$ as it contains an isomorphic copy of $\mathbb{F}_{4}$ so factorization from question c) will work.
\end{document}
