\documentclass[12pt]{scrartcl}
\usepackage{mathtools}
\usepackage{amsmath}
\usepackage{amsfonts}
\usepackage{hyperref}
\usepackage{amssymb}
\usepackage{ wasysym }
\usepackage{accents}
\usepackage{graphicx}
\usepackage[dvipsnames]{xcolor}
\usepackage[a4paper,top=5mm, bottom=5mm, left=10mm, right=2mm]{geometry}
%Markup
\newcommand{\TYPE}[1]{\textcolor{NavyBlue}{\mathtt{#1}}}
\newcommand{\FUNC}[1]{\textcolor{Cerulean}{\mathtt{#1}}}
\newcommand{\LOGIC}[1]{\textcolor{Blue}{\mathtt{#1}}}
\newcommand{\THM}[1]{\textcolor{Maroon}{\mathtt{#1}}}
%META
\renewcommand{\.}{\; . \;}
\newcommand{\de}{: \kern 0.1pc =}
\newcommand{\extract}{\LOGIC{Extract}}
\newcommand{\where}{\LOGIC{where}}
\newcommand{\If}{\LOGIC{if} \;}
\newcommand{\Then}{ \; \LOGIC{then} \;}
\newcommand{\Else}{\; \LOGIC{else} \;}
\newcommand{\IsNot}{\; ! \;}
\newcommand{\Is}{ \; : \;}
\newcommand{\DefAs}{\; :: \;}
\newcommand{\Act}[1]{\left( #1 \right)}
\newcommand{\Example}{\LOGIC{Example} \; }
\newcommand{\Theorem}[2]{& \THM{#1} \, :: \, #2 \\ & \Proof = \\ } 
\newcommand{\DeclareType}[2]{& \TYPE{#1} \, :: \, #2 \\} 
\newcommand{\DefineType}[3]{& #1 : \TYPE{#2} \iff #3 \\} 
\newcommand{\DefineNamedType}[4]{& #1 : \TYPE{#2} \iff #3 \iff #4 \\} 
\newcommand{\DeclareFunc}[2]{& \FUNC{#1} \, :: \, #2 \\}  
\newcommand{\DefineFunc}[3]{&  \FUNC{#1}\Act{#2} \de #3 \\} 
\newcommand{\DefineNamedFunc}[4]{&  \FUNC{#1}\Act{#2} = #3 \de #4 \\} 
\newcommand{\NewLine}{\\ & \kern 1pc}
\newcommand{\Page}[1]{ \begin{align*} #1 \end{align*}   }
\newcommand{ \bd }{ \ByDef }
\newcommand{\NoProof}{ & \ldots \\ \EndProof}
%LOGIC
\renewcommand{\And}{\; \& \;}
\newcommand{\ForEach}[3]{\forall #1 : #2 \. #3 }
\newcommand{\Exist}[2]{\exists #1 : #2}
%TYPE THEORY
\newcommand{\DFunc}[3]{\prod #1 : #2 \. #3 }
\newcommand{\DPair}[3]{\sum #1 : #2 \. #3}
\newcommand{\Type}{\TYPE{Type}}
%%STD
\newcommand{\Int}{\mathbb{Z} }
\newcommand{\NNInt}{\mathbb{Z}_{+} }
\newcommand{\Reals}{\mathbb{R} }
\newcommand{\Complex}{\mathbb{C}}
\newcommand{\Rats}{\mathbb{Q} }
\newcommand{\Nat}{\mathbb{N} }
\newcommand{\EReals}{\stackrel{\mathclap{\infty}}{\mathbb{R}}}
\newcommand{\ERealsn}[1]{\stackrel{\mathclap{\infty}}{\mathbb{R}}^{#1}}
\DeclareMathOperator*{\centr}{center}
\DeclareMathOperator*{\argmin}{arg\,min}
\DeclareMathOperator*{\id}{id}
\DeclareMathOperator*{\im}{Im}
\newcommand{\EqClass}[1]{\TYPE{EqClass}\left( #1 \right)}
\newcommand{\Cat}{\TYPE{Category}}
\newcommand{\Mor}{\mathcal{M}}
\newcommand{\Obj}{\mathcal{O}}
\newcommand{\Func}[2]{\TYPE{Functor}\left( #1, #2 \right)}
\mathchardef\hyph="2D
\newcommand{\Surj}[2]{\TYPE{Surjective}\left( #1, #2 \right)}
\newcommand{\ToInj}{\hookrightarrow}
\newcommand{\ToSurj}{\twoheadrightarrow}
\newcommand{\ToBij}{\leftrightarrow}
\newcommand{\Set}{\TYPE{Set}}
\newcommand{\du}{\; \triangle \;}
\renewcommand{\c}{\complement}
%%ProofWritting
\newcommand{\Say}[3]{& #1 \de #2 : #3, \\}
\newcommand{\Conclude}[3]{& #1 \de #2 : #3; \\}
\newcommand{\Derive}[3]{& \leadsto #1 \de #2 : #3, \\}
\newcommand{\DeriveConclude}[3]{& \leadsto #1 \de #2 : #3 ; \\}
\newcommand{\A}{\LOGIC{Assume} \;} 
\newcommand{\Assume}[2]{& \A #1 : #2, \\}
\newcommand{\As}{\; \LOGIC{as } \;} 
\newcommand{\QED}{\; \square}
\newcommand{\EndProof}{& \QED \\}
\newcommand{\ByDef}{\eth} 
\newcommand{\ByConstr}{\jmath}  
\newcommand{\Alt}{\LOGIC{Alternative} \;}
\newcommand{\CL}{\LOGIC{Close} \;}
\newcommand{\More}{\LOGIC{Another} \;}
\newcommand{\Proof}{\LOGIC{Proof} \; }
%SetTheory
%Cats
\newcommand{\SET}{\mathsf{SET}}
%Ordered Fiels
\newcommand{\OF}{\TYPE{OrderedField}}
\author{Uncultured Tramp} 
\title{Ordered Fields}
\begin{document}
\maketitle
\newpage
\tableofcontents
\newpage
\section{Totally Ordered Fields}
\subsection{Basic Definitions and Inequelity Algebra}
\Page{
\DeclareType{OrderedField}{  \sum S : \Set \. (S \times S \to S)^2 \times \TYPE{TotalOrder}(S)  }
\DefineType{(k,+,\cdot,\ge)}{OrderedField}{ (k,+,\cdot) : \TYPE{Field}  \And \forall x,y,z \in k \.  \NewLine
 x \le y \Rightarrow z + x \ge z + y \NewLine
 x \ge 0 \And y \ge 0 \Rightarrow  xy \ge 0  
}
\\
\DeclareFunc{addToIneq}{\prod k : \OF \. \prod a,b,c \in k \. a \ge b \to a + c \ge b + c }
\DefineNamedFunc{addToIneq}{ x }{x + c}{\bd \OF(k)(a,b,c)}
\\
\DeclareFunc{implicit}{\OF \to \TYPE{Field}}
\DefineFunc{implicit}{S,+,\cdot,\ge}{(S,+,\cdot)}
\\
\DeclareFunc{implicit}{\OF \to \TYPE{Poset}}
\DefineFunc{implicit}{S,+,\cdot,\ge}{(S,\ge)}
\\
\Theorem{NegateIneq}{ \forall k : \OF \. \forall a \in k \. -a \ge 0 \ge a | a \ge 0 \ge -a  }
\Assume{(1)}{ a \ge 0}
\Assume{(2)}{-a \ge 0}
\Say{(3)}{\bd \OF(k)( -a, 0 ,a)}{  0 \ge a   }
\Say{(4)}{\bd  \TYPE{Order}(1)(3)}{ 0 = a  }
\Conclude{()}{\THM{ZeroInverse}(4)}{0 \ge - a}
\Derive{(2)}{I(\Rightarrow)}{-a \ge 0 \Rightarrow 0 \ge -a }
\Say{(3)}{ \THM{SelfImplication}( 0 \ge -a )}{ 0 \ge -a \Rightarrow 0 \ge -a}
\Say{(4)}{ \bd \TYPE{Total}(\ge)(0, -a)}{ -a \ge 0 | 0 \ge -a }
\Say{(5)}{  E(|)(2,3,4)  }{ 0 \ge - a}
\Conclude{()}{I(I)(5)( -a \ge 0 \ge  a )}{ -a \ge 0 \ge a | a \ge 0 \ge -a    }
\Derive{(1)}{  I(\Rightarrow)  }{ a \ge 0 \Rightarrow -a \ge 0 \ge a | a \ge 0 \ge -a}
\Assume{(2)}{ 0 \ge a}
\Assume{(3)}{ 0 \ge -a}
\Say{(4)}{\bd \OF(k)( 0, -a , a)}{  a \ge 0   }
\Say{(5)}{\bd  \TYPE{Order}(1)(3)}{ 0 = a  }
\Conclude{()}{\THM{ZeroInverse}(4)}{0 \ge - a}
\Derive{(3)}{I(\Rightarrow)}{ 0 \ge -a \Rightarrow  -a \ge 0 }
\Say{(4)}{ \THM{SelfImplication}( -a \ge 0 )}{ -a \ge 0 \Rightarrow -a \ge 0}
\Say{(5)}{ \bd \TYPE{Total}(\ge)(0, -a)}{ -a \ge 0 | 0 \ge -a }
\Say{(6)}{  E(|)(2,3,4)  }{ -a \ge - 0 }
\Conclude{()}{I(I)(5)( -a \ge 0 \ge  a )}{ -a \ge 0 \ge a | a \ge 0 \ge -a    }
\Derive{(2)}{  I(\Rightarrow)  }{ 0 \ge a \Rightarrow -a \ge 0 \ge a | a \ge 0 \ge -a}
\Say{(3)}{ \bd \TYPE{Total}(\ge)(2) }{ 0 \ge a | a \ge 0   }
\Say{(*)}{ E(|)(3,2,1) }{ - a \ge  0 \ge a | a \ge 0 \ge -a}
\EndProof
}
\Page{
\DeclareFunc{neqIneq}{ \prod k : \OF \. \prod a.b \in k \. a \le b \to -b \le -a  }
\DefineNamedFunc{ neqIneq  }{x}{ -x}{\THM{NegateIneq}(x)(b - a)}
\\
\Theorem{AddIneq}{\forall k : \OF \. \forall a,b,c, d \in k \. a \le b \And c \le d \Rightarrow a + c \le b + d }
\Say{(1) }{  \bd\OF(k)(a,b,c)  }{ a + c \le b + c }
\Say{(2)}{ \bd\OF(k)(c,d,b)}{b + c \le c + d}
\Conclude{(*)}{\bd \TYPE{Transitive}(k)(1)(2)}{ a + c \le b + d}
\EndProof
\\
\DeclareFunc{addIneq}{\prod k : \OF \. \prod a,b,c,d \in k \. a \ge b \And c \ge d \to a  + c \ge b + d}
\DefineNamedFunc{addIneq}{x,y}{x + y}{\THM{AddIneq}(b,a,d,c)(x,y)} 
\\
\Theorem{UnityIsGreaterThenZero}{\forall k : \OF \. 1_k > 0_k}
\Assume{ (1)}{ 0_k \ge 1_k }
\Say{(2)}{\THM{NegateIneq}(1)}{-1_k \ge 0_k}
\Say{(3)}{\bd \OF(k)(-1_k,-1_k)}{1_k \ge 0_K}
\Say{(4)}{\bd \TYPE{Antisymmetric}(\ge)(1,3)}{0_k = 1_k}
\Conclude{(5)}{\THM{FieldContradiction}(4)}{\bot}
\DeriveConclude{(*)}{E(\bot)}{1_k \ge 0_k}
\EndProof
\\
\DeclareFunc{positivePart}{ \prod k : \OF \. ?k }
\DefineNamedFunc{positivePart}{}{k_{++}}{\{x \in k : x > 0  \}}
\\
\DeclareFunc{nonNegativePart}{\prod k : \OF \. ?k}
\DefineNamedFunc{positivePart}{}{k_{+}}{\{x \in k : x \ge 0 \}}
\\
\Theorem{MultIneq}{\forall k : \OF \. \forall a,b,c,d \in k_+ \. a \ge b \And c \ge d \Rightarrow ac \ge bd  }
\Say{(1)}{\bd \OF(k)(c,d,-d)}{c - d \ge 0}
\Say{(2)}{\bd \OF(k)(a,b,-b)}{a - b \ge 0}
\Say{(3)}{\bd \TYPE{Distributive}(k,+,\cdot)(c,d,a)\bd \OF(k)(c - d, a )(1)  }{ a(c - d) = ac - ad \ge 0   }
\Say{(4)}{\bd \TYPE{Distributive}(k,+,\cdot)(a,b,d)\bd \OF(k)(a -b, d)(2) }{ (a - b)d = ad - bd \ge 0 }
\Say{(5)}{\bd \OF(k)(ac - ad, 0,ad)(3)}{ ac \ge ad  }
\Say{(6)}{\bd \OF(k)(ad - bd,0,bd)(4)}{ad \ge bd}
\Say{(*)}{\bd \TYPE{Transitive}(\ge)(5,6)}{ ac \ge bd}
\EndProof
\\
\DeclareFunc{multIneq}{\prod k :  \OF \. \forall a,b,c,d \in k_+ \. a \ge b \And c \ge d \to ac \ge bd}
\DefineNamedFunc{multIneq}{x,y}{x \cdot y}{  \THM{MulIneq}(k)(a,b,c,d)(x,y)    } 
}
\Page{
\Theorem{InverseOfPositiveIsPositive}{\forall k : \OF \. \forall a \in k_{++} \. a^{-1} \in k_{++}}
\Say{(1) }{ \bd\FUNC{inverse}(a)\THM{UnityIsGreaterThenZero}}{aa^{-1} = 1 > 0}
\Assume{(2)}{ 0 > a^{-1} }
\Say{(3)}{ -(2)}{-a^{-1} > 0}
\Say{(4))}{ (1)\cdot(3) }{ -a > 0  }
\Conclude{(5)}{ \THM{StrictOrderContradiction}(\bd k_{++}(\bd a))(4)}{ \bot }
\DeriveConclude{(*)}{E(\bot)}{ a^{-1} > 0}
\\
\Theorem{InverseIneq}{\forall k : \OF \. \forall a,b \in k_{++} \. \forall (0) :  a \ge b  \.  b^{-1} \ge a^{-1}}
\Say{(1)}{\bd \TYPE{Reflexive}(\ge)(b^{-1})}{b^{-1} \ge b^{-1}}
\Say{(2)}{\bd \TYPE{Reflexive}(\ge)(a^{-1})}{a^{-1} \ge a^{-1}}
\Say{(3)}{ \THM{InversOfPositiveIsPositive}( a )   }{a^{-1} > 0}
\Say{(4)}{ \THM{InverseOfPositiveIsPositive}(b)}{b^{-1} > 0}  
\Say{(*)}{ (0)\cdot(1)\cdot(2) }{ b^{-1} \ge a^{-1}}
\EndProof
\\
\DeclareFunc{InverseIneq}{\prod k : \OF \. \forall a,b \in k_{++} \. a \ge b \to b^{-1} \ge a^{-1}}
\DefineNamedFunc{InverseIneq}{x}{x^{-1}}{\THM{InverseIneq}(x)}
\\
\Theorem{SquareIsNonNeg}{\forall k : \OF \. \forall a \in k \. a^2 \in k_{+}}
\Say{ (1)  }{\bd \TYPE{Antisimmetric}(\ge)(a,0)}{ a \ge 0 | 0 \ge a}
\Assume{(2)}{ a \ge 0 }
\Conclude{()}{ (2)^2 }{a^2 \ge 0}
\Derive{(2)}{I(\Rightarrow)}{a \ge 0 \Rightarrow a^2 \ge 0}
\Assume{(3)}{ 0 \ge a}
\Say{(4)}{-(3)}{ -a \ge 0 }
\Conclude{(5)}{ (4)^2  }{ a^2 \ge 0    }
\Derive{(3)}{I(\Rightarrow)}{ 0 \ge a \Rightarrow a^2 \ge 0}
\Conclude{(*)}{E(|)(1,2,3)}{ a^2 \ge 0  }
\EndProof
}
\subsection{Roots of Inequelities}
\Page{
	\Theorem{RootEq}{\forall R : \OF \. \forall x,y \in R_+ \. x^2 = y^2 \iff x = y}
	\Assume{(1)}{x^2 = y^2}
	\Say{(2)}{ \bd x^2  }{\Big(x : \TYPE{Root}(\Lambda u \in R \. x^2 - u^2)\Big)}
	\Say{(3)}{\bd y^2(1)}{\Big(y : \TYPE{Root}(\Lambda u \in R \. x^2 - u^2)\Big)}
	\Say{(4)}{\THM{SimpleQuadaraticRoots}(x)}{ \TYPE{Root}(\Lambda u \in R \. x^2 - u^2) = \{ -x,x\} }
	\Say{(5)}{ (3)(4)   }{ y = x | y = -x}
	\Say{(6)}{\THM{NegValue}(\bd R_+)(\bd x )}{ -x \le 0 }
	\Conclude{()}{\THM{LEM}(6)(5)(\bd R_+)(\bd y) }{ x = y}
	\Derive{(1)}{I(\Rightarrow)}{ x^2 = y^2 \Rightarrow x = y }
	\Assume{(2)}{x = y}
	\Conclude{()}{E(\to,=)(\FUNC{square})(2)}{x^2 = y^2 }
	\Derive{()}{I(\Rightarrow)}{x = y \Rightarrow x^2 = y^2}
	\Conclude{(*)}{I(\iff)(1)(2)}{x^2 = y^2 \iff x = y}
	\EndProof
	\\
	\Theorem{RootIneq}{\forall R : \OF \. \forall x,y \in R_+ \. x^2 \ge y^2 \iff x \ge y}
	\Assume{1}{x^2 \ge y^2}
	\Assume{2}{x < y}
	\Say{(3)}{(2)^2}{x^2 < y^2}
	\Conclude{()}{\bd \TYPE{Antisymmetric}(\FUNC{order})(1,3)}{\bot }
	\DeriveConclude{(2)}{E(\bot)}{x \ge y}
	\Derive{(1)}{I(\Rightarrow)}{x^2 \ge  y^2 \Rightarrow x \ge y}
	\Assume{(2)}{x \ge y}
	\Conclude{()}{(2)^2}{x^2 \ge y^2}
	\Derive{(3)}{I(\Rightarrow)}{x \ge y \Rightarrow x^2 \ge y^2}
	\Conclude{(*)}{I(\iff)}{x^2 \ge y^2 \iff x \ge y}
	\EndProof
	\\
	\DeclareFunc{rootEq}{\prod R : \OF \. \prod a,b \in R_+ \. a^2 = b^2 \to a = b}
	\DefineNamedFunc{RootEq}{x}{\sqrt{x}}{\THM{RootEq}(x) }
	\\
	\DeclareFunc{rootIneq}{\prod R : \OF \. \prod a,b \in R_+ \. a^2 \ge b^2 \to a \ge b}
	\DefineNamedFunc{rootIneq}{x}{\sqrt{x}}{\THM{RootIneq}(x)}
}
\subsection{Archimedean Property}
\Page{
\Theorem{OrderedFieldHasCharZero}{\forall k : \OF \. \mathrm{char} \, k = 0}
\Say{(0)}{\THM{UnityIsGreaterThenZero}(k)}{ 1_k > 0_k }
\Assume{n}{\Nat}
\Assume{(1)}{\mathrm{char} \, k = n}
\Say{(2)}{\bd \mathrm{char} (2) }{  n_k = 0   }
\Assume{m}{\Nat}
\Assume{(3)}{  m_k > 0_k}
\Conclude{(4)}{ ((3) + 1_k)(0) }{ m_k + 1_k \ge 1_k > 0_k }
\Derive{(2)}{ I(\forall)I(\Rightarrow)}{ \forall m \in \Nat \.  m_k  > 0_k \Rightarrow m_k +1 > 0_k }
\Say{(3)}{ E(\Nat)(0)(2)}{ \forall m \in \Nat \. m_k > 0}
\Say{(4)}{(3)(n)}{n_k > 0}
\Conclude{()}{\THM{StrictOrderContradiction}(1)(4)}{\bot}
\DeriveConclude{(*)}{\bd^1 \mathrm{char} I(\forall) E(\bot) }{ \mathrm{char} \, k = 0}
\EndProof
\\
\DeclareType{Archimedean}{ ? \OF}
\DefineType{k}{Archimedean}{ \forall a  \in k \. \exists n \in \Nat \. n_k \ge a}
\\
\Theorem{InverseArchimedean}{ \forall k : \TYPE{Archimedean} \. \forall x \in k^+ \. \exists n \in \Nat \.  \frac{1}{n} \le  x  }
\Say{(1)}{\THM{InverseOfPosistiveIsPositive}(k)(x)}{x^{-1} \in k^++}
\Say{(n,2)}{\bd \TYPE{Archimedean}(x^{-1})}{ \sum n \in \Nat \. n > x^{-1}  }
\Conclude{(*)}{ (2)^{-1}}{ n^{-1} < x }
}
\newpage
\subsection{Rational Numbers as Example}
\Page{
	\Theorem{RationalNumbersAreOrderedField}{\Rats : \OF}
	\Assume{\frac{a}{b},\frac{c}{d},\frac{x}{y}}{\Rats}
	\Assume{(1)}{\frac{a}{b} \ge \frac{c}{d}}
	\Say{(2)}{\bd (\ge_{\Rats})(2)}{ ad \ge  cb   }
	\Say{(3)}{  y \cdot_{\Int} (2) +_{\Int} bdx   }{ ady + bdx \ge bcy + bdx   }
	\Say{(4)}{ \left( \frac{1}{bdy} \right) \cdot_{\Rats} (3)   }{ \frac{ady + bdx}{bdy} \ge \frac{bcy + bdx}{bdy}   }
	\Conclude{()}{\bd^{-1} +_\Rats (4)}{ \frac{a}{b} + \frac{x}{y} \le \frac{c}{d} + \frac{x}{y}   }
	\Derive{(1)}{ I(\forall)I(\Rightarrow)}{ \forall a,b,c \in \Rats \. a \ge b \Rightarrow a + c \ge b + c}
	\Assume{ \frac{a}{b},\frac{c}{d}}{\Rats}
	\Assume{(2)}{ \frac{a}{b} \ge 0 \And \frac{c}{d} \ge 0}
	\Say{(3)}{\bd (\ge_\Rats) (2)}{ a \ge 0 \And c \ge 0}
	\Say{(4)}{(3)_1 \cdot_{\Int} (3)_2}{ac \ge 0}
	\Conclude{(5)}{ (4) \cdot_{\Rats}\frac{1}{bf}}{\frac{ac}{bd} \ge 0 }
	\Derive{(2)}{ I(\forall)I(\Rightarrow)}{ \forall x,y \in \Rats \.  x > 0 \And y > 0 \Rightarrow xy > 0    }
	\Conclude{(*)}{\bd^{-1} \TYPE{\OF}(\Rats)(1)(2)}{\Big(\Rats : \OF \Big)}
	\EndProof
	\\
	\Theorem{RationalNumbersAreArchimedean}{\Rats : \TYPE{Archimedean}}
	\Assume{\frac{a}{b}}{\Rats}
	\Say{ (1)  }{ \THM{IntegerIneq}(a) }{ a \le |a| + 1 }
	\Say{(2)}{ \THM{NaturalMultIneq}(2,b)}{ a \le b|a| + b  }
	\Conclude{()}{\bd(\le_{\Rats}) }{ \frac{a}{b} \le b|a| + b  }
	\Derive{(1)}{ I(\forall)I(\exists)(b|a| + b) }{ \forall q \in \Rats \. \exists n \in \Nat \. q \le n }
	\Conclude{(2)}{\bd^{-1} \TYPE{Archimedean}(\Rats)(1) }{ \Big( \Rats : \TYPE{Archimedean} \Big)   }
	\EndProof
}
\newpage
\section{Value  Fields}
\subsection{Absolute Values}
\Page{
	\DeclareType{AbsoluteValue}{ \prod K : \TYPE{Field} \. \prod R : \OF \. ?(K \to R_+) }
	\DefineType{a}{AbsoluteValue}{ \forall x \in K \.  a(x) = 0 \iff x = 0
		\And  \NewLine \And \forall x,y \in K \. a(xy) = a(x)a(y)
		\And \forall x,y \in K \. a(x + y) \le a(x) + a(y)
	}
\\
& \TYPE{AbsoluteValueField} = \prod R : \OF \. \sum K : \TYPE{Field} \. \TYPE{AbsoluteValueField}(K,R) \\
\\
\DeclareFunc{synecdoche}{ \TYPE{AbsoluteValueField} \to \TYPE{Field} }
\DefineFunc{synecdoche}{K,a}{K}
\\
\DeclareFunc{absoluteValue}{ \prod (K,a) : \TYPE{AbsoluteValueField}(R) \. \TYPE{AbsoluteValue}(K,R)}
\DefineNamedFunc{absoluteVlaue}{ }{|\cdot|_{(K,a)}}{a}
\\
\Theorem{IdValue}{\forall K : \TYPE{AbsoluteValueField}(R) \. |1_K| = 1}
\Say{(1)}{\bd \TYPE{Field}(K)}{1 \neq 0}
\Say{(2)}{\bd_2 \TYPE{AbsoluteValue}(|\cdot|_K)(1,1)}{|1|^2 = |1|}
\Say{(3)}{ |1|^{-1}((2) - |1|)  }{|1|(|1| - 1) = 0}
\Say{(4)}{\bd_1 \TYPE{AbsoluteValue}(1)}{ |1| \neq 0 }
\Conclude{(5)}{\THM{PolynomialRoots}(1)(4)}{|1| = 1}
\EndProof
\\
\Theorem{NegIdValue}{\forall K : \TYPE{AbsoluteValueField}(R) \. |-1_K| = 1}
\Say{(1)}{ \THM{IdValue}\THM{NegSquere}\bd_2 \TYPE{AbsoluteValue}(|\cdot|_K)(-1,-1)  }
{ 1 = | 1 | = \big| (-1)^2 \big| =  \big| -1 \big|^2   }
\Say{(2)}{\bd \TYPE{AbsoluteValue}(-1)}{ |-1| > 0 }
\Conclude{(*)}{\THM{PolinomialRoots}(1)(2)}{|-1| = 1}
\EndProof
\\
\Theorem{NegValue}{\forall K : \TYPE{AbsoluteValueField}(R) \. \forall x \in K \. |-x| = |x|}
\Conclude{(*)}{ \bd \FUNC{negate}(-x,|-x|) \bd_2 \TYPE{AbsoluteValue}(-1,x)  \THM{NegIdValue}}{ |-x| = |-1\cdot x| = |-1||x| = |x| }
\EndProof
\\
\DeclareFunc{positiveVersion}{ \prod R : \OF \. R \to R_+ }
\DefineNamedFunc{positiveVersion}{x}{|x|}{ \If x \ge 0 \Then x \Else -x  }
}
\Page{
\Theorem{PositiveIsNotLess}{\forall R : \OF \. \forall x \in R \. x \le |x|}
\Assume{(1)}{x \ge 0}
\Say{(2)}{\bd \FUNC{positiveVersion}}{|x| = x}
\Conclude{()}{\bd \TYPE{Reflexive}\Big( \FUNC{order}(R) \Big)(2)}{ x \le |x|  }
\Derive{(1)}{I(\Rightarrow)}{ x \ge 0 \Rightarrow x \le |x|  }
\Assume{(2)}{x \le 0}
\Say{(3)}{\bd \FUNC{positiveVersion}(x) \bd(R_+)}{|x| \ge 0}
\Conclude{()}{(2)(3)}{ x \le |x| }
\Derive{(2)}{I(\Rightarrow)}{x < 0 \Rightarrow x \le |x|}
\Say{(3)}{\bd \TYPE{Total}\Big(\FUNC{order}(R)\Big)(x,0)}{ x \ge 0 | x \le 0}
\Conclude{(*)}{E(|)(1,2,3)}{ x \le |x| }
\EndProof
\\
\Theorem{PositiveSquering}{\forall R : \OF \. \forall x \in R \.  x^2  = |x|^2}
\Assume{(1)}{x \ge 0}
\Say{(2)}{\bd \FUNC{positiveVersion}}{x = |x|}
\Conclude{(4)}{\bd x^2 (2)}{   x^2  = x\cdot x = |x||x| = |x|^2  }
\Derive{(1)}{I(\Rightarrow)}{ x \ge 0 \Rightarrow x^2 = |x|^2  }
\Assume{(2)}{x \le 0 }
\Say{(3)}{\bd \FUNC{positiveVersion}}{|x| = -x}
\Conclude{()}{\bd x^2(3)}{ x^2 = (-x) \cdot (-x) = |x||x| = |x|^2  }
\Derive{(2)}{I(\Rightarrow)}{x \le 0 \Rightarrow x^2 = |x|^2}
\Say{(3)}{\bd \TYPE{Total}\Big(\FUNC{order}(R)\Big)(x,0)}{ x \ge 0 | x \le 0}
\Conclude{(*)}{E(|)(1,2,3)}{ x = |x|^2 }
\EndProof
}
\Page{
\Theorem{PositiveVersionIsAbsoluteValue}{ \forall R : \OF \.  \FUNC{positiveVersion}(R) : \TYPE{AbsoluteValue}  }
\Assume{x}{\TYPE{In}(R)}
\Assume{(1)}{x \neq 0}
\Say{(2)}{\bd \FUNC{positiveVersion}}{ |x| = x | |x| = -x}
\Assume{(3)}{|x| =  x}
\Conclude{()}{ (2)(3) }{ |x| \neq 0  }
\Derive{(3)}{I(\Rightarrow) }{ |x| = x \Rightarrow |x| \neq 0 }
\Assume{(4)}{|x| = -x}
\Conclude{()}{ (1)\bd\FUNC{neg}(4) }{|x| \neq 0}
\Derive{(4)}{I(\Rightarrow)}{ |x|  = -x \Rightarrow |x| \neq 0  }
\Conclude{(5)}{E(|)(2,3,4)}{ |x| \neq 0  }
\Derive{(1)}{I(\Rightarrow) }{ x \neq 0 \Rightarrow |x| \neq 0 }
\Assume{(2)}{|x| \neq  0}
\Say{(3)}{\bd \FUNC{positiveVersion}(0)}{ |0| = 0 }
\Conclude{()}{ (2)(3) }{ x \neq 0 }
\Derive{(1)}{I(\iff)(1)I(\Rightarrow) }{ x \neq  0 \iff |x| \neq 0}
\Assume{x,y}{\TYPE{In}(R)}
\Say{(2)}{\THM{PositiveSquere}(R)(xy)}{ (xy)^2 \ge 0 }
\Say{(3)}{\THM{PositiveSquering}(R)(xy)}
{  |xy|^2 =   \Big|(xy )^2\Big| =  (xy)^2  = x^2y^2 = \Big| x^2 \Big| \Big|  y^2\Big| = |x|^2|y|^2   }
\Conclude{()}{ \sqrt{(3)} }{|xy| = |x||y|}
\Derive{(2)}{I(\forall)}{ x,y \in R \. |xy| = |x||y|}
\Assume{x,y}{\TYPE{In}(R)}
\Say{(3)}{ \THM{PositiveIsNotLess}(2xy)(2)(2,|x|,|y|)  }{ 2xy \le |2xy| = 2|x||y|  }
\Say{(4)}{ \THM{PositiveIsSquare}\bd \FUNC{positiveVersion}\Big(\THM{PositiveSquering}(x + y) \Big)   
\THM{Binom}(2)\, (2) \THM{Binom}^{-1}(2) 
}{ 
\NewLine :
| x + y |^2  = |(x + y)^2| =  (x + y)^2 \le x^2 + 2xy + y^2 \le |x|^2 + 2|x||y| + |y|^2 \le \Big(|x| + |y|\Big)^2  }
\Conclude{()}{\sqrt{(4)} }{ |x + y| \le |x| + |y|  }
\Derive{(3)}{I(\forall)}{\forall x,y \in R \. |x + y| \le |x| + |y|}
\Conclude{(*)}{\bd^{-1}\TYPE{AbsoluteValue}(1,2,3)}{|x + y| \le |x| + |y|}
\EndProof
\\
\Theorem{InverseAbsValue}{\forall K : \TYPE{AbsoluteValueField}(R) \. \forall x \in K \. \forall (0) : x \neq 0 \. 
\big| x^{-1} \big| = | x |^{-1}}
\Say{(1)}{\THM{IdValue}(K)\bd^{-1} \FUNC{inverse}(x) \bd_2 \FUNC{absoluteValue}(K)(x,x^{-1})   }
{1_R = |1_K| = |xx^{-1}| = |x|\big|x^{-1}\big|}
\Conclude{(*)}{|x|^{-1} \cdot (1)}{  |x|^{-1} = \big| x^{-1} \big|  }
\EndProof
}\Page{
\Theorem{InverseTriangleIneq}{\forall K : \TYPE{AbsoluteValueField}(R) \. \forall x,y \in K \. 
\big||x| - |y|\big| \le |x -  y|  }
\Assume{ (1) }{ |x| - |y| \ge 0 }
\Say{(2)}{\THM{AddSubstract}(x,y)(|x|)\bd_3\TYPE{AbsoluteValue}(R)}{ |x| = | x - y + y| \le |x - y| + |y| }
\Conclude{()}{( \bd \FUNC{positiveVersion}(1) \Big((2) - |y|\Big)}{  \big||x| - |y|\big| = |x| - |y| \le  |x - y| }
\Derive{(1)}{I(\Rightarrow)}{ |x| - |y| \ge 0 \Rightarrow | |x| - |y| | \le |x - y|  }
\Assume{ (2) }{ |x| - |y| \le 0 }
\Say{(2)}{\THM{AddSubstract}(x,y)(|x|)\bd_3\TYPE{AbsoluteValue}(R)}{ |y| = | x - x + y| \le |x - y| + |x| }
\Conclude{()}{( \THM{NegValue}\bd \FUNC{positiveVersion}(1) \Big((2) - |y|\Big)}{  \big||x| - |y|\big| = |y| - |x| \le  |x - y| }
\Derive{(2)}{I(\Rightarrow)}{ |x| - |y| \le 0 \Rightarrow | |x| - |y| | \le |x - y|  }
\Say{(3)}{\bd \TYPE{Total}\Big(\FUNC{order}(R)\Big)(|x| - |y|,0)}{ |x| - |y| \ge 0 \; \Big| \; |x| - |y| \le 0}
\Conclude{(*)}{E(|)(1,2,3)}{ \Big| |x| - |y| \Big| \le |x - y| }
\EndProof
\\
\Theorem{IteratedTriangleIneq}{\forall K : \TYPE{AbsoluteValueField}(R) \. \forall n \in \Nat \. \forall x : n \to K
 \. \left| \sum^n_{i = 1} x_i  \right| \le \sum^n_{i=1} |x_i|
}
\Assume{(1)}{n = 1}
\Conclude{()}{\bd\TYPE{Reflexive}(\FUNC{order}(R))(|x_1)}{ |x_1| \le |x_1|}
\Derive{(1)}{\bd^{-1}\THM{IteratedTriangleIneq}}{ \THM{TriangleTriangleIneq}(1) }
\Assume{n-1}{\Nat}
\Assume{(2)}{\THM{IteratedTriangleIneq}(n-1)}
\Conclude{()}{ \bd_3 \TYPE{AbsoluteValue}(\FUNC{absoluteValue}(K)) (2)}
{\left|  \sum^n_{i=1} x_i \right| \le |x_n | + \left| \sum^{n-1}_{i=1} x_i \right| 
 \le \sum^n_{i=1} |x_i|
}
\Derive{ (2)  }{I(\forall)I(\Rightarrow)\bd^{-1} \THM{IteratedTriangleIneq}(n)}
{
\NewLine
\forall n - 1 \in \Nat \. \THM{IteratedTriangleIneq}(n - 1) \Rightarrow \THM{IteratedTrangleIneq}(n)}
\Conclude{(*)}{E(\Nat)(1)(2)}{ \THM{IteratedTriangleIneq} }
\EndProof
\\
\Theorem{IteratedAbsHomogen}
{ \forall K : \TYPE{AbsoluteValueField}(R) \. \forall n \in \Nat \. \forall x : n \to K \. 
	\left| \prod^n_{i=1} x_i \right| = \prod^n_{i=1} \left| x_i  \right|   }
\Assume{(1)}{n = 1}
\Conclude{()}{I(=)\big(|x_1|\big)}{|x_1| = |x_1|}
\Derive{(1)}{ \bd^{-1} \THM{IteratedAbsHomogen} }{ \THM{IteratedAbsHomogen}(1) }
\Assume{(2)}{n-1 \in \Nat}
\Assume{(3)}{ \THM{IteratedAbsHomogen}(n - 1)  }
\Conclude{(4)}{ \bd_2 \TYPE{AbsoluteValue}\big(\FUNC{absoluteValue}(K)\big)\left(x_n, \prod^{n-1}_{i=1} x_i \right)   
 (2)(x_{|{n-1}})   }
{ \left| \prod^n_{i=1} x_i \right| = |x_n|\left| \prod^{n-1}_{i=1} x_i \right| \le
\prod^n_{i=1} |x_i|}
\Derive{ (2)  }{I(\forall)I(\Rightarrow)\bd^{-1} \THM{IteratedAbsHomogen}(n)}
{
\NewLine
\forall n - 1 \in \Nat \. \THM{IteratedAbsHomogen}(n - 1) \Rightarrow \THM{IteratedAbsHomogen}(n)}
\Conclude{(*)}{E(\Nat)(1)(2)}{ \THM{IteratedAbsHomogen} }
\EndProof
}
\subsection{Conjugation}
\Page{
	\DeclareType{ConjugationMap}{\prod K : \TYPE{AbsoluteValueField}(R) \. 
		\prod (1) : R \subset_{\mathsf{RING}} K \. 
		?(K \to K)
	}
\DefineType{\zeta}{ConjugationMap}{ \forall x,y \in K \.  x\zeta(x) = |x|^2 
\And  \zeta(xy) = \zeta(x)\zeta(y) 
\And \NewLine \And  \zeta(x + y) = \zeta(x) + \zeta(y) 
\And  \zeta\big( \zeta(x) \big) = x
}
\\
& \TYPE{ConjugationField} = \sum K : \TYPE{AbsoluteValueField}(R) \. \sum (1) : R \subset_{\mathsf{RING}} K \. 
 \TYPE{ConjugationMap}(k) \\
 \\
\DeclareFunc{synecdoche}{ \TYPE{ConjugationField}(R) \to \TYPE{AbsoluteValueField}(R) }
\DefineFunc{synecdoche}{(K,\ldots,\zeta)}{K}
\\
\DeclareFunc{conjugationMap }{ \prod (K, (1),\zeta) :  \TYPE{ConjugationField}(R) \. \TYPE{ConjugationMap}(K,(1))}
\DefineNamedFunc{conjugationMap}{}{\overline{\cdot}}{\zeta}
\\
\DeclareType{RealStrucure}{ \prod K : \TYPE{ConjugationField}(R) \. ?K  }
\DefineNamedType{x}{RealStructure}{ x \in \Reals (K) }{ \overline{x} = x  }
}
\end{document}
