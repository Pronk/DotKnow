\documentclass[12pt]{article}
\usepackage{mathtools}
\usepackage{amsmath}
\usepackage{amsfonts}
\usepackage{amssymb}
\usepackage{wasysym}
\usepackage{accents}
\usepackage[hidelinks]{hyperref}
\usepackage[dvipsnames]{xcolor}
\usepackage[top=20mm, bottom=20mm, left=30mm, right=10mm]{geometry}
\begin{document}

1) $K$ is  a splitting field of $P$, hence it is normal.  $P'(X) = -a \neq 0 $ which means that $P$ is separable, hence $K$ is Galois.

We can write
$$
 P(X) = \prod^p_{i = 1}(X - \alpha^i).
$$

By using structure of $P$, it is known that 

$$
  \sum_{I \in S(n, p)}  (-1)^n \prod_{i \in I} \alpha_i  = 0
$$
 for  $n : 1 \le n < p - 1$, and

$$
  \sum_{ I \in S(p - 1, p)} (-1)^{p - 1} \prod_{i \in i } \alpha_i = -a  
$$ 
Where $S(n,p)$ is set of subsets of $p$ of size $n$.

By expressing this value of $\alpha_i$ as functions  of $\alpha_{i + 1}, \ldots, \alpha_p $ as substituting them, it must finally yield

$$
  a = ( \alpha_1 - \alpha_2 )^{p - 1} = \beta^{p - 1}
$$ 

Other roots of $X^{p-1} - a$ will have form $ z \beta $ for all nonzero elements $z \in \mathbb{F}_p$ as $(z \beta )^{p - 1} = z^{p-1}\beta^{p-1} = \beta^{p - 1} $  . 

2) Hence, this polynomial must have cyclic Galois group generated by action $ \beta \mapsto 2\beta $ and isomorphic to multiplicative group  $\mathbb{F}_p^*$ . 

3) As order of $\alpha_i$ was arbitrary, $gx - x = z\beta$ in case $g \neq \mathrm{id}$ or otherwise $gx - x = 0$.

In case $g \in H$ when it is stable on roots $X^{p -1} -  a$. So if $g = \mathrm{id}$, it yields 0 on all $x$. Otherwise  $gx - x = gy - y$ for all $y$ in the orbit of $x$. But as splitting field of $P$ the group $H$ must be cyclic  (for root $\alpha$ element $\alpha + z\beta$ is also a root, so it must be spawned by map $\alpha \mapsto \alpha + \beta $ ), so the result doesn't really depends on choice of $x$.

4) In case $P$ splits over $L$ group $H$ is trivial ($|H| = 1$). In the other case group $H$ must be cyclic group rotating all roots of $P$ so the only choice is $\mathbb{Z}/p{\mathbb{Z}}$ ($|H| = p$).

5)  If $H = \mathbb{Z}{pZ}$  as it was said before this indicates that $P$ is irreducible over $L$ (otherwise $H$ is trivial),  moreover as $k \subset L$ this means that $P$ is irreducible over $k$. 

Now  assume that $E$ is trivial. As 

$$
 (X  + \beta)^p - a(X + \beta) - b = X^p + \beta^p - aX -a\beta - b =  X^p - aX - b + \beta( \beta^{p - 1} - a) = X^p - aX - b
$$

this means that $0 = \beta = a_i - a_j$.  As $P$ is separable this means that $P$ is reducible over $k$. So if  $P$ is irreducible over $k$ the group $ H = \mathbb{Z}/p\mathbb{Z}$. 

6) Polynomial $  X^{p-1} -T$ is irreducible in $\mathbb{F}_p(T)$. So $L$ is nontrivial extension of $k$. $P$ also does not split over $\mathbb{F}_p(S)$ so $H \cong \mathbb{Z}/p \mathbb{Z}$. This means that $ |\mathrm{Gal}(K/k)| =p(p-1)$.

\end{document}
