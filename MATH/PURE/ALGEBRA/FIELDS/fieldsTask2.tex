\documentclass[12pt]{article}
\usepackage{mathtools}
\usepackage{amsmath}
\usepackage{amsfonts}
\usepackage{amssymb}
\usepackage{wasysym}
\usepackage{accents}
\usepackage[hidelinks]{hyperref}
\usepackage[dvipsnames]{xcolor}
\usepackage[top=20mm, bottom=20mm, left=30mm, right=10mm]{geometry}
\begin{document}
\section{Problem about seventh roots of unity}
\subsection*{Question a)}
We show that polynomial  $P(X) = \sum^{p-1}_{k=0} X^{k}$ is irreducible for prime $p$.
fitstly w will make a substitution $X = Y + 1$. \
Then coeficient of $Y^k$ will have form 
$$\sum^{p - 1}_{n = k} \binom {n}{k} = \binom {p}{k+ 1} = \frac{p!}{(k+1)!(p - k - 1)!} $$ 
by Christmass stocking theorem. So $Y^{p-1}$ will have coefficient $1$, $Y^0$ will have coefficient $p$,
 and all other coeficients will be divisible by $p$. 
This means that by Eisenstein's criterion $P$ is irreducible over $ \mathbb{Q}$.
\subsection*{Question b)}
 $P(X) = \sum^{6}_{k=0} X^k = \frac{X^7 - 1}{X - 1}  $  is the minimal polynomial of $\zeta$ (irreducible by (a),monic and has $\zeta$ as root). This means that $[L : \mathbb{Q}] = \deg P = 6  $ 
\subsection*{Question c)} 
We will use the fact that $\zeta^{-1} = \zeta^6 $,$\zeta^6 + \zeta = 2\cos(2\pi/7) \in $
  Note that we can factor $P$ over $M$:
  \begin{multline*}
   P(X) =  \prod^6_{i = k}( X - \zeta^k) = (X^2  - (\zeta + \zeta^6)X + 1)
   (X^2 - (\zeta^2 + \zeta^5)X + 1)(X^2 - (\zeta^3 + \zeta^4)X + 1) = \\=
   (X^2 - 2 \cos(2\pi/7)X + 1)
   (X^2 - 2 \cos(4\pi/7)X + 1)
   (X^2 - 2 \cos(6\pi/7)X + 1).
  \end{multline*}
  So the minimal polynomial for $\zeta$ over $M$ is $ Q(X) = X^2 - 2 \cos(2\pi/7)X + 1$ . 
  So $[L:M] = 2$ and  as $[L : \mathbb{Q}] = [L : M][M : \mathbb{Q}]$ it is clear that $[M : \mathbb{Q}] = 3$.
\subsection*{Question d)}
  The group of automorphisms of $P$ is generated by $\zeta \mapsto \zeta^2$, so there exist following options for $f(\zeta)$:
$$
 \{ \zeta, \zeta^2,\zeta^3,\zeta^4,\zeta^5,\zeta^6\}
$$ 
and following options for $f(\cos(2\pi/7))$
$$
\{ \cos(2\pi/7),\cos(4\pi/7) , \cos(6\pi/7)   \}
 $$
because automorphisms always map inverse into inverse
$$
  f \left( \frac{1}{2}(\zeta + \zeta^{-1})\right) =  \frac{1}{2}(f(\zeta) + (f(\zeta))^{-1})
$$
\end{document}
