\documentclass[12pt]{scrartcl}% European-style article
\usepackage{mathtools}%For basic mathimatical symbols
\usepackage{amsmath}  %For basic mathimatical symbols
\usepackage{amsfonts} %For mathematical fonts
\usepackage{hyperref} %For clickable contents
\usepackage{amssymb}  %For more mathematical symbols
\usepackage{wasysym}  %For astronomic symbols
\usepackage{accents}  %For accents
\usepackage{graphicx} %For images
\usepackage{scalerel} %For resizing operators
\usepackage[dvipsnames]{xcolor}
\usepackage[a4paper,top=5mm, bottom=5mm, left=10mm, right=2mm]{geometry}
%Markup
%To visually distinguish different things
\newcommand{\TYPE}[1]{\textcolor{NavyBlue}{\mathtt{#1}}}% Types are things that have members
\newcommand{\FUNC}[1]{\textcolor{Cerulean}{\mathtt{#1}}}% Func are things that trandform typed values
\newcommand{\LOGIC}[1]{\textcolor{Blue}{\mathtt{#1}}}% Logical elements are beyond the scope of the type theory
\newcommand{\THM}[1]{\textcolor{Maroon}{\mathtt{#1}}}% Theorems are things, which need to be proven
%META
%Basic elements of the language
\renewcommand{\.}{\; . \;} %to separate elements of quantified statements
\newcommand{\de}{: \kern 0.1pc =} %to define values of objects
\newcommand{\extract}{\LOGIC{Extract}} %produces a whitness of the existentionally typed object !Legacy! use E(\exists) instead
\newcommand{\where}{\LOGIC{where}} % used to define values post-factum
\newcommand{\If}{\LOGIC{if} \;} % A part of a famous trenary operator
\newcommand{\Then}{ \; \LOGIC{then} \;} % A part of a famous trenary operator
\newcommand{\Else}{\; \LOGIC{else} \;} % A part of a famous trenary operator
\newcommand{\IsNot}{\; ! \;} % A negation for a compound type (Is not a member of the Type, but of the same essence)
\newcommand{\Is}{ \; : \;}  % Type membership
\newcommand{\DefAs}{\; :: \;} % Defuened ti be a membere of a Type (essence)
\newcommand{\Act}[1]{\left( #1 \right)} % Func acts on an object  !Legacy?
\newcommand{\Example}{\LOGIC{Example} \; } % Used to identify examples !Legacy! we don't have examples any more
\newcommand{\Theorem}[2]{& \THM{#1} \, :: \, #2 \\ & \Proof = \\ } % An environment for declaring and defining=prooving a theorem
\newcommand{\DeclareType}[2]{& \TYPE{#1} \, :: \, #2 \\}% An environment for declaring a type (name + essence)   
\newcommand{\DefineType}[3]{& #1 : \TYPE{#2} \iff #3 \\}% An environment for defining a type (member + name + defining Type )
\newcommand{\DefineNamedType}[4]{& #1 : \TYPE{#2} \iff #3 \iff #4 \\}%An environment for defining a type (member +  name + symbol + defining Type ) 
\newcommand{\DeclareFunc}[2]{& \FUNC{#1} \, :: \, #2 \\}% An environment for declaring a func (name + type)   
\newcommand{\DefineFunc}[3]{&  \FUNC{#1}\Act{#2} \de #3 \\}% An environment for defining a type (name + argument + value expression) 
\newcommand{\DefineNamedFunc}[4]{&  \FUNC{#1}\Act{#2} = #3 \de #4 \\}% An environment for defining a type (name + argument + symbol + value expression)  
\newcommand{\NewLine}{\\ & \kern 1pc}% A shorthand for breaking a line inside Page environment      
\newcommand{\Page}[1]{ \begin{align*} #1 \end{align*}  }% An environment for writting this shit
\newcommand{ \bd }{ \ByDef }% A shorthand                                                  
\newcommand{\NoProof}{ & \ldots \\ \EndProof}% An omission of the prove of the theorem
\renewcommand{\And}{\; \& \;}% A typological and logical and
\newcommand{\Type}{\TYPE{Type}}% A metatype of Types
%%STD
%Standard mathematical graphic
\newcommand{\Int}{\mathbb{Z}}% Integers
\newcommand{\NNInt}{\mathbb{Z}_{+}}% Positive Integers
\newcommand{\Reals}{\mathbb{R}}% Real Numbers
\newcommand{\Complex}{\mathbb{C}}% Complex Numbers
\newcommand{\Rats}{\mathbb{Q}}% Rational Numbres
\newcommand{\Nat}{\mathbb{N}}% Natural Numbers
\newcommand{\EReals}{\stackrel{\mathclap{\infty}}{\mathbb{R}}}% Extended real Numbers
\newcommand{\ERealsn}[1]{\stackrel{\mathclap{\infty}}{\mathbb{R}}^{#1}}% Extended Real Plane
\DeclareMathOperator*{\argmin}{arg\,min}% arg min
\DeclareMathOperator*{\id}{id}% identity map
\DeclareMathOperator*{\im}{Im}% an image of the function
\DeclareMathOperator*{\supp}{supp}% a support of something
\newcommand{\EqClass}[1]{\TYPE{EqClass}\left( #1 \right)}% An Equivalence Classes
\newcommand{\Cat}{\TYPE{Category}}% Type of categories
\newcommand{\Mor}{\mathcal{M}}% morphisms of the category
\newcommand{\Obj}{\mathcal{O}}% objects of the category
\newcommand{\Aut}{\mathrm{Aut}}% automorphisms of the object in the category
\newcommand{\End}{\mathrm{End}}% automorphisms of the object in the category
\mathchardef\hyph="2D % a hyphen for the use in the math mode 
\newcommand{\ToInj}{\hookrightarrow} % An arrow for injective maps
\newcommand{\ToSurj}{\twoheadrightarrow} % An arrow for the surjective maps
\newcommand{\ToBij}{\leftrightarrow} % A arrow for the bijective maps
\newcommand{\Set}{\TYPE{Set}} % Type of sets
\newcommand{\du}{\; \triangle \;} % symmetric difference
\renewcommand{\c}{\complement}% set-theoretic complement
\newcommand{\Imply}{\Rightarrow}
%%ProofWritting
% Commands to write proofs
\newcommand{\Say}[3]{& #1 \de #2 : #3, \\} % A Logical Statements (name + expression + type of the expression)
\newcommand{\Conclude}[3]{& #1 \de #2 : #3; \\}% A conclusion which ends a reflection (name with end pointer + expression + type of the expression )
\newcommand{\Derive}[3]{& \leadsto #1 \de #2 : #3, \\} % A Result produced by conlcuding the reflection, must follow comclusion (name + post-reflection + type)         
\newcommand{\DeriveConclude}[3]{& \leadsto #1 \de #2 : #3 ; \\} % Use to follow a conclusion by an another conclusion imedietely ( name with end pointer + post-reflection + type  )
\newcommand{\Assume}[2]{& \LOGIC{Assume} \; #1 : #2, \\} %Starts a reflection (name + type)
\newcommand{\As}{\; \LOGIC{as } \;} %An ambigous symbol (Legacy)
\newcommand{\QED}{\; \square} %A symbol to end the prove
\newcommand{\EndProof}{& \QED \\} %End of prove
\newcommand{\ByDef}{\rotatebox[origin=c]{-180}{$D$}}%\text{\textthorn}}  %Extracts defining type statement from the type member, may be inverted  (T -> Type)
\newcommand{\ByConstr}{\rotatebox[origin=c]{-180}{$C$}}%\text{\textopeno}} %Extract the defining statement from the defined value !Legacy! use \eth instead  
\newcommand{\Alt}{\LOGIC{Alternative} \;} % Can be used to check multiple alterntives inside the prove !Undeveloped!
\newcommand{\CL}{\LOGIC{Close} \;} % Was Intended for the use with the Alternative !Undeveloped!
\newcommand{\More}{\LOGIC{Another} \;} % Was Intended for the use with the Alternative !Undeveloped! 
\newcommand{\Proof}{\LOGIC{Proof} \; } % Begins a Prove
%FOUND
%Foundations of mathematics
%CAT
%Catgory Theory
\newcommand{\Arrow}[1]{\xrightarrow{#1}}% an arrow representatition of the morphism
\newcommand{\ToIso}[1]{\xleftrightarrow{#1}}% an arrow representation of the isomprphism
%CategoryTheorey
%Types
\newcommand{\Cov}{\TYPE{Covariant}}% A type of Covariant functors
\newcommand{\Contra}{\TYPE{Contravariant}}% A type of the Contravariant Functors
\newcommand{\NT}{\TYPE{NaturalTransform}}% A type of the Natural Transormations
\newcommand{\UMP}{\TYPE{UnversalMappingProperty}}% A type of catgories with the universal mapping property ?
\newcommand{\CMP}{\TYPE{CouniversalMappingProperty}}% A type of categories with the couniversal mapping property ?
\newcommand{\paral}{\rightrightarrows} %?
%functions
\newcommand{\op}{\mathrm{op}} %opposite cotegory
\newcommand{\obj}{\mathrm{obj}} %objects?
\DeclareMathOperator*{\dom}{dom} % domain
\DeclareMathOperator*{\codom}{codom}% codomain
\DeclareMathOperator*{\colim}{colim}% colimit
%variable
% Varianles for denoting categories
\newcommand{\C}{\mathcal{C}}
\newcommand{\A}{\mathcal{A}}
\newcommand{\B}{\mathcal{B}}
\newcommand{\D}{\mathcal{D}}
\newcommand{\I}{\mathcal{I}}
\newcommand{\J}{\mathcal{J}}
\newcommand{\R}{\mathcal{R}}
\newcommand{\G}{\mathsf{G}}
%Cats
\newcommand{\CAT}{\mathsf{CAT}} % 2-Category of all Categories
\newcommand{\SET}{\mathsf{SET}} % Category of Sets
\newcommand{\PARALLEL}{\bullet \paral \bullet} % A parallel category
\newcommand{\WEDGE}{\bullet \to \bullet \leftarrow \bullet} % Wedge category
\newcommand{\VEE}{\bullet \leftarrow \bullet \to \bullet} % Vee Category
%Algebra
%Abstract Algebra
%Groups
%Group Theory
%Types
\newcommand{\Group}{\TYPE{Group}} % Type of groups
\newcommand{\Abel}{\TYPE{Abelean}} % Type of abelean groups
\newcommand{\Sgrp}{\subset_{\mathsf{GRP}}} % Subgroup as a subset
\newcommand{\Nrml}{\vartriangleleft} % Normal Subgroup as a subset
\newcommand{\FG}{\TYPE{FiniteGroup}} % Finite Groups
\newcommand{\Stab}{\mathrm{Stab}}  % A stabilizer
%\newcommand{\FGA}{\TYPE{FinitelyGeneratedAbelean}} % A Finitely Generated abelean group
\newcommand{\DN}{\TYPE{DirectedNormality}} % A noramal complex
%Func
\DeclareMathOperator{\tor}{tor} % torsion
\DeclareMathOperator{\bool}{bool} % boolinization
\DeclareMathOperator{\rank}{rank} % a rank
%Cats
\newcommand{\GRP}{\mathsf{GRP}} % A category of Groups
\newcommand{\ABEL}{\mathsf{ABEL}} % a category of Abelean Groups
%Ops
\newcommand{\SDP}{\rightthreetimes} % A very special norm
%LINEAR
%Linear Algebra
%Types
\newcommand{\Basis}{\TYPE{Basis}} % Basis of the linear space
\newcommand{\submod}[1]{\subset_{\LMOD{#1}}}% submodule as a subset
\newcommand{\subvec}[1]{\subset_{\VS{#1}}}% vector subspace as a subset
\newcommand{\FGM}{\TYPE{FinitelyGeneratedModule}}% Finitely generated module
\newcommand{\LI}{\TYPE{LinearlyIndependent}}
\newcommand{\LIS}{\TYPE{LinearlyIndependentSet}}
\newcommand{\FM}{\TYPE{FreeModule}}
\newcommand{\IBP}{\TYPE{InvariantBasisProperty}}
\newcommand{\UTM}{\TYPE{UpperTriangularMatrix}}
\newcommand{\LTM}{\TYPE{LowerTriangularMatrix}}
\newcommand{\Diag}{\TYPE{DiagonalMatrix}}
\newcommand{\FP }{\TYPE{FinitelyPresented}}
\newcommand{\GL}{\mathrm{GL}}% General Linear Group
\newcommand{\SL}{\mathrm{SL}}% Special Linear group
\newcommand{\prsubvec}[1]{\subsetneq_{\VS{#1}}}	% poper vectoe subspace as a subset
\newcommand{\LC}{\TYPE{LinearComplement}} 
\newcommand{\IS}{\TYPE{InvariantSubspace}}
\newcommand{\RP}{\TYPE{ReducingPair}}
\newcommand{\RCF}{\TYPE{RationalCanonicalForm}}
\newcommand{\JCF}{\TYPE{JordanCanonicalForm}}
\newcommand{\Diagble}{\TYPE{Diagonalizable}}
\newcommand{\UT}{\TYPE{UpperTriangulizable}}
\newcommand{\LT}{\TYPE{LowerTriangulizable}}
\newcommand{\QF}{\TYPE{QuadraticForm}}
%Func
\DeclareMathOperator{\Span}{span} % spann by subset
\DeclareMathOperator{\Ann}{Ann}   % annihilator
\DeclareMathOperator{\Ass}{Ass}   % associated primes
\DeclareMathOperator{\diag}{diag} % diagonal
\DeclareMathOperator{\adj}{adj}   % an adjoint matrix
\DeclareMathOperator{\tr}{tr}     % trace
\DeclareMathOperator{\codim}{codim} % codimension
\DeclareMathOperator{\Cell}{\mathbf{C}} % a componion matrix
\DeclareMathOperator{\JC}{\mathbf{J}}  % a Jordan cell
\DeclareMathOperator{\bigboxplus}{\scalerel*{\boxplus}{\sum}} % a direct sum of operators in the sence of the reducing a pair
\DeclareMathOperator{\bigboxtimes}{\scalerel*{\boxtimes}{\sum}} % a direct sum of operators in the sence of the reducing a pair
\DeclareMathOperator{\Spec}{Spec} % Spectre
\DeclareMathOperator{\TensPow}{\mathbf{T}}
\DeclareMathOperator{\pf}{pf} % Pfaffian 
\DeclareMathOperator{\perm}{perm} % Permanent
%Cats
\newcommand{\VS}[1]{#1\hyph\mathsf{VS}} % a category of vector spaces (Field)
\newcommand{\FDVS}[1]{#1\hyph\mathsf{FDVS}} % a category of finite-dimensional vector spaces (Field)
\newcommand{\LMOD}[1]{#1\hyph\mathsf{MOD}} % a category of the left modules (Ring)
\newcommand{\RMOD}[1]{\mathsf{MOD}\hyph#1} % a category of the right modules (Ring)
\newcommand{\LLMAP}[1]{#1\hyph\mathsf{LMAP}} % a cagory of based linear maps with the left scalar multiplication (Ring)
\newcommand{\LMAT}[1]{#1\hyph\mathsf{MAT}}  % a category of based matrices with the left scalar multiplication (Ring)
\newcommand{\NMAT}[1]{#1\hyph\mathbb{N}} % a category of finite matrices (Field)
%Symbols
\renewcommand{\L}{\mathcal{L}}
%FIELDS
\newcommand{\Field}{\TYPE{Field}}
\newcommand{\ACF}{\TYPE{AlgebraicallyClosedField}}
%RINGS
%TYPE
\newcommand{\Ring}{\TYPE{Ring}}
\newcommand{\CR}{\TYPE{CommutativeRing}}
\newcommand{\Ideal}{\TYPE{Ideal}}
\newcommand{\ID}{\TYPE{IntegralDomain}}
\newcommand{\UFD}{\TYPE{UniqueFactorizationDomain}}
\newcommand{\PID}{\TYPE{PrincipleIdealDomain}}
\newcommand{\FGI}{\TYPE{FinitelyGeneratedIdeal}}
\newcommand{\ER}{\TYPE{EuclideanRing}}
\newcommand{\DVR}{\TYPE{DiscreteValuationRing}}
\newcommand{\MoFT}{\TYPE{MonoidOfFiniteType}}
\newcommand{\MS}{\TYPE{MultiplicativeSubset}}
%CATS
\newcommand{\RING}{\mathsf{RING}} % A category of Rings
\newcommand{\ANN}{\mathsf{ANN}} % A category of Commutative Rings
%FUNCS
\DeclareMathOperator{\lcd}{lcd} % least common devided 
\DeclareMathOperator{\lc}{lc} % leading coefficient of the polynomial
\DeclareMathOperator{\cont}{cont} % content of the polynomial
\DeclareMathOperator{\antideg}{antideg} % antideree if the foramal power series
%Symbols
\newcommand{\F}{\mathcal{F}}
%ALGEBRA
\newcommand{\Algebra}{\TYPE{Algebra}}
\newcommand{\LALG}[1]{#1\hyph\mathsf{ALG}}% Left associative unital algebras (Ring)
\newcommand{\RALG}[1]{\mathsf{ALG}\hyph#1}% Right associative unital  algebras (Rings)
\newcommand{\LALGE}[1]{#1\hyph\mathsf{ALGE}}% Left associative unital algebras (Ring)
\newcommand{\RALGE}[1]{\mathsf{ALGE}\hyph#1}% Right associative unital  algebras (Rings)
\newcommand{\LLGE}[1]{#1\hyph\mathsf{LGE}}% Left associative unital algebras (Ring)
\newcommand{\RLGE}[1]{\mathsf{LGE}\hyph#1}% Right associative unital  algebras (Rings)
\newcommand{\LLG}[1]{#1\hyph\mathsf{LG}}% Left associative unital algebras (Ring)
\newcommand{\RLG}[1]{\mathsf{LG}\hyph#1}% Right associative unital  algebras (Rings)
\newcommand{\LCALG}[1]{#1\hyph\mathsf{CALG}}% Left associative unital algebras (Ring)
\newcommand{\RCALG}[1]{\mathsf{CALG}\hyph#1}% Right associative unital  algebras (Rings)
\newcommand{\LCALGE}[1]{#1\hyph\mathsf{CALGE}}% Left associative unital algebras (Ring)
\newcommand{\RCALGE}[1]{\mathsf{CALGE}\hyph#1}% Right associative unital  algebras (Rings)
\newcommand{\LCLGE}[1]{#1\hyph\mathsf{CLGE}}% Left associative unital algebras (Ring)
\newcommand{\RCLGE}[1]{\mathsf{CLGE}\hyph#1}% Right associative unital  algebras (Rings)
\newcommand{\LCLG}[1]{#1\hyph\mathsf{CLG}}% Left associative unital algebras (Ring)
\newcommand{\RCLG}[1]{\mathsf{CLG}\hyph#1}% Right associative unital  algebras (Rings)
\newcommand{\FGA}{\TYPE{FinitelyGeneratedAlgebra}}
\newcommand{\FGCA}{\TYPE{FinitelyGeneratedCommutativeAlgebra}}
\newcommand{\PGA}{\TYPE{PolynomialGradedAlgebra}}
\newcommand{\COALG}[1]{#1\hyph\mathsf{COALG}}
\newcommand{\CCOALG}[1]{#1\hyph\mathsf{CCOALG}}
\newcommand{\SCOALG}[1]{#1\hyph\mathsf{SCOALG}}
\newcommand{\hit}{\rightharpoonup}
\newcommand{\hitBy}{\leftharpoonup}
\newcommand{\LAMOD}[1]{{\;}_{#1}\mathsf{MOD}}
\newcommand{\RAMOD}[1]{\mathsf{MOD}_{#1}}
\newcommand{\LCOMOD}[1]{{\;}^{#1}\mathsf{MOD}}
\newcommand{\RCOMOD}[1]{\mathsf{MOD}^{#1}}
\newcommand{\BIALG}[1]{#1\hyph\mathsf{BIALG}}
\newcommand{\LBALG}[1]{{\;}_{#1}\mathsf{ALGE}}
\newcommand{\RBALG}[1]{\mathsf{ALGE}_{#1}}
\newcommand{\RBCOALG}[1]{\mathsf{COALG}_{#1}}
\newcommand{\LBCOALG}[1]{{\;}_{#1}\mathsf{COALG}}
\newcommand{\HOPF}[1]{#1\hyph\mathsf{HOPF}}
\newcommand{\RHMOD}[1]{\mathsf{MOD}^{#1}_{#1}}
%HOMOLOG
\newcommand{\Exact}{\TYPE{Exact}}
\newcommand{\ShortExact}{\TYPE{ShortExact}}
\newcommand{\CH}[1]{#1\hyph\mathsf{CH}}
%Numbers
%Integers
%FUNCS
\DeclareMathOperator{\divi}{div} % devide withou reminder
\DeclareMathOperator{\remi}{rem} % reminder
\DeclareMathOperator{\Frac}{Frac} % Field of fractions
\title{Multilinear Algebra}
\author{Uncultured Tramp}
\begin{document}
\maketitle
\normalsize
\newpage
\tableofcontents
\newpage
\section{From Bilinear Maps to Tensor Products}
\subsection{Multilinear Maps and Forms}
\Page{
	\DeclareFunc{Multilinear}{\prod R \in \ANN \. \left(\prod n \in \Nat \. n \to \LMOD{R} \right) \to \LMOD{R} \to \LMOD{R}}
	\DefineNamedFunc{Multilinear}{1 \mapsto V,M}{\L(V;M)}{\Mor_{\LMOD{R}}(V,M)}
	\DefineNamedFunc{Multilinear}{V,M}{\L(V;M)}{\Mor_{\LMOD{R}}\big(V_1,\L(V';M)\big) \quad \where \quad V' \de \Lambda i \in n-1 \. V_{i+1}}
	\\
	\DeclareFunc{multiEval}{\prod R \in \ANN \. \prod n \in \Nat \. \prod V : n \to \LMOD{R} \. \prod W \in \LMOD{R} \. \NewLine \. 
		\L(V;W) \to \left(\prod i \in n \. V_i\right) \to W}
	\DefineNamedFunc{multiEval}{T,v}{T(v)}{T(v_1)(v') \quad \where \quad v' \de \Lambda i \in n-1 \. v_{i+1}}
	\\
	\DeclareType{NForm}{\prod R \in \ANN \. \prod V,W \in \LMOD{R} \. \Nat \to ?(V \to R)}
	\DefineType{F}{NForm}{\Lambda n \in \Nat \. \exists T \in \L(i \mapsto  V ; R) : \forall v \in V \. F(v) = L(i \mapsto v)}
	\\
	\DeclareFunc{coordinateTensor}{\prod n \in \Nat \. \prod V : n \to \FM(R) \. \prod W \in \FM(R) \. \NewLine \.  
		\L(V;W) \to \left( \prod i \in n \. \Basis(V_i)\right) \to \Basis(W)  \to R^{\left(\prod^n_{i=1}\rank V_i\right) \times W} }
	\DefineNamedFunc{coordinataTensor}{T,e,f}{T^{e;f}}
	{\Lambda j : \prod i \in n \. \rank V_i \. \Lambda k \in \rank W \. \alpha_k \quad \where \quad  \alpha f = L(i \mapsto e_{i,j_i}) }
	\\
	\DeclareFunc{multiFromCoordinates}{\prod n \in \Nat \. \prod V : n \to \FM\And \FGM(R) \. \NewLine \. \prod W \in \FM \And \FGM(R)  \.  
		 R^{\left(\prod^n_{i=1}\rank V_i\right) \times W}  \to \left( \prod i \in n \. \Basis(V_i)\right) \to \NewLine \to \Basis(W) \to  \L(V;W)  }
	\DefineNamedFunc{multiFromCoordinates}{A,e,f}{A_{e;f}}
	{\Lambda \alpha e_1  \in V_1 \. \sum^{\rank V_1}_{i=1} \alpha_i (A_i)_{e',f}  \quad \where \quad  e' \de \Lambda i \in n-1 \. e_{i+1} }
	\\
	\DeclareFunc{Bilinear}{\prod R \in \ANN \. \LMOD{R}^3 \to \LMOD{R} }
	\DefineNamedFunc{Bililinear}{A,B,W}{\L(A,B;W)}{\L(\lambda i \in 2 \. \If i == 1 \Then A \Else B ; W )}
	\\
	\Theorem{PermutationIsomorphism}{
		\forall R \in \ANN \. 
		\forall n \in \Nat \. 
		\forall V : n \to \LMOD{R} \. 
		\forall W \in \LMOD{R} \. \NewLine
		\forall \sigma \in S_n \.
		\L(V;W) \cong_{\LMOD{R}} \L(\sigma V;W)
		}
	&  \sigma^* T(v) \de T(\sigma v) \quad \text{definetly acts as an isomorphism with the inverse provided by the $\sigma^{-1}$.} \\
	\EndProof
	\\
	\Conclude{\QF}{\Lambda R \in \ANN \. \Lambda A,B \in \LMOD{R} \. \TYPE{NForm}(R,2)\Big(\Lambda i \in 2 \. \If i == 1 \Then A \Else B\Big)}
	{ \NewLine : \prod R \in \ANN \. \LMOD{R}^2 \to \Type}
}\Page{
	\Theorem{MultiAdditive}{
		\forall R \in \ANN \. 
		\forall n \in \Nat \. 
		\forall V : n \to \LMOD{R} \. 
		\forall W \in \LMOD{R}  \.  
		\forall T \in \L(V;W) \. \NewLine \.
		\forall v,v',v'' \. 
		\forall i \in n \. 
		\forall a,b \in V_i \. 
		\forall [0] : v_i = a + b \. 
		\forall [00] : v'_i = a \. 
		\forall [000] : v''_i = b \. \NewLine \. 
		\forall [0000] : \forall j \in n \. j \neq i \Rightarrow v_j = v'_j = v''_j \.
		T(v) = T(v') + T(v'')
	}
	\Say{\cancer}{\Lambda n \in \Nat \. \forall m \in \Nat  \.  m \le n \Rightarrow \LOGIC{This}(R)(n)}{\Nat \to \Type}
	\Assume{[1]}{n = 1}
	\Conclude{[1.*]}{\bd \L(V;W)(T)[1]\bd \LMOD{R}(V,W)(T)[0][00][000]}{T(v) = T(a + b) = T(a) + T(b) = T(v') + T(v'')}
	\Derive{[1]}{\ByConstr^{-1}\cancer}{\cancer(1)}
	\Assume{m}{\Nat}
	\Assume{[m.2]}{\cancer(m)}
	\Assume{[m.3]}{n = m + 1}
	\Assume{[m.4]}{i = 1}
	\Say{\hat V}{\Lambda j \in m \. V_{j+1}}{m \to \LMOD{R}}
	\Say{\hat v}{\Lambda j \in m \. v_{j+1} }{\prod_{j=1}^m \hat V_j}
	\Conclude{[m.4.*]}{ \bd \FUNC{multiEval}(T,v)\bd \LMOD{R}(V_1,\L(\hat V;W))[0][00][000]\ByConstr \bar v }
	{
		\NewLine :	T(v) = T(a + b)(\bar v) =  T(a)(\bar v) + T(b)(\bar v) = T(v') + T(v'')
	}
	\Derive{[m.4]}{I(\Rightarrow)}{i = 1 \Rightarrow  T(v) = T(v') + T(v'')}
	\Assume{[m.5]}{i \neq 1}
	\Say{\bar V}{ V_{|i-1}}{(i-1) \to \LMOD{R}}
	\Say{\hat V}{ \Lambda j \in m + 2 - i \.  V_{i + j - 1} }{(n + 2 - i) \to \LMOD{R}}
	\Say{[m.6]}{\THM{NonegativeAdditionNondecrease}[m.5][m,3] + (i - 2)}{m + 2 - i \le m}
	\Say{\bar v}{v_{|i-1}}{\prod_{j=1}^{i-1} \hat V_j}
	\Say{\hat v}{\Lambda j \in m + 2 - i \. v_{i + j-1}}{ \prod^{n+2-i}_{j=1} \hat V_j}
	\Say{\hat v'}{\Lambda j \in m + 2 - i \. v_{i + j-1}'}{\prod^{n+2-i}_{j=1} \hat V_j}
	\Say{\hat v''}{\Lambda j \in m + 2 - i \. v_{i + j-1}''}{\prod^{n+2-i}_{j=1} \hat V_j}
	\Conclude{[m.5.*]}{\ByConstr^{-1}\bar v \ByConstr^{-1}\hat v [m.2](n+ 2 -i ,[m.6])[0][00][000][0000] \ByConstr \hat v'' \ByConstr \hat v'}
	{
		\NewLine :	T(v) = T(\bar v)(\hat v) = T(\bar v)(\hat v') + T(\bar v)(\hat v'') = T(v') + T(v'')
	}
	\Derive{[m.5]}{I(\Rightarrow)}{i \neq 1 \Rightarrow T(v) = T(v') + T(v'')}
	\Say{[m.6]}{\THM{EqAlt}(\Nat,i,1)}{i = 1 | i \neq 1}
	\Conclude{[m.*]}{E(|)[m.4][m.5][m.6]}{T(v) = T(v') + T(v'')}
	\Derive{[*]}{\bd \TYPE{NaturalSet}(\Nat)[1]}{\LOGIC{This}(R)}
	\EndProof
}\Page{
	\Theorem{MultiHomogen}{
		\forall R \in \ANN \. 
		\forall n \in \Nat \. 
		\forall V : n \to \LMOD{R} \. 
		\forall W \in \LMOD{R}  \.  
		\forall T \in \L(V;W) \. \NewLine \.
		\forall v,v' \.
		\forall \omega \in A \.
		\forall i \in n \. 
		\forall a \in V_i \. 
		\forall [0] : v'_i = a  \. 
		\forall [00] : v_i = \omega a \. 
		\forall [000] : \forall j \in n \. j \neq i \Rightarrow v_j = v'_j  \.
		T(v) = \omega T(v')
	}
	\Say{\cancer}{\Lambda n \in \Nat \. \forall m \in \Nat  \.  m \le n \Rightarrow \LOGIC{This}(R)(n)}{\Nat \to \Type}
	\Assume{[1]}{n = 1}
	\Conclude{[1.*]}{\bd \L(V;W)(T)[1]\bd \LMOD{R}(V,W)(T)[0][00][000]}{T(v) = T(\omega a) = \omega T(a) = \omega T(v') }
	\Derive{[1]}{\ByConstr^{-1}\cancer}{\cancer(1)}
	\Assume{m}{\Nat}
	\Assume{[m.2]}{\cancer(m)}
	\Assume{[m.3]}{n = m + 1}
	\Assume{[m.4]}{i = 1}
	\Say{\hat V}{\Lambda j \in m \. V_{j+1}}{m \to \LMOD{R}}
	\Say{\hat v}{\Lambda j \in m \. v_{j+1} }{\prod_{j=1}^m \hat V_j}
	\Conclude{[m.4.*]}{ \bd \FUNC{multiEval}(T,v)\bd \LMOD{R}(V_1,\L(\hat V;W))[0][00][000]\ByConstr \bar v }
	{
		\NewLine :	T(v) = T(\omega a)(\bar v) =  \omega T(a)(\bar v)  = \omega T(v')
	}
	\Derive{[m.4]}{I(\Rightarrow)}{i = 1 \Rightarrow  T(v) =  \omega T(v')}
	\Assume{[m.5]}{i \neq 1}
	\Say{\bar V}{ V_{|i-1}}{(i-1) \to \LMOD{R}}
	\Say{\hat V}{ \Lambda j \in m + 2 - i \.  V_{i + j - 1} }{(n + 2 - i) \to \LMOD{R}}
	\Say{[m.6]}{\THM{NonegativeAdditionNondecrease}[m.5][m,3] + (i - 2)}{m + 2 - i \le m}
	\Say{\bar v}{v_{|i-1}}{\prod_{j=1}^{i-1} \hat V_j}
	\Say{\hat v}{\Lambda j \in m + 2 - i \. v_{i + j-1}}{ \prod^{n+2-i}_{j=1} \hat V_j}
	\Say{\hat v'}{\Lambda j \in m + 2 - i \. v_{i + j-1}'}{\prod^{n+2-i}_{j=1} \hat V_j}
	\Conclude{[m.5.*]}{\ByConstr^{-1}\bar v \ByConstr^{-1}\hat v [m.2](n+ 2 -i ,[m.6])[0][00][000][0000] \ByConstr \hat v'' \ByConstr \hat v'}
	{
		\NewLine :	T(v) = T(\bar v)(\hat v) = \omega T(\bar v)(\hat v')  = \omega T(v') 
	}
	\Derive{[m.5]}{I(\Rightarrow)}{i \neq 1 \Rightarrow T(v) = \omega T(v')}
	\Say{[m.6]}{\THM{EqAlt}(\Nat,i,1)}{i = 1 | i \neq 1}
	\Conclude{[m.*]}{E(|)[m.4][m.5][m.6]}{T(v) = \omega T(v')}
	\Derive{[*]}{\bd \TYPE{NaturalSet}(\Nat)[1]}{\LOGIC{This}(R)}
	\EndProof
	\\
	\Theorem{NFormNHomogen}{\forall R \in \ANN \. \forall n \in \Nat \. \forall V : \LMOD{R} \. \forall F : \TYPE{NForm}(R,V,n) \. \NewLine \. 
	\forall v \in V \. \forall \alpha \in R \. F(\alpha v) = \alpha^n F(v) }
	\Say{(T,[1])}{ \bd \TYPE{NForm}(R,V,n)}{ \sum T \in \L(\Lambda i \in n \. V; R) \. \forall v \in V \. F(v) = T(i \mapsto v)} 
	\Conclude{[*]}{[1] \THM{MultiHomogen}^n(T)(\ldots) [1]}{F(\alpha v) = T(i \mapsto \alpha v) = \alpha^n T(i \mapsto v) = \alpha^n F(v)}
	\EndProof
}\Page{
	\Theorem{PolarLemma}{ \forall R \in \ANN \. \forall V : \LMOD{R} \. \forall F : \QF(V) \. \forall v,w \in V \. 
		 \NewLine \. F(v+w) + F(v - w) = 2F(v) + 2F(w) }
	\Say{(T,[1])}{ \bd \TYPE{NForm}(R,V,n)}{ \sum T \in \L(V, V; R) \. \forall v \in V \. F(v) = T(v,v)} 
	\Conclude{[*]}{[1]\THM{MultiAdditive}^6(T)(\ldots)\THM{MultiHomogen}^4(T)(\ldots)\THM{SquareOfNegative}(-1)[1]}
	{
		\NewLine 
		F(v + w) + F(v - w) =  
		T(v+w,v+w) + T(v-w,v-w) = \NewLine =  
		T(v,v) + T(v,w) + T(w,v) + T(w,w) - 
		T(v,v) - T(v,w) - T(w,v) + T(v,v) = \NewLine = 
		2T(v,v) + 2T(w,w) =
		2F(v) + 2F(w)
	}
	\EndProof
	\\
	\DeclareFunc{multiNullset}{\prod n \in \Nat \. \prod V : n \to \LMOD{R} \. \prod i \in n \. \L(V;W) \to \TYPE{Submodule}(V_i)  }
	\DefineNamedFunc{multiNullset}{ T }{N_i(T)}
	{\left\{ v \in V_i : \forall w : \prod j \in n \setminus \{i\} \. V_j \. T\Big(j \mapsto \If j == i \Then v \Else w_j\Big) = 0  \right\}}
	\\
	\DeclareFunc{multiReduction}{ \prod T \in \L(V;W) \.   \L\Act{\frac{V}{N(T)};W}}
	\DefineNamedFunc{multiReduction}{T}{\tilde T}{  \FUNC{reduce}(T) \Else \FUNC{Reduce} \; \Lambda v \in V_1 \. \FUNC{multiReduce}(T(v))  }
	\\
	\Theorem{ReducedMultiIsReduced}{\forall T \in \L(V;W) \. \forall i \in n \. N_i(\tilde T) = \{0\}}
	\Assume{[v]}{N_i(\tilde T)}
	\Say{[1]}{\bd N_i(\tilde T)}{\forall [w] : \prod j \in n \setminus \{i\} \. \frac{V_j}{N_j(T)} \. 
		\tilde T\Big(j \mapsto \If j == i \Then [v] \Else [w_j] \Big) = 0 }
	\Assume{w}{\prod j \in n \setminus \{i\} \. V_j}
	\Say{[w.1]}{[1][w]}{\tilde T\Big(j \mapsto \If j == i \Then [v] \Else [w_j] \Big) = 0 }
	\Conclude{[w.*]}{\bd \tilde T [w.1]}{ T\Big(j \mapsto \If j == i \Then v \Else w_j \Big) = 0 }
	\DeriveConclude{[v.*]}{\bd^{-1} N_i(T)}{v \in N_i(T)}
	\DeriveConclude{[*]}{\bd \TYPE{QuotientModule}}{N_i(\tilde T) = \{0\} }
	\EndProof
	\\
	\Theorem{NullSpaceInclusion}{\forall T \in \L(V;W) \. \forall A : W \Arrow{\LMOD{R}} M \.  \forall i \in n \. N_i(T) \subset N_i(AT)}
	\Assume{v}{N_i(T)}
	\Say{[1]}{\bd N_i(\tilde T)}{\forall w : \prod j \in n \setminus \{i\} \. V_j \. 
		\tilde T\Big(j \mapsto \If j == i \Then v \Else w_j \Big) = 0 }
	\Assume{w}{\prod j \in n \setminus \{i\} \. V_j}
	\Say{[2]}{[1](w)}{T\Big(j \mapsto \If j == i \Then v \Else w_j \Big) = 0 }
	\Conclude{[w.*]}{\bd \LMOD{R}(W,M)(T)[2]}{ AT\Big(j \mapsto \If j == i \Then v \Else w_j \Big) = 0    }
	\DeriveConclude{[v.*]}{\bd^{-1} N_i}{v \in N_i(AT)}
	\DeriveConclude{[*]}{\bd^{-1}\TYPE{Subset}}{N_i \subset N_i(AT)}
	\EndProof
}
\Page{
        \\
	\DeclareType{Alternating}{\prod R \in \ANN \. \prod V,W \in \LMOD{R} \. \prod n \in \Nat \. ?\L( \Lambda i \in n \. V; W  )}
	\DefineType{T}{Alternating}{ \forall v \in V^n \. \forall i \in (n-1) \. v_i = v_{i+1} \Rightarrow  T(v) = 0}
	\\
	\Theorem{StrongAlternatingProperty}{\forall R \in \ANN \. \forall V,W \in \LMOD{R} \. \forall n \in \Nat \. \NewLine 
		\forall T : \TYPE{Alternating}(V,W,n) \. \forall i,j \in n \. \forall v \in V^n \. \forall [0] : i < j \.
		\forall [00] : v_i = v_j \. T(v) = 0
	}
	\Say{\mercury}{\Lambda k \in \Nat \. \forall i,j \in n \. \forall v \in V^n \. i < j \And v_i = v_j \Rightarrow T(v) = 0}{\Nat \to \Type}
	\Assume{i,j}{n}
	\Assume{v}{V^n}
	\Assume{[1]}{j - i = 1}
	\Assume{[2]}{v_i = v_j}
	\Conclude{[1.*]}{\bd \TYPE{Alternating}(T)[1][2]}{T(v) = 0}
	\Derive{[1]}{\ByConstr \mercury}{\mercury(1)}
	\Assume{k}{n-2}
	\Assume{[2]}{\mercury(k)}
	\Assume{i,j}{n}
	\Assume{v}{V^n}
	\Assume{[3]}{j-i = k+1}
	\Assume{[4]}{v_i = v_j}
	\\
	\Conclude{[k.*]}{\bd^k \TYPE{Alternating}(T) \THM{multiAdditive}^{k+1}(T)\bd \TYPE{Alternating}(T) \THM{MultAdditive}(T) 
		\bd \TYPE{Alternating}(T) \NewLine [2](i,j-1,\ldots)}
	{
		T(v) = 
		T(v) + \sum^k_{L=1} T\left( \Lambda m \in n \. \If i \le m \le i + L  \Then \sum_{l=i}^{\min(m,i+L-1)} v_l \Else v_m  \right) = \NewLine =  
		T\left( \Lambda m \in n \. \If i \le m \le i + k \Then \sum_{l=i}^m v_l \Else v_m \right) = \NewLine =
		T\left( \Lambda m \in n \. \If i \le m \le j \Then \sum_{l=i}^{\min(m,j-1)} v_l \Else v_m \right) - \NewLine
		-T\left( \Lambda m \in n \. \If i \le m < j \Then \sum_{l=i}^m v_l \Else \If m = j \Then \sum_{l=i+1}^{j-1} v_l \Else v_m \right) =
		\NewLine =
		-T\left( \Lambda m \in n \. \If i \le m < j \Then \sum_{l=i}^m v_l \Else \If m = j \Then \sum_{l=i+1}^{j-1} v_l \Else v_m \right) =
		\NewLine =
		-T\left( \Lambda m \in n \. \If i \le m < j-1 \Then \sum_{l=i}^m v_l \Else \If j-1 \le m \le j \Then \sum_{l=i+1}^{j-1} v_l \Else v_m \right) -
		\NewLine -
		T\left( \Lambda m \in n \. \If i \le m \le i+k \Then \sum_{l=i}^{\If m < i+k \Then m \Else i}
		v_l \Else \If m = j-1 \Then  \sum_{l=i+1}^{j-1} v_l \Else v_m \right) =  \NewLine = 0 
	}
	\DeriveConclude{[*]}{\bd \TYPE{InduciveSet}(\Nat)\ByConstr \mercury}{\LOGIC{This}}
	\EndProof
}\Page{
	\Theorem{LinearlyIndeprndentByAlternating}{
			\forall k : \Field \. 
			\forall V,W \in \VS{k} \.
			\forall n \in \Nat \. \NewLine \. 
			\forall T : \TYPE{Alternating}(V,W,n) \.
			\forall v  : V^n \.
			\forall [0] : T(v) \neq 0 \.
			v : \LI(n,V) 
	}
	\Assume{[1]}{v \IsNot \LI(n,V)}
	\Say{(i,\alpha,[2])}{\bd \LI(n,V)[1]}{\sum i \in n \.  \sum \alpha \in R^{n\setminus\{i\}} : \alpha v_{|n\setminus\{i\}} = v_i }
	\Say{[3]}{ [2]\THM{MultiAdditive}^{n-1}(T)\THM{MultiHomogen}^{n-1}(T)\THM{StrongAlternatingProperty}^{n-1}(T,\ldots)  }
	{
		\NewLine : T(v) = \sum_{j \in n\setminus\{i\}} \alpha_jT(\Lambda k \in n \. \If k=i \Then v_j \Else v_k) = 0
	}
	\Conclude{[1.*]}{[0][3]}{\bot}
	\DeriveConclude{[*]}{E(\bot)}{\Big(v : \LI(n,V)\Big)}
	\EndProof
	\\
	\DeclareType{Symmetric}{\prod R \in \ANN \. \prod V,W \in \LMOD{R} \. \prod n \in \Nat \. ?\L( \Lambda i \in n \. V; W  )}
	\DefineType{T}{Symmetric}{\forall \sigma \in S_n \. \forall v \in V^n \. T(\sigma^*v) = T(v)}
	\\
	\DeclareType{Antisymmetric}{\prod R \in \ANN \. \prod V,W \in \LMOD{R} \. \prod n \in \Nat \. ?\L( \Lambda i \in n \. V; W  )}
	\DefineType{T}{Antisymmetric}{\forall \sigma \in S_n \. \forall v \in V^n \. T(\sigma^*v) = (-1)^{\sigma}T(v)}
	\\
	\Theorem{SymmetricNForm}{\forall R \in \ANN \. \forall V \in \LMOD{R} \. \forall n \in \Nat \. \forall F : \TYPE{NForm}(V,n) \. \NewLine 
		\forall [0] : n! \in R^* \. \exists S : \TYPE{Symmetric}(V,R,n) : \forall v \in V  \. F(v) = S(i \mapsto v)
	}
	\Say{\Big(T,[1]\Big)}{\bd \TYPE{NForm}(V,n)(F) }{\sum T : \L(\Lambda i \in n \. V ; R) \. \forall v \in V \. F(v) = T(i \mapsto v)}
	\Say{S}{\frac{1}{n!}\sum_{\sigma \in S_n} \sigma^{**} T}{\TYPE{Symmetric}(V,R,n)}
	\Assume{v}{V}
	\Conclude{[v.*]}{ [1][0]\THM{NumberOfPermutations}(n)\bd^{-1}\sigma^{**} \ByConstr^{-1}(S)  }{ 
		\NewLine : 
		F(v) = T(i \mapsto v) =  
		\frac{1}{n!}\sum_{\sigma \in S_n} T(i \mapsto v) = 
		\frac{1}{n!}\sum_{\sigma \in S_n} \sigma^{**}T(i \mapsto v) = 
		S(i \mapsto v) }
	\DeriveConclude{[*]}{I(\forall)}{\LOGIC{This}}
	\EndProof
}\Page{
	\Theorem{MultilinearNullByIdealStructure}
	{
		\forall R \in \ANN \. 
		\forall n \in \Nat \.
		\forall I : n \to \Ideal(R) \.
		\forall [0] : R = \sum^n_{i=1} I_i \.
		\NewLine \.
		\forall W \in \LMOD{R} \.
		\L\left( \frac{R}{I};W\right) = \{0\} 
	}
	\Assume{T}{\L\left(\frac{R}{I};W\right)}
	\Assume{[\alpha]}{\prod^n_{i=1}\frac{R}{I_i}}
	\Say{\Big(\beta,[1]\Big)}{[0]\alpha}{\sum \beta : n \to \prod^n_{i=1} I_i \. \forall i \in n \. \alpha_i = \sum^n_{j=1} \beta_{i,j} }
	\Say{[2]}{\THM{MultiAdditive}(T)[1]}{T[\alpha] = \sum_{j : n \to n} T[\beta_j] }
	\Assume{j}{n \to n}
	\Assume{[3]}{T[\beta_j] \neq 0}
	\Say{\beta'}{\Lambda i \in n \. \If i = 1 \Then [1] \Else \If i = j_1 \Then [\beta_1 \beta_{j_i}] \Else [\beta_{i,j_i}]}{\prod^n_{i=1}\frac{R}{I_i}}
	\Conclude{[j.*]}{\THM{MultiHomogen}^2(T,\beta,\beta') \ByConstr \beta' \bd \TYPE{Ideal}(I_{j_1})\bd 
		\FUNC{quotientRing}\ByConstr \beta'\THM{MultiHomogen}}{ \NewLine : T[\beta_j] = T[\beta'] = 0}
	\Derive{[3]}{I(\forall)}{\forall j : n \to n \. T[\beta_j]=0}
	\Conclude{\big[[\alpha].*\big]}{[3][2]}{T[\alpha]=0}
	\DeriveConclude{[T.*]}{I(=,\to)}{T= 0}
	\DeriveConclude{[*]}{\bd^{-1}\TYPE{Subset}\bd^{-1} \{0\}}{\L\left(\frac{R}{I};W\right) = \{0\}}
	\EndProof
} 
\newpage
\subsection{The Tensor Product}
\Page{
	\DeclareType{TensorProduct}{\prod R \in \ANN \. \prod n \in \Nat \. \prod V : n \to \LMOD{R} \. ? \sum W \in \LMOD{R} \. \L(V;W) }
	\DefineType{\left(W ,\bigotimes \right)}{TensorProduct}{ \forall M \in \LMOD{R} \. \forall T : \L(V;M) \. \exists! A : W \Arrow{\LMOD{R}} M \. A\bigotimes = T}
	\\
	\Theorem{TensorProductExists}{\forall R \in \ANN \. \forall n \in \Nat \. \forall V : n \to \LMOD{R} \. 
		\exists \left(W,\bigotimes\right) : \TYPE{TensorProduct}(R,n,V)}
	\Say{F}{\FUNC{FreeModule}\Act{R,\prod^n_{i=1} V_i}}{\LMOD{R}}
	\Say{f}{\ByConstr F \bd \FUNC{FreeModulr}}{ \Basis\Act{\prod^n_{i=1} V_i} }
	\Say{U}{\Span \{  f(v) - f(v') - f(v'') | \ldots  \}\cup\{ f(v) - \alpha f(v') \}}{\TYPE{Submodule}(F)}
	\Say{W}{\frac{F}{U}}{\LMOD{R}}
	\Say{T}{\pi_U \circ f}{ \prod^n_{i=1} V_i  \to W }
	\Say{\venus}{\Lambda k \in n-1 \. \forall v \in \prod^{n-k}_{i=1} V_i \. T(v) \in \L(\Lambda i \in n-k \. V_i) }{\Nat \to \Type}
	\Assume{v}{\prod^{n-1}_{i=1} V_i}
	\Conclude{[v.*]}{\ByConstr U \ByConstr T(v) \bd^{-1} \L(V_n;W)}{ T(v) \in \L(V_n;W)  }
	\Derive{[1]}{\ByConstr \venus}{ \venus(1) } 
	\Assume{k}{n-2}
	\Assume{[2]}{\venus(k)}
	\Assume{v}{\prod_{i=1}^{n-k-1} V_i}
	\Conclude{[k.*]}{\ByConstr U \ByConstr T(v) \bd^{-1} \L\left(\prod^n_{n-k-1}V;W\right)}{ T(v) \in \L(V_n;W)  }
	\Derive{[2]}{\bd^{-1} \L(V;W)\bd \TYPE{InductiveSet}(n-1)\ByConstr U \ByConstr T}{T \in \L(V;W)}
	\Assume{M}{\LMOD{R}}
	\Assume{S}{\L(V;M)}
	\Say{\Big(A',[3]\Big)}{\bd \TYPE{Adjoint}(\FUNC{FreeModule})(S)}{\sum A :  F \Arrow{\LMOD{R}} M \. A' \circ f = S}
	\Say{[4]}{\bd \L(V;M)[3]}{  U \subset \ker A  }
	\Say{ (A,[5])  }{\THM{MorphismRestriction}[4]}{ \sum A  : W \Arrow{\LMOD{R}} M \. A \circ T = S   }
	\Assume{B}{\L(V;M)}
	\Assume{[6]}{B \circ T = S}
	\Say{[7]}{\THM{GenSurjection}(f,\pi_U)\ByConstr T }{ W = \Span \im T}
	\Conclude{[M.*] }{[7][6][5]\bd \FUNC{span}}{A = B}
	\DeriveConclude{[*]}{\bd^{-1} \TYPE{TensorProduct}}{\Big( (W,T) : \TYPE{TensorProduct}(R,n,V) \Big) }
	\EndProof
}\Page{
	\Theorem{TensorProductsAreEq}{\forall R \in \ANN \. \forall n \in \Nat \. \forall V : n \to \LMOD{R} \. 
		\forall (W,T),(U,S) : \TYPE{TensorProduct}(R,n,V) \. W \cong_{\LMOD{R}} U}
	\Say{\big(A,[1]\big)}{\bd \TYPE{TensorProduct}(R,n,V)(W,T)(S)}{\sum A : W \Arrow{\LMOD{R}} U \. S = A \circ T}
	\Say{\big(B,[2]\big)}{\bd \TYPE{TensorProduct}(R,n,V)(U,S)(T)}{\sum A : U \Arrow{\LMOD{R}} W \. T = B \circ S}
	\Say{[3]}{[1][2]}{S = A \circ B \circ S}
	\Say{[4]}{[2][1]}{T = B \circ A \circ T}
	\Say{[5]}{\bd \TYPE{TensorProduct}(R,n,V)[3]}{A \circ B = \id}
	\Say{[6]}{\bd \TYPE{TensorProduct}(R,n,V)[4]}{B \circ A = \id}
	\Conclude{[*]}{[5][6]\bd^{-1}\TYPE{Isomorphic}}{ W \cong_{\LMOD{R}} U  }
	\EndProof
	\\
	\DeclareFunc{tensorProduct}{\prod R \in \ANN \. \prod n \in \Nat \. \prod V : n \to \LMOD{R} \. \TYPE{TensorProduct}(R,n,V) }
	\DefineNamedFunc{tensorProduct}{}{\left(\bigotimes_{i=1}^n V_i,\bigotimes\right)}{\THM{TensorProductExists}(R,n,V)}
	\\
	\DeclareFunc{tensorisation}{\prod R \in \ANN \. \prod n \in \Nat \. \prod V : n \to \LMOD{R} \. \prod W \in \LMOD{R} \. 
		\NewLine \. L(V;W) \to \bigotimes^n_{i=1} V_i \Arrow{\LMOD{R}}  W   
	}
	\DefineNamedFunc{tensorisation}{T}{T^\otimes}{\bd \TYPE{TensorProduct}(R,n,V)\left( \bigotimes^n_{i=1} V_i, \bigotimes \right)(T)}
	\\
	\Theorem{TensorProductOfFreeModules}
	{
		\forall R \in \ANN \.
		\forall n \in \Nat \. 
		\forall V : n \to \FM(R) \. 
		\bigotimes^n_{i=1} V_i : \FM(R)
	}
	\Say{e}{\THM{FreeHasBasis}(V)}{\prod i \in n \. \Basis(V_i)}
	\Say{f}{\Lambda j : \prod i \in n \. \rank V_i \. \bigoplus^n_{i=1} e_{i,j_i}}{\left(\prod i \in n \. \rank V_i \right) \to \bigoplus^n_{i=1} V_i}
	\Assume{v}{\prod^n_{i=1} V_i}
	\Say{(\alpha,[1])}{\bd \Basis(v)}{ \sum \alpha : \left(\sum^n_{i=1} \rank V_i\right) \to R \.  \forall i \in n \. v_i = \alpha_i e_i }
	\Conclude{[*]}{[1]\bd \L(V;W)\bigoplus \ByConstr f}{ \bigoplus^n_{i=1} v_i = \sum_{j \in \prod^n_{i=1}\rank V_i} \prod^n_{i=1} \alpha_{i,j_i} f_j  }
	\Derive{[1]}{\bd \bigoplus^n_{i=1} V_i \bd^{-1}\Span}{\bigoplus^n_{i=1} V_i = \Span(f)}
	\Assume{[2]}{f \IsNot \LI}
	\Say{T}{\delta_{e;\prod e}}{\L\left(V;R^{\oplus\prod^n_{i=1}\rank V_i}\right)}
	\Say{(\alpha,[3])}{\bd \LI[2]}{ \sum \alpha \in R^{\oplus\prod^n_{i=1}\rank V_i} \. \alpha f = 0 \And \alpha \neq 0 }
	\Say{[4]}{ \bd \LMOD{R}(T^\otimes)[3]\bd\LMOD{R}(T^\otimes) \ByConstr(T) }{0 = T^\otimes(0) =  T^{\otimes}(\alpha f) = \alpha T^{\otimes}(f) = \alpha}
	\Conclude{[*]}{[3][4]}{ \bot }
}\Page{
	\Derive{[2]}{\bd^{-1}\Basis(\bot)}{f : \Basis\Act{ \prod^n_{i=1} \rank V_i, \bigoplus^n_{i=1} V_i }}
	\Conclude{[*]}{\THM{FreeByBasis}[2]}{\Act{\bigoplus^n_{i=1} V_i : \FM(R)}}
	\EndProof
	\\
	\Theorem{RankOfTensorProduct}{\forall R \in \ANN \. \forall n \in \Nat \. \forall V : n \to \FM(R) \. \rank \bigotimes^n_{i=1}V = \prod^n_{i=1} \rank V_i}
	\NoProof
	\\
	\Theorem{TensorProductTensorProduct}{
		\forall R \in \ANN \. 
		\forall n \in \Nat \. 
		\forall V : n \to \LMOD{R} \. 
		\forall k \in n-1 \. \NewLine \. 
		\bigotimes^k_{i=1} V_i  \otimes \bigotimes^n_{i=k+1}  V_i \cong_{\LMOD{R}} \bigotimes^n_{i=1} V_i  
		}
	\Say{T}{\Lambda  v : \prod^n_{i=1} V_i \. \left(\bigotimes^k_{i=1} v_i\right) \otimes \left( \bigotimes^n_{i=k+1} v_i \right) }
	{  \L\left( V ; \bigotimes^k_{i=1} V_i \otimes \bigotimes^n_{i=k+1} V_i  \right)}
	\Say{S}{
		\Lambda \left( \sum_{j=1} \bigotimes^k_{i=1} v_{i}^j,\sum_{j=1}
		\bigotimes^n_{i=k+1} w_i^j\right) :\bigotimes^k_{i=1} V_i  \times \bigotimes^n_{i=k+1} V_i  \.
		\sum_{j=1}\sum_{l=1}\bigotimes^n_{i=1}  \If i \le k \. v_i^j \Else w_i^l	
	}{ \NewLine : \L\left( \bigotimes^k_{i=1} V_i, \bigotimes^n_{i=k+1} V_i ; \bigotimes^n_{i=1} V_i \right) }
	\Say{[1]}{\ByConstr T \ByConstr S}{T^\oplus S^\oplus = \id \And S^\oplus S^\oplus = \id}
	\Conclude{[*]}{\bd \TYPE{Isomorphic}[1]}{\bigotimes^k_{i=1} V_i  \otimes \bigotimes^n_{i=k+1}  V_i \cong_{\LMOD{R}} \bigotimes^n_{i=1} V_i } 
	\EndProof
	\\
	\Theorem{AssociativeTensorProduct}{\forall R \in \ANN \. \forall A,B,C \in \LMOD{R} \. (A \otimes B) \otimes C \cong_{\LMOD{R}} A \otimes (B \otimes C)   }
	\Say{[1]}{\THM{TensorProductTensorProduct}(A,B,C,1)}{A \otimes (B \otimes C ) \cong A \otimes B \otimes C}
	\Say{[2]}{\THM{TensorProdctTensorProduct}(A,B,C,2)}{(A \otimes B) \otimes C \cong A \otimes B \otimes C}
	\Conclude{[3]}{\bd \TYPE{Transitive}(\TYPE{Isomorphic})[1][2]}{A \otimes  (B \otimes C) \cong_{\LMOD{R}}(A \otimes B)  \otimes A }
	\EndProof
}\Page{
	\Theorem{TensorProductPermutation}{
		\forall R \in \ANN \. 
		\forall n \in \Nat \. 
		\forall n \to  \LMOD{R} \. 
		\forall \sigma \in S_n \.  
		\bigoplus^n_{i=1} V_i \cong_{\LMOD{R}} \bigoplus^n_{i=1} V_{\sigma(i)} 
	}
	\Say{T}{\Lambda v \in \prod^n_{i=1} V_i  \. \bigotimes^n_{i=1} v_{\sigma(i)} }{ \L\left(V; \bigotimes^n_{i=1} V_i \right)  }
	\Say{S}{\Lambda v \in \prod^n_{i=1} V_{\sigma{i}} \. \bigotimes^n_{i=1} v_{\sigma^{-1} i}  }{\L\left(\sigma^* V, \bigotimes^n_{i=1} V_i \right)}
	\Say{[1]}{\ByConstr T \ByConstr S}{T^\otimes S^\otimes = \id \And S^\otimes T^\otimes = \id}
	\Conclude{[*]}{\bd \TYPE{Isomorphic}[1]}{\bigotimes^n_{i=1} V_i   \cong_{\LMOD{R}} \bigotimes^n_{i=1} V_{\sigma(i)} } 
	\EndProof
	\\
	\Theorem{TensorProductIdealQuotient}{
		\forall R \in \ANN \. 
		\forall I : \Ideal(R) \.
		\forall n \in \Nat \.
		\forall V : n \to \LMOD{R} \. \NewLine \. 
		\frac{{\bigotimes^n_{i=1}}_R V_i}{I{\bigotimes^n_{i=1}}_R V_i} \cong_{\LMOD{R}}
		{\bigotimes^n_{i=1}}_{\frac{R}{I}} \frac{V_i}{IV_i}
	}
	\Say{T}{\Lambda v \in \prod^n_{i=1} V_i \. \bigotimes^n_{i=1} [v_i] }{\L\left( V; \bigotimes^n_{i=1} \frac{V_i}{RV_i} \right)}
	\Say{[1]}{\bd^{-1} \ker \bd \FUNC{moduleQuotien}(\ldots) \THM{MultiHomogen}(\ldots)}{ I\bigotimes^n_{i=1} V_i \subset \ker T^\otimes   }
	\Say{\hat T^\otimes}{\THM{KerRestriction}[1]}{
		\sum \hat T^\otimes :\frac{{\bigotimes^n_{i=1}}_R V_i}{I{\bigotimes^n_{i=1}}_R V_i} 
		\Arrow{\LMOD{R}} 
		\bigotimes^n_{i=1} \frac{V_i}{RV_i} \.   T^\otimes = \hat T^\otimes
	}
	\Say{S}{\Lambda [v] \in \prod^n_{i=1}\frac{V_i}{SV_i} \. \left[ \bigoplus^n_{i=1} \right]  }
	{\L\left( \frac{V}{IV} ; \frac{\bigoplus^n_{i=1} V_i}{I\bigoplus^n_{i=1} V_i}  \right)}
	\Assume{w}{\prod^n_{i=1}IV_i}
	\Say{u}{\Lambda L : n \to \{0,1\} \. \Lambda i \in n \. \If L_i = 0 \Then v \Else w }{(n \to \{1,0\}) \to \prod^n_{i=1}V_i}
	\Conclude{[w.*]}{ \THM{MultiAdditive}(S)\bd\FUNC{quotientModule}\ByConstr u   }
	{ S\big([v + w]\big) = \sum_{L : n \to \{0,1\}} S[u] = = S[v] + \sum_{L\neq 0}S[u] = 0  }
	\Derive{[2]}{\bd \FUNC{quotientModule}}{S : \TYPE{WellDefined}}
	\Say{[1]}{\ByConstr T \ByConstr S}{T^\otimes S^\otimes = \id \And S^\otimes T^\otimes = \id}
	\Conclude{[*]}{\bd \TYPE{Isomorphic}[1]}{\frac{\bigotimes^n_{i=1} V_i}{ \bigotimes^n_{i=1}  V_i} \cong_{\LMOD{R}} \bigotimes^n_{i=1} \frac{V_i}{IV_i} } 
	\EndProof
}\Page{
	\Theorem{TrivialTensorProduct}{
		\forall R \in \ANN \.
		\forall V \in \LMOD{R} \.
		R \otimes V \cong_{\LMOD{R}} V
	}
	\Say{A}{\Lambda^\otimes (\alpha,v) \in R \times V \. \alpha v}{ R \otimes V  \Arrow{\LMOD{R}} V  }
	\Say{B}{\Lambda v \in V \. 1 \otimes v}{ V \Arrow{\LMOD{R}}R \otimes V    }
	\Assume{v}{V}
	\Conclude{[v.1]}{\ByConstr Bv \ByConstr A}{   ABv = A(1 \otimes v) = v } 
	\Derive{[1]}{I(=,\to)}{AB = \id}
	\Assume{\alpha_i \otimes v_i }{R \otimes V}
	\Conclude{[.*]}{\ByConstr A \ByConstr B  \THM{MultiHomogen}(\otimes) }
	{ BA(\alpha_i \otimes v_i) = B(\alpha_i v_i) = 1 \otimes \alpha_i v_i = \alpha_i \otimes v_i  }
	\Derive{[2]}{I(=,\to)}{BA = \id}
	\Conclude{[*]}{\bd^{-1}\TYPE{Isomorphic}[1][2]}{ R \otimes V \cong_{\LMOD{R}} V }
	\EndProof
	\\
	\Theorem{QuotientByTensorProduct}
	{\forall R \in \RING \. \forall I : \Ideal(R) \. \forall V \in \LMOD{R} \. \frac{R}{I} \otimes V \cong_{\LMOD{R}} \frac{V}{IV}}
	\Say{A}{\Lambda \big([\alpha] , v\big) \in \frac{R}{I} \times V \. [\alpha][v]}{\L\left( \frac{R}{I}\otimes V ; \frac{V}{IV}\right)}
	\Say{B}{\Lambda  [v] \in \frac{V}{IV} \. [1] \otimes v }{ \frac{V}{IV} \Arrow{\LMOD{R}} \frac{R}{I} \otimes V}
	\Assume{w}{IV}
	\Say{(n,\alpha,u,[1])}{}{\sum n \in \Nat \.  \alpha : n \to I \. \sum u : n \to  V : w = \sum^n_{i=1} \alpha_iv_i }
	\Conclude{[w.*]}
	{\ByConstr B \THM{MultiAdditive}^{n+1}(\otimes) \THM{MultiHomogen}^n(\otimes) \bd \FUNC{quotientRing} \THM{MultiHomogen}^n(\otimes) 
		\ByConstr^{-1}B }
	{
		\NewLine :
		B[v + w] =  
		[1] \otimes \left(v + \sum^n_{i=1} \alpha_i u_i \right) = 
		[1] \otimes v + \sum^n_{i=1} \alpha_i [1] \otimes u_i = 
		[1] \otimes v = 
		B[v]
	}
	\Derive{[1]}{\bd \FUNC{QuotientModule}}{(B: \LOGIC{WellDefined})}
	\Assume{[v]}{\frac{V}{VI}}
	\Conclude{[v.*]}{ \ByConstr B \ByConstr A}{  A^\otimes B[v] = A^{\otimes}([1]\otimes v) = [v]}
	\Derive{[2]}{I(=,\to)}{A^\otimes B = \id}
	\Assume{[\alpha]\otimes v }{ \frac{R}{I} \otimes V}
	\Conclude{[.*]}{\ByConstr A \THM{MultiHomogen}^2(\otimes) \bd \FUNC{quotientRing}(R,I)}
	{BA^\otimes([\alpha]\otimes v) = B[\alpha v] = [1] \otimes \alpha v = [\alpha] \otimes v}
	\Derive{[3]}{I(=,\to)}{B A^\otimes = \id}
	\Conclude{[*]}{\bd^{-1}\TYPE{Isomorphic}[2][3]}{ \frac{R}{I} \otimes V \cong_{\LMOD{R}} \frac{V}{IV}  }
	\EndProof
}\Page{
	\Theorem{NakayamaTensorCondition}{
			\forall R \in \ANN \. 
			\forall V \in \LMOD{R} \. \NewLine \.
			\Big( \forall N \in \FGM(R) \.  N = \{0\} \iff N \otimes V = \{0\} \Big) \iff \NewLine \iff 
			\forall I : \TYPE{MaximalIdeal}(R) \. IV \neq V
		}
	\Assume{[1]}{\forall N \in \FGM(R) \. N = \{0\} \iff N \otimes V = \{0\}}
	\Assume{I}{\TYPE{MaximalIdeal}(R)}
	\Say{[2]}{\bd \TYPE{MaximalIdeal}(I)}{\frac{R}{I} \neq \{0\}}
	\Say{[3]}{ \THM{QuotientByTensorProduct}(V,I)[1][2]}{  \frac{V}{IV} \cong_{\LMOD{R}} V \otimes  \frac{R}{I} \neq \{0\}}
	\Conclude{[*]}{ \bd \FUNC{quotientModule}[3] }{ V \neq IV }
	\Derive{[1]}{I(\Rightarrow)I(\forall)}{\LOGIC{Left} \Rightarrow \LOGIC{Right}}
	\Assume{[2]}{ \forall I : \TYPE{MaximalIdeal}(R) \. IV \neq V  }
	\Assume{N}{\FGM(R)}
	\Assume{[3]}{N \otimes V = \{0\}}
	\Assume{I}{\TYPE{maximalIdeal}(R)}
	\Say{[4]}{\bd \FUNC{quotientModule}[3]\THM{TensotProductIdealQuotient}(V,N;I)} 
	{ \NewLine : \{0\} = \frac{N \otimes_R V}{ I(N \otimes_R V)} \cong_{\LMOD{R}} \frac{N}{IN} \otimes_{\frac{R}{I}} \frac{V}{IV}   }
	\Say{[5]}{ [2](I) }{\frac{V}{IV} \neq 0}
	\Say{[6]}{\THM{MaximallQuotientIsField}(I)}{(\frac{R}{I} : \TYPE{Field})}
	\Say{[7]}{[5][6]}{ \frac{N}{IN} = \{0\} }
	\Conclude{[I.*]}{ \bd \FUNC{quotientModule}[7] }{ N = IN}
	\Derive{[4]}{I(\forall)}{\forall I : \TYPE{MaximalIdeal}(R) \. IN = N}
	\Conclude{[2.*]}{\THM{NakayamaLemma}[4]}{N = \{0\}}
	\Derive{[*]}{I(\Rightarrow)\bd \FUNC{tensrProduct}I(\iff)I(\Rightarrow)I(\iff)[1]}{\LOGIC{This}}
	\EndProof
	\\
	\DeclareFunc{tensorPower}{\prod R \in \ANN \. \prod n \in \Nat \. \LMOD{R} \to \LMOD{R}}
	\DefineNamedFunc{tensorPower}{V}{\mathbf{T}^n(V)}{\bigoplus^n_{i=1} V}
	\\
	\Theorem{ZeroTensorInFGM}{
		\forall R \in \ANN \. \forall A,B \in \LMOD{R} \.
		\forall t \in A \otimes B \. 
		\forall [0] : t =_{A \otimes B} 0 \. \NewLine : 
		\exists A' : \FGM(R) \And \TYPE{Submodule}(R,A) \. \NewLine : 
		\exists B' : \FGM(R) \And \TYPE{Submodule}(R,B) \.
		t =_{A' \otimes B'} 0 
	}
	\NoProof
} 
\newpage
\subsection{Tensor Product as Functor}
\Page{
	\DeclareFunc{tensorMap}{
			\prod R \in \ANN \. 
			\prod n \in \Nat \. 
			\prod V,W : n \to \LMOD{R} \. \NewLine \. 
			\left( \prod^n_{i=1} V_i \Arrow{\LMOD{R}} W_i\right) \to 
			\bigotimes^n_{i=1} V_i \Arrow{\LMOD{R}}  \bigotimes^n_{i=1} W_i
		}
	\DefineNamedFunc{tensorMap}{f}{\bigotimes^n_{i=1} f_i}{\FUNC{tensorisation} \; \Lambda v \in \prod^n_{i=1} V_i \. \bigotimes^n_{i=1} f_i(v_i)}
	\\
	\Theorem{TensorMapComposition}{
		\prod R \in \ANN \. 
		\prod n \in \Nat \.
		\prod V,W,U : n \to \LMOD{R} \. \NewLine \.
		\forall f : \prod^n_{i=1} V_i \Arrow{\LMOD{R}} W_i \.
		\forall g : \prod^n_{i=1} W_i \Arrow{\LMOD{R}} U_i \.
		\bigotimes^n_{i=1} g_i \circ \bigotimes^n_{i=1} f_i =
		\bigotimes^n_{i=1}  g_i \circ f_i
	}
	\Assume{v}{\prod^n_i V_i}
	\Conclude{[v.*]}{\bd^2 \FUNC{tensorMap}^{-1}(f,g) \bd^{-1} \FUNC{compose}}
	{   	\NewLine :
		\bigotimes^n_{i=1} g_i \circ \bigotimes^n_{i=1} f_i \bigotimes^n_{i=1} v_i = 
		\bigotimes^n_{i=1} g_i \bigotimes^n_{i=1} f_i(v_i) =
		\bigotimes^n_{i=1} g_i\Big(f_i(v_i)\Big) =
		\bigotimes^n_{i=1} g_i \circ f_i(v_i)
	}
	\DeriveConclude{[*]}{I(\forall)\bd \FUNC{tensorisation}}{\bigotimes^n_{i=1} g_i \circ \bigotimes^n_{i=1} f_i = \bigotimes^n_{i=1} g_i \circ f_i}
	\EndProof
	\\
	\DeclareFunc{tensorFunctor}{\prod R \in \ANN \. \prod n \in \Nat \. \LMOD{R}^n \Arrow{\CAT} \LMOD{R}}
	\DefineNamedFunc{tensorFunctor}{}{\bigotimes^n_{i=1}}{\left( \bigotimes^n_{i=1},\bigotimes^n_{i=1} \right)}
	\\
	\Theorem{TensorMapAdditive}{
		\forall R \in \ANN \. \forall n \in \Nat \. \forall V,U \in \LMOD{R}^n \. 
		\NewLine \.
		\forall f' : \prod^n_{i=1} V_i \Arrow{\LMOD{R}} U_i \. 
		\forall i \in n \. \forall g : V_i \Arrow{\LMOD{R}} U_i \. \bigotimes^n_{i=1} f_i = \bigotimes^n_{i=1} f'_i + \bigotimes^n_{i=1} f''_i
		\NewLine
		\quad \where \quad f = \Lambda j \in n \. \If i == j \Then g + f_i' \Else f_i', f'' = \Lambda j \in n \. \If i == j \Then g \Else f'_i
	}
	\NoProof
	\\
	\Theorem{TensorMapHomogen}{
		\forall R \in \ANN \. \forall n \in \Nat \. \forall V,U \in \LMOD{R}^n \. 
		\NewLine \.
		\forall f : \prod^n_{i=1} V_i \Arrow{\LMOD{R}} U_i \. 
		\forall i \in n \. \forall \alpha \in A \. \bigotimes^n_{i=1} f_i' = \alpha \bigotimes^n_{i=1} f_i 
		\NewLine
		\quad \where \quad f' = \Lambda j \in n \. \If i == j \Then \alpha f_i
	}
	\NoProof
}
\Page{
	\Theorem{ExactTensorMapLemma1}{
		\forall R \in \ANN \.
		\forall n \in \Nat \. 
		\forall (V,f) : \TYPE{RightShortExact}^n(R) \.
		\bigotimes^n_{i=1} f^i_0 : \bigotimes^n_{i=1} V_1^i \ToSurj \bigotimes^n_{i=1} V_0^i
	}
	\Assume{v}{\prod^n_{i=1} V^i_1}
	\Assume{i}{n}
	\Say{[i.1]}{\THM{SurjByExact}\bd \TYPE{RightShortExact}(V^i,f^i)}{(f^i_0 : V^i_1 \ToSurj V^1_0)}
	\Conclude{\Big(w_i,[i.*]\Big)}{\bd^{-1} \TYPE{Surjective}(f^i)(v_i)}{ \sum w_i \in V^i_1 \. f^i_0(w_i) = v_i }
	\Derive{(w,[v.1])}{}{\sum^n_{i=1} w_i \in V^i_1 \. f^i_0(w_i) = v_i}
	\Say{[v.2]}{\bd \FUNC{TensorFunc}(f_0)[v.1]}{\bigotimes^n_{i=1} f_0^i(w_i) = \bigotimes^n_{i=1} v_i}
	\Conclude{[v.*]}{ \bd^{-1} \FUNC{image} [v.2] }{ \bigotimes^n_{i=1} v_i \in \bigotimes^n_{i=1} f_0^i}
	\Derive{[1]}{I(\forall)}{\forall v \in \prod^n_{i=1} V^i_1 \. \bigotimes^n_{i=1} v_i \in \bigotimes^n_{i=1} f_i  }
	\Conclude{[*]}{\bd \FUNC{tensorProduct}[1]\bd^{-1}\TYPE{Surjective}}{\left( f^i_) : \bigotimes^n_{i=1} V_1^i \ToSurj V_0^i  \right)}
	\EndProof
	\\
	\Theorem{ExactTensorMapLemma2}{
		\forall R \in \ANN \.
		\forall n \in \Nat \. 
		\forall (V,f) : \TYPE{RightShortExact}^n(R) \. \NewLine
		\ker \bigotimes^n_{i=1} f^i_0 = \sum^n_{i=1} N_i  \quad \where \quad  N_i = \bigotimes^n_{j=1} \If j == i \Then \im f_1^i \Else V_0^i 
	}
	\Assume{\sum^n_{i=1} t_i}{\sum^n_{i=1} N_i}
	\Assume{i}{n}
	\Say{\Big( K, v, [i.1]  \Big)}{\bd \FUNC{linearSum}\ByConstr(N_i)(t_i)}
	{ 
		\NewLine :
		\sum K \in \Nat \. \sum v : K \to \sum^n_{j=1} \If i == j \Then \im f^i_1 \Else V^i_1 \.  t_i = \sum^K_{k=1} \bigotimes^n_{j=1} v_{k,j}
	}
	\Assume{k}{K}
	\Say{[k.1]}{\bd \TYPE{ChainComplex}(V^i,\varphi^i)(0)}{ \im f^i_1 \subset \ker f_i}
	\Say{[k.2]}{\bd \ker [k.1](v_{j,i})=0}{f^i_0(v_{k,i})= 0}
	\Conclude{[k.*]}{\bd^{-1} \FUNC{tensorMap}\bd \TYPE{TensorProduct}[k.2]}{ \bigotimes^n_{i=1} f_0^i \bigotimes^n_{i=1} v_{k,i} = \bigotimes^n_{i=1} f_0^i(v_{k,i}) = 0}
	\Derive{[i.2]}{I(\forall)}{\forall k \in K \. \bigotimes^n_{i=1} f^i_0 \bigotimes^n_{i=1} v_{k,i} = 0}
	\Conclude{[i.*]}{ \bd \LMOD{R} \left( \bigotimes^n_{i=1} V^i_1, \bigotimes^n_{i=1} V^i_0 \right)\left(\bigotimes^n_{i=1} f^i_0 \right)[i.1][i.2]}
	{\bigotimes^n_{j=1} f^j_0(t_i) = 0}
	\Derive{[t.1]}{ I(\forall)}{\forall i \in n \. \bigotimes^n_{j=1} f^j_0(t_i) = 0}
}\Page{
	\Conclude{[t.*]}{\bd \LMOD{R} \left( \bigotimes^n_{i=1} V^i_1,\bigotimes^n_{i=1} V^i_0 \right)\left(\bigotimes^n_{i=1} f^i_0\right)}
	{  \bigotimes^n_{i=1} f_0^i \sum^n_{i=1} t_i = 0  }
	\Derive{[1]}{\bd^{-1} \TYPE{Subset}\bd^{-1} \ker}{\sum^n_{i=1} N_i \subset \ker \bigotimes^n_{i=1} f_0^i}
	\Assume{i}{n}
	\Assume{v,w}{\prod^n_{i=1} V^i_1}
	\Assume{[i.1]}{f^i_0(v_i) = f^i_0(w_i)}
	\Assume{[i.2]}{\forall j \in n \. j \neq i \Rightarrow v_j = w_j}
	\Say{[i.3]}{\bd^{-1}\ker f^i_0 [w.2]}{ v_i - w_i \in \ker f^i_0 }
	\Say{[i.4]}{\bd \TYPE{RightShortExact}(R)(V^i,f^i) [w.3]}{v_i - w_i \in \im f^i_1}
	\Conclude{[i.*]}{\ByConstr N_i [w.4]}{\bigotimes^n_{i=1} v_i - \bigotimes^n_{i=1} w_i \in N_i}
	\Derive{[2]}{I^4(\forall)}{ \forall i \in n \. \forall v,w \in \prod^n_{i=1} V^i_1 \. 
		\Big( f^i_0(v_i) = f^i_0(w_i) \And \forall j \in n \. j \neq i \Rightarrow \. v_i = w_i \Big) \Rightarrow \NewLine \Rightarrow
		\bigotimes^n_{i=1} v_i - \bigotimes^n_{i=1} w_i \in N_i  
	}
	\Assume{v,w}{\prod^n_{i=1} V^i_1}
	\Assume{[w.1]}{\forall i \in n \. f_0^i(v_i) = f_0^i(w_i) }
	\Say{[w.2]}{\bd \LMOD{R}\bigotimes^n_{i=1} V_1^i}{ \bigotimes^n_{i=1} v_i - \bigotimes^n_{i=1} w_i = \NewLine = \sum^n_{i=0} 
	\left( \bigotimes^n_{j=1} \If j < i \Then w_i \Else v_i - \bigotimes^n_{j=1} \If j \le i \Then w_i \Else v_i \right)  }
	\Conclude{[w.*]}{\prod^n_{i=1}[2](i)[w.2]}{\bigotimes^n_{i=1} v_i - \bigotimes^n_{i=1} w_i \in \sum^n_{i=1} N_i}
	\Derive{[3]}{I(\forall)I(\Rightarrow)}{\forall v,w \in \prod^n_{i=1} V^i_1 \. \Big( \forall i \in n \. f_0^i(v_i) = f_0^i(w_i) \Big) \Rightarrow
		\bigotimes^n_{i=1} v_i - \bigotimes^n_{i=1} w_i \in \sum^n_{i=1} N_i}
	\Assume{v}{\prod^n_{i=1} V^i_0}
	\Say{[v.1]}{\THM{ExactTensorMap1}(v)}{\sum w \in \prod^n_{i=1} V^i_1 \.\bigotimes^n_{i=1} f^i_1 \bigotimes^n_{i=1} w_i =  \bigotimes^n_{i=1} v_i}
	\Say{G(v)}{\left[ \bigotimes^n_{i=1} w_i \right]_{\sum^n_{i=1} N_i}}{\frac{\bigotimes^n_{i=1}V^i_1}{\sum^n_{i=1}N_i}}
	\Assume{u}{\prod^n_{i=1} V^{i}_{1}}
	\Assume{[u.1]}{ \bigotimes^n_{i=1} f^i_1 \bigotimes^n_{i=1} u_i = \bigotimes^n_{i=1}  v_i }
	\Conclude{[u.*]}{[3](w,u)[v.1][v.2]}{ \left[ \bigotimes^n_{i=1}  w_i \right] = \left[ \bigotimes^n_{i=1} u_i \right]   }
	\DeriveConclude{[u.2]}{\ByConstr}{(F : \LOGIC{WellDefined})}
}\Page{
	\Derive{G}{I(\to)}{\L\left( V_0; \frac{\bigotimes^n_{i=1} V_1^i}{\sum^n_{i=1} N_i}\right)}
	\Say{g}{\bigotimes^n_{i=1} f^i_0 G^\otimes}{ \bigotimes^n_{i=1} V^i_1 \Arrow{\LMOD{R}} \frac{\bigotimes^n_{i=1} V^i_1}{\sum^n_{i=1} N_i} }
	\Assume{t}{\ker \bigotimes^n_{i=1} f_0^i}
	\Say{[t.1]}{\bd g(t)}{t \in \ker g}
	\Say{[t.2]}{\ByConstr g (t)}{g(t) = [t]}
	\Conclude{[t.*]}{[t.1][t.2]}{ t \in \sum^n_{i=1} N_i}
	\DeriveConclude{[*]}{\bd^{-1}\TYPE{SetEq}[1]\bd^{-1}\TYPE{Subset}}{\ker \bigotimes^n_{i=1}f_0^i = \sum^n_{i=1} N_i}
	\EndProof
	\\
	\DeclareFunc{tensorWith}{\prod R \in \ANN \. \LMOD{R} \to \LMOD{R} \Arrow{\CAT} \LMOD{R}}
	\DefineNamedFunc{tensorWith}{M}{T_M}{\Big(\cdot \otimes M, \cdot \otimes {\id}_M\Big)} 
	\\
	\Theorem{ExactTensorTHM}{\forall R \in \ANN \. \forall M \in \LMOD{R} \. T_M : \TYPE{RightExact}\Big(\LMOD{R},\LMOD{R}\Big)}
	\Assume{A \Arrow{f} B \Arrow{g} C \to 0}{\TYPE{RightShortExact}(\LMOD{R})}
	\Say{[1]}{\bd^{-1}\TYPE{RightShortExact}}{\left( 0 \to M \Arrow{{\id}_M} M  \to 0 : \right) }
	\Say{[2]}{\THM{ExactTensorLemma}}{ \ker g \otimes {{\id}_M}  = \im f \otimes M +  B \otimes 0 = \im f \otimes M  }
	\Conclude{[A.*]}{\bd^{-1} \FUNC{image}(\id)\bd^{-1}\TYPE{RightShortExact}}{  A \otimes M \Arrow{f \otimes \id} B \otimes M \Arrow{g \otimes \id} C \to 0 : 
		\NewLine : \TYPE{RightExact}(\LMOD{R},\LMOD{R})}
	\DeriveConclude{[*]}{\bd^{-1} \TYPE{RightExact}}{ \Big(T_M : \TYPE{RightExact}\big(\LMOD{R},\LMOD{R}\big) \Big)    } \EndProof
	\\
	\Theorem{TensorProductDistributive}{\forall R \in \ANN \. \forall A,B,M \in \LMOD{R} \. M \otimes (A \oplus B) = (M \otimes A) \oplus (M \otimes B) }
	\NoProof
	\\
	\Theorem{FreeTensoringDecomposition}{
			\forall R \in \ANN \. 
			\forall F : \FM(R) \. 
			\forall M \in \LMOD{R} \. 
			\forall E : \TYPE{Basis}(F) \. \NewLine \. 
			\forall t \in F \otimes M \. 
			\exists a \in M^{\oplus E} \.
			t = \sum_{e \in E} a \otimes e
		}
	\NoProof
}\Page{
	\Theorem{FreeTensoringIsExact}{\forall R \in \ANN \. \forall F : \FM(R) \. T_F : \TYPE{Exact}\Big(\LMOD{R},\LMOD{R}\Big)}
	\Assume{0 \Arrow{0} A \Arrow{f} B \Arrow{g} C \Arrow{0} 0  }{\TYPE{ShortExact}(R)}
	\Say{[1]}{\THM{InjectiveByExact}(0 \Arrow{0} A \Arrow{f} B \Arrow{g} C \Arrow{0} 0)}{ (f : A \ToInj B) }
	\Say{E}{\THM{FreeHasBasis}(F)}{\TYPE{Basis}(F)}
	\Assume{t}{A \otimes F}
	\Assume{[2]}{f \otimes \id (t) = 0}
	\Say{\big(a,[1]\big)}{\THM{TensorProductDistributive}(A,F)\bd \TYPE{Basis} E}{ \sum a : A^{\oplus E} \. t = \sum_{e \in E } a_e \otimes e }
	\Say{[3]}{[2]\bd \FUNC{TensorMap}(f,t)[1] }{ 0 = f \otimes \id (t) =  f(a_e) \otimes e }
	\Say{[4]}{\bd \TYPE{Basis}(E)[3]}{f(a) = 0}
	\Conclude{[t.*]}{\THM{ZeroKernelTHM}[1][4]}{a = 0}
	\Derive{[2]}{\THM{ZeroKernelTHM}}{f \otimes \id : A \otimes F \ToInj B \otimes F} 
	\Conclude{[A.*]}{\THM{ExactTensorTHM}[2]}{ 0 \Arrow{0} A \otimes F \Arrow{f \otimes \id} B \otimes F \Arrow{g \otimes \id} C \otimes F \Arrow{0} 0   : \TYPE{ShortExact}(R)}
	\DeriveConclude{[*]}{\bd^{-1} \TYPE{Exact} \bd^{-1} \FUNC{TensorWith}}{T_F : \TYPE{Exact}\Big(\LMOD{R},\LMOD{R}\Big)}
	\EndProof 
	\\
	\Theorem{ProjectiveTensoringIsExact}{\forall R \in \ANN \. \forall P : \TYPE{Projective}(R) \. T_P : \TYPE{Exact}\Big(\LMOD{R},\LMOD{R}\Big)}
	\Say{\Big(Q,[1]\Big)}{\bd }{\sum Q \in \LMOD{R} \. P \oplus Q : \FM(R)}
	\Assume{0 \Arrow{0} A \Arrow{f} B \Arrow{g} \Arrow{0} 0  }{\TYPE{ShortExact}(R)}
	\Say{[2]}{\THM{FreeTensoringIsExact}(P \oplus Q )[1]}{ 
		\NewLine : 
		0 \Arrow{0} A \otimes (P \oplus Q) \Arrow{f \otimes \id} B \otimes (P \oplus Q) \Arrow{g \otimes \id} C \otimes (P \oplus Q) \Arrow{0} 0 : \TYPE{ShortExact}(\LMOD{R})} 
	\Say{[3]}{\THM{InjectiveByExact}[2]}{ f \otimes \id : A \otimes (P \oplus Q) \ToInj B \otimes (P \oplus Q) }
	\Say{[4]}{\THM{TensorProductDistributive}(P,Q)\bd^{-1}\FUNC{image}(f \otimes \id)}{{\im f \otimes \id}_{|A \otimes P} = B \otimes P}
	\Say{[5]}{[4]\bd \FUNC{Restrict}[4]}{f \otimes {\id}_P = f \otimes {\id}_{P \oplus Q |A \otimes P}^{B \otimes P}}
	\Say{[6]}{\THM{RestrictionPreservesInj}}{f \otimes {\id}_P :  A \otimes P \ToInj B \otimes P }
	\Conclude{[A.*]}{\THM{ExactTensorTHM}[2]}{ 0 \Arrow{0} A \otimes P \Arrow{f \otimes \id} B \otimes P \Arrow{g \otimes \id} C \otimes P \Arrow{0} 0   : \TYPE{ShortExact}(R)}
	\DeriveConclude{[*]}{\bd^{-1} \TYPE{Exact} \bd^{-1} \FUNC{TensorWith}}{T_P : \TYPE{Exact}\Big(\LMOD{R},\LMOD{R}\Big)}
	\EndProof 
	\\
	\Theorem{KroneckerProduct}{ 
		\forall R \in \ANN \. 
		\forall A,B : \FM \And \FGM(R) \. \NewLine \. 
		\forall e : \TYPE{Basis}(A) \.
		\forall f : \TYPE{Basis}(B) \.
		\forall T : \End_{\LMOD{R}}(A) \. 
		\forall S ; \End_{\LMOD{R}}(B) \. \NewLine
		 (T \otimes S)^{e \otimes f, e \otimes f} =  \FUNC{fromBlocks}(\lambda i,j \in \rank A \. T^{e,e}_{i,j}S^{f,f}, \rank A \times \rank B,\rank B, \rank B )
	}
	\NoProof
} 
\Page{
	\Theorem{ImageOfTensor}{\forall R \in \ANN \. \forall n \in \Nat \. \forall  A,B  n \to  \LMOD{R} \. \forall T : \prod^n_{i=1} A \Arrow{\LMOD{R}} B \. 
		\im \bigotimes^n_{i=1} T_i = \bigotimes^n_{i=1} \im T_i}
	\NoProof
	\\
	\Theorem{TensorMapRank}{
			\forall R \in \ANN \. 
			\forall n \in \Nat \.
			\forall A : n \to \LMOD{R} \. \NewLine
			\forall B : n \to \FM(R) \.
			\forall T : \prod^n_{i=1} : \prod^n_{i=1} A_i \Arrow{\LMOD{R}} B_i \.
			\rank \bigotimes^n_{i=1} T_i = \prod^n_{i=1} \rank T_i
		}
	\Say{[1]}{\THM{ImageOfTensor}(R,n,A,B,T)}{\im \bigotimes^n_{i=1} T_i = \bigotimes^n_{i=1} \im T_i}
	\Conclude{[2]}{ \THM{RankOfTensorProduct}[1]}{\rank \bigotimes^n_{i=1} T_i} 
	\EndProof
	\\
	\Theorem{TensorMapTrace}{
		\forall R \in \ANN \.
		\forall n \in \Nat \.
		\forall A : n \to \FM(R) \And \FGM(R) \. \NewLine \. 
		\forall T :  \prod^n_{i=1} A_i \Arrow{\LMOD{R}} A_i \.
		\tr \bigotimes^n_{i=1} T_i = \prod^n_{i=1} \tr T_i
	}
	\Say{\big(m,e)}{\THM{FreeHasBase}(A)}{ \prod^n_{k=1} \sum m_k \in \Nat \. e : \TYPE{Basis}(m_k,A_k) } 
	\Assume{I}{\prod^n_{k=1}m_k}
	\Say{[I.1]}{
		\bd \FUNC{tensorMap}(T)\bigotimes^n_{i=1}
		\bd^{-1} \FUNC{matrixOfOperator}(T,e) 
		\THM{MultiAdditive}(\bigotimes) 
		\THM{MultiHomogen}(\bigotimes)
	}
	{
		\NewLine
		\bigotimes^n_{i=1} T_i \bigotimes^n_{i=1} e_{i,I_i} = 
		\bigotimes^n_{i=1} T_i(e_{i,I_i}) = 
		\sum J \in \prod^n_{i=1} \.  \bigotimes^n_{i=1}T^{e_i,e_i}_{i;J_{i},I_{o}}e_{J_{i}} =
		\sum J \in \prod^n_{i=1} \. \prod^n_{i=1} T^{e_i,e_i}_{i;J_{i},I_{i}}\bigotimes^n_{i=1} e_{J_{i}} 
	}
	\Conclude{[I.*]}{ \THM{BasisOfTensorProduct}[I.1](I)}{   
		 \left( \bigotimes^n_{i=1} T_i \bigotimes^n_{i=1} e_{i,I_i} \right)_I = 
		 \prod^n_{i=1} T^{e_i,e_i}_{I_i,I_i}
	}
	\Derive{[1]}{I(\forall)}{
		\forall I \in \prod^n_{i=1} m_i \.   
		 \left( \bigotimes^n_{i=1} T_i \bigotimes^n_{i=1} e_{i,I_i} \right)_I = 
		 \prod^n_{i=1} T^{e_i,e_i}_{i;I_i,I_i} }
	 \Conclude{[*]}{\bd\FUNC{trace}\bd\ANN(R)\bd^{-1}\FUNC{trace}}{ 
	 	\tr \bigotimes^n_{i=1} T_i = 
		\sum J \in \prod^n_{i=1} m_i \.  \prod^n_{i=1}  T^{e_i,e_i}_{i;I_i,I_i} =   
		\prod^n_{i=1} \sum^{m_i}_{j=1} T^{e_i,e_i}_{i;j,j} =
		\prod^n_{i=1} \tr T_i
	}
	\EndProof
}\Page{
	\Theorem{DoubleTensorMapDet}{
		\forall R \in \ANN \.
		\forall A, B \in \FM(R) \And \FGM(R) \. \NewLine \.
		\forall T : A \Arrow{\LMOD{R}} A \.
		\forall S : B \Arrow{\LMOD{R}} B \.
		\det T \otimes S = (\det T)^{\rank B}(\det S)^{\rank A}
	}
	\Say{e}{\THM{FreeHasBasis}(A)}{\Basis(\rank A,A)}
	\Say{f}{\THM{FreeHasBasis}(B)}{\Basis(\rank B,B)}
	\Assume{i}{\rank A}
	\Assume{j}{\rank B}
	\Say{[i.j.1]}{ \bd \FUNC{tensorMap}(T,\id)\bd^{-1} \FUNC{matrixOfOperator} }{
		\NewLine :
		T \otimes {\id}_B(e_i \otimes f_j) = 
		T(e_i) \otimes f_j = 
		\sum^{\rank A}_{a=1}   T^{e,e}_{i,i}e_a \otimes f_j =
		\sum^{\rank A}_{a=1}  T^{e,e}_{i,i}(e_a \otimes f_j)
	}
	\Assume{i'}{\rank A}
	\Assume{j'}{\rank B}
	\Conclude{[i.j.i'.j'.*]}{ \THM{BasisOfTensorProduct}[i.j.1]}
	{
		\left( T \otimes {id}_B(e_i \otimes f_j) \right)_{(i',j')} = \delta_{j'}^j T^{e,e}_{i',i} 	
	}
	\DeriveConclude{[i.j.*]}{I(\forall)}{ \forall i' \in \rank A \. \forall j' \in \rank B \.  \left( T \otimes \id(e_i \otimes f_j) \right)_{(i',j')} = \delta^{j}_{j'}T^{e,e}_{a,i}  }
	\DeriveConclude{[1]}{\bd^{-1} \FUNC{matrixOfOperator}(T \otimes {\id}_B)}{
		\forall i,i' \in \rank A \. \forall j,j' \in \rank B \.  (T \otimes {\id}_B)^{e\otimes f, e \otimes f}_{(i',j'),(i,j)} = \delta^{j}_{j'}T^{e,e}_{i',i} }
	\Say{[2]}{\bd^{-1} \TYPE{BlockDiagonal}[1] }{ \Big( (T \otimes {\id}_B)^{e \otimes f. e \otimes f} : \TYPE{BlockDaigonal}(\rank B, T^{e,e}) \Big)    }
	\Say{[3]}{ \THM{BlockDiagonalDet}[2] }{ \det T \otimes {\id}_B =  (\det T)^{\rank B}}
	\Assume{i}{\rank A}
	\Assume{j}{\rank B}
	\Say{[i.j.1]}{ \bd \FUNC{tensorMap}(\id,S)\bd^{-1} \FUNC{matrixOfOperator} }{
		\NewLine :
		{\id}_A \otimes S(e_i \otimes f_j) = 
		e_i \otimes S(f_j) = 
		\sum^{\rank B}_{b=1}   S^{f,f}_{b,j}e_i \otimes f_b =
		\sum^{\rank B}_{b=1}  S^{f,f}_{b,j}(e_i \otimes f_b)
	}
	\Assume{i'}{\rank A}
	\Assume{j'}{\rank B}
	\Conclude{[i.j.i'.j'.*]}{ \THM{BasisOfTensorProduct}[i.j.1]}
	{
		\left( {\id}_A \otimes S(e_i \otimes \id) \right)_{(i',j')} = \delta_{i'}^i S^{f,f}_{j',j} 	
	}
	\DeriveConclude{[i.j.*]}{I(\forall)}{ \forall i' \in \rank A \. \forall j' \in \rank B \.  \left( {\id}_A \otimes S)( e_i \otimes f_j) \right)_{(i',j')} = \delta^{i}_{i'}S^{f,f'}_{b,i}  }
	\DeriveConclude{[4]}{\bd^{-1} \FUNC{matrixOfOperator}({\id}_A \otimes S)}{
		\forall i,i' \in \rank A \. \forall j,j' \in \rank B \.  ({\id}_A \otimes S)^{e \otimes f, e \otimes f}_{(i',j'),(i,j)} = \delta^i_{i'}S^{f,f}_{j',j} }
	\Say{[5]}{\bd^{-1} \TYPE{BlockDiagonal}[1] }{ \Big( ({\id}_A \otimes T)^{e \otimes f. e \otimes f} : \TYPE{BlockDaigonal}(\rank A, S^{f,f}) \Big)    }
	\Say{[6]}{ \THM{BlockDiagonalDet}[2] }{ \det  {\id}_A \otimes S  =  (\det S)^{\rank A}}
	\Conclude{[*]}{\THM{DetProduct}[6][3]}{\det T \otimes S =  \det (T \otimes \id_{A}) \det ( \id_{B} \otimes S ) = (\det T)^{\rank B}(\det S)^{\rank A} }
	\EndProof
	\\
	\Theorem{TensorMapDet}{
		\forall R \in \ANN \.
		\forall n \in \Nat \.
		\forall A : n \to \FM(R) \And \FGM(R) \. \NewLine \.
		\forall T : \prod^n_{i=1}  A_i \Arrow{\LMOD{R}} A_i \.
		\det \bigotimes^n_{i=1} T_i = \prod^n_{i=1} \det (T_i)^{N_i}  
		\quad \where \quad 
		N = \Lambda i \in n \. \prod_{j \neq i}^n \rank A_j
	}
	\NoProof
}
\newpage
\subsection{Flatness}
\Page{
	\DeclareType{Flat}{\prod R \in \ANN \. ?\LMOD{R}}
	\DefineType{M}{Flat}{ T_M : \TYPE{Exact} }
	\\
	\Theorem{ProjectiveIsFlat}{\forall R \in \ANN \. \forall P : \TYPE{Projective}(R) \. P : \TYPE{Flat}(R)} 
	\NoProof
	\\
	\Theorem{NonFlatQuotient}{\forall R \in \ANN \. \forall I : \TYPE{ProperIdeal}(R) \. \forall a \in R^\times \. \forall [0] : a \in I \. \frac{R}{I} \IsNot \TYPE{Flat}(R)}
	\Say{[1]}{[0]\bd \TYPE{ProperIdeal}[0]}{ a \not\in R^*  }
	\Say{(b,[2])}{\bd \TYPE{ProperIdeal}(I)}{\sum b \in R \. b \not \in I}
	\Say{C}{0 \Arrow{0} R  \Arrow{\cdot a} R \Arrow{\pi_{(a)}} \frac{R}{(a)} \Arrow{0} 0}{\TYPE{ShortExact}}
	\Say{[3]}{\bd \TYPE{TensorProduct}[2] }{ [1] \otimes b \neq_{\frac{R}{I} \otimes R}} 
	\Say{[4]}{\bd \TYPE{TrivialModule}}{ R \otimes \frac{R}{I} \neq \{0\}  }
	\Say{[5]}{\bd \TYPE{TensorProduct}\bd \FUNC{quotientModule} \bd \FUNC{temsorMap}}{a \otimes {\id}_{R \otimes \frac{R}{I}} = 0}
	\Say{[6]}{\bd^{-1} \TYPE{Exact}[5][4]}{ C \otimes \frac{R}{I} \IsNot \TYPE{Exact}}
	\Conclude{[*]}{\bd^{-1} \TYPE{Flat}[6]}{\frac{R}{I} \IsNot \TYPE{Flat}}
	\EndProof
	\\
	\Theorem{FlatDirectSum}{\forall R \in \ANN \. \forall X \in \SET \. \forall M : X \to \TYPE{Flat}(R) \. \bigoplus_{x \in X} M_x : \TYPE{Flat}(R)}
	\Assume{(V,f)}{\TYPE{ShortExact}(\LMOD{R})}
	\Say{[1]}{\bd \TYPE{Flat}(M,V)}{\forall i \in n \. M_i \otimes (V,f) : \TYPE{Exact}}
	\Say{[2]}{\THM{ExactDirectSum}[1]}{ \left( \bigotimes^n_{i=1} (M_i \otimes (V,f)) : \TYPE{Exact} \right)}
	\Conclude{[(V,f).*]}{\THM{TensorProductDistributive}[2]}
	{\left( \left( \bigoplus^n_{i=1} M_i  \right) \otimes (V,f) : \TYPE{Exact} \right) }                                                          
	\EndProof
}\Page{	
	\Theorem{FlatTensorProduct}{\forall R \in \ANN \. \forall n \in \Nat \. \forall M : n \to \TYPE{Flat}(R) \.\bigotimes^n_{i=1} M_i : \TYPE{Flat}(R)}
	\Say{\mars}{\Lambda  n \in \Nat \. \forall M : n \to \TYPE{Flat}(R) \. \bigotimes^n_{i=1} M_i : \TYPE{Flat}(R)}{\Nat \to \Type}
	\Say{[1]}{\bd \FUNC{tensorProduct} \ByConstr^{-1}\mars}{\mars(1)}
	\Assume{n}{\Nat}
	\Assume{[n.1]}{\mars(n)}
	\Assume{M}{(n + 1) \to \TYPE{Flat}(R)}
	\Say{[M.1]}{\ByConstr \mars [n.1](M_{|n})}{(T_{\otimes^n_{i=1} M_i} : \TYPE{Exact})}
	\Say{[M.2]}{ \bd \TYPE{Flat}(M_{n+1})}{ (T_{M_{n+1}} : \TYPE{Exact})  }
	\Conclude{[M.*]}{ \THM{ExactCompose}[M.1][M.2] \THM{TensorProductAssoc}(M) }
	{
		\NewLine : 
		T_{\bigotimes^{n+1}_{i=1} M_i} = 
		T_{M_{n+1}\otimes\bigotimes^n_{i=1} M_i } =
		T_{M_{n+1}} T_{\bigotimes^n_{i=1} M_i} : \TYPE{Exact}
	}
	\DeriveConclude{[n.*]}{\ByConstr^{-1}\mars \bd^{-1} \TYPE{Flat}}{ \mars[n+1]}
	\DeriveConclude{[*]}{\bd \TYPE{NaturalSet}(\Nat)(n,M)}{ \left( \bigotimes^n_{i=1} M_i : \TYPE{Flat}(R) \right)}
	\EndProof
	\\	
	\Theorem{FlatBySubmodules}{\forall R \in \ANN \. \forall M \in \LMOD{R} \. 
		\NewLine \. \forall [0] : \forall N : \FGM(R) \And \TYPE{Submodule}(R,M) \. N : \TYPE{Flat}(R) \. M : \TYPE{Flat}(R)}
	\Assume{(V,f)}{\TYPE{ShortExact}(\LMOD{R})}
	\Assume{\sum^k_{i=1} m_i \otimes v_i}{M \otimes V_2}
	\Say{N}{\Span(m)}{\FGM(R) \and \TYPE{Submodule}(R,M)}
	\Say{[m.1]}{[0](N)}{(N : \TYPE{Flat}(R))}
	\Conclude{[m.*]}{\THM{ZeroKernelTHM}\left( m_i \otimes v_i\right)\THM{InjectiveByExactness}(f_2 \otimes {\id}_N)\bd \TYPE{Flat}(R)(N)}
	{ \NewLine :\sum^n_{i=1} m_i \otimes f_2(v_i)  = 0 \iff \sum^k_{i=1} v_i \otimes m_i = 0  }
	\Derive{[1]}{\THM{ZeroKernelTHM}I(\forall)}{ (f_2 \otimes {\id}_M : V_2 \otimes M \ToInj V_1) ш}
	\Assume{\sum^k_{i=1}  v_i \otimes m_i}{ V_0 \otimes M}	
	\Say{N}{\Span(m)}{\FGM(R) \and \TYPE{Submodule}(R,M)}
	\Say{[m.1]}{[0](N)}{(N : \TYPE{Flat}(R))}
	\Conclude{[m.*]}{ \bd \TYPE{Surjective}\THM{SurjectiveByExactness} \bd \TYPE{Flat}(R)(N)(V,f)}
	{ \NewLine : \exists \sum^{k'}_{i=1} v_i' \otimes m_i' \in  V_1 \otimes M \.   f_1 \otimes {\id}_N\left( \sum^{k'}_{i=1} v_i' \otimes m_i' \right) = \sum^k_{i=1} v_i \otimes m_i}
	\Derive{[2]}{\bd^{-1} \TYPE{Surjective} }{ f_1 \otimes {\id}_M : M \otimes V_1 \ToSurj M \otimes V_0}
}\Page{
	\Assume{ \sum^k_{i=1} v_i \otimes m_i}{ \ker f_1 \otimes {\id}_M }
	\Say{N}{\Span(m)}{\FGM(R) \and \TYPE{Submodule}(R,M)}
	\Say{[m.1]}{[0](N)}{(N : \TYPE{Flat}(R))}
	\Conclude{[m.*]}{ \bd \TYPE{Exact} \bd \TYPE{Flat}(R)(N)(V,f)}
	{\exists \sum^{k'}_{i=1} v_i' \otimes m_i' \in  V_2 \otimes M \.   f_2 \otimes {\id}_N\left( \sum^{k'}_{i=1} v_i' \otimes m_i' \right) = \sum^k_{i=1} v_i \otimes m_i}
	\DeriveConclude{[V.*]}{\bd^{-1} \TYPE{Exact}[1][2]}{(M \otimes (V,f) : \TYPE{Exact})}
	\DeriveConclude{[*]}{\bd^{-1}\TYPE{Flat}\bd^{-1}\TYPE{Exact}I(\forall)}{(M : \TYPE{Flat}(R))}
	\EndProof
	\\
	\Theorem{RatsAreFlatButNotProjective}{ \Rats : \TYPE{Flat}(\Int) \And \Rats \IsNot \TYPE{Projective}(\Int)}
	\Assume{N}{\FGM(\Int) \And \TYPE{Submodule}(\Int,\Rats)}
	\Say{\left(n,\frac{a}{b},[1] \right)}{ \bd N }{ \sum n \in \Nat \. \frac{a}{b} \in \Rats^n \. N = \Span_\Int\left( \frac{a}{b}\right) }
	\Say{[2]}{[1]\bd \Rats}{ N \subset \frac{\Int}{\prod^n_{i=1} b_i}}
	\Say{[3]}{\THM{CyclicSubsetIsCyclic}[2]}{ (N : \TYPE{Cyclic})  }
	\Say{[4]}{\THM{InfiniteCyclic}[3]}{N \cong_{\LMOD{\Int}} \Int }
	\Conclude{[N.*]}{\THM{FreeIsProjective} \; \THM{ProjectiveIsFlat}}{(N : \TYPE{Flat}(\Int))}
	\Derive{[1]}{\THM{FlatBySubmodules}}{(\Rats : \TYPE{Flat}(\Int))}
	\Assume{\varphi}{\Rats \Arrow{\LMOD{\Int}} \Int}
	\Assume{\frac{a}{b}}{\Rats}
	\Assume{[2]}{\varphi\Act{\frac{a}{b}} \neq 0}
	\Say{n}{\varphi\Act{\frac{a}{b}}}{\Int^\times}
	\Say{[3]}{\bd \LMOD{\Int}(\Rats,\Int)(\varphi) \ByConstr^{-1}(n)}{n\varphi\Act{\frac{a}{nb}} = \varphi\Act{\frac{a}{b}} =  n } 
	\Say{[4]}{\THM{InjMult}}{ \varphi\Act{\frac{a}{nb}} = 1   }
	\Say{[5]}{\bd \LMOD{\Int}}{ 2\varphi\Act{\frac{a}{2nb}} = \varphi\Act{\frac{a}{nb}} = 1  }
	\Say{[6]}{\THM{DivisorsOfUnity}[5]}{2 = 1}
	\Conclude{[\varphi.*]}{I(\bot)[6]}{\bot}
	\Derive{[2]}{I(\forall)I(=,\to)E(\bot)}{ \forall \varphi : \Rats \Arrow{\LMOD{\Int}} \Int}
	\Assume{[4]}{(\Rats : \TYPE{Projective}(\Int))}
	\Say{(X,P,[5])}{\bd \TYPE{Projective}[4]}{\sum X \in \SET \. \sum P \in \LMOD{\Int} \. \Rats \oplus P \cong X^{\Int}}
	\Assume{x}{X}
	\Say{f}{\iota_\Rats \pi_x}{ \Rats \Arrow{\LMOD{\Int}} \Int}
	\Conclude{[x.*]}{[2](f)}{f = 0}
	\Derive{[5]}{\bd \TYPE{Product}}{\iota_\Rats = 0}
	\Say{[6]}{ \bd \TYPE{Injective}(\iota)\THM{ZeroKernelTHM}[5]}{ \Rats = \{0\} }
	\Conclude{[4.*]}{I(\bot)[6]}{\bot }
	\DeriveConclude{[*]}{E(\bot)}{\Rats \IsNot \TYPE{Projective}(R)}
	\EndProof
}
\newpage
\subsection{Covariant Scalar Extension}
\Page{
	\DeclareType{Bimodule}{\prod R,S \in \RING \. ?\sum M \in \SET \. (M \times M \to M) \times (R \times M \to M) \times (S \times M \to M)}
	\DefineType{\Big(M,+,\odot_1,\odot_2\Big)}{Bimodule}{(M,+,\odot_1) \in \LMOD{R} \And (M,+,\odot_2) \in \LMOD{S} 
		\And \NewLine \And \forall \alpha \in R \. \forall \beta \in S \. \forall a \in  M \. \beta \odot_2 (\alpha \odot_1 a) = \alpha \odot_1 (\beta \odot_2 a)} 
	\\
	\DeclareFunc{bimoduleCategory}{\RING^2 \to \CAT}
	\DefineNamedFunc{bimoduleCategory}{R,S}{\LMOD{(R,S)}}{( \TYPE{Bimodule}, \LMOD{R} \cap \LMOD{S},\circ,\id )}
	\\
	\DeclareFunc{leftTensorBimodule}{\prod R,S \in \ANN \. \LMOD{(R,S)} \to \LMOD{R} \to \LMOD{(R,S)}}
	\DefineNamedFunc{leftTensorBimodule}{ V,M }{V \otimes_R M}{ \NewLine \de
		\Big(V \otimes_R M, +, \cdot, \Lambda s \in S \. \FUNC{tensorization}(\Lambda v \in V \. \Lambda m \in M \. (sv) \otimes m ) \Big)} 
	\\
	\DeclareFunc{rightTensorBimodule}{\prod R,S \in \ANN \. \LMOD{(R,S)} \to \LMOD{R} \to \LMOD{(R,S)}}
	\DefineNamedFunc{rightTensorBimodule}{ V,M }{M \otimes_R V}{ \NewLine \de
		\Big(M \otimes_R V, +, \cdot, \Lambda s \in S \. \FUNC{tensorization}(\Lambda v \in V \. \Lambda m \in M \. m \otimes (sv)  \Big)} 
	\\
	\Theorem{TensorCommutes}{\forall R,S \in \ANN \. \forall V \in \LMOD{(R,S)} \. \forall M \in \LMOD{R} \. M \otimes_R V \cong_{\LMOD{(R,S)}} V \otimes_R M }
	\NoProof
	\\
	\Theorem{TensorAssociativityLaw}{
		\forall R,S \in \ANN \. \forall V \in \LMOD{(R,S)} \. \forall A \in \LMOD{R} \. \forall B \in \LMOD{S} \.   
		\NewLine (A \otimes_R V) \otimes_S B  \cong_{\LMOD{(R,S)}} A \otimes_R (V \otimes_S B)
	}
	\NoProof
	\\
	\DeclareFunc{morphismExtension}{\prod R,S \in \ANN \.  \Big(S \Arrow{\ANN} R\Big) \to \LMOD{R} \to \LMOD{(R,S)}}
	\DefineNamedFunc{morphismExtension}{\varphi,M}{M_\varphi}{\Big( M,+,\cdot,\Lambda s \in S \. \Lambda m \in M \. \varphi(s)m \Big) }
}\Page{
	\Theorem{BasisOfCovariantExtension}{
		\forall R,S \in \ANN \. 
		\forall \varphi : S \Arrow{\ANN} R \. 
		\forall F : \FM(R) \. 
		\forall E : \Basis(F) \. \NewLine \.  
		E \otimes 1 : \Basis(S,F \otimes_R S_\varphi)
	}
	\Assume{t}{F \otimes_R S_\varphi}
	\Say{\Big(s,[1]\Big)}{\THM{FreeTensotingDecomposition}(E,t)}{\sum s : S^{\oplus E} \.  t = \sum_{e \in E} e \otimes s_e}
	\Conclude{[t.*]}{\bd \FUNC{rightTensorBimodule}}{t = \sum_{e \in E} s_e e \otimes 1 }
	\Derive{[1]}{\bd^{-1}\FUNC{span}}{ \big( F \otimes_R S = {\Span}_S( E \otimes 1) \big) }
	\Assume{s}{S^{\oplus E}}
	\Assume{[2]}{ s E \otimes 1 = 0}
	\Assume{e}{E}
	\Say{[3]}{[2](e)}{ e \otimes s_e = 0}
	\Conclude{[e.*]}{\bd \Basis(E)[3]}{ s_e = 0}
	\DeriveConclude{[s.*]}{I(=,\to)}{s = 0}
	\DeriveConclude{[*]}{\bd^{-1}\TYPE{Basis}[1]}{\big(E \otimes 1 : \Basis(F \otimes S)\big) }
	\EndProof
	\\
	\Theorem{FreeCovariantExtension}{ 
		\forall R,S \in \ANN \.
		\forall \varphi : S \Arrow{\ANN} R \.
		\forall F : \FM(R) \.
		F \otimes_R S_\varphi : \FM(S) 
	}
	\NoProof
	\\
	\Theorem{FreeCovariantExtensionRank}{ 
		\forall R,S \in \ANN \.
		\forall \varphi : S \Arrow{\ANN} R \.
		\forall F : \FM(R) \. \NewLine \. 
		\rank_S F \otimes_R S_\varphi = \rank_R F 
	}
	\NoProof
	\\
	\Theorem{ProjectiveCovariantExtension}{
		\forall R,S \in \ANN \.
		\forall \varphi : S \Arrow{\ANN} R \.
		\forall P : \TYPE{Projective}(R) \. \NewLine \.
		P \otimes R S_\varphi : \TYPE{Projectve}(S) 
	}
	\Say{(Q,[1])}{\bd \TYPE{Projective}(P)}{\sum Q \in \LMOD{R} \. Q \oplus P : \FM(R)}
	\Say{[2]}{ \THM{FreeCovariantExtension}(R,S,\varphi, Q \oplus P) }{\Big( Q \oplus P \otimes_R S_\varphi : \FM(S) \Big)}
	\Say{[3]}{\THM{TensorProductDistributive}(S_\varphi,P,Q)}{ Q \otimes_R S_\varphi \oplus P \otimes_R S_\varphi \cong_{\LMOD{R}} Q \oplus P \otimes S_\varphi  }
	\Say{[4]}{ \bd \FUNC{leftBimodule}[3] }{ Q \otimes_R S_\varphi \oplus P \otimes_R S_\varphi \cong_{\LMOD{S}} Q \oplus P \otimes S_\varphi   }
	\Conclude{[*]}{\bd^{-1} \TYPE{Projective}}{\Big( P \otimes_R S_\varphi : \TYPE{Projective}(S)\Big)}
	\EndProof
}
\Page{
	\Theorem{CovariantExtensionDistributive}{
			\forall R,S \in \ANN \. 
			\forall \varphi : R \Arrow{\ANN} S \.
			\forall A,B \in \LMOD{R} \. \NewLine \. 
			 (A \otimes_R S_\varphi) \otimes_S (B \otimes_R S_\varphi ) \cong_{\LMOD{R}} (A \otimes_R B) \otimes_R  S_\varphi
		}
	\Say{X}{
		\FUNC{tensorize}_S \bigg( 
			 \Lambda   \sum^n_{i=1} a_i \otimes \alpha_i \in A \otimes_R S_\varphi \.
			 \Lambda  \sum^m_{i=1} b_i \otimes \beta_i \in B \otimes_R S_\varphi  \.  
			 \sum^n_{i=1} \sum^m_{j=1} a_i \otimes b_j \otimes \alpha_i \beta_j 
		\bigg)
	}
	{   \NewLine \.(A \otimes_R S_\varphi) \otimes_S (A \otimes R S_\varphi)\Arrow{\LMOD{S}} (A\otimes_R B) \otimes_R S_\varphi } 
	\Say{Y}{
		\Lambda \sum n \in \Nat \. \sum^n_{i=1} a_i \otimes b_i \otimes s_i  \in (A \otimes_R B) \otimes_R S_\varphi \.
		\sum^n_{i=1}( a_i \otimes 1)  \otimes  (b_i \otimes s)
	}
	{ \NewLine : (A \otimes_R B) \otimes_R S_\varphi \Arrow{\LMOD{S}} (A \otimes_R S_\varphi) \otimes_S (B \otimes_R S_\varphi) }
	\Assume{t}{(A \otimes_R S_\varphi) \otimes_S (B \otimes_R S_\varphi)}
	\Say{\big(n,a,b,\alpha,\beta,[1]\big)}{ \bd^3 \FUNC{tensorProduct}(t)}{ \NewLine : 
		\sum n \in \Nat \. \sum a : n \to A \. \sum b : n \to B \. \sum \alpha, \beta : n \to S \.   
		t = \sum^n_{i=1} (a_i \otimes \alpha_i) \otimes (b_i \otimes \beta_i)
	}
	\Conclude{[t.*]}{[1]\bd \LMOD{S}\ByConstr X \ByConstr Y   \THM{MultiHomogen}^{2n}(\ldots,\alpha) [1]}
	{ 
			\NewLine : YX(t) = 
			\sum^n_{i=1} YX\big( (a_i \otimes \alpha_i) \otimes (b_i \otimes \beta_i) \big) =
			\sum^n_{i=1} Y(a_i \otimes b_i \otimes  \alpha_i\beta_i) =
			\sum^n_{i=1} (a_i \otimes 1) \otimes (b_i \otimes \alpha_i\beta_i) = \NewLine = 
			\sum^n_{i=1} (a_i \otimes \alpha_i) \otimes (b_i \otimes \beta_i) =
			t
	}
	\Derive{[1]}{I(=.\to)}{YX = \id}
	\Assume{t}{A \otimes_R B \otimes_R S_\varphi}
	\Say{\big(n,a,b,s,[1]\big)}{\bd \FUNC{tensorProduct}(t)}{ 
		\NewLine : \sum n \in \Nat \. \sum a : n \to A \. \sum b : n \to B \. 
			\sum s : n \to S \. t = \sum^n_{i=1} a_i \otimes b_i \otimes s_i
		}
	\Conclude{[t.*]}{[1]\bd \LMOD{S}(XY)\ByConstr Y \ByConstr X[1]}
	{
		\NewLine : XY(t) =
		\sum^n_{i=1} XY (a_i \otimes  b_i \otimes s_i) = 
		\sum^n_{i=1} X\big( (a_i \otimes 1) \otimes (b_i \otimes s_i ) \big)= 
		\sum^n_{i=1} a_i \otimes b_i \otimes s_i = t
	}
	\Derive{[2]}{I(=,\to)}{XY = \id}
	\Conclude{[*]}{\bd^{-1}\TYPE{Isomorphic}[2][3]}{(A \otimes_R S_\varphi) \otimes_S (B \otimes_R S_\varphi ) \cong_{\LMOD{R}} (A \otimes_R B) \otimes_R  S_\varphi}
	\EndProof
	\\
	\Theorem{CovariantExtensionDistributive2}{
			\forall R,S \in \ANN \. 
			\forall \varphi : R \Arrow{\ANN} S \.
			\forall n \in \Nat \.
			\forall A : n \to \LMOD{R} \. \NewLine \. 
			\bigotimes^n_{i=1} (A_i \otimes_R  S_\varphi) \cong_{\LMOD{R}} \left(\bigotimes^n_{i=1} A_i\right) \otimes_R  S_\varphi
		}
	\NoProof
}
\Page{
	\Theorem{FlatCovariantExtension}{
			\forall R,S \in \ANN \.
			\forall \varphi : R \Arrow{\ANN} S \.
			\forall M : \TYPE{Flat}(R) \.
			M \otimes_R S_\varphi : \TYPE{Flat}(S)
		}                  
	\Assume{(V,f)}{\TYPE{ShortExact}(\LMOD{S})}
	\Say{[1]}{\THM{TensorAssociativityLaw}(R,S,\varphi,M,V)}{ 
		\NewLine : (M \otimes_R S_{\varphi}) \otimes_S (V,f) = M \otimes_R (S \otimes_S (V,f))_{\varphi}  = M \otimes_R (V,f)_\varphi}
	\Say{[2]}{\THM{ExactInAllStructures}((V,f),\varphi)}{ \Big( (V,f)_\varphi : \TYPE{Exact}(\LMOD{R})\Big) }
	\Say{[3]}{\bd \TYPE{Flat}(M)[2]}{\Big( M \otimes_R (V,f)_\varphi : \TYPE{Exact}(\LMOD{R})  \Big)}        
	\Conclude{[.*]}{\THM{ExactInAllStructures}[3]}{ \Big( M \otimes_R (V,f)_\varphi : \TYPE{Exact}(\LMOD{S}) \Big)  }
	\DeriveConclude{[*]}{\bd^{-1}\TYPE{Flat}(S)}{\Big(M \otimes_R S_\varphi : \TYPE{Exact}(S)\Big)}
	\EndProof
	\\
	\Theorem{FractionTensorZeroCondition}{
		\forall  R \in \ANN \.
		\forall  \Sigma : \MS(R) \.
		\forall M \in \LMOD{R} \. \NewLine \.  
		\forall m \in M \.
		\forall \sigma \in \Sigma \.
		m \otimes \frac{1}{\sigma} = 0 \iff
		\exists \sigma' \in \Sigma : \sigma' m = 0
	}
	\Assume{[1]}{m \otimes \frac{1}{\sigma} = 0}
	\Say{(M',S,[2])}{\THM{ZeroTensorInFGM}[1]}
	{ \NewLine : \sum M',S : \FGM(R) \. M' \subset M \And S \subset \Sigma^{-1} R \And , m \otimes  \frac{1}{\sigma} =_{M' \otimes S} 0  }
	\Say{(\varsigma,[3])}{\THM{FGFractionSet}(S)[2]}{ \sum \varsigma \in \Sigma \. S \subset \frac{R}{\sigma\varsigma}}
	\Say{\varphi}{\Lambda r \in R \. \frac{r}{\sigma\varsigma}}{ R \Arrow{\LMOD{R}} \frac{R}{\sigma\varsigma}    }
	\Say{I}{\ker \varphi}{\LMOD{R}}
	\Say{C}{ I \ToInj R \Arrow{\varphi} \frac{R}{\sigma\varsigma}}{\TYPE{ShortExact}}
	\Assume{a}{I}
	\Say{[a.1]}{\ByConstr(a)(I)}{ \frac{a}{\sigma \varsigma} = 0 }
	\Conclude{(\alpha,[a.*])}{\bd \Sigma^{-1}R[a.1]}{ \sum \alpha \in \Sigma \. \alpha a = 0  }
	\Derive{(\alpha,[4])}{}{\sum \alpha I \to \Sigma \. \forall a \in I \. a \alpha_a = 0}
	\Say{[5]}{\THM{TensorProductRightExact}(M,C)}{ (M \otimes C : \TYPE{RightExact})}
	\Say{[6]}{\bd \TYPE{RightExact}(M \otimes C)[1]}{ m \otimes \varsigma \in M \otimes I }
	\Say{(\varsigma',a,[7])}{[4][6]}{ \sum \varsigma' \in \Sigma \. \varsigma \varsigma' m = 0 }
	\Conclude{[1.*]}{\bd \MS(\Sigma)(\varsigma,\varsigma')}{\varsigma\varsigma' \in \Sigma}
	\Derive{[1]}{I(\Rightarrow)}{\LOGIC{Left} \Rightarrow \LOGIC{Right}}
	\Assume{(\sigma',[2])}{ \sum \sigma' \in \Sigma : \sigma' m = 0}
	\Conclude{[2.*] }{ \THM{MultiHomogen}(\sigma')[2]\THM{MultiHomogen}(0)}{
		m \otimes \frac{1}{\sigma} = 
		\sigma' m \otimes \frac{1}{\sigma \sigma'} = 
		0 \otimes \frac{1}{\sigma\sigma'} = 
		0
	}
	\Derive{[*]}{I(\iff)[1]I(\Rightarrow)}{\LOGIC{This}}
	\EndProof
}\Page{
	\Theorem{FractionFormTensor}{ 
		\forall R \in \ANN \. 
		\forall \Sigma : \MS(R) \. 
		\forall M \in \LMOD{R} \.  
		\forall t \in M \otimes \Sigma^{-1}R \. \NewLine \.
		\exists m \in M : \exists \sigma \in \Sigma :
		t = m \otimes \frac{1}{\sigma}
	}
	\NoProof
	\\
	\Theorem{FractionsAreFlat}{\forall R \in \ANN \. \forall \Sigma : \MS(R) \. \Sigma^{-1} R : \TYPE{Flat}(R)}
	\Assume{A,B}{\LMOD{R}}
	\Assume{f}{A \Arrow{\LMOD{R}} B}
	\Assume{m \otimes \frac{1}{\sigma}}{\ker \Big(f \otimes {\id}_{\Sigma^{-1}R}\Big)}
	\Say{[1]}{\bd \ker \bd \FUNC{tensorMap}}{   0 =  f \otimes {\id}_{\Sigma^{-1}R} \left(m \otimes \frac{1}{\sigma} \right) = f(m) \otimes \frac{1}{\sigma}}
	\Say{(\sigma',[2])}{\THM{FractionZeroTensorCondition}}{\sum \sigma' \in \Sigma \. \sigma' f(m) = 0}
	\Conclude{[A.*]}{\bd^{-1} \FUNC{kernel} \bd }{\sigma' m \in \ker f}
	\Derive{[1]}{I^{3}(\forall)}
	{
		\forall A,B \in \LMOD{R} \. \forall f : A \Arrow{\LMOD{R}} B \. \forall m \otimes \frac{1}{\sigma} \in \ker f \otimes {\id}_{\Sigma^{-1}R} \. 
		\exists \sigma' \in \Sigma : \sigma' m \in \ker f
	}
	\Assume{(V,f)}{\TYPE{ShortExact}(\LMOD{R})}
	\Say{[2]}{[1](V_2,V_1,f_2)}{\forall v \otimes \frac{1}{\sigma}  \in \ker(f_2 \otimes {\id}_{\Sigma^{-1}R}) \. \exists \sigma' \in \Sigma \. \sigma'v = 0 }
	\Say{[3]}{\THM{ZeroFractionTensorCondition}[2]}{\ker(f_2 \otimes {\id}_{\Sigma^{-1}R})= \{0\}}
	\Say{[4]}{\THM{ZeroKernelTHM}[3]}{f_3 \otimes {\id}_{\Sigma^{-1}R} : V_2 \otimes \Sigma^{-1}R \ToInj V_1 \otimes \Sigma^{-1}R }
	\Assume{v \otimes \frac{1}{\sigma}}{\ker (f_1 \otimes {\id}_{\Sigma^{-1}R})} 
	\Say{(\sigma',[5])}{ [1]\left(V_1,V_0,f_1,v \otimes \frac{1}{\sigma} \right)}{ \sum \sigma' \in \Sigma \.  f_1(\sigma' v) = 0   }
	\Say{(w,[6])}{\bd \TYPE{ShortExact}(f_1)}{ \sum w \in V_2 \. f_2(w) = \sigma' v   }
	\Conclude{[v.*]}{ \bd \FUNC{tensorMap} [6] \THM{MultiHomogeb} }
	{ 
		f_2 \otimes  {\id}_{\Sigma^{-1}R}  w \otimes \frac{1}{\sigma \sigma'} = 
		\sigma' v \otimes \frac{1}{\sigma \sigma'} = 
		v \otimes \frac{1}{\sigma} 
	}
	\DeriveConclude{[V.*]}{\bd^{-1} \TYPE{Exact}[4]}{((V,f) \otimes \Sigma^{-1} R : \TYPE{Exact})}
	\DeriveConclude{[*]}{\bd^{-1}\TYPE{Flat}}{\Big(\Sigma^{-1}R : \TYPE{Flat}(R)\Big)}
	\EndProof
}
\newpage
\subsection{Composition Algebra}
\Page{
	\Theorem{TensorBilinearProduct}
	{
		\forall R \in \ANN \. 
		\forall V,W,U,X,Y,Z \in \LMOD{R} \. \NewLine \.
		\forall A \in \L(V,W;U) \. 
		\forall B \in \L(X,Y;Z) \.  
		\exists! C : \L(V \otimes X, W \otimes Y;U \otimes Z) : \NewLine : 
		\forall v \in V \. \forall x \in X \.
		\forall w \in W \. \forall y \in Y \.
		C(v \otimes x, w \otimes y) = A(v,x) \otimes B(w,y)
	}
	\NoProof
	\\
	\Theorem{TensorBilinearProduct2}
	{
		\forall R \in \ANN \. 
		\forall n,m \in \Nat \.
		\forall W : n \to \LMOD{R} \.
		\forall V : n \times m \to \LMOD{R} \. 
		\NewLine \.
		\forall A  : \prod^n_{i=1} \L(V_i;W_i) \. 
		\exists! C : \L\left( \bigotimes^n_{i=1} V_n ; \bigotimes^n_{i=1} W_n\right) : \NewLine : 
		\forall v \in \prod^{m \times n}_{i,j = 1} V_{i,j} \. 
		C\left(\bigotimes^n_{j=1} v_{i,j}\right)^m_{i=1} = \bigotimes^n_{i=1} A_n(v_i) 
	}
	\NoProof
	\\
	\DeclareFunc{bilinearMapTensorProduct}
	{ 
		\prod R \in \ANN \.
		\prod n,m \in \Nat \.  
		\prod W : n \to \LMOD{R} \. \NewLine \. 
		\prod V : n \times m \to \LMOD{R} \. 
		\prod^n_{i=1} \L(V_n;W_n)  \to \L\left(\bigotimes^n_{i=1} V_n; \bigotimes^n_{i=1} W_n\right) 
	}
	\DefineNamedFunc{bilinearMapTensorProduct}{A}{\bigotimes^n_{i=1} A_i}{ \THM{TensorBilinearProduct2}}
	\\
	\Theorem{BilinearFuncTensorProduct}
	{
		\forall R \in \ANN \.
		\forall n,m \in \Nat \.
		\forall V : n \times m \to \LMOD{R} \. \NewLine \.
		\forall A  : \prod^n_{i=1} \L(V_i;R) \. 
		\forall v \in \prod^{m \times n}_{i,j = 1} V_{i,j} \. 
		\bigotimes^n_{i=1} A_i \left(\bigotimes^n_{j=1} v_{i,j} \right)^m_{i=1} = \prod^n_{i=1} A_i(v_i)
	}
	\NoProof
	\\
	\Theorem{NondegenerateNensorProductCondition}
	{
		\forall k : \Field \.
		\forall n,m \in \Nat \.
		\forall V : n \times m \to \VS{k} \. \NewLine \.
		\forall A  : \prod^n_{i=1} \L(V_i;k) \. 
		\bigotimes^n_{i=1} A_i : \TYPE{Nondegenerate}\left( k,  \bigotimes^n_{i=1} V_i \right) \iff
		\forall i \in n \. A_i : \TYPE{Nondegenerate}(k, V_i)  
	}
	\NoProof
}
\Page{
	\Theorem{DualTensorProduct}
	{
		\forall k : \Field \.
		\forall V,W \in \VS{k} \.
		\big(V \otimes W\big)^* \cong_{\VS{k}} V^* \otimes W^*
	}
	\NoProof
	\\
	\Theorem{SelfdualTensorProduct}
	{
		\forall k : \Field \.
		\forall V \in \VS{k} \.
		\big(V \otimes V^*\big)^* \cong_{\VS{k}} V^* \otimes V
	}
	\NoProof
	\\
	\Theorem{TensorProductReflexive}
	{
		\forall k : \Field \.
		\forall V \in \VS{k} \.
		\big(V \otimes V^*\big)^* \cong_{\VS{k}} V^* \otimes V
	}
	\NoProof
	\\
	\Theorem{DulMappingTensorProduct}
	{
		\forall k : \Field \.
		\forall V,W,X,Y \in \VS{k} \.
		\forall f : V \Arrow{\VS{k}} W \.
		\forall g : X \Arrow{\VS{k}} Y \. \NewLine \. 
		\big(f \otimes g\big)^* \cong_{\VS{k}} f^* \otimes g
	}
	\NoProof
	\\
	\DeclareFunc{compositionAlgebra}{\prod R \in \ANN \. \LMOD{R} \to \LALG{R}}
	\DefineNamedFunc{compositionAlgebra}{V}{\mathsf{CA}(V)}
	{\NewLine \de\Big(V \otimes V^*,\cdot,+, \FUNC{tensorisation}\Lambda v \otimes f, w \otimes g \in V \otimes V^* \. f(w) (v \otimes g) \Big)}
	\\
	\DeclareFunc{asOperator}{\prod R \in \ANN \. \prod V \in \LMOD{R} \. \mathsf{CA}(V) \Arrow{\LALG{R}} \End_{\LMOD{R}} }
	\DefineFunc{asOperator}{\sum^n_{i=1} v_i \otimes f_i}{\Lambda w \in V \. \sum^n_{i=1} f_i(w)v_i}
	\\
	\Theorem{InclusionOdCompositionAlgebras}{\forall k : \Field  \. \forall V : \VS{k} \. \FUNC{asOperator} : \mathsf{CA}(V) \ToInj \End_{\LMOD{R}}}
	\NoProof
	\\
	\Theorem{CompositionAlgebraIsOperators}{
		\forall k : \Field  \. 
		\forall V : \VS{k} \. 
		\forall [0] : \dim V < \infty
		\FUNC{asOperator} : \mathsf{CA}(V) \ToIso{\LALG{k}} \End_{\LMOD{R}}
		}
	\NoProof
	\\
	\Theorem{CompositionAlgebraIsNotOperators}{
		\forall k : \Field  \. 
		\forall V : \VS{k} \. 
		\forall [0] : \dim V = \infty \. 
		\id \not \in \FUNC{asOperatoe}(\mathsf{CA}(V))
		}
	\NoProof
}
\newpage
\section{Tensorial Algebras}
\subsection{Tensor Algebra}
\Page{
	\DeclareType{TensorAlgebra}{\prod R \in \ANN \. \prod M \in \LMOD{R} \. ?\sum T : \LALGE{R} \.  M  \Arrow{\LMOD{R}} T }
	\DefineType{(T,\iota)}{TensorAlgebra}{\forall A \in \LALGE{R} \. \forall \varphi : M \LMOD{R} A \. \exists! f : T \Arrow{\LALGE{R}} A : \varphi = \iota f  }
	\\
	\Theorem{IsomorphicTensorAlgebras}{\forall R \in \ANN \. \forall M \in \LMOD{R} \.  \forall (T,\iota),(T',\iota') : \TYPE{TensorAlgebra}(M) \. 
		\NewLine \. T \cong_{\LALGE{R}} T'}
	\NoProof
	\\
	\Theorem{TensorAlgebraUniverslInjective}{\forall R \in \ANN \. \forall M \in \LMOD{R} \. \forall (T,\iota) : \TYPE{TensorAlgebra}(M) \. \iota : M \ToInj T }
	\Say{A}{\FUNC{leggedAlgebra}(M)}{\LALGE{R}}
	\Say{\varphi}{\Lambda m \in M \. (0,m)}{M \Arrow{\LMOD{R}} A}
	\Say{(f,[1])}{\bd \TYPE{TensorAlgebra}(T,\iota)}{\sum f : T \Arrow{\LALGE{R}} A \. \iota f = \varphi}
	\Assume{m}{M}
	\Assume{[1]}{\varphi(m) = 0}
	\Conclude{[2.*]}{\ByConstr \varphi}{m = 0}
	\Derive{[2]}{\THM{ZeroKernelTHM}}{(\varphi : M \ToInj A)}
	\Conclude{[*]}{\bd \TYPE{MonoComp}[1][2]}{(\iota : M \ToInj T)}
	\EndProof
	\\
	\DeclareFunc{tensorAlgebra}{\LMOD{R} \to \LALGE{R}(\Int_+)}
	\DefineNamedFunc{tensorAlgebra}{M}{M^{\otimes}}{\left( \Int, \left(\bigoplus_{n=0}^\infty M^{\otimes n},\otimes \right),\Lambda n \in \Int_+ \. M^{\otimes n}\right) } 
	\\
	\DeclareFunc{tensorImbedding}{ \prod M \in \LMOD{R} \. M \Arrow{\LMOD{R}} M^\otimes}
	\DefineNamedFunc{tensorImbedding}{m}{\iota_{M^\otimes}(m)}{\Lambda n \in \Int_+ \.  \If n == 1 \Then}
	\\
	\DeclareFunc{tensorAlgebraMap}{\prod R \in \ANN \. \prod X,Y \in \LMOD{R} \.  X \Arrow{\LMOD{R}} Y \to  X^\otimes \Arrow{\LALGE{R}} Y^\otimes }
	\DefineNamedFunc{tensorAlgebraMap}{f}{f^\otimes}{\bd \TYPE{TensorAlgebra}(M^\otimes,\iota_{M^\otimes})(f)  }                                
	\\
	\DeclareFunc{tensorAlgebraFunctor}{\prod R \in \ANN \. \Cov(\LMOD{R},\LALGE{R})}
	\DefineFunc{tensorAlgebraFunctor}{}{ (\FUNC{tensorAlgebra}, \FUNC{tensorAlgebraMap}) }
}\Page{
	\Theorem{TensorAlgebraTheorem}{\forall R \in \ANN \. \forall M \in \LMOD{R} \.  M^\otimes : \TYPE{TensorAlgebra}(M)}
	\Assume{A}{\LALGE{R}}
	\Assume{\varphi}{M \Arrow{\LMOD{R}} A}
	\Assume{n}{\Nat}
	\Assume{m}{n \to M}
	\Conclude{f\left( \bigotimes^n_{i=1} m_i \right)}{\prod^n_{i=1} \varphi(m_i)}{M}
	\Derive{f}{\bd M^\otimes}{M^\otimes \Arrow{\LMOD{R}} A}
	\Say{[1]}{\bd \FUNC{tensorProduct}\ByConstr f}{ (f : M^\otimes \Arrow{\LMOD{R}} A) }
	\Say{[2]}{\bd \FUNC{tensorProduct}\ByConstr f}{ (f : M^\otimes \Arrow{\LALGE{R}} A) }
	\Say{[3]}{\ByConstr f \bd \iota_{M^\otimes}}{\iota_{M^\otimes} f = \varphi}
	\Assume{f'}{M^\otimes \Arrow{\LALGE{R}} A}
	\Assume{[4]}{\varphi = \iota_{M^\otimes} f'}
	\Conclude{[f'.*]}{\ByConstr f}{f = f'}
	\Derive{[*]}{\bd^{-1}\TYPE{TensorProduct}}{ \Big( M^\otimes : \TYPE{TensorAlgebra}\Big) }
	\EndProof
	\\
	\Theorem{TensorAlgebraKer}{ \forall X,Y \in \LMOD{R} \. \forall f : X \Arrow{\LMOD{R}} Y \. \ker f^\otimes = \langle \ker f \rangle_{Y^\otimes}}
	\NoProof
	\\
	\Theorem{TensorMapSurjective}{\forall X,Y \in \LMOD{R} \. \forall f : X \Arrow{\LMOD{R}} Y \. \big(f : X \ToSurj Y \big) \to (f^\otimes : X^\otimes \ToSurj Y^\otimes)}
	\NoProof
	\\
	\Theorem{FromBasisToTensorMapExtension}{\forall R \in \ANN \. \forall M \in \LMOD{R} \. \forall A \in \LALGE{R} \. \forall E : \TYPE{Basis}(M) \. 
		\. \NewLine \. \forall f : E \to A \.  \exists! f' : M^\otimes \Arrow{\LALGE{R}} A 
		}
	\NoProof
	\\
	\Theorem{TensorAlgebraBasis}{\forall R \in \ANN \. \forall M \in \LMOD{R} \. \forall E : \Basis(M) \. 
		\NewLine \. \left\{ \bigotimes^n_{i=1} e_i \Big| n \in \Nat, e : n \to E \right\} : \Basis(M^\otimes)}
	\NoProof
}\Page{
	\Theorem{FreeTensorAlgebra}{\forall R \in \ANN \. \forall M \in \FM(R) \.  M^\otimes : \TYPE{FreeAssociativeAlgebra}(R) } 
	\NoProof
	\\
	\Theorem{FlatTensorAlgebra}{\forall R \in \ANN \. \forall M \in \TYPE{Flat}(R) \. M^\otimes : \TYPE{Flat}(R)}
	\Say{\Big(Q,[1]\Big)}{\bd \TYPE{Flat}(R) }{ \sum Q \in \LMOD{R} \.  Q \oplus M : \FM(R)}
	\Say{[2]}{\THM{FreeTensorAlgebra}[1]}{\Big(\big( Q \oplus M \big)^\otimes : \FM(R)\Big)}
	\Say{\alpha}{\Lambda m \in M \. (0,m)}{M \Arrow{\LMOD{R}} Q \oplus M }
	\Say{\beta}{\Lambda (q,m) \in Q \oplus M \. m }{Q \oplus M \Arrow{\LMOD{R}} M}
	\Say{[3]}{\ByConstr \alpha \beta}{\alpha^\otimes\beta^\otimes = \id}
	\Say{[4]}{\THM{IsomorphismDecompTHM}[3] }{  (Q \oplus M)^\otimes \cong_{\LMOD{R}} \ker \beta^\otimes \oplus M^\otimes  }
	\Conclude{[*]}{\bd^{-1} \TYPE{Flat} [2][4]}{\big( M^\otimes : \TYPE{Flat}(R) \big)}
	\EndProof
	\\
	\Theorem{CovariantExtensionOfTensorAlgebra}{
		\forall R \in \ANN \. 
		\forall S \in \ANN \. 
		\forall \omega  : R \Arrow{\RING} S \. \NewLine \.  
		\forall M \in \LMOD{R} \.
		(M \otimes_\omega S)^{\otimes_S} \cong_{\LALGE{S}} M^{\otimes_R} \otimes_\omega S
	}
	\NoProof
	\\
	\Theorem{TensorAlgebraIsPolynomial}{\forall R \in \ANN \. \forall M \in \LMOD{R} \. M^\otimes : \PGA(R)}
	\NoProof
	\\
	\Theorem{DerivationTensorAlgebraExtension}{\forall R \in \ANN \. \forall M \in \LMOD{R} \.  \forall f \in M^* \. \NewLine \.  \exists! D \in \D(M^\otimes) \. D_{|M} = f}
	\NoProof
	\\
	\Theorem{SkewDerivationTensorAlgebraExtension}{\forall R \in \ANN \. \forall M \in \LMOD{R} \.  \forall f \in M^* \. \NewLine \.  \exists! D \in \widetilde{\D}(M^\otimes) \. D_{|M} = f}
	\NoProof
}
\Page{
	\DeclareType{PseudoCyclic}{\prod R \in \ANN \. ?\LMOD{R}}
	\DefineType{M}{PseudoCyclic}{\forall N \submod{R} M \. N : \FGM(R) \Rightarrow \exists Z \submod{R} M \. Z : \TYPE{CyclicModule}(R) \And N \submod{R} Z }
	\\
	\Theorem{CommutativeTensorAlgebra}{ \forall R \in \ANN \. \forall M : \TYPE{PseudoCyclic}(R) \.   M^\otimes \in \LCALGE{R}}
	\Assume{n,m}{\Int_+}
	\Assume{x}{n \to M}
	\Assume{y}{m \to M}
	\Say{N}{\Big\langle \{x_i | i \in n\} \cup \{y_i | i \in m \} \Big\rangle_{M}}{ \TYPE{Submodule}(M)}
	\Say{[1]}{\ByConstr N}{  \Big(N : \FGM(R)\Big)    }
	\Say{\big(Z,[2]\big)}{\bd \TYPE{PseusoCyclic}(M)(N,[1])}{ \sum Z \submod{R} M \. Z : \TYPE{CyclicModule}(R) \And   N \submod{R} Z }
	\Say{\big(z,[3]\big)}{\bd \TYPE{Cyclic}(Z)}{\sum z \in M \. Z = Rz}
	\Say{\big(\alpha,\beta,[4]\big)}{[3]\ByConstr N \bd x \bd y}{\sum \alpha : n \to R \. \sum \beta : m \to R \. x = \alpha z \And y = \beta z}
	\Conclude{[n.*]}{[4]\bd \THM{MultiHomogen}^{2(n+m)}[4] }{
		\NewLine :
		\bigotimes^n_{i=1} x_i \otimes \bigotimes^m_{j=1} y_j = 
		\bigotimes^n_{i=1} \alpha_i z \otimes \bigotimes^m_{j=1} \beta_j z = 
		\prod^n_{i=1} \alpha_i \prod^m_{j=1} \beta_j \bigotimes^{n+m}_{i=1} z = 
		\bigotimes^m_{j=1} \beta_j z \otimes \bigotimes^n_{i=1} \alpha_i z =
		\bigotimes^m_{j=1} y_j \otimes \bigotimes^n_{j=1} x_j
	}
	\Derive{[1]}{\bd M^\otimes}{\forall t,s \in M^\otimes \. ts = st}
	\Conclude{[*]}{\bd \LCALGE{R}[1]}{ M^\otimes \in \LCALGE{R} }
	\EndProof
	\\
	\Theorem{TensorAlgebraOfIntegralDomain}{\forall R : \ID \. \forall M : \FM(R) \. M^\otimes : \ID }
	\Assume{n,m}{\Int_+}
	\Assume{x}{M^\otimes_n}
	\Assume{y}{M^\otimes_m}
	\Say{E}{\THM{FreeHasBasis}(M)}{\Basis(M)}
	\Say{(\alpha,[1])}{\THM{TensorAlgebraBasis}(M)(x)}{ \sum \alpha : (n \to E) \to R \. x = \sum_{e : n \to E} \alpha_e \bigotimes^n_{i=1} e_i }
	\Say{(\beta,[2])}{\THM{TensorAlgebraBasis}(M)(y)}{ \sum \alpha : (m \to E) \to R \. x = \sum_{e : m \to E} \beta_e \bigotimes^m_{i=1} e_i }
	\Say{[3]}{\bd \FUNC{tensorAlgebra}[1][2]}{x \otimes y = \sum_{e : m + n \to E} \alpha_{e_{|n}}\beta_{e_{+n}} \bigotimes^{n+m}_{i=1} e_i }
	\Conclude{[n.*]}{\bd \ID(R)[3]}{ x \otimes y = 0 \Rightarrow  x = 0 | y = 0   }
	\Derive{[4]}{\bd M^\otimes}{ \forall x, y \in M^\otimes \. x \otimes y = 0 \Rightarrow x = 0 | y = 0 }
	\Conclude{[*]}{\bd^{-1}\ID[4]}{\Big( M^\otimes : \ID \Big)}
	\EndProof
}
\Page{
	\Theorem{TensorAlgebraQuotient}{
		\forall R \in \ANN \. 
		\forall M \in \LMOD{R} \. 
		\forall I : \TYPE{Ideal}(R) \. 
		\frac{M^\otimes}{(IM)^\otimes} \cong_{\LALGE{\frac{R}{I}}}  
		\left( \frac{M}{IM} \right)^{\otimes_{\frac{R}{I}}}   
	}
	\Assume{[m]}{\frac{M}{IM}}
	\Conclude{\varphi[m]}{ [m]_{\frac{M^\otimes}{(IM)^\otimes}} }{\frac{M^\otimes}{(IM)^\otimes}} 
	\Derive{\varphi}{I(\to)}{\frac{M}{IM} \Arrow{\LMOD{\frac{R}{I}}} \frac{M^\otimes}{(IM)^\otimes}  }
	\Say{(f,[1])}{\bd \TYPE{TensorAlgebra}\left( \frac{M}{IM} \right)(\varphi)}{ \sum f : \left(  \frac{M}{IM}\right)^\otimes \Arrow{\LALGE{R}(\Int)} \frac{M^\otimes}{(IM)^\otimes} \. \varphi = \iota f }
	\Say{\psi}{\phi^{|\left(\frac{M^\otimes}{(IM)^\otimes} \right)_1 } }{  \frac{M}{IM} \Arrow{\LMOD{\frac{R}{I}}}  \frac{M^\otimes}{(IM)^\otimes}    }
	\Say{[2]}{\bd \psi \bd \phi}{ \psi : \TYPE{Surjective} }
	\Say{[3]}{\THM{SurjectiveTensorExtension}[2]}{(f : \TYPE{Surjective} )}
	\Assume{t}{\TYPE{Homogeneous}\left( \frac{M}{IM} \right)^\otimes}
	\Assume{[4]}{f(t) = 0}
	\Say{d}{\deg t}{\Int_+}
	\Say{(k,m,[5])}{\bd t}{\sum k \in \Nat \. m : k \to d \to M \ . t = \sum^k_{i=1} \bigotimes^d_{j=1} \big[m_{i,j}\big]  }
	\Say{(m',[6])}{[4][5]}{ \sum  m' : k \to d \to (IM)^\otimes \. \sum^k_{i=1} \bigotimes^d_{j=1} \Big( m_{i,j} +  m_{i,j}' \Big) \in (IM)^\otimes }
	\Say{[7]}{\THM{MultiAdditive}(\otimes)[6]}{\forall i \in k \. \forall j \in d \. m_{i,j} \in IM}
	\Conclude{[t.*]}{[5][7]}{t = 0}
	\DeriveConclude{[*]}{ \bd \TYPE{Isomorphic}\bd \TYPE{Iso}[3]\THM{ZeroKernelTHM}}{\LOGIC{This}}
	\EndProof
	\\
	\Theorem{DerivationTensorAlgebraExtension}{\forall R \in \ANN \. \forall M \in \LMOD{R} \. \forall n \in \Nat \.  \forall f \in M \to M^\otimes_n \. \NewLine \.  \exists! D \in \D^n(M^\otimes) \. D_{|M} = f}
	\NoProof
	\\
	\Theorem{SkewDerivationTensorAlgebraExtension}{\forall R \in \ANN \. \forall M \in \LMOD{R} \.  \forall f \in M^ \to M^\otimes_n \. \NewLine \.  \exists! D \in \widetilde{\D}^n(M^\otimes) \. D_{|M} = f}
	\NoProof
}
\newpage
\subsection{Mixed Tensor Algebra}
\Page{
	\DeclareFunc{mixedTensorAlgebra}{\prod R \in \ANN \. \LMOD{R} \to \LALGE{R}(\Int^2_+) }
	\DefineNamedFunc{mixedTensorAlgebra}{M}{M^{\otimes,*}}{M^\otimes \otimes \big(M^*\big)^\otimes}
	\\
	\DeclareFunc{totalDegree}{ \prod R \in \ANN \. \prod M \in \LMOD{R} \. \TYPE{Homogeneous}(M^{\otimes,*}) \to \Int_+}
	\DefineNamedFunc{totalDegree}{h}{\overline{\deg} h}{i + j \quad \where \quad (i,j) = \deg h}
	\\
	\DeclareType{DecomposableTensor}{\prod R \in \ANN \. \prod M \in \LMOD{R} \. ?M^{\otimes,*}}
	\DefineType{t}{DecomposableTensor}{\exists p,q \in \Int_+ : \exists m : M^p} 
	\\
	\DeclareFunc{contraction}{\prod R \in \ANN \. \prod M \in \LMOD{R} \. \prod p,q \in \Nat \. p \to q \to M^{\otimes,*}_{p,q} \to M^{\otimes,*}_{p-1,q-1} }
	\DefineNamedFunc{contraction}{k,l, t}{\tr_{k,l} t}{ \FUNC{tensorize}\left(\Lambda  v \in M^p \. \Lambda f \in (M^*)^q  \. 
		f_j(v_i)\bigotimes^{p-1}_{i=1} \widehat{v}_{k,i}   \otimes \bigotimes^{1-1}_{j=1} \widehat{f}_{l,j} \right)  }
	\\
	\Theorem{BasisContraction}{
			\forall R \in \ANN \. 
			\forall M \in \LMOD{R} \. 
			\forall E : \TYPE{Basis}(M) \. 
			\forall p,q \in \Nat \. \NewLine \. 
			\forall \sum_{e \in E^p} \sum_{f \in (E^*)^q} \alpha_{e,f} \bigotimes^p_{i=1} e_{i} \otimes \bigotimes^q_{i=1} f_{i} \.
			\forall i \in p \.
			\forall j \in q \. \NewLine \. 
			\tr_{i,j} \alpha_{t,s} \bigotimes^p_{i=1} e_{t_i} \otimes \bigotimes^q_{i=1} e^*_{s_i} =
			\sum_{e' \in E^{p-1}} \sum_{f' \in (E^*)^{q-1}} \left( \sum e \in E_p \. \sum f \in E^*_q \. e_i^* = f_j \And \hat{e}_i = e' \And \hat{f}_j = f' \. \alpha_{e,f} \right)                    
		 	\NewLine  \bigotimes^p_{i=1} e_{i}' \otimes \bigotimes^q_{i=1} f_{i}' 
		}
	\NoProof
	\\
	\DeclareFunc{mixedTensorMap}{\prod R \in \ANN \. \prod M,N \in \LMOD{R} \. \Big(M \ToIso{\LMOD{R}} N) \to (M^{\otimes,*} \Arrow{ \LALGE{R}} N^{\otimes,*}\Big)}
	\DefineNamedFunc{mixedTensorMap}{T}{T^{\otimes,*}}{ T^\otimes \otimes \big(T_*^{-1}\big)^\otimes  }
	\\
	\Theorem{MixedTensorFunctor}{\forall R \in \ANN \. (\FUNC{mixedTensorAlgebra},\FUNC{mixedTensorMap}) :  
		\NewLine : \Cov\Big( \FUNC{groupoid}(\LMOD{R}),\LALGE{R}\Big)(\Int_+^2) }
	\NoProof
	\\
	\DeclareType{Tensorial}{\prod R \in \ANN \. \prod M \in \LMOD{R} \.  \prod a,b,p,q \in \Int_+ \. ? M^{\otimes,*}_{a,b} \Arrow{\LMOD{R}} M^{\otimes,*}_{p,q}}
	\DefineType{T}{Tensorial}{\forall A \in \Aut_{\LMOD{R}}(M) \. TM^{\otimes,*} = M^{\otimes,*}T}
}
\newpage
\subsection{Exterior Algebra}
\Page{
	\DeclareFunc{alternator}{ \prod R \in \ANN \. \prod M \in \LMOD{R} \. M^\otimes \Arrow{\LMOD{R}} M^\otimes}
	\DefineNamedFunc{alternator}{ t }{ \wedge(t) }{ \bd \TYPE{TensorAlgebra} \Lambda k \in \Nat \. \lambda m \in M^k \. \sum_{\sigma \in S^n} (-1)^\sigma \bigotimes^k_{i=1} m_{\sigma(i)} } 
	\\
	\DeclareFunc{exteriorPower}{\prod R \in \ANN \. \LMOD{R} \to \Int_+ \to \LMOD{R}}
	\DefineNamedFunc{exteriorPower}{ M,n  }{ M^{\wedge n} }{ \frac{M^\otimes_n}{\ker \wedge}   }
	\\
	\DeclareFunc{exteriorAlgebra}{\prod R \in \ANN \. \LMOD{R}  \to \TYPE{SkewAlgebra}{R}(\Int_+)}
	\DefineNamedFunc{exteriorAlgebra}{ M  }{ M^{\wedge } }{ \frac{M^\otimes}{M^\otimes\{ a \otimes a\} M^\otimes} }
	\\
	\DeclareType{ExteriorAlgebra}{\prod R \in \ANN \. \prod M \in \LMOD{R} \. ?\sum E : \TYPE{SkewAlgebra}(R) \.  M  \Arrow{\LMOD{R}} E }
	\DefineType{(E,\iota)}{ExtoriarAlgebra}{\forall A : \TYPE{SkewAlgebra}(R) \. \forall \varphi : M \LMOD{R} A \. \exists! f : T \Arrow{\LALGE{R}} A : \varphi = \iota f  }
	\\
	\Theorem{IsomorphicExteriorAlgebras}{\forall R \in \ANN \. \forall M \in \LMOD{R} \.  \forall (E,\iota),(E',\iota') : \TYPE{ExteriorAlgebra}(M) \. 
		\NewLine \. T \cong_{\LALGE{R}} T'}
	\NoProof
	\\
	\Theorem{ExteriorAlgebraUniversalInjective}{\forall R \in \ANN \. \forall M \in \LMOD{R} \. \forall (E,\iota) : \TYPE{TensorAlgebra}(M) \. \NewLine \. \iota : M \ToInj T }
	\NoProof
	\\
	\DeclareFunc{exteriorImbedding}{ \prod M \in \LMOD{R} \. M \Arrow{\LMOD{R}} M^\wedge}
	\DefineNamedFunc{exteriorImbedding}{m}{\iota_{M^\wedge}(m)}{\Lambda n \in \Int_+ \.  \If n == 1 \Then m \Else 0}
	\\
	\Theorem{ExteriorAlgebraTheorem}{\forall R \in \ANN \. \forall M \in \LMOD{R} \. M^{\wedge} : \TYPE{ExteriorAlgebra}(M)}
	\NoProof
	\\
	\DeclareFunc{exteriorAlgebraMap}{\prod R \in \ANN \. \prod X,Y \in \LMOD{R} \.  X \Arrow{\LMOD{R}} Y \to  X^\wedge \Arrow{\LALGE{R}} Y^\wedge }
	\DefineNamedFunc{exteriorAlgebraMap}{f}{f^\wedge}{\bd \TYPE{ExteriorAlgebra}(M^\wedge,\iota_{M^\wedge})(f)  }                                
	\\
	\DeclareFunc{tensorAlgebraFunctor}{\prod R \in \ANN \. \Cov(\LMOD{R},\LALGE{R})}
	\DefineFunc{tensorAlgebraFunctor}{}{ (\FUNC{exteriorAlgebra}, \FUNC{exteriorAlgebraMap}) }
}
\Page{
	\DeclareFunc{exteriorProduct}{\prod R \in \ANN \. \prod M \in \LMOD{R} \. \L\Big( M^\wedge , M^\wedge \Big)} 
	\DefineNamedFunc{exteriorProduct}{t,s}{t \wedge s}{ \wedge(t \otimes s)}
	\\
	\Theorem{TensorAlgebraKer}{ \forall X,Y \in \LMOD{R} \. \forall f : X \Arrow{\LMOD{R}} Y \. \ker f^\wedge = \langle \ker f \rangle_{Y^\wedge}}
	\NoProof
	\\
	\Theorem{TensorMapSurjective}{\forall X,Y \in \LMOD{R} \. \forall f : X \Arrow{\LMOD{R}} Y \. \big(f : X \ToSurj Y \big) \to (f^\otimes : X^\wedge \ToSurj Y^\wedge)}
	\NoProof
	\\
	\Theorem{ExteriorAlgebraBasis}{\forall R \in \ANN \. \forall M \in \LMOD{R} \. \forall E : \Basis(M) \. 
		\NewLine \. \left\{ \bigwedge^n_{i=1} e_i \Big| n \in \Nat, e : \TYPE{Injective} \And \TYPE{Ascending}\Big(n,(E,o) \Big) \right\} : \Basis(M^\otimes)
		\NewLine \where \quad o = \bd \THM{WellOrderingPrinciple}(E)}
	\NoProof
	\\
	\Theorem{FreeExteriorAlgebra}{\forall R \in \ANN \. \forall M \in \FM(R) \. M^\wedge : \FM(R) }
	\NoProof
	\\
	\Theorem{ExteriorHPAlgebraRank}{\forall R \in \ANN \. \forall M \in \FM(R) \. \forall n,k \in \Int_+ \. \NewLine \. \forall [0] : \rank M = n \. \rank  M^\wedge_k = C^k_n }
	\NoProof
	\\
	\Theorem{ExteriorAlgebraRank}{\forall R \in \ANN \. \forall M \in \FM(R) \. \forall n \in \Int_+ \. \NewLine \. \forall [0] : \rank M = n \. \rank  M^\wedge = 2^n }
	\NoProof
}
\Page{
	\\
	\Theorem{ExteriorProductDirectSum}{
		\forall R \in \ANN \.
		\forall n \in \Nat \.
		\forall M : n \to \LMOD{R} \. \NewLine \, 
		\left( \bigoplus^n_{i=1} M_i \right)^\wedge \cong_{\LALGE{R}(\Int_+)}
		\widetilde{\bigotimes}_{i=1}^n M_i^\wedge
	}
	\Assume{A}{\LMOD{R}}
	\Assume{B}{\LMOD{B}}
	\Assume{a}{A}
	\Assume{b}{B}
	\Conclude{\varphi(a,b)}{  a \otimes 1 + 1 \otimes b}{  A^\wedge \widetilde{\otimes} B^\wedge}
	\Derive{\varphi}{\bd \TYPE{ExteriorAlgebra}(A \oplus B)}{ (A \oplus B)^\wedge \Arrow{\LALGE{R}} A^\wedge \widetilde{\otimes} B^\wedge   }
	\Say{T}{  \pi_A^\wedge  \wedge \pi_B^\wedge  }{ \L\left( A^\wedge, B^\wedge  ; (A \oplus B )^\wedge  \right)}
	\Say{\psi}{\FUNC{tensorize}(T)}{  A^\wedge \widetilde{\otimes} B^\wedge \Arrow{\LMOD{R}} (A \oplus B)^\wedge  }
	\Assume{n,m}{\Int_+}
	\Assume{t}{n \to \TYPE{Homogeneous}(A^\wedge)}
	\Assume{t'}{m \to \TYPE{Homogeneous}(A^\wedge)}
	\Assume{s}{n \to \TYPE{Homogeneous}(B^\wedge)}
	\Assume{s'}{m \to \TYPE{Homogeneous}(B^\wedge)}
	\Say{k}{ i,j \mapsto \deg s_i \deg t'_j}{ n \times m \to \Int }
	\Say{[1]}{\bd \FUNC{skewTensorProduct} \ByConstr \psi    }
	{
		\psi\Big( (t_i \otimes s_i) (t_j' \otimes s'_j) \Big) = 
		\psi( (-1)^{k_{i,j}}t_i \wedge t'_j \otimes s_i \wedge s'_j ) = \NewLine =
		(-1)^{k_{i,j}} \bigwedge^{\deg t_i}_{l=1} (t_{i,l},0) \wedge \bigwedge^{\deg t'_j}_{l=1} (t_{j,l}',0) 
		\wedge
		\bigwedge^{\deg s_i}_{l=1} (0,s_{i,l}) \wedge \bigwedge^{\deg s'_j}_{l=1} (0,s_{j,l}') 
	}
	\Say{[2]}{\ByConstr \psi \bd \FUNC{exteriorProduct}   }
	{
		\psi (t_i \otimes s_i) \psi (t_j' \otimes s'_j)  = 
		\bigwedge^{\deg t_i}_{l=1} (t_{i,l},0) \wedge \bigwedge^{\deg s_i}_{l=1} ( 0,s_{i,l}) 
		\wedge
		\bigwedge^{\deg t_j'}_{l=1} (t'_{j,l},0) \wedge \bigwedge^{\deg s'_j}_{l=1} (0,s_{j,l}') = \NewLine =
		(-1)^{k_{i,j}} \bigwedge^{\deg t_i}_{l=1} (t_{i,l},0) \wedge \bigwedge^{\deg t'_j}_{l=1} (t_{j,l}',0) 
		\wedge
		\bigwedge^{\deg s_i}_{l=1} (0,s_{i,l}) \wedge \bigwedge^{\deg s'_j}_{l=1} (0,s_{j,l}') 
	}
	\Conclude{[*]}{[1][2]}{\psi\Big((t_i \otimes s_i)(t_j \otimes s_j)\Big) = \psi(t_i \otimes s_i)\psi(t_j' \otimes s_j')}
	\Derive{[1]}{\bd \LALGE{R}}{ \psi : A^\wedge  \widetilde{\otimes} B^\wedge  \Arrow{\LALGE{R}} (A \oplus B)^\wedge }
	\Assume{(a,b)}{A \oplus B }
	\Say{[(a,b).*]}{ \ByConstr \varphi \ByConstr \psi \bd \FUNC{directSum} }{ \psi \varphi(a,b) = \psi( a \otimes 1 + 1 \otimes b  ) = (a,0) + (0,b) = (a,b)   }
	\Say{(a,b).*}{  \ByConstr \psi \ByConstr \varphi \bd \FUNC{tensorProduct} }{\varphi \psi (a \otimes 1)  =  \varphi (a,0) =  a \otimes 1 + 1 \otimes  0 = a \otimes 1  }
	\Conclude{(a,b),*}{\ByConstr \psi \ByConstr \varphi \bd \FUNC{tensorProduct}}{ \varphi \psi (1 \otimes b) = \varphi (0,b) = 0 \otimes a + 1 \otimes b = 1 \otimes b     } 
	\Derive{[2]}{\bd \FUNC{exteriorAlgebra} \bd \FUNC{tensorProduct}}{  \varphi \psi = \id  \And \psi \varphi = \id }
	\Conclude{[*]}{\bd \TYPE{Isomotphic}[2]}{\LOGIC{This}}
	\EndProof
}\Page{
	\Theorem{CovariantExtensionOfExteriorAlgebra}{
		\forall R \in \ANN \. 
		\forall S \in \ANN \. 
		\forall \omega  : R \Arrow{\RING} S \. \NewLine \.  
		\forall M \in \LMOD{R} \.
		(M \otimes_\omega S)^{\wedge_S} \cong_{\LALGE{S}} M^{\wedge_R} \otimes_\omega S
	}
	\NoProof
	\\
	\Theorem{ExteriorAlgebraIsPolynomial}{\forall R \in \ANN \. \forall M \in \LMOD{R} \. M^\wedge : \PGA(R)}
	\NoProof
	\\
	\Theorem{SkewDerivationExteriorAlgebraExtension}{\forall R \in \ANN \. \forall M \in \LMOD{R} \.  \forall f \in M^* \. \NewLine \.  \exists! D \in \widetilde{\D}(M^\wedge) \. D_{|M} = f}
	\NoProof
	\\
	\DeclareFunc{skewExtension}{\prod R \in \ANN \. \prod M \in \LMOD{R} \. M^* \Arrow{\LMOD{R}} \widetilde{D}(M^\wedge)}
	\DefineNamedFunc{skewExtension}{f }{D_f}{\bd \THM{SkewDerivarionExteriorAlgebraExtension}(f)}
	\\
	\DeclareFunc{skewBilinearExtension}{\prod R \in \ANN \. \prod M,U \in \LMOD{R} \. \L(U,M;R) \to U \to \widetilde{D}(M^\wedge) } 
	\DefineNamedFunc{skewBilinearExtension}{ T,u }{D_{T,u}}{D_f \quad \where \quad f = \Lambda m \in M \. T(u,m)}
	\\
	\Theorem{skewBilinearExteriorAsComp}
	{
		\forall R \in \ANN \. 
		\forall M,U \in \LMOD{R} \.
		\forall T \in  \L(U,M;R) \. 
		\forall n \in \Nat \. 
		\forall u : n \to U \. \NewLine \.  
		D_T^{\wedge} \bigwedge^n_{i=1} u_i =  
		\prod^{n-1}_{i=0} D_{T,u_{n-i}}
	}
	\NoProof
	\\
	\Theorem{skewBilinearExteriorAppByDet}
	{
		\forall R \in \ANN \.
		\forall M,U \in \LMOD{R} \.
		\forall T \in \L(U,M;R) \. 
		\forall n \in \Nat \. 
		\forall m \in n \. \NewLine \. 
		\forall u : m \to U \.
		\forall v : n \to M \. 
		D_T^\wedge \left( \bigwedge^m_{i=1} u_i \right)\left( \bigwedge^n_{i=1} v_i \right) 
		= (-1)^m \sum_{ k: m \to n} (-1)^{|k|} \det \Lambda i,j \in m \. T(u_{i},v_{k_i}) 
	}
	\NoProof
	\\
	\Theorem{skewBilinearExteriorAppByDet2}
	{
		\forall R \in \ANN \.
		\forall M,U \in \LMOD{R} \.
		\forall T \in \L(U,M;R) \. 
		\forall n \in \Nat \. 
		 \NewLine \. 
		\forall u : n \to U \.
		\forall v : n \to M \. 
		D_T^\wedge \left( \bigwedge^n_{i=1} u_i \right)\left( \bigwedge^n_{i=1} v_i \right) 
		= (-1)^{\frac{n(n+3)}{2}}   \det \Lambda i,j \in n \. T(u_{i},v_{i}) 
	}	
	\NoProof
}\Page{
	\Theorem{ExteriorAlgebraQuotient}{
		\forall R \in \ANN \. 
		\forall M \in \LMOD{R} \. 
		\forall I : \TYPE{Ideal}(R) \. 
		\frac{M^\wedge}{(IM)^\wedge} \cong_{\LALGE{\frac{R}{I}}}  
		\left( \frac{M}{IM} \right)^{\wedge_{\frac{R}{I}}}   
	}
	\\
	\Theorem{SkewDerivationExteriorAlgebraExtension}{
			\forall R \in \ANN \. 
			\forall M \in \LMOD{R} \. 
			\forall n \in \Nat \.
			\forall f \in M^{\wedge n*}   \. \NewLine \.  
			\exists! D \in \widetilde{\D}^n(M^\wedge) \. D_{|M^{\wedge n}} = f
			}
	\NoProof
	\\
	\DeclareType{Decomposable}{\prod R \in \ANN \. \prod M \in \LMOD{R} \. ?M^\wedge }
	\DefineType{x}{Decomposable}{\exists p \in \Nat : \exists m: p \to M \. x = \bigwedge^p_{i=1} m_i }
	\\
}
\newpage
\subsection{Determinant Identities}
\Page{
	\Theorem{DeterminantTHM}{\forall k : \TYPE{Field} \. \forall V \in \FDVS{k} \, \forall T \in \End_{\VS{k}}(V) \. 
		\forall e : \TYPE{Basis}(V,\dim V) \.  \NewLine \.  
		T^\wedge\left(\bigwedge^n_{i=1} e_i\right) = \det T \bigwedge^n_{i=1} e_i
	}
	\NoProof
	\DeclareFunc{complementaryIncreaingSequance}{ \prod n \in \Nat \. \TYPE{Increasing}(n,2n) \to \TYPE{Increasing}(n,2n)}
	\DefineNamedFunc{complementatyIncreasingSequance}{I}{I^\c}{ \Lambda i \in n \. \If i = 1 \Then u(1) \Else u\Big(I^\c(n-1)+1\Big) 
		\NewLine \quad \where \quad      
		u = \Lambda i \in 2n \. \If i \in \im I \Then u(i+1) \Else  i
	}
	\\
	\DeclareType{IndependentIncreasing}{\prod n \in \Nat \. ?\TYPE{Increasing}^2(n,2n)}
	\DefineNamedType{I,J}{IndependentIncreasing}{I \bot J}{\im I \cap \im J = \emptyset}
	\\
	\DeclareFunc{independentAsPermutation}{ \prod n \in \Nat \. \TYPE{IndependentIncreasing}(n) \to S_{2n}}
	\DefineNamedFunc{independentAsPermutation}{I,J}{I \oplus J}{ \Lambda i \in 2n \. \If i \le n  \Then I(n) \Else J(i-n)}
	\\
        \Theorem{IndependentComplements}{\forall n \in \Nat \. \forall I : n \uparrow 2n \. (I,I^\c) : \TYPE{IndependentIncreasing}(n)}
	\NoProof
	\\
	\Theorem{LaplaceDeterminantIdentity}
	{
		\forall R \in \ANN \.
		\forall n \in \Nat \. 
		\forall A \in R^{2n \times n} \. \NewLine \. 
		\sum_{I :  n \uparrow 2n }  (-1)^{I \oplus I^\c} \det (A_{I_i,j})^n_{i,j=1} \det (A_{I_i^\c,j})^n_{i,j=1} = 0 
	}
	\Say{K}{ \sum_{I :n \uparrow 2n} (-1)^{I \oplus I^\c}  }{\Int}
	\Say{[1]}
	{
		\bd \LMOD{R}(M^\wedge) \bd M^\wedge \;
		\THM{determinantTHM}
		\bd^{-1} \FUNC{ExteriorAlgebraFunctor} \NewLine
		\bd \LALGE{R}(M^\wedge,M^\wedge)(A\oplus A)^\wedge_e
		\bd \LMOD{R}(M^\wedge,M^\wedge)
		\ByConstr^{-1} K \;
		\THM{DeterminantTHM} \;
		\THM{SingularDeterminantTHM} 
	}
	{
		\NewLine : 
		\left(\sum_{I :  n \uparrow 2n }  (-1)^{I \oplus I^\c} \det (A_{I_i,j})^n_{i,j=1} \det (A_{I_i^\c,j})^n_{i,j=1}\right) \bigwedge^{2n}_{i=1} e_i = 
		\sum_{I :  n \uparrow 2n }  (-1)^{I \oplus I^\c} \det (A_{I_i,j})^n_{i,j=1} \det (A_{I_i^\c,j})^n_{i,j=1} \bigwedge^{2n}_{i=1} e_i = \NewLine =  
		\sum_{I :  n \uparrow 2n }   \det (A_{I_i,j})^n_{i,j=1} \bigwedge^n_{i=1} e_{I_i} \wedge \det (A_{I_i^\c,j})^n_{i,j=1} \bigwedge^{n}_{i=1} e_{I^\c_i} = 	
		\sum_{I :  n \uparrow 2n }   A_{e_I}^\wedge \bigwedge^n_{i=1} e_{I_i} \wedge A_{e_{I^\c}}^\wedge \bigwedge^{n}_{i=1} e_{I^\c_i} = \NewLine =  
		\sum_{I :  n \uparrow 2n }   (A \oplus A )^\wedge_e \bigwedge^n_{i=1} e_{I_i} \wedge (A \oplus A)^\wedge_e \bigwedge^{n}_{i=1} e_{I^\c_i} =  
		\sum_{I :  n \uparrow 2n }   (A \oplus A)^\wedge_e \left( \bigwedge^n_{i=1} e_{I_i}  \wedge  \bigwedge^{n}_{i=1} e_{I^\c_i} \right) =  
		\NewLine =  
		(A \oplus A)_e^\wedge  \left( K  \bigwedge^{2n}_{i=1} e_{ i } \right) = 
		\det (A \oplus A)  K \bigwedge^{2n}_{i=1} e_i
		 = 0
	}
	\Conclude{[*]}{\bd \LMOD{R}(M^\wedge)[1]}{  \sum_{I :  n \uparrow 2n }  (-1)^{I \oplus I^\c} \det (A_{I_i,j})^n_{i,j=1} \det (A_{I_i^\c,j})^n_{i,j=1} = 0} 
	\EndProof  
}
\Page{
	\Theorem{IncreasingSwapLemma}{\forall n : \TYPE{Even} \. \forall I : n \uparrow 2n \. (-1)^{I \oplus I^\c} = (-1)^{I^\c \oplus I}}
	\Say{\sigma}{\prod^n_{k=1} (k, n-k+1)}{S_{2n}}
	\Say{[1]}{\ByConstr \sigma \bd \FUNC{independentAsPermutation}}{ I \oplus I^\c \sigma = I^\c \oplus I }
	\Say{[2]}{ \bd \TYPE{Even}\bd \THM{SignByTransposition}(\sigma)}{(-1)^\sigma = 1}
	\Conclude{[*]}{\THM{SignIsHomo}[1][2]}{(-1)^{I^\c \oplus I} = (-1)^{\sigma I \oplus I^\c} = (-1)^\sigma (-1)^{I \oplus I^\c} = (-1)^{I \oplus I^\c}}
	\EndProof
	\\
	\Theorem{SpecialLaplaceDeterminantIdentity}
	{
		\forall R \in \ANN \.
		\forall n : \TYPE{Even} \. 
		\forall k \in n \. 
		\forall A \in R^{2n \times n} \.
		\NewLine \. 
		\sum_{I :  n \uparrow 2n : k \in \im I }  (-1)^{I \oplus I^\c} \det (A_{I_i,j})^n_{i,j=1} \det (A_{I_i^\c,j})^n_{i,j=1} = 0 
	}
	\Say{(s,I)}{\FUNC{enumerate}\{ I : n \uparrow 2n : k \in \im I  \}}{\sum s \in \Nat \. s \to n \uparrow 2n }
	\Say{[1]}{ \THM{LaplaceDeterminantIdentity}(R,n,A)\ByConstr^{-1} I \bd I^\c  \THM{IncreasingSwapLemma}(n,I) }
	{
		\NewLine :
		0 = \sum_{I :  n \uparrow 2n }  (-1)^{I \oplus I^\c} \det (A_{I_i,j})^n_{i,j=1} \det (A_{I_i^\c,j})^n_{i,j=1} = \NewLine = 
		\sum^s_{t=1} (-1)^{I_t \oplus I^\c_t}\det (A_{I_{t,i},j})^n_{i,j=1} \det (A_{I_{t,i}^\c,j})^n_{i,j=1} 
		-
		\sum^s_{t=1} (-1)^{I_t^\c \oplus I_t}\det (A_{I_{t,i},j})^n_{i,j=1} \det (A_{I_{t,i}^\c,j})^n_{i,j=1} = \NewLine = 
		2\sum^s_{t=1} (-1)^{I_t \oplus I^\c_t}\det (A_{I_{t,i},j})^n_{i,j=1} \det (A_{I_{t,i}^\c,j})^n_{i,j=1} 
	}
	\Conclude{[*]}{\text{Argue for the ring $\Int[\Int_+^{n \times n}]$ and apply homomorphism $x_{i,j} \mapsto A_{i,j}$}[1]}
	{\LOGIC{This} }
	\EndProof
	\\
	\Theorem{DeterminantPermutation}{ 
		\forall R \in \ANN \.
		\forall n \in \Nat \.
		\forall I : n \uparrow 2n \.
		\forall A : R^{2n \times 2n} \.
		 \det A_{I \oplus I^\c,\cdot} = (-1)^{I \oplus I^\c} \det A
	}
	\NoProof
	\\
	\DeclareFunc{Antiminor}{\prod R \in \ANN \. \prod n \in \Nat \. R^{2n \times 2n} \to (n \uparrow 2n)^2 \to R}
	\DefineNamedFunc{Antiminor}{A,I,J}{\Gamma_{I,J}(A)}{ (-1)^{|I| + |J|} \det A_{I^\c,J^\c}  }
	\\
	\Theorem{AntiminorSummationTHM}
	{
		\forall R \in \ANN \. 
		\forall n \in \Nat \. 
		\forall A \in R^{2n \times 2n} \.
		\forall k \in n \.
		\forall I,J  :  (k \uparrow n) \. \NewLine \. 
		\sum_{K : n - k \uparrow n} \Gamma_{I,K}(A) \det A_{J,K}   = 
		\If I == J \Then \det A \Else 0
	}
	\NoProof
	\\
	\DeclareFunc{exteriorPowerOfTheMatrix}
	{
		\prod R \in \ANN \.
		\prod n \in \Nat \.
		\prod k \in n \. 
		R^{n \times n} \to R^{\frac{n!}{k!(n-k)!}\times\frac{n!}{k!(n-k)!}} 
	}
	\DefineNamedFunc{exteriorPowerOfTheMatrix}{A}{A^{\wedge k}}
	{
		\Big(A_{e,e}^{\wedge k} \Big)^{e^{\wedge k},e^{\wedge k }}
	}
}\Page{
	\Theorem{ExteriorDeterminantMult}
	{
		\forall R \in \ANN \.
		\forall n \in \Nat \.
		\forall A \in R^{n \times n} \.
		\forall k \in n \.
		\exists U \in R^{\frac{n!}{k!(n-k)!}\times \frac{n!}{k!(n-k)!}}  \. \NewLine \.  
		U A^{\wedge k} = \det A I = A^{\wedge k} U
	}
	\Assume{I}{k \uparrow n}
	\Conclude{[I.*]}{\bd \FUNC{ExteriorAlgebraFunctor}}{A^{\wedge k}_{e,e} \bigwedge^k_{i=1}e_{I_i} = \sum_{J: k \to 2n}\det A_{I,J} \bigwedge^{k}_{i=1} e_{J_i} }
	\Derive{[1]}{\bd \FUNC{exteriorPowerOfTheMatrix}}{ \forall I,J : k \uparrow n \.  A^{\wedge k}_{I,J} = \det A_{I,J}}
	\Say{U}{\Lambda I,J : k \uparrow n \. \Gamma_{I,J}(A)}{ R^{\frac{n!}{k!(n-k)!}\times \frac{n!}{k!(n-k)!}} }
	\Conclude{[*]}{\THM{AntiminorSummationTHM}\ByConstr U [1]}{\LOGIC{This}}
	\EndProof
	\\
	\Theorem{IrreducibleDeterminant}
	{
		\forall n \in \ANN \.
		\det (x_{i,j})^n_{i,j=1} : 
		\TYPE{Irreducible} \; \Int\Big[ \Int^{n \times n}_+ \Big] 
	}
	\NoProof
	\\
	\Theorem{ExteriorPowerDeterminant}{
		\forall R \in \ANN \. 
		\forall n \in \Nat \. 
		\forall A \in R^{n \times n} \. 
		\forall k \in n \. \NewLine \.  
		\det A^{\wedge k } =  (\det A )^{\frac{(n-1)!}{(k-1)!(n-k)!}} 
	}
	\Say{X(x)}{(x_{i,j})^n_{i,j=1}}{\bigg( \Int\Big[ \Int^{n\times n}_{+}\Big]\bigg)^{n \times n}}
	\Say{d(x)}{\det X(x)}{\Int\Big[\Int^{n\times n}_{+}\Big]} 
	\Say{(U,[1])}{ \THM{ExteriorDeterminantMult}\Big(X(x)\Big)^{\wedge k}}{
		\NewLine : 
		\sum U(x) \in \bigg(\Int\Big[ \Int^{n \times n}_{i=1}\Big]\bigg)^{\frac{n!}{(n-k)!k!}\times \frac{n!}{(n-k)!k!}}
		\. U(x) \Big( X(x) \Big)^{\wedge k} = \det X(x)  I
	}
	\Say{[2]}{ \THM{DetHomo}[1] }{ \det U(x) \det \Big(X(x)\Big)^{\wedge k} = \Big(\det X(x) \Big)^{\frac{n!}{k!(n-k)!}}  = \Big(d(x)\Big)^{\frac{n!}{k!(n-k!)}}  }
	\Say{[3]}{\THM{IrreducibleDeterminant}(n)}{(d(x) : \TYPE{Irreducible})}
	\Say{(p,q,s,[4])}{[2][3]}{  \sum p,q \in \Int_+ \. \sum s \in  \{1,-1\}  \. \NewLine \.  \det U(x) = sd^p(x) \And  \det X^{\wedge k}(x) = sd^q(x) \And p + q = \frac{n!}{k!(n-k!)}   }
	\Say{[5]}{\text{Use special dummy matrices $e_1 \mapsto \alpha e_1$ and $E$}}{ p = \frac{(n-1)!}{(n-k)!(k-1)!} \And s = 1 }
	\Conclude{[*]}{\text{Map dummy variable to the enties of $A$}[4][5]}{\LOGIC{This}}
	\EndProof
}
\newpage
\subsection{Interior Product}
\Page{
	\DeclareFunc{leftExteriorMult}{ 
		\prod R \in \ANN \. 
		\prod M \in \LMOD{R} \. 
		M \Arrow{\LMOD{R}} M^{\wedge} \Arrow{\LMOD{R}} M^\wedge 
	}
	\DefineNamedFunc{leftExteriorMult}
	{
		m,t
	}
	{
		L_m(t)
	}
	{
		m \wedge t
	}
	\\
	\DeclareFunc{dualExteriorApp}{ \prod R  \in \Ann \. \prod M \in \LMOD{R} \. \L( M^{*\wedge},M^\wedge;M^\wedge) }
	\DefineNamedFunc{dualExteriorApp}{ f,v  }{ f(v) }
	{
		\bd M^{\wedge}\bd M^{*\wedge}
		\Lambda f : \TYPE{Decomposable}(M^{*\wedge}) \.
		\Lambda v : \TYPE{Decomposable}(M^\wedge) \.
		\NewLine \. 
		\sum_{I : \deg f \uparrow \deg v}\sum_{\sigma \in S_{\deg f}} (-1)^{I^*\sigma} \prod^{n}_{i=1} f_i(v_{I_{\sigma(i)}}) \bigwedge^{\deg v - \deg f}_{i=1} v_{I^\c_i}                             
	}
	\\
	\DeclareFunc{interiorProduct}
	{
		\prod R \in \ANN \.
		\prod M \in \LMOD{R} \.
		M^\wedge \to ?\Big(M^{*\wedge} \Arrow{\LALGE{E}} M^{*\wedge}\Big) 
	}
	\DefineNamedFunc{interiorProduct}{a,f,v}{ \mathbf{i}_a(f)(v) }{ f(a \wedge v)  }  
	\\
	\Theorem{InteriorProductComposition}{  
		\forall R \in \ANN \.
		\forall M \in \LMOD{R} \.
		\forall a,b \in M^\wedge \.
		\mathbf{i}_{a \wedge b} = \mathbf{i}_b \circ \mathbf{i}_a
	}
	\Assume{f}{M^{*\wedge}}
	\Assume{v}{M^\wedge}
	\Conclude{[f.*]}{ \bd \mathbf{i}}
	{
		\mathbf{i}_{a \wedge b}(f)(v) = 
		f( a \wedge b \wedge v ) =
		\mathbf{i}_a(f)(b \wedge v  ) =
		\mathbf{i}_b\Big( \mathbf{i}_a(f)\Big)(v)
	}
	\DeriveConclude{[*]}{I(\to,=)}{\mathbf{i}_a\mathbf{i}_b = \mathbf{i}_{a \wedge b}}
	\EndProof
	\\
	\Theorem{NonAnnihilatingInteriorProductExists}
	{
		\forall k : \Field \.
		\forall V \in \VS{k} \. 
		\forall t \in V^{\star\wedge2} \.
		\exists x \in V :  
		\mathbf{i}_x(t) \neq 0           
	}
	\Say{E}{\THM{FreeHasBasis}(V^*)}{\TYPE{Basis}(V^*)}
	\Say{o}{\THM{WellOrderingTHM}(E)}{\TYPE{WellOrderingTHM}(E)}
	\Say{(\alpha,[1])}{\THM{ExteriorAlgebraBasis}(E)(t)} 
	{
		\alpha \in k^{\oplus E \times E} \. 
		t = \sum_{f \in E} \sum_{g >_o f} \alpha_{f,g} f \wedge g
	}
	\Say{f}{\min_o \{ f \in E : \exists g \in E \. \alpha_{f,g} \neq 0 \}}{E}
	\Say{(e,[2])}{\THM{CanonicalIsoTHM}(f)}{ \sum e \in V \. e^{**} = f^*}
	\Say{[3]}{\bd \mathbf{i}_e(t)[1][2] }{\mathbf{i}_e(t) = \sum_{g \in E} \alpha_{f,g} g}
	\Conclude{[*]}{\bd \TYPE{Basis}\ByConstr f [3]}{\mathbf{i}_e(t) \neq 0}
	\EndProof
}\Page{
	\Theorem{DecomposableByAnnihilator}{
		\forall k : \Field \.
		\forall V \in \VS{k} \. 
		\forall t \in V^{\wedge2} \.
		\exists x \in V \. \NewLine \. 
		\forall [0] x \neq 0 \.
		\forall [00] v \wedge x = 0 \.
		v : \TYPE{Decomposable}(V^\wedge)
	}
	\Say{(E,[1])}{\THM{BasisExtension}(\{x\})[0]}
	{
		\sum E \subset V \.  \{x\} \cap E : \TYPE{Basis}(V) 
	}
	\Say{o}{\THM{WellOrderingTHM}(E)}{\TYPE{WellOrderingTHM}(E)}
	\Say{(\alpha,[2])}{\THM{ExteriorAlgebraBasis}(E)(t)} 
	{
		\alpha \in k^{\oplus \{x\} \cup E \times \{x\} \cup E} \. 
		t =  \sum_{f \in E} \alpha_{x,g} x \wedge f +  \sum_{g >_o f} \alpha_{f,g} f \wedge g
	}
	\Say{[3]}{[00][1][2]}{ 0  = v \wedge x = \sum_{f \in E} \sum_{g >_o f} \alpha_{f,g} f \wedge g \wedge x }
	\Say{[4]}{\THM{ExteriorAlgebraBasis}\bd \TYPE{Basis}}{ \forall f, g \in E \. \alpha_{f,g} = 0 }
	\Say{[5]}{[4][3]}{t = x \wedge \sum_{e \in E} \alpha_{x,e} e}
	\Conclude{[i]}{\bd \TYPE{Decomposable}[5]}{\LOGIC{This}}
	\EndProof
	\\
	\Theorem{InteriorProductAntiderivation}{
		\forall R \in \ANN \.
		\forall M \in \LMOD{R} \.
		\forall a \in M \. 
		\mathbf{i}_a \in \widetilde{\D}(M^{*\wedge})                                                         
	}
	\Assume{f,g}{\TYPE{Decomposable}(M^{*\wedge})}
	\Say{p}{\deg f}{\Int_+}
	\Say{q}{\deg g}{\Int_+}
	\Say{N}{p+q}{\Int_+}
	\Assume{v}{\TYPE{Decomposable}(M^\wedge)}
	\Say{m}{\deg v}{\Int_+}
	\Say{M}{m+1}{\Int_+}
	\Conclude{[v,*]}{\bd \mathbf{i}_a(f \wedge g)(v) \bd \FUNC{leftExteriorMult}(f \wedge g,a \wedge v) \bd (-1)^{I^*\sigma} \bd}
	{
		\NewLine :
		\mathbf{i}_a(f \wedge g)(v) = 
		(f \wedge g)(a \wedge v) = \NewLine =  
		\sum_{I : N \to M } 
		\sum_{\sigma \in S_N}  (-1)^{I^*\sigma} f_{(\sigma I)^{-1}(1)}(a)
		\prod_{i \in (\sigma I)^{-1}([1] + n) \cap p} f_{i}(v_{I\sigma(i)}) \NewLine 
		\prod_{i \in (\sigma I)^{-1}([1] + n) \cap [q] + p} g_{i}(v_{I\sigma(i)})                                   
		\bigwedge_{  i \in  (I^\C)^{-1}\{1\}  } a
		\wedge
		\bigwedge_{i \in I^\c \cap [m] + n} v_i  
		+ \NewLine + 
		\sum_{I : N \to M } 
		\sum_{\sigma \in S_N}  (-1)^{I^*\sigma} g_{(\sigma I)^{-1}(1)}(a) 
		\prod_{i \in (\sigma I)^{-1}([1] + n) \cap p} f_{i}(v_{I\sigma(i)}) \NewLine 
		\prod_{i \in (\sigma I)^{-1}([1] + n) \cap [q] + p} g_{i}(v_{I\sigma(i)})                                   
		\bigwedge_{ i \in  ( I^\C)^{-1}\{1\}  } a
		\wedge
		\bigwedge_{i \in I^\c \cap [m] + n} v_i = \NewLine
		\sum_{I : N \to M } 
		\sum_{\sigma \in S_N}  (-1)^{I^*\sigma} f_{(\sigma I)^{-1}(1)}(a)
		\prod_{i \in (\sigma I)^{-1}([1] + n) \cap p} f_{i}(v_{I\sigma(i)}) \NewLine 
		\prod_{i \in (\sigma I)^{-1}([1] + n) \cap [q] + p} g_{i}(v_{I\sigma(i)})                                   
		\bigwedge_{  i \in  (I^\c)^{-1}\{1\}  } a
		\wedge
		\bigwedge_{i \in I^\c \cap [m] + n} v_i  
		+ \NewLine + (-1)^p
		\sum_{I : N \to M } 
		\sum_{\sigma \in S_N}  (-1)^{I^*\sigma} g_{(\sigma I)^{-1}(1)}(a) 
		\prod_{i \in (\sigma I)^{-1}([1] + n) \cap  q} f_{i}(v_{I\sigma(i)}) \NewLine 
		\prod_{i \in (\sigma I)^{-1}([1] + n) \cap [p] + q} g_{i}(v_{I\sigma(i)})                                   
		\bigwedge_{ i \in  ( I^\C)^{-1}\{1\}  } a
		\wedge
		\bigwedge_{i \in I^\c \cap [m] + n} v_i = \NewLine
		\mathbf{i}_a(f)(v) + (-1)^p\mathbf{i}_a(g)(v)
	}
	\DeriveConclude{[(f,g).*]}{\bd M^{\wedge}}{\mathbf{i}_a(f \wedge g) = \mathbf{i}_a(f) + (-1)^p\mathbf{i}_a(g)}
	\DeriveConclude{[*]}{\bd \widetilde{\D}(M^{\star\wedge})}{ \iota_a \in \widetilde{\D}(M^\star\wedge)}
	\EndProof
}\Page{
	\Theorem{ExtDecomposableProperty}
	{
		\forall k : \TYPE{NonBinary}  \. 
		\forall V : \VS{k} \. 
		\forall x \in V^{\wedge 2} \.
		x \wedge x = 0 \iff x : \TYPE{Decomoposable}(V)
	}
	\Assume{[1]}{x \wedge x = 0}
	\Assume{[2]}{ x \neq 0 }
	\Say{(f,[3])}{\THM{NonAnnihilatingInteriorProductExists}(x,[2])\THM{CanonicalIsomorphismTHM}}
	{
		\NewLine : \sum f \in V^* \. \mathbf{i}_f(x) \neq 0
	}
	\Say{[4]}{  \bd \VS{k}(V^\wedge,V^\wedge) (\mathbf{i}_f) [1] \bd \widetilde{\D}(V^\wedge) \bd V^{wedge}  }{   
		0 = 
		\mathbf{i}_f(0) = 
		\mathbf{i}_f(x \wedge x) = 
		\mathbf{i}_f(x) \wedge x + x \wedge \mathbf{i}_f(x)  =
		2 \mathbf{i}_f(x) \wedge x 
	}
	\Conclude{[2.*]}{\THM{DecomposableByAnnihalator}[2]}{(x : \TYPE{Decomposable}(V))} 
	\Derive{[2]}{I(\Rightarrow)}{ x \neq 0 \Rightarrow  x  : \TYPE{Decomposable}(V)}
	\Say{[3]}{\bd V^\wedge}{ x = 0 \Imply x  : \TYPE{Decomposable}(V) }
	\Conclude{[1.*]}{E(|)\LOGIC{LEM}[2][3]}{\Big( x : \TYPE{Decomposable}(V) \Big)}
	\Derive{[1]}{I(\Imply)}{x \wedge x = 0 \Imply x : \TYPE{Decompasble}(V)}
	\Say{[4]}{\bd \TYPE{Decomposable}(V)\bd V^\wedge I(\Imply)}{  (x : \TYPE{Decomposable}(V)) \Imply x \wedge x = 0  }
	\Conclude{[*]}{[3][4]}{\LOGIC{This}}
	\EndProof
	\\
	\Theorem{DecomposableByMatrix}{ 
		\forall k : \Field \.  
		\forall V : \FDVS{k} \. 
		\forall e : \Basis(V,\dim V) \.   
		\forall t \in V^\wedge_2 \. \NewLine \. 
		\forall \alpha : \dim V \times \dim V \to  k \.
		\forall [0] : t = \alpha_{i \wedge j} e_i \wedge e_j \. 
		\rank \alpha = 1 \Rightarrow t : \TYPE{Decomposable}(V) 
	}
	\Assume{[1]}{\rank \alpha = 1}
	\Say{(C,\beta,[2])}{\bd \rank \alpha [1]}{ \sum C \in \GL(k,\dim V) \. \sun \beta \in k \dim V \.  C^{-1} \alpha C = \Lambda i,j \in \dim V \. \delta^j_1 \beta_i}
	\Say{f}{Ce}{\TYPE{Basis}(V)}
	\Say{[3]}{\ByConstr f [2]\bd V^\wedge}{t = \sum^{\dim V}_{i=1} \beta_i f_1 \wedge f_i = f_1 \wedge \sum^{\dim V}_{i=1} \beta_i f_i  }
	\Conclude{[1.*]}{\bd^{-1}\TYPE{Decomposable}}{\Big( t : \TYPE{Decomposable}(V)  \Big)}
	\DeriveConclude{[8]}{I(\Rightarrow)}{\rank \alpha = 1 \Rightarrow t : \TYPE{Decomposable}(V) }
	\EndProof
	\\
	\Theorem{InterorProductMapping}
	{
		\forall R \in \ANN \.
		\forall A,B \in \LMOD{R} \.
		\forall \varphi : A \Arrow{\LMOD{R}} B \. 
		\forall a \in A^{\wedge}  \. 
		\varphi^{*\wedge} \mathbf{i}_a = 
		\mathbf{i}_{\varphi^{\wedge}(a) } \varphi^{*\wedge}
	}
	\Assume{f}{B^{\star \wedge}}
	\Assume{v}{A^\wedge}
	\Conclude{[f.*]}{ \ldots}
	{
		\varphi^{*\wedge} \mathbf{i}_a(f)(v) =
		\varphi^{*\wedge}(f)( a \wedge v ) =
	        f\Big( \varphi^\wedge (a \wedge v)  \Big) = \NewLine = 
		f\Big( \varphi^\wedge(a) \wedge \varphi^\wedge(v) \Big) =
		\mathbf{i}_{\varphi^\wedge(a)}(f) \Big( \varphi^\wedge(v)  \Big) =
		\varphi^{*\wedge} \mathbf{i}_{\varphi^\wedge(a)}(f)(v)                           
	}
	\DeriveConclude{[*]}{I(=,\to)}{\LOGIC{This}}
	\EndProof
	\\
	\Theorem{InteriorProductDerivations}{
		\forall R \in \ANN \. 
		\forall M \in \LMOD{R} \.
		\forall a \in M^\wedge \.
		\forall f \in M^* \. 
		\widetilde{D}_f \mathbf{i}_a = \mathbf{i}_{D_f(a)}  + \mathbf{i}_a \widetilde{D}_f
	}
	\NoProof
}
\newpage
\subsection{Mixed Exterior Algebra}
\Page{
	\DeclareFunc{mixedExteriorAlgebra}{\prod R \in \ANN \. \LMOD{R} \to \LALGE{R}}
	\DefineNamedFunc{mixedExteriorAlgebra}{M}{M^{\wedge,*}}{ M^\wedge \otimes M^{*\wedge} }
	\\
	\DeclareFunc{exteriorPower}{\prod k  : \TYPE{NumberField}  \. \prod V \in \VS{R} \.  V^{\wedge,*} \to \Int_+ \to V^{\wedge,*}}
	\DefineNamedFunc{exteriorPower}{t,0}{t^0}{1\otimes 1}
	\DefineNamedFunc{exteriorPower}{t,n}{t^n}{\frac{1}{n!}\prod^n_{i=1} t} 
	\\
	\Theorem{AnticommutativeMixedProduct}
	{
		\forall R \in \ANN \. 
		\forall M \in \LMOD{R} \. 
		\forall p,p',q,q' \in \Int_+ \.
		\forall t \in M^{\wedge,*}_{(p,q)} \.
		\forall s \in M^{\wedge,*}_{(p',q')} \. \NewLine \.  
		t \wedge s = (-1)^{pp' + qq'} s \wedge t
	}
	\NoProof
	\\
	\Theorem{MixedExteriorBinomialFormula}
	{
		\forall k : \TYPE{NumberField} \. 
		\forall M \in \LMOD{R} \. 
		\forall k \in \Int_+ \.
		\forall x,y \in M^{\wedge,*} 
		\. \NewLine \. 
		(x + y)^k = \sum_{n+m=k} x^ny^m
	}
	\NoProof
	\\
	\DeclareFunc{dualExteriorInnerProduct}{ \prod R  \in \ANN \. \prod M \in \LMOD{R} \. \L( M^{*\wedge},M^\wedge;R) }
	\DefineNamedFunc{dualExteriorInnerProduct}{ f,v  }{ \langle f, v \rangle }
	{
		\bd M^{\wedge}\bd M^{*\wedge} \NewLine 
		\Lambda f : \TYPE{Decomposable}(M^{*}) \.
		\Lambda v : \TYPE{Decomposable}(M) \. 
		\If \deg f = \deg v \. 
		\Then \det (f_i(v_j))^{\deg f}_{i,j=1} 
		\Else 0 
	}
	\\
	\DeclareFunc{mixedExteriorInnerProduct}{ \prod R  \in \ANN \. \prod M \in \LMOD{R} \. \TYPE{InnerProduct}( M^{\wedge,*}) }
	\DefineNamedFunc{mixedExteriorInnerProduct}{ f,v  }{ \langle f, v \rangle }
	{
		\bd M^{\wedge,*} 
		\Lambda f  \in M^{*\wedge} \.
		\Lambda v  \in M^\wedge \. 
		\Lambda g \in M^{*\wedge} \.
		\Lambda w \in M^{\wedge} \. \NewLine \.  
		\langle f,w\rangle\langle g, v \rangle
	}
	\\
	\DeclareType{MixedInteriorProduct}
	{
		\prod R \in \ANN \.
		\prod M \in \LMOD{R} \. 
		M^{\wedge,*} \to M^{\wedge,*} \to M^{\wedge,*} 
	}
	\DefineType{T}{MixedInteriorProduct}{ \forall a,b,c \in M^{\wedge,*} \. \langle T(a)(b), c \rangle = \langle b , ac \rangle  }
	\\
	\Theorem{MixedInteriorProducUnique}{\forall R \in \ANN \. \forall M \in \LMOD{R} \. \forall T,S : \TYPE{MixedIneriorProduct}(M,R) \. T = S}
	\NoProof
	\\
	\DeclareFunc{mixedInterioreProduct}
	{
		\prod k : \Field \. 
		\prod V : \VS{k} \.
		\TYPE{MixedInteriorProduct}(V)
	}
	\DefineNamedFunc{mixedInteriorProduct}{a,b}{\mathbf{i}_a(b)}{L_a^\star(b)}
}
\Page{
	\Theorem{DegreeOfmixedInteriorProcuct}
	{
		\forall k : \Field \.
		\forall V \in \VS{k} \. 
		\forall a : \TYPE{Homogeneous}\Big(  M^{\wedge,*} \Big)\.
		\forall n,m \in \Int_+ \.
		\NewLine \.
		\forall [0]  : \deg a = (n,m) \.
		\deg \mathbf{i}_a = (-n,-m)
	}
	\NoProof
	\\
	\Theorem{MixedInteriorProductComposition }
	{
		\forall k : \Field \.
		\forall V \in \VS{k} \.
		\forall a,b \in M^{\wedge,*} \.
		\mathbf{i}_a \mathbf{i}_b = \mathbf{i}_{ab}
	}
	\NoProof
	\\
	\DeclareFunc{diagonalSubalgebra}
	{
		\prod R \in \ANN \. 
		\LMOD{R} \to \LALGE{R}
	}
	\DefineNamedFunc{diagonalSubalgebra}{M}{M^\Delta}{ \bigoplus^\infty_{n=0} M^{\wedge,\star}_{(n,n)}}
	\\
	\DeclareFunc{mixedExteriorMap}{
		\forall R \in \ANN \. 
		\forall A,B \in \LMOD{R} \.
		(A \Arrow{\LMOD{R}} B) 
		\times
		(B \Arrow{\LMOD{R}} A) \to
		A^{\wedge,*} \Arrow{\LALGE{R}} B^{\wedge,*}
	}
	\DefineNamedFunc{mixedExteriorMap}{f,g}{(f,g)^{\wedge,*}}{f^{\wedge}\otimes g^{*\wedge}}
	\\
	\Theorem{MixedExteriorMapTHM}{
		\forall R \in \ANN \. 
		\forall A,B \in \LMOD{R} \.
		\forall \phi : A \Arrow{\LMOD{R}} B \.
		\forall \psi : B \Arrow{\LMOD{R}} A \.
		\NewLine \. 
		\forall a \in  A^{\wedge,*} \.
		(\psi,\phi)^{\wedge,*} \mathbf{i}_a =
		\mathbf{i}_{(\phi,\psi)^{\wedge,*}a} (\psi,\phi)^{\wedge,*}
	}
	\NoProof
	\\
	\DeclareFunc{asExteriorLinearMap}
	{
		\prod R \in \ANN \.
		\prod M \in \LMOD{R} \.
		M^{wedge,*} \Arrow{\LMOD{R}} \L(M^{\wedge},M^{\wedge})
	}
	\DefineNamedFunc{asExteriorLinearMap}{x}{T_x}{\Lambda t \in M^\wedge \. f_i(t)v_i \quad \where \quad x = f_i \otimes v_i}
	\\
	\DeclareFunc{boxproduct}
	{
		\prod R \in \ANN \.
		\prod M \in \LMOD{R} \.
		\prod n \in \Nat \.
		(n \to M \Arrow{\LMOD{R}} M) \. 
		M^{\wedge}_n \Arrow{\LMOD{R}} M^\wedge_n 
	}
	\DefineNamedFunc{boxProducut}{\phi}{\bigboxtimes^n_{i=1} \phi_i }{ \bd M^\wedge \Lambda \bigwedge^n_{i=1} v_i \in M^\wedge_n \. \sum_{\sigma \in S_n} (-1)^\sigma \bigwedge^n_{i=1} \phi_i(v_{\sigma(i)})  }
	\\
	\Theorem{PermutationPreservesBoxProduct}
	{
		\forall R \in \ANN \.
		\forall M \in \LMOD{R} \.
		\forall n \in \Nat \.
		\forall \phi : n \to M \Arrow{\LMOD{R}} M \. \NewLine \.  
		\forall \sigma \in S_n \.
		\bigboxtimes^n_{i=1} \phi_{\sigma(i)} = \bigboxtimes^n_{i=1} \phi_i
	}
	\NoProof
}
\Page{
	\Theorem{BoxProductDistributivity}
	{
		\forall R \in \ANN \.
		\forall M \in \LMOD{R} \.
		\forall n \in \Nat \.
		\forall x : n \to M \otimes M^{*} \.
		\NewLine 
		T_{\prod^n_{i=1}x_i} = \bigboxtimes^n_{i=1} T_{x_i}
	}
	\Assume{f}{n \to M^*}
	\Assume{u}{n \to M}
	\Assume{[1]}{x = u \otimes f}
	\Assume{\bigwedge^n_{i=1} v_i}{\TYPE{Decomposable}(M)}
	\Conclude{[1.*]}{  
		[1]
		\bd \FUNC{mixedExteriorAlgebra}
		\bd \FUNC{asExteriorLinearMap}  
		\bd \FUNC{dualExteriorInnerProduct}
		\bd \det
		\bd M^\wedge \NewLine 
		\bd^{-1} \FUNC{asExteriorLinearMap}
		\bd^{-1} \FUNC{bixProduct}
	}
	{
		\NewLine : 
		T\Act{\prod^n_{i=1}x_i}\Act{\bigwedge^n_{i=1} v_i} =
		T\Act{\prod^n_{i=1}u_i \otimes f_i} \Act{ \bigwedge^n_{i=1}v_i }  =
		T\Act{\bigwedge^n_{i=1} u_i \otimes \bigwedge^n_{i=1} f_i }\Act{\bigwedge^n_{i=1} v_i} = \NewLine = 
		\bigwedge^n_{i=1} f_i  \Act{\bigwedge^n_{i=1} v_i} \bigwedge^n_{i=1} u_i  = 
		\det \big( f_i(v_j)  \big)^n_{i,j=1} \bigwedge^n_{i=1} u_i = 
		\sum_{\sigma \in S_n} (-1)^\sigma \Act{\prod^n_{i=1} f_i(v_{\sigma(i)})} \bigwedge^n_{i=1} u_i  = \NewLine =  
		\sum_{\sigma \in S_n} (-1)^\sigma  \bigwedge^n_{i=1} f_i(v_{\sigma(i)}) u_i =  
		\sum_{\sigma \in S_n} (-1)^\sigma \bigwedge^n_{i=1} T_{x_i}(v_{\sigma(i)})   =
		\bigboxtimes^n_{i=1} T_{x_i} \Act{\bigwedge^n_{i=1} v_i } 
	}
	\DeriveConclude{[*]}{\bd M^\wedge}{ T_{\prod^n_{i=1} x_i} = \bigboxtimes^n_{i=1} T_{x_i}}
	\EndProof
	\\
	\DeclareFunc{exteriorCompositionProduct}
	{
		\prod R \in \ANN \. 
		\prod M \in \LMOD{R} \.
		\L\Big(M^{\wedge,*},M^{\wedge,*};M^{\wedge,*})
	}
	\DefineNamedFunc{exteriorCompositionProduct}
	{ f \otimes v, g \otimes u  }{f \otimes v \circ g \otimes u}{  (g_i(v_i)) f_i \otimes u_i}
	\\
	\Theorem{compositionProductProperty}
	{
		\forall R \in \ANN \.
		\forall M \in \LMOD{R} \.
		\forall f \otimes v, g \otimes u \im M^{\wedge,*} \.
		T_{f \otimes v \circ g \otimes u} = T_{f \otimes v} \circ T_{g \otimes u}
	}
	\NoProof
}\Page{
	\Theorem{InteriorProductOfMixedProductFormula}{
		\forall R \in \ANN \. 
		\forall M \in \LMOD{R} \.
		\forall x \in M \otimes M^* \. \NewLine 
		\forall n \in \Nat \.
		\forall y : n \to M \otimes M^* \.
		\mathbf{i}_{x} \prod^n_{i=1} y_i  =
		\sum^n_{i=1}  {\langle x, y_i \rangle} \prod^{n-1}_{j=1} \hat{y}_{i,j} -
		\sum^n_{i=1} \sum^n_{j=i+1} ( y_i \circ x \circ y_j + y_j \circ x \circ y_i) \prod^{n-2}_{k=1} \hat y_{(i,j),k}
	}
	\Assume{f}{M^*}
	\Assume{v}{M}
	\Assume{g}{n \to M^*}
	\Assume{u}{n \to M}
	\Assume{[1]}{ x =v \otimes f}
	\Assume{[2]}{ y = u \otimes g}
	\Conclude{[\ldots*]}{ [1][2] \bd \mathbf{i} \bd \ABEL(M^{\wedge,\star})\bd M^\wedge \bd M^{\star \wedge} [1][2]    }{  
		\NewLine :
		\mathbf{i}_x \prod^n_{i=1} y_i = 
		\mathbf{i}_{v \otimes f} \bigwedge^n_{i=1} u_i \otimes \bigwedge^n_{i=1} f_i = 
		\sum^n_{i,j=1} (-1)^{i + j} f(u_i) g_i(v) \bigwedge^{n-1}_{k=1} \hat u_{i,k} \otimes \bigwedge^{n-1}_{k=1} f_i = \NewLine 
		\sum^n_{i=1}    \langle v \otimes f , u_i \otimes g_i \langle \bigwedge^{n-1}_{k=1} \hat u_{i,k} \otimes \bigwedge^{n-1}_{k=1} \hat f_{j,k} + 
		\sum^n_{i=1}\sum^n_{j=i+1} (-1)^{2i + 2j + 1 } f(u_i) g_j(v) v_j \otimes g_i \bigwedge^{n-1}_{k=1} \hat u_{i,k} \otimes \bigwedge^{n-w}_{k=1} f_{(i,j),k} + \NewLine +
		\sum^n_{i=1}\sum^{i-1}_{j=1} (-1)^{2i + 2j} f(u_i) g_j(v) v_j \otimes g_i \bigwedge^{n-2}_{k=1} \hat u_{(i,j),k} \otimes \bigwedge^{n-2}_{k=1} \hat f_{(i,j),k} =  \NewLine = 
		\sum^n_{i=1}  {\langle x, y_i \rangle} \prod^{n-1}_{j=1} \hat{y}_{i,j} -
		\sum^n_{i=1} \sum^n_{j=i+1} ( y_i \circ x \circ y_j + y_j \circ x \circ y_i) \prod^{n-2}_{k=1} \hat y_{(i,j),k}
	}
	\DeriveConclude{[*]}{\bd M \otimes M^*}{\LOGIC{This}}
	\EndProof
	\\
	\Theorem{InteriorProductOfMixedProductFormula}{
		\forall R \in \ANN \. 
		\forall M \in \LMOD{R} \.
		\forall x \in M \otimes M^* \. \NewLine 
		\forall n \in \Nat \.
		\forall y \in  M \otimes M^* \.
		\mathbf{i}_{x}  y^n  = \langle x, y  \rangle y^{n-1} -  (y \circ x \circ y) y^{n-2}      }
	\NoProof
}
\newpage
\subsection{Algebraic Poincare Duality}
\Page{
	\DeclareFunc{asLinearMapInDegrees}{ \prod R \in \ANN \. \prod V \in \LMOD{R} \. \prod p,q \in \Int_+ \. M^{\wedge,*}_{(p,q)} \ToIso{\LMOD{R}} \L(V^\wedge_p,V^\wedge_q) }
	\DefineNamedFunc{asLinearMapInDegrees}{ v}{T_{v|V^{\wedge p}}^{V^{\wedge q}}}{T^{p,q}_v} 
	\\
	\Theorem{TraceInnerProduct}
	{
		\forall k :  \Field \.
		\forall V \in \FDVS{k} \.
		\forall p,q \in \Int_+ \.
		\forall x \in V^{\wedge,*}_{(p,q)} \.
		\forall y \in V^{\wedge,*}_{(q,p)} \. \NewLine
		\tr T_x^{p,q} \circ T_y^{q,p}  = \langle x,y \rangle  
	}
	\NoProof
	\\
	\Theorem{CompositionIsomorphism}
	{
		\forall k : \Field \.
		\forall V \in \FDVS{k} \.
		\forall p \in \Int_+ \.
		T^{p,p} : (V^\Delta_p,\circ) \ToIso{\LALGE{k}} \L(V^{\wedge p}; V^{\wedge p})
	}
	\NoProof
	\\
	\DeclareFunc{unitTensor}{ \prod k : \Field \. \prod V \in \FDVS{k}  \. V^{\wedge,*}   }
	\DefineNamedFunc{unitTensor}{}{\mercury}{T^{-1}({\id}_{V^\wedge})}
	\\
	\DeclareFunc{unitTensorOfDegree}{ \prod k : \Field \. \prod V \in \FDVS{k}  \. \prod p \in \dim V \. V^\Delta_{p}   }
	\DefineNamedFunc{unitTensorOfDegree}{}{\mercury_p}{(T^{p,p})^{-1}({\id}_{V^\wedge_p})}
	\\
	\Theorem{UnitTensorDecomposition}{ \forall k : \Field \.  \forall V \in \FDVS{k} \. \mercury  = \sum^{\dim V}_{i = 0} \mercury_i  }
	\NoProof
	\\
	\Theorem{UnitTensorPower}{\forall k : \Field \. \forall V \in \FDVS{k} \. \forall p \in \dim V \.  \mercury_p = \mercury_1^p}
	\NoProof
	\\
	\Theorem{UnitTensorTraceRelation}{
		\forall k : \Field \. 
		\forall V \in \FDVS{k} \. 
		\forall p \in \dim V \. 
		\forall w \in V^\Delta_p \.
		\langle \mercury_p, w \rangle = \tr T_w^{p,p} 
	}
	\NoProof
	\\
	\Theorem{UnitTensorMult}{
		\forall k : \Field \. 
		\forall V \in \FDVS{k} \. 
		\forall p,q \in \dim V \. 
		\mercury_p \mercury_q = C_{p+q}^p \mercury_{p+q}
	}
	\NoProof 	
}
\Page{
	\Theorem{UnitTensorIneriorMult}{
		\forall k : \Field \.
		\forall V \in \FDVS{k} \.
		\forall p,q \in  \dim V \.
		\mathbf{i}_{\mercury_p} \mercury_q =  C_{\dim V - q + p}^p \mercury_{q-p}
	}
	\Assume{[1]}{p = 1}
	\Conclude{[5.*]}{\THM{UnitTensorPower}(q)\THM{InteriorProductOfMixedPower}(\mercury_1,\mercury_1,q)\THM{UnitTensorTraceRelation}\bd \NewLine \id \bd^{-1} \FUNC{binomialCoeficient}}{
		\NewLine :
		\mathbf{i}_{\mercury_1}(\mercury_{q}) = 
		\mathbf{i}_{\mercury_1}\big( \mercury_1^{q} ) = 
		 \Big( \langle \mercury_1,\mercury_1 \rangle \mercury_1^{1-1}  - \mercury^{\circ 3}_1 \mercury_1^{q-2}  \Big) =
		 \Big(  (\dim V) \mercury^{q-1}_1 - (p - 1)\mercury^{q-1}_1  \Big) =  \NewLine 
		  (\dim V - q + 1 )    \mercury_{q-1} =  C^1_{\dim V - q + 1} \mercury_{q-1}
	}
	\Derive{[1]}{I(\Rightarrow)}{p=1 \Rightarrow \mathbf{i}_{\mercury_p} \mercury_q = C^p_{\dim v - q + p} \mercury_{q-p} }
	\Assume{n}{\Nat}
	\Assume{[2]}{\forall k \in n \. p =n \Rightarrow \mathbf{i}_{\mercury_p} \mercury_q = C^p_{\dim v - q + p} \mercury_{q-p} }
	\Assume{[3]}{p = n+ 1}
	\Conclude{[n.*]}{ \THM{UnitTensorMult} \THM{MixedExtrioorProductComp}[1][2][3] \bd \FUNC{BinomialCoefficient}  }
	{
		\NewLine :
		\mathbf{i}_{\mercury_p} \mercury_q = 
		\frac{1}{p}\mathbf{i}_{  \mercury_1 \mercury_n  } \mercury_q = 
		\frac{1}{p} \mathbf{i}_{\mercury_1} \mathbf{i}_{\mercury_n} \mercury_q =  
		\frac{1}{p} C^n_{\dim V - q + n} \mathbf{i}_{\mercury_1} \mercury_{q - n} = \NewLine = 
		\frac{1}{p} C^n_{\dim V - q + n} (\dim V - q + p) \mercury_{q - p} =
		\frac{(\dim V - q + p)!}{ (\dim V - q)! p!   } \mercury_{q - p}  = 
		C^{p!}_{(\dim V - q + p)!} \mercury_{q-p}
	}
	\DeriveConclude{[*]}{\bd \TYPE{NaturalSet}(\dim V)[1]I(\Imply)}{ \LOGIC{This}  }
	\EndProof
	\\
	\DeclareFunc{flatPoincareIsomorphism}{  \prod k : \Field \. \prod V : \FDVS{k} \. \prod e : \Basis(V,\dim V) \. V^\wedge \ToIso{\LALGE{k}} V^{*\wedge}}
	\DefineNamedFunc{flatPoincareIsomorphism}{ t }{ D_{\flat e} t}{ \mathbf{i}_t \bigwedge^{\dim V}_{i=1} e^*_i}
	\\
	\DeclareFunc{sharpPoincareIsomorphism}{  \prod k : \Field \. \prod V : \FDVS{k} \. \prod e : \Basis(V,\dim V) \. V^{*\wedge} \ToIso{\LALGE{k}} V^{\wedge}}
	\DefineNamedFunc{sharpPoincareIsomorphism}{ t }{ D_{\sharp e} t}{\mathbf{i}_t \bigwedge^{\dim V}_{i=1} e_i}
	\\
	\DeclareFunc{flatPoincareIsomorphismScalarMult}
	{
		\forall k : \Field \.
		\forall V \in \FDVS{k} \.
		\forall e : \Basis(V,\dim V) \.
		\forall \alpha \in k^* \.\NewLine \.
		D_{\flat \alpha e} = \alpha^{-\dim V} D_{\flat e} 
	}
	\NoProof
	\\
	\DeclareFunc{flatPoincareIsomorphismScalarMult}
	{
		\forall k : \Field \.
		\forall V \in \FDVS{k} \.
		\forall e : \Basis(V,\dim V) \.
		\forall \alpha \in k^* \. \NewLine \. 
		D_{\sharp \alpha e} = \alpha^{\dim V} D_{\sharp e} 
	}
	\NoProof
}\Page{
	\DeclareFunc{flatPoincareIsomorphismExteriorMult}
	{
		\forall k : \Field \.
		\forall V \in \FDVS{k} \.
		\forall e : \Basis(V,\dim V) \.
		\forall x,y \in V^\wedge \. \NewLine \.
		D_{\flat e}(x \wedge y) = \mathbf{i}_y  D_{\flat e}(x)
	}
	\NoProof
	\\
	\DeclareFunc{flatPoincareIsomorphismScalarMult}
	{
		\forall k : \Field \.
		\forall V \in \FDVS{k} \.
		\forall e : \Basis(V,\dim V) \.
		\forall f,g \in V^{*\wedge} \. \NewLine \.
		D_{\sharp e}(f \wedge g) = \mathbf{i}_g D_{\sharp e}(x)
	}
	\NoProof
	\\
	\Theorem{PoincareIsometry}{
		\forall k : \Field \. 
		\forall V \in \FDVS{k} \.
		\forall e : \Basis(V,\dim V) \. \NewLine \. 
		\forall v \in V^{\wedge} \. 
		\forall f \in V^{*\wedge} \.
		\langle D_{\sharp e} f, D_{\flat e} v  \rangle = \langle f, v \rangle
	}
	\Assume{[1]}{\Act{v : \TYPE{Homogeneous}(V^\wedge)}}
	\Assume{[2]}{\Act{f : \TYPE{Homogeneous}(V^{*\wedge})}}
	\Say{p}{\deg v}{\Int_+}
	\Assume{[3]}{\deg g = p}
	\Say{[4]}{
		\bd D_\sharp \bd D_\flat  
		\bd^{-1} \FUNC{mixedExteriorInnerProduct}   
		\bd^{-1} \FUNC{mixedInteriorProduct}
		\bd \TYPE{MixedInteriorProduct} \NewLine 
		\bd M^\Delta
		\bd \TYPE{MixedIneriorProduct}
		\THM{UnitTensorInteriorMult
		\bd \FUNC{mixedExteriorInnerProduct}}
	}
	{
		\NewLine
		\langle D_{\sharp e} f, D_{\flat e} v  \rangle =   
		\left\langle \mathbf{i}_f \bigwedge^n_{i=1} e_i , \mathbf{i}_v \bigwedge^n_{i=1} e^*_i \right\rangle =
		\left\langle \mercury_{n-p},\mathbf{i}_f \bigwedge^n_{i=1} e_i \otimes \mathbf{i}_v \bigwedge^n_{i=1} e^*_i \right\rangle =  
		\langle \mercury_{n-p},\mathbf{i}_{f \otimes v} \mercury_n \rangle = \NewLine =
		\langle  (f \otimes v)\mercury_{n-p}, \mercury_n \rangle = 
		\langle \mercury_{n-p}(f \otimes v), \mercury_n \rangle = 
		\langle f \otimes v, \mathbf{i}_{\mercury_{n-p}} \mercury_n \rangle =
		\langle f \otimes v,  \mercury_{p} \rangle =
		\langle f, v \rangle
	}
	\DeriveConclude{[*]}{\bd V^\wedge \bd V^{*\wedge} }{\LOGIC{This}}
	\EndProof
	\\
	\Theorem{FlatPoincareDuality}{
		\forall k : \Field \.
		\forall V \in \FDVS{k} \.
		\forall e : \Basis(V,\dim V) \.
		\forall p \in \dim V \. \NewLine \.
		D^*_{\flat e| V^{\wedge}_p} = (-1)^{p(n-p)} D_{\flat e| V^{\wedge}_{n-p}}
	}
	\Assume{v}{V^\wedge_p}
	\Assume{u}{V^\wedge_{n-p}}
	\Conclude{[v.*]}{ \bd D_\flat \bd \TYPE{InteriorProduct} \bd \FUNC{exteriorProduct} \bd \TYPE{InteriorProduct} \bd^{-1} D_\flat  }
	{
		\NewLine :
		\langle D_{\flat e} v, u \rangle = 
		\left\langle \mathbf{i}_v \bigwedge^n_{i=1} e_i^* , u  \right\rangle = 
		\left\langle \bigwedge^n_{i=1} e^*_i ,  v \wedge u \right\rangle = 
		(-1)^{p(n-p)} \left\langle \bigwedge^n_{i=1} e^*_i, u \wedge v \right\rangle = \NewLine = 
		(-1)^{p(n-p)} \left\langle \mathbf{i}_v \bigwedge^n_{i=1} e_i^*,  v \right\rangle = 
		(-1)^{p(n-p)} \langle D_{\flat e} u, v \rangle  = 
		\langle v,  (-1)^{p(n-p)} D_{\flat e} u \rangle
	}
	\Derive{[*]}{ \bd \TYPE{DualMap}  }
	{
		D^*_{\flat e| V^{\wedge}_p} = (-1)^{p(n-p)} D_{\flat e| V^{\wedge}_{n-p}}
	}
	\EndProof
	\\
	\Theorem{SharpPoincareDuality}{
		\forall k : \Field \.
		\forall V \in \FDVS{k} \.
		\forall e : \Basis(V,\dim V) \.
		\forall p \in \dim V \. \NewLine \.
		D^*_{\sharp e| V^{*\wedge}_p} = (-1)^{p(n-p)} D_{\sharp e| V^{*\wedge}_{n-p}}
	}
	\NoProof
	\\
}\Page{
	\Theorem{FlatPoincareSemiinversion}{
		\forall k : \Field \.
		\forall V \in \FDVS{k} \.
		\forall e : \Basis(V,\dim V) \.
		\forall p \in \dim V \. \NewLine \.
		D_{\flat e| V^{\wedge}_p} D_{\sharp e| V^{\wedge}_{n-p}} = (-1)^{p(n-p)} \id  }
	\NoProof
	\\
	\Theorem{SharpPoincareSemiinversion}{
		\forall k : \Field \.
		\forall V \in \FDVS{k} \.
		\forall e : \Basis(V,\dim V) \.
		\forall p \in \dim V \. \NewLine \.
		D_{\sharp e| V^{\wedge}_p} D_{\flat e| V^{\wedge}_{n-p}} = (-1)^{p(n-p)} \id  }
	\NoProof
	\\
	\Theorem{FlatPoincareNaturality}{
		\forall k : \Field \.
		\forall V,U \in \FDVS{k} \.
		\forall e : \Basis(V,\dim V) \.
		\forall \varphi : V \ToIso{\VS{k}} U \. \NewLine \. 
		\varphi^{\wedge} D_{\flat \varphi(e)} = D_{\flat e} \varphi^{-1*\wedge}  	
	}
	\NoProof
	\\
	\Theorem{SharpPoincareNaturality}{
		\forall k : \Field \.
		\forall V, U \in \FDVS{k} \.
		\forall e : \Basis(V,\dim V) \.
		\forall \varphi : V \ToIso{\VS{k}} U  \. \NewLine \.
		\varphi^{-1*\wedge} D_{\sharp e}  = D_{\sharp \varphi(e)} \varphi  
	}
	\NoProof
	\\
	\DeclareFunc{NaturalPoincareIsomorphism}
	{
		\prod k : \Field \.
		\prod V : \FDVS{k} \. 
		V^{\wedge,*} \ToIso{\LALGE{R}} V^{\wedge,*} 
	}
	\DefineNamedFunc{NaturalPoincareIsomorphism}{t}{D_\natural(t)}{ \mathbf{i}_t(\mercury) }
	\\
	\Theorem{NaturalPIDecomposition}
	{
		\forall k : \Field \.
		\forall V : \FDVS{k} \.
		\forall e : \Basis{k} \. 
		\forall v \in V^{\wedge} \.
		\forall f \in V^{*\wedge} \. \NewLine \. 
		D_\natural v \otimes f = D_{\flat e} (v) \otimes D_{\sharp e}(f) 
	}
	\NoProof	
	\\
	\Theorem{NaturalPIMult}
	{
		\forall k : \Field \.
		\forall V : \FDVS{k} \. 
		\forall t,s \in V^{*,\wedge} \. 
		D_\natural(t \cdot s) = \mathbf{i}_t D_\natural(s)
	}
	\NoProof
	\\
	\DeclareFunc{PoincareInvolution}{\prod k : \Field \. \prod V : \FDVS{k} \. V^{*,\wedge} \to V^{*,\wedge}}
	\DefineNamedFunc{PoincareInvolution}{}{\omega_\natural}{ \bd V^{\wedge,*} \Lambda p,q \in \Int_+ \. \Lambda v \in V^{\wedge p} \. \Lambda f \in V^{*\wedge q} \. (-1)^{q(n-p) + p(n-q)} v \otimes f \quad \NewLine \quad n = \dim V  }
}
\Page{
	\Theorem{BalancedPoincareIsometry}{
		\forall k : \Field \. 
		\forall V \in \FDVS{k} \.
		\forall t,s \in V^{\wedge,*} \. 
		\langle D_\natural t, D_\natural s  \rangle = \langle t, s \rangle
	}
	\NoProof
	\\
	\Theorem{BalancedPoincareDuality}{
		\forall k : \Field \.
		\forall V \in \FDVS{k} \.
		 \NewLine \.
		D^*_{\natural} = \omega_\natural \circ D_{\natural}
	}
	\NoProof
	\\
	\Theorem{FlatPoincareSemiinversion}{
		\forall k : \Field \.
		\forall V \in \FDVS{k} \.
		D_\natural^{\circ 2} = \omega_\natural  }
	\NoProof
	\\
	\Theorem{FlatPoincareNaturality}{
		\forall k : \Field \.
		\forall V,U \in \FDVS{k} \.
		\forall \varphi : V \ToIso{\VS{k}} U \. \NewLine \. 
		\varphi^{\wedge,*} D_{\natural} = D_{\natural} \varphi^{\wedge,*}  	
	}
	\NoProof
	\\
	\DeclareFunc{intersectionProduct}{ \prod k : \Field \. \prod V : \FDVS{k} \. \prod e : \Basis(V)  \. V^{\wedge} \times V^{\wedge} \to V^{\wedge} }
	\DefineNamedFunc{intersectionProduct}{ t,s}{t \cap_e s}{ D_{\sharp e}(D_{\sharp e}^{-1} t \wedge D_{\sharp e}^{-1} s )}
	\\
	\Theorem{IntersectionProductAnticommute}
	{
		\forall k : \Field \. 
		\forall V : \FDVS{k} \. 
		\forall e : \Basis(V) \.
		\forall t,s : \TYPE{Homogeneous}(V^{\wedge}) \. \NewLine \. 
		\forall p \in \Int_+ \. 
		\forall q \in \Int_+ \. 
		\forall [0] : \deg t = p \.
		\forall [00] : \deg s = q \.
		 t \cap_e s = (-1)^{(n-p)(n-q)}s \cap_e t
		\quad \where \quad
		n = \dim V 
	}
	\NoProof
	\\
	\Theorem{IntersectionProductWithBasis}
	{
		\forall k : \Field \. 
		\forall V : \FDVS{k} \. 
		\forall e : \Basis(V) \.
		\forall t : V^{\wedge} \. \NewLine \. 
		t \cap_e \bigwedge^n_{i=1} e_i = t
	}
	\NoProof
}\Page{
	\Theorem{PoincareAlgebraHomo}
	{
		\forall k : \Field \.
		\forall V : \FDVS{k} \.
		\forall e : \Basis(V) \.
		\forall t,s \in V^{\wedge} \. \NewLine \. 
		D_{\flat e}(t \cap_e s) = D_{\flat e}(t) \wedge D_{\flat e}(s) 
	}
	\Say{n}{\dim V}{\Int_+}
	\Conclude{[*]}{\bd \FUNC{intersectionProduct}(V,e) \THM{FlatPoincarePseudoInverse}^2(V,e) \bd \L(V^\wedge,L^\wedge;L^\wedge)(\wedge) \bd(-1)}{
		\NewLine
		D_{\flat e}( t \cap_e s) = 
		D_{\flat e} D_{\sharp e} \Big(D_{\sharp e}^{-1}(t) \wedge D_{\sharp e}^{-1}(s) \Big) =
		\sum^n_{p,q= 0} (-1)^{(2n - p - q)(p + q - n)} D_{\sharp e}^{-1}(t_p) \wedge D_{\sharp e}^{-1}(s_q) = \NewLine = 
		\sum^n_{p,q = 0}(-1)^{(2n-p-q)(p+q-n)} \Big( (-1)^{p(n- p)} D_{\flat e}t_p \wedge (-1)^{q(n - q)}s_q\Big) =  
		\sum^n_{p,q = 0}(-1)^{2np + 2nq - 2p^2 - 2q^2  } D_{\flat e} t_p \wedge D_{\flat e} s_q = \NewLine = 
		D_{\flat e} t \wedge D_{\flat e} s
	}
	\EndProof
	\\
	\Theorem{PoincareAlgebraHomo2}
	{
		\forall k : \Field \.
		\forall V : \FDVS{k} \.
		\forall e : \Basis(V) \.
		D_{\flat e} : (V^\wedge,\cap_e) \Arrow{\LALGE{k}} (V^\wedge,\wedge) 
	}
	\NoProof
	\\
	\DeclareFunc{externalProduct}{ \prod k : \Field \. \prod V : \FDVS{k} \. \prod e : \Basis(V) \. V^{(\dim V) - 1} \to V}
	\DefineNamedFunc{externalProduct}{ v}{ [v]_e }{ D_{\flat e} \bigwedge^{\dim V - 1}_{i=1} v_i  }
	\\
	\Theorem{ExternalProductOrthogonality}{
		\forall k : \Field \. 
		\forall V : \FDVS{k} \. 
		\forall e : \Basis(V) \.  \NewLine \. 
		\forall v : (\dim V - 1) \to V \.
		\forall i \in \dim V - 1 \.\Big \langle [v]_e, v_i \rangle = 0 
	}
	\NoProof
	\\
	\Theorem{LagrandgeIdentity}{
		\forall k : \Field \.
		\forall V : \FDVS{k} \.
		\forall e : \Basis(V) \. 
		\forall v : (n-1) \to V \. 
		\forall f : (n-1) \to V^* \. \NewLine \.
		\Big\langle [f]_{e^*}, [v]_{e} \Big\rangle = \det \Big( f_i(v_j)\Big)_{i,j=1}^{n-1}
		\quad \where \quad n = \dim V
	}
	\NoProof
}
\newpage
\subsection{Pfaffian}
\Page{
	\DeclareFunc{leftDiffeomult}{\prod R \in \ANN \. \prod M \in \LMOD{R} \. \TYPE{Alternating}(M,R)  \to M \Arrow{\LMOD{R}} M^\wedge \Arrow{\LMOD{R}} M^\wedge}
	\DefineNamedFunc{leftDiffeomult}{T,m,t}{\Lambda_{T,m}(t)}{  L_m(t) + D_{T,m}(t)  }
	\\
	\Theorem{AlternatingDiffeomult}
	{
		\forall R \in \ANN \. 
		\forall M \in \LMOD{R} \.
		\forall T : \TYPE{Alternating}(M,R) \. 
		\forall m \in M \. 
		\Lambda_{T,m}^2 = 0
	}
	\Assume{t}{\TYPE{Alternationg}(M,R)}
	\Conclude{[t.*]}{\bd \Lambda_{T,m} \bd \FUNC{exteriorAlgebra}(M) \bd D_{T,m} \bd \TYPE{Alernating}(M,r)(T) \bd \ABEL(M^\wedge)  }{ 
		\NewLine :
		\Lambda_{T,m}^2(t) = 
		\Lambda_{T,m} \left(  L_m(t)  + D_{T,m}(t) \right) = 
		L^2_m(t)   + L_m D_{T,m}(t)  + D_{T,m} L_m(t) + D^2_{T,m}(t) = \NewLine =  
		m \wedge m \wedge t  + m \wedge D_{T,m}(t) + D_{T,m}(m \wedge t) =
		m \wedge D_{T,m}(t)  + T(m,m) \wedge t  - m \wedge D_{T,m}(t) = 0
	}
	\Derive{[*]}{I(=,\to)}{ \Lambda_{T,m}^2 = 0 }
	\EndProof
	\\
	\Theorem{DiffeomultExteriorAsComp}
	{
		\forall R \in \ANN \. 
		\forall M \in \LMOD{R} \.
		\forall T : \TYPE{Alternating}(M,R) \. 
		\forall n \in \Nat \. 
		\forall u : n \to M \. \NewLine \.  
		\Lambda_T^{\wedge} \bigwedge^n_{i=1} u_i =  
		\prod^{n-1}_{i=0} \Lambda_{T,u_{n-i}}
	}
	\NoProof
	\\
	\DeclareFunc{higherDiffeomult}
	{
		\prod R \in \ANN \. 
		\prod M \in \LMOD{R} \.
		\TYPE{Alternating}(M,R) \to
		M^\wedge \Arrow{\LMOD{R}} M^\wedge
	}
	\DefineNamedFunc{higherDiffeomult}
	{
		T,t
	}
	{
		\Omega_T(t)
	}
	{
		\Lambda^\wedge_{T}(t)(1)
	}
	\\
	\DeclareFunc{higherAntidiffeomult}
	{
		\prod R \in \ANN \. 
		\prod M \in \LMOD{R} \.
		\TYPE{Alternating}(M,R) \to
		M^\wedge \Arrow{\LMOD{R}} M^\wedge
	}
	\DefineNamedFunc{higherAntidiffeomult}
	{
		T,t
	}
	{
		\overline{\Omega}_T(t)
	}
	{
		\Omega_{-T}(t)
	}
	\\
	\Theorem{DiffeomultDecomp}
	{
		\forall R \in \ANN \.
		\forall M \in \LMOD{R} \. 
		\forall T : \TYPE{Alternating}(M,R) \.
		\forall t \in M^\wedge \.
		\forall m \in M \. \NewLine \. 
		\Omega_T(m \wedge t) = \Lambda_{T,m}\Omega_{T}(t)
	}
	\NoProof
}\Page{
	\Theorem{DiffeomultAndAntiderivationCommute}
	{
		\forall R \in \ANN \.
		\forall M \in \LMOD{R} \. 
		\forall T : \TYPE{Alternating}(M,R) \.
		\NewLine \. 
		\forall D \in \widetilde{\D}(M) \.  
		D\Omega_T = \Omega_{T} D
	}
	\Assume{t}{\TYPE{Decomposable}(M^\wedge)}
	\Assume{[0]}{\deg t = 0}
	\Say{[1]}{\bd \Omega_T}{ \Omega_T(t) \in R }
	\Conclude{[t.*]}{\bd \FUNC{MapOfDegree}(M^\wedge,-1)(D)[1]}{  D\omega_T(t) = 0 = \omega_TD(t) }
	\Derive{[0]}{I(\forall)}{ \forall t : \TYPE{Decomposable}(M^\wedge) \. \deg t = 0 \Rightarrow D \omega_T(t) = \omega_T D(t) } 
	\Assume{n}{\Int_+}
	\Assume{[1]}{\forall t : \TYPE{Decomposable}(M^\wedge) \. \deg t \le n \Rightarrow D \omega_T(t) = \omega_T D(t) }
	\Assume{t}{\TYPE{Decomposable}(M^\wedge)}
	\Assume{[2]}{\deg t = n + 1}
	\Say{\big(m,s,[3]\big)}{\bd \TYPE{Decomposable}(t)}{\sum m \in M \. \sum s : \TYPE{Decomposable}(M^\wedge) \. t = m \wedge s}
	\Say{[4]}{[2][3]}{\deg s = n}
	\Say{[5]}{ [3]\bd \TYPE{SkewDerivation}(D) \bd \LMOD{R}(M^\wedge,M^\wedge)(\Omega_T) \THM{DiffeomultComp}(T) \bd \Lambda_{T,m}  }
	{
		\NewLine :
		\Omega_T D(t) =
		\Omega_T D(m \wedge s) =
		\Omega_T \Big( D(m) s -  m \wedge D(s) \Big) = \NewLine =
		D(m) \Omega_T(s) - \Lambda_{T,m} \Omega_T(s) =
		D(m) \Omega_T(s) -  m \wedge \Omega_T(D(s)) -  D_{T,m} \Omega_T(D(s))
	}
	\Say{[6]}{[3] \THM{DiffeomultDecomp} \bd \Lambda_{T,m} \bd \TYPE{skewDerivation}(D) \THM{SkewDerivationAnticommute}(D_{T,m}, D)[1]\big(s,[4]\big)  }
	{
		\NewLine : 
		D \Omega_T (t) = 
		D \Omega_T (m \wedge s) =
		D\Big( m \wedge \Omega_T(s) +  D_{T,m} \Omega_T(s) \big) = \NewLine = 
		D(m) \Omega_T(s) -  m \wedge D\Omega_T(s) + D D_{T,m} \Omega_T(s) =
		D(m) \Omega_T(s) -  m \wedge D\Omega_T(s) - D_{T,m} D \Omega_T(s) = \NewLine = 
		D(m) \Omega_T(s) -  m \wedge \Omega_T(D(s)) - D_{T,m}  \Omega_T(D(s))  
	}
	\Conclude{[1.*]}{[5][6]}{\Omega_T D(t) = D \Omega_T(t)}
	\Derive{[1]}{\bd \TYPE{NaturalSet}(\Int_+)[0][1]}{\forall t : \TYPE{Decomposable}(M^\wedge) \. \Omega_T D(t) = D \Omega_T(t)}
	\Conclude{[*]}{\bd M^\wedge [1]}{\Omega_T D = D \Omega_T}
	\EndProof
}\Page{
	\Theorem{DiffeomultInverse}{
		\forall R \in \ANN \.
		\forall M \in \LMOD{R} \.
		\forall T : \TYPE{Alternating}(M,R) \.
		\Omega_T^{-1} = \overline{\Omega}_T
	}
	\Assume{t}{\TYPE{Decomposable}(M^\wedge)}
	\Assume{[0]}{\deg t = 0}
	\Say{[1]}{\bd \Omega_T\bd \overline{\Omega}_T(t)}{ \Omega_T(t) = t = \overline{\Omega}_T(t) }
	\Conclude{[t.*]}{[1]^2}{  \Omega_T \overline{\Omega}_T(t) = t \And \overline{\Omega}_T \Omega_T(t) = t }
	\Derive{[0]}{I(\forall)}{ \forall t : \TYPE{Decomposable}(M^\wedge) \. \deg t = 0 \Rightarrow  \Omega_T  \overline{\Omega}_T(t) = \overline{\Omega}_T \Omega_T(t) = t } 
	\Assume{n}{\Int_+}
	\Assume{[1]}{\forall t : \TYPE{Decomposable}(M^\wedge) \. \deg t \le n \Rightarrow \Omega_T  \overline{\Omega}_T(t) = \overline{\Omega}_T \Omega_T(t) = t }
	\Assume{t}{\TYPE{Decomposable}(M^\wedge)}
	\Assume{[2]}{\deg t = n + 1}
	\Say{\big(m,s,[3]\big)}{\bd \TYPE{Decomposable}(t)}{\sum m \in M \. \sum s : \TYPE{Decomposable}(M^\wedge) \. t = m \wedge s}
	\Say{[4]}{[2][3]}{\deg s = n}
	\Say{[5]}{ [3] \THM{DiffeomultComp}^2(T) \bd \overline{\Omega}_T\bd^2 \Lambda_{T,m}  \THM{DiffeomultAndAntiderivativeCommute}\bd \ABEL(M^\wedge)   }
	{
		\NewLine :
		\Omega_T \overline{\Omega}_T(t) =
		\Omega_T \overline{\Omega}_T(m \wedge s) =
		\Omega_T \Big( m \wedge \overline{\Omega}_T(s) -  D_{T,m} \overline{\Omega}_T(s) \Big) = \NewLine =
		m \wedge \Omega_T \overline{\Omega}_T(s) + D_{T,m} \Omega_T \overline{\Omega}_T \overline{\Omega}_T(s) - \Omega_T D_{T,m} \overline{\Omega}_T(s) =
		m \wedge s    + D_{T,m} \Omega_T \overline{\Omega}_T (s) - D_{T,m} \Omega_T  \overline{\Omega}_T(s) =
		t 
	}
	\Say{[6]}{[3] \THM{DiffeomultComp}^2(T) \bd \overline{\Omega}_T\bd^2 \Lambda_{T,m}  \THM{DiffeomultAndAntiderivativeCommute}\bd \ABEL(M^\wedge) }
	{
		\NewLine :
		\overline{\Omega}_T \Omega_T(t) =
		\overline{\Omega}_T \Omega_T(m \wedge s) =
		\overline{\Omega}_T \Big( m \wedge \Omega_T(s) +  D_{T,m} \Omega_T(s) \Big) = \NewLine =
		m \wedge \overline{\Omega}_T \Omega_T(s) - D_{T,m} \overline{\Omega}_T \Omega_T (s) - \overline{\Omega}_T D_{T,m} \Omega_T(s) =
		m \wedge s    + D_{T,m}  \overline{\Omega}_T \Omega_T(s) - D_{T,m} \overline{\Omega}_T  \Omega_T(s) =
		t 
	}
	\Conclude{[1.*]}{[5][6]}{\Omega_T \overline{\Omega}_T(t) = t \And \overline{\Omega}_T \Omega_T(t) = t }
	\Derive{[1]}{\bd \TYPE{NaturalSet}(\Int_+)[0][1]}{\forall t : \TYPE{Decomposable}(M^\wedge) \. \Omega_T \overline{\Omega}_T(t) = t \And \overline{\Omega}_T \Omega_T(t) = t }
	\Conclude{[*]}{\bd M^\wedge [1]}{\Omega_T \overline{\Omega}_T = \id \And \overline{\Omega}_T \Omega_T = \id }
	\EndProof
}\Page{
	\Theorem{PfaffTHM}
	{
		\forall R \in \ANN \. 
		\forall M \in \LMOD{R} \. 
		\forall T : \TYPE{Alternating}(M,R) \. 
		\forall n \in \Nat \.
		\forall m : n \to M \. \NewLine \.  
		\det \Lambda i,j \in n \. T(m_i,m_j)  = \pi_0^2\Omega_T \bigwedge^n_{i=1} m_i
	}
	\Say{x}{ \bigwedge^n_{i=1} m_i }{ M^\wedge }
	\Say{[1]}{ \THM{SkewExtertiotAppByDet2}(T) \THM{SkewExtensionExteriorComp}\bd^{-1} \pi_0 \THM{DiffeomultExteriorAsComp}(T)  }
	{
		\NewLine :
		\det \Lambda i,j \in n \. T(m_i,m_j) = 
		(-1)^{n(n-3)/2} D^{\wedge}_{T,x}(x) =
		(-1)^{n(n-3)/2} \prod^{n-1}_{i=0} D_{T,m_{n-i}} \bigwedge^n_{i=1} m_i = \NewLine = 
		(-1)^{n(n-3)/2} \pi_0 \prod^{n-1}_{i=0} ( L_{m_{n-i}} +  D_{T,m_{n-i}} ) \bigwedge^n_{i=1} m_i =
		(-1)^{n(n-3)/2} \pi_0 \Lambda^\wedge_{T,x}(x)
	}
	\Say{\bar x}{\overline{\Omega}_T(x)}{M^\wedge }
	\Say{[2]}{\THM{DiffeomultInverse}(T)\ByConstr(\bar X) \bd \Omega_T \THM{DiffeomultExteriorComp} \bd^{-1} \Omega_T}{
		\NewLine : 
		\Lambda^\wedge_{T,x}(x) = 
		\Lambda^\wedge_{T,x}\big( \Omega_T(\bar x) \big) =
                \Lambda^\wedge_{T,x} \Lambda_{T,\bar x}^\wedge(1) =
		\Lambda^\wedge_{T,x \wedge \bar x}(1) = 
		\Omega_T(x \wedge \bar x) 
	}
	\Say{[3]}{\bd \overline{\Omega}_T \ByConstr \bar(x) \bd^{-1} \FUNC{genAlgebra} }
	{
		\bar x \in \Big\langle \{  m_i | i \in n  \} \Big\rangle
	}
	\Say{[4]}{\bd \FUNC{exteriorAlgebra}[3]}{x \wedge \bar x = \pi_0(\bar x) x}
	\Say{[6]}{[1][2][4]}{ \det \Lambda i,j \in n \. T(m_i,m_j) =  (-1)^{n(n-3)/2}\pi_0(\Omega_T(x))\pi_0(\overline{\Omega}_T(x))   }
	\Say{\Big(F,[7]\Big)}{\THM{DiffeomultExteriorComp}\bd^{-1}\TYPE{MapOfDegree}}{ \sum n \in \Nat \. \prod i \in n \.  \NewLine \. F_i : \TYPE{MapOfDegree}(M^\wedge,n -2i) \.  \Lambda^\wedge_{x} = \sum^n_{i=1} F_i }
	\Say{[8]}{\bd \bar  x [7]}{   \Lambda^\wedge_{\bar x}  = \sum^n_{i=1} (-1)^i F_i     } 
	\Say{[9]}{[8][7]\bd^{-1} \pi_0}{  \pi_0( \overline{\Omega}_T(x) ) = (-1)^{n/2}\pi_0(\Omega_T(x))  }
	\Say{[10]}{\bd (-1)}{  n : \TYPE{Even} \Rightarrow  (-1)^{n/2} = (-1)^{k(k-2)/3}   }
	\Say{[11]}{[8][7]}{ n : \TYPE{Odd} \Rightarrow  \pi_0(\Omega_T(x)) = 0 = \pi_0(\overline{\Omega}_T(x))   }
	\Say{[12]}{\THM{OddOrEven}[10][11]}{  \pi_0\Omega_T(x) = (-1)^{n(n+3)/2} \pi_0(\overline{\Omega}_T(x))  }
	\Conclude{[*]}{[12][6]}{  \det \Lambda i,j \in n \.  T(m_i,m_j) = \pi^2_0(\Omega_T(x))    }
	\EndProof
	\\
	\DeclareFunc{pfaffian}{
		\prod R \in \ANN \. 
		\prod M : \FM \And \FGM(R) \. \NewLine \. 
		\prod e : \TYPE{Basis}(M) \.
		\TYPE{Alternating}(M,R) \to R
	}
	\DefineNamedFunc{pfaffian}{T}{\pf_e T}{  \pi_0\left(\Omega_T \; \bigwedge^n_{i=1} e_i\right)  }
	\\
	\Theorem{PfaffianProperty}{
		\forall R \in \ANN \. 
		\forall M : \FM \And \FGM(R) \. \NewLine \. 
		\forall e : \TYPE{Basis}(M) \.
		\forall T : \TYPE{Alternating}(M,R) \.
		\pf_e^2 T = \det T^e
	}
	\NoProof
}
\Page{
	\Theorem{PfaffianChangeOfBasis}{
		\forall R \in \ANN \. 
		\forall M : \FM \And \FGM(R) \. 
		\NewLine \. 
		\forall e,e' : \TYPE{Basis}(M) \. 
		\pf_{e'} T = \det C_{e \to e'} \pf_{e} T
	}
	\Conclude{[*]}{ \bd \pf_{e'} T \bd^{-1} C_{e \to e'} \bd^{-1} \FUNC{exteriorAlgebraFunctor} \THM{DeterminantTHM}\bd \LMOD{R}(M^\wedge,R)(\Omega\pi_0)\bd^{-1} \pf_{e} T  }
	{
		\NewLine :
		\pf_{e'} T  = 
		\pi_0 \Omega\left( \bigwedge^p_{i=1} e_i'  \right) = 
		\pi_0 \Omega\left( \bigwedge^p_{i=1} C_{e \to e'} e_i  \right) =
		\pi_0 \Omega\left( C_{e \to e'}^{\wedge} \bigwedge^p_{i=1} e_i \right) =
		\pi_0 \Omega\left( (\det C_{e \to e'}) \bigwedge^p_{i=1} e_i \right) = \NewLine = 
		(\det C_{e \to e'})\pi_0 \Omega \left( \bigwedge^p_{i=1} e_i \right) =
		\det C_{e \to e'} \pf_e T
	}
	\EndProof
	\\
	\DeclareFunc{matrixPfaffian}{\prod R \in \ANN \. \prod n \in \Nat \. \TYPE{AlternatingMatrix}(R,n) \to R }
	\DefineNamedFunc{matrixPfaffian}{A}{\pf A}{\pf_e A_{e,e}}
	\\
	\Theorem{BlockDiagonalPfaffian1}
	{
		\forall R \in \ANN \.
		\forall n,m \in \Nat \. 
		\forall A : \TYPE{AlternatingMatrix}(R,n) \. \NewLine \. 
		\forall B : \TYPE{AlternatingMatix}(R,m) \. 
		\pf A \oplus B = \pf A \pf B 
	}
	\Say{[1]}{\bd \FUNC{matrixPfaffian} (A \oplus B) \bd \FUNC{pfaffian} \bd \Omega \bd A \oplus B  }
	{
		\NewLine  
		\pf (A \oplus B) = 
		\pi_0 \Big( \Omega_{(A \oplus B)_{e,e}} \; \bigwedge^{n+m}_{i=1} e_i \Big) =    
		\pi_0  \prod^{n + m}_{i=1} ( L_{e_i} + T_{(A\oplus B)_{e,e},e_i} ) \bigwedge^{n+m}_{i=1} e_i =  \NewLine =  
		\pi_0  \prod^n_{i=1} ( L_{e_i} + T_{(A \oplus B)_{e,e},e_i} )  \prod^{n+m}_{i=n + 1} (L_{e_i} + T_{B_{e,e},e_i}) \bigwedge^{n+m}_{i=1} e_i 
	}
	\Say{\Big( x, [2] \Big)}{ \bd \pf B[1]  }
	{
		\sum x : m \to R^{n + m \wedge} \.  
		\pf (A \oplus B) = \pi_0 \prod^n_{i=1} ( L_{e_i} + T_{(A \oplus B)_{e,e},e_i} ) \left( \pf B \bigwedge^n_{i=1} e_i  + \sum^m_{i=1} e_{i+1} \wedge x_i  \right)
	}
	\Say{\Big(y, [*] \Big)}
	{
		\bd T \bd^{-1} \pf A \bd \pi_0[2]
	}
	{
		\sum y : m \to R^{n + m \wedge} \. 
		\pf (A \oplus B) = \pi_0 \left(  \pf A \pf B + \sum^m_{i=1} e_{i+1} \wedge y_i \right) = \pf A \pf B 
	}
	\EndProof
	\\
	\DeclareFunc{doublyReducedMatrix}{\prod X \in \SET \. \prod n \in \Nat \. X^{n \times n} \to n \times n \to X^{(n-2) \times (n-2)} }
	\DefineNamedFunc{doublyReducedMatrix}{ A,(i,j) }{ \widehat A_{((i,j))} }{ \widehat{\Big(\widehat A_{(i,j)}\Big)}_{(j,i)} }
	\\
	\Theorem{PfaffianFormula}
	{
		\forall R \in \ANN \.
		\forall n \in \Nat \. 
		\forall A : \TYPE{AlternatingMatrix}(R,n) \. \NewLine \. 
		\pf A = \sum^{n}_{i=1} (-1)^i A_{1,i} \pf  \widehat A_{((1,i))}
	}
	\NoProof
}
\newpage
\subsection{Symmetric Algebra}
\Page{
	\DeclareType{SymmetricAlgebra}{\prod R \in \ANN \. \prod M \in \LMOD{R} \. ? \sum S : \LCALGE{R} \. M \Arrow{\LMOD{R}} S }
	\DefineType{(S,\iota)}{SymmetricAlgebra}{ \forall A \in \LCALGE{R} \. \forall \varphi : M \Arrow{\LMOD{R}} A \. \exists! f : S \Arrow{\LALGE{R}} A : \varphi = \iota f  }
	\\
	\Theorem{IsomorphicTensorAlgebras}{\forall R \in \ANN \. \forall M \in \LMOD{R} \.  \forall (S,\iota),(S',\iota') : \TYPE{SymmetricAlgebra}(M) \. 
		\NewLine \. T \cong_{\LCALGE{R}} T'}
	\NoProof
	\\
	\Theorem{SymmetricAlgebraUniversalInjective}{\forall R \in \ANN \. \forall M \in \LMOD{R} \. \NewLine \. \forall (S,\iota) : \TYPE{SymmetricAlgebra}(M) \. \iota : M \ToInj T }
	\NoProof
	\\
	\DeclareFunc{symmetricAlgebra}{ \prod R \in \ANN \. \LMOD{R} \to \LCALGE{R}  }
	\DefineNamedFunc{symmetricAlgebra}{M}{M^{\vee}}{\frac{M^\otimes}{\Big\langle \{ x \otimes y - y \otimes x | x,y \in M  \}  \Big\rangle }  }
	\\
	\DeclareFunc{symmetricProduct}{ \prod R \in \ANN \. \prod M \in \LMOD{R} \. \L( M^\vee,M^\vee;M^\vee)  }
	\DefineNamedFunc{symmetricProduct}{ [t],[s]  }{ [t] \vee [s]}{  [ t \otimes s ] }
	\\
	\DeclareFunc{symmetricEmbedding}{ \prod R \in \ANN \. \prod M \in \LMOD{R} \. M \Arrow{\LMOD{R}} M^\vee  }
	\DefineNamedFunc{symmetricEmbedding}{ m  }{ \iota^\vee_M(m)}{ \Big[\iota^\otimes_M(m)\Big]   }
	\\
	\Theorem{SymmetricAlgebraTHM}
	{
		\forall R \in \ANN \. \forall M \in \LMOD{R} \. 
		(M^\vee,\iota^\vee_M) : \TYPE{SymmetricAlgebra}(R,M)
	}
	\NoProof
	\\
	\DeclareFunc{symmetricMapping}{\prod R \in \ANN \. \prod M,N \in \LMOD{R} \. ( M \Arrow{\LMOD{R}} N) \to (M^\vee \Arrow{\LCALGE{R}} N^\vee) } 
	\DefineNamedFunc{symmetricMapping}{f }{f^\vee }{ \bd \TYPE{SymmetricAlgebra}(R,M)(M^\vee)( f \iota^\vee_N)}
	\\
	\DeclareFunc{symmetricFunctor}{ \prod R \in \ANN \. \LMOD{R} \Arrow{\CAT} \LCALGE{R} }
	\DefineFunc{symmetricFunctor}{ }{\Big(\FUNC{symmetricAlgebra},\FUNC{symmetricMap}\Big)}
}
\Page{
	\Theorem{BasisOfSymmetricAlgebra}
	{
		\forall R \in \ANN \. 
		\forall M \in \FM(R) \. 
		\forall E : \Basis(R) \. \NewLine \. 
		\left\{  \bigvee^n_{i=1} e_i | e : \TYPE{Nondecreasing}\Big(n,(E,o)\Big)  \right\} : \TYPE{Basis}(M^\vee)
		\quad \where \quad  o = \THM{wellOrderingTHM}(E)
	}
	\NoProof
	\\
	\Theorem{FreeSymmetricAlgebra}
	{
		\forall R \in \ANN \.
		\forall M \in \FM(R) \. 
		M^{\vee} : \FM(R) 
	}
	\NoProof
	\\
	\Theorem{SymmetricAlgebra}
	{
		\forall R \in \ANN \.
		\forall M \in \FM(R) \. 
		M^{\vee} : \FM(R) 
	}
	\NoProof
	\\
	\Theorem{SymmetricAlgebraDirectSum}
	{
		\forall R \in \ANN \. 
		\forall n \in \Nat
		\forall M : n \to \LMOD{R} \. 
		\forall \left( \bigoplus^n_{i=1} M_i  \right)^\vee \cong_{\LCALGE{R}}  \bigotimes^n_{i=1} M_i^\vee 
	}
	\NoProof
	\\
	\Theorem{SymmetricAlgebraPoincareSeries}
	{
		\forall R \in \ANN \.
		\forall M : \FM \And \FGM(R) \. \NewLine \.
		P(M^\vee)(x) =  \frac{1}{(1 - x)^n}  \quad \where \quad n = \rank M  
	}
	\Say{[1]}{\THM{SymmetricAlgebraBasis}}{\forall n \in \Nat \. \dim R^\vee_n = 1}
	\Say{[2]}{\bd \FUNC{seriesOfPoincare}[1]}{P(R^\vee)(x) = \frac{1}{1 - x}}
	\Say{[3]}{\THM{FreeAsSum}(M)}{M = \bigoplus^n_{i=1} R}
	\Say{[4]}{\THM{SymmetrcAlgebraDirectSum}}{ M^\wedge \cong_{\LCALGE{R}} \bigotimes^n_{i=1} R^\wedge }
	\Conclude{[*]}{\THM{PoincareSeriesProduct}[2][4]}{P(M^\vee)(x) = \frac{1}{(1-x)^n}}
	\EndProof
}\Page{
	\Theorem{SymmetricAlgebraDimension}
	{
		\forall R \in \ANN \. 
		\forall M : \FM \And \FGM(R) \. \NewLine \.
		\forall p \in \Int_+ \. 
		\dim M^\vee_p = \binom{n+p-1}{p}           
		\quad \where \quad n = \rank M 
	}
	\Say{[1]}{\THM{SymmetricAlgebraPoincareSeries}(M)\THM{FractionDiff}\;\THM{GeometricSeries}\;\THM{SeriesDiff}\bd^{-1}\FUNC{binom} }
	{
		\NewLine
		P(M^\wedge)(x) =
		\frac{1}{(1-x)^n} = 
		\frac{d^{(n-1)}}{dx^{n-1}} \frac{1}{(n-1)!(1-x)} = 
		\frac{d^{(n-1)}}{dx^{n-1}} \sum^\infty_{p=0}   \frac{x^p}{(n-1)!}  =
		\sum^\infty_{p=0} \frac{d^{n-1}}{dx^{n-1}}  \frac{x^p}{(n-1)!} = \NewLine =  
		\sum^{\infty}_{p=0} \frac{(n+p-1)!}{ (n-1)! p!  } x^p = 
		\sum^{\infty}_{p=0} \binom{n+p-1}{p} x^p
	}
	\Conclude{[*]}{\bd \FUNC{PoincareSeries}[1] }{ \forall p \in \Int_+ \. \dim M^\vee_p = \binom{n+p-1}{p} }
	\EndProof
	\\
	\Theorem{ProjectiveSymmetricAlgebra}{
		\forall R \in \ANN \. 
		\forall M : \TYPE{Projective}(R) \.
		 M^\vee : \TYPE{Projective} 
	}
	\NoProof
	\\
	\Theorem{SymmetricCovariantExtension}
	{
		\forall A,B \in \ANN \.
		\forall \omega A \Arrow{\RING} B \.
		\forall M \in \LMOD{A} \. \NewLine \.  
		M^\vee \otimes_\omega B \cong_{\LALGE{B}} (M \otimes_\omega B)^\vee 
	}
	\NoProof
	\\
	\Theorem{SymmetricDerivationExtension}
	{
		\forall R \in \ANN \. 
		\forall M \in \LMOD{R} \.
		\forall f \in M^* \. 
		\exists! D \in \D(M^\vee) :
		D_{|M^\vee_1} = f
	}
	\NoProof
	\\
	\Theorem{SymmetricGeneralisedDerivationExtension}
	{
		\forall R \in \ANN \.
		\forall M \in \LMOD{R} \.
		\forall n \in \Nat \. \NewLine \. 
		\forall  f \in  \Big(M^\wedge_n\Big)^* \.  
		\exists! D \in \D^n(M^\vee) :
		D_{M^\vee_n} = f 
	}
	\NoProof 
	\\
	\Theorem{SymmetricAlgebraQuotient}
	{
		\forall R \in \ANN \.
		\forall I : \TYPE{Ideal}(R) \. 
		\forall M \in \LMOD{R} \. 
		\left( \frac{M}{IM} \right)^\vee
		\cong_{\LALGE{\frac{R}{I}}}  
		\frac{M^\vee}{IM^\vee}
	}
	\NoProof
}
\newpage
\subsection{Algebraic Differentiation}
\Page{
	\DeclareType{DifferentialOperator}{ \prod R \in \ANN \. \prod M \in \LMOD{R} \. \prod n \in \Nat \.  \TYPE{MapOfDegree}(M^\vee,n) }
	\DefineNamedType{F}{DifferentialOperator}{F \in \nabla^n M}{
		\exists! f \in (M^\vee_n)^* : \forall k : \TYPE{After}(n) \. \NewLine \.  
		\forall m : k \to M \. F\left( \bigvee^k_{i=1} m_i \right) = \sum_{I : n \uparrow k} f\left( \bigvee^{m}_{i=1} m_{I_i} \right) \bigvee^{k - n}_{i=1} m_{I^\c_i}   
	}
	\\
	\Theorem{DifferentialOperatorComposition}{\forall R \in \ANN \. \forall M \in \LMOD{R} \. \forall n,m \in \Nat \. \forall A \in \nabla^n(M) \. 
		\NewLine \. \forall B \in \nabla^m(M) \. AB \in \nabla^{n + m}(M)}
	\Say{(a,[1])}{\bd \nabla^n(M)(A) }{\sum a : M^\vee_n \Arrow{\LMOD{R}} R \. \ldots}
	\Say{(b,[2])}{\bd \nabla^m(M)(A)}{ \sum b : M^\vee_m \Arrow{\LMOD{R}} R \. \ldots}
	\Assume{K}{\Nat}
	\Assume{[3]}{K \ge n + m}
	\Assume{x}{K \to M}
	\Say{[K.*.1]}
	{ [1](K,x)\bd\LMOD{R}(M^\wedge)(b)[3][2](K-n,\ldots) \THM{RearangeArange}(\ldots) }
	{
		\NewLine :
		AB\left( \bigvee^K_{i=1} x_i\right) =
		B\left(  \sum_{I : n \uparrow K} a\left( \bigvee^n_{i=1} x_{I_i} \right) \bigvee^{K-n}_{i=1} x_{I^\c_i}   \right) = 
	        \sum_{I : n \uparrow K} a\left( \bigvee^n_{i=1} x_{I_i} \right) B \bigvee^{K-n}_{i=1} x_{I^\c_i} = \NewLine =   
		\sum_{I : n \uparrow K} a\left( \bigvee^n_{i=1} x_{I_i} \right) \sum_{J : m \uparrow K - n} b\left( \bigvee^m_{i=1} x_{I_{J_i}^\c} \right) \bigvee^{K-n-m}_{i=1} x_{I_{J_i^\c}^\c} = \NewLine = 
		\sum_{L : n + m \uparrow K} \left( \sum_{H : n \uparrow n + m} a\left( \bigvee^n_{i=1 }x_{L_{H_i}} \right) b\left( \bigvee^m_{i=1} x_{L_{H_i^\c}} \right) \right) \bigvee^{K-n-m}_{i=1} x_{L^\c_i}
	}
	\Say{f}{ \bd M^\vee_{n+m} \Lambda x : (n +m) \to M \. \sum_{H : n \uparrow n + m} a\left( \bigvee^n_{i=1 }x_{H_i} \right) b\left( \bigvee^m_{i=1} x_{H_i^\c} \right) }{ M^\vee_{n+m} \to R }
	\Conclude{[K.*.2]}{\ByConstr f\bd \LMOD{R}(M^\wedge_a,R)(a) \bd \LMOD{R}(M^\wedge_m,R)(b)}{\Big( f : M^\wedge_{n+m} \Arrow{\LMOD{R}} R \Big)}
	\DeriveConclude{[*]}{\bd \nabla^{n+m}(M)}{AB \in \nabla^{n+m}(M)}
	\EndProof
}\Page{
	\Theorem{DifferentialOperatorsCommute}{
		\forall R \in \ANN \. 
		\forall M \in \LMOD{R} \. 
		\forall n,m \in \Nat \. \NewLine \.  
		\forall A \in \nabla^n(M) \. 
		\forall B \in \nabla^m(M) \.
		AB = BA
	}
	\Say{(a,[1])}{\bd \nabla^n(M)(A) }{\sum a : M^\vee_n \Arrow{\LMOD{R}} R \. \ldots}
	\Say{(b,[2])}{\bd \nabla^m(M)(A)}{ \sum b : M^\vee_m \Arrow{\LMOD{R}} R \. \ldots}
	\Assume{K}{\Nat}
	\Assume{[3]}{K \ge n + m}
	\Assume{x}{K \to M}
	\Conclude{[K.*]}
	{ [1](K,x)\bd\LMOD{R}(M^\wedge)(b)[3][2](K-n,\ldots) \THM{RearangeArange}(\ldots) }
	{
		\NewLine :
		AB\left( \bigvee^K_{i=1} x_i\right) =
		B\left(  \sum_{I : n \uparrow K} a\left( \bigvee^n_{i=1} x_{I_i} \right) \bigvee^{K-n}_{i=1} x_{I^\c_i}   \right) = 
	        \sum_{I : n \uparrow K} a\left( \bigvee^n_{i=1} x_{I_i} \right) B \bigvee^{K-n}_{i=1} x_{I^\c_i} = \NewLine =   
		\sum_{I : n \uparrow K} a\left( \bigvee^n_{i=1} x_{I_i} \right) \sum_{J : m \uparrow K - n} b\left( \bigvee^m_{i=1} x_{I_{J_i}^\c} \right) \bigvee^{K-n-m}_{i=1} x_{I_{J_i^\c}^\c} = \NewLine = 
		\sum_{L : n + m \uparrow K} \left( \sum_{H : n \uparrow n + m} a\left( \bigvee^n_{i=1 }x_{L_{H_i}} \right) b\left( \bigvee^m_{i=1} x_{L_{H_i^\c}} \right) \right) \bigvee^{K-n-m}_{i=1} x_{L^\c_i}
		= \NewLine = 
		\sum_{L : n + m \uparrow K} \left( \sum_{H : m \uparrow n + m}  b\left( \bigvee^m_{i=1} x_{L_{H_i^\c}} \right) a\left( \bigvee^n_{i=1 }x_{L_{H_i}} \right)\right) \bigvee^{K-n-m}_{i=1} x_{L^\c_i}
		= \NewLine = 
		BA\left( \bigvee^K_{i=1} x_i \right)
	}
	\DeriveConclude{[*]}{\bd M^\vee I(=,\to)}{ AB = BA }
	\EndProof
	\\
	\Theorem{ExtensionToDifferentialOperator}{
		\forall R \in \ANN \.
		\forall M \in \LMOD{R} \.
		\forall n \in \Nat \. \NewLine \. 
		\forall f \in M^\vee_n \Arrow{\LMOD{R}} R \.
		\exists! A \in \nabla^n(M) \. 
		A_{|M^\vee_{n}} = f
	}
	\NoProof
	\\
	\Theorem{DifferentialOperatorsAsFunctional}
	{
		\forall R \in \ANN \.
		\forall M \in \LMOD{R} \.
		\forall n \in \Nat \. \NewLine \.
		\Big(M^\vee_n\Big)^* \cong_{\LMOD{R}}  \nabla^n(M)  
	}
	\\
	\DeclareFunc{algebraOfDifferentialOperators}{\prod R \in \ANN \. \LMOD{R} \to \LCALGE{R}(\Int)  }
	\DefineNamedFunc{algebraOfDifferentialOperators}{M}{\nabla M}{ \left( \bigoplus_{n=0}^\infty \nabla^n M, \Lambda n \in \Int \. \If n \ge 0 \Then \nabla^n M \Else 0   \right)}
}
\newpage
\Page{
	\DeclareFunc{partialDifferentiation}{\prod R \in \ANN \. \prod X \in \SET \. X \to \nabla^1 R^X}
	\DefineNamedFunc{partialDifferential}{\alpha}{\frac{\partial}{\partial x_\alpha}}{ \bd \THM{ExtensionToDifferentialOperators}(e_\alpha^*)}
	\\
	\DeclareFunc{higherPartialDifferentiation}{\prod R \in \ANN \. \prod X \in \SET \. \prod n \in \Nat \. \Big(n \ToInj X \And n \to \Nat\Big) \to \nabla R^X }
	\DefineNamedFunc{higherPartialDifferential}
	{ \alpha,m  }{  \frac{\partial^{\sum^n_{i=1}m_i}}{\prod^n_{i=1} \partial x_{\alpha_i}^{m_i}} }
	{\prod^n_{i=1} \left( \frac{\partial}{\partial x_{\alpha_i} } \right)^{m_i}  }
	\\
	\Theorem{HigherPolynomialDifferentiation}{\forall R \in \ANN \. \forall n \in \Nat \. \forall \alpha : n \to \Int_+ \. \forall \beta : n \to \Int_+ \.  
		\NewLine \.  
		\prod^n_{i=1} \left( \frac{\partial}{\partial x_i} \right)^{\alpha_i} \bigwedge^n_{i=1} e_i^{\vee \beta_i} = \prod^n_{i=1} \frac{\beta_i!}{(\beta_i - \alpha_i)!} \bigwedge^n_{i=1} e_i^{\vee \beta_i-\alpha_i}  
	}
	\NoProof
	\\
	\DeclareFunc{standartDifferentialOperator}{\prod R \in \ANN \. \prod X \in \SET \. \prod n \in \Nat \. \Big(n \ToInj X \And n \to \Nat\Big) \to \nabla R^X }
	\DefineNamedFunc{standartDifferentialOperator}
	{ \alpha,m  }{  \mathrm{D}^{\alpha,m} }
	{  \prod^n_{i=1} \frac{1}{\alpha_i!} \frac{\partial^{\sum^n_{i=1} \alpha_i}}{\prod^n_{i=1}\partial x_i}  }
	\\
	\Theorem{MultinomialComposition}
	{
		\forall R \in \ANN \.
		\forall X \in \SET \.
		\forall n \in \Nat \.
		\forall \alpha,\beta : n \to  \Int_+ \.
		\mathrm{D}^\alpha \mathrm{D}^\beta = \binom{\alpha + \beta}{\alpha} D^{\alpha + \beta} 
	}
	\NoProof
	\\
	\DeclareFunc{permanent}
	{
	    \prod R \in \ANN \. 
	    \prod n \in \Nat \.
	    R^{n \times n} \to R
	}
	\DefineNamedFunc
	{permanent}{A}{\perm(A)}{ \sum_{\sigma \in S_n} A_{i,\sigma(i)}}
	\\
	\DeclareFunc{billinearAsDifferential}
	{
	    \prod R \in \ANN \. 
	    \prod A,B \in \LMOD{R} \.
	    \L(A,B;R) \to A \to \nabla B
	}
	\DefineNamedFunc
	{billinearAsDifferential}{\gamma, a}{D^\gamma_a}{  \THM{ExtensionToDifferentialOperators}(\Lambda b \in B \. \gamma(a,b))  }
	\\
	\Theorem{PermanentComposition}
	{
		\forall R \in \ANN \.
		\forall A,B \in \LMOD{R} \. 
		\forall n \in \Nat \. 
		\forall a : n \to A \.
		\forall b : n \to B \. \NewLine \. 
		\prod^n_{i=1} D^\gamma_{a_i} \bigwedge^n_{i=1} b_i = \perm ( \gamma(a_i,b_i))^n_{i,j=1}
	}
	\NoProof
}
\newpage
\subsection{Grassmann Algebra}
\Page{
	\DeclareFunc{exteriorComultiplication}
	{
		\prod R \in \ANN \.
		\prod A \in \LMOD{R} \.
		A^\wedge \Arrow{\LALGE{R}(\Int)} A^\wedge \widetilde{\otimes} A^\wedge
	}
	\DefineNamedFunc{exteriorComultiplication}
	{ }{\Delta}{\bd A^\wedge \Lambda a \in A \. a \otimes 1 + 1 \otimes a }
	\Assume{a,b}{A}
	\Say{[1]}{\bd \Delta \bd \FUNC{twistedTensorProduct}}{
		\NewLine :
		\Delta\Big(a \wedge b\Big)
		(a \otimes 1 + 1 \otimes a)(b \otimes 1 + 1 \otimes b) = 
		(a \wedge b) \otimes 1 +  a \otimes b  - b \otimes a + 1 \otimes (a \wedge b)  
	}
	\Say{[2]}{\bd \Delta \bd \FUNC{twistedtensorProduct} \bd A^\wedge}{
		\NewLine :
		\Delta\Big( b \wedge a \Big) 
		(b \otimes 1 + 1 \otimes b)(a \otimes 1 + 1 \otimes a) = 
		(b \wedge a) \otimes 1 +  b \otimes a  - a \otimes b + 1 \otimes (b \wedge a) = \NewLine =  
	       -(a \wedge b) \otimes 1 -  a \otimes b  + b \otimes a - 1 \otimes (a \wedge b)  
	}
	\Conclude{[a,b.*]}{\bd \Delta[1][2]}{ \Delta(a \wedge b + b \wedge a) = \Delta(a \wedge b) + \Delta(b \wedge a) = 0} 
	\DeriveConclude{[*]}{\bd \Delta \bd E^\wedge}{\LOGIC{WellDefinied}(\Delta)}  
	\EndProof
	\\
	\DeclareFunc{exteriorCounit}
	{
		\prod R \in \ANN \.
		\prod A \in \LMOD{R} \.
		A^\wedge \Arrow{\LALGE{R}(\Int)} R
	}
	\DefineNamedFunc{exteriorCounit}{}{\eta}{\bd \LALGE{R}(A^\wedge,R)(0)}
	\\
	\DeclareFunc{exteriorAntipode}
	{
		\prod R \in \ANN \.
		\prod A \in \LMOD{R} \.
		A^\wedge \Arrow{\LALGE{R}(\Int)} A^\wedge
	}
	\DefineNamedFunc{exteriorAntipode}{}{\sigma}{\bd \LALGE{R}(A^\wedge,R)(-{\id}_A)}
	\\
	\Theorem{ExteriorAlgebraIsASkewCoalgebra}
	{
		\forall R \in \ANN \.
		\forall A \in \LMOD{R}
		(A^\wedge,\Delta,\eta) \in \SCOALG{R}(\Int) 
	}
	\NoProof
	\\
	\Theorem{ExteriorAlgebraIsATwistedHopfAlgebra}
	{
		\forall R \in \ANN \.
		\forall A \in \LMOD{R} \.
		A^\wedge \in \widetilde{\HOPF{R}}
	}
	\NoProof
	\\
	\Theorem{ExteriorAlgebraMapIsAHopfMorphism}
	{
		\forall R \in \ANN \.
		\forall A,B \in \LMOD{R} \.
		\forall f : A \Arrow{\LMOD{R}} B \. 
		\NewLine \. 
		f^\wedge : A^\wedge \Arrow{\widetilde{\HOPF{R}}} B^\wedge
	}
	\NoProof
} \Page{
	\DeclareFunc{disjointSequenceSum}
	{
		\prod A \in \ABEL
		\prod n \in \Int_+ \. \NewLine \. 
		\left( \sum k,l \in \Int_+ \. \sum [0] : k + l = 0 \.  
		\bigg(\Big(k \uparrow [n]_\Nat\Big) \times \Big(l \uparrow [n]_\Nat\Big)\bigg)  
		\to A \right) \to A
	}
	\DefineNamedFunc{disjointSequenceSum}{F}{\sum_{I \sqcup J \uparrow n} F(I,J) } 
	{ \NewLine \de \sum (I,J) \in	\bigg\{ (I,J) \in   \bigg(k \uparrow [n]_\Nat\Big) \times \Big(l \uparrow [n]_\Nat\Big) 
		:  \im I \cap \im J = \emptyset
		\Big| k,l \in \Int_+ : k + l = n\bigg\} \. F(I,J)  }
	\\
	\Theorem{ComultiplicationOfExteriorProduct}{
		\forall R \in \ANN \.
		\forall A \in \LMOD{R} \.
		\forall n \in \Nat \.
		\forall a : n \to A \. \NewLine \.
		\Delta\left(\bigwedge^n_{i=1} a_i \right) =
		\sum_{I \sqcup J \uparrow n}  (-1)^{I,J}\left( \bigwedge_{i \in \dom I} a_{I_i} \right) \otimes
		\left( \bigwedge_{j \in \dom J} a_{J_j} \right) 
	}
	\Say{\mars}{
		\Lambda n \in \Nat \. 
		\forall m \in n \. 
		\forall a : m \to A \.  
		\Delta\left(\bigwedge^n_{i=1} a_i \right) =
		\sum_{I \sqcup J \uparrow n}  (-1)^{I,J}\left( \bigwedge_{i \in \dom I} a_{I_i} \right) \otimes
		\left( \bigwedge_{j \in \dom J} a_{J_j} \right)  
	}
	{\Nat \to \Type}
	\Say{[1]}{\bd \Delta \bd^{-1}(-1)^{I,J} \bd^{-1}\FUNC{disjointSequenceSum}}{
		\mars(1)
	}
	\Assume{m}{\Nat}
	\Assume{[2]}{\mars(m)}
	\Assume{a}{(m+1) \to A}
	\Say{\I}{\bigg\{ (I,J) \in   \bigg(k \uparrow [n]_\Nat\Big) \times \Big(l \uparrow [n]_\Nat\Big) 
		:  \im I \cap \im J = \emptyset
		\Big| k,l \in \Int_+ : k + l = m\bigg\}}{\SET} 
	\Say{\I_+}{\bigg\{ (I,J) \in   \bigg(k \uparrow [n]_\Nat\Big) \times \Big(l \uparrow [n]_\Nat\Big) 
		:  \im I \cap \im J = \emptyset
		\Big| k,l \in \Int_+ : k + l = m+1\bigg\}}{\SET} 
	\Conclude{\Big(s,[3]\Big)}
	{
		\bd \LALGE{R}(A^\wedge,A^\wedge \otimes A^\wedge)
		\bd \Delta
		\bd \LALGE{R}(A^\wedge)\bd^{-1}
		\bd^{-1}\FUNC{disjointSequenceSum}
	}
	{
		\NewLine :
		\sum s : \I_+ \to \{-1,1\} \. \NewLine \. 
		\left( \bigwedge_{i=1}^{m+1} a_i \right) =
		\bigwedge^n_{i=1} \Delta(a_i) =
		\bigwedge^n_{i=1} (a_i \otimes 1 + 1 \otimes a_i) =  
		\sum_{I \sqcup J \uparrow (m+1)}  s_{I,J}\left( \bigwedge_{i \in \dom I} a_{I_i} \right) \otimes
		\left( \bigwedge_{j \in \dom J} a_{J_j} \right) 
	}
	\EndProof
	\Say{[4]}{[2](a_{|m})}{
		\Delta\left(\bigwedge^m_{i=1} a_i \right) =
		\sum_{I \sqcup J \uparrow m}  (-1)^{I,J}\left( \bigwedge_{i \in \dom I} a_{I_i} \right) \otimes
		\left( \bigwedge_{j \in \dom J} a_{J_j} \right) 
	}
	\Say{[5]}{[1]\bd \LALGE{R}(A^\wedge,A^\wedge \widetilde{\otimes} A^\wedge)[2] }{ \NewLine : 
		\sum_{I \sqcup J \uparrow (m+1)}  s_{I,J}\left( \bigwedge_{i \in \dom I} a_{I_i} \right) \otimes
		\left( 
		\bigwedge_{j \in \dom J} a_{J_j} \right)  = 
		\Delta\left(\bigwedge_{i=1}^{m+1} a_i \right)  =
		\Delta\left(\bigwedge_{i=1}^m a_i \right) \Delta(a_{m+1}) = \NewLine = 
		\left(\sum_{I \sqcup J \uparrow m}  (-1)^{I,J}\left( \bigwedge_{i \in \dom I} a_{I_i} \right) \otimes
		\left( \bigwedge_{j \in \dom J} a_{J_j} \right) \right)\Big( (a_{m+1} \otimes 1) + (1 \otimes a_{m+1})\Big) 
	}
}\Page{
	\Assume{(I,J)}{\I}
	\Say{[6]}{ \bd \FUNC{twistedTensorProduct}  }
	{
	   (-1)^{I,J}\left( \bigwedge_{i \in \dom I} a_{I_i} \right) \otimes
		\left( \bigwedge_{j \in \dom J} a_{J_j} \right) 
		 (a_{m+1} \otimes 1) = \NewLine =
	   (-1)^{I,J}(-1)^{|J|}  \left( \bigwedge_{i \in \dom \big(I \sqcup (m+1)\big)} a_{(I\sqcup(m+1))_i} \right) 
	   \otimes \left( \bigwedge_{j \in J} a_{J_j} \right)
	}
	\Say{[7]}{ \bd \FUNC{twistedTensorProduct}  }
	{
	   (-1)^{I,J}\left( \bigwedge_{i \in \dom I} a_{I_i} \right) \otimes
		\left( \bigwedge_{j \in \dom J} a_{J_j} \right) 
		 (1 \otimes a_{m+1}) = \NewLine = 
	   (-1)^{I,J}  \left( \bigwedge_{i \in \dom \big(I \sqcup (m+1)\big)} a_{(I\sqcup(m+1))_i} \right) 
	   \otimes \left( \bigwedge_{j \in \dom(J\sqcup(m+1))} a_{(J_j\sqcup(m+1))_j} \right)
	}
	\Say{[8]}{\bd \FUNC{permutatioSign}\bd \FUNC{doubleIncreasingAsPermutation}}{
		(-1)^{I,J}(-1)^|J| = (-1)^{I\sqcup(m+1),J} 
	}
	\Say{[9]}{\bd \FUNC{permutationSign} \bd \FUNC{doubleIncreasignAsPermutation}}
	{
		(-1)^{I,J} = (-1)^{I,J\sqcup (m+1)}
	}
	\Conclude{\Big[(I,J).*\Big]}{[3][6][7][8][9]}
	{
		s_{I\sqcap(m+1),J} = (-1)^{I\sqcap(m+1),J} 
		\And
		s_{I,J\sqcap(m+1)} = (-1)^{I,J\sqcap(m+1)}
	}
	\DeriveConclude{[m.*]}{I(\forall)[3]\bd^{-1}\mars}
	{
		\mars(m+1)
	}
	\DeriveConclude{[*]}{\bd \Nat \bd \mars}{ \LOGIC{This}}
	\EndProof
	\\
	\DeclareFunc{exteriotDualProduct}
	{
		\prod R \in \ANN \. 
		\prod A \in \LMOD{R}
		A^{\wedge*}\otimes A^{\wedge*} \to A^{\wedge*}
	}
	\DefineNamedFunc{exteriorDualProduct}{f,g}{f \wedge g}{\Delta_A(f \otimes g)\mu_R}
	\\
	\Theorem{ExteriorDualProductAction}{
		\forall R \in \ANN \.
		\forall A \in \LMOD{R} \.
		\forall n \in \Nat \.
		\forall a : n \to A \. \NewLine \.
		\forall f,g \in A^{\wedge*}
		(f \wedge g)\left(\bigwedge^n_{i=1} a_i \right) =
		\sum_{I \sqcup J \uparrow n}  (-1)^{I,J}f\left( \bigwedge_{i \in \dom I} a_{I_i} \right) 
		g\left( \bigwedge_{j \in \dom J} a_{J_j} \right) 
	}
	\NoProof
	\\
	\DeclareFunc{GrassmannAlgebra}
	{
		\prod R \in \ANN \.
		\LMOD{R}  \Arrow{\CAT} \widetilde{\HOPF{R}} 
	}
	\DefineNamedFunc{GrassmannAlgebra}{A}{\mathfrak{G}(A)}{\mathfrak{D}(A^\wedge)}
	\\
	\Theorem{GrassmannIsomorphism}
	{
		\forall R \in \LMOD{R} \.
		\forall A : \FGM(R) \.
		\mathfrak{G}(A) \cong_{\LMOD{R}(\Int)} A^\wedge
	}
	\NoProof
	\\
	\Theorem{StrongGrassmannIsomorphism}
	{
		\forall R \in \LMOD{R} \.
		\forall A : \FGM(R) \.
		\mathfrak{G}(A) \cong_{\widetilde{\HOPF{R}}} A^{*\wedge}
	}
	\NoProof
}\Page{
	\Theorem{ExteriorDualProductAction}{
		\forall R \in \ANN \.
		\forall A \in \LMOD{R} \.
		\forall n \in \Nat \.
		\forall a : n \to A \. \NewLine \.
		\forall f : m \to  A^{*}
		\left(\bigwedge^n_{i=1} f_i \right) \left(\bigwedge^n_{i=1} a_i \right) =
		\det \Big( f_i(m_j) \Big)_{i,j=1}^n 
	}
	\Say{\venus}{
			\Lambda n \in \Int_+ \. 
			\forall m \in n \.
			\forall a : m \to A \.
			\forall f : m \to A^* \.
			\left(\bigwedge^n_{i=1} f_i\right)
			\left(\bigwedge^n_{i=1} a_i\right) 
			= \det (f_i(a_j))^n_{i,j=1}
		}{\Int_+ \to \Type}
	\Say{[1]}{\bd^{-1} \det \bd^{-1}\venus}{\venus(0)\And \venus(1)}
	\Assume{m}{\Nat}
	\Assume{[2]}{\venus(m)}
	\Assume{a}{(m+1) \to A}
	\Assume{f}{(m+1) \to A^*}
	\Say{g}{ \bigwedge^m_{j=1} \widehat{f}_{1,j}}{ (m+1) \to A^{\wedge*} }
	\Conclude{[m.*]}
	{
		\ByConstr^{-1}(g) \THM{ExteriorDualProductAction}
		\ByConstr(g)\bd \venus[2] 
		\THM{DeterminantDecomposition}
	}
	{      
		\NewLine : 
		\left(\bigwedge^{m+1}_{i=1} f_i \right)
		\left( \bigwedge^{m+1}_{i=1} a_i \right) =
		(f_1 \wedge g)\left( \bigwedge^{m+1}_{i=1} a_i \right) = 
		\sum^n_{i=1}  (-1)^i f_1(a_i) g\left( \bigwedge^n_{j=1} \widehat{a}_{i,j}  \right) = \NewLine = 
		\sum^n_{i=1}  (-1)^i f_1(a_i) \det\Big( \hat f_{1,j}(\hat a_{i,l}) \Big)^m_{j,l} =
		\det (f_i(a_j))^n_{i,j=1}
	}
	\DeriveConclude{[*]}{[1]\bd \Int_+}{\LOGIC{This}}
	\EndProof
	\\
	\Theorem{ExterirorIntegrals}
	{
		\forall k : \Field \.
		\forall V \in \FDVS{k} \.
		\forall n \in \Nat \.
		\forall [0] : \dim V = n \.
		\int^l_{V^\wedge} = \int^r_{V^\wedge} = V^\wedge_n
	}
	\NoProof
	\\
	\Theorem{GrassmannIntegrals}
	{
		\forall k : \Field \.
		\forall V \in \FDVS{k} \.
		\forall n \in \Nat \.
		\forall [0] : \dim V = n \.
		\int^l_{\mathfrak{G}(V)} = \int^r_{\mathfrak{G}(V)} = \mathfrak{G}_n(V)
	}
	\NoProof
	\\
	\DeclareFunc{integralOfBerezin}
	{
		\prod k : \Field \.
		\prod V \in \FDVS{k} \.
		\prod n \in \Nat \.
		\prod e : \Basis(n,V) \.
		\int^l_{\mathfrak{G}(V)}
	}
	\DefineNamedFunc{integralOfBerezin}{\sum_{I \subset n} \alpha_I \bigwedge_{i \in I} e_i}
	{ \mathbf{B}_e \sum_{I \subset n} \alpha_I \bigwedge_{i \in I}  }{\alpha_n}
}
\newpage
\subsection{Graded Duality In Symmetric Algebras}
\Page{
	\DeclareFunc{symmetricComultiplication}
	{
		\prod R \in \ANN \.
		\prod A \in \LMOD{R} \.
		A^\vee \Arrow{\LALGE{R}(\Int)} A^\vee \widetilde{\otimes} A^\vee
	}
	\DefineNamedFunc{symmetricComultiplication}
	{ }{\Delta}{\bd A^\vee \Lambda a \in A \. a \otimes 1 + 1 \otimes a }
	\Assume{a,b}{A}
	\Say{[1]}{\bd \Delta \bd \FUNC{TensorProduct}}{
		\NewLine :
		\Delta\Big(a \vee b\Big)
		(a \otimes 1 + 1 \otimes a)(b \otimes 1 + 1 \otimes b) = 
		(a \vee b) \otimes 1 +  a \otimes b  + b \otimes a + 1 \otimes (a \vee b)  
	}
	\Say{[2]}{\bd \Delta \bd \FUNC{twistedtensorProduct} \bd A^\wedge}{
		\NewLine :
		\Delta\Big( b \vee a \Big) 
		(b \otimes 1 + 1 \otimes b)(a \otimes 1 + 1 \otimes a) = 
		(b \vee a) \otimes 1 +  b \otimes a  + a \otimes b + 1 \otimes (b \vee a) = \NewLine =  
	       (a \vee b) \otimes 1  +  a \otimes b  + b \otimes a + 1 \otimes (a \vee b) = 
	}
	\Conclude{[a,b.*]}{\bd \Delta[1][2]}{ \Delta(a \vee b - b \vee a) = \Delta(a \vee b) - \Delta(b \vee a) = 0} 
	\DeriveConclude{[*]}{\bd \Delta \bd E^\wedge}{\LOGIC{WellDefinied}(\Delta)}  
	\EndProof
	\\
	\DeclareFunc{SymmetricCounit}
	{
		\prod R \in \ANN \.
		\prod A \in \LMOD{R} \.
		A^\vee \Arrow{\LALGE{R}(\Int)} R
	}
	\DefineNamedFunc{exteriorCounit}{}{\eta}{\bd \LALGE{R}(A^\vee,R)(0)}
	\\
	\DeclareFunc{exteriorAntipode}
	{
		\prod R \in \ANN \.
		\prod A \in \LMOD{R} \.
		A^\vee \Arrow{\LALGE{R}(\Int)} A^\vee
	}
	\DefineNamedFunc{symmetricAntipode}{}{\sigma}{\bd \LALGE{R}(A^\vee,R)(-{\id}_A)}
	\\
	\\
	\Theorem{SymmetricAlgebraIsACocommutativeCoalgebra}
	{
		\forall R \in \ANN \.
		\forall A \in \LMOD{R}
		(A^\vee,\Delta,\eta) \in \CCOALG{R}(\Int) 
	}
	\NoProof
	\\
	\Theorem{SemmetricAlgebraIsAHopfAlgebra}
	{
		\forall R \in \ANN \.
		\forall A \in \LMOD{R} \.
		A^\vee \in \HOPF{R}(\Int)
	}
	\NoProof
	\\
	\Theorem{SymmetricAlgebraMapIsAHopfMorphism}
	{
		\forall R \in \ANN \.
		\forall A,B \in \LMOD{R} \.
		\forall f : A \Arrow{\LMOD{R}} B \. 
		\NewLine \. 
		f^\vee : A^\vee \Arrow{\HOPF{R}(\Int)} B^\vee
	}
	\NoProof
}\Page{
	\Theorem{DifferentialOperatorsAsGradedDuals}      
	{
		\forall R \in \ANN \.
		\forall A  : \FM \And \FGM(R) \. \NewLine
		\nabla  A \cong_{\LALGE{R}(\Int)} \mathfrak{D}(A)   
	}
	\NoProof
	\\
	\DeclareFunc{symmetricDualEmbedding}
	{
		\prod R \in \ANN \.
		\prod A \in \LMOD{R} \. 
		A^{*\vee} 
		\Arrow{\HOPF{R}(\Int)} \mathfrak{D}(A^\vee)
	}
	\DefineNamedFunc{symmeticDualEmbedding}{f}{\lambda(f)}{\bd \HOPF{R}(\Int)\bd \mathfrak{D}(A)(f)}
	\\
	\Theorem{SymmetricDualAction}
	{
		\forall R \in \ANN \.
		\forall A \in \LMOD{R} \.
		\forall n \in \Int_+ \.
		\forall a : n \to A \.
		\forall f : n \to A^* \. \NewLine \.  
		\left( \bigvee^n_{i=1} f_i \right)\left( \bigvee^n_{i=1} a_i \right) =
		\perm (f_i(a_j))^{n}_{i,j=1}
	}
	\NoProof
	\\
	\Theorem{SymmetricDualBasisAction1}
	{
		\forall R \in \ANN \.
		\forall A \in \LMOD{R} \.
		\forall n \in \Nat \.
		\forall x : \TYPE{Bais}(n,A) \.
		\NewLine \. 
		\forall p : n \to \Int_+ \.
		\left( \bigvee^n_{i=1} \Big(\mathrm{d}x_i\Big)^{p_i} \right)\left( \bigvee^n_{i=1} x_i^{p_i} \right) =
		\prod^n_{i=1} p_i!
	}
	\NoProof
	\\
	\Theorem{SymmetricDualBasisAction2}
	{
		\forall R \in \ANN \.
		\forall A \in \LMOD{R} \.
		\forall n \in \Nat \.
		\forall x : \TYPE{Bais}(n,A) \.
		\NewLine \. 
		\forall p,q : n \to \Int_+ \.
		\forall [0] : p \neq q \.
		\left( \bigvee^n_{i=1} \Big(\mathrm{d}x_i\Big)^{p_i} \right)\left( \bigvee^n_{i=1} x_i^{q_i} \right) = 0
	}
	\NoProof
}
\newpage
\subsection{Pl{\"u}cker's Equations}
\Page{
	\Theorem{DecomposableByHighDegree}
	{
		\forall k : \TYPE{Numeric} \.
		\forall V \in \FDVS{k} \.
		\forall t \in V^{\wedge(n-1)} \.
		t : \TYPE{Decomposable}(V)
		\NewLine
		\quad n = \dim V
	}
	\Say{\pluto}
	{
		\Lambda m \in \Nat \. 
		\forall V \in \FDVS{k} \.
		\dim V = m \Imply
		\forall t \in V^{\wedge(m-1)} \.
		t : \TYPE{Decomposable}(V)
	}
	{
		\Nat \to \Type
	}
	\Say{[0]}{\ByConstr \pluto \bd \FUNC{exteriorPower}(0)}
	{\pluto(0)}
	\Assume{n-1}{\Nat}
	\Assume{[1]}{\pluto(n-1)}
	\Assume{V}{\FDVS{k}}
	\Assume{[01]}{\dim V = n}
	\Assume{t}{t \in V^{\wedge(n-1)}}
	\Say{e}{\THM{FreeHasBasis}(V)}{\TYPE{Basis}(V,n)}
	\Say{\Big(\alpha,[2]\Big)}
	{
		\THM{ExteriorAlgebraBasis}(V,e)\bd \TYPE{Basis}                 
	}
	{
		\sum \alpha : k^n \.
		t = \sum_{i=1}^n \alpha_i\bigwedge_{j\neq i} e_i
	}
	\Say{U}{\Span \{ e_i\}_{i=2}^n}{\TYPE{VectorSubspace}(V)}
	\Assume{[00]}{\exists i \in \overline{2n} \. \alpha_i \neq 0}
	\Say{\Big( u,[3]\Big)}{[1](U)\bd \TYPE{Decomposable}[00]}
	{
		\sum u : \LI(U,n-2) \. 
		\sum^n_{i=2} \alpha_i\bigwedge^n_{j\neq i,1} e_j = 
		\bigwedge^{n-1}_{i=0} u_i
	}
	\Say{\Big(w,[4]\Big)}{\THM{BasisExtension}(U,u)}
	{
		\sum w \in U \. w \oplus u : \Basis(U,n-1)
	}
	\Say{\Big(\beta,[5] \Big)}{\bd U^{\wedge n-1}
		\THM{ExteriorAlgebraBasis}(U,w \oplus u)}
	{
		\alpha_1 \bigwedge^n_{i=2} e_i = 
		\beta w \wedge \bigwedge^{n-1}_{i=1} u_i 
	}
	\Say{[6]}{[2]\bd V^\wedge [3][5]\bd V^\wedge}
	{
		\NewLine :
		t = \alpha_n \sum^n_{i=1} \bigwedge_{j\neq i} e_i  = 
		e_1 \wedge \left( \sum^n_{i=2} \alpha_i \bigwedge^n_{j\neq 1,i}e_j
		\right)  + \alpha_1 \bigwedge^n_{i\neq 1} e_i =
		e_1 \wedge \bigwedge^{n-1}_{i=1} u_i 
		+
		\beta w \wedge \bigwedge^{n-1}_{i=1} u_i =
		(e_1 + \beta w) \wedge \bigwedge^{n-1}_{i=1} u_i  
	}
	\Conclude{[00.*]}{\bd^{-1}\TYPE{Decomposable}[6]}
	{
		\Big(
			t : \TYPE{Decomposable}(V)
		\Big)
	}
	\Derive{[00]}{I(\Imply)}{
		\Big(\exists i \in \overline{2n} \. 
		\alpha_i \neq 0 \Big)
		\Imply t  : \TYPE{Decomposable}(V)}
	\Say{[10]}{[2]\bd^{-1}\TYPE{Decomposable}I(\Imply)}
	{
		\Big(\forall i \in \overline{2n} \. \alpha_i = 0 \Big)
		\Imply
		t : \TYPE{Decomposable}(V)
	}
	\Conclude{n.*}{\LOGIC{LEM}[00][10]}
	{
		\Big(t : \TYPE{Decomposable}(V)\Big)
	}
	\DeriveConclude{[*]}{\bd \Nat \ByConstr \pluto }
	{
		\LOGIC{This}
	}
	\EndProof
	\\
	\DeclareFunc{tensorRank}
	{
		\prod k : \TYPE{Numeric} \.
		\prod V : \FDVS{k} \.
		V^{\wedge } \to \Int_+ 
	}
	\DefineNamedFunc{tensorRank}{t}{\rank t}{
		\min \{ \dim U : U \subvec{k} V \And t \in U^\wedge  \}}
	\\
	\DeclareFunc{tensorAnnihilator}
	{
		\prod k : \TYPE{Numeric} \.
		\prod V : \FDVS{k} \.
		V^{\wedge } \to \TYPE{VectorSubspace}(V^*)  
	}
	\DefineNamedFunc{tensorAnnihilator}{t}{\Ann t}
	{
		\{ f \in V^* \. \mathbf{i}(f)(t) = 0  \}
	}
}
\Page{
	\Theorem{DecomposableByRank}
	{
		\forall k : \TYPE{Numeric} \.
		\forall V : \FDVS{k} \.
		\forall n \in \dim V \.
		\forall t \in V^{\wedge n} \.
		\forall [0] : t \neq 0
		\NewLine \. 
		t : \TYPE{Decomposable}(V)
		\iff
		\rank t = n
	}
	\Assume{[1]}{\Big( t : \TYPE{Decomposable}(V) \Big)}
	\Say{\Big(v,[2]\Big)}
	{
		\bd \TYPE{Decomposable}[1]
	}
	{
		\sum v : n \to V \. t = \bigwedge^n_{i=1} v_i
	}
	\Say{[3]}{[2][0]}{\Big( v : \LI(V,n) \Big)}
	\Say{U}{\Span\{v_i\}^n_{i=1}}{\TYPE{VectorSubspace}(V)}
	\Say{[4]}{[2]\ByConstr}{t \in U^\wedge}
	\Say{[5]}{[3]\ByConstr}{\dim U = n}
	\Conclude{[1.*]}{\THM{ExteriorPowerRank}[4][5][0]}
	{
		\rank t = n
	}
	\Derive{[1]}{I(\Imply)}{
		t : \TYPE{Decomposable}(V) \Imply \rank t = n 
	}
	\Assume{[2]}{  \rank t = n }
	\Say{\Big( U, [3] \Big)}
	{
		\bd \FUNC{tensorRank}(t)[2]
	}
	{
		\sum U \subvec{k} V \.  t \in U^\wedge \And \dim U = n
	}
	\Conclude{[2.*]}{
			\THM{ExteriorPowerRank}[3]
			\bd^{-1} \TYPE{Decomposable}
		}
		{
			\Big(t : \TYPE{Decomosable}\Big) 
		}
	\DeriveConclude{[*]}{I(\Imply)I(\iff)[1]}
	{
		\Big(t : \TYPE{Decomposable}(V)
		\iff
		\rank t = n\Big)	
	}
	\EndProof
	\\
	\Theorem{RankAnnTHM}
	{
		\forall k : \TYPE{numeric} \.
		\forall V : \FDVS{k}  \.
		\forall t \in V^\wedge \.
		\Ann(t) + \rank(t) =\dim V 
	}
	\Say{n}{\rank t}{\Nat}
	\Say{\Big(U, [1] \Big)}{\bd \FUNC{tensorRank}[t]}
	{
		\sum U \subvec{k} V \.  t \in U^\wedge \And \dim U = n
	}
	\Say{u}{\THM{FreeHasBasis}[1]}
	{
		\TYPE{Basis}(U)
	}
	\Say{\Big( w, [2] \Big)}{\THM{BasisExtension}(u)}
	{
		\sum w : \LI(V,n) \. u \oplus w : \Basis(V)
	}
	\Say{e}{u \oplus w}{\Basis(V)}
	\Say{d}{\deg t}{\Int_+}
	\Assume{\beta}{n \to k}
	\Assume{[4]}{\beta \neq 0}
	\Say{f}{\sum^n_{i=1} \beta_i u^*_i}
	{
		V^{*}
	}
	\Say{v}{\sum^n_{f=1} \beta_i u^1_i }{V}
	\Say{[5]}{\ByConstr(v)[4]}{ v \neq 0}
	\Say{\Big(u',[6] \Big)}{\THM{OrthogonalBasisExtension}(v)[5]}
	{
		\sum u' : (n-1) \to V \. 
		v \oplus u' : \TYPE{Basis}(V) \And \NewLine \And 
		\forall i \in (n-1) \. f(u'_i) = 0
	}
	\Say{[7]}{\ByConstr f \ByConstr v}{f(v) \neq 0}
}\Page{
	\Say{\Big(\alpha,\gamma,[3] \Big)}
	{
		\THM{ExteriorAlgebraBasis}(t)[1]
	}
	{
		\sum \alpha : \TYPE{Skew}(d-1,n,k) \.
		\sum \gamma : \TYPE{Skew}(d,n,k) \.
		\NewLine \.
		 t = 
		 \sum^{d-1}_{m=1}\sum_{I : m \uparrow {n-1}} 
		 \alpha_I v \wedge \bigwedge^m_{i=1} u_{I_i}'
		 +
		 \sum^{d-1}_{m=1}\sum_{I : m \uparrow {n-1}} 
		 \gamma_I \bigwedge^m_{i=1} u_{I_i}'
	}
	\Say{[8]}{ [3]\ByConstr \beta \bd \FUNC{interiorProduct}[6]  }{
		\NewLine :
		\mathbf{i}(f)(t) =
		 \sum^{d-1}_{m=1}\sum_{I : m \uparrow {n-1} } 
		 \alpha_I \mathbf{i}(f)v \wedge \bigwedge^m_{i=1} u_{I_i}' 
		 +
		 \sum^{d-1}_{m=1}\sum_{I : m \uparrow {n-1} } 
		 \gamma_I \mathbf{i}(f)\bigwedge^m_{i=1} u_{I_i}'
		 =
		 \sum^{d-1}_{m=1} \sum_{I : m \uparrow {n-1}} 
		 f(v) \alpha_I\bigwedge^m_{i=1} u'_{I_i}
	}
	\Say{U'}{\Span \{ u'_i \}^n_{i=1}}{ \TYPE{VectorSubspace}(U)} 
	\Assume{[9]}{\mathbf{i}(f)(t) = 0}
	\Say{[10]}{[9][8]}{\alpha=0}
	\Say{[11]}{[3][10]}{t \in U'^\wedge}
	\Say{[12]}{\ByConstr U'[11]}{ \rank t = n-1}
	\Conclude{[9.*]}{[12]\ByConstr(n)\THM{PrivIneq}}{\bot}
	\DeriveConclude{[\beta.*]}{E(\bot)}{\mathbf{i}(f)(t) \neq 0 }
	\Derive{[5]}{ I(\forall)\bd^{-1}U^* \bd^{-1} \Ann t }
	{U^* \bot \Ann t = \{0\}}
	\Say{W}{\Span\{ w_i \}_{i=1}^n}{ \TYPE{VectorSubspace}(V) }
	\Say{[6]}{\ByConstr(W)\bd^{-1}\Ann(t)}{\forall f \in W^* \. f \in \Ann t} 
	\Say{[7]}{\bd \dim [6]}{\dim \Ann t \ge \dim V - n}
	\Say{[8]}{[5]\THM{NonitersectDim}}{\dim \Ann(t) + \dim U^* \le \dim V}
	\Conclude{[*]}{\ByConstr U [8][7] \THM{DoubleIneq}}
	{
		\dim \Ann(t) + \rank(t) = \dim V
	}
	\EndProof
	\\
	\Theorem{SkewMatrixRank}
	{
		\forall k : \TYPE{Numeric} \.
		\forall n \in \Nat \.
		\forall \alpha : \TYPE{SkewMatrix}(k,n) \.
		\rank \alpha = 
		\rank \sum^n_{i=1} \sum^n_{j=i+1} \alpha_{i,j} e_i \wedge e_j
	}
	\Say{t}{\sum^n_{i=1} \sum^n_{j=i+1} \alpha_{i,j} e_i \wedge e_j}
	{
		k^{n\wedge}
	}
	\Assume{\beta}{\ker \alpha}
	\Say{f}{\sum^n_{i=1} \beta_i e^*_i}{k^{n*}}
	\Say{[1]}{
		\bd \ker \alpha(\beta)
		\bd \FUNC{matrixMultiplication}
		\bd \TYPE{SkewMatrix}(\alpha)
		\bd^{-1} \FUNC{exteriorMultiplication}
	}{
		\NewLine :
		0 = 
		\alpha \beta =
		\sum^n_{i=1} \left( \sum^n_{j=1} \beta_j \alpha_{i,j} \right)e_i=
		\sum^n_{i=1} \sum^n_{j=i+1} \alpha_{i,j} (\beta_i e_j - \beta_j e_i)=
		\mathbf{i}(f)(t)
	}
	\Conclude{[f.*]}{\bd^{-1} \Ann f}{f \in \Ann t}
	\Derive{[1]}{\THM{IsoDim}}{\dim \ker \alpha = \dim \Ann t}
	\Conclude{[*]}{
		\THM{RankKerTHM}(\alpha)
		\THM{RankAnnTHM}(t)[1]
	}
	{
		\rank t = \rank \alpha
	}
	\EndProof
	\\
	\Theorem{BivectorRank}
	{
		\forall k : \TYPE{Numeric} \.
		\forall V \in \VS{k} \. 
		\forall t \in V^{\wedge2} \.
		\rank t : \TYPE{Even}
	}
	\NoProof
}
\Page{
	\Theorem{ProductTensorRank}
	{
		\forall k : \TYPE{Numeric} \.
		\forall V \in \VS{k} \.
		\forall U,W \subvec{k} V \.
		\forall [0] : U \cap W = \{0\} \.
		\forall n,m \in \Int_+ \.
		\NewLine \.
		\forall t \in U^{\wedge n} \.
		\forall s \in W^{\wedge m} \.
		\forall [00] : t \neq 0 \neq s \.
		\rank t \wedge s = \rank t + \rank s
	}
	\NoProof
	\\
	\Theorem{SumTensorRank}
	{
		\forall k : \TYPE{Numeric} \.
		\forall V \in \VS{k} \.
		\forall U,W \subvec{k} V \.
		\forall [0] : U \cap W = \{0\} \.
		\forall n,m  : \Nat   \.
		\NewLine \.
		\forall t \in U^{\wedge n} \.
		\forall s \in W^{\wedge m} \.
		\forall [00] : (n,m) \neq (1,1) \. 
		\rank (t + s) = \rank t + \rank s
	}
	\NoProof
	\\
	\DeclareFunc{mapOfPl\ddot{u}cker}
	{
		\prod k : \TYPE{Numeric} \.
		\prod V \in \FDVS{k} \.
		\prod n \in \Nat \.
		V^{\wedge n} \Arrow{\VS{k}} V^* \Arrow{\VS{k}} 
		V^{\wedge(n-1)}
	}
	\DefineNamedFunc{mapOfPl\ddot{u}cker}{t,f}{p_t(f)}
	{ \mathbf{i}(f)(t) }
	\\
	\DeclareFunc{dualMapOfPl\ddot{u}cker}
	{
		\prod k : \TYPE{Numeric} \.
		\prod V \in \FDVS{k} \.
		\prod n \in \Nat \.
		V^{\wedge n} \Arrow{\VS{k}} 
		V^{*\wedge(n -1) } \Arrow{\VS{k}} V
	}
	\DefineNamedFunc{dualMapOfPl\ddot{u}cker}{t,s}{b_t(s)}
	{ \mathbf{i}(s)(t) }
	\\
	\Theorem{Pl\ddot{u}ckerDuality}
	{
		\forall k : \TYPE{Numeric} \.
		\forall V \in \FDVS{k} \.
		\forall n \in \Nat \.
		\forall t \in V^{\wedge n}  \. 
		\exists \sigma \in \{-1,+1\} \. 
		p^*_t = sb_t
	}
	\Assume{F}{V^{\wedge(n-1)*}}
	\Say{(m,f,[1])}{\THM{FiniteExteriorDuality}(F)}
	{
		\sum m \in \Nat \.
		\sum f : m \to (n-1) \to V^* \.
		F = 
		\mathbf{i}\left(\sum^m_{i=1} \bigwedge^{n-1}_{j=1} f_{i,j}
		\right)
	}
	\Say{s}{\sum^m_{i=1} \bigwedge^{n-1}_{j=1} f_{i,j}}
	{V^{*\wedge(n-1)}}
	\Assume{g}{V^*}
	\Conclude{[g.*]}{
		\bd \FUNC{dualMap}
		\bd p_t
		\bd \LALGE{k} \mathbf{i}
		\bd V^\wedge
	}
	{
		\NewLine
		p^*_t F(f) =
		F\Big( p_t g  \Big) =
		\mathbf{i}\left(\sum^m_{i=1} \bigwedge^{n-1}_{j=1} f_{i,j}
		\right)
		\mathbf{i}(g)(t) = 
		\mathbf{i}\left(\sum^m_{i=1} g \wedge \bigwedge^{n-1}_{j=1} 
		f_{i,j} 
		\right)
		(t) = \NewLine 
		\mathbf{i}\left((-1)^{n-1} 
			\sum^m_{i=1} \left(\bigwedge^{n-1}_{j=1} 
			f_{i,j} \right) \wedge g
		\right)
		(t) = 
		\mathbf{i}(g)\left((-1)^{n-1}
			\sum^m_{i=1} \bigwedge^{n-1}_{j=1} 
			f_{i,j} 
		\right)
		(t) = 
		(-1)^{n-1}\mathbf{i}(g)\mathbf{i}(s)(t) = \NewLine = 
		(-1)^{n-1} g\Big(b_t(s)\Big) =
		(-1)^{n-1} \epsilon(b_t(s))(g)
	}
	\DeriveConclude{[F.*]}{I(=,\to)\THM{NaturalIsomorphism}(V) }
	{
		p^*_t(F) = (-1)^{n-1} b_t(s)
	}
	\DeriveConclude{[*]}{I(=,\to)\THM{ExteriorDuality}(V)}
	{
		p^*_t = b_t
	}
	\EndProof
	\\
	\Theorem{Pl\ddot{u}ckerRankLemma}
	{
		\forall k : \TYPE{Numeric} \.
		\forall V \in \FDVS{k} \.
		\forall n \in \Nat \.
		\forall t \in V^{\wedge n} \.
		\rank p_t = \rank t
	}
	\Say{[1]}{\bd \Ann t \bd \ker p_t}
	{
		\Ann t = \ker p_t
	}
	\Conclude{[*]}
	{     
		\THM{RankAnnTHM}(t)
		\THM{RankkerTHM}(p_t)
	}
	{
		\rank p_t = \rank t
	}
	\EndProof
}
\Page{
	\Theorem{DualPl\ddot{u}ckerRankLemma}
	{
		\forall k : \TYPE{Numeric} \.
		\forall V \in \FDVS{k} \.
		\forall n \in \Nat \.
		\forall t \in V^{\wedge n} \.
		\rank b_t = \rank t
	}
	\Say{[1]}{\THM{Pl\ddot{u}ckerRankLemma}(k,V,n,t)}
	{
		\rank p_t = \rank t
	}
	\Conclude{[*]}
	{     
		\THM{DualRankTheorem}[1]
		\THM{Pl\ddot{u}ckerDuality}(k,V,n,t)
	}
	{
		\rank b_t = \rank t
	}
	\EndProof
	\\
	\Theorem{Pl\ddot{u}ckerEquations}
	{
		\forall k : \TYPE{Numeric} \.
		\forall V \in \FDVS{k} \.
		\forall n \in \Nat \.
		\forall t \in V^{\wedge n} \.
		\NewLine
		\Big(t : \TYPE{Decomposable}(V)\Big) 
		\iff
		\forall 
		\xi \in  V^{*\wedge (n-1)}  \.
		b_t(\xi) \wedge t = 0
	}
	\Assume{[1]}{ \Big( t : \TYPE{Decomposable}(V) \Big)  }
	\Say{\Big(v,[2]\Big)}
	{
		\bd \TYPE{Decomposable}[1]
	}
	{
		\forall v : n \to V \.
		t = \bigwedge^n_{i=1} v_i
	}
	\Conclude{[*]}{\bd b_t [2] \bd V^\wedge}
	{
		\forall \xi \in V^{*\wedge (n-1)} \.
		b_t(\xi) \wedge t = 0
	}
	\Derive{[1]}{I(\Imply)}
	{
		\Big(t : \TYPE{Decomposable}\Big)(V) 
		\Imply
		\forall \xi \in V^{*\wedge(n-1)} \.
		b_t(\xi) \wedge t = 0
	}
	\Assume{[2]}
	{
		\forall \xi \in V^{*\wedge(n-1)} \.
		b_t(\xi) \wedge t = 0
	}
	\Say{N}{\rank t}{\Int_+}
	\Say{\Big( U, [3] \Big)}
	{
		\bd \rank t
	}
	{
		\sum U \subvec{k} V \.
		t \in U^\wedge 
		\And
		\dim U = N
	}
	\Say{[4]}{\THM{DualPl\ddot{u}ckerRankLemma}(t)\ByConstr N}
	{
		\rank b_t = N
	}
	\Say{[5]}{\bd b_t [3] }
	{
		\im b_t \subset U
	}
	\Say{[6]}
	{
		[3][4][5]
	}
	{
		\im b_t = U
	}
	\Say{u}{\THM{VectorSpaceIsFree}(U)\THM{FreeHasBasis}(U)}
	{
		\Basis(U)
	}
	\Say{\Big(\alpha,[7]\Big)}{\THM{ExteriorBasis}(u)(t)[3]}
	{
		\sum \alpha : (n \uparrow N) \to k \.
		t = \sum_{I : (n \uparrow N)} 
		\alpha_I \bigwedge^n_{i=1}  u_{I_i}
	}
	\Assume{i}{N}
	\Say{\Big(f,[8]\Big)}{
		\bd \FUNC{image}[6](u_i) 
	}
	{
		\sum f \in V^{*\wedge(n-1)} \. b_t(f) = u_i
	}
	\Say{[9]}{[8][7][2](f)}
	{ 
		\sum_{I : (n \uparrow N)}  
		u_i \wedge \alpha_I\bigwedge^n_{i=1} u_{I_i}     =
		u_i \wedge t = 
		b_t(f) \wedge t = 
		0
	}
	\Conclude{[*.1]}{[9]\bd V^\wedge}{
		\forall I : n \uparrow N \. 
		i \not \in \im I \Imply
		\alpha_I = 0                           
	}
	\Derive{[8]}
	{
		I(\forall)
	}
	{
		\forall i \in N \.
		\forall I : n \uparrow N \.
		(\alpha_I \neq 0) 
		\Imply
		( i \in \im I  )
	}
	\Assume{[9]}{ t \neq 0}
	\Say{[10]}{[9][3]}{\alpha \neq 0}
	\Say{[11]}{[10][8]}{N = n}
	\Conclude{[9.*]}{\ByConstr N \THM{DecomposableByRank}}
	{
		\Big( t : \TYPE{Decompasble}\Big)(V)
	}
	\Derive{[9]}{I(\Imply)}{t \neq 0 \Imply t :\TYPE{Decomposable}(V)}
	\Say{[10]}{\bd^{-1} \TYPE{Decomposable}\bd V^\wedge}
	{
		t = \Imply t : \TYPE{Decomposable}(V)
	}
	\Conclude{[2.*]}{E(|)\LOGIC{LEM}(t=0)[9][10]}
	{
		\Big( t : \TYPE{Decomposable}(V)\Big)
	}
	\DeriveConclude{[*]}{I(\Imply)[1]I(\iff)}
	{
		\LOGIC{This}
	}
	\EndProof
}
\newpage
\section{Noncommutative Tensor Product}
\end{document}
