\documentclass[12pt]{scrartcl}
\usepackage{mathtools}
\usepackage[T2A]{fontenc}
\usepackage[utf8]{inputenc}
\usepackage{amsmath}
\usepackage{amsfonts}
\usepackage{hyperref}
\usepackage{amssymb}
\usepackage{ wasysym }
\usepackage{ upgreek }
\usepackage{scalerel}
\usepackage[dvipsnames]{xcolor}
\usepackage[a4paper,top=5mm, bottom=20mm, left=10mm, right=2mm]{geometry}
\renewcommand\pagemark{{\usekomafont{pagenumber}\thepage\ }}
%Markup
\newcommand{\TYPE}[1]{\textcolor{NavyBlue}{\mathtt{#1}}}
\newcommand{\FUNC}[1]{\textcolor{Cerulean}{\mathtt{#1}}}
\newcommand{\LOGIC}[1]{\textcolor{Blue}{\mathtt{#1}}}
\newcommand{\THM}[1]{\textcolor{Maroon}{\mathtt{#1}}}
%META
\renewcommand{\.}{\; . \;}
\newcommand{\de}{: \kern 0.1pc =}
\newcommand{\extract}{\LOGIC{Extract}}
\newcommand{\where}{\LOGIC{where}}
\newcommand{\If}{\LOGIC{if} \;}
\newcommand{\Then}{ \; \LOGIC{then} \;}
\newcommand{\Else}{\; \LOGIC{else} \;}
\newcommand{\IsNot}{\; ! \;}
\newcommand{\Is}{ \; : \;}
\newcommand{\DefAs}{\; :: \;}
\newcommand{\Act}[1]{\left( #1 \right)}
\newcommand{\Example}{\LOGIC{Example} \; }
\newcommand{\Theorem}[2]{& \THM{#1} \, :: \, #2 \\ & \Proof = \\ } 
\newcommand{\DeclareType}[2]{& \TYPE{#1} \, :: \, #2 \\} 
\newcommand{\DefineType}[3]{& #1 : \TYPE{#2} \iff #3 \\} 
\newcommand{\DefineNamedType}[4]{& #1 : \TYPE{#2} \iff #3 \iff #4 \\} 
\newcommand{\DeclareFunc}[2]{& \FUNC{#1} \, :: \, #2 \\}  
\newcommand{\DefineFunc}[3]{&  \FUNC{#1}\Act{#2} \de #3 \\} 
\newcommand{\DefineNamedFunc}[4]{&  \FUNC{#1}\Act{#2} = #3 \de #4 \\} 
\newcommand{\NewLine}{\\ & \kern 1pc}
\newcommand{\Page}[1]{ \begin{align*} #1 \end{align*}   }
\newcommand{ \bd }{ \ByDef }
\newcommand{\NoProof}{ & \ldots \\ \EndProof}
%LOGIC
\renewcommand{\And}{\; \& \;}
\newcommand{\ForEach}[3]{\forall #1 : #2 \. #3 }
\newcommand{\Exist}[2]{\exists #1 : #2}
\newcommand{\Imply}{\Rightarrow} 
\newcommand{\Intro}{\LOGIC{I}}
\newcommand{\Elim}{\LOGIC{E}}
%TYPE THEORY
\newcommand{\Type}{\TYPE{Type}}
%\DeclareMathOperator*{\dom}{dom}
%%STD
\newcommand{\Int}{\mathbb{Z} }
\newcommand{\NNInt}{\mathbb{Z}_{+} }
\newcommand{\Reals}{\mathbb{R} }
\newcommand{\Complex}{\mathbb{C}}
\newcommand{\Rats}{\mathbb{Q} }
\newcommand{\Sphere}{\mathbb{S}}
\newcommand{\Ball}{\mathbb{B}}
\newcommand{\Nat}{\mathbb{N} }
\newcommand{\EReals}{\stackrel{\mathclap{\infty}}{\mathbb{R}}}
\newcommand{\ERealsn}[1]{\stackrel{\mathclap{\infty}}{\mathbb{R}}^{#1}}
\DeclareMathOperator*{\centr}{center}
\DeclareMathOperator*{\argmin}{arg\,min}
\DeclareMathOperator*{\id}{id}
\DeclareMathOperator*{\im}{Im}
\DeclareMathOperator*{\supp}{supp}
\newcommand{\EqClass}[1]{\TYPE{EqClass}\left( #1 \right)}
\newcommand{\End}{\mathrm{End}}
\newcommand{\Aut}{\mathrm{Aut}}
\mathchardef\hyph="2D
\newcommand{\ToInj}{\hookrightarrow}
\newcommand{\ToMono}{\xhookrightarrow}
\newcommand{\ToSurj}{\twoheadrightarrow}
\newcommand{\ToEpi}{\xtwoheadrightarrow}
\newcommand{\ToBij}{\leftrightarrow}
\newcommand{\ToIso}{\xleftrightarrow}
\newcommand{\Arrow}{\xrightarrow}
\newcommand{\Set}{\TYPE{Set}}
\newcommand{\du}{\; \triangle \;}
\renewcommand{\c}{\complement}
\renewcommand{\i}{\mathbf{i}}
\newcommand{\Eqmod}[3]{#1 = #2 \quad \mathrm{mod} \quad #3}
%%ProofWritting
\newcommand{\Say}[3]{& #1 \de #2 : #3, \\}
\newcommand{\SayIn}[3]{& #1 \de #2 \in #3, \\}
\newcommand{\Conclude}[3]{& #1 \de #2 : #3; \\}
\newcommand{\Derive}[3]{& \leadsto #1 \de #2 : #3, \\}
\newcommand{\DeriveConclude}[3]{& \leadsto #1 \de #2 : #3 ; \\} 
\newcommand{\Assume}[2]{& \LOGIC{Assume} \; #1 : #2, \\}
\newcommand{\AssumeIn}[2]{& \LOGIC{Assume} \; #1 \in #2, \\}
\newcommand{\As}{\; \LOGIC{as } \;} 
\newcommand{\ByDef}{\LOGIC{E}}
\newcommand{\QED}{\; \square}
\newcommand{\EndProof}{& \QED \\}
\newcommand{\Proof}{\LOGIC{Proof} \; }
\newcommand{\Explain}[1]{& \text{#1.} \\}
\newcommand{\ExplainFurther}[1]{& \text{#1} \\}
\newcommand{\Exclaim}[1]{& \text{#1!} \\}
%SetTheory
\newcommand{\NonEmpty}{\TYPE{NonEmpty}}
\newcommand{\Finite}{\TYPE{Finite}}
\newcommand{\Countable}{\TYPE{Countable}}
\newcommand{\Uncountable}{\TYPE{Uncountable}}
\newcommand{\Ideal}{\TYPE{Ideal}}
\newcommand{\Inj}{\TYPE{Injective}}
\newcommand{\Surj}{\TYPE{Surjective}}
\newcommand{\Bij}{\TYPE{Bijective}}
\newcommand{\SIdeal}{\TYPE{\sigma\hyph \Ideal}}
%\newcommand{\SA}{\TYPE{\sigma \hyph Algebra}}
\newcommand{\Eq}{\TYPE{Equivalence}}
%CategoryTheory
%Types
\newcommand{\Cov}{\TYPE{Covariant}}
\newcommand{\Contra}{\TYPE{Contravariant}}
\newcommand{\NT}{\TYPE{NaturalTransform}}
\newcommand{\UMP}{\TYPE{UnversalMappingProperty}}
\newcommand{\CMP}{\TYPE{CouniversalMappingProperty}}
\newcommand{\paral}{\rightrightarrows}
%functions
\newcommand{\op}{\mathrm{op}}
\newcommand{\obj}{\mathrm{obj}}
\DeclareMathOperator*{\dom}{dom}
\DeclareMathOperator*{\codom}{codom}
\DeclareMathOperator*{\colim}{colim}
%variable
\renewcommand{\C}{\mathcal{C}}
\newcommand{\A}{\mathcal{A}}
\newcommand{\B}{\mathcal{B}}
\newcommand{\I}{\mathcal{I}}
\newcommand{\J}{\mathcal{J}}
\newcommand{\R}{\mathcal{R}}
%Cats
\newcommand{\CAT}{\mathsf{CAT}}
\newcommand{\SET}{\mathsf{SET}}
\newcommand{\PARALLEL}{\bullet \paral \bullet}
\newcommand{\WEDGE}{\bullet \to \bullet \leftarrow \bullet}
\newcommand{\VEE}{\bullet \leftarrow \bullet \to \bullet}
%OrderTheory
%Types
\newcommand{\Poset}{\TYPE{Poset}}
\newcommand{\Toset}{\TYPE{Toset}}
\newcommand{\Pres}{\TYPE{PreorderedSet}}
\newcommand{\WF}{\TYPE{WellFounded}}
\newcommand{\WO}{\TYPE{WellOrdered}}
\newcommand{\II}{\TYPE{InitialInterval}}
\newcommand{\UB}{\TYPE{UpperBound}}
\newcommand{\LUB}{\TYPE{LowerUpperBound}}
\newcommand{\LB}{\TYPE{LowerBound}}
\newcommand{\ULB}{\TYPE{UpperLoweBound}}
%Cats
\newcommand{\POSET}{\mathsf{POSET}}
\newcommand{\ORD}{\mathsf{ORD}}
%Symbols
\renewcommand{\P}{\mathsf{P}}
%\newcommand{\F}{\mathsf{F}}
%\newcommand{\U}{\mathsf{U}}
%Algebra
%Groups
%Types
\newcommand{\Group}{\TYPE{Group}}
\newcommand{\Abel}{\TYPE{Abelean}}
\newcommand{\Sgrp}{\subset_{\mathsf{GRP}}}
\newcommand{\Nrml}{\vartriangleleft}
\newcommand{\FG}{\TYPE{FiniteGroup}}
\newcommand{\Stab}{\mathrm{Stab}}
\newcommand{\FGA}{\TYPE{FinitelyGeneratedAbelean}}
\newcommand{\DN}{\TYPE{DirectedNormality}}
\newcommand{\ActsOn}{\curvearrowright}
%Func
\DeclareMathOperator{\tor}{tor}
\DeclareMathOperator{\ord}{ord}
\DeclareMathOperator{\bool}{bool}
\DeclareMathOperator{\rank}{rank}
%Cats
\newcommand{\GRP}{\mathsf{GRP}}
\newcommand{\ABEL}{\mathsf{ABEL}}
%LINEAR
%Linear Algebra
%Types
\newcommand{\Basis}{\TYPE{Basis}} % Basis of the linear space
\newcommand{\submod}[1]{\subset_{\LMOD{#1}}}% submodule as a subset
\newcommand{\subvec}[1]{\subset_{\VS{#1}}}% vector subspace as a subset
\newcommand{\FGM}{\TYPE{FinitelyGeneratedModule}}% Finitely generated module
\newcommand{\LI}{\TYPE{LinearlyIndependent}}
\newcommand{\LIS}{\TYPE{LinearlyIndependentSet}}
\newcommand{\FM}{\TYPE{FreeModule}}
\newcommand{\IBP}{\TYPE{InvariantBasisProperty}}
\newcommand{\UTM}{\TYPE{UpperTriangularMatrix}}
\newcommand{\LTM}{\TYPE{LowerTriangularMatrix}}
%\newcommand{\Diag}{\TYPE{DiagonalMatrix}}
\newcommand{\FP }{\TYPE{FinitelyPresented}}
\newcommand{\GL}{\mathbf{GL}}% General Linear Group
\newcommand{\SL}{\mathbf{SL}}% Special Linear group
\newcommand{\SO}{\mathbf{SO}}
\newcommand{\SU}{\mathbf{SU}}
\newcommand{\prsubvec}[1]{\subsetneq_{\VS{#1}}}	% poper vector subspace as a subset
\newcommand{\LC}{\TYPE{LinearComplement}} 
\newcommand{\IS}{\TYPE{InvariantSubspace}}
\newcommand{\RP}{\TYPE{ReducingPair}}
\newcommand{\RCF}{\TYPE{RationalCanonicalForm}}
\newcommand{\JCF}{\TYPE{JordanCanonicalForm}}
\newcommand{\Diagble}{\TYPE{Diagonalizable}}
\newcommand{\UT}{\TYPE{UpperTriangulizable}}
\newcommand{\LT}{\TYPE{LowerTriangulizable}}
\newcommand{\IPS}{\TYPE{InnerProductSpace}}
\newcommand{\OBasis}{\TYPE{OrthonormalBasis}}
\newcommand{\FDIPS}{\TYPE{FiniteDimensionalInnerProductSpace}}
\newcommand{\NO}{\TYPE{NormalOperator}}
\newcommand{\NM}{\TYPE{NormalMatrix}}
\newcommand{\SA}{\TYPE{SelfAdjoint}}
\newcommand{\SSA}{\TYPE{SkewSelfAdjoint}}
\newcommand{\PI}{\TYPE{Pseudoinverse}}
%\newcommand{\OVS}{\TYPE{OrthogonalVectorSpace}}
\newcommand{\SVS}{\TYPE{SymplecticVectorSpace}}
\newcommand{\MVS}{\TYPE{MetricVectorSpace}}
\newcommand{\FDMVS}{\TYPE{FiniteDimensionalMetricVectorSpace}}
\newcommand{\Sp}{\mathbf{Sp}}
%Func
\DeclareMathOperator{\Span}{span} % spann by subset
\DeclareMathOperator{\Ann}{Ann}   % annihilator
\DeclareMathOperator{\Ass}{Ass}   % associated primes:
\DeclareMathOperator{\adj}{adj}   % an adjoint matrix
\DeclareMathOperator{\tr}{tr}     % trace
\DeclareMathOperator{\codim}{codim} % codimension
%\DeclareMathOperator{\Cell}{\mathbf{C}} % a componion matrix
\DeclareMathOperator{\JC}{\mathbf{J}}  % a Jordan cell
\DeclareMathOperator{\bigboxplus}{\scalerel*{\boxplus}{\sum}} % a direct sum of operators in the sence of the reducing a pair
\DeclareMathOperator{\Spec}{Spec} % Spectre
\DeclareMathOperator{\bigbot}{\scalerel*{\bot}{\sum}} % an othogonal direct sum
\DeclareMathOperator{\GS}{\mathbf{GS}} %Gramm-Smmidt process
\DeclareMathOperator{\NGS}{\mathbf{NGS}} %Normalized Gramm-Smmidt process
\DeclareMathOperator{\WI}{\mathrm{WI}} %Witt Index
%Cats
\newcommand{\VS}[1]{#1\hyph\mathsf{VS}} % a category of vector spaces (Field)
\newcommand{\FDVS}[1]{#1\hyph\mathsf{FDVS}} % a category of finite-dimensional vector spaces (Field)
\newcommand{\LALGE}[1]{#1\hyph\mathsf{ALGE}}
\newcommand{\LMOD}[1]{#1\hyph\mathsf{MOD}} % a category of the left modules (Ring)
\newcommand{\RMOD}[1]{\mathsf{MOD}\hyph#1} % a category of the right modules (Ring)
\newcommand{\LLMAP}[1]{#1\hyph\mathsf{LMAP}} % a cagory of based linear maps with the left scalar multiplication (Ring)
\newcommand{\LMAT}[1]{#1\hyph\mathsf{MAT}}  % a category of based matrices with the left scalar multiplication (Ring)
\newcommand{\NMAT}[1]{#1\hyph\mathbb{N}} % a category of finite matrices (Field)
%Symbols
\renewcommand{\L}{\mathcal{L}}
\renewcommand{\O}{\mathbf{O}}
\renewcommand{\S}{\mathcal{S}}
%FIELDS
\newcommand{\Field}{\TYPE{Field}}
\newcommand{\ACF}{\TYPE{AlgebraicallyClosedField}}
%RINGS
%TYPE
\newcommand{\Ring}{\TYPE{Ring}}
\newcommand{\CR}{\TYPE{CommutativeRing}}
%\newcommand{\Ideal}{\TYPE{Ideal}}
\newcommand{\ID}{\TYPE{IntegralDomain}}
\newcommand{\UFD}{\TYPE{UniqueFactorizationDomain}}
\newcommand{\PID}{\TYPE{PrincipleIdealDomain}}
\newcommand{\FGI}{\TYPE{FinitelyGeneratedIdeal}}
\newcommand{\ER}{\TYPE{EuclideanRing}}
\newcommand{\DVR}{\TYPE{DiscreteValuationRing}}
\newcommand{\MoFT}{\TYPE{MonoidOfFiniteType}}
%CATS
\newcommand{\RING}{\mathsf{RING}} % A category of Rings
\newcommand{\ANN}{\mathsf{ANN}} % A category of Commutative Rings
%FUNCS
\DeclareMathOperator{\lcd}{lcd} % least common devided 
\DeclareMathOperator{\lc}{lc} % leading coefficient of the polynomial
\DeclareMathOperator{\cont}{cont} % content of the polynomial
\DeclareMathOperator{\antideg}{antideg} % antideree if the foramal power series
%Symbolsqq
%ALGEBRA
\newcommand{\LALG}[1]{#1\hyph\mathsf{ALG}}% Left associative unital algebras (Ring)
\newcommand{\RALG}[1]{\mathsf{ALG}\hyph#1}% Right associative unital  algebras (Rings)
%Numbers
%Integers
%FUNCS
\DeclareMathOperator{\divi}{div} % devide withou reminder
\DeclareMathOperator{\remi}{rem} % reminder
\DeclareMathOperator{\Frac}{Frac} % Field of fractions
%Complex
%Symb
\newcommand{\Herm}{\mathbf{H}}
\newcommand{\p}{\mathbf{p}}
\newcommand{\Inv}{\mathrm{Inv}}
\newcommand{\Stg}{\mathrm{Stg}}
\newcommand{\M}{\mathcal{M}}
%Geometry
%Affine
%Type
\newcommand{\AS}{\TYPE{AffineSpace}}
\newcommand{\ASS}{\TYPE{AffineSubspace}}
\newcommand{\AI}{\TYPE{AffineIndepend}}
\newcommand{\WL}{\TYPE{WithLines}}
\newcommand{\SAFF}{\mathbf{SAFF}}
\newcommand{\AFF}{\mathbf{AFF}}
\newcommand{\SLI}{\mathcal{SL}}
\newcommand{\SGL}{\mathbf{SGL}}
\newcommand{\GVS}{\TYPE{GeometricVectorSpace}}
\newcommand{\TP}{\TYPE{TrigonometricPlane}}
\newcommand{\OrVS}{\TYPE{OrientatedVectorSpace}}
\newcommand{\OTP}{\TYPE{OrientatedTrigonometricPlane}}
\newcommand{\MAS}{\TYPE{MetricAffineSpace}}
%Func
\newcommand{\Gr}{\mathrm{Gr}}
\newcommand{\Di}{\mathrm{Di}}
\newcommand{\Sc}{\mathrm{Sc}}
\newcommand{\Tr}{\mathrm{Tr}}
\DeclareMathOperator{\Aff}{Aff}
\DeclareMathOperator{\rat}{rat}
%Symbol
\newcommand{\tri}{\triangle}
%Projective
%TYPE
\newcommand{\subproj}[1]{\subset_{\PROJ{#1}}}
%FUNC
\newcommand{\vs}{\mathsf{VS}}
\newcommand{\PGL}{\mathbf{PGL}}
%\CAT
\newcommand{\PROJ}[1]{#1\hyph\mathsf{PROJ}}
%Symbol
\renewcommand{\P}{\mathbb{P}} 
%TOPVS
\newcommand{\TOPVS}[1]{#1\hyph\mathsf{TOPVS}} % a category of topological vector spaces (Field)
%Convex
%
\newcommand{\Convex}{\TYPE{Convex}}
\newcommand{\CB}{\TYPE{ConvexBody}}
\newcommand{\CC}{\TYPE{ConvexCone}}
\newcommand{\Mink}{\TYPE{MinkowskySpace}}
\newcommand{\Euc}{\TYPE{EucledeanSpace}}
\newcommand{\Cone}{\TYPE{Cone}}
\newcommand{\Wedge}{\TYPE{Wedge}}
\newcommand{\PVS}{\TYPE{PreorderedVectorSpace}}
\newcommand{\OVS}{\TYPE{OrderedVectorSpace}}
\newcommand{\AVS}{\TYPE{ArchemedeanVectorSpace}}
\newcommand{\RS}{\TYPE{RieszSpace}}
%FUNC
\newcommand{\rint}{{\mathrm{rel}  \intx}}
\DeclareMathOperator{\lina}{lina}
\DeclareMathOperator{\lin}{lin}
\DeclareMathOperator{\core}{core}
\DeclareMathOperator{\conv}{conv}
\DeclareMathOperator{\cone}{cone}
\DeclareMathOperator{\cconv}{\overline{conv}}
\DeclareMathOperator{\CCONV}{\mathsf{CCONV}}
\renewcommand{\H}{\mathrm{H}}
%Topology
%General Topology
%Types
\newcommand{\TS}{\TYPE{TopologicalSpace}} 
\newcommand{\LF}{\TYPE{LocallyFinite}}
%\newcommand{\LC}{\TYPE{LocallyCompact}}
\newcommand{\PN}{\TYPE{PerfectlyNormal}}
\newcommand{\Open}{\TYPE{Open}}
\newcommand{\Compact}{\TYPE{Compact}}
\newcommand{\Compacts}{\TYPE{CompactSubset}}
\newcommand{\SCompact}{\TYPE{\sigma\hyph Compact}}
\newcommand{\LCompact}{\TYPE{LocallyCompact}}
\newcommand{\Perfect}{\TYPE{Perfect}}
\newcommand{\Limit}{\TYPE{Limit}}
\newcommand{\Clopen}{\TYPE{Clopen}}
\newcommand{\Closed}{\TYPE{Closed}}
\newcommand{\Separable}{\TYPE{Separable}}
\newcommand{\Dense}{\TYPE{Dense}}
\newcommand{\ND}{\TYPE{NowhereDense}}
\newcommand{\Meager}{\TYPE{Meager}}
\newcommand{\Comeager}{\TYPE{Comeager}}
\newcommand{\Bair}{\TYPE{Baire}}
%FUNC
\DeclareMathOperator*{\intx}{int}
\DeclareMathOperator*{\cl}{cl} 
\DeclareMathOperator*{\boundary}{\partial} 
\DeclareMathOperator{\combo}{\triangledown} 
\DeclareMathOperator{\diag}{\triangle} 
\DeclareMathOperator{\rem}{rem}
%CATS
\newcommand{\TOP}{\mathsf{TOP}}
\newcommand{\HC}{\mathsf{HC}}
\newcommand{\CG}{\mathsf{CG}}
%Symbols
\newcommand{\T}{\mathcal{T}}
\renewcommand{\U}{\mathcal{U}}
\newcommand{\V}{\mathcal{V}}
\renewcommand{\O}{\mathcal{O}}
\renewcommand{\d}{\mathrm{d}}
\newcommand{\F}{\mathcal{F}}
\newcommand{\X}{\mathcal{X}}
%\newcommand{\d}{\mathrm{d}}
%Metric Topology
%TYPE
\newcommand{\Lip}{\mathrm{Lip}}
\newcommand{\Complete}{\TYPE{Complete}}
\newcommand{\Metrizable}{\TYPE{Metrizable}}
%FUNC
\DeclareMathOperator{\diam}{diam}
\DeclareMathOperator{\osc}{osc}
\newcommand{\Cell}{\mathbb{B}}
%CATS
\newcommand{\Semiiso}{\mathsf{SMS}_{\circ \to \cdot}}
\newcommand{\Iso}{\mathsf{MS}_{\circ \to \cdot}}
\newcommand{\SMS}{\mathsf{SMS}}
\newcommand{\MS}{\mathsf{MS}}
\newcommand{\UNI}{\mathsf{UNI}}
%Analysis
%Convex Analysis
%TYPE
\newcommand{\CF}{\TYPE{ConvexFunction}}
\newcommand{\Concave}{\TYPE{Concave}}
\newcommand{\PCF}{\TYPE{ProperConvexFunction}}
\newcommand{\PHomog}{\TYPE{PositivelyHomogeneous}}
\newcommand{\Conic}{\TYPE{Conic}}
\newcommand{\SCF}{\TYPE{StrictConvexFunction}}
\newcommand{\ClF}{\TYPE{ClosedFunction}}
\newcommand{\PClF}{\TYPE{ProperClosedFunction}}
%FUNC
\DeclareMathOperator{\epi}{epi}
\DeclareMathOperator*{\bigsquare}{\scalerel*{\square}{\textstyle\sum}}
%Linear
\newcommand{\BAN}[1]{#1\hyph\mathsf{BAN}}
\newcommand{\HIL}[1]{#1\hyph\mathsf{HIL}}
%Differential Analysis
\newcommand{\D}{\mathbf{D}}
%Real Analysis
\newcommand{\LsC}{\TYPE{LowerSemicontinuous}}
\newcommand{\UsC}{\TYPE{LowerSemicontinuous}}
\author{Uncultured Tramp}
\title{Convex Analysis}
\begin{document}
\maketitle
\thispagestyle{empty}
\newpage
\tableofcontents
\thispagestyle{empty}
\newpage
\pagenumbering{arabic}
\section{Convex  Functions}
\subsection{Subject}
\Page{
	\DeclareFunc{epigraph}{\prod V : \VS{\Reals} \. \prod D \subset V \. \Big(D \to \EReals\Big) \to ?(V \oplus \Reals)}
	\DefineNamedFunc{epigraph}{f}{\epi f}{ \{ (x,\phi) | x \in D, \phi \in \Reals,  \phi \ge f(x)  \}  }
	\\
	\DeclareType{\Convex}{\prod V : \VS{\Reals} \.  \prod D \subset V \. ?\Big(D \to \EReals\Big)}
	\DefineType{f}{\Convex}{\Convex(V\oplus\Reals,\epi f) }
	\\
	\DeclareFunc{effectiveDomain}{\prod V : \VS{\Reals} \. \prod D \subset V \. \Convex(V,D) \to ?D}
	\DefineNamedFunc{effectiveDomain}{f}{\dom f}{ \pi_1 \epi f  }
	\\
	\Theorem{DomainIsConvex}{
		\forall V \in \VS{\Reals} \. 
		\forall D \subset V \. 
		\forall f : \Convex(V,D) \.
		\Convex(V,\dom f)
	}
	\Explain{ As a linear image of convex set}
	\EndProof
	\\
	\DeclareType{ProperConvexFunction}
	{
		\prod V : \VS{\Reals} \. ?\Convex(V,V) \.
	}
	\DefineType{f}{ProperConvexFunction}{\forall x \in V \. f(x) > -\infty \And \exists x \in V \. f(x) < + \infty}
	\\
	\Theorem{InterpolationProperty}
	{
		\NewLine ::		
		\forall V : \VS{\Reals} \.
		\forall C : \Convex(V) \.
		\forall f : C \to (-\infty,+\infty] \. \NewLine \.
		\Convex(V,C,f)
		\iff
		\forall x,y \in C \.
		\forall \lambda \in [0,1] \. \NewLine \.
		f\Big( \lambda x + (1- \lambda) y \Big) \le \lambda f(x) + (1-\lambda) f(y)
	}
	\Explain{
		$(\Rightarrow):$ assume that $f$ is convex}
	\Explain{
		Then $f$ has convex epigraph}
	\Explain{
		Take arbitrary $x,y \in C$ and $\lambda \in [0,1]$}
	\Explain{
		If $f$ takes value $+\infty$ either in $x$ or $y$, then the inequality follows,
		so assume the contrary}
	\ExplainFurther{
		Then $\Big(x,f(x)\Big),\Big(y,f(y)\Big)$ trivially belong to the epigraph,} 
	 \Explain{
	 	so by convexity $\Big( \lambda x + (1-\lambda)y, \lambda f(x) + (1-\lambda) f(y)\Big)$ 
	 	is also in epigraph}
	 \Explain{
	 	The definition of epigraph produces the inequality
	 	$f\Big( \lambda x + (1 - \lambda) y\Big) \le  \lambda f(x) + (1-\lambda) f(y)$
	 }
	 \Explain{
		$(\Leftarrow):$ now assume that inequality always hold}
	\Explain{
		Assume $(x,\phi),(y,\psi)$ belong to the epigraph and $\lambda \in [0,1]$}
	\Explain{
		Then $\lambda \phi + (1-\lambda)\psi \ge \lambda  f(x) + (1-\lambda)f(y) \ge 
		f\Big(\lambda x +(1-\lambda)y\Big)$}
	\Explain{
		So $\lambda (x,\phi) + (1-\lambda) (y,\psi )$ belong to the epigraph}
	\Explain{
		Thus, epigraph is convex and $f$ is convex}
}
\Page{
	\Theorem{JensensIneq}
	{
		\NewLine ::		
		\forall V : \VS{\Reals} \.
		\forall C : \Convex(V) \.
		\forall f : C \to (-\infty,+\infty] \. \NewLine \.
		\forall n \in \Nat \.
		\forall \lambda \in \Reals^n_+ \.
		\forall \aleph : \sum^n_{k=1} \lambda_k = 1 \.
		\forall v \in V^n \.
		f\left( \sum^n_{k=1} \lambda_k v_k \right) \le \sum^n_{k=1} \lambda_k f(v_k)
	}
	\Explain{ Iterate the interpolation property}
	\EndProof
	\\
	\Theorem{SecondDerivativeConvexityTest}
	{
		\forall  I : \TYPE{OpenInterval}(\Reals) \.
		\forall f \in C^2(I) \. \NewLine \.
		\Convex(\Reals,I,f)
		\iff
		f'' \ge 0
	}
	\Explain{
		$(\Rightarrow):$ assume there is a $t \in I$ such that $f''(t) < 0$}
	\Explain{
		As $f''$ must be continous there is whole open interval $(a,b)$ 
		such that $f''(j) < 0$ for all $j \in (a,b)$}
	\Explain{
		Take some $x,y \in (a,b)$ with $x < y$ 
		and define$z = \lambda x + (1-\lambda) y$
		for siome $\lambda \in (0,1)$}
	\Explain{
		Then 
		$f(z) - f(x) = \int^z_x f'(t) \; dt  > f'(z)(z - x) $
		and
		$f(y) - f(z) = \int^y_z f'(t) \; dt < f'(z)(y- z)$
	}
	\Explain{
		Then from definiton of $z$ we get
		$f(z) > f(x)   - (1-\lambda) f'(z)(y - x)$
		and
		$f(z) > f(y)  + \lambda f'(z)(y-x) $
	}
	\Explain{
		By adding two inequalities with multipliers $\lambda$ and $(1-\lambda)$ one gets
		$f\Big( \lambda x + (1-\lambda)y\Big) > \lambda f(x) + (1-\lambda) f(y)$}
	\Explain{
		But this contradicts a convexity.
	}
	\Explain{$(\Rightarrow):$ use same inequalities but with different sign to prove the convexity}
	\EndProof
	\\
	\Theorem{ExponentIsConvexity}
	{
		\forall \alpha \in \Reals \. \Convex\Big(\Reals,\Reals,\Lambda t \in \Reals \. e^{\alpha t} \Big) 
	}
	\Explain{
		write $f(t) = e^{\alpha t}$}
	\Explain{
		Then $f''(t) = \alpha^2 e^{\alpha t} \ge 0$}
	\EndProof
	\\
	\Theorem{MonomialConvexity1}
	{
		\forall p \in [1,+\infty) \. \Convex\Big(\Reals,\Reals_{++},\Lambda t \in \Reals \. t^p \Big) 
	}
	\Explain{
		Write $f(t) = t^{p}$}
	\Explain{
		Then $f''(t) =  p(p-1)t^{p-2} \ge 0$ for $t > 0$}
	\EndProof
	\\
	\Theorem{MonomialConvexity2}
	{
		\forall p \in [0,1) \. \Convex\Big(\Reals,\Reals_{++},\Lambda t \in \Reals \. -t^p \Big) 
	}
	\Explain{
		Write $f(t) = t^{p}$}
	\Explain{
		Then $f''(t) =  p(1-p)t^{p-2} \ge 0$ for $t > 0$}
	\EndProof
}\Page{
	\Theorem{MonomialConvexity3}
	{
		\forall p \in (-\infty,0] \. \Convex\Big(\Reals,\Reals_{++},\Lambda t \in \Reals \. t^p \Big) 
	}
	\Explain{
		write $f(t) = t^{p}$}
	\Explain{
		Then $f''(t) =  p(p-1)t^{p-2} \ge 0$ for $t > 0$}
	\EndProof
	\\
	\Theorem{GeneralizedArcsinDerivativeIsConvex}
	{
		\forall \alpha \in \Reals_{++} \.
		\Convex\left(\Reals, (-\alpha,\alpha), \Lambda t \in \Reals \. \frac{1}{\sqrt{\alpha^2 - t^2}}\right)
	}
	\Explain{
		Write $f(t) = \frac{1}{\sqrt{\alpha^2 - t^2}}$}
	\Explain{
		Then $f'(t) = \frac{t}{\sqrt{\alpha^2 - t^2}^3}$}
	\Explain{
		And $f''(t) =  \frac{1}{\sqrt{\alpha^2 - t^2}^3} + \frac{3t^2}{\sqrt{\alpha^2 - t^2}^5} > 0$
		for $t \in (-\alpha,\alpha)$}
	\EndProof
	\\
	\Theorem{NegativeLogIsConvex}
	{
		\Convex\left(\Reals, \Reals_{++}, \Lambda t \in \Reals \. - \ln(t)  \right)
	}
	\Explain{
		Write $f(t) = - \ln(t)$}
	\Explain{
		Then $f''(t) = \frac{1}{t^2} > 0$ for $t > 0$}
	\EndProof
	\\
	\Theorem{NegativeEntropyIsConvex}
	{
		\Convex\left(\Reals, \Reals_{++}, \Lambda t \in \Reals \. t\ln(t)  \right)
	}
	\Explain{
		Write $f(t) = t\ln(t)$}
	\Explain{
		Then $f'(t) =  \ln(t) + 1$
	}
	\Explain{
		And $f''(t) = \frac{1}{t} > 0$ for $t > 0$}
	\EndProof
	\\
	\DeclareType{\Concave}{\prod V : \VS{\Reals} \.  \prod D \subset V \. ?\Big(D \to \EReals\Big)}
	\DefineType{f}{\Concave}{\Convex(V, D, -f) }
	\\
	\Theorem{SecondDerivativeConvexityTest2}
	{
		\forall V : \Euc \.
		\forall  U : \Open \And \Convex(V) \.
		\forall f \in C^2(U) \. \NewLine \.
		\Convex(\Reals,U,f)
		\iff
		\D^2 f \ge 0
	}
	\Explain{For $x \in U$ and $v \in V \setminus \{0\}$ define $\phi_{x,v}(t) = f(x + tv)$
		with a domain $I_{x,v} = \{ t \in \Reals | x + tv \in C \}$}
	\Explain{
		Then $f$ is convex iff every $\phi_{x,v}$ does}
	\Explain{
		But $\phi''_{x,v}(t) = \langle v, \D^2 f|_y v \rangle$, 
		where $y =  x + tv$}
	\Explain{
		So $f$ is convex iff $\D^2 f$ is positive-semidefinite}
	\EndProof
}
\Page{
	\Theorem{GeometricMeanIsConcave}
	{
		\NewLine ::		
		\forall V : \Euc \.
		\Concave\left( V, V_{++}, \Lambda x \in V \.  \prod^n_{k=1} \sqrt[n]{x_k}   \right)
		\quad \where \quad n = \dim V
	}
	\Explain{ write $f(x) = \prod^n_{k=1} \sqrt[n]{x_k}$}
	\Explain{
		Then $\nabla f|_x  = \left( \frac{1}{n\sqrt[n]{x_i}^{n-1}} \prod^n_{j\neq i} \sqrt[n]{x_j} \right)^n_{i=1} $}
	\Explain{
		And $\D^2_{i,j} f|_x = \frac{1}{n^2\sqrt[n]{x_i x_j}^{n-1}} \prod^n_{k \neq i,j} \sqrt[n]{x_k}$ 
		when $i \neq j$, and
		$\D^2_{i,i} f|_x =   -\frac{n-1}{n^2\sqrt[n]{x_i}^{2n-1}} \prod^n_{j\neq i} \sqrt[n]{x_j}$ }
	\ExplainFurther{
		So,
		$\D^2 f|_x(v,v) = -\frac{n-1}{n^2}\sum^n_{i=1}  
			\frac{v_i^2}{\sqrt[n]{x_i}^{2n-1}} \prod^n_{j\neq i} \sqrt[n]{x_j}  
			+   \frac{1}{n^2}\sum^n_{i\neq j} \frac{v_i v_j}{\sqrt[n]{x_i x_j}^{n-1}} \prod^n_{k \neq i,j} \sqrt[n]{x_k}	
			=$}
	\Explain{
		$
				=f(x)\left(  -\frac{n-1}{n^2} \sum^n_{i=1} \frac{v_i^2}{x_i^2} +        
				\frac{1}{n^2} \sum^n_{i\neq j} \frac{v_i v_j}{x_i x_j} \right) =
				-\frac{f(x)}{n^2}\left( n \sum^n_{i=1} \frac{v_i^2}{x_i^2} - 
				\left(\sum^n_{i=1} \frac{v_i}{x_i}\right)^2     \right) \le 0	
		$
		}
	\Explain{ This follows from obvious matching schema}
	\EndProof
	\\
	\Theorem{NormsAreConvex}
	{
		\forall V : \VS{\Reals} \.
		\forall \eta : \TYPE{Norm}(V)
		\Convex(V,V,\eta)
	}
	\Explain{
		Write $\eta(v) = \|v\|$}
	\Explain{ 
		Just use triangle inequality
		$\Big\| \lambda x + (1 - \lambda) y \Big\|  \le 
		\| \lambda x \| + \Big\| (1 - \lambda) y \Big\| = 
		\lambda \|x\| + (1 - \lambda) \|y\|$	}
	\EndProof
	\\
	\DeclareFunc{convexIndicator}
	{
		\forall V : \VS{\Reals} \.
		\Convex(V) \to \Convex(V,V)
	}
	\DefineNamedFunc{convexIndicator}{C}{\Lambda x \in V \. \chi(x|C)}{
			\Lambda x \in V \.  \infty\big[x \in C^\c\big]
	}
	\\
	\DeclareFunc{supportFunction}
	{
		\forall V : \Reals\hyph\mathsf{HIL} \.
		\Convex(V) \to \Convex(V,V)
	}
	\DefineNamedFunc{supportFunction}{C}{\Lambda x \in V \. \chi^*(x|C)}{
			\sup_{y \in C} \langle x,y \rangle
	}
	\\
	\DeclareFunc{gauge}
	{
		\forall V : \VS{\Reals} \.
		\Convex(V) \to \Convex(V,V)
	}
	\DefineNamedFunc{gauge}{C}{\Lambda x \in V \. \gamma(x|C)}{
			\Lambda x \in V \. \inf \Big\{ \lambda \in  \Reals_{++} \Big| x \in \lambda C \Big\}
	}
	\\
	\Theorem{ConvexFunctionHasConvexLevelSets}
	{
		\NewLine ::		
		\forall V \in \VS{\Reals} \.
		\forall f : \Convex(V,V) \.
		\forall \alpha \in \EReals \.
		\Convex\Big(V, \{ v \in V : f(v) \ge \alpha  \} \Big)
	}
	\NoProof
	\\
	\Theorem{ConvexFunctionHasConvexStrictLevelSets}
	{
		\NewLine ::		
		\forall V \in \VS{\Reals} \.
		\forall f : \Convex(V,V) \.
		\forall \alpha \in \EReals \.
		\Convex\Big(V, \{ v \in V : f(v) > \alpha  \} \Big)
	}
	\NoProof
}\Page{
	\Theorem{ConvexlyBoundedRegionIsConvex}
	{
		\NewLine ::
		\forall V \in \VS{\Reals} \.
		\forall I \in \SET \.
		\forall \alpha : I \to \EReals \.
		\forall f : I \to \Convex(V,V) \.
		\Convex\Big(V, \{ v \in V : \forall i \in I \. f_i(v) > \alpha_i  \} \Big)
	}
	\\
	\Theorem{GeneralizedAMGMIneq}
	{
		\forall n \in \Nat \.
		\forall \lambda : \Reals_{+}^n \.
		\forall x : \Reals_{++}^n \.
		\forall \aleph : \sum^n_{i=1} \lambda_i = 	
		\sum^n_{i=1} \lambda_i x_i \ge \prod^n_{i=1} x_i^{\lambda_i}
	}
	\Explain{ By Jensen inequality for natural logarithm 
	$\ln\left(\sum^{n}_{i=1} \lambda_i x_i \right) \ge \sum^n_{i=1} \lambda_i\ln(x_i)$}
	\Explain{ Then by exponentiating both parts
		$ \sum^n_{i=1} \lambda_i x_i \ge \prod^n_{i=1} x_i^{\lambda_i} $}
	\EndProof
	\\
	\DeclareType{\PHomog}{\prod V : \VS{\Reals} \. ?\Big(V \to (-\infty,+\infty] \Big)}
	\DefineType{f}{\PHomog}{ \forall v \in V \. \forall \alpha \in \Reals_{++} \. f(\alpha v) = \alpha f(v)   }
	\\
	\Theorem{PositiveHomogeneousZeroPositivity}
	{
		\forall V : \VS{\Reals} \.
		\forall f : \PHomog(V) \.
		f(0) \ge 0
	}
	\Explain{ 
		Note that $f(0) = f(t0) = t f(0)$ for all $t \in \Reals_{++}$}
	\Explain{
		This means that $f(0)$ is either $0$ or $+\infty$}
	\EndProof
	\\
	\Theorem{PositiveHomogeneousConvexity}
	{
		\forall V : \VS{\Reals} \.
		\forall f : \PHomog(V) \. \NewLine \.
		\Convex(V,V,f)
		\iff
		\forall x,y \in V \. f(x + y) \le f(x) + f(y)
	}
	\Explain{
		$(\Rightarrow):$ assume $f$ is convex}
	\Explain{
		Then 
		$f(x + y) = f\left(\frac{2}{2}x + \frac{2}{2}{y}\right) \le \frac{1}{2} f(2x) + \frac{1}{2} f(2y) = f(x) + f(y)$
		for any $x,y \in V$}
	\Explain{
		$(\Leftarrow):$ assume the inequality holds
	}
	\Explain{
		Then 
		$f\Big(\lambda x + (1+\lambda)y\Big) \le f(\lambda x) + f\Big( (1-\lambda)y\Big) = \lambda f(x) + (1-\lambda)f(y)$
		when $\lambda \in (0,1)$ and $x,y \in V$
	}
	\Explain{
		Otherwise, when $\lambda = 0,1$, convexity condition holds trivially}
	\EndProof
	\\
	\Conclude{\Conic}{\lambda V \in \VS{\Reals} \. \Convex(V,V) \times \PHomog(V) }
	{
		 \VS{\Reals} \to \Type
	}
	\\
	\Theorem{ConicIneq}
	{
		\forall V : \VS{\Reals} \.
		\forall f : \Convex(V,V) \And \PHomog(V) \.
		\forall n \in  \Nat \.
		\forall x \in V^n \. \NewLine \.
		\forall \lambda \in \Reals^n_{++} \. 
		f\left( \sum^n_{i=1} \lambda_i x \right)  \le \sum^n_{i=1} \lambda_i f(x_i)
	}
	\Explain{ Iterate previous theorem}
	\EndProof
}\Page{
	\Theorem{ConicEpigraph}
	{
		\forall V \in \VS{\Reals} \.
		\forall f : V \to (-\infty,+\infty) \.
		\Conic(V,f)
		\iff
		\CC(V,\epi f)
	}
	\NoProof
	\\
	\Theorem{ConicIsSupersymmetric}
	{
		\forall V \in \VS{\Reals} \.
		\forall f \in \Conic(f) \.
		\forall v \in V \.
		f(v) \ge - f(-v)
	}
	\Explain{ Write
		$
			f(x) + f(-x) \ge f( x - x) = f(x) \ge 0
		$}
	\Explain{
		So
		$
				f(x) \ge -f(-x)
		$}
	\EndProof
	\\
	\Theorem{ConicIsLinearIffSymmetric}
	{
		\forall V \in \VS{\Reals} \.
		\forall f \in \Conic(f) \.
		f \in V^* 
		\iff
		\forall v\in V \. f(-v) = -f(v)
	}
	\Explain{
		$(\Rightarrow):$ this is trival}
	\Explain{
			$(\Leftarrow):$ assume that the property holds
		}
	\Explain{
		Let $x,y \in V$
	}
	\Explain{
		Then $f(x) + f(y) \ge f(x + y) \ge -f(-x-y) \ge -f(-x) - f(-y) = f(x) + f(y)$
	}
	\Explain{
		This mean $f(x) + f(y) = f(x + y) $
	}
	\Explain{
		But as $x$ and $y$ were arbitrary $f$ must be additive and hence linear}
	\EndProof
	\\
	\DeclareType{StrictlyConvexFunction}{\prod V : \VS{\Reals} \. ?\PCF(V)}
	\DefineType{f}{StrictlyConvexFunction}
	{
		\NewLine \iff		
		\forall \lambda \in [0,1] \.
		\forall x,y \in V \.
		\forall \aleph : x \neq y \.
		f\Big(\lambda x + (1-\lambda)y \Big) < \lambda f(x) + (1-\lambda)f(y)
	}
	\\
	\Theorem{SquareNoremIsStrictlyConvex}
	{
		\forall V : \TYPE{NormedSpace}(\Reals) \. \SCF\Big( V, \|\bullet\|^2 \Big)
	}
	\ExplainFurther{
		$\forall x,y \in V, \lambda \in (0,1) \. 
				\Big\| \lambda x + (1 - \lambda) y \Big\|^2 \le    
				\Big( \lambda \|x\| + (1-\lambda) \|y\| \Big)^2  <
				\lambda^2\|x\|^2 + (1-\lambda)^2\|y\|^2 <$}
		\Explain{
				$ < \lambda \|x\|^2 + (1-\lambda)\|y\|^2 
		$}
	\EndProof
}
\newpage
\subsection{Convexity Preserving Operations}
\Page{
	\Theorem{ConvexComposition}
	{
		\forall V \in \VS{\Reals} \.
		\forall D \subset V
		\forall f : \Convex(V,D) \.
		\forall \phi : \Convex \And \TYPE{Increasing}(\Reals,\Reals) \.
		\Convex(V,D,\phi \circ f)
	}
	\Explain{ Assume $x,y \in \dom f, \lambda \in [0,1]$}
	\Explain{
		Then $\phi\Big( f\big( \lambda x + (1-\lambda) y \big)\Big) \le \phi\Big( \lambda f(x) + (1-\lambda) f(y)\Big)
		\le \lambda \phi \circ f(x) + (1-\lambda) \phi \circ f(y)$}
	\EndProof
	\\
	\Theorem{ConvexFunctionFromSet}
	{
		\forall V \in \VS{\Reals} \.
		\forall C : \Convex(V \oplus \Reals) \.
		\Convex\Big(V,V, \Lambda v \in V \. \inf \big\{t|(v,t)\in C \big\}\Big)
	}
	\Explain{
	This is function has cinvex epigraph}
	\EndProof	
	\\
	\\
	\Theorem{InfimalConvolutionIsConvex}
	{
		\forall V \in \VS{\Reals} \.
		\forall n \in \Nat \.
		\forall f : \{1,\ldots,n\} \to \PCF(V) \. \NewLine \.
		\Convex\left( V, V, \Lambda x \in V \. \inf \left\{ \sum^n_{k=1} f_k(v_k) \Bigg| v \in V^n, 
			\sum^n_{k=1} v_k = x \right\} \right)
	}
	\Explain{
		Let $g = \inf \left\{ \sum^n_{k=1} f_k(v_k) \Bigg| v \in V^n, 
			\sum^n_{k=1} v_k = x \right\}$}
	\Explain{$C = \sum^n_{k=1} \epi f_k$ is convex}
	\ExplainFurther{ 
		A tuple $(x,\phi) \in C$ if there is a sequence $(v,\psi) \in (V\oplus\Reals)^n$
		such that $x = \sum^n_{k=1} v_k,\phi = \sum^n_{k=1} \psi_k$}
	\Explain{		
		 and 
		$f(v_k) \le \psi_k$ for every $k \in \{1,\ldots,n\}$}
	\Explain{
		Thus $\phi = \sum^n_{k=1} \psi_k \ge \sum^n_{k=1} f(v_k) \ge  g(x)$,
		so $(x,\phi) \in \epi g$}
	\Explain{ Then $g$ can be constructed from set $C$}
	\EndProof
	\\
	\DeclareFunc{infimalConvolution}{\prod_{V \in \VS{\Reals}} \prod^\infty_{n=1} 
		\Big(\{1,\ldots,n\} \to \PCF(V) \Big) \to \Convex(V,V)  }
	\DefineNamedFunc{infimalConvolution}{f}{\bigsquare^n_{k=1} f_i}
	{
		\inf \left\{ \sum^n_{k=1} f_k(v_k) \Bigg| v \in V^n, \sum^n_{k=1} v_k = x \right\} 
	}
	\\
	\DeclareFunc{ConvexDelta}{\prod_{V \in \VS{\Reals}}  V \to \PCF(V)  }
	\DefineNamedFunc{convexDelta}{a}{\delta_a}
	{
		\Lambda x \in V \. \If x = a \Then 0 \Else +\infty 
	}
}\Page{
	\Theorem{GraphTranslationByInfimalConvolution}
	{		
		\forall V \in \VS{\Reals} \.
		\forall f : \PCF(V) \.		
		\forall a,v \in V \.
		(\delta_a \square f)(v) = f(v - a)
	}
	\Explain{ Clearly $(\delta_a \square f)(v) = \min \Big\{ f(v-a),+\infty \Big\}$ }
	\EndProof
	\\
	\Theorem{InfimalConvolutionDomain}
	{		
		\forall V \in \VS{\Reals} \.
		\forall f : \PCF(V) \.		
		\forall a,v \in V \.
		\dom ( f \square g) = \dom f + \dom g
	}
	\Explain{ Obvious }
	\EndProof
	\\
	\Theorem{DistanceExpression}
	{
		\forall V : \TYPE{NormedSpace}(\Reals) \.
		\forall C : \Convex(V) \.
		d_V(C,\bullet) = \| \bullet \|  \square \chi( \bullet | C)
	}
	\EndProof
	\\
	\Theorem{InfimalConvolutionDefinesCommutativeMonoid}
	{
		\NewLine ::		
		\forall V \in \VS{\Reals} \. \NewLine \.
		\TYPE{CommutativeMonoid}\Big( 
			\Convex(V,V), 
			\Lambda f,g \in \Convex(V,V) \.
			\Lambda x \in  V \. \inf \Big\{ \phi \Big| (v,\phi) \in (\epi f + \epi g)  \Big\}  			
			 \Big)
	}
	\Explain{ $\delta_0$ is a neutral element, comutativity and associativity is almost obvious}
	\EndProof
	\\
	\DeclareFunc{rightScalarMultiplication}
	{
		\prod_{V \in \VS{\Reals}} \.
		\Convex(V,V) \to \Reals_+ \to \Convex(V,V)
	}
	\DefineNamedFunc{rightScalarMultiplication}{f,\lambda}{f\lambda}
	{
		\THM{ConvexFunctionFromSet}(V,\lambda \epi f)
	}
	\\
	\Theorem{RightScalarMultiplicationExpression}
	{
		\NewLine ::		
		\forall V \in \VS{\Reals} \. \forall f : \Convex(V,V) \.
		\forall \lambda \in \Reals_{++} \.
		\forall x \in X \.
		f\lambda(x) =\lambda f(\lambda^{-1}x)
	}
	\Explain{ Obvious}
	\EndProof
	\\
	\Theorem{RightScalarMultiplicationByZero}
	{
		\forall V \in \VS{\Reals} \. \forall f : \Convex(V,V) \.
		f0(x) = \delta(0)
	}
	\Explain{ Obvious}
	\EndProof
	\\
	\Theorem{ConicityByRightMultiplication}
	{
		\forall V \in \VS{\Reals} \.
		\forall f : \Convex(V,V) \.
		\Conic(V,f)
		\iff
		\forall \lambda \in \Reals_{++} \. f\lambda = f
	}
	\Explain{ Follows from the expression for right multiplication}
	\EndProof
}\Page{
	\DeclareFunc{conicClosure}
	{
		\prod_{ V \in \VS{\Reals} }
		 \CF(V) \to \Conic(V)
	}
	\DefineFunc{conicClosure}{\cone f }{\THM{ConvexFunctionFromSet}\Big(V. \cone \epi f\Big)}
	\\
	\Theorem{GaugeExpression}
	{
		\forall V \in \VS{\Reals} \. 
		\forall C : \Convex \And \TYPE{NonEmpy}(V) \.
		\gamma(\bullet|C) = \cone \Big( \chi(\bullet|C) + 1  \Big)
	}
	\Explain{ $(x,\phi) \in \epi \gamma(\bullet|C)$ iff $x \in \lambda C$ and $0 < \lambda \le \phi $}
	\Explain{This means that $(x,\lambda) \in \cone C \times \{1\} \subset \cone \epi\Big( \chi(\bullet|C) + 1  \Big)$}
	\Explain{ So $(x,\phi) \in \cone C \times \{ \phi/\lambda \} \subset
		\cone \epi \Big( \chi(\bullet|C) + 1  \Big) =  \epi \cone \Big( \chi(\bullet|C) + 1  \Big)$}
	\Explain{
		On the other hand id $(x,\psi) \in \epi \cone \Big( \chi(\bullet|C) + 1$ then there
		exists $\lambda \in \Reals_{++}$ such that $\lambda x \in C$ and $\lambda \psi \ge 1$
	}
	\Explain{  
		But this means that $\psi \ge \lambda^{-1} \ge \gamma(x|C)$}
	\Explain{
			Thus $(x,\psi) \in \gamma(\bullet|C)$		
	}
	\Explain{
		And both functions are equal by equality of epigraphs}
	\EndProof
	\\
	\Theorem{SupremumIsConvex}
	{
		\forall V \in \VS{\Reals} \.
		\forall I \in \SET \.
		\forall f : I \to \CF(V) \.
		\CF(V,\sup_{i \in I} f_i)
	}
	\Explain{ 
		$\epi \sup_{i \in I} f_i = \bigcap_{i \in I} \epi f_i$ is convex}
	\\
	\DeclareFunc{convexHull}
	{
		\prod_{V \in \VS{\Reals}} \prod_{I \in \SET} 
		\Big(I \to V \to \EReals) \to \CF(V)
	}
	\DefineNamedFunc{convexHull}{f}{\conv_{i \in I} f_i}
	{
		\THM{ConvexFunctionFromSet}\left(V, \conv \bigcup_{i \in \I} \epi f_i\right)	
	}
	\\
	\Theorem{ConvexHullExpression}
	{
		\forall V \in \VS{\Reals} \.
		\forall I \in \SET \.
		\forall f : I \to V \to (-\infty,+\infty] \.
		\forall v \in V \. \NewLine \.
		\conv_{i \in I} f_i(x) =
		\inf \left\{  \sum_{i \in I} \lambda_if_i(v_i) \Bigg| \lambda \in \Reals_+^{\oplus I}, v : I \to V,
		\sum_{i \in I} \lambda_i = 1, \sum_{i \in I} \lambda_i v_i = x \right\}
	}
	\Explain{ This follows from the thorough  examination of the definition}
	\\
	\DeclareFunc{convexPullback}
	{
		\prod V,W \in \VS{\Reals} \.
		\VS{\Reals}(V,W) \to \CF(W) \to \CF(V)
	}
	\DefineNamedFunc{convexPullback}{f,T}{fT}
	{
		f \circ T
	}
	\\
	\DeclareFunc{convexPushforward}
	{
		\prod V,W \in \VS{\Reals} \.
		\VS{\Reals}(V,W) \to \CF(V) \to \CF(W)
	}
	\DefineNamedFunc{convexPullback}{f,T}{T_* f}
	{
		\Lambda w \in W \. \inf \{ f(v) | w = Tv \}
	}
}
\newpage
\subsection{Metric and Topological Properties}
\Page{
	\Theorem{SphericalBound}
	{
		\NewLine ::		
		\forall V \in \mathsf{BAN}(\Reals) \.
		\forall f : \PCF(V) \. 
		\forall c \in \dom f \.
		\forall \rho \in \Reals_{++} \.
		\forall \aleph : \eta < + \infty \.
		\forall \alpha \in (0,1) \. \NewLine \.
		\forall x \in \Cell_V(c,\alpha\rho) \.
		\Big| f(x) - f(c) \big| \le \alpha\Big( \eta  - f(c)  \Big) \NewLine 
		\quad \where \quad \eta = \sup f\Big(\Cell_V(c,\rho)\Big)
	}
	\Explain{
		By convexity
		$f(x) - f(c) =
			f\left(  (1-\alpha)c + \alpha\left(\frac{x - (1-\alpha)c}{\alpha}\right)\right) - f(c) \le	
			\alpha \left( f\left(c + \frac{x-c}{\alpha}\right) - f(c) \right)	
		$}
	\Explain{
		If $x \in \Cell_V(c,\alpha\rho)$ then $c + \frac{x-c}{\alpha} \in \Cell_V(c,\rho)$
	}
	\Explain{
		So $f(x) - f(c) \le \alpha\Big( \eta - f(c)\Big)$}
	\ExplainFurther{
		On the other hand
		$f(c) - f(x) =
			f\left(  \frac{x}{1+\alpha} + \frac{\alpha}{1-\alpha} \frac{(1+\alpha)c-x}{\alpha}\right) - f(x) \le$}
	\Explain{$	
			\le \frac{\alpha}{1+\alpha} \left( f\left(c + \frac{c-x}{\alpha}\right) - f(x) \right)	
			\le  \frac{\alpha}{1+\alpha} ( \eta - f(x)) = 
			\frac{\alpha}{1+\alpha} \Big(\eta  - f(c)\Big) +   \frac{\alpha}{1+\alpha} \Big(f(c)  - f(x)\Big)
		$}
	\Explain{
		So by rearanging inequalities one gets
		$ f(c) - f(x) \le \alpha \Big(\eta - f(x)\Big)$}
	\\
	\Theorem{
		LocalLipschitzContinuity
	}
	{
		\NewLine ::		
		\forall V \in \mathsf{BAN}(\Reals) \.
		\forall f : \PCF(V) \. 
		\forall c \in \dom f \.
		\forall \rho \in \Reals_{++} \.
		\forall \aleph : \delta < + \infty \. \NewLine \ .
		\left(\frac{\delta}{\rho}\right)\hyph\Lip\Big( \Cell(c,\rho),\Reals, f_{|\Cell(c,\rho)} \Big)
		\NewLine		
		\quad \where \quad \eta = \diam f\Big(\Cell_V(c,\rho)\Big)
	}
	\Explain{
		Assume $x,y \in \Cell_V(c,\rho)$ sych that $x \neq y$}
	\Explain{
		Let $\alpha=\frac{\|x-y\|}{\|x-y\| + \rho} < \frac{\|x-y\|}{\rho}$	and 
		$z = x + \frac{1-\alpha}{\alpha}(x - y)) $}
	\Explain{
		Then, $\| z - c \| \le \|z - x\| + \| x - c\| \le \frac{1-\alpha}{\alpha}\|x-y\|  + \rho \le 2\rho$,
		so $z \in \Cell_V(c,2\rho)$}
	\Explain{
		Thus, by convexity
		$
			f(x) = 
			f\Big( \alpha z + (1 - \alpha)y \Big) \le
		 	f(y)  + \alpha( f(z) - f(y ) \le 
		 	f(y) + \alpha \delta \le 
		 	f(y) + \frac{\delta}{\rho} \| x - y \| 
		$
	}
	\Explain{ From symmetry
		$	
			|f(x) - f(y)| \le \frac{\delta}{\rho} \|x - y\|
		$ 
	}
	\EndProof
}\Page{
	\Theorem{ContinuityByBound}
	{
		\NewLine ::		
		\forall V \in \mathsf{BAN}(\Reals) \.
		\forall f : \PCF(V) \.
		\forall x \in V \.
		\forall U \in \U(x) \. \NewLine \.
		\forall \aleph : \TYPE{Bounded}(V,U,f_{|U}) \.
		\intx \dom f \Arrow{f_{|\intx \dom f}} \Reals : \TOP
	}
	\Explain{ 
		Assume $v \in U \cap \rint \dom f$}
	\ExplainFurther{
		As $v$ belongs to relative intrior there are $w \in \dom f, \rho \in Reals_+, \lambda \in (0,1)$}
	\Explain{
		such that $v \in \lambda\Cell(x,\rho) + (1-\lambda)w$
		and $\Cell(x,\rho) \subset U$}
	\ExplainFurther{
		Then $f(x) \le \lambda f(y) + (1-\lambda)f(w) \le \lambda \beta + (1-\lambda)f(w)$,}
	\Explain{
		where $x \in \lambda\Cell(x,\rho) + (1-\lambda)w,y \in \Cell(y,\rho)$ and $\beta$
		is the bound for $U$}
	\ExplainFurther{
		So by the previous theorem $f$ is locally Lipshitz and continuous on $\rint \dom f$}
	\Explain{ 
		and so $f$ is actually continuous on $\rint \dom f$}
	\EndProof
	\\
	\Theorem{ContinuityByFiniteDimension}
	{
		\NewLine ::		
		\forall V \in \Euc \.
		\forall f : \PCF(V) \. 
		\intx \dom f \Arrow{f_{|\intx \dom f}} \Reals : \TOP
	}
	\Explain{
		Let $n = \dim V$}
	\Explain{
		Assume  $x \in V$ and $\rho \in \Reals_{++}$ such that $\Cell(x,\rho) \subset \dom f$}
	\Explain{
		Then there exists a simplicital set  $\{v_1, \ldots, v_{n+1}\}$ such that
		$\Cell(x,\rho) \subset \conv(v_1,\ldots,v_{n+1}) \subset \dom f $}
	\Explain{
		But this means that $\sup f\Big( \Cell(x,\rho)\Big) \le \max_{i=1,\ldots,n+1} f(v_i) $
	}
	\ExplainFurther{
		So by the previous theorem $f$ is locally Lipshitz and continuous on $\rint \dom f$}
	\Explain{ 
		and so $f$ is actually continuous on $\rint \dom f$}
	\EndProof
	\\
	\Theorem{NonEmptyInteriorCondition}
	{
		\forall V \in \mathsf{BAN}(\Reals) \.	
		\forall f : \PCF(V) \.
		\forall x \in V \.
		\forall U \in \U(x) \. \NewLine \.
		\forall \aleph : \TYPE{Bounded}(V,U,f_{|U}) \.
		\intx \epi f \neq \emptyset
	}
	\Explain{
		We can find radius $\rho$ Lipschitz constant $\beta$ such that
		$\Big| f(v) - f(w)\Big| \le \beta \| v - w \| \le \beta \rho$
		for every  $v,w \in \Cell(x,\rho)$}
	\Explain{
		Select $\delta \in (2\beta\rho,+\infty)$
		and	set $\gamma = \min(\rho,\delta/2) > 0$}
	\Explain{
		And let $(y,\phi) \in V \oplus \Reals$ such that 
		$\Big\| \big(y,\phi\big) - \big(x,f(x)+\rho\big)\Big\|^2 \le \gamma^2$ }
	\Explain{
		Then $\| y - x \| \le \gamma \le \rho$ and
		$\Big|\phi - \big( f(x) + \delta\big)\Big| \le \gamma \le \delta/2$}
	\Explain{
		So,
		$
			f(y) < f(x) + \delta/2 = f(x) + \delta - \delta/2 \le f(x) + \delta - \gamma \le \phi
		$
	}
	\Explain{ 
		Thus $(y,\phi) \in \epi f$}
	\Explain{ 
		As $(y,\phi)$ was arbitrary, the whole sphere is a subset of $\epi f$}
	\Explain{
		So $\intx \epi f \neq \emptyset$
	}
	\EndProof
}\Page{
	\Theorem{InteriorLevelSet}
	{
		\NewLine ::		
		\forall V : \BAN{\Reals} \.
		\forall f : \UsC\Big(V,(-\infty,+\infty]\Big) \.
		\forall x \in V \.
		\forall \aleph : f(x) < 0 \.
		x \in \intx f^{-1}(-\infty, 0]
	}
	\NoProof
	\\
	\Theorem{ConvexLevelSetInterior}
	{
		\forall V : \BAN{\Reals} \.
		\forall f : \CF(V) \.
		\forall x  \in V \.
		\forall \aleph : f(x) < 0 \. \NewLine \.
		\intx f^{-1}(-\infty, 0] \subset f^{-1}(-\infty,0)
	}
	\NoProof
	\\
	\Theorem{ConvexUsCLevelSetInteriorEq}
	{
		\NewLine \. 		
		\forall V : \BAN{\Reals} \.
		\forall f : \CF(V) \.
		\forall \aleph : \UsC\Big(f_{|f^{-1}(-\infty,0)},(-\infty,0)\Big) \. \NewLine \.
		\intx f^{-1}(-\infty,0] = \intx f^{-1}(-\infty,0)
	}
	\NoProof
	\\
	\Theorem{ConvexEucLevelSetInteriorEq}
	{
		\NewLine \. 		
		\forall V \in \Euc \.
		\forall f : \CF(V) \.
		\forall \aleph : \dom f \in \T(V) \.
		\intx f^{-1}(-\infty,0] = \intx f^{-1}(-\infty,0)
	}
	\NoProof
}
\newpage
\subsection{Closures}
\Page{
	\Theorem{LowerSemicontinuityByEpigraph}
	{
		\forall V \in \Euc \.
		\forall f : V \to \TOPVS{\Reals} \.
		\NewLine 
		\LsC(V,\EReals,f)
		\iff
		\Closed(V\oplus \Reals, \epi f)
	}
	\Explain{
		If $f$ is lower semicontinuous then $\lim \inf_{x \to v} f(x) = f(v)$}
	\Explain{
		So the epigraph must be closed}
	\Explain{
		Thus, result is basically obvious}
	\\
	\DeclareFunc{closure}
	{
		\prod_{V \in \TOPVS{\Reals}} \CF(V) \to \Big(\CF(V) \And \LsC(V,\Reals)\Big)
	}
	\DefineNamedFunc{closure}{f}{\cl f}
	{
		\If  f > -\infty \Then \THM{FunctionFromSet}(V,\cl \epi f) \Else -\infty
	}
	\\
	\DeclareType{ClosedFunction}{\prod V \in \TOPVS{\Reals} \. ?\CF(V) }
	\DefineType{f}{ClosedFunction}{\cl f = f}
	\\
	\Theorem{ImproperDomain}
	{
		\forall V \in \TOPVS{\Reals} \.
		\forall f : \CF(V) \. \NewLine \.
		\forall \aleph : -\infty \in \im f \.
		\forall x \in \rint \dom f \.
		f(x) = -\infty
	}
	\Explain{
		If there is a point $p \in V$ such that $f(p) = -\infty$ then $p \in \dom f$}
	\Explain{
		Also assume that $u \in \rint \dom f$		
	}
	\Explain{
		By propertirs of relative interiot there exists $x \in \dom f$ suxh that $u \in (p,x)$}
	\Explain{
		So there is $\lambda \in (0,1)$ suxh that $u = \lambda u + (1-\lambda)x$
	}
	\Explain{
		But $(p,\alpha) \in \epi f$ for any  arbitrary $\alpha \in \Reals$}
	\Explain{
		Thus $\Big( u, \lambda \alpha + (1-\lambda)f(x) \Big) \in \epi f$ fo any $\alpha$
	}
	\Explain{
		And by taking the limit $\alpha \to -\infty$ we see that it must be the case
		that $f(u) = -\infty$
	}
	\EndProof
	\\
	\Theorem{ContinuityByClosedness}
	{
		\NewLine ::		
		\forall V \in \mathsf{BAN}(\Reals) \.
		\forall f : \ClF(V) \. 
		\intx \dom f \Arrow{f_{|\intx \dom f}} \Reals : \TOP
	}
	\NoProof
	\\
	\Theorem{ConvexLsCLevelSetInteriorEq}
	{
		\NewLine \. 		
		\forall V \in \Euc \.
		\forall f : \ClF(V) \.
		\forall \aleph : \dom f \in \T(V) \. \NewLine \.
		\intx f^{-1}(-\infty,0] = \intx f^{-1}(-\infty,0)
	}
	\NoProof
}\Page{
	\Theorem{ClosedFunctionSupremum}
	{
		\forall V \in \TOPVS{\Reals} \.
		\forall I \in \SET \.
		\forall f : I \to \ClF(V) \. \NewLine \.
		\sup f  \in \ClF(V)  
	}
	\Explain{
		The epigraph of the supremum is the intersection of epigraphs}
	\Explain{
		Then use the fact that intersection of closed sets is closed}
	\EndProof	
	\\
	\Theorem{ClosedFunctionSum}
	{
		\NewLine \. 		
		\forall V \in \TOPVS{\Reals} \.
		\forall f, g \in \ClF \And \PCF(V) \. \NewLine \.
		f + g \in  \ClF \And \PCF(V)
	}
	\NoProof
	\\
	\Theorem{ClosedFunctionSum2}
	{
		\NewLine \. 		
		\forall V \in \TOPVS{\Reals} \.
		\forall I \in \SET \.
		\forall f : \Nat \to \ClF(V) \. \NewLine \.
		\forall \aleph :  \inf_{i \in I} f_i \ge 0 \.
		\sum_{i \in I} f_i  \in  \ClF(V)
	}
	\NoProof
	\\
	\Theorem{DegenerateClosedForm}
	{
		\NewLine \.
		\forall V \in \TOPVS{\Reals} \.
		\forall f : \ClF(V) \.
		\forall \aleph : -\infty \in \im f \.
		\im f = \{-\infty, +\infty\} 
	}
	\NoProof
	\\
	\Theorem{ClosureLevelSets}
	{
		\NewLine:
		\forall V \in \TOPVS{\Reals} \.
		\forall f : \CF(V)  \.
		\cl \Big( f^{-1}(-\infty,0) \Big) = \cl \Big(  f^{-1}(-\infty,0] \Big) =
		\Big( \cl f\Big)^{-1}(-\infty,0) 
	}
	\NoProof
	\\
	\Conclude{\PClF}{\PCF \And \ClF}{\TOPVS{\Reals} \to \Type}
	\\
	\Theorem{ConvexLimit}
	{
		\forall V \in \TOPVS{\Reals} \. 
		\forall f : \PClF(V) \.
		\forall x \in \dom f \.
		\forall y \in V \. 
		\lim_{\lambda \downarrow 0} g(\lambda) = f(y)
		\NewLine
		\quad \where \quad
		g = \Lambda \lambda \in [0,1] \. f\Big(\lambda x  + (1-\lambda)y \Big)
	}
	\Explain{
			In $  f\Big(\lambda x  + (1-\lambda)y \Big) \le \lambda f(x) + (1 - \lambda) f(y) $
			the majorant is continuous in $lambda$}
	\Explain{ 
		So in  $f(x) \le \lim_{\lambda \to \infty} f\Big(\lambda x  + (1-\lambda)y \Big) \le f(x)$   holds}
	\EndProof
}\Page{
	\Theorem{ProperConvexContinuity}
	{
		f \in \PClF(\Reals) \.
		f_{|\cl \dom f} \in \TOP\Big(\cl \dom f ,\EReals\Big)
	}
	\Explain{Use convex limits as above}
	\EndProof
	\\
	\Theorem{ConvexExtension}
	{
		\NewLine		
		\forall V \in \TOPVS{\Reals} \.
		\forall f,g \in \PClF(V \oplus \Reals) \. \NewLine \. 
		\forall \aleph : \dom f \cup \dom g \subset V \oplus \Reals_+ \.
		\forall \beth :  f_{|V \times \Reals_{++}} = g_{|V \times \Reals_{++}} \.
		f = g
	}
	\NoProof
}
\newpage
\subsection{Affine Minorization}
\subsection{Recession}
\section{Duality}
\section{(Sub)differential Calculus}
\section{From Optimization to Convex Algebra}
\section*{Sources}
\begin{enumerate}
\item Convex Analysis --- R. T. Rockaffeler  1972
\item Convex Analysis  and Monotone Operator Theory in Hilbert Spaces -- H. H. Bauschke, P. L. Combetes 2010

\end{enumerate}
\end{document}