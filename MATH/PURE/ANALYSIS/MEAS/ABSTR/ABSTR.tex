\documentclass[12pt]{scrartcl}
\usepackage{mathtools}
\usepackage[T2A]{fontenc}
\usepackage[utf8]{inputenc}
\usepackage{amsmath}
\usepackage{amsfonts}
\usepackage{hyperref}
\usepackage{amssymb}
\usepackage{ wasysym }
\usepackage{ upgreek }
\usepackage[dvipsnames]{xcolor}
\usepackage[a4paper,top=5mm, bottom=20mm, left=10mm, right=2mm]{geometry}
\renewcommand\pagemark{{\usekomafont{pagenumber}\thepage\ }}
%Markup
\newcommand{\TYPE}[1]{\textcolor{NavyBlue}{\mathtt{#1}}}
\newcommand{\FUNC}[1]{\textcolor{Cerulean}{\mathtt{#1}}}
\newcommand{\LOGIC}[1]{\textcolor{Blue}{\mathtt{#1}}}
\newcommand{\THM}[1]{\textcolor{Maroon}{\mathtt{#1}}}
%META
\renewcommand{\.}{\; . \;}
\newcommand{\de}{: \kern 0.1pc =}
\newcommand{\extract}{\LOGIC{Extract}}
\newcommand{\where}{\LOGIC{where}}
\newcommand{\If}{\LOGIC{if} \;}
\newcommand{\Then}{ \; \LOGIC{then} \;}
\newcommand{\Else}{\; \LOGIC{else} \;}
\newcommand{\IsNot}{\; ! \;}
\newcommand{\Is}{ \; : \;}
\newcommand{\DefAs}{\; :: \;}
\newcommand{\Act}[1]{\left( #1 \right)}
\newcommand{\Example}{\LOGIC{Example} \; }
\newcommand{\Theorem}[2]{& \THM{#1} \, :: \, #2 \\ & \Proof = \\ } 
\newcommand{\DeclareType}[2]{& \TYPE{#1} \, :: \, #2 \\} 
\newcommand{\DefineType}[3]{& #1 : \TYPE{#2} \iff #3 \\} 
\newcommand{\DefineNamedType}[4]{& #1 : \TYPE{#2} \iff #3 \iff #4 \\} 
\newcommand{\DeclareFunc}[2]{& \FUNC{#1} \, :: \, #2 \\}  
\newcommand{\DefineFunc}[3]{&  \FUNC{#1}\Act{#2} \de #3 \\} 
\newcommand{\DefineNamedFunc}[4]{&  \FUNC{#1}\Act{#2} = #3 \de #4 \\} 
\newcommand{\NewLine}{\\ & \kern 1pc}
\newcommand{\Page}[1]{ \begin{align*} #1 \end{align*}   }
\newcommand{ \bd }{ \ByDef }
\newcommand{\NoProof}{ & \ldots \\ \EndProof}
%LOGIC
\renewcommand{\And}{\; \& \;}
\newcommand{\ForEach}[3]{\forall #1 : #2 \. #3 }
\newcommand{\Exist}[2]{\exists #1 : #2}
\newcommand{\Imply}{\Rightarrow} 
\newcommand{\Intro}{\LOGIC{I}}
\newcommand{\Elim}{\LOGIC{E}}
%TYPE THEORY
\newcommand{\Type}{\TYPE{Type}}
%\DeclareMathOperator*{\dom}{dom}
%%STD
\newcommand{\Int}{\mathbb{Z} }
\newcommand{\NNInt}{\mathbb{Z}_{+} }
\newcommand{\Reals}{\mathbb{R} }
\newcommand{\Complex}{\mathbb{C}}
\newcommand{\Rats}{\mathbb{Q} }
\newcommand{\Sphere}{\mathbb{S}}
\newcommand{\Ball}{\mathbb{B}}
\newcommand{\Nat}{\mathbb{N} }
\newcommand{\EReals}{\stackrel{\mathclap{\infty}}{\mathbb{R}}}
\newcommand{\ERealsn}[1]{\stackrel{\mathclap{\infty}}{\mathbb{R}}^{#1}}
\DeclareMathOperator*{\centr}{center}
\DeclareMathOperator*{\argmin}{arg\,min}
\DeclareMathOperator*{\id}{id}
\DeclareMathOperator*{\im}{Im}
\DeclareMathOperator*{\supp}{supp}
\newcommand{\EqClass}[1]{\TYPE{EqClass}\left( #1 \right)}
\newcommand{\End}{\mathrm{End}}
\newcommand{\Aut}{\mathrm{Aut}}
\mathchardef\hyph="2D
\newcommand{\ToInj}{\hookrightarrow}
\newcommand{\ToMono}{\xhookrightarrow}
\newcommand{\ToSurj}{\twoheadrightarrow}
\newcommand{\ToEpi}{\xtwoheadrightarrow}
\newcommand{\ToBij}{\leftrightarrow}
\newcommand{\ToIso}{\xleftrightarrow}
\newcommand{\Arrow}{\xrightarrow}
\newcommand{\Set}{\TYPE{Set}}
\newcommand{\du}{\; \triangle \;}
\renewcommand{\c}{\complement}
\renewcommand{\i}{\mathbf{i}}
\newcommand{\Eqmod}[3]{#1 = #2 \quad \mathrm{mod} \quad #3}
%%ProofWritting
\newcommand{\Say}[3]{& #1 \de #2 : #3, \\}
\newcommand{\SayIn}[3]{& #1 \de #2 \in #3, \\}
\newcommand{\Conclude}[3]{& #1 \de #2 : #3; \\}
\newcommand{\Derive}[3]{& \leadsto #1 \de #2 : #3, \\}
\newcommand{\DeriveConclude}[3]{& \leadsto #1 \de #2 : #3 ; \\} 
\newcommand{\Assume}[2]{& \LOGIC{Assume} \; #1 : #2, \\}
\newcommand{\AssumeIn}[2]{& \LOGIC{Assume} \; #1 \in #2, \\}
\newcommand{\As}{\; \LOGIC{as } \;} 
\newcommand{\ByDef}{\LOGIC{E}}
\newcommand{\QED}{\; \square}
\newcommand{\EndProof}{& \QED \\}
\newcommand{\Proof}{\LOGIC{Proof} \; }
\newcommand{\Explain}[1]{& \text{#1.} \\}
\newcommand{\ExplainFurther}[1]{& \text{#1} \\}
\newcommand{\Exclaim}[1]{& \text{#1!} \\}
%SetTheory
\newcommand{\NonEmpty}{\TYPE{NonEmpty}}
\newcommand{\Finite}{\TYPE{Finite}}
\newcommand{\Countable}{\TYPE{Countable}}
\newcommand{\Uncountable}{\TYPE{Uncountable}}
\newcommand{\Ideal}{\TYPE{Ideal}}
\newcommand{\Inj}{\TYPE{Injective}}
\newcommand{\Surj}{\TYPE{Surjective}}
\newcommand{\Bij}{\TYPE{Bijective}}
\newcommand{\SIdeal}{\TYPE{\sigma\hyph \Ideal}}
\newcommand{\SA}{\TYPE{\sigma \hyph Algebra}}
\newcommand{\Eq}{\TYPE{Equivalence}}
%CategoryTheory
%Types
\newcommand{\Cov}{\TYPE{Covariant}}
\newcommand{\Contra}{\TYPE{Contravariant}}
\newcommand{\NT}{\TYPE{NaturalTransform}}
\newcommand{\UMP}{\TYPE{UnversalMappingProperty}}
\newcommand{\CMP}{\TYPE{CouniversalMappingProperty}}
\newcommand{\paral}{\rightrightarrows}
%functions
\newcommand{\op}{\mathrm{op}}
\newcommand{\obj}{\mathrm{obj}}
\DeclareMathOperator*{\dom}{dom}
\DeclareMathOperator*{\codom}{codom}
\DeclareMathOperator*{\colim}{colim}
%variable
\renewcommand{\C}{\mathcal{C}}
\newcommand{\A}{\mathcal{A}}
\newcommand{\B}{\mathcal{B}}
\newcommand{\D}{\mathcal{D}}
\newcommand{\I}{\mathcal{I}}
\newcommand{\J}{\mathcal{J}}
\newcommand{\R}{\mathcal{R}}
%Cats
\newcommand{\CAT}{\mathsf{CAT}}
\newcommand{\SET}{\mathsf{SET}}
\newcommand{\PARALLEL}{\bullet \paral \bullet}
\newcommand{\WEDGE}{\bullet \to \bullet \leftarrow \bullet}
\newcommand{\VEE}{\bullet \leftarrow \bullet \to \bullet}
%OrderTheory
%Types
\newcommand{\Poset}{\TYPE{Poset}}
\newcommand{\Toset}{\TYPE{Toset}}
\newcommand{\Pres}{\TYPE{PreorderedSet}}
\newcommand{\WF}{\TYPE{WellFounded}}
\newcommand{\WO}{\TYPE{WellOrdered}}
\newcommand{\II}{\TYPE{InitialInterval}}
\newcommand{\UB}{\TYPE{UpperBound}}
\newcommand{\LUB}{\TYPE{LowerUpperBound}}
\newcommand{\LB}{\TYPE{LowerBound}}
\newcommand{\ULB}{\TYPE{UpperLoweBound}}
%Cats
\newcommand{\POSET}{\mathsf{POSET}}
\newcommand{\ORD}{\mathsf{ORD}}
%Symbols
\renewcommand{\P}{\mathsf{P}}
%\newcommand{\F}{\mathsf{F}}
%\newcommand{\U}{\mathsf{U}}
%Algebra
%Groups
%Types
\newcommand{\Group}{\TYPE{Group}}
\newcommand{\Abel}{\TYPE{Abelean}}
\newcommand{\Sgrp}{\subset_{\mathsf{GRP}}}
\newcommand{\Nrml}{\vartriangleleft}
\newcommand{\FG}{\TYPE{FiniteGroup}}
\newcommand{\Stab}{\mathrm{Stab}}
\newcommand{\FGA}{\TYPE{FinitelyGeneratedAbelean}}
\newcommand{\DN}{\TYPE{DirectedNormality}}
\newcommand{\ActsOn}{\curvearrowright}
%Func
\DeclareMathOperator{\tor}{tor}
\DeclareMathOperator{\ord}{ord}
\DeclareMathOperator{\bool}{bool}
\DeclareMathOperator{\rank}{rank}
%Cats
\newcommand{\GRP}{\mathsf{GRP}}
\newcommand{\ABEL}{\mathsf{ABEL}}
%Boolean Algebra
%TYPE
\newcommand{\Bool}{\mathbb{B}}
\newcommand{\Alg}{\TYPE{Algebra}}
\newcommand{\BR}{\TYPE{BooleanRing}}
\newcommand{\BA}{\TYPE{BooleanAlgebra}}
\newcommand{\PD}{\TYPE{PairwiseDisjointElements}}
\newcommand{\PoU}{\TYPE{PartitionOfUnity}}
\renewcommand{\SS}{\TYPE{StoneSpace}}
\newcommand{\TK}{\mathcal{TK}}
\newcommand{\BL}{\TYPE{BooleanLattice}}
\newcommand{\Fix}{\mathrm{Fix}}
\newcommand{\OC}{\TYPE{OrderClosed}}
\newcommand{\SOC}{\TYPE{SequentiallyOrderClosed}}
\newcommand{\oC}{\TYPE{OrderContinuous}}
\newcommand{\sC}{\TYPE{\sigma\hyph Continuous}}
\newcommand{\OD}{\TYPE{OrderDense}}
\newcommand{\REing}{\TYPE{RegularEmbedding}}
\newcommand{\REed}{\TYPE{RegularEmbeded}}
\newcommand{\REable}{\TYPE{RegularEmbedable}}
\newcommand{\OComplete}{\TYPE{OrderDedekindComplete}}
\newcommand{\TAlgebra}{\TYPE{\tau\hyph Algebra}}
\newcommand{\OCompletes}{\TYPE{OrderDedekindCompleteSubset}}
\newcommand{\SComplete}{\TYPE{\sigma\hyph DedekindComplete}}
\newcommand{\SCompletes}{\TYPE{\sigma\hyph DedekindCompleteSubset}}
\newcommand{\LS}{\mathcal{LS}}
\newcommand{\POpen}{\TYPE{PseudoOpen}}
\newcommand{\od}{\mathbf{OD}}
\newcommand{\mgr}{\mathbf{MGR}}
\newcommand{\nd}{\mathbf{ND}}
\newcommand{\CCC}{\TYPE{WithCountableChainCondition}}
\newcommand{\CSI}{\TYPE{\omega_1\hyph SaturatedIdeal}}
\newcommand{\WD}{\TYPE{(\sigma,\infty)\hyph WeaklyDistributive}}
\newcommand{\Aless}{\TYPE{Atomless}}
\newcommand{\PA}{\TYPE{PurelyAtomic}}
\newcommand{\Homog}{\TYPE{Homogeneous}}
%\newcommand{\FS}{\TYPE{FullSubgroup}}
%\newcommand{\CFS}{\TYPE{CountablyFullSubgroup}}
%\newcommand{\EI}{\TYPE{ExchangingInvolution}}
%\newcommand{\SwS}{\TYPE{SubgroupWithSeparators}}
%\newcommand{\SwmI}{\TYPE{SubgroupWithManyInvolutions}}
%FUNC
\DeclareMathOperator{\upr}{upr}
\DeclareMathOperator{\Atom}{Atom}
%\DeclareMathOperator{\Supp}{Supp}
%\newcommand{\genFS}[1]{\left\langle #1 \right\rangle_\mathrm{F}}
%\newcommand{\genCFS}[1]{\left\langle #1 \right\rangle_\mathrm{CF}}
%\DeclareMathOperator{\Sep}{Sep}
%\DeclareMathOperator{\Tr}{Tr}
%CATS
\newcommand{\BOL}{\mathsf{BOL}}
\newcommand{\BOOL}{\mathsf{BOOL}}
%SYMBOL
\newcommand{\Z}{\mathsf{Z}}
%Descriptive Set Theory
%TYPE
%\newcommand{\Bool}{\mathbb{B}}
\newcommand{\IS}{\TYPE{InitialSegement}}
\newcommand{\FS}[1]{{#1}{}^*}
\newcommand{\Ext}{\TYPE{Extension}}
\newcommand{\Tree}{\TYPE{Tree}}
\newcommand{\Pruned}{\TYPE{Pruned}}
\newcommand{\PTM}{\TYPE{ProperTreeMorphism}}
\newcommand{\LTM}{\TYPE{LipschitzTreeMorphism}}
\newcommand{\Polish}{\TYPE{Polish}}
\newcommand{\IIPG}{\TYPE{InfiniteIterativeTwoPlayersGame}}
\newcommand{\FPS}{\TYPE{FirstPlayerStrategy}}
\newcommand{\SPS}{\TYPE{SecondPlayerStrategy}}
\newcommand{\FPWS}{\TYPE{FirstPlayerWinningStrategy}}
\newcommand{\SPWS}{\TYPE{SecondPlayerWinningStrategy}}
\newcommand{\CS}{\TYPE{ChoquetSpace}}
\newcommand{\SCS}{\TYPE{StrongChoquetSpace}}
\newcommand{\BP}{\mathbf{BP}}
\newcommand{\MGR}{\mathbf{MGR}}
\newcommand{\cat}{\mathbf{CAT}}
\newcommand{\BM}{\TYPE{BairMeasurable}}
\newcommand{\CGSA}{\TYPE{CountablyGeneratedSigmaAlgebra}}
\newcommand{\MC}{\TYPE{MonotonicClass}}
\newcommand{\PSA}{\TYPE{PointSeparatingAlgebra}}
\newcommand{\SBS}{\TYPE{StandardBorelSpace}}
\newcommand{\IH}{\TYPE{InducedHomomorphism}}
%FUNC
\DeclareMathOperator{\len}{len}
\newcommand{\inits}[2]{{#1}_{|\left[1,\ldots,#2\right]}}
\DeclareMathOperator{\lb}{lb}
\DeclareMathOperator{\WFpart}{WF}
\DeclareMathOperator{\Tr}{Tr}
\DeclareMathOperator{\PTr}{PTr}
\DeclareMathOperator*{\Tll}{{T\;\underline{lim}}}
\DeclareMathOperator*{\Tul}{{T\;\overline{lim}}}
\DeclareMathOperator*{\Tl}{{T\;lim}}
\DeclareMathOperator{\rankcb}{rank_{CB}}
\DeclareMathOperator{\lp}{lp}
\newcommand{\alg}{\mathsf{A}}
%CATS
\newcommand{\TREE}{\mathsf{TREE}}
\newcommand{\FSF}{\mathsf{FS}}
\newcommand{\CRONE}{\mathsf{CRONE}}
\newcommand{\BODY}{\mathsf{BODY}}
\newcommand{\BOR}{\mathsf{BOR}}
\newcommand{\bor}{\mathsf{B}}
\newcommand{\Effros}{\mathsf{EFF}}
%symbols
\newcommand{\K}{\mathsf{K}}
\renewcommand{\H}{\mathrm{H}}
\renewcommand{\L}{\mathcal{L}}
\renewcommand{\P}{\mathcal{P}}
\renewcommand{\S}{\mathcal{S}}
%LINEAR
%Linear Algebra
%Types
\newcommand{\Basis}{\TYPE{Basis}} % Basis of the linear space
\newcommand{\submod}[1]{\subset_{\LMOD{#1}}}% submodule as a subset
\newcommand{\subvec}[1]{\subset_{\VS{#1}}}% vector subspace as a subset
\newcommand{\FGM}{\TYPE{FinitelyGeneratedModule}}% Finitely generated module
\newcommand{\LI}{\TYPE{LinearlyIndependent}}
\newcommand{\LIS}{\TYPE{LinearlyIndependentSet}}
\newcommand{\FM}{\TYPE{FreeModule}}
\newcommand{\IBP}{\TYPE{InvariantBasisProperty}}
\newcommand{\UTM}{\TYPE{UpperTriangularMatrix}}
%\newcommand{\LTM}{\TYPE{LowerTriangularMatrix}}
\newcommand{\Diag}{\TYPE{DiagonalMatrix}}
\newcommand{\FP }{\TYPE{FinitelyPresented}}
\newcommand{\GL}{\mathbf{GL}}% General Linear Group
\newcommand{\SL}{\mathbf{SL}}% Special Linear group
\newcommand{\SO}{\mathbf{SO}}
\newcommand{\SU}{\mathbf{SU}}
\newcommand{\prsubvec}[1]{\subsetneq_{\VS{#1}}}	% poper vector subspace as a subset
\newcommand{\LC}{\TYPE{LinearComplement}} 
%\newcommand{\IS}{\TYPE{InvariantSubspace}}
\newcommand{\RP}{\TYPE{ReducingPair}}
\newcommand{\RCF}{\TYPE{RationalCanonicalForm}}
\newcommand{\JCF}{\TYPE{JordanCanonicalForm}}
\newcommand{\Diagble}{\TYPE{Diagonalizable}}
\newcommand{\UT}{\TYPE{UpperTriangulizable}}
\newcommand{\LT}{\TYPE{LowerTriangulizable}}
\newcommand{\IPS}{\TYPE{InnerProductSpace}}
\newcommand{\OBasis}{\TYPE{OrthonormalBasis}}
\newcommand{\FDIPS}{\TYPE{FiniteDimensionalInnerProductSpace}}
\newcommand{\NO}{\TYPE{NormalOperator}}
\newcommand{\NM}{\TYPE{NormalMatrix}}
%\newcommand{\SA}{\TYPE{SelfAdjoint}}
\newcommand{\SSA}{\TYPE{SkewSelfAdjoint}}
\newcommand{\PI}{\TYPE{Pseudoinverse}}
\newcommand{\OVS}{\TYPE{OrthogonalVectorSpace}}
\newcommand{\SVS}{\TYPE{SymplecticVectorSpace}}
\newcommand{\MVS}{\TYPE{MetricVectorSpace}}
\newcommand{\FDMVS}{\TYPE{FiniteDimensionalMetricVectorSpace}}
\newcommand{\Sp}{\mathbf{Sp}}
%Func
\DeclareMathOperator{\Span}{span} % spann by subset
\DeclareMathOperator{\Ann}{Ann}   % annihilator
\DeclareMathOperator{\Ass}{Ass}   % associated primes:
\DeclareMathOperator{\adj}{adj}   % an adjoint matrix
\DeclareMathOperator{\tr}{tr}     % trace
\DeclareMathOperator{\codim}{codim} % codimension
\DeclareMathOperator{\Cell}{\mathbf{C}} % a componion matrix
\DeclareMathOperator{\JC}{\mathbf{J}}  % a Jordan cell
\DeclareMathOperator{\bigboxplus}{\scalerel*{\boxplus}{\sum}} % a direct sum of operators in the sence of the reducing a pair
\DeclareMathOperator{\Spec}{Spec} % Spectre
\DeclareMathOperator{\bigbot}{\scalerel*{\bot}{\sum}} % an othogonal direct sum
\DeclareMathOperator{\GS}{\mathbf{GS}} %Gramm-Smmidt process
\DeclareMathOperator{\NGS}{\mathbf{NGS}} %Normalized Gramm-Smmidt process
\DeclareMathOperator{\WI}{\mathrm{WI}} %Witt Index
%Cats
\newcommand{\VS}[1]{#1\hyph\mathsf{VS}} % a category of vector spaces (Field)
\newcommand{\FDVS}[1]{#1\hyph\mathsf{FDVS}} % a category of finite-dimensional vector spaces (Field)
\newcommand{\LALGE}[1]{#1\hyph\mathsf{ALGE}}
\newcommand{\LMOD}[1]{#1\hyph\mathsf{MOD}} % a category of the left modules (Ring)
\newcommand{\RMOD}[1]{\mathsf{MOD}\hyph#1} % a category of the right modules (Ring)
\newcommand{\LLMAP}[1]{#1\hyph\mathsf{LMAP}} % a cagory of based linear maps with the left scalar multiplication (Ring)
\newcommand{\LMAT}[1]{#1\hyph\mathsf{MAT}}  % a category of based matrices with the left scalar multiplication (Ring)
\newcommand{\NMAT}[1]{#1\hyph\mathbb{N}} % a category of finite matrices (Field)
%Symbols
\renewcommand{\L}{\mathcal{L}}
%\renewcommand{\O}{\mathbf{O}}
%\renewcommand{\S}{\mathbf{S}}
%%Measure theorty
%Types
\newcommand{\Measure}{\TYPE{Measure}}
\newcommand{\MS}{\TYPE{MeasureSpace}}
\newcommand{\CMS}{\TYPE{CompleteMeasureSpace}}
\newcommand{\Null}{\mathcal{N}}
\renewcommand{\ae}{\mathrm{a.e.}}
\newcommand{\OM}{\TYPE{OuterMeasure}}
\newcommand{\IM}{\TYPE{InnerMeasure}}
\newcommand{\Thick}{\TYPE{Thick}}
\newcommand{\Integrable}{\mathsf{I}}
\newcommand{\ME}{\TYPE{MeasurableEnvelope}}
\newcommand{\Probability}{\TYPE{Probability}}
\newcommand{\sFinite}{\TYPE{\sigma \hyph  Finite}}
\newcommand{\Semifinite}{\TYPE{Semifinite}}
\newcommand{\Decomposition}{\TYPE{Decomposition}}
\newcommand{\SLoc}{\TYPE{StrictlyLocalizable}}
\newcommand{\Loc}{\TYPE{Localizable}}
\newcommand{\LocDet}{\TYPE{LocallyDetermined}}
\newcommand{\PtSupp}{\TYPE{PointSupported}}
\newcommand{\SF}[1]{\TYPE{\sigma \hyph  Finite}\left( #1 \right) }
\newcommand{\DRP}{\TYPE{DiscreteRandomProcces}}
\newcommand{\MwLDNS}{\TYPE{MeasureWithLocallyDeterminedNullSets}}
\newcommand{\AF}{\TYPE{AdditiveFunctional}} 
\newcommand{\CAF}{\TYPE{CountablyAdditiveFunctional}}
\newcommand{\TC}{\TYPE{TrulyContinuous}} 
\newcommand{\CE}{\TYPE{ConditionalExpectation}}
%Functions and Operators
\DeclareMathOperator{\esssup}{ess\sup}
%Symbols
\newcommand{\F}{\mathcal{F}}
\newcommand{\E}{\mathcal{E}}
%CATS
\newcommand{\MEAS}{\mathsf{MEAS}}
\newcommand{\Simple}{\mathsf{S}}
\newcommand{\caf}{\mathsf{ca}}
\newcommand{\af}{\mathsf{a}}
\newcommand{\baf}{\mathsf{ba}}
\newcommand{\ac}{\mathsf{ac}}
\newcommand{\tc}{\mathsf{tc}}
\author{Uncultured Tramp} 
\title{Abstract Measure Theory}
\begin{document}
\maketitle
\thispagestyle{empty}
\newpage
\thispagestyle{empty}
\tableofcontents
\newpage
\thispagestyle{empty}
\section*{Intro}
This memoire is supposed to cover purely abststact topic in measure theory.
By purely abstract I undestand complete lack of assumptions about topology
or metric structure of underlying measurable space. So, it could have been 
to talk measurable sets equipped with measures, if such lingo was not 
over confusing.

Th may need for this memoire is the need to put basic definitions of 
measure theory somewhere. And it was not desirable to invoke
any associations to topology, algebra or geometry. 
So, first and third part of this treatise cover pretty standard results,
while the second part is about somethat exotic notions 

Another reason for this memo to exis, as I already got a memo
on basic measure theory, was the desire to untangle the
abstract part of the theory and the construction of the 
Lebesgue measure. This construction, in my opinion,
has undoubtful geometric merit, as it explixitely uses
intuitions provided by affine geometry of real line and plane.
So this topics concerning the constructuon of the Lebesgue
and Hausdorff measure will be put in folder of Real Analysis.
Nevertheless,  topics  of the current ducument also belong to the field of Analysis.
Their relations with analysis comes from 
1) Use of sigma-algebras which makes this discussion allready related to basic boolean structures,
eve without assumin their Borel, and hence topological nature
2) Use of infinite real serieal, which are covered in the Analysis on the Real Line memo.
By the way, these two topics are essential prerequisites here.  
Although, we assume here measures to be real-valued, we do not make any assumptions about their domains. So, this seems to be strong enough foundations to separate this memo from the real anaysis directory.
\newpage
\pagenumbering{arabic}
\section{Classical Theory}
\subsection{Measures}
\subsubsection{Subject}
\Page{
	\DeclareType{\Measure}{\prod X \in \BOR \. ?\Big(X \to {\EReals}_+\Big)}
	\DefineType{\mu}{\Measure}
	{
		\mu(\emptyset) = 0
		\And		
		\forall A : \TYPE{DisjointSequence}(\alg \; X) \. 
		\mu\left( \bigcup^\infty_{n=1} A \right) = \sum^\infty_{n=1} \mu(A_n) 
	}
	\\
	\Conclude{\MS}{\sum X \in \BOR \. \Measure(X)}{\Type}
	\\
	\DeclareFunc{measureFromFunction}
	{\prod X \in \SET \. \Big(X \to \EReals_+\Big) \to \Measure(X)}
	\DefineNamedFunc{measureFromFunction}{f}{\mu_f}
	{
		\Lambda A \subset X \. \sup \left\{ \sum^n_{i=1} f(a_i) \Bigg| n \in \Nat, 
		a :\{1,\ldots,n\} \to A  \right\}	
	}
	\\
	\DeclareFunc{measureOfDirac}
	{\prod X \in \SET \. \Big(X \to \EReals_+\Big) \to \Measure(X)}
	\DefineNamedFunc{measureOfDirac}{x}{\delta_x}
	{
		\Lambda A \subset X \. \If A(x) \Then 1 \Else 0	
	}
	\\
	\DeclareFunc{countingMeasure}
	{\prod X \in \SET \.  \Measure(X)}
	\DefineNamedFunc{countingMeasure}{A}{\# A}
	{
		\If |A| < \infty \Then  	|A| \Else +\infty
	}
	\\
	\Theorem{DisjointPairAdditivity}
	{
		\NewLine :		
		\forall (X,\Sigma,\mu) : \MS \.
		\forall (A,B) : \TYPE{DisjointPair}(X,\Sigma) \.
		\mu(A \cup B) = \mu(A) + \mu(B)
	}
	\Say{C}{(A,B,\emptyset,\ldots,\emptyset,\ldots)}{\Nat \to \Sigma}
	\Say{[1]}{\THM{EmptySetIntersection}(X)\Elim C}{\TYPE{DisjointSequence}(X,\Sigma,C)}
	\Say{[2]}{ \THM{UnionIteration}(X) \THM{EmptySetUnion}(X)}
	{
		\bigcup^\infty_{n=1} C_n = 
		A \cup B \cup \bigcup^\infty_{n=1} \emptyset = 
		A \cup B
	}
	\Conclude{[*]}{
		[2][1] 
		\Elim_2 \Measure(X,\Sigma,\mu) 
		\THM{SumIteration} \Elim C
		\THM{ZeroSum}
	}
	{
		\NewLine :		
		\mu(A \cup B) =
		\mu\left( \bigcup^\infty_{n=1} C_n \right) =
		\sum^\infty_{n=1} \mu(C_n) =
		\mu(A) + \mu(B) + \sum^\infty_{n=1} \mu(\emptyset) = 
		\mu(A) + \mu(B) + \sum^\infty_{n=1} 0 =
		\mu(A) + \mu(B)
	}
	\EndProof
}
\Page{
	\Theorem{Monotonicity}
	{
		\NewLine :		
		\forall (X,\Sigma,\mu) : \MS \.
		\forall A,B \in \Sigma \.
		\forall [0] : A \subset B \.
		\forall \mu(A) \le \mu(B)
	}
	\Say{[1]}{\THM{tDisjointPairByComplement}\Big(X,A,B\Big)}
	{
		\TYPE{DisjointPair}\Big( X, A, B \setminus A \Big)
	}
	\Say{[2]}{\THM{SubbsetComplement}\Big(X,A,B,[0]\Big)}
	{
		B = A \cup (B \setminus A)
	}
	\Conclude{[*]}{[2]\THM{DisjointPairAdditivity}[1]\THM{NonDecreasingAddition}\Big( \EReals_+\Big)}
	{
		\NewLine :		
		\mu(B) =
		\mu\Big( A \cup (B \setminus A) \Big) =
		\mu(A) + \mu(B \setminus A) \ge	
		\mu(A) 
	}
	\EndProof
	\\
	\Theorem{PairSubadditivity}
	{
		\NewLine :		
		\forall (X,\Sigma,\mu) : \MS \.
		\forall A,B \in \Sigma \.
		\mu(A \cup B) \le \mu(A) + \mu(B)
	}
	\Say{[1]}{\THM{UnionSymmetricDecomposition}(X,A,B)}
	{
		A \cup B = (A \setminus B) \sqcup (A \cap B) \sqcup (B \setminus A)
	}
	\Say{[2]}{\THM{SetDecomposition}(X,A,B)}
	{
		A  = (A \setminus B) \sqcup (A \cap B)
	}
	\Say{[3]}{\THM{UnionSymmetricDecomposition}(X,B,A)}
	{
		B = (A \cap B) \sqcup (B \setminus A)
	}
	\Conclude{[*]}{
		\THM{DisjointPairAdditivity}(X,\Sigma,\mu)[1]	
		\THM{NonDecreasingAddition}\Big( \EReals_+\Big) \NewLine
		\THM{DisjointPairAdditivity}(X,\Sigma,\mu)[2,3]
	}
	{
		\NewLine :		
		\mu(A \cup B ) =
		\mu(A \setminus B) + \mu(A \cap B) + \mu(B \setminus A) \le 
		\mu(A \setminus B) + 2\mu(A \cap B) + \mu(B \setminus A) = 
		\mu(A) + \mu(B)
	}
	\EndProof
	\\
	\Theorem{Subadditivity}
	{
		\NewLine :		
		\forall (X,\Sigma,\mu) : \MS \.
		\forall A : \Nat \to \Sigma \.
		\mu\left( \bigcup^\infty_{n=1} A_n \right) \le \sum^\infty_{n=1} \mu(A_n)
	}
	\Say{B}{\Lambda n \in \Nat \. A_n \setminus \bigcup^{n-1}_{k=1} A_k }
	{
		\Nat \to \Sigma
	}
	\Say{[1]}{\THM{DisjoinedUnion}(X,A)\Intro B}
	{
		\bigcup^\infty_{n=1} A_n = \bigcup^\infty_{n=1} B_n
	}
	\Say{[2]}{\THM{ComplementIntersection}(X)\Intro B}
	{
		\TYPE{DisjointSequence}(X, \Sigma, B)
	}
	\Say{[3]}{ \Lambda n \in \Nat \. \Elim B_n \THM{DifferenceIsSubset}(X)}
	{
		\forall n \in \Nat \. B_n \subset A_n
	}
	\Say{[4]}{ \THM{Monotonicity}(X,\Sigma,\mu) [3]}
	{
		\forall n \in \Nat \. \mu(B_n) \le \mu(A_n)	
	}
	\Conclude{[*]}{ [1]\Elim \Measure(X,\Sigma,\mu)[2][4]}
	{
		\mu\left( \bigcup^\infty_{n=1} A_n \right) =
		\mu\left( \bigcup^\infty_{n=1} B_n \right) =
		\sum^\infty_{n=1} \mu(B_n) \le
		\sum^\infty_{n=1} \mu(A_n)
	}
	\EndProof
	\\
	\Theorem{Difference}
	{
		\NewLine :		
		\forall (X,\Sigma,\mu) : \MS \.
		\forall A,B \in \Sigma \.
		\forall [01] : A \subset B \.
		\forall [02] : \mu(A) < \infty \.
		\mu(B\setminus A) = \mu(B) - \mu(A)
	}
	\Say{[1]}{\THM{tDisjointPairByComplement}\Big(X,A,B\Big)}
	{
		\TYPE{DisjointPair}\Big( X, A, B \setminus A \Big)
	}
	\Say{[2]}{\THM{SubbsetComplement}\Big(X,A,B,[0]\Big)}
	{
		B = A \cup (B \setminus A)
	}
	\Say{[3]}{\THM{DisjointPairAdditivity}(X,\Sigma,\mu,A,(B \setminus A))[1][2]}
	{
		\mu(B) = \mu(A) + \mu(B \setminus A)
	}
	\Conclude{[*]}{[3] - \mu(A)}
	{
		\mu(B \setminus A) = \mu(B) - \mu(A)
	}
	\EndProof
}\Page{
	\Theorem{LowerContinuity}
	{
		\forall (X,\Sigma,\mu) : \MS \.
		\forall A : \Nat \uparrow \Sigma \.
		\mu\left( \bigcup^\infty_{n=1} A_n \right) = \lim_{n \to \infty} \mu(A_n)
	}
	\Say{B}{\Lambda n \in \Nat \. A_n \setminus \bigcup^{n-1}_{k=1} A_k }
	{
		\Nat \to \Sigma
	}
	\Say{[1]}{\Lambda n \in \Nat \. \THM{DisjoinedUnion}(X,A|n)\Intro B}
	{
		\forall n \in \Nat		
		\bigcup^\infty_{n=1} A_n = \bigcup^\infty_{n=1} B_n
	}
	\Say{[2]}{\THM{ComplementIntersection}(X)\Intro B}
	{
		\TYPE{DisjointSequence}(X, \Sigma, B)
	}
	\Say{[3]}{ 
		\Lambda n \in \Nat \.
		\THM{MonotonicNondecreasingUnion}(X,n,A|n)
		[1](n)
		\THM{DisjointPairAdditivity}^{n-1}[2]	
	}
	{
		\NewLine :		
		\forall n \in \Nat \. 
		\mu(A_n) = 
		\mu\left(  \bigcup^n_{i=1} A_i \right) = 
		\mu\left(  \bigcup^n_{i=1} B_i \right) =
		\sum^n_{i=1} \mu(B_i)
	}
	\Say{[4]}{\THM{DisjoinedUnion}(X,A)\Intro B}
	{
		\bigcup^\infty_{n=1} A_n = \bigcup^\infty_{n=1} B_n
	}
	\Conclude{[*]}{
		[4]
		\Elim \Measure(X,\Sigma,\mu)[2]
		\Elim \TYPE{SeriesLimit}
		[3] 
	}
	{
		\NewLine :		
		\mu\left(\bigcup^\infty_{n=1} A_n\right) = 
		\mu\left(\bigcup^\infty_{n=1} B_n\right) =
		\sum^\infty_{n=1} \mu(B_n) =
		\lim_{n \to \infty} \sum^n_{i=1} \mu(B_i) =
		\lim_{n \to \infty} \mu(A_n)
	}
	\EndProof
	\\
	\Theorem{UpperContinuity}
	{
		\forall (X,\Sigma,\mu) : \MS \.
		\forall A : \Nat \downarrow \Sigma \.
		\mu(A_1) < \infty \Imply
		\mu\left( \bigcap^\infty_{n=1} A_n \right) = \lim_{n \to \infty} \mu(A_n)
	}
	\Conclude{B}{\Lambda n \in \Nat \. A_1 \setminus A_n}{\Nat \uparrow \Sigma}
	\Say{[1]}{
		\THM{Difference}\left(X,\Sigma,\mu, A_1, \bigcap^\infty_{n=1} A_n \right)	
		\Intro B	
		\THM{LowerContinuity}(X,\Sigma,\mu,B)
		\Elim B
		\NewLine
		\Lambda n \in \Nat \. 
		\THM{Difference}\left(X,\Sigma,\mu, A_1,  A_n \right)
		\THM{LimitSum}\Big( 
			\Lambda n \in \Nat \. \mu(A_1),
			\Lambda n \in \Nat \. -\mu(A_n)
		\Big)	\NewLine
		\THM{ConstantLimit}\Big( 
			\Reals,
			\mu(A_1)
		\Big) 
	}
	{
		\NewLine :	 	
	 	\mu(A_1) - \mu\left(	\bigcap^\infty_{n=1} A_n \right) =
	 	\mu\left( A_1 \setminus \bigcap^\infty_{n=1} A_n  \right) =
		\mu\left( \bigcup^\infty_{n=1} B_n \right) =
		\lim_{n \to \infty} \mu(B_n) =
		\lim_{n \to \infty} \mu(A_1) - \mu(A_n) = \NewLine =
		\mu(A_1) - \lim_{n \to \infty} \mu(A_n)	
	}
	\Conclude{[*]}{ \mu(A_1) -[1]}
	{
		\mu\left(\bigcap^\infty_{n=1} A_n \right) = \lim_{n \to \infty} \mu(A_n)
	}
	\EndProof 
}\Page{
	\Theorem{DeMoivreFormula}
	{	
		\forall (X,\Sigma,\mu) : \MS\.
		\forall n \in \Int_+ \.
		\forall A : \{1,\ldots,n\} \to \Sigma \. \NewLine
		\mu\left( \bigcup^n_{i=1} A_i \right)
		+ 
			\sum^{\lfloor n/2 \rfloor}_{k=1} 
			\sum_{I \subset \{1,\ldots,n\},|I|=2k} 
			 \mu\left( \bigcap_{i \in I} A_i \right)
		=
		\sum^{\lfloor n/2 \rfloor}_{k=0} 
			\sum_{I \subset \{1,\ldots,n\},|I|=2k + 1} 
		\mu\left( \bigcap_{i \in I} A_i \right)	
	}
	\Explain{We will prove this in the form
		$
			\mu\left(\bigcup^n_{i=1} A_i\right)=
			\sum^n_{k=1} 
			\sum_{I \subset \{1,\ldots,n\},|I|=k} (-1)^{k+1} 
			\mu\left( \bigcap_{i \in I} A_i \right)
		$
	}
	\Explain{ Clearly, in  case $n=0$ we have relation $\mu(\emptyset) = 0$, which is true }
	\Explain{ Clearly, in  case $n=1$ we have relation $\mu(A_1) = \mu(A_1)$, 
		which is also obvious }
	\Explain{ Use this as the basis for induction}
	\Explain{
		Clearly from iterating disjoint additivity of measure it follows that}
	\Explain{ 
		$\forall A,B \in \Sigma \. \mu(A \cup B) = \mu(A) + \mu(B) - \mu(A \cap B)$}	
	\Explain{ We will use this for induction step}
	\Explain{ Now assume the statement hold for $n=1,\ldots,m$}
	\Explain{ Let $A_1,\ldots,A_{m+1}$ be measurable}
	\Explain{ Masqurade them as $B_i = A_i$ for $i<m$ and $B_m  = A_m \cup A_{m+1}$}
	\Explain{ Then, by hypothesis 
		$
			\mu\left(\bigcup^m_{i=1} B_i\right)=
			\sum^m_{k=1} 
			\sum_{I \subset \{1,\ldots,m\},|I|=k} (-1)^{k+1} 
			\mu\left( \bigcap_{i \in I} B_i \right)
		$	
	}
	\Explain{ Here the left part clearly corresponds to $\mu\left(\bigcup^m_{i=1} A_i\right)$}
	\Explain{ On the other hand, summands in the right part which don't depend on $B_m$ 
		will stand the same}
	\Explain{ 
		And ones which depend, by associativity of basic boolean operation
		will turn into}
	\Explain{
		$
		 \mu\left(\bigcap_{i\in I \setminus \{m\}} A_i \cap (A_m \cup A_{m+1}) \right) = 
		 \mu\left(\bigcap_{i\in I \setminus \{m\}} A_i \cap A_m  \cup 
		 	 \bigcap_{i\in I \setminus \{m\}} A_i \cap A_{m+1} \right) =
		$	
	}
	\Explain{
		$
		 =
		 \mu\left(\bigcap_{i\in I \setminus \{m\}} A_i \cap A_m  \right)
		 +
		 \mu\left(\bigcap_{i\in I \setminus \{m\}} A_i \cap A_{m+1}  \right)
		 -
		 \mu\left(\bigcap_{i\in I} A_i \cap A_{m+1}  \right)
		$
	}
	\Explain{ This kind of transformation will produce all possible subsets of $\{1\ldots,m+1\}$}
	\Explain{  With signs correctly corresponding to parities}
	\Explain{  So, it holds that 
		$
			\mu\left(\bigcup^{m+1}_{i=1} A_n\right)=
			\sum^{m+1}_{k=1} 
			\sum_{I \subset \{1,\ldots,n\},|I|=k} (-1)^{k+1} 
			\mu\left( \bigcap_{i \in I} A_i \right)
		$
	}
	\EndProof
}\Page{
	\Theorem{LimInfBound}
	{
		\forall (X,\Sigma,\mu) : \MS \.
		\forall  A : \Nat \to \Sigma \.
		\mu\left( \bigcup^\infty_{n=1} \bigcap^\infty_{m=n} A_m\right) \le 
		\lim \inf_{n \in \Nat} \mu(A_n)
	}
	\Explain{ Note, that the sequens $B_n = \bigcap^\infty_{m=n} A_m$ is increasing}
	\Explain{ So, by lower continuity 
		$
			\mu\left( \bigcup^\infty_{n=1} B_n \right)
			= \lim_{n \to \infty } \mu(B_n)
		$
	}
	\Explain{ But $\mu(B_n) \le \mu(A_m)$ for any $m \ge n$ by measure monotonicity}
	\Explain{ So, $\mu(B_n) \le \inf \Big\{ \mu(A_n), \mu(A_{n+1}), \ldots   \Big\}$}
	\Explain{Thus, $  
		\mu\left( \bigcup^\infty_{n=1} B_n \right) 
		\le \lim_{n \to \infty} \inf \Big\{ \mu(A_n), \mu(A_{n+1}), \ldots   \Big\}$
		by limiting inequality}
	\Explain{
		But this is exactly the same as
		$
			\mu\left( \bigcup^\infty_{n=1} \bigcap^\infty_{m=n} A_m\right) \le 
			\lim \inf_{n \in \Nat} \mu(A_n)
		$
	}
	\EndProof
	\\
	\Theorem{SymmetricDifferenceExpression}
	{
		\NewLine :		
		\forall (X,\Sigma,\mu) : \MS \.
		\forall A,B \in \Sigma \.
		\forall \mu(A) < \infty \.
		\mu(A \du B) =  \mu(B) - \mu(A) + 2 \mu(A \setminus B)
	}
	\Explain{ Write
		$A \du B = (A \setminus B) \sqcup (B \setminus A)$}
	\Explain{
		So,
		$
			\mu(A \du B) =
			\mu(A \setminus B)
			+
			\mu(B \setminus A)
		$}
	\Explain{ Note that $B \setminus A = B \setminus (A \cap B)$ }
	\Explain{
		So, by difference formula
		$
			\mu(A \du B) = 
			\mu(B) - \mu(A \cap B) + \mu(A \setminus B)
		$
	}
	\Explain{ Now view $A \cap B = A \setminus (A \setminus B)$}
	\Explain{ Then, by difference law 
		$  \mu(A \du B) = \mu(B) - \mu(A) + 2\mu(A \setminus B)$}
	\EndProof
	\\
	\Theorem{LimSupBound}{
			\forall (X,\Sigma,\mu) : \MS \.
		\forall  A : \Nat \to \Sigma \.
		\forall [0] : \mu\left( \bigcup^\infty_{n=1} A_m \right) < \infty
		\. \NewLine \.
		\mu\left( \bigcap^\infty_{n=1} \bigcup^\infty_{m=n} A_m\right) \ge 
		\lim \sup_{n \in \Nat} \mu(A_n)
	}
	\Explain{Dualize proof of lim inf bound}
	\EndProof
	\\
	\Theorem{LimSupLimInfEq}
	{
		\forall (X,\Sigma,\mu) : \MS \.
		\forall A : \Nat \to \Sigma \.
		\forall B \in \Sigma \. \NewLine \.
		\forall [01] : \mu\left( \bigcup^\infty_{n=1} A_m \right) < \infty \.
		\forall [02] : \bigcap^\infty_{n=1} \bigcup^\infty_{m=n} A_m = B =
		\bigcup^\infty_{n=1} \bigcap^\infty_{m=n} A_m \. \NewLine \.
		\lim \inf_{n \in \Nat} \mu(A_n) = \lim \sup_{n \in \Nat}\mu(A_n) = \mu(B)
	}
	\Explain{ Use lim sup and lim inf bounds to get}
	\Explain{
		$
			\mu(B) =
			\mu\left(\bigcap^\infty_{n=1} \bigcup^\infty_{m=n} A_m \right) \ge
			\lim \sup_{n \in \Nat} \mu(A_n) \ge
			\lim \inf_{n \in \Nat} \mu(A_n) \ge
			\mu\left(\bigcup^\infty_{n=1} \bigcap^\infty_{m=n} A_m \right) =
			\mu(B)
		$
	}
	\EndProof
}
\newpage
\subsubsection{Quantification}
\Page{
	\DeclareType{NullSet}{\prod (X,\Sigma,\mu) : \MS \. ??X}
	\DefineNamedType{Z}{NullSet}{Z \in \Null_\mu}
	{
		\exists A \in \Sigma \. \mu(A) = 0 \And Z \subset A	
	}
	\\
	\Theorem{EmptyIsNull}
	{
		\forall (X,\Sigma,\mu) : \MS \.
		\emptyset \in \Null_\mu
	}
	\Say{[1]}{\Elim_1 \Measure(X,\Sigma,\mu)}{\mu(\emptyset)=0}
	\Say{[2]}{\THM{SelfContainment}(X,\emptyset)}{\emptyset \subset \emptyset}
	\Conclude{[*]}{\Intro \Null_\mu [1][2]}{\emptyset \in \Null_\mu}
	\EndProof
	\\
	\Theorem{NullSubset}
	{
		\forall (X,\Sigma,\mu) : \MS \.
		\forall A \in \Null_\mu \.
		\forall B \subset A \.
		B \in \Null_\mu
	}
	\Say{\Big(Z,[1],[2]\Big)}{\Elim \Null_\mu(A)}{
		\sum Z \in \Sigma \.   \Big(\mu(Z) = 0 \Big)\times\Big( A \subset Z \Big)	
	}
	\Say{[3]}{\THM{TransitiveSubset}(X) \Elim B [2]}
	{
		B \subset Z
	}
	\Conclude{[*]}{\Elim \Null_\mu\Big([1], [3]\Big)}
	{
		B \in \Null_\mu
	}
	\EndProof
	\\
	\Theorem{NullSum}
	{
		\forall (X,\Sigma,\mu) : \MS \.
		\forall A : \Nat \to \Null_\mu \.
		\bigcup^\infty_{n=1} A_n \in \Null_\mu \.
	}
	\Say{\Big(Z,[1],[2]\Big)}{\Elim \Null_\mu(A)}{
		\sum Z \Nat \to \Sigma \.   \Big( \forall n \in \Nat \. \mu(Z_n) = 0 \Big)
		\times\Big( \forall n \in \Nat \. A_n \subset Z_n \Big)	
	}
	\Say{[3]}{\THM{UnionOfSubsets}\Big(X,A,Z,[4]\Big)}
	{
		\bigcup_{n=1} A_n \subset \bigcup_{n=1} Z_n
	}
	\Say{[4]}{\THM{Subbaditivty}(X,\Sigma,\mu,Z)[1]\THM{ZeroSum}}
	{
		\mu\left(\bigcup^\infty_{n=1} Z_n\right) \le	
		\sum^\infty_{n=1} \mu(Z_n) =
		\sum^\infty_{n=1} 0 =
		0	
	}
	\Say{[5]}{\THM{MinimaUpperBound}[4]}
	{
		\mu\left(\bigcup^\infty_{n=1} Z_n\right) = 0
	}
	\Conclude{[1.*]}{\Elim \Null_\mu\Big( [3], [4] \Big)}
	{
		\bigcup^\infty_{n=1} A_n \in \Null_\mu
	}
	\EndProof
	\\
	\Theorem{NullSetsAreSigmaIdeal}
	{
		\forall (X,\Sigma,\mu) : \MS \. 
		\SIdeal(\Sigma,\Null_\mu)
	}
	\Explain{By definition}
	\EndProof
}
\Page{
	\DeclareType{ConullSet}
	{
		\prod (X,\Sigma, \mu) : \MS	
	}
	\DefineNamedType{C}{ConullSet}{C \in \Null'_\mu}{C^\c \in \Null_\mu}
	\\
	\Theorem{UniversumIsConull}
	{
		\forall (X,\Sigma,\mu) : \MS \.
		X \in \Null_\mu' \.
	}
	\Explain{ By duallity}
	\EndProof
	\\
	\Theorem{ConullSuperset}
	{
		\forall (X,\Sigma,\mu) : \MS \.
		\forall A \in \Null_\mu' \.
		\forall A \subset B \.
		B \in \Null_\mu'
	}
	\Explain{ By duallity}
	\EndProof
	\\
	\Theorem{ConullProduct}
	{
		\forall (X,\Sigma,\mu) : \MS \.
		\forall A : \Nat \to \Null_\mu' \.
		\bigcap^\infty_{n=1} A_n \in \Null_\mu' \.
	}
	\Explain{ By duallity }
	\EndProof
	\\
	\DeclareFunc{almostEverywhere}
	{
		\prod (X,\Sigma,\mu) : \MS \. 
		?X \to \Type
	}
	\DefineNamedFunc{almostEverywhere}{P}
	{ \forall_\mu P = \forall_\mu x \in X \. P(x) =  P(x)\;\mu\hyph\ae\;(x)  }{P \in \Null'_\mu}
	\\
	\DeclareFunc{somewhere}
	{
		\prod (X,\Sigma,\mu) : \MS \. 
		?X \to \Type
	}
	\DefineNamedFunc{somewhere}{P}
	{ \exists_\mu P = \exists_\mu x \in X \. P(x) =  P(x)\;\mu\hyph\ae\;(x)  }{P \in \Null^\c_\mu}
	\\
	\DeclareFunc{almostDefinedFunctions}{\MS \to \Type}
	\DefineNamedFunc{almostDefinedFunctions}{X,\Sigma,\mu}
	{\F_\mu}
	{
		\sum A \in \Null'_\mu \. A \to \Reals
	}
	\\
	\DeclareType{GEAlmostEverywhere}{\prod (X,\Sigma,\mu) : \MS \. ?(\F_mu^2)}
	\DefineNamedType{(f,g)}{GEAlmostEverywhere}{f \ge_{\ae} g}
	{
		\exists A \in \Null'_\mu \. 
		A \subset \dom(f) \cap \dom(g)  \And
		\forall a \in A \. f(a) \ge g(a)
	}
	\\
	\Theorem{GeAlmostEverywhereIsPreorder}
	{
		\forall (X,\Sigma,\mu) : \MS \.
		\TYPE{Preorder}\Big( \F_\mu, \ge_\mu \Big)
	}
	\Explain{ To get reflexivity use $A = \dom f$  and use reflicibity of order in $\Reals$}
	\Explain{ Let $f,g,h \in \F_\mu$ such that $f \geq_\ae g $ and $g \geq_\ae h$ }
	\Explain{ Then, there are sets $A,B \in \Null'_\mu$ such that first inequality holds on $A$
		and second on $B$}
	\Explain{ Now, $C = A \cap B \in \Null'_\mu$ and both inequalities hold on $C$ }
	\Explain{ So using transitivty of order on $\Reals$ we get $f \geq_\ae h$}
	\EndProof
	\\
	\DeclareType{EqAlmostEverywhere}{\prod (X,\Sigma,\mu) : \MS \. \Eq(\F_\mu)}
	\DefineNamedType{(f,g)}{EqAlmostEverywhere}{f  =_{\ae} g}
	{
		\exists A \in \Null'_\mu \. 
		A \subset \dom(f) \cap \dom(g)  \And
		\forall a \in A \. f(a) = g(a)
	}
}
\Page{
	\DeclareType{\CMS}{?\MS}
	\DefineType{(X,\Sigma,\mu)}{\CMS}{\Null_\mu \subset \Sigma}
	\\
	\Theorem{ConullAreFilter}
	{
		\forall (X,\Sigma,\mu) \.
		\forall \aleph : \mu > 0 \.
		\TYPE{Filter}(X,\Null'_\mu,)
	}
	\Explain{
		 $X \in \Null'_\mu$ by the fact that $\mu(\emptyset) = 0$, so $\exists\Null'_\mu$
	}
	\Explain{ 
		$\emptyset \not \in \Null'_\mu$  as $\mu(\emptyset) = 0 $ an $\aleph$}
	\Explain{
		If $A,B \in \Null'_\mu$, then so is $A \cap B$}
	\Explain{Also by monotonicity if $A \in \Null'_\mu$ and $A \subset B$, then $B \in \Null'_\mu$}
	\EndProof
}
\newpage
\subsubsection{Elementary Transforms}
\Page{
	\DeclareFunc{pushforwardMeasureSpace}
	{
		\prod (X,\Sigma,\mu) : \MS \.
		\prod Y \in \Set \. 
		(X \to Y) \to \MS
	}
	\DefineNamedFunc{pushforwardMeasureSpace}
	{\varphi}{\Big(Y,\varphi_*\Sigma,\varphi_*\mu\Big)}
	{
		\Big( Y, \big\{ B \subset Y : \varphi^{-1}(B) \in \Sigma  \big\},
		\Lambda B \in  \varphi_*\Sigma \. \mu\big( \varphi^{-1}(B) \big)\Big)
	}
	\Explain{ By elementary set theory it is evident that $\varphi_*\Sigma$ is sigma-algebra}
	\Explain{ Clearly, 
		$\varphi_*\mu(\emptyset) = 
		\mu\big( \varphi^{-1}(\emptyset)\Big) = 
		\mu(\emptyset) = 0
		$ }
	\Explain{ Using the fact that for disjoint sets their preimages are also disjoint we get additivity of 
		$\varphi_*\mu$ }
	\EndProof
	\\
	\Theorem{MeasureSum}
	{
		\forall (X,\Sigma,\mu),(X,\Sigma',\nu) : \MS \.
		\MS(X,\Sigma \cap \Sigma',\mu + \nu)
	}
	\Explain{ $ (\mu + \nu)(\emptyset) = \mu(\emptyset) + \nu(\emptyset)  = 0 + 0 = 0$ }
	\Explain{
		$
			(\mu + \nu)\left( \bigsqcup^\infty_{n=1} A_n  \right) = 
			\mu\left( \bigsqcup^\infty_{n=1} A_n  \right)
			+
			\nu\left( \bigsqcup^\infty_{n=1} A_n  \right) =
			\sum^\infty_{n=1} \mu(A_n)
			+
			\sum^\infty_{n=1} \nu(A_n) =
			\sum^\infty_{n=1} \mu(A_n) + \nu(A_n) 
		$
	}
	\Explain{ Here, in the last step we used non-negativity of summands}
	\Explain{ So, 
		$  (\mu + \nu)\left( \bigsqcup^\infty_{n=1} A_n  \right)  =
		 \sum^\infty_{n=1} (\mu + \nu)(A_n)$
	}
	\EndProof
	\\
	\DeclareFunc{countingTransform}
	{
		\prod (X,\Sigma) \in \BOR \.
		(\Nat \to \Sigma) \to (\Nat \to \Sigma)
	}
	\DefineNamedFunc{countingTransform}{A}{A^\#}
	{
		\Lambda n \in \Nat \.
		\Big\{
			x \in X : \big| \{ k \in \Nat : x \in A_k  \} \big| \ge n 	
		\Big\}
	}
	\Explain{ Note, that each $A^\#_n \in \Sigma$}
	\Explain{ Express $A^\#_n = \bigcup_{I \subset \Nat,|I|=n} \bigcap_{i \in I} A_i  $  } 
	\EndProof
	\\
	\Theorem{CountingTransformSum}
	{
		\forall (X,\Sigma,\mu) : \MS \.
		\forall A : \Nat \to \Sigma \.
		\sum^\infty_{n=1} \mu(A_n) = \sum^\infty_{n=1} \mu(A^\#_{n})
	}
	\Explain{Let $f(x) = \big| \{ k \in \Nat : x \in A_k  \} \big| = \sum^\infty_{n=1} \chi_{A_n}(x)$,
		this function takes only inegral and infinite values}
	\Explain{
	Then, $\int f  \; d\mu = \sum^\infty_{n=1} \mu(A_n)$}
	\Explain{ 
		On the other hand level sets of $f$ 
		for the value $n$ are exactly $A^\#_n \setminus A^\#_{n+1}$ }
	\Explain{
		Thus, $f(x) = \sum^\infty_{n=1} \chi_{A^\#_{n}}(x)$
		and assuming all $\mu(A_n)$ are finite
		}
	\Explain{
		$
			\sum^\infty_{n=1} \mu(A_n) = 
			\sum^\infty_{n=1} n\mu(A^\#_{n} \setminus A^\#_{n+1}) =
			\sum^\infty_{n=1} n\mu(A^\#_{n}) - n\mu(A^\#_{n+1}) =
			\sum^\infty_{n=1} \mu(A^\#_{n})
		$
	}
	\Explain{
		Otherwise, both sums are infinite}
	\EndProof
	\Explain{
		This prove is bogus as the references integration, 
		the proper proof must be purely combinatorial}
}
\Page{
	\Theorem{FiniteSumAlmostFinite}
	{
			\NewLine :			
			\forall (X,\Sigma,\mu) : \MS \.
			\forall A : \Nat \to \Sigma \.
			\forall [0] : \sum^\infty_{n=1} \mu(A_n) < \infty \.
			\forall_\mu x \in X \.  
			\big| \{ k \in \Nat : x \in A_k  \} \big| < \infty
	}
	\Explain{ 
		Assume the contrary}
	\Explain{
		Then, there is some $a \in \Reals_{++}$
		such that every $\mu(A^\#_n) \ge a$	}
	\Explain{
		So,  $\infty = \sum^\infty_{n=1} \mu(A^\#_{n}) =  \sum^\infty_{n=1} \mu(A_n)$}
	\Exclaim{
		A contradiction}
	\EndProof
}
\newpage
\subsubsection{Infimum and Supremum Measures}
\Page{
	\DeclareFunc{infMeasure}{\prod (X,\Sigma) \in \BOR \. 
		?\Measure(X,\Sigma) \to \Measure(X,\Sigma)
	}
	\DefineNamedFunc{infMeasure}{\mathcal{M}}{\inf{\mathcal{M}} = 
		\bigwedge_{\mu \in \mathcal{M}} \mu}
	{
		\Lambda A \in \Sigma \.
		\inf\left\{
			\sum^\infty_{n=1} \mu_n(B_n) \Bigg| 
			\mu : \Nat \to \mathcal{M}, B : \Nat \to \Sigma,
			A \subset \bigcup^\infty_{n=1} B_n
		\right\}
	}
	\Explain{Clearly, $\inf {\mathcal{M}}(\emptyset) =0$ As we can cover $\emptyset$ by $\emptyset$}
	\Explain{Assume $A : \Nat \to \infty$}
	\Explain{Then for any cover $B$ of $\bigcup^\infty_{n=1} A_n$
		we can construct a system of covers $C_{n,m} = A_n \cap B_m$ for $A_n$}
	\Explain{ Conversly any such system by relabling can be transformed to a cover for 
		$\bigcup^\infty_{n=1} A_n$  }
	\Explain{As we can cover each $A_n$ independently we will get additivity}
	\EndProof
	\\
	\Theorem{InfMeasureMaximality}
	{
		\NewLine ::		
		\forall (X,\Sigma) \in \BOR \.
		\forall \mathcal{M} : ?\Measure(X,\Sigma) \. \NewLine \.
		\inf {\mathcal{M}} = 
		\max \Big\{ 
			\mu : \Measure(X,\Sigma), 
			\forall A \in \Sigma \. 
			\forall \nu \in \mathcal{M}  \.
			\mu(A) \le \nu(A)
		\Big\}
	}
	\Explain{ Clearly, for each $\nu \in \mathcal{M}$ and $A \in \Sigma$ 
		we can take cover of $B_1=A$ and $\mu_1=\nu$, so
		$\inf {\mathcal{M}}(A) \le \nu(A)$}
	\Explain{ Now assume 
		$\mu$ is another measure and such that $\forall A \in \Sigma$ 
		that $\mu(A)  \le  \inf_{\nu \in \mathcal{M }} \nu(A)$ 
	}
	\Explain{ Then, Clearly, by definition $\inf \mathcal{M} \ge \mu$}
	\EndProof
	\\
	\Theorem{InfimumProperty}
	{
		\NewLine ::
		\forall (X,\Sigma) \in \BOR \.
		\forall \mathcal{M} : \TYPE{DownwardsDirected}\;\Measure(X,\Sigma) \. \NewLine \.
		\forall A \in \Sigma \.
		\Big( \inf \mathcal{M}\Big)(A) = \inf \Big(\mathcal{M}(A)\Big) 
	}
	\Explain{ If there are$\nu_i,\nu_j \in \mathcal{M}$  for the cover with $B_i,B_j$  
		  by using downward direction of $\mathcal{M}$ select $\nu' \le \nu_i,\nu_j$	
	}
	\Explain{ Then $\nu'(B_i \cup B_j) \le \nu'(B_i) + \nu'(B_j) \le \nu_i(B_i) + \nu_j(B_j) $}
	\Explain{So, by definition $\Big( \inf \mathcal{M}\Big)(A) $ 
		will convergere to $\inf \Big(\mathcal{M}(A)\Big)$}
	\EndProof
	\\
	\DeclareFunc{supMeasure}{\prod (X,\Sigma) \in \BOR \. 
		?\Measure(X,\Sigma) \to \Measure(X,\Sigma)
	}
	\DefineNamedFunc{supMeasure}{\mathcal{M}}{\sup{\mathcal{M}} = 
		\bigvee_{\mu \in \mathcal{M}} \mu}
	{
		\Lambda A \in \Sigma \. \NewLine \.
		\sup\left\{
			\sum^\infty_{n=1} \mu_n(B_n) \Bigg| 
			\mu : \Nat \to \mathcal{M}, B : \TYPE{DisjointSequence}(X,\Sigma),
			\bigcup^\infty_{n=1} B_n \subset A
		\right\}
	}
}
\Page{
	\Theorem{SupMeasureMinimality}
	{
		\NewLine ::		
		\forall (X,\Sigma) \in \BOR \.
		\forall \mathcal{M} : ?\Measure(X,\Sigma) \. \NewLine \.
		\sup {\mathcal{M}} = 
		\min \Big\{ 
			\mu : \Measure(X,\Sigma), 
			\forall A \in \Sigma \. 
			\forall \nu \in \mathcal{M}  \.
			\mu(A) \ge \nu(A)
		\Big\}
	}
	\Explain{ Dualize result for inf measure }
	\EndProof
	\\
	\Theorem{SupremumProperty}
	{
		\NewLine ::
		\forall (X,\Sigma) \in \BOR \.
		\forall \mathcal{M} : \TYPE{UpwardDirected}\;\Measure(X,\Sigma) \. \NewLine \.
		\forall A \in \Sigma \.
		\Big( \sup \mathcal{M}\Big)(A) = \sup \Big(\mathcal{M}(A)\Big) 
	}
	\Explain{ Dualize result for sup measure }
	\EndProof
	\\
	\Theorem{MeasuresAreCompleteLattice}
	{
		\forall X \in \BOR \.
		\TYPE{CompleteLattice}\Big( X, \Measure(X)  \Big)
	}
	\Explain{ Use results on minimality and maximality }
	\EndProof
}
\newpage
\subsubsection{Applications of Dynkin Classes}
\Page{
	\Theorem{MeasureEqTHM}
	{
		\NewLine :		
		\forall (X,\Sigma,\mu),(X,T,\nu)  : \MS \.
		\forall I \subset \Sigma \cap T \.
		\forall \aleph : \forall A \in I \. \mu(A) = \nu(A) \.
		\forall \beth  : \mu(X) = \nu(X) \. \NewLine \.
		\forall \gimel : \forall A,B \in I \. A \cap B \in I \.
		\forall B \in \sigma(I) \. 
		\mu(B) = \nu(B)
	}
	\Explain{ 
		Define $\A = \big\{ E \in \Sigma \cap T | \mu(E) = \nu(E) \big\}$}
	\Explain{
		Then $\A$  contains $\emptyset, X$ and is closed under intersections
		and disjoint unions}
	\Explain{
		So, $\A$ is a $\lambda$-class}
	\Explain{
		But $I$ clearly is a $\pi$-class, so $\sigma(I) \subset \A$}
	\EndProof
	\\
	\Theorem{SubalgebraApproximationTHM}
	{
		\NewLine :
		\forall (X,\Sigma,\mu) : \MS \.
		\forall A \subset_BOOL \Sigma \.
		\forall E \in \sigma(A) \.
		\forall \varepsilon \in \Reals_{++} \.
		\exists F \in A \.
		\mu(E \du F) < \varepsilon
	}
	\Explain{ 
		This is also an application of $\pi\hyph\lambda$ theorem}
}
\newpage
\subsection{Outer Measures}
\subsubsection{Subject}
\Page{
	\DeclareType{\OM}
	{
		\prod_{X \in \SET} ?X \to \EReals_+
	}
	\DefineType{\theta}{\OM}
	{
		\theta(\emptyset) = 0 \And \NewLine \And
		\forall A \subset X \. \forall B \subset A \. \theta(A) \ge \theta(B) \And \NewLine \And
		\forall A : \Nat \to X \. \theta\left( \bigcup^\infty_{n=1} A_n\right) \le \sum^\infty_{n=1} \theta(A_n)
	}
	\\
	\DeclareFunc{meaurableSets}
	{
		\prod_{X \in \SET} \OM(X) \to ??X
	}
	\DefineNamedFunc{measurableSets}{\theta}{\Sigma_\theta}
	{
		\Big\{ E \subset X : \forall A \subset X \. \theta(A) = \theta(A \setminus E) + \theta(A \cap E)   \Big\}
	}
	\\
	\Theorem{SumOfOuterMeasures}
	{
		\forall X \in \SET \.
		\forall \alpha,\beta : \OM(X) \.
		\OM(X,\alpha + \beta)
	}
	\Say{[1]}{\Elim (\alpha + \beta)(\emptyset)  \Elim_1 \OM(X,\alpha \And \beta) \Elim 
		\TYPE{Neutral}(\Reals)(0)
	}
	{
		(\alpha + \beta)(\emptyset) =
		\alpha(\emptyset) + \beta(\emptyset) = 
		0 + 0 =
		0
	}
	\Say{[2]}
	{
		\Lambda A,B \subset X \.
		\Lambda T : A \subset B \.
		\Elim 	(\alpha + \beta)(A)
		\Elim_2 \OM(X,\alpha + \beta, T) 
		\THM{NonNegSumIneq}\big(\Reals,\alpha(B) \And \beta(B)\big) \NewLine
		\Intro (\alpha + \beta)(B)
	}
	{
		\Lambda A,B \subset X \.
		\Lambda T : A \subset B \.
		(\alpha + \beta)(A) =
		\alpha(A) + \beta(A) \le 
		\alpha(B) + \beta(B)  =
		(\alpha + \beta)(B)
	}
	\Say{[3]}
	{
		\Lambda A : \Nat \to 2^X \.
		\Elim (\alpha + \beta)\left( \bigcup^\infty_{n=1} A_n \right)
		\Elim_3 \OM(X,\alpha + \beta,A)
		\THM{NonNegSumIneq}\big(\Reals,\ldots\big) \NewLine
		\THM{NonNegSumPermutation}
		\Lambda n \in \Nat \.
		\Intro (\alpha + \beta)(A_n)		
	}
	{
		\NewLine :		
		\forall A : \Nat \to 2^X \.
		(\alpha + \beta)\left( \bigcup^\infty_{n=1} A_n \right) =
		\alpha\left( \bigcup^\infty_{n=1} A_n \right) +
		\beta\left( \bigcup^\infty_{n=1} A_n \right) \le
		\sum^\infty_{n=1} \alpha(A_n) +
		\sum^\infty_{n=1} \beta(A_n) =
		\sum^\infty_{n=1} (\alpha + \beta)(A_n)		
	}
	\Conclude{[*]}{
		\Intro \OM [1][2][3]	
	}
	{
		\OM(X,\alpha + \beta)
	}
	\EndProof
	\\
	\Theorem{OuterMeasureSup}
	{
		\forall X \in \SET \.
		\forall \Theta : ?\OM(X) \.
		\OM(X,\sup \Theta)
	}
	\Say{[1]}{\Elim (\sup \Theta )(\emptyset)  
		\Lambda \theta \in \Theta \.		
		\Elim_1 \OM(X,\theta) \Elim 
		\TYPE{Neutral}(\Reals)(0)
	}
	{
		(\sup \Theta)(\emptyset) =
		\sup_{\theta \in \Theta} \theta(\emptyset) =
		\sup_{\theta \in \Theta} 0
		=
		0
	}
	\Say{[2]}
	{
		\Lambda A,B \subset X \.
		\Lambda T : A \subset B \.
		\Elim 	(\sup \Theta)(A)
		\Lambda \theta \in \Theta \. \Elim_2 \OM(X,\theta, T)  \NewLine
		\Intro (\sup \Theta)(B)
	}
	{
		\Lambda A,B \subset X \.
		\Lambda T : A \subset B \.
		(\sup \Theta )(A) =
		\sup_{\theta \in \Theta} \theta(A) \le
		\sup_{\theta \in \Theta} \theta(B) = 
		(\sup_{\theta \in \Theta})(B)
	}
	\Say{[3]}
	{
		\Lambda A : \Nat \to 2^X \.
		\Elim ( \sup \Theta)\left( \bigcup^\infty_{n=1} A_n \right)
		\Lambda \theta \in \Theta \.
		\Elim_3 \OM(X,\theta,A)
		\THM{SumIneq}(\Reals)
		\THM{SupSumIneq}(\Reals)
		\NewLine
		\Lambda n \in \Nat \.
		\Intro (\sup \Theta)(A_n)		
	}
	{
		\NewLine :		
		\forall A : \Nat \to 2^X \.
		(\sup \Theta)\left( \bigcup^\infty_{n=1} A_n \right) =
		\sup_{\theta \in \Theta} \theta\left( \bigcup^\infty_{n=1} A_n \right) \le
		\sup_{\theta \in \Theta} \sum^\infty_{n=1} \theta(A_n) \le
		\sum^\infty_{n=1} \sup_{\theta \in \Theta} \theta(A_n) =
		\sum^\infty_{n=1} (\sup \Theta)(A_n)
	}
	\Conclude{[*]}{
		\Intro \OM [1][2][3]	
	}
	{
		\OM(X,\sup \Theta)
	}
	\EndProof
}
\Page{
	\DeclareFunc{outerMeasureMin}
	{
		\prod_{ X \in \SET } \.
		\OM^2(X) \to \OM(X)
	}
	\DefineNamedFunc{outerMeasureMin}{\alpha,\beta}{\alpha \wedge \beta}
	{
		\Lambda A \subset X \.
		\inf \Big\{ \alpha( E ) + \beta(A \setminus E) \Big|  E \subset A  \Big\}	
	}
	\Say{[1]}{\Elim \alpha \wedge \beta \Elim \emptyset \Elim_1 \OM(X,\alpha \And \beta) \Elim \inf}
	{
			\alpha \wedge \beta (\emptyset) =
			\inf  \Big\{ \alpha( E ) + \beta(\emptyset) \Big|  E \subset A  \Big\} =
			\inf \{ 0 \} = 0
	}
	\Say{[2]}{
		\Lambda A \subset B \subset X \.
		\Elim \alpha \wedge \beta 
		\THM{DifferneceWithSelfIntersection}
		\Elim_2 \OM(X,\alpha \And \beta)
		\Intro \alpha \wedge B
	}
	{
			\NewLine :
			\forall A \subset B \subset X \.			
			\alpha \wedge \beta (A) =
			\inf  \Big\{ \alpha( E ) + \beta(A \setminus E) \Big|  E \subset A  \Big\} =
			\inf  \Big\{ \alpha( E \cap A ) + \beta(A \setminus E) \Big|  E \subset B  \Big\}
			\le \NewLine \le
			\inf  \Big\{ \alpha( E ) + \beta(B \setminus E) \Big|  E \subset B  \Big\} =
			\alpha \wedge \beta(B)
	}
	\Assume{A}{\Nat \to 2^X}
	\Say{[3]}{
		\Lambda B \subset \bigcup^\infty_{n=1}	A_n \.
		\THM{UnionDecompositon}(X,A)
		\Elim_3 \OM(X,\alpha)
	}
	{
		\NewLine \.		
		\forall B \subset \bigcup^\infty_{n=1}	A_n \.
		\alpha(B) \le \sum^\infty_{n=1} \alpha(A_n \cap B) 
	}
	\Say{[4]}{
		\Lambda B \subset \bigcup^\infty_{n=1}	A_n \.
		\THM{UnionDifferenceDecompositon}(X,A)
		\Elim_3 \OM(X,\beta)
	}
	{	
		\NewLine \.		
		\forall B \subset \bigcup^\infty_{n=1}	A_n \.
		\beta\left(\bigcup^\infty_{n=1} A_n \setminus B\right) 
		\le \sum^\infty_{n=1} \beta\left(A_n \setminus B \right) 
	}
	\Conclude{[A.*]}{
		\Elim \alpha \wedge	\beta
		[3][4]
		\Elim_2 \OM(X,\beta)
		\THM{IndependentInfSum}(\Reals)
		\Intro \alpha \wedge \beta
	}
	{
		\NewLine :		
		\alpha \wedge \beta\left( \bigcup^\infty_{n=1} A_n \right) =
		\inf \left\{
			\alpha\left(  B \right) +
			\beta\left(  \bigcup^\infty_{n=1} A_n \setminus B \right)
		\Bigg|
			B \subset \bigcup^\infty_{n=1} A_n		
		\right\} 
		= \NewLine =
		\inf \left\{
			\alpha\left(  \bigcup^\infty_{n=1} B_n \right) +
			\beta\left(  \bigcup^\infty_{n=1} A_n \setminus \bigcup^\infty_{n=1} B_n  \right)
		\Bigg|
			B_n \subset  A_n		
		\right\}
		\le \NewLine \le
		\inf \left\{
			\sum^\infty_{n=1} \alpha( B_n \cap A_n ) +  
			\beta\left(A_n \setminus \bigcup^\infty_{n=1}B_n\right)
		\Bigg|
			B_n \subset A_n		
		\right\} 
		\le \NewLine \le 
		\inf \left\{
			\sum^\infty_{n=1} \alpha( B_n \cap A_n ) +  
			\beta\left(A_n \setminus B_n\right)
		\Bigg|
			B_n \subset A_n		
		\right\}
		= \NewLine =
		\sum^\infty_{n=1} 
		\inf \Big\{
			\alpha(B \cap A_n) + \beta(A_n \setminus B)  \Big|
			B \subset A_n
		\Big\} =
		\sum^\infty_{n=1} \alpha \wedge \beta (B)
	}
	\Derive{[3]}{\Intro \forall}
	{
		\forall A : \Nat \to 2^X \. 
		\alpha \wedge \beta\left( \bigcup^\infty_{n=1} A_n \right)	\le 
		\sum^\infty_{n=1} \alpha \wedge \beta (B)
	}
	\Conclude{[*]}{\Intro \OM [1][2][3]}
	{
		\OM(X,\alpha \wedge \beta)	
	}
	\EndProof
}\Page{
	\DeclareFunc{outerMeasurePushfoward}
	{
		\prod_{X,Y \in \SET}  \OM(X) \to (X \to Y) \to \OM(Y)	
	}
	\DefineNamedFunc{outerMeasurePushfoward}{\theta,f}{f_* \theta}
	{
		\Lambda A \subset Y \.  \theta\Big( f^{-1}(A) \Big)	
	}
	\Say{[1]}{\Elim f_* \theta \THM{EmptyPreimage} \Elim_1 \OM(X,\theta)}
	{
		f_* \theta(\emptyset) = 
		\theta \Big( f^{-1}(\emptyset) \Big)	=
		\theta(\emptyset) = 0
	}
	\Say{[2]}{
		\Lambda A,B \subset Y \.
		\Lambda T : A \subset B \.
		\Elim f_* \theta \THM{PreimageMonotonicitity}(X,Y,f,A,B,T) \Elim_2 \OM(X,\theta) \Intro f_* \theta}
	{
		\NewLine :		
		\forall A \subset B \subset Y \.		
		f_* \theta(A) = 
		\theta \Big( f^{-1}(A) \Big)	\le 
		\theta \Big( f^{-1}(B) \Big) =
		f_* \theta(B)	
	}
	\Say{[3]}{
		\Lambda A : \Nat \to 2^Y \.		
		\Elim f_* \theta \THM{UnionPreimage}(X,Y,f,A) \Elim_3 \OM(X,\theta) \Intro f_* \theta
	}
	{
		\NewLine :		
		\forall A : \Nat \to 2^X \.		
		f_* \theta\left(  \bigcup^\infty_{n=1} A_n \right) = 
		\theta \left( f^{-1}\left(\bigcup^\infty_{n=1} A_n\right) \right)	=
		\theta \left( \bigcup^\infty_{n=1} f^{-1}(A_n) \right) \le 
		\sum^\infty_{n=1} \theta\Big( f^{-1}(A_n) \Big)  =
		\sum^\infty_{n=1} f_* \theta(A_n)
	}
	\Conclude{[*]}{\Intro \OM[1,2,3]}
	{
		\OM(Y, f_* \theta)
	}
	\\
	\DeclareFunc{outerMeasurePullback}
	{
		\prod_{X,Y \in \SET}  \OM(Y) \to (X \to Y) \to \OM(X)	
	}
	\DefineNamedFunc{outerMeasurePullback}{\theta,f}{f^* \theta}
	{
		\Lambda A \subset X \.  \theta\Big( f(A) \Big)	
	}
	\Say{[1]}{\Elim f_* \theta \THM{EmptyImage} \Elim_1 \OM(X,\theta)}
	{
		f^* \theta(\emptyset) = 
		\theta \Big( f(\emptyset) \Big)	=
		\theta(\emptyset) = 0
	}
	\Say{[2]}{
		\Lambda A,B \subset X \.
		\Lambda T : A \subset B \.
		\Elim f^* \theta \THM{ImageMonotonicitity}(X,Y,f,A,B,T) \Elim_2 \OM(X,\theta) \Intro f^* \theta}
	{
		\NewLine :		
		\forall A \subset B \subset X \.		
		f^* \theta(A) = 
		\theta \Big( f(A) \Big)	\le 
		\theta \Big( f(B) \Big) =
		f^* \theta(B)	
	}
	\Say{[3]}{
		\Lambda A : \Nat \to 2^X \.		
		\Elim f^* \theta \THM{UnionImage}(X,Y,f,A) \Elim_3 \OM(X,\theta) \Intro f^* \theta
	}
	{
		\NewLine :		
		\forall A : \Nat \to 2^X \.		
		f^* \theta\left(  \bigcup^\infty_{n=1} A_n \right) = 
		\theta \left( f\left(\bigcup^\infty_{n=1} A_n\right) \right)	=
		\theta \left( \bigcup^\infty_{n=1} f(A_n) \right) \le 
		\sum^\infty_{n=1} \theta\Big( f(A_n) \Big)  =
		\sum^\infty_{n=1} f^* \theta(A_n)
	}
	\Conclude{[*]}{\Intro \OM[1,2,3]}
	{
		\OM(X, f^* \theta)
	}
	\\
	\Theorem{OuterMeasureEquation}
	{
		\NewLine ::		
		\forall X \in \SET \.
		\forall \theta : \OM(X) \.
		\forall E \in \Sigma_\theta \.
		\forall A \subset X \.
		\theta(A \cap E) + \theta(A \cup E) = \theta(A) + \theta(E)
	}
	\Explain{ Assume $\theta(A \setminus E)$ is finite}
	\Explain{ Otherwise we get $\infty = \infty$}
	\Say{[1]}{
		\Elim \Sigma_\theta(E,A)
		\Elim \Sigma_\theta(E,A \cup E)
		\LOGIC{CheckingBooleanTablese}
		\Elim \TYPE{Inverse}\Big( \Reals, \theta(A \setminus E) \Big) 	
	}
	{
		\NewLine ::		
		\theta(A \cap E) + \theta(A \cup E) = 
		\theta(A) - \theta(A \setminus E) + \theta(A \cup E)  = \NewLine =
		\theta(A) - \theta(A \setminus E) 
		+ \theta\big( (A \cup E) \cap E\big) 
		+ \theta\big( (A \cup E) \setminus E \big) = 
		\theta(A) - \theta(A \setminus E) 
		+ \theta(E) + \theta(A \setminus E) = \NewLine =
		\theta(A) + \theta(E)
	}
	\EndProof
}
\newpage
\subsubsection{Caratheodory Construction}
\Page{
	\Theorem{CaratheodoryConstruction1}
	{
		\forall X \in \SET \.
		\forall \theta : \OM(X)	\.
		\SA\Big(X,\Sigma_\theta\Big)
	}
	\Say{[1]}{\THM{DifferenceDecomposition}(X)}
	{
		\forall A,E \subset X \.   (A \cap E) \cup (A \setminus E) = A
	}
	\Say{[2]}{\Elim_3 \OM(X,\theta)[1]}
	{
		\forall A,E \subset X \.
		\theta(A \cap E)  + \theta(A \setminus E) = \theta(A)
	}
	\Say{[3]}{\Elim \Sigma_\theta [2]}
	{
		\Sigma_\theta =
		\Big\{
				E \subset X : 
				 \forall A \subset X \. \theta(A) \ge \theta(A \setminus E) + \theta(A \cap E)
		\Big\}
	}
	\Say{[4]}{
		\THM{IntersectionWithEmptySet}(X)
		\THM{EmptysetDifference}(X)
		\Elim \TYPE{Neutral}(\Reals,+,0)
		\Elim_1 \OM(X,\theta)
	}	
	{
			\NewLine :
			\forall A \subset X \.
			\theta(A \cap \emptyset) + \theta(A \setminus \emptyset) =
			\theta(\emptyset) + \theta(A) = 
			0 + \theta(A) =
			\theta(A)	
	}
	\Say{[5]}{\Elim \Sigma_\theta [4]}
	{
		 \emptyset \in \Sigma_\theta
	}
	\Say{[6]}
	{
		\Lambda E \in \Sigma_\theta \.
		\Lambda A \subset X \.
		\THM{IntersectionWithComplement}(X,A,E)
		\THM{DifferenceWithComplement}(X,A,E)
		\NewLine 
		\Elim \Sigma_\mu(E)(A)
	}
	{
		\forall E \in \Sigma_\theta \.
		\forall A \subset X \.
		\mu(A \cap E^\c)   +\mu(A \setminus E^\c)=
		\mu(A \cap E)   +\mu(A \setminus E) =
		\mu(A)
	}
	\Say{[7]}
	{
		\Lambda E,F \in \Sigma_\theta(X) \.
		\Lambda A \subset X \.
		\Elim \Sigma_\theta(E, A \cap (E \cup F))
		\LOGIC{CheckingBooleanTable}(X)
		\Elim \Sigma_\theta(F, A \setminus F) 
		\Elim \Sigma_\theta(E,A)
	}
	{
		\NewLine :		
		\forall E,F \in \Sigma_\theta \.
		\forall A \subset X \.
		\theta\big(A \cap (E \cup F)\big) + 
		\theta\big( A \setminus (E \cup F) \big) = \NewLine =
		\theta\big(A \cap (E \cup F) \cap E \big) +
		\theta\big(A \cap (E \cup F) \setminus E \big)  +
		\theta\big( A \setminus (E \cup F) \big) = \NewLine =
		\theta(A \cap E ) +
		\theta\big( (A \setminus F) \cap  E \big) +
		\theta\big( (A \setminus F) \setminus E \big) = 
		\theta(A \cap E) + \theta(A \setminus E)	=
		\theta(A)	
	}
	\Say{[8]}{\Intro \Alg [5][6][7]}
	{
		\Alg(X,\Sigma_\theta)
	}
	\Assume{E}{\Nat \to \Sigma_\theta}
	\Say{F}{\Lambda n \in \Nat \. \bigcup^n_{k=1} E_k}
	{
		\Nat \to \Sigma_\theta
	}
	\Say{G}{\Lambda n \in \Nat \. \If n = 1 \Then E_1 \Else F_n \setminus F_{n-1}}
	{
		\Nat \to \Sigma_\theta
	}
	\Say{[9]}{\Elim G \Elim F
		}{
			\bigcup^\infty_{n=1} E_n = 
			\bigcup^\infty_{n=1} F_n =
			\bigcup^\infty_{n=1} G_n
		}
	\AssumeIn{A}{2^X}
	\AssumeIn{n}{\Nat}
	\Assume{[10]}{n > 1}
	\Conclude{[A.*]}{
		\Elim \Sigma_\theta\Big( F_{n-1},A \cap F_n\Big)
		\Intro G_n
		\Elim F_n	
	}
	{
		\NewLine :		
		\theta( A \cap F_n) =
		\theta(A \cap F_n \cap F_{n-1})  + \theta(A \cap F_n \setminus F_{n-1})    =
		\theta(A \cap F_{n-1}) + \theta(A \cap G_n)
	}
	\Derive{[10]}{\Elim \Nat}
	{
		\forall A \subset X \.
		\forall n \in \Nat \.
		\theta(A \cap F_n) = \sum^n_{k=1} \theta(A \cap G_k)		
	}
	\AssumeIn{A}{2^X}
	\Say{[11]}{[9]\THM{UnionIntersectDistributivity}(X)
		\Elim_3 \OM(X,\theta)	
		\Elim \TYPE{SeriesLimit} 
		[10]
	}
	{
		\NewLine :		
		\theta\left(A \cap \bigcup^\infty_{n=1} E_n \right) =
		\theta\left(\bigcup^\infty_{n=1} A \cap G_n \right) \le
		\sum^\infty_{n=1} \theta (A \cap G_n) =
		\lim_{m \to \infty} \sum^m_{n=1} \theta (A \cap G_n) =
		\lim_{n \to \infty} \theta(A \cap F_n)
	}
	\Say{[12]}{[9]\THM{UnionIntersectDistributivity}(X)
		\Elim_3 \OM(X,\theta)	
		\THM{MonotonicInfLimit} 
		[10]
	}
	{
		\NewLine :		
		\theta\left(A \setminus \bigcup^\infty_{n=1} E_n \right) =
		\theta\left(A \setminus \bigcup^\infty_{n=1} F_n \right)\le
		\inf_{n=1} \theta\left( A \setminus F_n \right)
        =
		\lim_{n \to \infty} \theta(A \setminus F_n)
	}
	\Conclude{[A.*]}{
			[11][10]
			\THM{LimitSum}(\ldots)
			\THM{ConstantLimit}(\theta(A))
	}
	{
		\NewLine :		
		\theta\left(A \cap \bigcup^\infty_{n=1} E_n \right) + 
		\theta\left(A \setminus \bigcup^\infty_{n=1} E_n \right) \le 
		\lim_{n \to \infty} \theta(A \cap F_n) + \lim_{n \to \infty} \theta(A \setminus F_n) =
		\lim_{n \to \infty } \theta(A \cap F_n) + \theta(A \setminus F_n) = \NewLine =
		\lim_{n \to \infty} \theta(A) =
		\theta(A)
	}
}\Page{
	\Derive{[11]}{\Intro \forall \Intro \Sigma_\theta}
	{
		\forall A : \Nat \to \Sigma_\theta \. 
		\bigcup^\infty_{n=1} A_n \in \Sigma_\theta
	}
	\Conclude{[*]}{\Intro \SA[8][11]}{\SA(X, \Sigma_\theta)}	
	\EndProof
	\\
	\Theorem{CaratheodoryConstruction2}
	{
		\forall X \in \SET \.
		\forall \theta : \OM(X)	\.
		\MS(X,\Sigma_\theta,\theta_{|\Sigma_\theta})
	}
	\Say{[1]}{\Elim_1 \OM(X,\theta)}
	{
		\theta(\emptyset) = 0
	}
	\Assume{A}{\TYPE{DisjointSequence}(X,\Sigma_\theta)}
	\Say{[2]}{\Elim_3 \OM(X,\theta)}
	{
		\theta\left(\bigcup^\infty_{n=1} A_n \right) \le \sum^\infty_{n=1} \theta\left(A_n\right)
	}
	\Say{F}{\Lambda n \in \Nat \. \bigcup^n_{k=1} A_k}{\Nat \to \Sigma_\theta}
	\Say{[3]}{\Elim F}{\bigcup^\infty_{n=1} \theta(A_k) = \bigcup^\infty_{n=1} \theta(F_k) }
	\Say{[4]}{\Elim F \ldots}
	{
		\forall n \in \Nat \. \theta(A_{n+1}) = \theta(F_{n+1}) + \theta(A_n)
	}
	\Say{[5]}{[3] \Elim_2 \OM(X,\theta) [4]}
	{
		\forall n \in \Nat \. 
		\theta\left( \bigcup^\infty_{k=1} A_k \right) =
		\theta\left( \bigcup^\infty_{k=1} F_k \right) \ge 
		\theta\left(  \sum^n_{k=1} F_k \right) =
		\sum^n_{k=1} \theta(A_k)
	}
	\Say{[6]}{\lim_{n \to \infty} [5](n)}
	{
		\theta\left( \bigcup^\infty_{n=1} A_n \right) \ge \sum^\infty_{n=1} \theta(A_n) 
	}
	\Conclude{[A.*]}{[2][5]}
	{
		\theta\left( \bigcup^\infty_{n=1} A_n \right) = \sum^\infty_{n=1} \theta(A_n) 
	}
	\DeriveConclude{[*]}{[1]\Intro \Measure}{\Measure(X,\Sigma_\theta,\theta_{|\Sigma_\theta})}
	\EndProof
	\\
	\Theorem{CaratheodoryExtensionIsComplete}
	{
		\forall X \in \SET \.
		\forall A \subset X \.
		\forall [0] : \theta(A) = 0 \.
		A \in \Sigma_\theta
	}
	\AssumeIn{B}{2^X}
	\Say{[1]}{\THM{IntersectionDifferenceDecomposition}(X,B,A)}
	{
		B = (B \cap A) \cup (B \setminus A)
	}
	\Say{[2]}{\Elim_3 \OM(X,\theta)[1]}
	{
		\theta(B) \leq \theta(B \cap A) + \theta(B \setminus A)
	}
	\Say{[3]}{\Elim_2 \OM(X,\theta)[0] \Elim_2 \OM(X,\theta) }
	{
			\theta(A \cap B) + \theta(B \setminus A) =
			\theta(B \setminus A) \leq  \theta(B)
	}
	\Conclude{[B.*]}{[2][3]}{
		\theta(A \cap B) + \theta(B \setminus A) = \theta(B)	
	}
	\DeriveConclude{[*]}{\Elim \Sigma_\theta}{ A \in \Sigma_\theta  }
	\EndProof
}
\newpage
\subsubsection{Outer Measures from Measures}
\Page{
	\DeclareFunc{outerMeasure}
	{
		\prod_{X \in \BOR} \Measure(X) \to \OM(X)
	}
	\DefineNamedFunc{outerMeasure}{\mu}{\mu^\star}
	{
		\Lambda A \subset X \.  
		\inf \big\{  \mu(E) \big|   A \subset E \in \S_X     \big\}
	}
	\AssumeIn{A}{2^X}
	\Say{\Big( E, [1] \Big)}{\Elim \mu^\star(A)}
	{
		\sum E : \Nat \to \S_X  \.
		\forall n \in \Nat \. A \subset E_n \And
		\lim_{n \to \infty} \mu(E_n) = \mu^\star(A)
	}
	\Say{[2]}{\Elim \mu^\star \Elim \inf [1.1][1.2]}
	{
		\forall n \in \Nat \. \mu(E_n) \ge \mu^\star(A)
	}
	\Say{F}{\Lambda n \in \Nat \. \bigcap^\infty_{n=1} E_n }{\Nat \downarrow \S_X}
	\Say{[3]}{\Elim F \THM{CommonSubsetIntersection} [1.1]}
	{
		\forall n \in \Nat \. A \subset F_n \subset E_n
	}
	\Say{[4]}{
			\Elim \mu^\star(A) \Elim \Measure(X,\mu)[3]	
	}
	{
		\forall n \in \Nat \.  \mu(E_n) \ge \mu(F_n) \ge \mu^\star(A)
	}
	\Say{[5]}{\THM{DoubleInqLemma}[4][1.1]}
	{
		\lim_{n \to \infty} \mu(F_n) = \mu^\star(A)
	}
	\Conclude{[A.*]}{\THM{UpperContinuity}(X,\mu)[5]}
	{
		\mu\left( \bigcap^\infty_{n=1} F_n \right) = \mu^\star(A)
	}
	\Derive{[1]}{\Intro \forall \Elim \exists}
	{
		\forall A \subset X\. \exists E \in \S_X \.  A \subset E 
		\And \mu(E) = \mu^\star(A)  
	}
	\Say{[2]}{\Elim\mu^\star(\emptyset) \Elim \Measure (X, \mu) }
	{\mu^\star(\emptyset) = \mu(\emptyset) = 0}
	\Say{[3]}{
		\Lambda A,B \subset X \. 
		\Lambda T : A \subset B
		\Elim	\mu^\star(A)
		\THM{AntitoneInf}(T)
		\Intro \mu^\star(B)
	}
	{
		\NewLine :		
		\mu^\star(A) =  \inf \big\{  \mu(E) \big|   A \subset E \in \S_X     \big\} 
		\le \inf \big\{  \mu(E) \big|   B \subset E \in \S_X     \big\}  =
		\mu^\star(B)
	}
	\Assume{A}{\Nat \to 2^X}
	\Say{\Big(E,[4]\Big)}{[1](A)}
	{
		\sum E : \Nat \to \S_X \. \forall n \in \Nat \. 
		\mu(E_n) = \mu^\star(A_n) \And A_n \subset E_n
	}
	\Conclude{[4.*]}{
		\Elim \mu^\star\left( \bigcup^\infty_{n=1} A_n \right)
		\THM{InfBasicBoiunf} \left( \bigcup^\infty E_n  \right)[4.2]
		\THM{Subadditivity}(X,\mu)
		[4.1] 	
	}
	{
		\NewLine :		
		\mu^\star\left( \bigcup^\infty_{n=1} A_n \right) =
		\inf \left\{  \mu(F)  \Bigg|   \bigcup^\infty_{n=1} A_n \subset  F \in \S_X  \right\} \le 
		\mu\left( \bigcup^\infty_{n=1} E_n \right) \le
		\sum^\infty_{n=1} \mu(E_n) =
		\sum^\infty_{n=1} \mu^\star(A_n)
	}
	\Derive{[4]}{\Intro \forall}
	{
		\forall A : \Nat \to 2^X \.
		\mu^\star\left( \bigcup^\infty_{n=1} A_n \right) \le
		 \sum^\infty_{n=1} \mu^\star(A_n)
	}
	\Conclude{[*]}{\Intro \OM [2][3][4]}{\OM(X,\mu^\star)}
	\EndProof
	\\
	\Theorem{OuterMeasureMeasurableRepresentation}
	{
		\NewLine : 		
		\forall X \in \BOR \. \forall \mu : \Measure(X) \.		
		\forall A \subset X\. \exists E \in \S_X \.  A \subset E 
		\And \mu(E) = \mu^\star(A)
	}
	\Explain{ It was proved just above}
	\EndProof
	\\
	\DeclareFunc{subsetSigmaAlgebra}{\prod_{X \in \BOR} 2^X \to \BOR}
	\DefineNamedFunc{subsetSigmaAlgebra}{A}{(A,\S_X|A)}
	{
		\Big(A, \{ A \cap E | E \in \S_X \} \Big)	
	}
}
\Page{
	\Theorem{OriginalSigmaAlgebraIsMeasurable}
	{
		\forall (X,\Sigma,\mu) : \MS \.
		\Sigma \subset \Sigma_{\mu^\star}	
	}
	\AssumeIn{E}{\Sigma}
	\AssumeIn{A}{2^X}
	\Say{\Big(F,[1]\Big)}{\THM{OuterMeasureMeasurableRepresentation}(X,\mu,A)}
	{
		\sum F \in \Sigma \. A \subset F  \And \mu(F) = \mu^\star(A)
	}
	\Say{[2]}{ [1.2] \THM{PairAdditivity}(X,\Sigma,\mu, F \cap E, F \setminus E) \Intro \mu^\star [1.1] }
	{
		\NewLine :		
		\mu^\star(A) = \mu(F) =
		\mu(F \cap E) + \mu(F \setminus E) \ge 
		\mu^\star(A \cap E) + \mu^\star(A \setminus E) 
	}
	\Conclude{[A.*]}{\Elim \OM(X,\mu^\star)[2]}
	{\mu^\star(A) = \mu^\star(A \cap E) + \mu^\star(A \setminus E)}
	\DeriveConclude{[E.*]}{\Intro \Sigma_{\mu^\star}}{E \in \Sigma_{\mu^\star}}
	\DeriveConclude{[*]}{\Intro \subset }{\Sigma \subset \Sigma_{\mu^\star}}	
	\EndProof
	\\	
	\DeclareFunc{subsetMeasure}{\prod (X,\Sigma,\mu) : \MS \. 2^X \to \MS}
	\DefineNamedFunc{subsetMeasure}{A}{(A,\Sigma|A,\mu|A)}
	{
		\Big(A,\Sigma|A,  \mu^\star_{\Sigma|A}  \Big)
	}
	\Explain{ In terms of outer measures this is a pushforward for natural embedding 
		$\iota : A \to X$}
	\Explain{ We need just to show that each $E \in \Sigma |A$ is measurable}
	\Explain{ Represent $E = F \cap A$ with $F \in \Sigma$}
	\Explain{ Then for arbitrary $B \subset A$}
	\Explain{
		$
			\mu^\star(B \cap E) + \mu^\star(B \setminus E)  =
			\mu^\star(B \cap F \cap A) + \mu^\star\big(B \setminus (F \cap A)\big) = 
			\mu^\star(B \cap F) + \mu^\star(B \setminus F) =
			\mu^\star(B) 
		$
	}
	\EndProof
}
\newpage
\subsubsection{Outer Measures and Measures from Functionals}
\Page{
	\DeclareType{UrMeasure}
	{
		\prod_{X \in \SET} ?\Big(2^X \to \EReals_+ \Big)	
	}
	\DefineType{\tau}{UrMeasurel}{\tau(\emptyset) = 0}
	\\
	\DeclareFunc{generateOuterMeasure}
	{
		\prod_{X \in \SET} \TYPE{UrMeasure}(X) \to \OM(X)
	}
	\DefineNamedFunc{generateOuterMeasure}{\tau}{\theta_\tau}
	{
		\Lambda A \subset X \. 
		\inf \left\{
			\sum^\infty_{n=1} \tau(C_n) \Bigg|
				 C : \Nat \to 2^X,  A \subset \bigcup^\infty_{n=1} C_n
		\right\}
	}
	\\
	\DeclareFunc{infOuterMeasure}
	{
		\prod_{X \in \SET}  ?\OM(X) \to \OM(X)
	}
	\DefineNamedFunc{infOuterMeasure}{\Theta}{\inf \Theta = \bigwedge_{\theta \in \Theta} \theta}
	{
		\theta_\tau \quad \where \quad \tau = \Lambda A \subset X \. 
		\bigwedge_{\theta \in \Theta} \theta(A)
	}
	\\
	\Theorem{InfOuterMeasureIsMaximal}
	{
		\NewLine ::		
		\forall X \in \SET \. 
		\forall \Theta : \OM(X) \. 
		\inf \Theta = \max \Big\{ \eta : \OM(X), \forall \theta \in \Theta \. \eta \le \theta  \Big\} 
	}
	\Explain{ Let $\tau = \Lambda A \subset X \. \bigvee_{\theta \in \Theta} \theta(A)$ 
		as in definition of $\inf \Theta$}
	\Explain{ Let $\eta$ be such outer measure that $\forall \theta \in \Theta \. \eta \le \theta$}
	\Explain{ Let $A \subset X$ and take $C$ as in definition of $\theta_\tau(A)$ above }
	\Explain{ Then, by definition of infima and ouer measure
				$
							\sum^\infty_{n=1} \tau(C_n)  =
							\sum^\infty_{n=1} \inf_{\theta \in \Theta}\theta(C_n) \ge 
							\sum^\infty_{n=1} \eta(C_n) \ge 
							\eta(A)
				$
	}
	\Explain{ So, $\eta \le \inf \Theta$ }
	\Explain{ Clearly $\forall \theta \in \Theta \. \inf \Theta \le \theta$, so the theorem holds.}
	\EndProof
	\\
	\Theorem{OuterMeasuresAreCompleteLattice}
	{
		\forall X \in \SET \. 
		\TYPE{CompleteLattice}\Big( \OM(X) \Big)
	}
	\Explain{ Use constructions of inf and sup as above}
	\EndProof
}
\newpage
\subsubsection{Inner Measures}
\Page{
	\DeclareType{\IM}{\prod_{X \in \SET} \Big(X \to \EReals_+ \Big)}
	\DefineType{\theta}{\IM}
	{
		\theta(\emptyset) = 0 
		\And \NewLine \And
		\forall A,B \subset X \. \theta(A \cup B) \le \theta(A) \cup \theta(B)
		\And \NewLine \And
		\forall A : \Nat \downarrow 2^X \. \theta(A_1) < \infty \Imply 
		\theta\left( \bigcap^\infty_{n=1} A_n \right) = \lim_{n \to \infty} \theta(A) 
		\And \NewLine \And
		\forall A \subset X \. \forall \alpha \in \Reals \.  \theta(A) = \infty
		\Imply \exists B \subset A \. \alpha \le \theta(B) < \infty
	}
	\\
	\DeclareFunc{meaurableSets}
	{
		\prod_{X \in \SET} \IM(X) \to \SA(X)
	}
	\DefineNamedFunc{measurableSets}{\theta}{\Sigma_\theta}
	{
		\Big\{ E \subset X : \forall A \subset X \. \theta(A) = \theta(A \setminus E) + \theta(A \cap E)   \Big\}
	}
	\\
	\Theorem{CaratheodoryConstruction3}
	{
		\forall X \in \SET \.
		\forall \theta : \IM(X)	\.
		\MS(X,\Sigma_\theta,\theta_{|\Sigma_\theta})
	}
	\NoProof
	\\
	\DeclareFunc{innerMeasure}
	{
		\prod_{X \in \BOR} \Measure(X) \to \OM(X)
	}
	\DefineNamedFunc{innerMeasure}{\mu}{\mu_\star}
	{
		\Lambda A \subset X \.  
		\sup \big\{  \mu(E) \big|   E \subset  A, E \in \S_X, \mu(E) < \infty     \big\}
	}
	\\
	\Theorem{OriginalSigmaAlgebraIsMeasurable}
	{
		\forall (X,\Sigma,\mu) : \MS \.
		\mu(X) < \infty \Imply
		\Sigma \subset \Sigma_{\mu_\star}	
	}
	\NoProof
}
\newpage
\subsubsection{Some Category Theory}
	Let $\mathcal{B}$ be some subcategory of category of measurables spaces and Let $\mathcal{C}$ be some subcategory of category of complete Lattices. Note, that $\mathcal{C}$ not necessarily has lattice morphism as morphisms. View $\mathsf{OM} : \mathsf{SET} \to \mathcal{C}$ as a functor with $\mathsf{OM}(X)$ is a complete lattice of outer measures on $X$ and $\mathsf{OM}_{X,Y}(f)(\theta) = f^{-1}\theta$. Respectively view $\mathsf{MEAS}:\mathcal{B} \to \mathcal{C}$ as a functor such that $\mathsf{MEAS}(X,\Sigma)$ is a complete lattice of all measures on $(X,\Sigma)$ and $\mathsf{MEAS}_{X,Y}(f)(\mu) = f^{-1}\mu$. If $\mathsf{U}:\mathcal{B} \to \mathsf{SET}$ is a forgetful functor, then there is a 'natural transform' $(\bullet)^* : \mathsf{MEAS} \Rightarrow \mathsf{OM \circ U}$ defined by $\mu^*(A) = \inf \{ \mu(E) | A \subset E \}$.

The problem is to identify categories $\mathcal{B}$ and $\mathcal{C}$, so $(\bullet)^*$ is actually a naturally transform. Ideally, I also want $f^{-1}$ acting on (outer) measures to be lattice morphisms. But this is another problem. At least I can always claim that they are monotonic. Probably, they may be shown to be suplattice morphisms.

So I want to identify the measurable spaces $(X,\Sigma_X),(Y,\Sigma_Y)$ and a measurable 
 maps $f : X \to Y$ such that $ f^{-1}(\mu^*) = (f^{-1}\mu)^*$ for every measure $\mu \in \mathsf{MEAS}(X,\Sigma_X)$. 

Let $f : X \to Y$ be measurable, $A \subset Y$. 

Then $$
f^{-1}(\mu^*)(A) = \mu^*\big(f^{-1}(A)\big) = 
\min\Big\{  \mu(E) \Big|  f^{-1}(A) \subset   E \in \Sigma_X \Big\} 
$$ 
and
$$
(f^{-1}\mu)^*(A)  =  \min\Big\{  f^{-1}\mu(E) \Big|  A \subset   E \in \Sigma_Y \Big\}=
 \min\Big\{  \mu\big(f^{-1}(E)\big) \Big|  A \subset   E \in \Sigma_Y \Big\}
$$

It seems that $f^{-1}(\mu^\star) \le (f^{-1} \mu)^\star$ always true. The converse may be true if $f(E)$ is measurable for every measurable $E \in \Sigma_X$ an $f$ is injective. This holds, for example, if $\mathcal{B}$ consists of standard Borel spaces (in sense of classical descriptive set theory) and every morphism is injective. Or there may be some way to use more general trick, but I haven't thought anything yet.
\newpage
\subsubsection{Measurable Envelopes}
\Page{
	\DeclareType{MeasurableEnvelope}
	{
		\prod (X,\Sigma,\mu) : \MS \.
		\prod A \subset X \. ?\Sigma	
	}
	\DefineType{E}{MeasurableEnvelope}
	{
		A \subset E \And 
		\forall F \in \Sigma \.  \mu(F \cap E) = \mu^*(F \cap A)
	}
	\\
	\Theorem{MeasurableEnvelopeByNullSets}
	{
		\forall (X,\Sigma,\mu) : \MS \.
		\forall A \subset X \. 
		\forall A \subset E \in \Sigma \. \NewLine \.
		\ME(A,E) \iff
		\forall F \in \Sigma \.
		F \subset E \setminus A \Imply
		\mu(F) = 0	
	}
	\Assume{[1]}{\ME(A,E)}
	\AssumeIn{F}{\Sigma}
	\Assume{[2]}{F \subset E \setminus A}
	\Conclude{[1.*]}
	{
		[2]\THM{SupersetIntersection} 
		\Elim \ME(A,E)(F)
		[2]\THM{DifferenceIntersection}
		\Elim_1 \OM(X,\mu^*)	
	}
	{
		\mu(F) = \mu(F \cap E) = 
		\mu^*(F \cap A ) = 
		\mu^*(\emptyset) = 0
	}
	\Derive{[1]}{\Intro \Imply}
	{
		\ME(A,E) \Imply
		\Big(\forall F \in \Sigma \.
		F \subset E \setminus A \Imply
		\mu(F) = 0 \Big)	
	}
	\Assume{[2]}
	{
		\forall F \in \Sigma \.
		F \subset E \setminus A \Imply
		\mu(F) = 0	
	}
	\AssumeIn{H}{\Sigma}
	\Conclude{[H.*]}{
		\Elim \mu^* \THM{Monotonicity}(X,\Sigma,\mu)
		\THM{Difference}(X,\Sigma,\mu)[1]^2
	}
	{
		\NewLine :		
		\mu^*(A \cap H) =
		\inf \Big\{ \mu(G) \Big|  A \cap H \subset  G \in \Sigma \Big\} =
		\inf \Big\{ \mu\big((H \cap E) \setminus F\big)  \big| F \in \Sigma,  F \subset E \setminus A \Big\}= \NewLine =
		\inf \Big\{ \mu(H \cap E)  - \mu( F)  \big| F \in \Sigma,  F \subset E \setminus A \Big\} =
		\mu(H \cap E)
	}
	\DeriveConclude{[2.*]}{\Intro \ME}{\ME(A,E)}
	\Conclude{[*]}{\Intro (\iff)}
	{
		\ME(A,E) \iff
		\forall F \in \Sigma \.
		F \subset E \setminus A \Imply
		\mu(F) = 0	
	}
	\EndProof
	\\
	\Theorem{MeasurableEnvelopeByEq}
	{
		\forall (X,\Sigma,\mu) : \MS \.
		\forall A \subset X \. 
		\forall A \subset E \in \Sigma \. \NewLine \.
		\forall \aleph : \mu(E) < \infty  \.
		\ME(A,E) \iff
		\mu(E) = \mu^*(A)
	}
	\Assume{[1]}{\ME(A,E)}
	\Conclude{[1.*]}{
				\THM{Selfintersrction}(E)
				\Elim \ME(A,E)
				\THM{IntersectionWithSubset}(E \cap A)
	}{
		\NewLine :		
		\mu(E) = \mu(E \cap E) = \mu^*( E \cap A) = \mu^*(A)
	}
	\Derive{[1]}{\Intro \Imply}
	{
		\ME(A,E) \Imply
		\mu(E) = \mu^*(A)
	}
	\Assume{[2]}{\mu(E) = \mu^*(A)}
	\AssumeIn{F}{\Sigma}
	\Assume{[3]}{F \subset E \setminus A}
	\Say{[4]}{		
		\THM{Difference}(\mu,E,F)[3] 
		\Intro \mu^*(A \setminus F)
		\THM{DisjointDifference}[3]		
		[2]	
	}
	{
		\NewLine :		
		\mu(E) - \mu(F)	=
		\mu(E \setminus F) \ge 
		\mu^*(A \setminus F) 	=
		\mu^*(A) = 
		\mu(E)
	}
	\Conclude{[F.*]}{\Elim \mu(F)\Elim \aleph \Big([4] - \mu(E)\Big)}
	{
		\mu(F) = 0
	}
	\DeriveConclude{[2.*]}{\THM{MeasurableEnvelopeByNullSets}}
	{
			\ME(A,E)
	}
	\DeriveConclude{[*]}{\Intro (\iff) [1]}
	{
		\ME(A,E) \iff
		\mu(E) = \mu^*(A)
	}
	\EndProof
}\Page{
	\Theorem{Intersection}
	{
		\NewLine ::		
		\forall (X,\Sigma,\mu) \in \MEAS \.
		\forall A \subset X \.
		\forall E : \ME(X,\Sigma,\mu,A) \.
		\forall H \in \Sigma \. \NewLine
		\ME(A \cap H,E \cap H)
	}
	\Explain{
		Pretty simple result}
	\Explain{
		Assume $G\in \Sigma$}
	\Explain{
		Then $\mu^*(A \cap H \cap G) = \mu(E \cap H \cap G)$ by definition of measurable envelope $E$}
	\EndProof
	\\
	\Theorem{CountableUnion}
	{
		\NewLine ::		
		\forall (X,\Sigma,\mu) \in \MEAS \.
		\forall A : \Nat \to 2^X \.
		\forall E : \prod^\infty_{n=1} \ME(X,\Sigma,\mu,A) \. \NewLine \.
		\ME\left( \bigcup^\infty_{n=1} A_n, \bigcup^\infty_{n=1} E_n \right)
	}
	\AssumeIn{F}{\Sigma}
	\Assume{[1]}{F \subset \bigcup^\infty_{n=1} E_n \setminus \bigcup^\infty_{n=1} A_n }
	\AssumeIn{n}{\Nat}
	\SayIn{Z}{F\cap E_n}{\Sigma}
	\Say{[2]}{\Elim Z [1]  \THM{UnionDifference}(X)}
	{Z \subset E_n \setminus \bigcup^\infty_{m=1} A_m \subset E_n \setminus A_n}
	\Conclude{[n.*]}{\THM{MeasurableEnvelopeByZeroSets}[2]}
	{
		\mu(Z) = 0
	}
	\Derive{[2]}{\Intro \forall}{\forall n \in \Nat \. \mu(F \cap E_n) = 0}
	\Say{[3]}{\THM{DifferenceSubset}[1]}{F \subset \bigcup^\infty_{n=1} E_m}	
	\Conclude{[F.*]}{
		\THM{UnionSubsetDecomposition}[3]
		\THM{Subadditivity}(X,\Sigma,\mu)
		[2] \THM{ZeroSum}(\Reals)	
	}{
		\NewLine		
		\mu(F) = 
		\mu\left( \bigcup^\infty_{n=1} F \cap E_n \right) \le
		\sum^\infty_{n=1} \mu(F \cap E_n) =
		\sum^\infty_{n=1} 0 = 0
	}
	\DeriveConclude{[*]}{\Intro \ME}
	{
		\ME\left( \bigcup^\infty_{n=1} A_n, \bigcup^\infty_{n=1} E_n \right)
	}
	\EndProof
}\Page{
	\Theorem{MeasurableEnvelopeByFiniteMeasureCover}
	{
		\NewLine :: 		
		\forall (X,\Sigma,\mu) \in \MEAS \.
		\forall A \subset X \.
		\forall E :\Nat \to \Sigma \.
		\forall \aleph : A \subset \bigcup^\infty_{n=1} E_n \.
		\forall \beth : \forall n \in \Nat \. \mu(E_n) < \infty \. \NewLine \.
		\exists \ME(X,\Sigma,\mu,A)
	}
	\AssumeIn{n}{\Nat}
	\SayIn{B_n}{A \cap E_n}{2^X}
	\Say{[1]}{\Elim B_n \THM{IntersectionIsSubset}(X) \Elim \OM(X,\mu^*) \Elim \mu^*(E_n) \Elim \beth}
	{
		\mu^*(B_n) \le \mu^*(E_n) = \mu(E_n) < \infty
	}
	\Say{\Big(F_n,[2]\Big)}{\THM{OuterMeasureRepresention}(X,\Sigma,\mu,B_n)}
	{
		\sum F_n \in \Sigma \.  B_n \subset F_n \And \mu(B_n) = \mu(F_n) 
	}
	\Say{[3]}{[2.2][1]}{\mu(F_n)<\infty}
	\Conclude{[n.*]}{\THM{MeasurableEnvelopeByEq}[2.2][3]}
	{
		\ME(B_n,F_n)
	}
	\Derive{F}{\Intro \prod}
	{
			\prod^\infty_{n=1} \ME(A \cap E_n)  	
	}
	\Say{[1]}{\Elim \aleph \THM{UnionDecomposition}(X)\Lambda n \in \Nat \. \Elim_0 \ME(A \cap E_n,F_n)}
	{
		A = \bigcup^\infty_{n=1} A \cap E_n \subset \bigcup^\infty_{n=1} F_n
	}
	\Conclude{[*]}{\THM{CountableUnion}[1]}{\ME\left( A, \bigcup^\infty_{n=1} F_n\right)}
	\EndProof
	\\
	\DeclareType{Thick}{\prod (X,\mu) \in \MEAS \. ??X }
	\DefineType{A}{Thick}{\ME(X,\mu,X,A)}
}
\newpage
\subsection{Lebesgue Integration}
\subsubsection{Real-Valued Measurable Functions}
\Page{
	\Theorem{OpenRaysMeasurabilityCondition1}
	{
		\NewLine :		
		\forall X \in \BOR \. 
		\forall D \subset X \.
		\forall f : D \to \Reals \. 
		\forall \aleph : \forall t \in \Reals \. f^{-1}(-\infty, t) \in \S_X \. 
		f \in \BOR(D,\Reals)
	}
	\Explain{ Use basic set-algebra of intervals}
	\Explain{ Express $[a,b)$ by complementation $(-\infty,b)\setminus (-\infty,a]$}
	\Explain{ Then express $(c,b) = \bigcap^\infty_{n=1}[c-2^{-n},b)$ }
	\Explain{ It is possible to continue so on to get all Borel sets}
	\Explain{ As preimage $f^{-1}$ commutes with basic set theoretic operations the function
	$f$ is measurable}
	\EndProof
	\Theorem{OpenRaysMeasurabilityCondition2}
	{
		\NewLine:		
		\forall X \in \BOR \. 
		\forall D \subset X \.
		\forall f : D \to \Reals \. 
		\forall \aleph : \forall t \in \Reals \. f^{-1}(t, + \infty) \in \S_X \. 
		f \in \BOR(D,\Reals)
	}
	\NoProof
	\Theorem{ClosedRaysMeasurabilityCondition1}
	{
		\NewLine :		
		\forall X \in \BOR \. 
		\forall D \subset X \.
		\forall f : D \to \Reals \. 
		\forall \aleph : \forall t \in \Reals \. f^{-1}(-\infty, t] \in \S_X \. 
		f \in \BOR(D,\Reals)
	}
	\NoProof
	\Theorem{ClosedRaysMeasurabilityCondition1}
	{
		\NewLine : 		
		\forall X \in \BOR \. 
		\forall D \subset X \.
		\forall f : D \to \Reals \. 
		\forall \aleph : \forall t \in \Reals \. f^{-1}[t, + \infty) \in \S_X \. 
		f \in \BOR(D,\Reals)
	}
	\NoProof
	\\
	\Theorem{IncreasingIsBorelMeasurable}
	{
		\forall D \subset \Reals \.
		\forall f : D \uparrow \Reals \.
		\forall f \in \BOR(D,\Reals)
	}
	\Explain{ Note, that there is always an $x\in \Reals$ such that $f^{-1}(t,+\infty)= (x,+\infty) \cap D$
	or $f^{-1}(t,+\infty)= [x,+\infty) \cap D$}
	\Explain{
		Both are Borel measurable in $D$}
	\Explain{
		To be concrete 
		$x = \inf \{ x \in D : f(x) > t   \}$}
	\Explain{
		Then by last theorem $f$ is measurable
	}
	\EndProof
	\\
	\Theorem{DecreasingIsBorelMeasurable}
	{
		\forall D \subset \Reals \.
		\forall f : D \downarrow \Reals \.
		\forall f \in \BOR(D,\Reals)
	}
	\NoProof
}\Page{
	\Theorem{SumIsMeasurable}
	{
		\forall X \in \BOR \.
		\forall A,B \subset X \.
		\forall f : \BOR(A,X) \.
		\forall g : \BOR(B,X) \.
		f + g \in \BOR(A \cap B, X)
	}
	\Explain{
		Let $Z$ be Borel subset of $\Reals$}
	\Explain{
		Then $(+)^{-1}(Z)$ is a Bore subset of $\Reals^2$}
	\Explain{
		So, $(f \times g)^{-1}(+)^{-1}(Z)$ is a Borel subset of $A \times B$
	}
	\Explain{ Then,  $ (f+g)^{-1}(Z) = (f \times g)^{-1}(+)^{-1}(Z) \cap \Delta(A \cap B)$ 
		is measurable in $\Delta(A \cap B)$}
	\Explain{
		Associating $\Delta(A \cap B)$ with $(A \cap B)$, so $f+g$ is measurable
	}
	\EndProof
	\\
	\Theorem{ProductIsMeasurable}
	{
		\NewLine ::			
		\forall X \in \BOR \.
		\forall A,B \subset X \.
		\forall f : \BOR(A,X) \.
		\forall g : \BOR(B,X) \.
		f \cdot g \in \BOR(A \cap B, X)
	}
	\NoProof
	\\
	\Theorem{DivisionIsMeasurable}
	{
		\NewLine ::			
		\forall X \in \BOR \.
		\forall A,B \subset X \.
		\forall f : \BOR(A,X) \.
		\forall g : \BOR(B,X) \.
		\frac{f}{g} \in \BOR\Big((A \cap B) \setminus g^{-1}\{0\} , X\Big)
	}
	\NoProof
	\\
	\Theorem{DivisionIsMeasurable}
	{
		\NewLine ::			
		\forall X \in \BOR \.
		\forall A,B \subset X \.
		\forall f : \BOR(A,X) \.
		\forall g : \BOR(B,X) \.
		\frac{f}{g} \in \BOR\Big((A \cap B) \setminus g^{-1}\{0\} , X\Big)
	}
	\NoProof
	\\
	\Theorem{MinIsMeasurable}
	{
		\NewLine ::			
		\forall X \in \BOR \.
		\forall A,B \subset X \.
		\forall f : \BOR(A,X) \.
		\forall g : \BOR(B,X) \.
		\min(f,g) \in \BOR\Big(A \cap B, X\Big)
	}
	\NoProof
	\\
	\Theorem{MaxIsMeasurable}
	{
		\NewLine ::			
		\forall X \in \BOR \.
		\forall A,B \subset X \.
		\forall f : \BOR(A,X) \.
		\forall g : \BOR(B,X) \.
		\max(f,g) \in \BOR\Big(A \cap B, X\Big)
	}
	\NoProof
}\Page{
	\Theorem{InfIsMeasurable}
	{
		\forall X \in \BOR \.
		\forall A : \Nat \to 2^X \.
		\forall f : \prod^\infty_{n=1} \BOR(A_n,X) \.
		\inf f \in \BOR\left(\bigcap^\infty_{n=1} A_n,X\right)
	}
	\Explain{
		Define $g_n = \min(f_1,\ldots,f_n)$	
	}
	\Explain{ 
		Then $\lim_{n \to \infty} g_n = \inf f$ is measurable as a limit of measurable functions
	}
	\EndProof
	\\
	\Theorem{SupIsMeasurable}
	{
		\forall X \in \BOR \.
		\forall A : \Nat \to 2^X \.
		\forall f : \prod^\infty_{n=1} \BOR(A_n,X) \.
		\sup f \in \BOR\left(\bigcap^\infty_{n=1} A_n,X\right)
	}
	\NoProof
	\\
	\Theorem{limInfIsMeasurable}
	{
		\forall X \in \BOR \.
		\forall A : \Nat \to 2^X \.
		\forall f : \prod^\infty_{n=1} \BOR(A_n,X) \.
		\lim\inf f \in \BOR\left( \bigcap^\infty_{n=1} \bigcap^\infty_{n=m} A_n ,X\right)
	}
	\Explain{Use limit of $\lim_{m \to \infty} \inf_{n \in \Nat} f_{n+m}$}
	\EndProof
	\\
	\Theorem{LimSupIsMeasurable}
	{
		\forall X \in \BOR \.
		\forall A : \Nat \to 2^X \.
		\forall f : \prod^\infty_{n=1} \BOR(A_n,X) \.
		\lim\sup f \in \BOR\left(\bigcup^\infty_{m=1} \bigcap^\infty_{n=m} A_n,X\right)
	}
	\NoProof
}
\newpage
\subsubsection{Simple Function}
\Page{
	\DeclareFunc{MeasureCategory}{\CAT}
	\DefineNamedFunc{MeasureCategory}{}{\MEAS}
	{
		\Big(\MS,
			\Lambda (X,\Sigma_X,\mu), \NewLine
			(Y,\Sigma_Y,\nu) \in \MEAS \.  
				\{ \varphi \in \BOR(X,Y) : 
					\forall E \in \Sigma_Y \. 
					\nu(E) < \infty \Imply \varphi_*\mu(E) < \infty
			 \}, \NewLine
			\circ,\id\Big)
		}
	\\
	\DeclareType{FiniteMeasure}{\prod (X,\Sigma,\mu) \in \MEAS \. \Ideal(X,\Sigma)}
	\DefineNamedType{E}{FiniteMeasure}{E \in \Sigma_\mu}{\mu(E) < \infty}
	\Explain{ This follows from basic axioms of measure}
	\Explain{ $\mu(\emptyset) = 0 < \infty $}
	\Explain{ If $A \in \Sigma_\mu$ and $B \in \Sigma$, then   
		$\mu(A \cap B) \le \mu(A) < \infty$}
	\Explain{ And finally, if $A,B \in \Sigma_\mu$, then
			$\mu(A \du B) \le \mu(A \cup B) \le \mu(A) + \mu(B) < \infty$}
	\EndProof	
	\\
	\DeclareFunc{Simple}{\Contra(\BOR,\VS{\Reals})}
	\DefineNamedFunc{Simple}{X,\mu,\Sigma}{\Simple(X,\mu,\Sigma)}
	{
		\left\{ \Lambda x \in X \. \sum^n_{i=1} \alpha_i \delta_x(E_i) 
			\Bigg| 
				n \in \Int_+ , 
				\alpha : \{1,\ldots,n\} \to \Reals, 
				E_i : \{1,\ldots,n\} \to \Sigma_\mu                
		\right\}
	}
	\DefineNamedFunc{Simple}{X,Y,\varphi}{\Simple_{X,Y}(\varphi)}
	{
		\varphi^* = \Lambda f \in \Simple(Y) \.    \varphi f	
	}
	\Explain{ It is trivial to check that $\Simple(X)$ is a vector space}
	\Explain{ Next we show that $\varphi^*$ indeed maps $\Simple(Y)$ to $\Simple(X)$}
	\Explain{ Let $f(y) = \sum^m_{i=1} \alpha_i \delta_y(E_i) \in \Simple(Y)$}
	\Explain{ Then $ \varphi^* f(x) = 
		\sum^m_{i=1} \alpha_i \delta_{\varphi(x)} (E_i) =
		\sum^m_{i=1} \alpha_i \delta_x\Big( \varphi^{-1} (E_i)\Big) \in \Simple(X) $}
	\Explain{ Clearly $\varphi^{-1}(E_i) \in \Sigma_\mu$ as 
		$\mu\Big( \varphi^{-1}(E_i) \Big) =  \varphi_*  \mu(E_i) < \infty  $	
	}
	\Explain{Also the composition law holds $\varphi^* \psi^* = (\psi\varphi)^*$}
	\EndProof
	\\
	\Theorem{SimpleFunctionsAreMeasurable}
	{
		\forall (X,\Sigma,\mu) \in \MEAS \.
		\Simple (X,\Sigma,\mu) \subset \BOR\Big((X,\Sigma),\Reals\Big)
	}
	\ExplainFurther{
		Clearly $f(x) = \delta_x(E)$ is measurable,
	 	as for every Borel set $B$ we have $f^{-1}(B) = X $ if $0,1 \in B$} 
	\Explain{
	 	or $f^{-1}(B) = E$ if   $1\in B,0 \not \in B$,
		or $f^{-1}(B) = E^\c$	if  $0 \in B,1 \not \in B$, otherwise $f^{-1}(B) = \emptyset$}
	\Explain{ Thus, every simple function is a linear combination of measurable functions, hence, measurable.}
	\EndProof
}\Page{
	\Theorem{Decomposition}
	{
		\NewLine ::		
		\forall (X,\Sigma,\mu) \in \MEAS \.
		\forall n \in \Int_+ \.
		\forall E : \{1, \ldots,n\} \to \Sigma_\mu \. \NewLine \.
		\exists m : \{1,\ldots,n\} \to \Nat \ .
		\exists F : \sum^n_{k=1}\{1,\ldots,m_k\} \to \Sigma_\mu  \.	
		\TYPE{PairwiseDisjoint}(\im F)	\And \NewLine \And	
		\forall k \in \{1,\ldots,n\} \.
		E_k = \bigcup^{m_k}_{l=1} F_{k,l}
	}
	\Explain{
		Set $m_k = 2^{n-1}$}
	\Explain{
		For each $k \in \{1,\ldots,n\}$ let $(I_{k,l})^{m_k}_{l=1}$ be an enumeration of subsets
		of $\{1,\ldots,n\}$ which contain $k$}
	\Explain{ 
		Then define $F_{k,l} = \bigcap_{i \in I_{k,n}} E_i  \setminus \bigcup_{j \in I_{k,n}^\c} E_j  $}
	\Explain{ 
		It is obvious, tha each pair $F_{k,l},F_{k',l'}$ is either equal or disjoint}
	\Explain{
		Also each $F_{k,l} \in \Sigma_\mu$ as  $F_{k,l} \subset E_k$ and $\Sigma_\mu$ is an ideal.
	}
	\Explain{
		Now, assume $x \in E_k$}
	\Explain{
		Then, there is an $l \in \{1,\ldots,m_k\}$ such that 
		$I_{k,l} = \Big\{ i \in \{1,\ldots,n\} : x \in E_i  \Big\}$
	}
	\Explain{ 
		such number $l$ clearly exists as $x \in E_k$, so $K \in I_{k,l}$}
	\Explain{ 
		But, then by construction $x \in F_{k,l}$}
	\Explain{
		So, $E_k \subset \bigcup^{m_k}_{l=1} F_{k,l}$
	}
	\Explain{ But, as it was mentioned above, each set $F_{k,l}$ is a subset of $E_k$, so
		$E_k = \bigcup^{m_k}_{l=1} F_{k,l}$	
	}
	\Explain{ 
		Finally, it is possible to refine decomposition by removing empty $F_{k,l}$}
	\EndProof 
	\\
	\Theorem{DisjointRepresentation}
	{
		\NewLine ::		
		\forall (X,\Sigma,mu) \in \MEAS \.
		\forall f \in  \Simple(X,\Sigma,\mu) \. \NewLine \.
		\exists m \in \Int_+ \.
		\exists \beta : \{1,\ldots,m\} \to \Reals \.
		\exists G : \TYPE{DisjointFamily}\Big(\{1,\ldots,m\},\Sigma_\mu \Big) \.
		f(x) = \sum^m_{i=1} \beta_i \delta_x(G_i)
	}
	\Explain{ We can assert that  $f(x) = \sum^n_{i=1} \alpha_i \delta_x(E_i)$}
	\Explain{ 
		Then construct finite decomposition of $E$ as in Decomposition and enumerate it
		as $G : \{1,\ldots,m\} \to \Sigma_\mu$}
	\Explain{ Then $f(x) = \sum^n_{i=1} \alpha_i \delta_x(E_i)  = 
		\sum^n_{i=1} \alpha_i \sum^m_{j=1}  [\exists E_i \cap G_j]\delta_x(G_j)$  
		as $G_j$ are disjoint}
	\Explain{
		Then recompute $\beta$ by basic rules of algebra.
		Namely, $\beta_i = \sum^n_{j=1} \alpha_j [\exists E_j \cap G_i] $
	}
	\EndProof
}\Page{
		\Theorem{PositiveEvaluation}
		{
			\NewLine ::		
			\forall (X,\Sigma,\mu) \in \MEAS \.
			\forall f(x) =\sum^n_{i=1} \alpha_i \delta_x(E_i) \in  \Simple(X,\Sigma,\mu) \. 
			\forall \aleph : \forall x \in X \. f(x) \ge 0 \.
			\sum^n_{i=1} \alpha_i \mu(E_i) \ge 0
		}
		\Explain{ 
			Construct disjoint representation $f(x) = \sum^m_{i=1} \beta_i \delta_x(G_i)$}
		\Explain{ 
			Then from $\aleph$ it follows that for each $\forall i \in \{1,\ldots,m\} \. \beta_i \ge 0$}
		\Explain{
			But then clearly $0 \le \sum^m_{i=1} \beta_i \mu(G_i) = \sum^n_{j=1} \alpha_j \mu(E_j)$
			by non-negativity of measure}
		\Explain{ Here we used the expression for $\beta$ again and additivity of $\mu$}
		\EndProof
		\\
		\Theorem{UniqueEvaluation}
		{
			\NewLine ::			
			\forall (X,\Sigma,\mu) \in \MEAS \.
			\forall 
			f(x) =\sum^n_{i=1} \alpha_i \delta_x(E_i) =  \sum^m_{i=1} \beta_i \delta_x(G_i)
			\in \Simple(X,\Sigma,\mu) \.
			\sum^n_{i=1} \alpha_i \mu(E_i)  = \sum^m_{i=1} \beta_i \mu(G_i)
		}
		\Explain{ 
			Clearly, $ \forall x \in X \. 
			\sum^n_{i=1} \alpha_i \delta_x(E_i) -  \sum^m_{i=1} \beta_i \delta_x(G_i) = 0 \ge 0 $}
		\Explain{ 
			So, $\sum^n_{i=1} \alpha_i \mu(E_i)  - \sum^m_{i=1} \beta_i \mu(G_i) \ge 0$ }
		\Explain{
			On the other hand $ \forall x \in X \. 
		\sum^n_{i=1} \alpha_i \delta_x(E_i) -  \sum^m_{i=1} \beta_i \delta_x(G_i) = 0 \le 0 $}
		\Explain{ 
			So, $\sum^n_{i=1} \alpha_i \mu(E_i)  - \sum^m_{i=1} \beta_i \mu(G_i) \le 0$ }
		\Explain{
			But this mean that $\sum^n_{i=1} \alpha_i \mu(E_i)  - \sum^m_{i=1} \beta_i \mu(G_i) = 0$}
		\Explain{ Thus, the equality holds}
		\EndProof
		\\
		\DeclareFunc{simpleIntegral}{\prod (X,\Sigma,\mu) \in \MEAS \. 
			\VS{\Reals}\Big(\Simple(X,\Sigma,\mu), \Reals\Big)}
		\DefineNamedFunc{simpleIntegral}{\sum^n_{i=1} \alpha_i \delta(E_i)}
		{
			\int_X  \sum^n_{i=1} \alpha_i \delta_x(E_i) \; d\mu(x)
		}
		{
			\sum^n_{i=1} \alpha_i \mu(E_i)
		}
		\Explain{ Linearity is pretty obvious}
		\EndProof
}
\Page{
	\Theorem{SimpleIntergralMonotonicity}
	{
		\NewLine ::		
		\forall (X,\Sigma,\mu) \in \MEAS \.
		\forall f,g \in \Simple(X,\Sigma,\mu) \.
		\forall \aleph  : f \le g \.
		\int_X f(x) \; d\mu(x) \le \int_X g(x) \; d\mu(x)
	}
	\Explain{
		Assume $f(x) = \sum^n_{i=1} \alpha_i \delta_x(E_i)$ and 
		$g(x) = \sum^m_{j=1} \beta_j \delta_x(G_j)$}
	\Explain{
		Then by $\aleph$ for every $x \in X$ we have inequality
		$
			\sum^m_{j=1} \beta_j \delta_x(G_j) - 
			\sum^n_{i=1} \alpha_i \delta_x(E_i) \ge 0
		$}
	\Explain{ 
		But this means that 
		$
			\int_X g(x) \; d\mu(x) - \int_X f(x) \; d\mu(x)  =
			\  \sum^m_{i=1} \beta_i \mu(G_i) - \sum^n_{i=1} \alpha_i \mu(E_i) \ge 0 
		$}
	\Explain{
		So, the desired inequality follows}
	\EndProof
	\\
	\Theorem{SimpleIntegralLowerContinuity}
	{
		\NewLine :
		\forall (X,\sigma,\mu)  \in \MEAS\.
		\forall f : \Nat \downarrow \Simple(X,\sigma,\mu) \.
		\forall \aleph  :   \lim_{n \to \infty} f_n =_{\ae \; \mu} 0 \.
		\lim_{n \to \infty} \int_X f_n(x) \; d\mu(x) = 0
	}
	\Explain{ 
		Assume that  $\lim_{n\to \infty} \int f_n \neq 0$ }
	\Explain{
		Then, as they form a decreasing sequence there must be some number 
		$\omega > 0$ such that $\lim_{n\to \infty} \int f_n = \omega$	
	}
	\Explain{ 
		Set $\alpha = \max f_1(x)$, which exists as $f_1$ is simple}
	\ExplainFurther{ 
		Then by  integral monotonicity and some $\beta \in (0,\alpha)$ }
	\ExplainFurther{
		$	
			\omega \le \int f_n \le 
			\alpha \mu\Big( f_n^{-1}[\beta,\alpha]\Big) + \beta \mu\Big( f_n^{-1}(0,\beta) \Big)  =
			\alpha f_n^*\mu[\beta,\alpha] + 
			\beta \big( f_n^*\mu(0,\alpha ) - f_n^*\mu(\beta,\alpha) \big) \le$}  
 	\Explain{		
 		$
 			\le \alpha f_n^*\mu[\beta,\alpha] + 
			\beta \big( f_1^*\mu(0,\alpha ] - f_n^*\mu[\beta,\alpha] \big)
		$}
	\Explain{
		Which can be rewritten as 
		$
			\gamma =  \frac{\omega -   \beta f_1^*\mu(0,\alpha ]}{\alpha - \beta} \le  f_n^*\mu[\beta,\alpha]
		$}
	\Explain{
		For $\beta$ small enough the value $\gamma>0$,
		so by upper continuity of measures
		$\mu\left( \bigcap f^{-1}_n[\beta,\alpha] \right) \ge \gamma > 0$		}
	\Explain{
		Thus, the the set $E = \bigcap f^{-1}_n[\beta,\alpha] $ is nonempty with positive measure}
	\Explain{
		define $g(x) = \beta \delta_x[\beta,\alpha]$}
	\Exclaim{ 
		But $\forall n \in \Nat \.  f_n \ge g > 0$, a contradiction with $\aleph$}
	\EndProof
}\Page{
	\Theorem{PositiveAndNegativeParts}
	{
		\forall X \in \MEAS \.
		\forall f \in \Simple(X) \.
		(f)_+,(f)_- \in \Simple(X)
	}
	\Explain{ This is obvious by removing elements in disjoint representation}
	\EndProof
	\\
	\Theorem{SimpleIntegralSupIneq}
	{
		\NewLine ::		
		\forall (X,\Sigma,\mu) \in \MEAS \.
		\forall f \in \Simple(X,\Sigma,\mu) \.
		\forall g : \Nat \uparrow \Simple(X,\Sigma,\mu) \.
		\forall \aleph :  f \le_{\ae \; \mu} \sup_{n \in \Nat} g_n(x) \. \NewLine \.
		\int_X f(x) \; d\mu(x) \le \sup_{n \in \Nat} \int_X g_n(x) \; d\mu(x)	
	}
	\Explain{
		Rewrite $\aleph$ as
		$ \sup_{n \in \Nat} (g_n - f) \ge 0 $	}
	\Explain{ But this means that $(g_n - f)_- \downarrow 0$}
	\Explain{    So,   
		$ 
			\lim_{n\to \infty } \int_X \big(g_n(x) - f(x)\big) \; d\mu(x)  \ge 
			- \lim_{n\to \infty } \int_X	 (g_n - f)_-(x) \; d\mu(x)  = 0
		$}
	\Explain{
		But this means that
		$
			\lim_{n\to \infty } \int_X g_n(x) \; d\mu(x) \ge 
			\int_X f(x) \; d\mu(x)
		$
	} 
	\EndProof
}
\newpage
\subsubsection{Nonnegative Integrable Functions }
\Page{
	\Conclude{\TYPE{AlmostDefinedMeasurable}}
	{
		\Lambda (X,\mu) \in \MEAS \.
		\Lambda  Y \in \BOR \.
		\BOR_\mu(X,Y)  = \NewLine 
		\Lambda (X,\mu) \in \MEAS \.
		\Lambda  Y \in \BOR \.
		\Big\{  f \in \F_\mu(Y)  \Big|  f \in \BOR\big( (\dom f,\Sigma|\dom f),Y\big)   \Big\}
	}{
		\MEAS \to \BOR \to \VS{\Reals}
	}
	\\
	\DeclareFunc{NonNegativeWithIntegral}{
		\prod (X,\mu) \in \MEAS \. \TYPE{Cone}\Big(\BOR_\mu(X,\Reals)\Big)}
	\DefineNamedFunc{NonNegativeWithIntegral}{}{\Integrable_+(X,\mu)}
	{
		\NewLine \de		
		\Big\{ 
			f \in   \BOR_\mu(X,\Reals) : f \geq 0 \And \exists \sigma \in \Simple(X,\mu)  \.
				\forall_\mu x \in X \. \sigma(x) \uparrow f(x)    
		\Big\}
	}
	\\
	\Theorem{LebesgueIntegralUnique}
	{
		\NewLine ::		
		\forall (X,\mu) \in \MEAS \.
		\forall f \in \Integrable_+(X, \mu) \.
		\forall \sigma : \Nat \uparrow \Simple(X,\mu) \.
		\forall \aleph : \forall_\mu x \in X \.    \sigma(x) \uparrow f(x) \. \NewLine \.
		\lim_{n \to \infty}\int_X \sigma_n(x) \; d\mu(x) = 
		\sup \left\{  \int \tau  \bigg| \tau \in \Simple(X,\mu) \And \tau \le_{\ae \; \mu} f \right\} 
	}
	\Explain{
		As every $\int_X \sigma_n $ belongs to the set, clearly
		$\lim_{n \to \infty} \int \sigma_n \le 
		\sup \left\{  \int \tau  \bigg| \tau \in \Simple(X,\mu) \And \tau \le_{\ae \; \mu} f \right\} $
	}
	\Explain{ Now, assume $\tau \in \Simple(X,\mu)$ with  $\tau \le_{\ae \; \mu} f $}
	\Explain{ But $\aleph$ 
		witnesses $\tau \le_{\ae \; \mu} f \le _{\ae \; \mu} \sup_{n \in \Nat } \sigma_n $}
	\Explain{
		So, by simple integrals sup inequelity
		$ \int \tau \le \lim_{n \to \infty} \int \sigma_n$
	}
	\Explain{This mean that desired equality holds}
	\EndProof
	\\
	\DeclareFunc{integralOfLebesgue}
	{
		\prod (X,\mu) \in \MEAS \. \Integrable_+(X,\mu) \to \EReals_+
	}
	\DefineNamedFunc{integralOfLebesgue}{f}{\int_X f(x) \; d\mu(x)}	
	{
		\sup \left\{  \int \tau  \bigg| \tau \in \Simple(X,\mu) \And \tau \le_{\ae \; \mu} f \right\}
	}
	\\
	\DeclareType{NonnegativeIntegrable}
	{
		\prod (X,\mu) \in \MEAS \. ?\Integrable_+(X,\mu)
	}
	\DefineNamedType{f}{NonnegativeIntegrable}{f \in L_{1}^{+}(X,\mu)}	
	{
		\int f < \infty
	}
	\\
	\Theorem{NonnegativeIntegrableIsACone}
	{
		\forall (X,\mu) \. \TYPE{Cone}\Big( \BOR_\mu(X,\Reals), L_{1}^{+}(X,\mu) \Big)
	}
	\Explain{ Compute integrals as limits of integrals of simple functions}
	\Explain{ Then use linearity of simple integrals}
	\EndProof
	\\
	\DeclareType{VirtuallyMeasurable}
	{\prod (X,\Sigma,\mu) \in \MEAS \. \prod Y \in \BOR \.  ?\F_\mu(Y)}
	\DefineNamedType{f}{VirtuallyMeasurable}
	{
	  		f \in \BOR_\mu^*\Big((X,\Sigma), Y \Big)
	}
	{
		\exists E \subset \dom f  \cap \Null'_\mu \. 
		f_{|E} \in \BOR(E,Y)		
	}
}	
\Page{
	\Theorem
	{
		IntegrabityCondition	
	}
	{
		\forall (X,\Sigma,\mu) \in \MEAS \.
		\forall f \in \F_\mu(\Reals_+) \.
		f \in L_{1+}(X,\Sigma,\mu) \iff \NewLine \iff
		\exists E \subset \dom f \cap \Sigma \. 
		f_{|E} \in \BOR\Big((X,\Sigma),\Reals\Big)
		\And 
		\forall \alpha \in \Reals_{++} \. f_{|E*}\mu(\alpha,+\infty) < \infty
		\And \NewLine \And
		 \sup \left\{  \int \tau  \bigg| \tau \in \Simple(X,\mu) \And \tau \le_{\ae \; \mu} f \right\} < \infty
	}
	\Explain{ 
		Assume $f \in L_{1+}(X,\Sigma,\mu)$}
	\Explain{
		Then, there is a sequence $\sigma$ of simple functions $f =_{\ae\;\mu} \sigma_{n}$
	}
	\Explain{ 
		Let $E$ be the set of convergence}
	\Explain{
		As $\Reals$ are complete, the convergence is equivalent to being Cauchy}
	\Explain{ 
		So by simple set-algebraic manipulations (see Descriptive Set Theory) set $E$
		must be measurable}
	\Explain{
		Clearly sets of form $F = f^{-1}_{|E}(\alpha,+\infty)$ must have finite measure}
	\Explain{ 
		Otherwise we have $\int f = \lim_{ n \to \infty} \int \sigma_n = \infty$
	}
	\Explain{
		To see this write
		$
			\int \sigma_n \ge   
			\left(\alpha - \frac{\varepsilon}{n} \right)
			\mu\left(\sigma^{-1}_{n}\left(\alpha - \frac{\varepsilon}{n},+\infty\right)\right)
			\ge 
			\left(\alpha - \frac{\varepsilon}{n} \right)
			\mu\big(  \sigma^{-1}_{n} (\alpha,+\infty)\big)
		$}
	\ExplainFurther{ See that $E \cap  f^{-1} (\alpha,+\infty) 
		\subset \bigcup^\infty_{n=1}  \sigma_n^{-1} (\alpha,+\infty) $,}
	\Explain{ 
		as if $f(x) > \alpha$ for some $x \in E$ then there must be some $n$ for which  
		$\sigma_n(x) > \alpha + \frac{f(x)-\alpha}{2} > \alpha$}
	\Explain{
		But $\lim_{n \to \infty} \left(\alpha - \frac{\varepsilon}{n}\right) = \alpha > 0$
		and by lower continuity of measures
		$
			\lim_{n \to \infty}
			\mu\big(  \sigma^{-1}_{n} (\alpha,+\infty)\big) \ge
			\mu\big(   f_{|E}^{-1} (\alpha,+\infty) \big) = \infty
		$}
	\Explain{ the third property is trivial by definition of the integral }
	\Explain{ Now assume the three properties hold for $f \in \F_\mu$ }
	\Explain{  Define the sequence of simple function as 
		$
		\sigma_n(x) = \sum^{2^{2n}}_{k=1} \frac{k}{2^n}\delta_x
		\left( f^{-1}_{|E}\left[\frac{k}{2^n},\frac{k+1}{2^n}\right)\right)
		+ 2^n \delta_x\big( f^{-1}_{|E}[2^n, + \infty )\big)		
		$
	}
	\Explain{
		Clearly by second property each interval in this construction has finite measure
	}
	\Explain{ 
		By construction each $\sigma_n \le f$ on $E$ and the sequence is increasing. 
	}
	\Explain{ now consider some $x \in E$ and $\varepsilon \in \Reals_{++}$
	}
	\Explain{
		Then by archimedian property of reals there is som $N$ such that $2^N > f(x)$
		and $\varepsilon <  2^{-N}$
	}
	\Explain{
		This means that for all $n \ge N$ the difference $ f(x) - \sigma_n(x) < \varepsilon$}
	\Explain{ So, $\sigma_n \to f$ on $E$}
	\Explain{ The finiteness of the integral follows from the third property}
	\EndProof 
	\\
	\Theorem{FunctionsWithIntergralsAreVirtuallyMeasurable}
	{
		\forall (X,\mu) \in \MEAS \.
		\Integrable_+(X,\mu) \subset \BOR^*_\mu(X,\Reals_+)
	}
	\Explain{
		See proof above for measurable subset $E$}
	\Explain{
		Then function with the integral is measurable on $E$ as a limit of measurable functions}
}
\newpage
\subsubsection{Integrable Functions}
\Page{
	\DeclareFunc{withIntegral}{
		\prod (X,\mu) \in \MEAS \. ?\BOR_\mu^*(X,\Reals)}
	\DefineNamedFunc{withIntegral}{}{\Integrable(X,\mu)}
	{
		\Big(\Integrable_+(X,\mu) - L_{+}^1(X,\mu)\Big)
		\cap
		\Big( L_{+}^1(X,\mu) - \Integrable_+(X,\mu)\Big)
	}
	\\
	\DeclareFunc{integralOfLebesgue}{
		\prod (X,\mu) \in \MEAS \. \Integrable(X,\mu) \to \EReals}
	\DefineNamedFunc{integralOfLebesgue}{f}
	{\int_X f(x) \; d\mu(x)}
	{
		\int f_+ - \int f_-	
	}
	\\
	\DeclareFunc{Integrable}{
	 		\MEAS \to \VS{\Reals}
	}
	\DefineNamedFunc{Integrable}{}{L_1(X,\mu)}
	{
		L_{+}^1(X,\mu) - L_{+}^1(X,\mu)
	}
	\\
	\Theorem{IntegralIsFunctional}
	{
		\forall (X,\mu) \in \MEAS \.
		\int_X \bullet\; d\mu(x) \in \VS{\Reals}\Big( L^1(X,\mu), \Reals \Big)
	}
	\Explain{Express integrals by definitions and use linearity of limits}
	\EndProof
	\\
	\Theorem{IntegralAbsBound}
	{
		\forall (X,\mu) \in \MEAS \.
		\forall f \in L_1(X,\mu) \.
		\int |f| \ge \left| \int f \right|
	}
	\Explain{
		Write $\int |f| = \int f_+ + \int f_-$}
	\Explain{
		Then,  clearly, $\int f_+ + \int f_- \ge \int f_+ - \int f_-$}
	\Explain{
		Also, $\int f_+ + \int f_- \ge -\int f_+ + \int f_-$
	}
	\Explain{
		So, by definition of the absolute value and the integral the result follows
	}
	\EndProof
}
\newpage
\subsubsection{Integration over Subsets}
\Page{
	\DeclareFunc{subsetIntegration}
	{
		\prod (X,\Sigma,\mu) \in \MEAS \.
		\prod E \in \Sigma \. 
		\Integrable(E,\Sigma|E,\mu|E) \to \EReals
	}
	\DefineNamedFunc{subsetIntegration}{f}{\int_E f(x) \; d\mu(x)}
	{
		\int_E f(x) d(\mu|E)(x)	
	}
	\\
	\DeclareFunc{subsetIntegration2}
	{
		\prod (X,\Sigma,\mu) \in \MEAS \.
		\prod E \in \Sigma \. 
		\Integrable(E,\Sigma,\mu) \to \EReals
	}
	\DefineNamedFunc{subsetIntegration2}{f}{\int_E f(x) \; d\mu(x)}
	{
		\int_E f_{|E}(x) d\mu(x)	
	}
	\\
	\DeclareFunc{zeroExtension}
	{
		\prod X \in \SET \.
		\prod E \subset X \. 
		(E \to \Reals) \to (X \to \Reals)
	}
	\DefineNamedFunc{zeroExtension}{f}{f\delta(E)}
	{
		\Lambda x \in X \. \If x \in E \Then f(x) \Else 0
	}
	\\
	\Theorem{IntegrableOveSubsetByZeroExtenstion}
	{
		\forall  (X,\Sigma,\mu) \in \MEAS \.
		\forall E \in \Sigma \.
		\forall f : E \to \Reals \. \NewLine \.
		f \in \Integrable(X,\Sigma|E,\mu|E) 
		\iff
		f\delta(E) \in \Integrable(X,\Sigma|E,\mu|E)
	}
	\Explain{ Note, that if $f$ is simple then $f\delta(E)$ is simple, 
		also  if $f$ is  non-negative then $f\delta(E)$ is nonnegative}
	\ExplainFurther{
		Thes $f_{\pm}$ can be approximated by simple functioncs $\sigma$ from below}
	\Explain{
		iff  $f_{\pm} \delta(E)$ can be approximated by simple functions $\sigma \delta(E)$}
	\Explain{
		So the equivalence follows
	}
	\\
	\Theorem{IntegralOverSubsetByZeroExtenstion}
	{
		\forall  (X,\Sigma,\mu) \in \MEAS \.
		\forall E \in \Sigma \.
		\forall f : \Integrable(X,\Sigma|E,\mu|E) \. \NewLine \.
		\int_E f(x) \; d\mu(x) = \int_X f(x)\delta_x(E) \; d\mu(x)
	}
	\Explain{
		The integrals above can be computed as limits of integrals of simple functions 
	}
	\Explain{
		Note, that  $\int_E \sigma = \int_X \sigma \delta(E)$ for simple functions $\sigma$
	}
	\Explain{
		So, the limits and the integrals are equal}
	\EndProof
	\\
	\Theorem{IntegralOverSubsetByZeroExtenstion}
	{
		\forall  (X,\Sigma,\mu) \in \MEAS \.
		\forall E \in \Sigma \.
		\forall f : E \to \Reals \. \NewLine \.
		f \in L^1(E,\Sigma|E,\mu|E) \iff f\delta(E) \in L^1(E,\Sigma,\mu)
	}
	\Explain{ Obvious}
	\EndProof
}\Page{
	\Theorem{IntegralOverZeroSetIsZero}
	{
		(X,\Sigma,\mu) \in \MEAS \.
		\forall Z \in \Null_\mu \cap \Sigma
		\forall f : L_1(X,\Sigma,\mu) \. \NewLine \. 
		\int_Z f = 0
	}
	\Explain{ Clearly for a simple function $\int_Z \sigma \le \mu(Z)\sum^n_{i=1} \alpha_i = 0$}
	\Explain{ So, by defintion of the integral $\int_Z f = 0$}
	\EndProof	
	\\
	\Theorem{NonNegativeByNonNegativeIntegrals}
	{
		\forall (X,\Sigma,\mu) \in \MEAS \.
		\forall f : L_1(X,\Sigma,\mu) \. \NewLine \. 
		f \ge_{\ae\;\mu} 0 \iff \forall E \in \Sigma \. \int_E f \ge 0
	}
	\Explain{
		Clearly, $f\delta_E$ is still nonegative a. e. $\mu|E$,
		so left to right implication is almost trivial
	}
	\Explain{
		Now, assume $\forall E \in \Sigma \. \int_E f \ge 0$}
	\Explain{
		Then $H = f^{-1}(-\infty,0)$ is measurable and $f_{|H} < 0$ by construction}
	\Explain{
		But  $\int_H f \ge 0$}
	\Explain{
		This means that $\mu(H) = 0$
	}
	\EndProof
	\\
	\Theorem{ZeroByZeroIntegrals}
	{
		\forall (X,\Sigma,\mu) \in \MEAS \.
		\forall f : \Integrable(X,\Sigma,\mu) \. \NewLine \. 
		f =_{\ae\;\mu} 0 \iff \forall E \in \Sigma \. \int_E f = 0
	}
	\Explain{Apply previous theorem to $f$ and $-f$}
	\EndProof
	\\
	\Theorem{IneqByIntegrals}
	{
		\forall (X,\Sigma,\mu) \in \MEAS \.
		\forall f,g : L_1(X,\Sigma,\mu) \. \NewLine \. 
		f \ge_{\ae\;\mu} g \iff \forall E \in \Sigma \. \int_E f \ge \int_E g
	}
	\NoProof
	\\
	\Theorem{EqByIntegrals}
	{
		\forall (X,\Sigma,\mu) \in \MEAS \.
		\forall f,g : L_1(X,\Sigma,\mu) \. \NewLine \. 
		f =_{\ae\;\mu} g \iff \forall E \in \Sigma \. \int_E f = \int_E g
	}
	\NoProof
}\Page{
	\Theorem{DisjointIntegrationAsSum}
	{
		\NewLine ::		
		\forall (X,\Sigma,\mu) \in \MEAS \.
		\forall E,H : \TYPE{DisjointPair}(X,\Sigma) \
		\forall f : \Integrable(E \cup H,\Sigma,\mu)
		\int_{E \cup H} f = \int_E f + \int_H f
	}
	\Explain{
		Let $\sigma = \sum^n_{i=1} \alpha_i \delta(G_i)$ be a simple function over $E \cup H$
	}
	\Explain{
		Then by additivity of measure
		$
			\int_{E \cup H} \sigma = 
			\sum^n_{i=1} \alpha_i \mu(G_i) =
			\sum^n_{i=1} \alpha_i(G_i \cap E) + \sum^n_{i=1} \alpha_i(G_i \cap H) =
			\int_E \sigma + \int_H \sigma 
		$}
	\ExplainFurther{
		Then for a non-negative $f$ with an integral it is equal to}
	\ExplainFurther{$
			\int_{E \cup H} f = 
			\sup \left\{ 
						\int_{E \cup H} \sigma \Bigg| 
						\sigma \in \Simple(E \cup H,\Sigma,\mu), \sigma \le f  
					\right\} =
			\sup \left\{ 
						\int_{E} \sigma + \int_H \sigma \Bigg| 
						\sigma \in \Simple(E \cup H,\Sigma,\mu),
						\sigma \le f  
					\right\} =$}
			\ExplainFurther{$=\sup \left\{ 
						\int_{E} \sigma + \int_H \tau \Bigg| 
						\sigma \in \Simple(E,\Sigma,\mu),
						\tau \in \Simple(H,\Sigma,\mu),
						\sigma \le f_{|E}, \tau \le f_{|H}  
					\right\} =
		$}
		\Explain{
			$= 
			\sup \left\{ 
						\int_{E} \sigma \Bigg| 
						\sigma \in \Simple(E,\Sigma,\mu), \sigma \le f_E  
					\right\} +
			\sup \left\{ 
						\int_H \sigma \Bigg| 
						\sigma \in \Simple(H,\Sigma,\mu),
						\sigma \le f_{|H}  
					\right\} = \int_E f + \int_H f$} 
	\ExplainFurther{
		This derivation works as as there is a bijecttion of simple functions 
		$\sigma \mapsto (\sigma_{|E},\sigma_{|H}),(\sigma,\tau)\mapsto \sigma + \tau$}
	\Explain{which preserves integrals, as was shown above}
	\\
	\Theorem{InfiniteDisjointIntegrationAsSum}
	{
		\NewLine ::		
		\forall (X,\Sigma,\mu) \in \MEAS \.
		\forall E : \TYPE{DisjointSequence}(X,\Sigma) \. 
		\forall f : \Integrable\left( \bigcup^\infty_{n=1} E_n,\Sigma,\mu\right)\.
		\int_{\bigcup^\infty_{n=1} E_n,} f = \sum^\infty_{n=1} \int_{E_n} f
	}
	\Explain{
		Basically rewrite a proof above but with an infinite sum}
	\Explain{
		The only complication here is that $\sum^n_{i=1} \sigma_i$
		for a sequence of simples 
		$\sigma$ is not a simple function anymore}
	\Explain{
		However $\sum^\infty_{i=1} \sigma_i$ clearly has integral as it can be approximated
		by simples $\sum^n_{i=1} \sigma_i$
	}
	\Explain{
		And  $\sum^n_{i=1} \sigma_i < f$, the proof as above still works}
	\\
	\Theorem{ConegledgibleIntegralEquality}
	{
		\forall (X,\Sigma,\mu) \.
		\forall f \in \Integrable(X,\Sigma,\mu) \.
		\forall E \in \Sigma \cap \Null' \.
		\int_E f = \int_X f
	}
	\Explain{
		$E^\c = X\setminus E$ is measurable with measure zero}
	\Explain{
		So write, 
		$
		\int_X f = \int_{E \sqcup E^\c} f = \int_E f + \int_{E^\c} f = \int_E f
		$}
	\EndProof
	\\
	\DeclareFunc{measureByDensity}
	{
		\prod (X,\Sigma,\mu) \in \MEAS \.
		\Integrable_+(X,\Sigma,\mu) \to \Measure(X,\Sigma)	
	}
	\DefineNamedFunc{measureByDensity}{f}{\mu^f}
	{
		\Lambda E \in \Sigma \. \int_E f \; d\mu	
	}
}
\newpage
\subsubsection{Complex-Valued Integrals}
\Page{
	\Theorem{ComplexPartialDefinedRepresentation}
	{
		\forall (\Omega,\Sigma,\mu) \in \F \.
		\forall z \in \F_\mu(\Complex) \.
		\exists! x,y \in \F_\mu(\Reals) \.
		z = x + \i y
	}
	\Explain{ Use  $\Re$ and $\Im$}
	\EndProof
	\\
	\DeclareType{ComplexIntegrable}{\prod (\Omega,\Sigma,\mu) \in \MEAS \. ?\F_\mu(\Complex)}
	\DefineNamedType{z}{ComplexIntegrable}{
		z \in L^1(\Omega,\Sigma,\mu) \iff z \in \Complex \hyph L^1(\Omega,\Sigma,\mu)
	}{
				\Re z,\Im z \in L_1(\Omega,\Sigma,\mu)		
	}
	\\
	\DeclareFunc{complexIntegral}
	{
		\prod (\Omega,\Sigma,\mu)  \in \MEAS \.
		\VS{\Complex}\Big(\Complex \hyph L^1(\Omega,\Sigma,mu) ,\Complex\Big)
	}
	\DefineNamedFunc{complexIntegral}
	{z}{\int_\Omega z(\omega) \; d\mu(\omega)}
	{
		 \int_\Omega x \; d\mu + \i \int_\Omega y \; d\mu
		 \NewLine
		 \where \quad x = \Re z, y = \Im z	
	}
}
\newpage
\subsubsection{Upper and Lower Integrals}
\Page{
	\DeclareFunc{upperIntegral}
	{
		\prod (X,\mu) \in \MEAS \. \F_\mu \to \EReals
	}
	\DefineNamedFunc{upperIntegral}{f}{\overline{\int}_X f(x) \; d\mu(x)}
	{
		\inf \left\{ \int g \bigg| g \in \Integrable(X,\mu), f \le g \right\}
	}
	\\
	\DeclareFunc{lowerIntegral}
	{
		\prod (X,\mu) \in \MEAS \. \F_\mu \to \EReals
	}
	\DefineNamedFunc{lowerIntegral}{f}{\underline{\int}_X f(x) \; d\mu(x)}
	{
		\sup \left\{ \int g \bigg| g \in \Integrable(X,\mu), g \le f \right\}
	}
	\\
	\Theorem{UpperIntegralRepresentation}
	{
		\NewLine ::		
		\forall (X,\mu) \in \MEAS\.
		\forall f \in \F_\mu \.
		\forall \aleph : \overline{\int}f  d\mu < \infty \.
		\exists g \in \Integrable(X,\mu) \.
		f \le_{\ae} g \And \int g  = \overline{\int} f 
	}
	\Explain{ 
		It must be possible to choose a decreasing sequence of integrable $h$ such that
		$\int h \downarrow \overline{\int} f $ }
	\Explain{ 
		Then the sequence $-h$ is monotonic increasing and 
		$\sup_{n \in \Nat} \int -h_n \le -\overline{\int} f $}
	\Explain{
		So, by monotonic convergence theorem there exists integrable $g = \lim_{n \to \infty} h_n$
		such that $\int g = \lim_{n \to \infty}\int h_n$}
	\Explain{
		But this means that $\int g = \overline{\int} f  $
	}
	\EndProof
	\\
	\DeclareType{UpperIntegrable}{\prod \prod (X,\mu) \in \MEAS \. ?\F_\mu }
	\DefineNamedType{f}{UpperIntegrable}{f \in \mathsf{UI}(X,\mu)}{\left| \overline{\int} f \right| < \infty}
	\\
	\DeclareFunc{upperPresentation}{\prod \prod (X,\mu) \in \MEAS \. \F_\mu \to \L^1(X,\mu) }
	\DefineNamedFunc{upperPresentation}{f}{\overline{f}}{\THM{UpperIntegralPresentation}}
	\\
	\Theorem{UpperPresentationBoundIsThick}
	{
		\NewLine ::		
		\forall (X,\mu) \in \MEAS \.
		\forall f \in \F_mu \.
		\forall g \in L^1_+(X,\mu) \.
		\TYPE{Thick}\Big( X, \mu, \big\{ x \in \dom f \cap \dom g : \overline{f}(x) \le f(x) + g(x)  \big\}  \Big)
	}
}
\newpage
\subsubsection{Infinity-Valued Upper and Lower Integrals}
\newpage
\subsection{Convergence Theorems}
\subsubsection{Beppi Levi's Monotonic Convergence Theorem}
\Page{
	\Theorem{
		LeviConvergenceTheorem
	}{
		\NewLine :		
		\forall (X,\mu) : \MS \.
		\forall f : \Nat \uparrow L^1(X,\mu) \. 
		\forall \aleph : \sup_{n \in \Nat} \int f_n < \infty \.
		\exists F \in L^1(X,\mu)  \. F = \lim_{n \to \infty} f_n 
	}
	\Explain{
		Set $\alpha = \sup_{n \in \Nat} \int f_n$}
	\Explain{
			Then $\aleph$ witnesses $f^*_n\mu(\beta,+\infty) \le \frac{\alpha}{\beta} $ for every $\beta > 0$
	}
	\Explain{
	 	So, by measure monotonicy it must be the case that 
	 	$\mu\left( \bigcap^\infty_{m=1}\bigcup^\infty_{n=1} f_n^{-1}(m,+\infty)\right) 
	 	\le \frac{\alpha}{m} $ }
	\Explain{
		Taking the limit one gets
		$
			\mu\left( \bigcap^\infty_{m=1}\bigcup^\infty_{n=1} f_n^{-1}(m,+\infty)\right)  = 0
		$
	}
	\Explain{
		This means that $\forall_\mu x \in X \. \sup_{n \in \Nat} f_n(x) < \infty$}
	\Explain{ 
		So, the limit $F = \lim_{n \to \infty} f_n$ exists almost evertwhere $\mu$}
	\Explain{
		Consider $g = f_+$ and $G = F_+$}
	\Explain{
		Then $g \uparrow G$ and $\sup \int g_n < \infty$}
	\Explain{ For each $g_n$ choose $\sigma^n: \Nat \uparrow \Simple(X,\mu)$ so 
				$\sigma^n \uparrow g_n$, which is possible as all $g_n$ has integrals}
	\ExplainFurther{
		There are  only countable number of pairs of functions $\sigma^n_m$ and $g_n$,}
	\Explain{
		so where is conegledgible set $A$ there all inequelities hold}
	\Explain{ Then $\tau_n(x) \de \sup \big\{ \sigma^i_j(x) \big| i,j \in \{1,\ldots,n \}\big\}$ 
	is an increasing sequence of simple functions} 
	\Explain{
		Assume  $x \in A$ and  $\varepsilon \in \Reals_{++}$}
	\Explain{
		Then there is $n \in \Nat$ such that $G(x) - g_n(x) < \varepsilon$}
	\Explain{
		Moreover, there is an integer $m \ge n$ such that $g_n(x) - \sigma^n_m(x) < \varepsilon$}
	\Explain{
		But by construction $\sigma^n_m(x) \le \tau_m(x) \le g_m(x) \le G(x)$
	}
	\Explain{
		So, $G(x) - \tau_m(x) < 2\varepsilon$, and same is also true for integer greater when $m$}
	\Explain{
		Hence, $\lim_{n \to \infty} \tau_m = G$ on $A$, and $G$ has integral}
	\Explain{
	But, as $\tau_n \le_{ae} g_n$ we have 
	$\int G = \sup_{n \in \Nat}\int \tau_n \le \sup_{n\in \Nat}\int g_n < \infty$ }
	\Explain{
		So, $G=F_+$ is integrable}
	\Explain{
		The same reasoning  works also with $-F_-,-f_-$. So, $F$ is also integrable}
	\EndProof
}\Page{
	\Theorem{
		MonotonicConvergenceTHM
	}{
		\NewLine :		
		\forall (X,\mu) \.
		\forall f : \Nat \uparrow L^1(X,\mu) \. 
		\forall \aleph : \sup_{n \in \Nat} \int f_n < \infty \.
		\int \lim_{n \to \infty} f_n  = \lim_{n \to \infty} \int f_n
	}
	\Explain{ Let $F = \lim_{n \to \infty} f_n$ be as in the first theorem  and 
		$G = F_+, g = f_+$}
	\Explain{ 
		Then, there is a bound $\int G \le \sup_{n\in \Nat}\int g_n = \lim_{n \to \infty} \int g_n$
		as was shown above}
	\Explain{ 
		But clearly, $g_n \le_{\ae} G$ for all $n \in \Nat$, so $\lim_{n \to \infty} \int g_n \le \int G$}
	\Explain{ 
		So $\lim_{n\to \infty} \int g_n = \int G = \int F_+$, and the dual result can be proved for $F_-$}
	\Explain{
		So the desired result $\int \lim_{n \to \infty} f_n  = \lim_{n \to \infty} \int f_n$ follows}	
	\EndProof
}
\newpage
\subsubsection{Fatou's Lemma}
\Page{
	\Theorem{FatouLemma1}
	{
		\NewLine ::		 
		 \forall (X,\mu) \in \MEAS \.
		 \forall f : \Nat \to  L^1(X,\mu) \.
		 \forall \aleph : f \ge_{\ae\;\mu} 0 \.
		 \forall \beth : \sup_{n \in \Nat} \int f_n \le \infty \.  
		 {\lim\inf}_{n \to \infty} f_n \in L^1(X,\mu)
	}
	\Explain{
		Define $g_n(x) = \inf\{  f_m(x) | m \in \Nat, m \ge n  \}$ on a set $A$ with $f \ge 0$}
	\Explain{
		Then $g_n$ is increasing and $g_n \le f_n$}
	\Explain{
		So, $\sup_{n\in \Nat}\int g_n \le \sup_{n \in \Nat} \int f_n < \infty$	
	}
	\Explain{
		So, by monotonic convergence theorem 
		$ {\lim\inf}_{n \to \infty} f_n  =\lim_{n \to \infty} g_n$ is defined and integrable}
	\EndProof
	\\
	\Theorem{FatouLemma2}
	{
		 \NewLine ::		 
		 \forall (X,\mu) \in \MEAS \.
		 \forall f : \Nat \to  L^1(X,\mu) \.
		 \forall \aleph : f \ge_{\ae\;\mu} 0 \.
		 \forall \beth : \sup_{n \in \Nat} \int f_n \le \infty \.   \NewLine
		 \int {\lim\inf}_{n \to \infty} f_n \le {\lim \inf}_{n \to \infty}  \int f_n
	}
	\Explain{
		Write $ {\lim\inf}_{n \to \infty} f_n  =\lim_{n \to \infty} g_n$
	}
	\Explain{
		Then, As $g_n \le f_m$ for all $m \ge n$, 
		$
			\int g_n \le \liminf_{n \to \infty} \int f_n
		$}
	\Explain{
		So	, using monotonic convergence theorem
		$
			\int {\lim\inf}_{n \to \infty} f_n =
			\int \lim_{n \to \infty} g_n  =
			\lim_{n \to \infty}  \int g_n \le 
			\liminf_{n \to \infty} \int f_n
		$
	}
	\EndProof
}
\newpage
\subsubsection{Lebesgue's Dominated Convergence Theorem}
\Page{
	\Theorem{DominatedConvergenceTHM1}
	{
		\NewLine ::		 
		\forall (X,\mu) \in \MEAS \.
		\forall f : \Nat \to L^1(X,\mu) \.
		\forall F \in \F_\mu \.
		\forall g \in L^1_+(X,\mu) \. \NewLine \. 
		\forall \aleph : \forall_\mu \lim_{n \to \infty} f(x) = F(x) \.
		\forall \beth  : |f| \le_{\ae \; \mu} g \.   
		F \in L^1(X,\mu)
	}
	\Explain{ 
		Say $h= f_+$ and $H = F_+$, then $h \le_{\ae} g$ and $\lim_{n \to \infty} h_n =_{\ae} H$}
	\Explain{
		Note, that $H \le g$}
	\Explain{
		Also $H$ is measurable as a limit of measurable functions
	}
	\Explain{
		By integrability condition we know that 
		$ g_*\mu(\alpha,+\infty) < \infty$		
		for any $\alpha \in \Reals_{++}$
	}
	\Explain{
		Then, the same holds  for $H$}
	\Explain{
		Also the $\dom H$ is measurable in $X$ (see Descriptive Set Theory)
	}
	\Explain{
		Clearly, any simple $\sigma\le H$ has $\int \sigma \le \int g$ as $\sigma \le \H \le g$
	}
	\Explain{
		So, all conditions of integrability are satisfied for $H$, and henceforth for $F$ too}
	\EndProof
	\\
	\Theorem{DominatedConvergenceTHM2}
	{
		\NewLine ::		 
		\forall (X,\mu) \in \MEAS \.
		\forall f : \Nat \to L^1(X,\mu) \.
		\forall F \in \F_\mu \.
		\forall g \in L^1_+(X,\mu) \. \NewLine \. 
		\forall \aleph : \forall_\mu \lim_{n \to \infty} f(x) = F(x) \.
		\forall \beth  : |f| \le_{\ae \; \mu} g \.   
		\int \lim_{n \to \infty} f_n = \lim_{n \to \infty} \int f_n 
	}
	\Explain{ 
		By Fatou Lemma
		$
			\int (f + g) =
			\int \lim_{n \to \infty}	(f_n + g) \le 
			\liminf_{n \to \infty} \int (f_n + g)
	$}
	\Explain{
		As $g$ cancels out
		$
			\int f \le \liminf \int f_n
		$}
	\Explain{
		On the other hand 
		$
			\int (g - f) = 
			\int \lim_{n \to \infty}	(g - f_n) \le 
			\liminf_{n \to \infty} \int (g - f_n)
		$ }
	\Explain{
		So,we have 
		$
 			 \int  -f \le   \liminf_{n \to \infty} \int - f_n
		$ or dually
		$
			\int f \ge \limsup_{n \to \infty} \int f_n
		$}
	\Explain{
		But this means that the limit $\lim_{n \to \infty} \int f_n$ exists and equal to $\int f$
	}
	\EndProof
}\Page{
	\Theorem{DifferentiationUnderIntegralSign}
	{
		\NewLine \.		
		\forall X \in \MEAS \.
		\forall (a,b) : \TYPE{OpentInterval}(\Reals) \.
		\forall f : X\times(a,b) \to \Reals \.
		\forall g \in L^1_+(X,\mu) \. \NewLine \.
		\forall \aleph :		
		\forall t \in (a,b) \.
		f(\bullet,t ) \in L^1_+(X,\mu) \. 
		\forall \beth :
		\forall_\mu x \in X \.
		f(x,\bullet) \in \mathsf{DIFF}\Big( \Reals , \Reals, (a,b) \Big) \.	\NewLine \.
		\forall \gimel :
		\forall_\mu \in X \.
		\forall t \in (a,b) \.
		\left|\frac{\partial f}{\partial t}(x,t)\right| \le g(x) \.
		\forall t \in (a,b) \.
		\int_X \frac{\partial f}{\partial t}(x,t) \; d\mu(x) = 
		\frac{\partial}{\partial t} \int_X f(x,t) \; d\mu(x) 			
	}
	\Explain{
		Take $x \in X$ such that $\beth$ and $\gimel$ hold and $t \in (a,b)$
	}
	\Explain{
		Let $s$ a sequence in $(a,b)$ converging to $t$ bur never equall}
	\Explain{ 
		Define $f_n(x) = \frac{f(x,s_n) - f(x,t)}{s_n - t} \in L^1(X,\mu)$ }
	\Explain{ 
		Then, by mean value theorem there is a $\tau_{n,x}$ suc that
		$
			f_n(x) = \frac{\partial f}{\partial t}(x,\tau_{n,x})
		$} 
	\Explain{
		But this means that $|f_n(x)| \le g(x)$ for every $n$}
	\Explain{
		Also note, 
		that $\lim_{n \to \infty} \int f_n =  \frac{\partial}{\partial t} \int_X f(x,t) \; d\mu(x) $
	}
	\Explain{
		So, by using Monotonic convergence theorem one gets the result 
		as $\lim_{n \to \infty} f_n(x) = \frac{\partial f}{\partial t}(x,t)$
	}
	\EndProof
	\\
	\Theorem{ComplexDominatedConvergenceTheorem1}
	{
		\NewLine ::		
		\forall (X,\mu) \in \MEAS \.
		\forall f : \Nat \to \Complex \hyph L^1(X,\mu) \.
		\forall F \in \F_\mu(\Complex) \.
		\forall g \in L^1_+(X,\mu) \. \NewLine \. 
		\forall \aleph : \forall_\mu \lim_{n \to \infty} f(x) = F(x) \.
		\forall \beth  : |f| \le_{\ae \; \mu} g \.   
		F \in \Complex\hyph L^1(X,\mu)
	}
	\Explain{
		Apply dominated convegences theorem twice ot real and imaginary part with $g$ used as a dominator}
	\Explain{ We use here that $|x| = \sqrt{x^2} \le \sqrt{x^2 + y^2} = |z|$ when  $z = x + \i y$}
	\EndProof
	\\
	\Theorem{ComplexDominatedConvergenceTheorem2}
	{
		\NewLine ::		
		\forall (X,\mu) \in \MEAS \.
		\forall f : \Nat \to \Complex \hyph L^1(X,\mu) \.
		\forall F \in \F_\mu(\Complex) \.
		\forall g \in L^1_+(X,\mu) \. \NewLine \. 
		\forall \aleph : \forall_\mu \lim_{n \to \infty} f(x) = F(x) \.
		\forall \beth  : |f| \le_{\ae \; \mu} g \.   
		\lim_{n \to \infty}  \int f_n = \int \lim_{n \to \infty} f_n
	}
	\Explain{
		Basically same proof as in the last theorem}
	\EndProof
	\\
	\Theorem{ComplexDifferentiationUnderIntegralSign}
	{
		\NewLine \.		
		\forall X \in \MEAS \.
		\forall (a,b) : \TYPE{OpentInterval}(\Reals) \.
		\forall f : X\times(a,b) \to \Complex \.
		\forall g \in L^1_+(X,\mu) \. \NewLine \.
		\forall \aleph :		
		\forall t \in (a,b) \.
		f(\bullet,t ) \in L^1_+(X,\mu) \. 
		\forall \beth :
		\forall_\mu x \in X \.
		f(x,\bullet) \in \mathsf{DIFF}\Big( \Reals , \Complex, (a,b) \Big) \.	\NewLine \.
		\forall \gimel :
		\forall_\mu \in X \.
		\forall t \in (a,b) \.
		\left|\frac{\partial f}{\partial t}(x,t)\right| \le g(x) \.
		\forall t \in (a,b) \.
		\int_X \frac{\partial f}{\partial t}(x,t) \; d\mu(x) = 
		\frac{\partial}{\partial t} \int_X f(x,t) \; d\mu(x) 			
	}
	\Explain{ Same proof, complex version}
	\EndProof
}
\newpage
\subsubsection{Egoroffs Theorem}
\Page{
	\Theorem{EgoroffsTHM}{
		\NewLine ::		
		\forall (X,\Sigma,\mu) \in \MEAS \.
		\forall f : \Nat \to L_1(X,\Sigma,\mu) \.
		\forall F \in L_1(X,\Sigma,\mu) \. \NewLine \.
		\forall \aleph : \forall n \in \Nat \. \dom f_n \in \Sigma \.
		\forall \beth : \lim_{n \to \infty} =_\ae F \.	
		\forall \gimel : \mu(X) < \infty \.
		\forall \varepsilon \in \Reals_{++} \. \NewLine \.
		\exists E \in \Sigma \.
		\mu(E^\c) \le \varepsilon \And
		f_{|E}  \rightrightarrows F_{|E}
 	}
 	\Say{[1]}{\Elim \beth}
 	{
 		\forall \delta \in \Reals_{++} \. 
 		\mu\left( 
 		\bigcap^\infty_{k=1} 
 		\bigcup^\infty_{n=k} \Big\{ x \in \dom F : \big|f_n(x) - F(x)\big|\ge\delta  \Big\} \right)
 		= \mu(\emptyset) = 0
 	}
 	\Say{\Big(m,[2]\Big)}{\Lambda n \in \Nat \. [1](n^{-1}) \THM{LowerContinuity}(\mu,\varepsilon)}
 	{
		\NewLine : 		
 		\sum m : \Nat \to \Nat \.
 		\forall t \in \Nat \.
 		\mu\left( 
 			\bigcap^{m_t}_{k=1}
 			\bigcup^{\infty}_{n=k} 
 			\Big\{ x \in \dom F : \big|f_n(x) - F(x)\big|\ge t^{-1}  \Big\} 
 		\right) < 2^{-t}\varepsilon
 	}
 	\SayIn{E}
 	{
 		\left(
		\bigcup^\infty_{t=1} 		
 		\bigcap^{m_t}_{k=1} 
 		\bigcup^{\infty}_{n=k} \Big\{ x \in \dom F : \big|f_n(x) - F(x)\big| \ge t^{-1}  \Big\}
 		\right)^\c
 	}{\Sigma}
 	\Say{[0]}{\Elim E}
 	{
 		E = \bigcap^\infty_{t=1} \bigcup^{m_t}_{k=1} 
 		\bigcap^{\infty}_{n=k} \Big\{ x \in \dom F : \big|f_n(x) - F(x)\big| < t^{-1}  \Big\}
 	}
 	\Say{[3]}{\Elim E \THM{Subadditivity}(\mu)}
 	{
 		\mu(E^\c) \le \varepsilon
 	}
 	\AssumeIn{\delta}{\Reals_{++}}
 	\Say{\Big(n,[4]\Big)}{\Elim \TYPE{Archimedean}(\Reals)}
 	{	
 		\sum n \in \Nat \. \frac{1}{n} < \delta
 	}
 	\AssumeIn{t}{\Nat}
 	\Assume{[5]}{t > m_n}
 	\AssumeIn{x}{E}
 	\Conclude{[\delta.*]}{[0][5][4]}
 	{
 		\big| f_t(x) - F(x) \big| < \frac{1}{n} < \delta
 	}
 	\DeriveConclude{[*]}{\Intro \rightrightarrows}
 	{
 		f_{|E}  \rightrightarrows F_{|E}
 	}
 	\EndProof
}
\newpage
\subsection{Lower and Upper Integrals}
\subsubsection{Subject}
\Page{
	\DeclareFunc{upperIntegral}
	{
		\prod (X,\mu) \in \MEAS \. \F_\mu \to \EReals
	}
	\DefineNamedFunc{upperIntegral}{f}{\overline{\int}_X f(x) \; d\mu(x)}
	{
		\inf \left\{ \int g \bigg| g \in \Integrable(X,\mu), f \le g \right\}
	}
	\\
	\DeclareFunc{lowerIntegral}
	{
		\prod (X,\mu) \in \MEAS \. \F_\mu \to \EReals
	}
	\DefineNamedFunc{lowerIntegral}{f}{\underline{\int}_X f(x) \; d\mu(x)}
	{
		\sup \left\{ \int g \bigg| g \in \Integrable(X,\mu), g \le f \right\}
	}
	\\
	\Theorem{UpperIntegralRepresentation}
	{
		\NewLine ::		
		\forall (X,\mu) \in \MEAS\.
		\forall f \in \F_\mu \.
		\forall \aleph : \overline{\int}f  d\mu < \infty \.
		\exists g \in \Integrable(X,\mu) \.
		f \le_{\ae} g \And \int g  = \overline{\int} f 
	}
	\Explain{ 
		It must be possible to choose a decreasing sequence of integrable $h$ such that
		$\int h \downarrow \overline{\int} f $ }
	\Explain{ 
		Then the sequence $-h$ is monotonic increasing and 
		$\sup_{n \in \Nat} \int -h_n \le -\overline{\int} f $}
	\Explain{
		So, by monotonic convergence theorem there exists integrable $g = \lim_{n \to \infty} h_n$
		such that $\int g = \lim_{n \to \infty}\int h_n$}
	\Explain{
		But this means that $\int g = \overline{\int} f  $
	}
	\EndProof
	\\
	\DeclareType{UpperIntegrable}{ \prod (X,\mu) \in \MEAS \. ?\F_\mu }
	\DefineNamedType{f}{UpperIntegrable}{f \in \mathsf{UI}(X,\mu)}{\left| \overline{\int} f \right| < \infty}
	\\
	\DeclareFunc{upperPresentation}{ \prod (X,\mu) \in \MEAS \. \mathsf{UI}(X,\mu) \to \L^1(X,\mu) }
	\DefineNamedFunc{upperPresentation}{f}{\overline{f}}{\THM{UpperIntegralPresentation}}
}\Page{
	\Theorem{UpperPresentationBoundIsThick}
	{
		\NewLine ::		
		\forall (X,\mu) \in \MEAS \.
		\forall f \in \mathsf{UI}(X,\mu) \.
		\forall g \in L^1_+(X,\mu) \.
		\forall \aleph : g >_{\ae} 0 \. \NewLine \.
		\TYPE{Thick}\Big( X, \mu, \big\{ x \in \dom f \cap \dom g : \overline{f}(x) \le f(x) + g(x)  \big\}  \Big)
	}
	\SayIn{A}{\big\{ x \in \dom f \cap \dom g : \overline{f}(x) \le f(x) + g(x)  \big\} }{2^X}
	\AssumeIn{E}{\Sigma}
	\Assume{[1]}{\mu^*(E \cap A) \neq \mu(E)}
	\Say{[2]}{\Elim \mu^* [1]}{ \mu^*(E \cap A) < \mu(E)}
	\Say{\Big(F,[3]\Big)}{\Elim \mu^* [2] \Elim A}
	{
		\sum F \in \Sigma \. \mu(F) > 0 \And \forall x \in F \. \overline{f}(x) > f(x) + g(x)
	}
	\SayIn{h}{(\overline{f}-g)\delta(F) + \overline{f}\delta\Big(F^\c\Big)}
	{
		L^1(X,\mu)
	}
	\Say{[4]}{\Elim h[3]}
	{
		  f(x) \le h(x) \le \overline{f}(x)	
	}
	\Say{[5]}{\Elim \aleph}{F = \bigcup^\infty_{n=1} F \cap g^{-1}(n^{-1},+\infty)}
	\Say{[6]}{[2.1][5]\THM{LowerContinuity}(X,\mu)}
	{
			0 < \mu(F) = 
			\mu\left( \bigcup^\infty_{n=1} F \cap g^{-1}(n^{-1},+\infty) \right) =
			\lim_{n \to \infty} \mu\Big(F \cap g^{-1}(n^{-1},+\infty) \Big) 	
	}
	\Say{[7]}{\THM{PositiveLimit}[6]}
	{
		\exists n \in \Nat \.  \lim_{n \to \infty} \mu\Big(F \cap g^{-1}(n^{-1},+\infty) \Big) > 0
	}
	\Say{[8]}{[7]\Intro \int}{\int_F g > 0}
	\Say{[9]}{\Elim \overline{f} \Elim \FUNC{upperIntegral}(f)[4]\Elim h [8]}
	{
		\int \overline{f} = 
		\overline{\int} f \le 
		\int h < \int \overline{f}
	}
	\Conclude{[1.*]}{\THM{TrichtomyPrinciple}[9]\THM{EquivalenceLaw}\left(\int \overline{f}\right)}
	{
		\bot
	}
	\DeriveConclude{[*]}{\Intro \TYPE{Thick}}{\TYPE{Thick}(X,\Sigma,\mu,A)}
	\EndProof
	\\
	\\
	\Theorem{LowerIntegralRepresentation}
	{
		\NewLine ::		
		\forall (X,\mu) \in \MEAS\.
		\forall f \in \F_\mu \.
		\forall \aleph : \underline{\int}f  d\mu < \infty \.
		\exists g \in \Integrable(X,\mu) \.
		f \ge_{\ae} g \And \int g  = \underline{\int} f 
	}
	\Explain{  True by duallity}
	\\
	\DeclareType{LowerIntegrable}{ \prod (X,\mu) \in \MEAS \. ?\F_\mu }
	\DefineNamedType{f}{LowerIntegrable}{f \in \mathsf{LI}(X,\mu)}{\left| \underline{\int} f \right| < \infty}
	\\
	\DeclareFunc{lowerPresentation}{ \prod (X,\mu) \in \MEAS \. \mathsf{LI}(X,\mu) \to \L^1(X,\mu) }
	\DefineNamedFunc{lowerPresentation}{f}{\underline{f}}{\THM{LowerIntegralPresentation}(f)}
}\Page{
	\Theorem{LowerPresentationBoundIsThick}
	{
		\NewLine ::		
		\forall (X,\mu) \in \MEAS \.
		\forall f \in \mathsf{LI}(X,\mu) \.
		\forall g \in L^1_+(X,\mu) \.
		\forall \aleph : g >_{\ae} 0 \. \NewLine \.
		\TYPE{Thick}\Big( X, \mu, \big\{ x \in \dom f \cap \dom g : \underline{f}(x) \ge f(x) - g(x)  \big\}  \Big)
	}
	\Explain{ True by duallity}
	\EndProof
	\\
	\Theorem{LowerUpperBound}
	{
		\forall (X,\mu) \in \MEAS \.
		\forall f \in \F_\mu \.
		\underline{\int} f \le \overline{\int} f
	}
	\Explain{ Obvious}
	\EndProof
	\\
	\Theorem{UpperIntegralSubbaditivity}
	{
		\NewLine ::		
		\forall (X,\mu) \in \MEAS \.
		\forall f,g \in \F_\mu \. 
		\forall \aleph : \left(\overline{\int} f,\overline{\int} g\right) \not \in \{(-\infty,+\infty),(+\infty,-\infty)\} \. 
		\overline{\int} f + g \le \overline{\int} f + \overline{\int} g
	}
	\Explain{ 
		If either $\overline{\int} f$ or $\overline{\int} g$ is infinite, then inequality holds trivially}
	\Explain{
		Otherwise, $\overline{f} + \overline{g} \ge_{ae} f + g$, so
		$
			\int (f + g) \le \int (\overline{f} + \overline{g}) \le 
			\int \overline{f} + \int \overline{g} = \int f + \int g
		$}
	\EndProof
	\\
	\Theorem{UpperIntegralPositiveHomogenity}
	{
		\forall (X,\Sigma,\mu) \in \MEAS \.
		\forall f \in \F_\mu \.
		\forall \alpha \in \Reals_{++} \.
		\overline{\int} \alpha f = \alpha \overline{f} 
	}
	\Explain{
		If one integral infinite then the other also is infinite}
	\Explain{
		Otherwise, consider upper representations $\overline{f}$ and $\overline{\alpha f}$}	
	\Explain{
		Assume $\alpha\overline{f} \neq \overline{\alpha f}$
	}
	\Explain{
		Then, as $\alpha \overline{f} \ge \alpha f$, 
			$\alpha \overline{\int} f = \alpha \int \overline{f} = \int \alpha \overline{f} \ge  
							\overline{\int} \alpha f = \int \overline{\alpha f}
			$}
	\Explain{
		But by trichtomy principle this means that $\int \alpha \overline{f} > \int \overline{\alpha f}$}
	\Explain{
		But, as $\frac{1}{\alpha} \overline{\alpha f} \ge_{\ae} f$ we have 
		$\overline{\int} f < \int \overline{f}$}
	\Explain{
		But this contradicts the definition of upper representation}
	\EndProof
}\Page{
	\Theorem{UpperLowerInversion}
	{
		\forall  (X,\mu) \in \MEAS \.
		\forall f \in F_\mu \.
		\overline{\int} -f =  -\underline{\int} f
	}
	\Explain{ 
		Use duality of inf and sup in definitons}
	\ExplainFurther{
		$
			\overline{\int} -f = 
			\inf \left\{ \int g \bigg| g \in \Integrable(X,\mu), -f \le g   \right\} = 
			\inf \left\{ \int g \bigg| g \in \Integrable(X,\mu), f \ge -g   \right\} =$}
	\Explain{
			$	
			\inf \left\{ -\int g \bigg| g \in \Integrable(X,\mu), f \ge g   \right\} =
			- \sup \left\{ \int g \bigg| g \in \Integrable(X,\mu), f \ge g   \right\} =
			-\underline{\int}  g
		$}
	\EndProof
	\\
	\Theorem{UpperIntegralSupadditivity}
	{
		\NewLine ::		
		\forall (X,\mu) \in \MEAS \.
		\forall f,g \in \F_\mu \.
		\forall \aleph : \left(\underline{\int} f,\underline{\int} g\right) \not \in \{(-\infty,+\infty),(+\infty,-\infty)\} \. 
		\underline{\int} f + g \ge \underline{\int} f + \underline{\int} g
	}
	\Explain{ 
		Combine inversion result and subadditivity for upper integral
	}
	\EndProof
	\\
	\Theorem{LowerIntegralPositiveHomogenity}
	{
		\forall (X,\Sigma,\mu) \in \MEAS \.
		\forall f \in \F_\mu \.
		\forall \alpha \in \Reals_{++} \.
		\underline{\int} \alpha f = \alpha \underline{\int} f 
	}
	\Explain{
		Combine positive homogenity of upper integral and the inversion result}
	\EndProof
}
\newpage
\subsubsection{Convergence Theorems}
\Page{
	\Theorem{MonotonicConvergenceTHM1}
	{
		\forall (X,\Sigma,\mu) \in \MEAS \.
		\forall f : \Nat \uparrow_{\ae \mu} \F_\mu \. \NewLine \.
		\forall \aleph :  -\infty < \sup_{n \in \Nat} \overline{\int} f_n < \infty \.
		-\infty <_{\ae \mu} \sup_{n \in \Nat}  f_n <_{\ae \mu} +\infty 
	}
	\Explain{
		$\aleph$ witnesses that $ -\infty < \sup_{n \in \Nat} \overline{\int} f_n$}
	\Explain{
		So, $\mu^*\Big( f^{-1}_n(-\infty) \Big) = 0$ starting from som $n$}
	\Explain{ 
		Thus, $-\infty <_{\ae \mu} \sup_{n \in \Nat}  f_n $}
	\Explain{
		As $\infty \ge_{\ae {\mu}} \sup_{n \in \Nat} \overline{f}_n \ge \sup_{n \in \Nat} f_n$ and is integrable,  
		so $\sup_{n \in \Nat}  f_n \le_{\ae \mu} + \infty$}
	\EndProof
	\\
	\Theorem{MonotonicConvergenceTHM2}
	{
		\forall (X,\Sigma,\mu) \in \MEAS \.
		\forall f : \Nat \uparrow_{\ae \mu} \F_\mu \. \NewLine \.
		\forall \aleph :  -\infty < \sup_{n \in \Nat} \overline{\int} f_n < \infty \.
		\overline{\int} \sup_{n \in \Nat}  f_n  =  \overline{\int} f_n  \.
	}
	\Explain{ 
		Note, that  $\overline{f}_n$ also must be monotonically increasing almost everywhere}
	\ExplainFurther{
		As it was pointed out $\sup_{n \in \Nat} \overline{f}_n$ is integrable by classical monotonic convergence 
		theorem, so   
	}
	\Explain{
		$
			\overline{\int} \sup_{n \in \Nat} f_n \le 
			\overline{\int} \sup_{n \in \Nat} \overline{f}_n = 
			\int \sup_{n \in \Nat} \overline{f}_n =
			\sup_{n \in \Nat} \int \overline{f}_n =
			\sup_{n \in \Nat} \overline{\int} f_n 
		$
	}
	\Explain{
		Moreover, now $\aleph$ witnesses that
		$\overline{\int} \sup_{n \in \Nat} f_n < \infty$
	}
	\Explain{
		So, we can use function $g = \overline{\sup_{n \in \Nat} f_n} \ge \sup_{n \in \Nat} f_n \ge f_n$}	
	\Explain{
		which means in case of integral that $\forall n \in \Nat \. \int g \ge \overline{\int} f_n$}	
	\Explain{
		Hence,
		$
			\overline{\int} \sup_{n \in \Nat} f_n = 
			\int g \ge \sup \overline{\int} f_n
		$}
	\Explain{
		This proves equality}
	\EndProof
}\Page{
	\Theorem{FatousLemma1}
	{
		\forall (X,\Sigma,\mu) \in \MEAS \.
		\forall f : \Nat \to \F_\mu \. \NewLine \.
		\forall \aleph :  \forall n \in \Nat \. f_n \ge_{\ae \mu} 0 \.
		\forall \beth : \liminf_{n \to \infty} \overline{\int} f_n < \infty \.
		-\infty <_{\ae \mu} \liminf_{n \in \Nat}  f_n <_{\ae \mu} +\infty 
	}
	\Explain{ fo every $x$ in the domain of definition 
		the set $\{ f_n(x) | n \ge N  \}$ is bounded from below by $0$, so the inf exists}
	\Explain{ $y_m = \inf \{ f_n(x) | n \ge N  \}$ is an increasing sequence}
	\Explain{
		So, $\liminf_{n \to \infty} f_n$ is defined with codomain $[0,+\infty]$
	}
	\Explain{ 
		Now consider $\liminf_{n \to \infty} f_n$ to be a limit of functions $g_n$ defined as $y_m$}
	\Explain{
		Then, $\overline{\int} g_n \le \overline{\int} f_n$ for each $m \ge n$}
	\Explain{
		So,   $\sup_{n \in \Nat} \overline{\int} g_n \le \liminf_{n \to \infty} \overline{\int} f_n < \infty$
	}
	\Explain{ Thus, by monotonic convergence theorem $\liminf_{n \to \infty} f_n \le_{\ae} \infty$}
	\EndProof
	\\
	\Theorem{FatousLemma2}
	{
		\forall (X,\Sigma,\mu) \in \MEAS \.
		\forall f : \Nat \to \F_\mu \. \NewLine \.
		\forall \aleph :  \forall n \in \Nat \. f_n \ge_{\ae \mu} 0 \.
		\forall \beth : \liminf_{n \to \infty} \overline{\int} f_n < \infty \.
		\overline{\int} \liminf_{n \to \infty} f_n \le \liminf_{n \to \infty} \overline{\int} f_n
	}
	\Explain{ This was shown above}
	\EndProof
}
\newpage
\subsubsection{Measurable Distributivity}
\Page{
	\Theorem{MeasurableDistributivity}
	{
		\NewLine ::		
		\forall (X,\Sigma,\mu) \in \MEAS \.
		\forall f \in \F_\mu \.
		\forall h,h' \in \Integrable_+\Big( X,\Sigma \Big)
		\forall \aleph :  \infty - \infty = \infty \. \NewLine \.  
		\overline{\int} f(h + h') \; d\mu = \overline{\int} fh \; d \mu + \overline{\int} fh' \; d \mu
	}
	\Explain{
		I will use the fact that virtually measurable functions have integrals}
	\Explain{
		Define measures $\nu = h\; d\mu$ and $\nu' = h'\; d\mu$}
	\Explain{
		From positivity it follows tha every simple function for $\nu + \nu'$ 
		is simple for $\nu$ and $\nu'$, 
	}
	\Explain{
		And so every function with integral for $\nu + \nu'$ is
		has an integral for $\nu$ and $\nu'$}
	\ExplainFurther{
		Then,
		$
			\overline{\int} f(h + h') \; d\mu =
			\overline{\int} f \; d(\nu + \nu') =
			\inf \left\{  
				\int g \; d(\nu + \nu') 
				\bigg| 
				g \in \Integrable(\nu + \nu'),
				f \le g    
			\right\} =$}
	\ExplainFurther{
		$=\inf \left\{  
				\int g \; d\nu + \int g \; d\nu' 
				\bigg| 
				g \in \Integrable(\nu + \nu'),
				f \le g    
			\right\} \geq$}
	\ExplainFurther{ 
			$ \geq \inf \left\{  
				\int g \; d\nu  
				\bigg| 
				g \in \Integrable(\nu),
				f \leq g    
			\right\} +
			\inf \left\{  
				\int g \; d\nu' 
				\bigg| 
				g \in \Integrable( \nu'),
				f \leq g    
			\right\} 
			=
			\overline{\int} f \; d\nu +
			\overline{\int} f \; d\nu' =
			\overline{\int} f h \; d\mu +
			\overline{\int} f h' \; d\mu'
		$}
	\Explain{ 
		On the other hand
		$
			\overline{\int} f(h + h') \; d\mu \le  \overline{\int} f h \; d\mu +
			\overline{\int} f h' \; d\mu'
		$}
	\Explain{ 
		Hence,
		$
			\overline{\int} f(h + h') \; d\mu =  \overline{\int} f h \; d\mu +
			\overline{\int} f h' \; d\mu'
		$}
	\EndProof
}
\newpage
\section{Generalities}
\subsection{Types of Measures}
\subsubsection{Definitions}
\Page{
	\DeclareType{Probability}
	{
		? \MEAS
	}
	\DefineType{(\Omega,\Sigma,P)}{Probability}{P(\Omega) = 1}
	\\
	\DeclareType{Finite}
	{
		? \MEAS
	}
	\DefineType{(\Omega,\Sigma,\mu)}{Finite}{\mu(\Omega) < \infty}
	\\
	\DeclareType{\sFinite}
	{
		? \MEAS
	}
	\DefineType{(\Omega,\Sigma,\mu)}{\sFinite}{
		\exists E : \Nat \to \Sigma \. 
		\Big(\forall n \in \Nat \. \mu(E_n) < \infty\Big)
		\And
		\Omega = \bigcup^\infty_{n=1} E_n
	}
	\\
	\Theorem{SigmaFiniteDisjointDecomposition}
	{
		\NewLine :		
		\forall (\Omega,\Sigma,\mu) : \sFinite \.
		\exists F : \TYPE{DisjointSequence}(\Omega,\Sigma) \.
		\Big(\forall n \in \Nat \. \mu(F_n) < \infty\Big)
		\And
		\Omega = \bigcup^\infty_{n=1} F_n
	}
	\Explain{
		Take $E$ as in definition above}
	\Explain{
		Then define $F_n = E_n \setminus \bigcup^{n-1}_{k=1} E_n $	
	}
	\Explain{
		By monotonicity $\mu(F_n) \le \mu(E_n) < \infty$}
	\Explain{
		For every $\omega \in \Omega$ there are least $n$ such that $\omega \in E_n$, 
		but then $\omega \in F_n$}
	\EndProof
	\\
	\Theorem{SigmaFiniteIncreasingDecomposition}
	{
		\NewLine :		
		\forall (\Omega,\Sigma,\mu) : \sFinite \.
		\exists H : \Nat \uparrow \Sigma \.
		\Big(\forall n \in \Nat \. \mu(H_n) < \infty\Big)
		\And
		\Omega = \bigcup^\infty_{n=1} H_n
	}
	\Explain{ 
		Take $F$ as in the statement above}
	\Explain{
		Define $H_n = \bigcup^n_{k=1} F_k$}
	\Explain{
		Then $\mu(H_n) = \sum^n_{k=1} \mu(F_k) < \infty$}
	\EndProof
}\Page{
	\DeclareType{Decomposition}{\prod (X,\Sigma,\mu) \in \MEAS \. ?\TYPE{PairwiseDisjoint}(\Sigma)}
	\DefineType{\mathcal{E}}{Decomposition}
	{
		\Sigma = \{ A \subset X : \forall E \in \mathcal{E} \. A \cap E \in \Sigma   \}
		\And
		\forall A \in \Sigma \. 
		\mu(A) = \sum_{E \in \mathcal{E}} \mu(A \cap E)
		\And \NewLine \And
		\forall E \in \mathcal{E} \. \mu(E) < \infty
	}
	\\
	\DeclareType{StrictlyLocalizable}{?\MEAS}
	\DefineType{(X,\Sigma,\mu)}{StrictlyLocalizable}{\exists \Decomposition(X,\Sigma,\mu)} 
	\\
	\DeclareType{Semifinite}{?\MEAS}
	\DefineType{(X,\Sigma,\mu)}{Semifinite}
	{
		\forall E \in \Sigma \. 
		\forall \aleph : \mu(E) = \infty \.
		\exists F \in \Sigma :
		F \subset E \And  0 < \mu(F) < \infty	
	}
	\\
	\DeclareType{EssentialSupremum}{\prod (X,\Sigma,\mu) \in \MEAS \. ?\Sigma \to ?\Sigma}
	\DefineNamedType{H}{EssentialSupremum}
	{
		\Lambda \E \subset \Sigma \. H \in \esssup \E	
	}
	{
		\NewLine \iff		
		\Big( \forall E \in \E \. \mu(E \setminus H) = 0 \Big)
		\And
		\forall G \in \Sigma \. 
		\forall \aleph : \forall E \in \E \. \mu(G \setminus E) = 0 \.
		\mu(H \setminus G) = 0
	}
	\\
	\DeclareType{\Loc}{?\Semifinite}
	\DefineType{(X,\Sigma,\mu)}{\Loc}
	{
		\forall \E \subset \Sigma \. 
		\exists \esssup \E
	}
	\\
	\DeclareType{\LocDet}{?\Semifinite}
	\DefineType{(X,\Sigma,\mu)}{\LocDet}
	{
		\Sigma = \{
			E \subset X  :
			\forall F \in \Sigma \.
			\mu(F) < \infty \Imply E \cap F \in \Sigma  
		\}
	}
	\\
	\DeclareType{Atom}{\prod (X,\Sigma,\mu) \in \MEAS \. ?\Sigma}
	\DefineNamedType{A}{Atom}{A\in \Atom(X,\Sigma,\mu)}
	{
		\mu(A) > 0 
		\And \forall B \in \Sigma \. 
		\forall \aleph : B \subset A \.
		\mu(B) = 0 | \mu(A \setminus B) = 0
	}
	\\
	\DeclareType{\Aless}{?\MEAS}
	\DefineType{X}{\Aless}{\Atom(X) = \emptyset}
	\\
	\DeclareType{\PA}{?\MEAS}
	\DefineType{(X,\Sigma,\mu)}{\PA}{
		\forall E \in \Sigma \.
		\forall \aleph : \mu(E) > 0 \.  
		\exists A \in \Atom(X,\Sigma,\mu) \.
		A \subset E
	}
	\\
	\DeclareType{\PtSupp}{?\MEAS}
	\DefineType{(X,\Sigma,\mu)}{\PtSupp}{
		\Sigma = 2^X 
		\And
		\forall E \in \Sigma \.
		\mu(E) = \sum_{x \in E} \mu\{x\}
	}
	\\
	\Theorem{PointSupportedIsPurelyAtomic}
	{
		\forall (X,\Sigma,\mu) : \PtSupp \.
		\PA(X,\Sigma,\mu)
	}
	\Explain{
		Assume $E \in \Sigma$ such that $\mu(E) > 0$}
	\Explain{
		Then as $\mu$ is point supported there must be
		some $x \in E$ such that $\mu\{x\} > 0$}
	\Explain{ 
		But then $\{x\}$ trivially is an atom}
	\EndProof
}
\newpage
\subsubsection{Degrees of Finiteness}
\Page{
	\Theorem{ProbabilityIsFinite}
	{
		\forall (\Omega,\Sigma,P) : \Probability \.
		\Finite(\Omega,\Sigma,P)
	}
	\Explain{ 
		$P(\Omega) = 1 < \infty $, This is obvious}
	\EndProof
	\\
	\Theorem{FiniteIsSigmaFinite}
	{
		\forall (X,\Sigma,\mu) : \Finite \.
		\sFinite(X,\Sigma,\mu)
	}
	\Explain{ 
		Take $E_n = X$, This is obvious}
	\EndProof
	\\
	\Theorem{SigmaFiniteIsStrictlyLocalizable}
	{
		\forall (X,\Sigma,\mu) : \sFinite \.
		\SLoc(X,\Sigma,\mu)
	}
	\Explain{
		Take $F$ to be a disjoint partition of $X$ into sets of finite measure $\mu$}
	\Explain{
		Then every set $E$ can be represented as
		$E = \bigcup^\infty_{n=1} E \cap F_n$}
	\Explain{
		But if all sets in union are measurable, then $E$ is also measurable, 
		as the union is countable}
	\Explain{
		Moreover, $\mu(E) = \sum^\infty_{n=1} \mu(E \cap F_n)$ as $F$ is a disjoint sequence}
	\Explain{
		So $F$ is a decomposition of $\mu$}
	\EndProof
	\\
	\Theorem{StrictlyLocalizableIsSemifinite}
	{
		\forall (X,\Sigma,\mu) : \SLoc \.
		\Semifinite(X,\Sigma,\mu)
	}
	\Explain{
		Take $\E$ to be a decomposition of $\mu$}
	\Explain{
		Assume $F \in \Sigma$ such that $\mu(F) = \infty$}
	\Explain{
		Then $\mu(F) = \sum_{E \in \E} \mu(F \cap E)$,
	so there must be some $E \in \E$ such that $\mu(E \cap F) > 0$}
	\Explain{
		Also by monotonicity $\mu(E \cap F) \le \mu(E) < \infty$
	}
	\EndProof
}\Page{
	\Theorem{StrictlyLocalizableIsLocalizable}
	{
		\forall (X,\Sigma,\mu) : \SLoc \.
		\Loc(X,\Sigma,\mu)
	}
	\Explain{
		 Take $\E$ to be a decomposition of $\mu$}
	\Explain{
		Assume $\F \subset \Sigma$}
	\Explain{
		Define $\A = \{A\in \Sigma : \forall F \in \F \. \mu(A \cap  F) = 0 \}$}
	\Explain{ 
		Then $\A$ is a $\sigma$-subring and ideal of $\Sigma$}
	\Explain{
		Define $\gamma : \E \to \EReals_{++}$ as $\gamma(E) = \sup\Big\{ \mu(A \cap E) \Big| A \in \A  \Big\}
		\le \mu(E) < \infty$}
	\Explain{
		For Eeach $E \in \E$ define $A_{E} : \Nat \to \A$ to be such a sequence of sets that
		$\gamma(E) = \lim_{n \to \infty} \mu(A_{E,n} \cap E)$
	}
	\Explain{
		Define $A'_E = \bigcup^\infty_{n=1} A_{E,n} \in \A, A''= \bigcup_{E \in \E} A'_E \cap E, H = X\setminus A''$}
	\Explain{
		Then, $\forall E \in \E \. E \cap A'' = A'_E \cap E \in \Sigma$ so by definition of decomposition
		$A'' \in \Sigma$}
	\Explain{ 
		And so $H \in \Sigma$}
	\Explain{
		Assume $F \in \F$}
	\Explain{
		Then, 
			$
				\mu(F \setminus H) = 
				\mu(F \cap A'') =
				\sum_{E \in \E} \mu\big( F \cap A'' \cap E\big) 
				\sum_{E \in \E} \mu\big(   F \cap A'_E \cap E\big) =
				\sum_{E \in \E} 0 =
				0 
			$}
	\Explain{
		On the other hand, assume $G \in \Sigma$ is such that $\forall F \in \F \. \mu(F\setminus G) = 0 $
	}
	\Explain{
		Then $B = A'' \cup G^\c \in \A$}
	\Explain{
		This means that $\forall E \in \E \. \mu(E \cap B) \le \gamma(E)$}
	\Explain{
		But by construction
		$\mu(A'' \cap E) \ge \sup_{n \in \Nat} \mu(A_{E,n} \cap E) = \gamma(E)$
	}
	\Explain{
		So, 
		$
			\mu(B \cap E) \ge \mu(A'' \cap E) = \gamma(E)		
		$	
		and finally  $\mu(B \cap E) = \gamma(E)$}
	\Explain{
		Moreover, 
		$\gamma(E) \ge \mu(A'_E \cap E) \ge \sup_{n \in \Nat} \mu(A_{E,n} \cap E) = \gamma(E)$}
	\ExplainFurther{
		Hence,
		$
			\mu(H \setminus G) = 
			\sum_{E \in \E} \mu(H \cap G^\c \cap E) \le  
			\sum_{E \in \E} \mu( (B \cap E) \setminus (A_E' \cap E) )   =
			\sum_{E \in \E} \mu(B \cap E)  - (A_E' \cap E)  =$}
	\Explain{	
			$=\sum_{E \in \E} \gamma(E) - \gamma(E) = 
			0
		$}
	\Exclaim{
		So, indeed, $H$ is an essential supremum for $\F\;$}
	\EndProof
	\\
	\Theorem{StrictlyLocalizableIsLocallyDetermined}
	{
		\NewLine ::		
		\forall (X,\mu, \Sigma) : \SLoc \.
		 \LocDet(X,\mu,\Sigma)
	}
	\Explain{
		 Take $\E$ to be a decomposition of $\mu$}
	\Explain{
		Assume $A \subset X$ such that $A \cap F \in \Sigma$ forall $F \in \Sigma$ 
		such that $\mu(F) < \infty$}
	\Explain{
		But then $\forall E \in \E \. A \cap E \subset \Sigma$
	}
	\Explain{
		So, the definition of decomposition $A \in \Sigma$
	}
	\EndProof
}
\Page{
	\Theorem{SigmaFinitenessConditionForSemifinite}
	{
		\NewLine ::		
		\forall (X,\Sigma,\mu) : \Semifinite \.
		\sFinite(X,\Sigma,\mu)
		\iff
		\exists  \nu : \Finite(X,\Sigma) \.
		\Null_\nu = \Null_\mu
	}
	\Explain{
		Assume $\mu$ is $\sigma$-finite}
	\Explain{
		Let $F$ be a countable partition of $X$ into sets of finite $\mu$-measure}
	\Explain{
		Then, if $\mu(F_n) \neq 0$ and $E \subset F$ is measurable 
		define $\nu(E) = \frac{2^{-n}\mu(E)}{\mu(F_n)}$, otherwise define $\nu(E) = 0$
	}
	\Explain{
		By construction $\nu |F_n$ is a measure for each $n$}
	\Explain{
		As $F$ is countable and disjoint $\nu$ can be extended as a measure on
		$(X,\Sigma)$}
	\Explain{ 
		But
		$
			\nu(X) = \nu\left( \bigcup^\infty_{n=1} F_n \right) =
			\sum^\infty_{n=1} \nu(F_n) \le 
			\sum^\infty_{n=1}   2^{-n} = 1
		$}
	\Explain{
		So $\nu$ is finite}
	\Explain{
		Clearly, from construction $\Null_\nu = \Null_\mu$
	}
	\Explain{
		Now, let $\mu$ be semifinite, and $\nu$ with properties as above}
	\Explain{
		Assume $\mu(X) = \infty$, otherwise we are done}
	\Explain{
		Let $A = \{ \mu(E) | E \in \Sigma, 0 < \mu(E) < \infty \}$ }
	\Explain{
		$A$ is non-empty as $\mu$ is semifinite}
	\Explain{
		If $\sup A < \infty$ there must be a sequence $E$ of sets in $A$
		such that $\lim_{n\to \infty} \mu(E_n) = \sup A$}
	\Explain{
		Then, $\mu\left( \bigcup^\infty_{n=1} E_n \right) = \sup A < \infty$
		by lower continuity}
	\Explain{
		But this means that $\mu\left(X \setminus \bigcup^\infty_{n=1} E_n \right) = \infty$
	}
	\Explain{
		And there is a measureble $F$ with $0 < \mu(F) < 0$ disjoint from each $E_n$}
	\Exclaim{
		Then, $\infty > \mu\left( F \cup \bigcup^\infty_{n=1} E_n  \right) > \sup A$,
		a contradiction}
	\Explain{
		So  $\sup A = \infty$
	}
	\Explain{
		Denote by $\A$ set of increasing sequences $E$ in $\Sigma$ such that
		$0 < \mu(E_n) < \infty$ and $\lim_{n \to \infty} \mu(E_n) = \infty$
	}
	\Explain{
		We know that $\A$ is not empty}
	\Explain{
		Take $\alpha = \sup_{E \in \A} \lim_{n \to \infty} \nu(E_n) \le \nu(X) < \infty$
	}
	\Explain{
		Then there exists $E \in \A$ such that $\mu\left( F \right) = \alpha$ with $F = \bigcup E$
		 (consider the diagonal)}
	\Explain{
		But if $\alpha < \nu(X)$ then $\nu(F^\c) > 0$ and so $\mu(F^\c) > 0$		
	}
	\Explain{
		So there must be some $G$ with $0 < \mu(G) < \infty$ and so with $0 < \nu(G)$
	disjoint from every $E_n$}
	\Explain{
		Thus $E_n \cup G \in A$ and $\lim_{n \to \infty} \nu(E_n \cup G) = \nu(G) + \lim_{n \to \infty} \nu(E_n)
		> \alpha =  \sup_{E \in \A} \lim_{n \to \infty} \nu(E_n) $}
	\Exclaim{
		A contradiction}
	\Explain{
		And so $\alpha = \nu(X)$ and $\nu(F^\c) = 0$}
	\Explain{
		Hence, $\mu(F^\c)=0$
	}
	\Explain{
		But $X = F \cup F^\c$ and $\mu$ is clearly $\sigma$-finite on $F$,
		so $\mu$ is also $\sigma$-finite on $X$}
	\EndProof
}\Page{
	\Theorem{PointSupportedIsComplete}
	{
		\forall (X,\Sigma,\mu) : \PtSupp \.
		\CMS(X,\Sigma,\mu)
	}
	\Explain{ $\mu$ measures every subset of $X$ by definiton}
	\EndProof
	\\
	\Theorem{PointSupportedStrictlyLocalizableIfSemifenite}
	{
		\NewLine ::
		\forall (X,\Sigma,\mu) : \PtSupp \.
		\Semifinite(X,\Sigma,\mu)
		\iff
		\SLoc(X,\Sigma,\mu)
	}
	\Explain{
		Every strictly localizable space is semifinite}
	\Explain{
		So, consider the case then $\mu$ is semifinite }
	\Explain{ If $\{x\}$ is a singleton, then $\mu\{x\} < \infty$}
	\Explain{
		Consider the contrary}
	\Explain{ 
		Then there must be $E \subset \{x\}$ such that $0 < \mu(E) < \infty$}
	\Explain{ 
		But this is imposible}
	\Explain{
		So, take $\E = \Big\{ \{x\} \Big| x \in X \Big\}$ to be a decomposition
	}
	\Explain{ This works as $\mu$ is point-supported}
	\EndProof
}\Page{
	\Theorem{AtomlessSemifiniteCondition}
	{
		\NewLine :		
		\forall (X,\Sigma,\mu) : \Semifinite \.
		\Aless (X,\Sigma,\mu) \iff \NewLine \iff
		\forall \varepsilon \in \Reals_{++} \.
		\forall E \in \Sigma \.
		\forall \aleph : \mu(E) < \infty \. 
		\exists \A : \TYPE{Partition}(E,\Sigma) \.
		|\A| < \infty 
		\And \forall A \in \A \. \mu(A) \le \varepsilon 
	}
	\Explain{
		Assume that $\mu$ is atomless}
	\Explain{
		Take $E\in \Sigma$ such that $\mu(E) < \infty$ and $\varepsilon\in\Reals_{++}$}
	\Explain{
		Define $\F =  \{  F : \Sigma : F_n \subset E \and 0 < \mu(F) < \mu(E) \}$	}
	\Explain{
		Then $\exists\F$ as $\mu$ is atomless
	}
	\Explain{
		I claim that $\inf_{F \in \F}  \mu(F) = 0$
	}
	\Explain{
		As $\mu$ is atomless it is possible to select $F$
		such that $0 < \mu(F) <  \mu(E)$ 
	}
	\Explain{
		But then either $\mu(F)$ or $\mu(E\setminus F)$ has measure less then $\frac{\mu(E)}{2}$}
	\Explain{
		But then it is possible to extract sequence $F$ with $\mu(F_n) \le \frac{\mu(E)}{2^n}$
	}	
	\Explain{
		Note, that if $\mu(E) \le \varepsilon$, then we are done}
	\Explain{
		Otherwise, we can select $F \subset E$ with $F \in \Sigma$
		and $\mu(F) \le \varepsilon$}
	\Explain{
		I want to show that it is possible to selrct $F$ with $\mu(F) = \varepsilon$}
	\Explain{
		Define $\F = \{  F : \Nat \uparrow \Sigma : F_n \subset E \and 0 < \mu(F_n) \le \varepsilon \}$
	}
	\ExplainFurther{
		We know that $\F$ is non-empty}
	\Explain{ 
		As we can keep selecting subsets of small measure 
		in the complements and adding that}
	\Explain{
		If $\sup_{F \in \F} \lim_{n \to \infty} \mu(F_n) = \alpha < \varepsilon$
		we can select a sequen $F \in \F$ with $\sup_{n\in \infty} \mu(F_n) = \alpha$
	}
	\Explain{
		Indeed if $G$ and $F$ are in $F$ we can select a max by taking $H \subset G_n \cup F_n$
		with measure less then $\varepsilon$ }
	\Explain{ 
		Then we can take $H$ such that $\mu(H) \ge \max\Big(\mu(G_n),\mu(F_n)\Big)$}
	\Explain{
		So,by diagonal construction $F \in \A$ with $\sup_{n\in \infty} \mu(F_n) = \alpha$ exists
	}
	\Explain{
		Then $\mu\left(\bigcup^\infty_{n=1} F_n\right) = \alpha < \varepsilon < \mu(E)$}
	\Explain{
		So $\mu\left(E \setminus \bigcup^\infty_{n=1} A_n\right) > 0$, 
		so we can extract some $H$ with $\mu(H) \le \varepsilon - \alpha$ and
		disjoint from all $F_n$}
	\Explain{
		But, then $F \cup H \in \F$ and $\lim_{n \to \infty} \mu(F_n \cup H) > \alpha $, 
		a contradiciton.
	}
	\Explain{ So we can keep extracting disjoint sets $F$ of $\mu(F) = \varepsilon$
		untill $\mu\left( E \setminus \bigcup^n_{i=1} F_n\right) \le \varepsilon$}
	\Explain{
		Such $n$ should exist as $\mu(E) < \infty$}
	\Explain{
		Now consider the case, then $\mu$ is semifinite and the righthandside property holds}
	\Explain{
			Then if $0 < \mu(E) < \infty$ ther must be a subset $F$ of $E$ with
			$0 < \mu(F) \le \frac{\mu(E)}{2}$, so $E$ is not an atom.}
	\Explain{
		If $\mu(E) = \infty$ there must be $F \subset E$ with $0 < \mu(F) < \infty$, so
		$E$ again is not an atom}
	\EndProof
}
\Page{
	\Theorem{AtomlessStrictlyLocalizableCondition}
	{
		\NewLine ::		
		\forall (X,\Sigma,\mu) : \SLoc \.
		\Aless (X,\Sigma,\mu) \iff \NewLine \iff
		\forall \varepsilon \in \Reals_{++} \.
		\exists \F : \TYPE{Decomposition}(X,\Sigma,\mu) \.
		\forall F \in \F \. \mu(F) \le \varepsilon
	}
	\Explain{ 
		Let $\E$ be a decomposition of $\mu$}
	\Explain{ 
		Note, that every strictly localizable space is semifinite}
	\Explain{
		Assume that $\mu$ is purely atomic}
	\ExplainFurther{
		Then as every set $E \in \E$ has $\mu(E) < \infty$,}
	\Explain{
		there must be a finite partition $\mathcal{P}_E$ of $E$ into
		sets $P \in \mathcal{P}_E$ with $\mu(P) \le \varepsilon$}
	\Explain{
		We claim that $\F = \bigcup_{E \in \E} \mathcal{P}_E$ is a
		decomposition of $\mu$}
	\Explain{
		Clearly, by construction $\F$ consisits of pairwise disjoint sets}
	\Explain{
		If $A \subset X$ is such that $\forall F \in \F \. A \cap F \in \Sigma$, then
		$A\in \Sigma$}
	\Explain{
		If $E \in \E$, then $A \cap E = \bigcup_{P \in \mathcal{P}_E} A \cap P \in \Sigma $ }
	\Explain{
		As $\E$ is a decomposition $A \in \Sigma$} 
	\Explain{
		Consider any $H \in \Sigma$}
	\Explain{
		Then $\mu(H) = 
			\sum_{E \in \E} \mu(H \cap E) = 
			\sum_{E \in \E} \sum_{P \in \mathcal{P}_E} \mu(H \cap P) =
			\sum_{F \in \F} \mu(H \cap F)$}
	\Explain{ 
		So $\F$ is indeed a decomposition of $\mu$}
	\Explain{
		Now let $\mu$ be just strictly localizable and le righthandside statement be true.
	}
	\Explain{
		Assume $E\in \Sigma$ with $\mu(E) > 0$}
	\Explain{
		Then construct a decomposition $\F$ of $\mu$
		such that $\forall F \in \F \. \mu(F) < \mu(E)$}
	\Explain{
		Then there must be some $F \in \F$, so $\mu(E \cap F) > 0$}
	\Explain{ 
		But also $ 0 < \mu(E \cap F) \le \mu(F) < \mu(E)$,
		so $E$ is not an atom}
	\EndProof
	\\
}\Page{
	\Theorem{AtomlessStrictlyLocalizableFunctionalCondition}
	{
		\NewLine ::		
		\forall (X,\Sigma,\mu) : \SLoc \.
		\Aless (X,\Sigma,\mu) \iff \NewLine \iff
		\exists f \in \BOR\Big((X,\Sigma),\Reals_{++}) \.
		\forall t \in \Reals \.
		\mu\Big(f^{-1}\{t\}\Big) = 0
	}
	\Explain{ 
		Let $\E$ be a decomposition of $\mu$}
	\ExplainFurther{
		By previous theorem we can construct a sequence of decomposisitions
		$\F$ such that $\F_0 = \E$,} 
	\Explain{
		$\F_{n+1}$ is a finite refinement of $\F_n$,
		and $\mu(F) \le \frac{1}{n}$ forall $F \in \F_n$ for $n \ge 1$}
	\Explain{
		Take $g_0(x) = 0$
	}
	\Explain{
		Assert that $g_n$ is a constant on each $F \in \F_n$, so $g_n(F) = \{\alpha\}$}
	\Explain{
		Denote by $\mathcal{P}$ a partition of $F$ in $\F_{n+1}$}
	\Explain{
		Let $m = |\mathcal{P}| < \infty$, also assume $\mathcal{P}$ is ordered
		as $\{P_1,\ldots,P_m\}$
	}
	\Explain{
		Construct $g_{n+1}$ by setting $g_{n+1}(x) = \alpha - \frac{1}{2^n} - \frac{1}{2^nm} + \frac{2k}{2^n m}$
		for $x \in P_k$ } 
	\Explain{ 
		So, the values of $g_{n+1}$ over $F$ changes from 
		$\alpha - \frac{1}{2^{n+1}}$ to $\alpha+ \frac{1}{2^{n+1}}$ }
	\Explain{
		If such construction do not intersect values of neighbouring elements of partition
	}
	\ExplainFurther{
		Otherwise set $g_{n+1}(x) = \frac{m+k}{4m}\alpha_+ 
		+ \left(1 - \frac{m+k}{4m} \right)\alpha_-$
		for $x \in P_k$,}
	\Explain{
		where $\alpha_-$ and $\alpha_+$ are values of $g_n$ at neighbouring partition cells}
	\Explain{
		Note that $g_n^{-1}(a,b) \cap F$ is either $\emptyset$ or $F$,  if $g_n(F) \subset (a,b)$,
	    for every $F \in \F_n$}
	\Explain{
		So, $g_n^{-1}(a,b) \in \Sigma$ as $\F_n$ is a decomposition}
	\Explain{
		Thus, each $g_n$ is measurable}
	\Explain{ 
		Note, that $g_n(x)$ is Cauchy for every $x \in X$}
	\Explain{ 
		Then $f = \lim_{n \to \infty} g_n$ is also measurable}
	\Explain{ 
		By construction $f^{-1}(t) \cap E \subset P_k$ for each partition level}
	\Explain{
		Thus, $\mu\Big( f^{-1}(t) \cap E  \Big) \le \frac{1}{n}$ for all $n$}
	\Explain{
		So $\mu\Big( f^{-1}(t) \cap E  \Big) = 0$}
	\Explain{
		But $\mu\Big(f^{-1}(t)\Big) = \sum_{E \in \E} \mu\Big( f^{-1}(t) \cap E  \Big) = 0$}
	\Explain{
		The other direction is trivial}
	\EndProof
}
\newpage
\subsubsection{Counting Measure Example}
\Page{
	\Theorem{CountingMeasureIsComplete}
	{
		\forall X \in \SET \.
		\CMS(X,2^X,\#)
	}
	\Explain{ 
		If $\#A=0$ for $A \subset X$, then $A = \emptyset$}
	\EndProof
	\\
	\Theorem{CountingMeasureIsStrictlyLocalizable}
	{
		\forall X \in \SET \.
		\SLoc(X,2^X,\#)
	}
	\Explain{ 
		Take $\E = \Big\{\{x\} \Big| x \in X\Big\}$}
	\Explain{
		Then the first condition of being a decompsition holds trivialy for $\E$}
	\Explain{
		Also, notice that
		$\# A = 
		\sum_{x \in A} \#\{x\}=
		\sum_{E \in \E} \#(E \cap A)
		$}
	\Explain{
		So $\E$ is a decomposition, indeed}
	\EndProof
	\\
	\Theorem{CountingMeasureIsPurelyAtomic}
	{
		\forall X \in \SET \.
		\PA(X,2^X,\#)
	}
	\Explain{
		Consider $A \subset X$ with $\#A > 0$}
	\Explain{ 
		Then $A$ must be non empty}
	\Explain{
		So there is $x \in A$}
	\Explain{
		But clearly $\#\{ x \} = 1$, so $\{x\} \subset A$ is an atom}
	\EndProof
	\\
	\Theorem{CountingMeasureSigmaFiniteIfCountable}
	{
		\NewLine ::
		\forall X \in \SET \.
		\sFinite(X,2^X,\#)
		\iff
		\Countable(X)
	}
	\Explain{If $\#$ is $\sigma$-finitite then $X$ is representable as a countable union of finite sets}
	\Explain{ So, $X$ is countable}
	\Explain{ If $X$ is countable, write $X = \bigcup_{x \in X} \{x\}$}
	\Explain{ Then $\#$ is $\sigma$-finite as $\#\{x\} = 1$}
	\EndProof
	\\
	\Theorem{CountingMeasureFiniteIfFinite}
	{
		\forall X \in \SET \.
		\Finite(X,2^X,\#)
		\iff
		\Finite(X)
	}
	\Explain{Obvious}
	\EndProof
	\\
	\Theorem{CountingMeasureProbabilityIfSingleton}
	{
		\forall X \in \SET \.
		\Probability(X,2^X,\#)
		\iff
		\TYPE{Singleton}(X)
	}
	\Explain{Obvious}
	\EndProof
}
\Page{
	\Theorem{CountingMeasureAtomlessIfEmpty}
	{
		\forall X \in \SET \.
		\Aless(X,2^X,\#)
		\iff
		X = \empty
	}
	\Explain{
		Clearly, every $\{x\} \subset X$ will constitute an atom}
	\EndProof
	\\
	\Theorem{CountingMeasureIsPointSupported}
	{
		\forall X \in \SET \.
		\PtSupp(X,2^X,\#)
	}
	\Explain{
		This is obvious as $\# A = 
		\sum_{x \in A} \#\{x\}$
	}
	\Explain{
		If $A$ is infinite, then the righthandside sum is infinte
	}
	\Explain{
		Otherwise proceed by induction on cardinalitys of the set
	}
	\Explain{
		From definitions $\# \emptyset = 0 = sum_{x \in \emptyset} \#\{x\}$
	}
	\Explain{
		Now consider we know the results holds for set with cardinality at most $n$
	}
	\Explain{
		Assum $|A| = n+1$, so $\# A = n + 1$
	}
	\Explain{ Choose on $a$ in $A$, $A$ must be non-empty as $n+1 \ge 1$}
	\Explain{
		Then $\# A = 
			\#(\{a\} \cup A \setminus \{a\}) =
			\#\{a\} + \#(A \setminus \{a\}) = 
			\#\{a\} + \sum_{x \in A \setminus \{a\}} \#\{x\} =
			\sum_{x \in A} \#\{x\}		
		$
	}
	\EndProof
}
\newpage
\subsubsection{Countable-Cocountable Measure}
\Page{
	\DeclareFunc{countableCocountableSigmaAlgebra}
	{
		\prod X \in \SET \. \SA(X)
	}
	\DefineNamedFunc{countableCocountableSigmaAlgebra}{}{
		\Upomega(X)
	}
	{
		\Big\{ A \subset X : \min\big(|A^\c|,|A|\big) \le \aleph_0 \Big\} 	
	}
	\Explain{
		Clearly $X,\emptyset \in \Upomega(X)$ and it is closed by complements 
	}
	\Explain{
		Now, consider $E : \Nat \to \Upomega(X)$}
	\Explain{
		If  $|E_n| \le \aleph_0$ for at  least one $n \in \Nat$
		then the intersection of $E$ is countable}
	\Explain{
		In the other case every set $E_n$ has a countable complement}
	\Explain{
		And a countable union of countable sets is again countable}
	\Explain{
		So their intersection has a countable complement and belongs to $\Upomega(X)$
	}
	\EndProof	
	\\
	\DeclareFunc{countableCocountableMeasure}
	{
		\prod X \in \SET \. \Measure\Big(X,\Upomega(X)\Big)
	}
	\DefineNamedFunc{countableCocountableMeasure}{E}{\omega(E)}{\Big[ |E| > \aleph_0 \Big]}
	\Explain{
		As $\emptyset$ is finite, $\omega(\emptyset) = 0$}
	\Explain{
		Now consider a disjoint sequence $E$ with $E_n \in \Sigma$}
	\Explain{
		If $E_n$ is uncountable for some $n$ then it must have a countable complement}
	\Explain{
		So, all other sets $E_m$ with $m \neq n$ must be countable}
	\Explain{
		Thus, $\omega\left(\bigcup^\infty_{i=1} E_i \right) =1 = \omega(E_n) = \sum^\infty_{i=1} \omega(E_i)$
	}
	\Explain{
		Conversly, if all $E_n$ are countable, then
		$
			\omega\left(\bigcup^\infty_{i=1} E_i \right) = 0 = \sum^\infty_{i=1} \omega(E_i)
		$}
	\EndProof
	\\
	\Theorem{CountableCocountableIsProbability}{\forall X : \Uncountable \. \Probability\Big(X,\Upomega(X),\omega\Big)}
	\Explain{
		$X^\c = \emptyset$ is finite}
	\Explain{
		So, $\omega(X) = 1$}
	\EndProof
	\\
	\Theorem{CountableCocountableIsPurelyAtomic}
	{ \forall X \in \SET \. \PA\Big(X,\Upomega(X),\omega\Big)}
	\Explain{
		Assume $E \in \Upomega(X)$ such that$\omega(E) = 1$}
	\Explain{
		Then every measurable subset $F \subset E$ either countable or has a countable complement}
	\Explain{
		So, either $\omega(F) = 0$ or $\omega(F)=1$, so $E$ is an atom}
	\EndProof
}\Page{
	\Theorem{CountableCocountableIsNotPointSupported}
	{
		\forall X : \Uncountable \. \neg \PtSupp\Big(X,\Upomega(X),\omega\Big)
	}
	\Explain{
		Of course,  $\omega(X) = 1 \neq 0 = \sum_{x \in X} \omega\{x\}$
	}
	\EndProof
	\\
	\Theorem{CountingMeasureIsNotLocalizable}
	{
		\neg \Loc(\Reals,\Upomega(\Reals),\#)
	}
	\Explain{ Take $\A = \{ A \subset \Reals_+ : |A| \le \aleph_0\}$}
	\Explain{ Then the $E =\esssup \A$ must be cocountable}
	\Explain{ But, then its intersection with $\Reals_{--}$ is also cocountable}
	\Explain{ So, it is possible to construct a smaller set $G$ by discarding a finite number $n$ of negative points}
	\Explain{ Then $\#(E \setminus G) = n$, a contradiction}
	\EndProof
	\\
	\Theorem{CountingMeasureIsNotLocallyDetermined}
	{
		\neg \LocDet(\Reals,\Upomega(\Reals),\#)
	}
	\Explain{ If $\# E < \infty$, then $E$ must be finite}
	\Explain{ So for $E \cap A \in \Upomega(\Reals)$ for every set $A \subset E$}
	\Explain{ But clearly $\Upomega(\Reals) \neq 2^\Reals$}
	\EndProof
}
\newpage
\subsubsection{Measures Induced by Sigma-Ideals}
\Page{
	\DeclareFunc{idealsSigmaAlgebra}
	{
		\prod_{X \in \SET}  \SIdeal(X) \to \SA(X)
	}
	\DefineNamedFunc{idealsSigmaAlgebra}{I}
	{
		\Upomega(I)
	}
	{
		\Big\{ E \subset X :  E \subset I | E^\c \subset I  \Big\}
	}
	\Explain{
		Same proof as with countable-cocountable case
	}
	\EndProof
	\\
	\DeclareFunc{idealsMeasure}
	{
		\prod_{X \in \SET}  \prod I : \SIdeal(X) \. \Measure\Big( X, \Upomega(I) \Big) 
	}
	\DefineNamedFunc{idealsMeasure}{E}
	{
		\omega_I(E)
	}
	{
		[E \not \subset I]
	}
	\Explain{
		Same proof as with countable-cocountable case
	}
	\EndProof
	\\
	\Theorem{IdealsMeasureIsProbability}{
		\NewLine ::		
		\forall X \in \SET \. 
		\forall I : \SA(X) \.
		\forall \alpha : X \neq I \.		
		\Probability\Big(X,\Upomega(I),\omega_I\Big)
	}
	\Explain{
		$X^\c = \emptyset \in I$ }
	\Explain{
		So, $\omega_I(X) = 1$}
	\EndProof
	\\
	\Theorem{IdealsMeasureIsPurelyAtomic}
	{ \forall X \in \SET \. \forall I : \SIdeal(X) \. \PA\Big(X,\Upomega(I),\omega_I\Big)}
	\Explain{
		Assume $E \in \Upomega(I)$ such that $\omega_I(E) = 1$}
	\Explain{
		Then every measurable subset $F \subset E$ either in $I$ or has a complement in $I$}
	\Explain{
		So, either $\omega_I(F) = 0$ or $\omega_I(F^\c)=0$, so $E$ is an atom}
	\EndProof
}
\newpage
\subsection{Completeness}
\subsubsection{Integrability in a Complete space}
\Page{
	\Theorem{VirtualMeasurabilityIsReal}
	{
		\NewLine ::		
		\forall (X,\Sigma,\mu) : \CMS \.
		\forall f \in \BOR_\mu^*\Big((X,\Sigma),\EReals\Big) \.
		f \in \BOR_\mu\Big((X,\Sigma),\EReals\Big)
	}
	\Explain{
		By definition of $\BOR_\mu^*\Big((X,\Sigma),\EReals\Big)$, 
		There is an $A \subset \dom f \cap \Null'_\mu$
		such that $f_{|E}$ is measurable}
	\Explain{
		But as $\mu$ is complete, $A$ has a measurable complement}
	\Explain{
		So $A$ is measurable and conull}
	\Explain{
		But this means that $\dom f$ is also measurable and so is $E  = \dom f \setminus A$.
	}
	\Explain{
		Then for every $B \in \B\Big(\EReals\Big)$ there is representation 
		$f^{-1}(B) = f^{-1}_{|A}(B) \cup C$		
		for some $C \subset E$}
	\Explain{
		But $C$ is measurable as $\mu(E) = 0$ and $\mu$ is complete 
	}
	\Explain{
		So $f$ is measurable}
	\EndProof
	\\
	\Theorem{IntegrableIsMeasurable}
	{
		\NewLine ::
		\forall (X,\Sigma,\mu) : \CMS \.
		\forall A \in \Null'_\mu \. 
		\forall f X \to \EReals \.
		f \in L_1(X,\Sigma,\mu) 
		\iff \NewLine \iff
		f \in \BOR_\mu\Big( \EReals  \Big)
		\And
		|f| \in L_1(X,\Sigma,\mu)
	}
	\Explain{
		If $f$ is integrable, then it must be virtually measurable}
	\Explain{
		But we just proved that it 	must be measurable}
	\Explain{
		Also it follows that $|f| = f_+ + f_-$ is integrable}
	\Explain{
		This proves one direction}
	\Explain{ 
		On the other hand $E = f^{-1}(0,+\infty]$ must be measurable}
	\Explain{
		So take $\sigma$ be an increasing sequence of simples producing $\int |f| = \lim_{n\to \infty} \int \sigma_n$}
	\Explain{
		Then $\int f_+  = \int_E |f| = \lim_{n\to \infty} \int \sigma_n$}
	\Explain{
		So, $f_+$ has integral and a simmilar argument works for $f_-$ and $f$ has integral}
	\Explain{
		Also $-\infty < \int |f| = \int f_+ + \int f_- < + \infty$, 
		so	$-\infty <  \int f_+ < + \infty$ and $-\infty <  \int f_+ < + \infty$}
	\Explain{
		Thus $-\infty < \int f = \int f_+ - \int f_- < + \infty$}
	\Explain{
		And $f$ is integrable}
	\EndProof
}\Page{
	\Theorem{IntegrableByDomination}
	{
		\NewLine ::
		\forall (X,\Sigma,\mu) : \CMS \.
		\forall A \in \Null'_\mu \. 
		\forall f X \to \EReals \.
		f \in L_1(X,\Sigma,\mu) 
		\iff \NewLine \iff
		f \in \BOR_\mu\Big( \EReals  \Big)
		\And
		\exists g \in L_1(X,\Sigma,\mu) \. 
		|f| \le_{\ae \mu} g
	}
	\Explain{
		 This is simmilar to the previous result}
	\EndProof
}
\newpage
\subsubsection{Completion}
\Page{
	\DeclareFunc{sigmaAlgebraCompletion}{\MEAS \to \BOR}
	\DefineNamedFunc{sigmaAlgebraCompletion}{X,\Sigma,\mu}{(X,\hat \Sigma_\mu)}
	{
		\Big( X, 
			\big\{ 
				A \subset X : \exists E,E' \in \Sigma \. 
				E \subset A \subset E' \And \mu(E'\setminus E) 
			\big\}
		\Big)	
	}
	\Explain{
		Clearly $\Sigma \subset \hat \Sigma$, so $\emptyset \in \hat \Sigma$}
	\Explain{
		Assume $A \in \hat \Sigma$}
	\Explain{
		Then there are $E,F \in \Sigma$ such that $E \subset A \subset F$ and $\mu(F \setminus E) =0$}
	\Explain{
		But $E^\c \subset A^\c \subset F^\c$ and  $\mu(E^\c \setminus F^\c) = \mu(F \cap E^\c) = 
		\mu(F \setminus E) =0$ by duality}
	\Explain{
		So, $A^\c \in \hat \Sigma$
	}
	\Explain{
		Now consider a sequence $A : \Nat \to \hat \Sigma$}
	\Explain{
		Then there is a sequences $E,F : \Nat \to \Sigma$ such that 
		$E_n \subset A_n \subset F_n$ and $\mu(F_n \setminus E_n) = 0$ for every $n \in \Nat$}
	\Explain{
		But clearly $\bigcup^\infty_{n=1} E_n 
			\subset \bigcup^\infty_{n=1} A_n 
			\subset \bigcup^\infty_{n=1} F_n$ and
		$\mu\left(\bigcup^\infty_{n=1} F_n \setminus \bigcup^\infty_{n=1} E_n\right) \le 
			\mu\left( \bigcup^\infty_{n=1} F_n \setminus E_n \right) \le 
			\sum^\infty_{n=1} \mu(F_n \setminus E_n) = 0	$}
	\Explain{
		So, $\bigcup^\infty_{n=1} A_n \in \hat \Sigma$}
	\Explain{ 
		This proves that $\hat \sigma$ is an $\sigma$-algebra}
	\EndProof	
	\\
	\DeclareFunc{measureCompletion}{\MEAS \to \CMS}
	\DefineNamedFunc{measureCompletion}{X,\Sigma,\mu}{(X,\hat \Sigma_\mu,\hat \mu)}
	{
		\Big(X,\hat \Sigma_\mu,  \mu^*_{|\hat \Sigma_\mu} \Big)
	}
	\Explain{
		We need to show that $\hat \Sigma_\mu \subset \Sigma_{\mu^*}$ to prove that $\hat \mu$ is a measure}
	\Explain{
		Consider $E \in \hat \Sigma$}
	\Explain{
		Then there are $G,F \in \Sigma$ such that $G \subset E \subset F$ and  
		$\mu(F \setminus G) = 0$}
	\Explain{ 
		 	Now consider arbitraty subset $A \subset X$}
	\Explain{
		Then $A \cap F \subset H \cup (F \setminus G) $ for every $H\in \Sigma$ with $A \cap G \subset H$}
	\Explain{
		But $\mu(H) \le \mu\big(H \cup (F \setminus G)\big) \le \mu(H) + \mu(F \setminus G) = \mu(H)$}
	\Explain{
		Thus $\mu\big(H \cup (F \setminus G)\big) = \mu(H)$}
	\Explain{
		As $H$ was arbitrary this means that $\mu^*(A \cap G) = \mu^*(A \cap F)$}
	\Explain{
		The simmilar argument may be used to show that 
		$\mu^*{A \setminus G} = \mu^*(A \cap G^\c) = \mu^*(A \cap F^\c) = \mu^*(A \setminus F)$}
	\Explain{
		But $\mu^*(A \cap G) \subset \mu^*(A \cap E) \subset \mu^*(A \cap F)$ 
		and $\mu^*(A \setminus F) \subset \mu^*(A \setminus E) \subset \mu^*(A \setminus F)$	
		proving equlity}
	\Explain{
	Thus, $\mu^*(A) = \mu^*(A \cap G) + \mu^*(A \setminus G) 
	= \mu^*(A \cap E) + \mu^*(A \setminus E)$, so $E \in \Sigma_{\mu^*}$}
	\Explain{ Now we want to show that $\hat \mu$ is complete}
	\Explain{ 
		Take some $Z \in \hat \Null_{\hat \mu}$ }
	\Explain{
		Then there is some $E \in \hat \Sigma$ such that 
		$Z \subset E$ and $\hat \mu(E) = \mu^*(E) = 0$}
	\Explain{
		But this means that there is an $F \in \Sigma$
		such that $\mu(F) =0$ and $E \subset F $}
	\Explain{  
		So $\emptyset \subset Z \subset F$ and $0 = \mu(F) = \mu(F\setminus \emptyset)$}
	\Explain{
		But this means that exactly that $Z \in \hat \Sigma$
	}
	\Explain{
		As $Z$ was arbitrary $\hat \mu$ is complete
	}
	\EndProof
}\Page{
	\DeclareFunc{measurableZeroCategory}{\CAT}
	\DefineNamedFunc{measurableZeroCategory}{}{\MEAS_0}
	{
		\NewLine \de
		\bigg( 
			\MEAS, 
			\Lambda (X,\Sigma,\mu),(Y,T,\nu) \in \MEAS \. \NewLine \. 
		 	\Big\{ f \in \MEAS\Big( (X,\Sigma,\mu),(Y,T,\nu) \Big) : 
		 		\forall E \in T \. \nu(E) = 0 \Imply \mu\Big( f^{-1}(T) \Big)    
		 \Big\},
		 \circ,		 
		 \id	\bigg)
	}
	\\
	\DeclareFunc{sigmaAlgebraCompletionFunctor}{\Cov(\MEAS_0,\BOR)}
	\DefineNamedFunc{SigmaAlgebraCompletionFunctor}{(X,\Sigma,\mu)}{\mathsf{C}^\sigma(X,\Sigma,\mu)}
	{
		(X, \hat \Sigma_\mu)
	}
	\DefineNamedFunc{sigmaAlgebraCompletionFunctor}{(X,\Sigma,\mu),(Y,T,\nu),\phi}
	{\mathsf{C}^\sigma_{(X,\Sigma,\mu),(Y,T,\mu)}(\phi)}{\phi}
	\Explain{
		Consider a set $E \in \hat T_\nu$}
	\Explain{ 
		Then there are $G,F \in T$ such that $G \subset E \subset F$ and
		$\nu(F \setminus G) = 0$}
	\Explain{
		Then clearly $\phi^{-1}(G) \subset \phi^{-1}(E) \subset \phi^{-1}(F)$
	}
	\Explain{
		Bu also		
		$
			\mu\Big(\phi^{-1} (F) \setminus \phi^{-1}(E) \Big) = 
			\mu\Big( \phi^{-1}(F \setminus E)  \Big)	 =0
		$
	by definition of $\MEAS_0$}
	\Explain{
		So $f^{-1}(E) \in \hat \Sigma_\mu$ 
	}
	\Explain{
		This shows that $f$ is still measurable for a completion}
	\EndProof
	\\
	\DeclareFunc{measureCompletionFunctor}{\Cov(\MEAS,\BOR)}
	\DefineNamedFunc{measureCompletionFunctor}{(X,\Sigma,\mu)}{\mathsf{C}(X,\Sigma,\mu)}
	{
		(X, \hat \Sigma_\mu, \hat\mu)
	}
	\DefineNamedFunc{measureCompletionFunctor}{(X,\Sigma,\mu),(Y,T,\nu),\phi}
	{\mathsf{C}_{(X,\Sigma,\mu),(Y,T,\mu)}(\phi)}{\phi}
	\Explain{
		Consider $E \in \hat{T}$ such that $\hat nu (E) < \infty$}
	\Explain{
		Thus, $\nu^*(E) < \infty$ and there id $F \in T$ such that $E \subset F$ and $\nu(F) < \infty$}
	\Explain{
		But that $\phi^{-1}(E) \subset f^{-1}(F)$ and $\phi_*\mu(F) < \infty$
	}
	\Explain{ So $\phi_*\hat \mu(E) < \infty$}
	\Explain{ The same strategy works for $E \in \hat{T}$ with $\hat \nu(E) = 0$}
	\EndProof
	\\
	\Theorem{CompleteMeasuresPreservation}
	{
		\forall (X,\Sigma,\mu) : \CMS \. 
		\hat \Sigma_\mu = \Sigma
	}
	\Explain{ 
		It is obvious that $\Sigma \subset \hat \Sigma$}
	\Explain{ 
		Now, consider $E \in \hat \Sigma $}
	\Explain{
		Then, there are $F,G \in \Sigma$ such that $F \subset E \subset G$ and
		$\mu(G \setminus F) = 0$}
	\Explain{
		But then $E \setminus F \subset G \setminus F$ is $\Sigma$-measurable as $\mu$ is complete}
	\Explain{
		So, $E = (E \setminus F) \cup F \in \Sigma$}
	\EndProof
}\Page{
	\Theorem{OuterMeasuresPreservation}
	{
		\forall (X,\Sigma,\mu) \in \MEAS \.
		\mu^* = \hat \mu^*
	}
	\Explain{ 
		Clearly, $\hat \mu^* \le \mu^*$ }
	\Explain{
		Now consider $A \subset X$}
	\Explain{
		Then there exists a sequence $E \in \hat \Sigma$
		such that $\hat \mu^*(A) = \hat \mu(E) = \mu^*(E)$ and $A \subset E$}
	\Explain{
		But then there is a $F \in \Sigma$ such that $\mu^*(E) = \mu(A)$
	}
	\Explain{
		This shows $ \mu^*(A) \le \hat \mu^*(A)$ and proves equality}
	\EndProof
	\\
	\Theorem{NullPreservation}
	{
		\forall (X,\Sigma,\mu) \in \MEAS \.
		\Null_\mu = \Null_{\hat \mu}
	}
	\Explain{
		More or less trivial from equality of outer measures}
	\EndProof
	\\
	\Theorem{ThickPreservation}
	{
		\forall (X,\Sigma,\mu) \in \MEAS \.
		\Thick(X,\Sigma,\mu) = \Thick(X,\hat \Sigma,\hat \mu)
	}
	\Explain{ If $A$ is $\mu$-Thick, the $\mu^*(A \cap E) = \mu(E)$ for any
		$E \in \Sigma$	
	}
	\Explain{ 
		Now, consider $E \in \hat \Sigma$}
	\Explain{ Then there are $F,G \in \Sigma$ such that $F \subset E \subset G$ and 
		$\mu(G \setminus F) = 0$}
	\Explain{
		Then trivially 
		$
			\hat \mu(E) \ge 
			\hat \mu^*(E \cap A) = 
			\mu^*(E \cap A) \ge   \mu^*(F \cap A) = \mu(F) = 
			\mu(G) = \hat \mu(G) \ge \hat \mu(E)$
	}
	\Explain{  So $A$ is also $\hat \mu$-thick}
	\Explain{ If $A$ is $\hat \mu$-thick, then it obviously $\mu$-thick as $\Sigma \subset \hat \Sigma$}
	\EndProof
	\\
	\Theorem{CompletionIsUnique}
	{
		\forall (X,\Sigma,\mu) \in \MEAS \.
		\forall nu \in \Measure(X,\hat \Sigma) \.
		\forall \aleph : \nu_{|\Sigma} = \mu \.
		\nu = \hat \mu
	}
	\Explain{
		Assume $E \in \hat \Sigma$}
	\Explain{ 
		Then there are $F,G \in \Sigma$ such that $F \subset E \subset G$ and 
		$\mu(G \setminus F) = 0$}
	\Explain{ 
		But then $ \nu(G) = \hat \mu(G)= \mu(G) = \mu(F) = \hat \mu(F) = \nu(F)$}
	\Explain{
		So, $\nu(E) = \mu(G) = \hat \mu(E)$}
	\Explain{
		And measures are equal}
	\EndProof
	\\
	\Theorem{Decomposition}
	{
		\forall (X,\Sigma,\mu) \in \MEAS \. 
		\forall A \subset X \.
		A \in \hat \Sigma
		\iff
		\exists E \in \Sigma \. 
		\exists Z \in \Null_{\mu} \.
		A = E \du Z
	}
	\Explain{
		If $A \in \hat \Sigma$, there are $E,F \in \Sigma$ such that $E \subset A \subset F$ and 
		$\mu(F \setminus E) = 0$}
	\Explain{
		Then $\mu^*(F \setminus A) \le \mu( F \setminus E) = 0$, so we can take 
		$Z = F \setminus A$}
	\Explain{
		On the other hand $Z\in \hat \Sigma$ as $\hat \mu$ is complete  
	}
	\Explain{
		So $A=E \du Z \in \hat \Sigma$}
	\EndProof
}\Page{
	\Theorem{Measurability}
	{
		\forall (X,\Sigma,\mu) \in \MEAS \.
		\forall A \in \Null_\mu' \.
		\forall f : A \to \EReals \. \NewLine \.
		f \in \BOR^*_\mu\Big((X,\Sigma),\EReals\Big)
		\iff
		f \in \BOR_\mu\Big((X,\hat \Sigma),\EReals\Big)
	}
	\Explain{
		This is obvious as $\dom f \in \hat \Sigma$}
	\EndProof
	\\
	\Theorem{ExistanceOfIntegrals}
	{
		\forall (X,\Sigma,\mu) \in \MEAS \.
		\forall A \in \Null_\mu' \.
		\forall f : A \to \EReals \. \NewLine \.
		f \in \Integrable(X,\Sigma,\mu)
		\iff
		f \in \Integrable(X,\hat \Sigma,\hat \mu)
	}
	\Explain{
		If $\sigma(x) = \sum^n_{i=1} \alpha_i \delta_x(E_i)$ is a simple function for $\hat \mu$,
		one can select sets $F_i \in \Sigma$ such that $\mu^*(E_i \du F_i) = 0$
	}
	\Explain{
		Then $\sigma'(x) = \sum^n_{i=1} \alpha_i \delta_x(F_i)$ 
		is a simple function for $\mu$ and $\int \sigma \; d\hat \mu = \int \sigma' \; d\mu$}
	\Explain{
		Thus, existance of integrals is equivalent}
	\EndProof
	\\
	\Theorem{EqualIntegrals}
	{
		\forall (X,\Sigma,\mu) \in \MEAS \.
		\forall f \in \Integrable(X,\Sigma,\mu) \.
		\int f \; d\mu = \int f \; d\hat \mu
	}
	\Explain{
		Follows from previous argument}
	\EndProof
	\\
	\Theorem{Integrability}
	{
		\forall (X,\Sigma,\mu) \in \MEAS \.
		\forall A \in \Null_\mu' \.
		\forall f : A \to \EReals \. \NewLine \.
		f \in L^1(X,\Sigma,\mu)
		\iff
		f \in L^1(X,\hat \Sigma, \hat \mu) 
	}
	\Explain{
		Follows from previous argument
	}
	\EndProof
	\\
	\Theorem{ProbabilityEquivalence}
	{
		\forall (X,\Sigma,\mu) \in \MEAS \.
		\Probability(X,\Sigma,\mu)
		\iff
		\Probability(X,\hat \Sigma,\hat \mu)
	}
	\Explain{
		Obvious, as  $\hat \mu(X) = \mu(X)$
	}
	\EndProof
	\\
	\Theorem{FiniteEquivalence}
	{
		\forall (X,\Sigma,\mu) \in \MEAS \.
		\Finite(X,\Sigma,\mu)
		\iff
		\Finite(X,\hat \Sigma,\hat \mu)
	}
	\Explain{
		Obvious, as  $\hat \mu(X) = \mu(X)$
	}
	\EndProof
}\Page{
	\Theorem{SigmaFiniteEquivalence}
	{
		\forall (X,\Sigma,\mu) \in \MEAS \.
		\sFinite(X,\Sigma,\mu)
		\iff
		\sFinite(X,\hat \Sigma,\hat \mu)
	}
	\Explain{
		One direction is obvious: just use cover of $\mu$ for $\hat \mu$ also}
	\Explain{
		Assume, $E$ is a cover of $\hat \mu$}
	\Explain{
		So $E_n \in \hat \Sigma, \hat \mu(E_n) < \infty$ ans $X = \bigcup^\infty_{n=1} E_n$}
	\Explain{
		Select $F_n \in \Sigma$ for each $E_n$ such that $E_n \subset F_n$ and $\mu(F_n) = \hat \mu(E_n)$}
	\Explain{
		Then, $F$ is a cover for $\mu$}
	\EndProof
	\\
	\Theorem{SemifiniteEquivalence}
	{
		\forall (X,\Sigma,\mu) \in \MEAS \.
		\Semifinite(X,\Sigma,\mu)
		\iff
		\Semifinite(X,\hat \Sigma,\hat \mu)
	}
	\Explain{
		Assume $\mu$ is semifinite First}
	\Explain{
		Take $E \in \hat \Sigma$ to be such that $\hat \mu(E) = \infty$}
	\Explain{
		Then there is $F \in \Sigma$, such that  $\mu(F) = \infty$ and $F \subset E$}
	\Explain{
		By semifiniteness of $\mu$ there is $G \in \Sigma$ such that $G \subset F$
		and $0 < \mu(G) < \infty$.
	}
	\Explain{
		But then $G \subset E$ and $G \in \hat \Sigma$}
	\Explain{
		But as $E$ was arbitrary it means that $\hat \mu$ is semifinite}
	\Explain{
		Now, assume $\hat \mu$ is semifinite} 
	\Explain{
		Take $E \in \Sigma$ such that $\mu(E) = \infty$
	}
	\Explain{
		Then there are $F \in \hat \Sigma$ such that $0 <\hat \mu(F) < \infty $
		and $F \subset E$}
	\Explain{
		By definition of completion there is $G \in \Sigma$ such that
		$G \subset F$ and $\mu(G) = \hat \mu(F)  $ }
	\Explain{
		But this means that $G \subset E$ and $0 < \mu(G) < \infty$
	}
	\Explain{
		But as $E$ was arbitrary it means that $\mu$ is semifinite}
	\EndProof
}\Page{
	\Theorem{LocalizableEquivalence}
	{
		\forall (X,\Sigma,\mu) \in \MEAS \.
		\Loc(X,\Sigma,\mu)
		\iff
		\Loc(X,\hat \Sigma,\hat \mu)
	}
	\Explain{
		Firstly, assume $\mu$ is localizable}
	\Explain{
		Assume $\A \subset \hat \Sigma$ 
	}
	\ExplainFurther{
		Then costruct set $\B \subset \Sigma$ such for every $B \in \B$
		there is an $A \in \A$ such that $\hat \mu(A \du B) = 0$,}
	\Explain{ 
		and also for every $A \in \A$ there is such $B$}
	\Explain{
		Take $H = \esssup_\mu \B$}
	\Explain{
		Then $\hat \mu(A \setminus H) = \hat \mu(B \setminus H) = \mu(B \setminus H) = 0 $
		 for every $A \in \A$}
	\Explain{
		Assume $G \in \hat \Sigma$ is such that 
		for $\hat \mu(A \setminus G) = 0$every $A \in \A$}
	\Explain{
		Then there is $F \in \Sigma$ such that  $\hat \mu(F \du G) = 0$
	}
	\Explain{
		Then $\mu(B \setminus F) = \hat\mu(B \setminus F) = \hat \mu(A \setminus G) = 0$ 
	}
	\Explain{
		So, $\hat \mu(H \setminus G) = \hat \mu(H \setminus F) = \mu(H \setminus F) = 0$
	}
	\Explain{
		Thus, $H = \esssup \A$}
	\Explain{
		And as $\A$ was arbitrary, $\hat \mu$ is localizable
	}
	\Explain{
		Now, assume $\hat \mu$ is localizable}
	\Explain{
		Assume $\A \subset \Sigma$}
	\Explain{
		Take $H = \esssup_{\hat \mu} \A \in \hat \Sigma$}
	\Explain{
		By completion there is $F \in \Sigma$ such that $\hat \mu(H \du F) = 0$}
	\Explain{
		Then $\mu(A \setminus F) = \hat \mu(A \setminus F) = \hat \mu(A \setminus H) = 0$}
	\Explain{
		Also supose $G \in \Sigma$ such that $\mu(A \setminus G) = 0$ for every $A \in \A$}
	\Explain{
		Then, $ \mu(F \setminus G) = \hat \mu(F \setminus G) =\hat \mu(H \setminus G) = 0$
	}
	\Explain{
		But this means that $F = \esssup_\mu \A$
	}
	\Explain{
		So, as $\A$ was arbitraty, $\mu$ is localizable}
	\EndProof
	\\
	\Theorem{DecompositionPreservation}
	{
		\NewLine	::	
		\forall (X,\Sigma,\mu) \in \MEAS \.
		\forall \E : \Decomposition(X,\Sigma,\mu) \.
		\Decomposition(X,\hat \Sigma,\hat \mu,\E)
	}
	\Explain{
		Assume $A \subset X$ such that $E \cap A \in \hat \Sigma$ for every $E \in \E$}
	\Explain{
		For every $E \in \E$ select $F_E,G_E \in \Sigma$ such that 
		$\hat \mu\Big( G_E \setminus F_E\Big) = 0$
		and $F_E \subset E \cap A \subset G_E subset E$}
	\Explain{
		 $\bigcup_{E \in \E} F_E \cap E = F_E \in \Sigma$
		 and
		 $
		 	\bigcup_{E \in \E} G_E \cap E = G_E \in \Sigma
		 $
		 by construction}
	\Explain{
		So, by definition of decomposition $\bigcup_{E \in \E} F_E,\bigcup_{E \in \E} G_E \in \Sigma$}
	\Explain{
		Then  $\bigcup_{E \in \E} F_E \subset A \subset \bigcup_{E \in \E} G_E$
	}
	\Explain{
		Also 
		$
			\mu\left( \bigcup_{E \in \E} G_E \setminus \bigcup_{E \in \E} F_E \right) \le 
			\mu\left( \bigcup_{E \in \E} G_E \setminus F_E \right) = 
			\sum_{E \in \E} \mu(G_E \setminus F_E) 	 = 0	
		$
	} 
	\Explain{Thus, $A \in \hat \Sigma$}
	\Explain{
		With simmilar nomenclature
		$
			\hat \mu(A) = 
			\hat \mu\left( \bigcup_{E \in \E} G_E \right) = 
			\mu\left( \bigcup_{E \in \E} G_E \right) =
			\sum_{E \in \E} \mu(G_E) =
			\sum_{E \in \E}  \hat \mu(E \cap A)
		$}
	\Explain{
		So, indeed, $\E$ is a decomposition for $\hat \mu$
	}
	\EndProof
}\Page{
	\Theorem{StrictlyLocalizablePreservation}
	{
		\forall (X,\Sigma,\mu) : \SLoc \.
		\SLoc(X,\hat\Sigma,\hat\mu)
	}
	\Explain{
		Follows from previous result}
	\\
	\Theorem{AtomCriterion}
	{
		\NewLine ::		
		\forall (X,\Sigma,\mu) \in \MEAS \.
		\forall A \in \hat \Sigma \.
		A \in \Atom(X,\hat \Sigma,\hat \mu) 
		\iff
		\exists B \in \Atom(X,\sigma,\mu) \.
		\hat \mu(A \du B) = 0
	}
	\Explain{
		Firstly, assume that $A \in \Atom(X,\hat \Sigma,\hat \mu)$}
	\Explain{
		Then, there is $B \in \Sigma$ such that $B \subset A$
		such that $\hat \mu (A \setminus B) = 0$
	}
	\Explain{
		But then, $B$ also must be an atom as any subset of $B$ is also an subset of $A$}
	\Explain{
		Now assume just $A \in \hat \Sigma$ and that such $B$ exists}
	\Explain{
		Take some $E \in \hat \Sigma$ such that $E \subset A$}
	\Explain{
		Then there is $F \in \Sigma$ such that $\mu(E \du F) = 0$
	}
	\Explain{
		So $\mu(F \cap B) = \hat \mu(F \cap B) = \hat \mu(E \cap A) = \hat \mu(E)$,
		which is either $0$ or $\mu(B) = \hat \mu(A)$
	}
	\Explain{
		Thus, $A$ is an atom fo $\hat \mu$}
	\EndProof
	\\
	\Theorem{AtomlessEquivalence}
	{
		\forall (X,\Sigma,\mu) \in \MEAS \.
		\Aless(X,\Sigma,\mu)
		\iff
		\Aless(X,\hat \Sigma,\hat \mu)
	}
	\Explain{Follows straight from the theorem about atoms}
	\EndProof
	\\
	\Theorem{SigmaFiniteEquivalence}
	{
		\forall (X,\Sigma,\mu) \in \MEAS \.
		\PA(X,\Sigma,\mu)
		\iff
		\PA(X,\hat \Sigma,\hat \mu)
	}
	\Explain{
		Also, follows  from the theorem about atoms
	}
	\EndProof
}
\newpage
\subsubsection{Selecta}
\Page{
	\Theorem{MeasurableByFiniteSupersetDecomposition}
	{
		\NewLine ::		
		\forall (X,\Sigma,\hat \mu) : \CMS \.
		\forall E \in \Sigma \.
		\forall \aleph : \mu(E) < \infty \.
		\forall A \subset E \. \NewLine \.
		\forall \beth :  \mu^*(A) + \mu^*(E \setminus A) = \mu(E) \.
		A \in \Sigma	
	}
	\Explain{
		Take $F,G \in \Sigma$ such that $\mu(F) = \mu^*(A \cap E),\mu(G) = \mu^*(E \setminus A )$ 
		and $A \cap E \subset F$ and $E \setminus A \subset G$}
	\Explain{
		Then $\beth$ witnesses that $\mu(E) = \mu(F) + \mu(G)$}
	\Explain{
		So by $\aleph$ and difference furmula $\mu(E \setminus G) = \mu(F)$}
	\Explain{
		But $  (E \setminus G) \subset A \subset F$, 
		so $\mu( F \setminus A)  \le \mu\Big( F \setminus (E \setminus G) \Big) = 0 $}
	\Explain{
		As $\mu$ is complete $F \setminus A$ is measurable, so $A = F \du (F \setminus A)$ 
		is also measurable}
	\EndProof
	\\
	\Theorem{OuterMeasuresEquality}
	{
		\NewLine ::		
		\forall X \in \SET \.
		\forall (\Sigma,\mu),(T,\nu) : \Measure(X) \.
		\mu^* = \nu^*
		\iff
		\NewLine		
		\iff
		\bigg(
				\forall E \in \hat \Sigma \cup \hat T \.
				\Big( \hat\mu( E) < \infty | \hat \nu(E) < \infty \Big)
				\Imply
				E \in \hat \Sigma \cap \hat T 
				\And
				\hat \mu(E) = \hat \nu(E)
		\bigg)
		\iff
		\NewLine
		\iff
		\forall f \in L^1(X,\Sigma,\mu) \cap L^1(X,T,\nu) \. 
		f \in  L^1(X,\Sigma,\mu) \cup L^1(X,T,\nu) \And
		\int f \; d \mu = \int  f \; d \nu
	}
	\Explain{
		Firstly, assume $\mu^* = \nu^*$}
	\Explain{ 
		Take $E \in \hat \Sigma$  such that $\hat \mu(E)<\infty$}
	\Explain{
		Then, $\nu^*(E) =\mu^*(E) = \hat \mu(E) < \infty$
	}
	\Explain{
		So, there is $F \in T$ such that $E \subset F$ and $\nu(F) = \nu^*(E) < \infty$
	}
	\Explain{   
		As $E$ is $\mu^*$-measurable		
		$ 
			\nu^*(E)  + \nu^*(F \setminus E)  = 
			\mu^*(E) + \mu^*(F \setminus E)  = 
			\mu^*(F) =
			\nu^*(F) =
			\nu(F) = \hat \nu (F)
		$    
	}
	\Explain{
		As $\hat \nu$ is complete $E \in \hat T$ by theorem above, 
		and $\hat \mu (E) = \mu^*(E) = \nu^*(E) = \hat \nu(E)$}
	\Explain{
		This argument works symmetrically, so we proved $(1) \Imply (2)$}
	\Explain{
		Now assume this implication and take $f \in  L^1(X,\Sigma,\mu)$
	}
	\Explain{
		Then it must be virtually measurable for $\mu$}
	\Explain{
			So $\dom f$ is $\hat \mu$ measurable}
	\Explain{
		Then by assumption $\hat \nu \Big( f^{-1}_+(t,+\infty)\Big) < \infty$
		and $\hat \nu \Big( f^{-1}_-(t,+\infty)\Big) < \infty$ are defined for every $t \in \Reals_{++}$}
	\Explain{
		Also by assumption the set of simple functions agree both for $\hat \mu$ and $\hat \nu$
	}
	\Explain{
		And $f = \lim_{n \to \infty} \sigma_n$ can be computed as a limit of measurable functions 
	}
	\Explain{
		But the set of convergence must be measurable,
		so summing all up $f$ is $\hat \nu$-measurable and integrable}
	\Explain{
		But $L^1(X,T,\nu) = L^1(X,\hat T,\hat \nu)$ so we are done
	}
	\Explain{
		This argument works symmetrically, so we proved $(2) \Imply (3)$
	}
	\Explain{ Now assume the condition about integrals is true}
	\ExplainFurther{
		I will compute $\mu^*(A) = \overline{\int} \delta_x(A) \; d\mu(x) = 
			\inf \left\{  \int g d \mu   \bigg|  g \in \Integrable(X,\Sigma,\mu), \delta(A) \le g \right\}$} 
	\Explain{
		$=
			\inf \left\{  \int g d \nu   \bigg|  g \in \Integrable(X,T,\nu), \delta(A) \le g \right\} = 
			\overline{\int} \delta_x(A) \; d\nu(x)  =
			\nu^*(A)   
		$
	}
	\EndProof
}
\newpage
\subsection{Localization}
\subsubsection{Thick Decomposition}
\Page{
	\Theorem{ThickDecompostition}
	{
		\NewLine ::		
		\forall (X,\Sigma,\mu) : \SLoc
		\forall \aleph : \bigg(\forall n \in \Nat \. 
		\exists D : \TYPE{DisjointFamily}\Big( \{1,\ldots,n\}, \TYPE{Thick}(X,\Sigma,\mu) \Big) \bigg) \.
		\NewLine \.
		\exists D : \TYPE{DisjointSequence}\Big(\TYPE{Thick}(X,\Sigma,\mu) \Big) 
	}
	\NoProof
}
\newpage
\subsubsection{Semifinite Measures}
\Page{
	\DeclareFunc{finiteMeasure}{\prod (X,\Sigma,\mu) \in \MEAS \.  \Ideal(\Sigma)}
	\DefineNamedFunc{finiteMeasure}{}{\Sigma^f}{\Big\{ E \in \Sigma \Big| \mu(E) < \infty \Big\}}
	\\
	\Theorem{SemifiniteMeasureComputation}
	{
		\forall (X,\Sigma,\mu) : \Semifinite \. 
		\forall E \in \Sigma \.
		\mu(E) = \sup\Big\{ \mu(F) \Big| F \in \Sigma^f,  F \subset E  \Big\}	
	}
	\Explain{
		If $\mu(E) < \infty$ then we are done}
	\Explain{
		Consider  case $\mu(E) = \infty$	}
	\Explain{
		Define $\A = \Big\{  F : \Nat \uparrow \Sigma^f \Big| \forall n \in \Nat \. F_n \subset E    \Big\}$ }
	\Explain{
		As $\mu$ is semifinite $\A$ must be non-empty}
	\Explain{
		$E$ must contain some $F_1$ with $0 < \mu(F_1) < \infty $,
		then $\mu(E \setminus F_1) = 0$} 
	\Explain{ And we may select some $G \subset E \setminus F_1$ with $0 < \mu(G) < \infty $
		and let $F_2 = F_1 \cup G$ and go so on}
	\Explain{
		Assume $\alpha = \sup_{F \in \A} \lim_{n \to \infty} \mu(F_n) < \infty$
	}
	\Explain{
		Then there exists sequence of sequneces $F : \Nat \to \Nat \uparrow \Sigma$,
		such that $\alpha = \lim_{n \to \infty} \lim_{m \to \infty} \mu(F_{n,m})$}
	\Explain{
		Construct a new sequence $ G_n = \bigcup^n_{k=1} F_{k,n} \in \A$
		and take $H = \bigcup^\infty_{n=1} G_n$}
	\Explain{
		Then $\mu(H) = \lim_{n \to \infty} \mu(G_n) \le \alpha < \infty$
	}
	\Explain{
		So we can take $Z \subset E \setminus H$ with $0 < \mu(Z) < \infty$
	}
	\Explain{
		Then 
		$
			\lim_{n \to \infty} \mu(G_n \cup Z) =
			\mu(Z) + \lim_{n \to \infty} \mu(G_n) \ge 
			\mu(Z) + \lim_{n \to \infty} \mu(F_{n,n}) =
			\mu(Z) + \alpha > \alpha  
		$}
	\Explain{ 
		As $G_n \cup Z \in \A$ we produced a contradiction, so $\alpha = \infty = \mu(E)$}
	\EndProof
	\\
	\Theorem{FiniteIntegalApproximation}
	{
		\forall (X,\Sigma,\mu) \in \MEAS \.
		\forall f \in \Integrable_+\Big(X,\Sigma,\mu\Big) \.
		\int_X f  = \sup_{E \in \Sigma^f} \int_E f
	}
	\Explain{
		Note that every simple function is localized on a set of the finite measure	
	}
	\ExplainFurther{
		$\int f
		= \sup \left\{   \int g \bigg|  g \in \Simple(X,\Sigma,\mu), g \le_{ae} f\right\}  
		= \sup \left\{   \int_E g \bigg| g \in \Simple(X,\Sigma,\mu), g \le_{ae} f, E \in \Sigma^f \right\}=$}
	\Explain{  
		$= \sup_{F \in \Sigma^f} \left\{   \int_E g \bigg| g \in \Simple(X,\Sigma,\mu), g \le_{ae} f \right\}  
		= \sup_{F \in \sigma^f} \int_F f 
		$ }
}\Page{
	\Theorem{SemifiniteIntegrabilty}
	{
		\NewLine ::		
		\forall (X,\Sigma,\mu) : \Semifinite \.
		\forall f \in \BOR^*_\mu\Big(X,\EReals_+\Big) \. \NewLine \.
		f \in L^1(X,\Sigma,\mu) 
		\iff
		\sup \left\{   \int g \bigg|  g \in \Simple(X,\Sigma,\mu), g \le_{ae} f\right\} < \infty
	}
	\Explain{One implication is trivial}
	\Explain{ So assume that $\sup \left\{   \int g \bigg|  g \in \Simple(X,\Sigma,\mu), g \le_{ae} f\right\} < \infty$  }
	\Explain{
		Take some $t \in \EReals_{++}$ and consider the case when 
		$\mu\Big( f^{-1}_{|E}(t,+\infty] \Big) = \infty$
	}
	\Explain{
		Then it is possible to find $F_n$ with arbitraty large measure, say $n$, 
		such that $F \subset \mu\Big( f^{-1}_{|E}(t,+\infty] \Big)$
	}
	\Explain{
		But then $t \delta_x(F_n) \le f$ and so 
		$\sup \left\{   \int g \bigg|  g \in \Simple(X,\Sigma,\mu), g \le_{ae} f\right\} \ge tn \to \infty$, 
		which is impossible}
	\Explain{
		So $f$ must be integrable
	}
	\EndProof
}
\newpage
\subsubsection{Locally Determined Completion}
\Page{
	\Theorem{CLDCaratheodoryExtensionIsItself}
	{
		\NewLine ::		
		\forall (X,\Sigma,\mu) : \CMS \And \LocDet \.
		\Sigma_{\mu^*} = \Sigma
	}
	\Explain{ 
		Take $E \in \Sigma_{\mu^*}$, so $\forall A \subset X \. \mu^*(A) = \mu(E \cap A) + \mu(A \setminus E)$ }
	\Explain{
		Also take $F \in \Sigma^f$}
	\Explain{
		Then $\infty > \mu(F) = \mu^*(F) = \mu^*(F \cap E) + \mu^*(F \setminus E ) = 
			\mu^*(E \cap A) + \mu^*\Big(F \setminus (E \cap F ) \Big)$}
	\Explain{
		So,  as $\mu$ is complete we can assert that $AE\cap F \in \Sigma$}
	\Explain{
		But as $F$ was arbitrary and $\mu$ is locally determined $E \in \Sigma$}
	\EndProof
	\\
	\DeclareFunc{locallyDetermindeCompletion}
	{
		\MEAS \to \CMS \And \LocDet
	}
	\DefineNamedFunc{locallyDeterminedCompletion}
	{X,\Sigma,\mu}{(X,\tilde \Sigma,\tilde \mu)}
	{
		\NewLine \de		
		\Big( X, \{ H \subset X : \forall E \in \Sigma^f \. H \cap E \in \hat \Sigma \},    
		\Lambda H \in \tilde \Sigma = \sup \big\{  \hat \mu(H \cap E) | E \in \Sigma^f  \big\} \Big)
	}
	\Explain{1
		Firstly, we show that $\tilde \Sigma$ is $\sigma$-algebra}
	\Explain{
		Clearly by definition $\hat \Sigma \subset \tilde \Sigma$ so $X,\emptyset \in \tilde \Sigma$}
	\Explain{
		If $E \in \tilde \Sigma$, and $F \in \Sigma^f$ then $E \cap F \in \hat \Sigma$}
	\Explain{
		Then $F = \Big(E \cap F\Big) \du \Big(E^\c \cap F\Big)$, 
		so $E^\c \cap F \in \hat \Sigma$}
	\Explain{
		And as $F$ was arbitrary $E^\c \in \tilde\Sigma$}
	\Explain{ 
		If $E : \Nat \to \tilde \Sigma$ and 
		$ F \in \Sigma^f$ then 
		$F \cap \bigcap^\infty_{n=1} E_n  =  \bigcap^\infty_{n=1} (E_n \cap F) \in \hat \Sigma $}
	\Explain{
		So $\tilde \Sigma$ is $\sigma$-algebra}
	\Explain{
		From definition clearly $\tilde \mu(\emptyset) = 0$}
	\ExplainFurther{
		If $H : \Nat \to \tilde \Sigma$ is a disjoint family, then
		$
			\tilde \mu\left( \bigcup_{n=1} H_n \right) =
			\sup  \left\{  \hat \mu\left( \bigcup^\infty_{n=1 } H_n \cap E\right) \Bigg| E \in \Sigma^f  \right\} =
		$	
	}
	\Explain{
		$
			= \sup  \left\{ \sum^\infty_{n=1} 
			\hat \mu( H_n \cap E) \Big| E \in \Sigma^f  \right\}
			\le
			\sum^\infty_{n=1} \sup \big\{ \hat \mu(H_n \cap E) | E \in \Sigma^f \big\} =
			\sum^\infty_{n=1} \tilde \mu(H_n)
		$
	}
	\ExplainFurther{
		Assume the inequality above is strict}
	\Explain{
		So there must be some $m \in \Nat$
		such that 
		$
			\tilde \mu\left( \bigcup_{n=1} H_n \right)
			 < \sum^m_{n=1} \tilde \mu(H_n)
		$	}
	\Explain{
		Select some $E_{n,k} \in \Sigma^f$ producing supremums on the righthandside}
	\Explain{
		We can construct sets $F_k= \bigcup^m_{n=1} E_{n,k} \in \Sigma^f$}
	\Explain{
		Then   
		$
		\hat \mu\left( \bigcup_{n=1}^\infty H_n \cap F_k \right)
		=
		\sum^\infty_{n=1}	\hat \mu(H_n \cap F_k) \ge 
		\sum^m_{n=1} \hat \mu(H_n \cap E_{n,k})	
		$}
	\Exclaim{
		So by taking limit in $k$ we see that
		$
			\tilde \mu\left( \bigcup_{n=1} H_n \right)  \ge \sum^m_{n=1} \tilde \mu(H_n),
		$
		a contradiction}
	\Explain{
		So, $\tilde \mu$ is a measure}
}\Page{
	\Explain{Clearly, every null-set belongs to $\tilde \Sigma$ and has measure $0$}
	\Explain{ So $\tilde \mu$ is complete}
	\Explain{
		Consider set $E \in \tilde \Sigma$ such that $\tilde \mu(E) = \infty$	
	}
	\Explain{
		Then there must exist $F \in \Sigma_f$ such that $\hat \mu(E \cap F) > 0$
	}
	\Explain{
		But $ \tilde \mu(E \cap F)  =  \hat \mu(E \cap F) \le \hat \mu(E) = \mu(E) < \infty$}
	\Explain{
		So $\tilde \mu$ is semifinite}
	\Explain{
		As $\Sigma^f \subset \tilde \Sigma^f$ the measure $\tilde \mu$ is locally determined
		by construction}
	\EndProof
	\\
	\Theorem{CLDPreservesMeasurebility}
	{
		\forall (X,\Sigma,\mu) \in \MEAS \.
		\Sigma \subset \tilde \Sigma
	}
	\Explain{
		If $E\in\Sigma$ and 
		$F \in \Sigma^f $, then
		$ E \cap F \in \Sigma \subset \hat \Sigma$}
	\Explain{
		So  $E \in \tilde \Sigma$}
	\EndProof
	\\
	\Theorem{CLDPreservesFiniteMeasure}
	{
		\forall (X,\Sigma,\nu) \in \MEAS \.
		\forall E \in \Sigma^f \.
		\tilde \mu(E) = \mu(E)
	}
	\Explain{
		Use definition and monotonicity of measure}
	\Explain{		
		Then $\hat \mu(E \cap E) = \hat \mu(E)  = \mu(E)$	
	}
	\EndProof
	\\
	\Theorem{CLDPreservesFiniteOuterMeasure}
	{
		\forall (X,\Sigma,\nu) \in \MEAS \.
		\forall A \subset  X \.
		\mu^*(A) < \infty \Imply
		\tilde \mu^*(A) = \mu^*(A)
	}
	\Explain{  $\tilde \mu^*(A) \leq \mu^*$ as $\Sigma \subset \tilde \Sigma$}
	\Explain{
		So consider the case $\tilde \mu^*(A) < \mu^*(A)$}
	\Explain{
		Then there is an envelope  $E \in \Sigma$ such that
		$\infty > \mu^*(A) = \mu(E)$ and $A \subset E$}
	\Explain{
		Also consider an envelope  $F \in \tilde\Sigma$ such that
		$\mu(E) > \tilde \mu^*(A) = \tilde \mu(F)$ and $A \subset F$}
	\Explain
	{
		Then $A \subset F \cap E \in \hat \Sigma$
		and $\tilde \mu(F \cap E) \le \tilde \mu(F) < \mu(E) < \infty$
	}
	\Explain{
		So there exists a sequence $G : \Sigma$ such that 
		$A \subset F \cap E \subset G$ and $\mu(G) = \hat \mu(F \cap E) = \tilde \mu(F \cap E) < \mu(E)$
	}
	\Exclaim{
		But this shows that $\mu^*(A) \le \mu(G) < \mu(E) = \mu^*(A)$, a contradiction}
	\EndProof
	\\
	\Theorem{OuterMeasureIneq}
	{
		\forall (X,\Sigma,\mu) \in \MEAS \.
		\tilde \mu^* \le \mu^*
	}
	\Explain{
		If $\mu^*(A)$ is finite, then $\mu^*(A) = \tilde \mu^*(A)$}
	\Explain{
		So in case of inequality it must be the case that
		$\mu^*(A) = \infty$ and this value is maximal}
	\EndProof
	\\
	\Theorem{NullSetsPreservation}
	{
		\forall (X,\Sigma,\mu) \in \MEAS \.
		\Null_\mu = \Null_{\tilde \mu}
	}
	\Explain{
		Use tha fact that $A \in \Null_\mu$ iff $\mu^*(A) = 0$
	}
	\EndProof
}\Page{
	\Theorem{ConullSetsPreservation}
	{
		\forall (X,\Sigma,\mu) \in \MEAS \.
		\Null_\mu' = \Null_{\tilde \mu}'
	}
	\Explain{
		By duallity}
	\EndProof
	\\
	\Theorem{MeasureComputation}
	{
		\forall (X,\Sigma,\mu) \.
		\forall E \in \tilde \Sigma \.
		\tilde \mu(E) = \sup \Big\{ \mu(F) \Big| F \in \Sigma^f, F \subset E  \Big\}
	}
	\Explain{ By definition of $\tilde \mu$ there is a sequence of sets $G : \Nat \uparrow  \hat \Sigma$
		such that $\tilde \mu(E) = \lim_{n \to \infty} \hat \mu(G_n)$	}
	\Explain{
		Also $\hat \mu(G_n) < \infty$ and $G_n \subset E$ for every $n \in \Nat$}
	\Explain{
		By definition of $\hat \mu$ there is a sequence $F : \Nat \to \Sigma$ such that
		$F_n \subset G_n$ and $\hat \mu(G_n) = \mu(F_n)  $}
	\Explain{
		Then $F_n \subset E$ and $\mu(F_n) < \infty$ for each $n \in \Nat$}
	\Explain{
		And $\lim_{n \to \infty} \mu(F_n) = \lim_{n \to \infty} \hat \mu(G_n)  = \tilde \mu(E)$
	}
	\Explain{
		Clearly, $\mu(F) = \tilde \mu(F) \le \tilde \mu(E)$ for every such set $F \in \Sigma^f$
		with $F \subset E$, so the result follows}
	\EndProof
	\\
	\Theorem{ApproximationFromBelow}
	{
		\forall (X,\Sigma,\mu) \in \MEAS \.
		\forall E \in \tilde \Sigma \.
		\exists G \in \Sigma \.
		G \subset E \And \mu(G) = \tilde \mu(E)
	}
	\Explain{
		Take Sequence $F$ as in Previous Theorem}
	\Explain{
		Then $G = \bigcup^\infty_{n=1} F_n \in \Sigma$
		and $\mu(G) = \lim_{n \to \infty} \mu(F_n) = \tilde \mu(F_n)$}
	\Explain{
		Also $G \subset E$ as each $F_n \subset E$	
	}
	\EndProof
	\\
	\Theorem{Measurability}
	{
		\forall (X,\Sigma,\mu) \in \MEAS \.
		\forall f \in \BOR_\mu^*(X,\Reals) \.
		f \in \BOR_{\tilde \mu}(X,\Reals)
	}
	\Explain{ 
		$\tilde \mu$ is complete }
	\EndProof
	\\
	\Theorem{Integrability}
	{
		\forall (X,\Sigma,\mu) \in \MEAS \.
		\forall f \in L^1(X,\Sigma,\mu) \.
		f \in \L^1(X,\tilde \Sigma,\tilde \mu)
	}
	\Explain{
		Use equality on finite sets  to prove result on finite functions
	 }
	 \ExplainFurther{
	 	Then by monotonic convergence theorem and approximation from below} 
	 \Explain{
	 	it can be extended to positive functions}
	\EndProof
	\\
	\Theorem{IntegralEquality}
	{
		\forall (X,\Sigma,\mu) \in \MEAS \.
		\forall f \in L^1(X,\Sigma,\mu) \.
		\int f  \; d\mu = \int f \; d \tilde \mu
	}
	\Explain{ See Fremlin 213Gb}
	\NoProof
	\\
}\Page{
	\Theorem{
		InegrableApproximation
	}
	{
		\forall (X,\Sigma,\mu) \in \MEAS \.
		\forall f \in L^1(X,\tilde \Sigma,\tilde \mu) \.
		\exists \tilde f \in L^1(X,\Sigma,\mu) \.
		f =_{\ae \mu} \tilde f 
	}
	\Explain{
		Let $\sigma(x) = \sum^n_{k=1} \alpha_k \delta_x(E_k)$ be a simple function for $\tilde \mu$}
	\Explain{
		As $\tilde \mu(E_k) < \infty$ there must exist a set $F_k \in \Sigma^f$ with $\tilde \mu_k(E_k \du F_k) =0$}
	\Explain{
		Define $\tau(x) = \sum^n_{k=1} \alpha_k \delta_x(E_k) \in \Simple(X,\Sigma,\mu)$}
	\Explain{
		Then $\sigma = \tau$ evetywhere expect on the set $H \subset \bigcup^n_{k=1} E_k \du F_k$
	}
	\Explain{
		But $\tilde \mu(H) \le \sum^n_{k=1} \tilde \mu(E_k \du F_k) = 0$, so
		$\mu^*(H) = \tilde \mu^*(H) = \tilde \mu(H) = 0$}
	\Explain{
		Thus, $\sigma$ and $\tau$ agree almost everywhere}
	\Explain{
		Now, take $f \in L^1(X,\tilde\Sigma,\tilde\mu)$}
	\Explain{
		Then there is 
		an increasing sequence of simples 
		$\sigma : \Nat \to \Simple(X,\tilde \Sigma,\tilde\mu) $ such that
		$f =_{\ae \tilde \mu} \lim_{n \to \infty} \sigma_n $}
	\Explain{
		Then there is a sequence $\tau :  \Nat \to \Simple(X,\Sigma,\mu) $ constructed as above}
	\Explain{
		Then it is still increasing and bounded almost everywhere}
	\Explain{
		Moreover, there is also a common conegledgible set, where $\sigma_n = \tau_n$ for every $n \in \Nat$}
	\Explain{
		Thus, $\tau_n$ converge to $f$ almost everywhere}
	\Explain{
		So define $\tilde f = \lim_{n \to \infty} \tau_n$
	}
	\EndProof
}\Page{
	\\
	\Theorem{ProbabilityPreservation}
	{
		\forall (X,\Sigma,\mu) : \Probability \. 
		\Probability(X,\tilde \Sigma,\tilde \mu)
	}
	\Explain{
		Obvious, as  $\tilde \mu(X) = \mu(X)$
	}
	\EndProof
	\\
	\Theorem{FiniteEquivalence}
	{
		\forall (X,\Sigma,\mu) : \Finite(X,\Sigma,\mu)\.
		\Finite(X,\tilde \Sigma,\tilde \mu)
	}
	\Explain{
		Obvious, as  $\tilde \mu(X) = \mu(X)$
	}
	\EndProof
	\\
	\Theorem{SigmaFinitePreservation}
	{
		\forall (X,\Sigma,\mu) :
		\sFinite
		\.
		\sFinite(X,\tilde \Sigma,\tilde \mu)
	}
	\Explain{
		Just use cover of $\mu$ for $\tilde \mu$ also}
	\EndProof
	\\
	\Theorem{StriclyLocalizablePreservation}
	{
		\forall (X,\Sigma,\mu) : \SLoc
		\.
		\SLoc(X,\tilde \Sigma,\tilde \mu)
	}
	\Explain{
		Let $\E$ be a decomposition for $\mu$}
	\Explain{
		Assume $A \subset X$ is such that 
		$
			\forall E \in \E \. A \cap E \in \tilde \Sigma \.
		$}
	\Explain{
		Then for $A \cap E \cap F \in \hat \Sigma $ any $F\in \Sigma^f$ and $E \in \E$}
	\Explain{
		But this means that $A \cap F \in \hat \Sigma$ as $\E$ is also a decomposition 
		for $\hat \mu$}
	\Explain{
		As $F$ was arbitrary $A \in \tilde \Sigma$}
	\ExplainFurther{
		Also note that
		$
		 	\sum_{E \in \E} \tilde \mu(E \cap A) =
		 	\sum_{E \in \E} \mu(E \cap B)
		 	\mu(B) = \tilde \mu(A), 
		$}
	\Explain{ 
		if $\tilde \mu(A) < \infty$ and $B \in \Sigma^f$ is such that $\tilde \mu(A \du B) = 0$ anf $B \subset A$}
	\Explain{
		Otherwise the equality must Follow as there exists $F \in \Sigma$ with $F \subset A$
		with arbitrary large $\mu$-measure}
	\Explain{
		Say $\mu(F_n) \ge n$
	}
	\Explain{
		Then $\sum_{E \in \E} \tilde \mu(E \cap A) \ge \sum_{E \in \E} \tilde \mu(E \cap F_n) = 
		\sum_{E \in \E}  \mu(E \cap F_n) = \mu(F_n) = n\to\infty$
	}
	\Explain{
		So $\E$ is a decomposition}
	\EndProof
}\Page{
	\Theorem{LocalizablePreservation}
	{
		\forall (X,\Sigma,\mu) : \Loc
		\.
		\Loc(X,\tilde \Sigma,\tilde \mu)
	}
	\Explain{
		Assume $\A\subset \tilde \Sigma$}
	\Explain{
		Construct $\A' = \{ A \cap F | A \in \A, F \in \Sigma^f  \} \subset \hat \Sigma^f$}
	\Explain{
		For each $A \in \A'$ denote by $B_A$ its envelope in $\Sigma$, so
		$\hat \mu(A \du B_A) = 0$
	}
	\Explain{
		Then there exists $H = \esssup_{A \in \A'} B_A \in \Sigma$}
	\Explain{
		$\hat \mu\Big((A\setminus H) \cap F\Big) = \hat \mu( (A \cap F) \setminus H) 
		= \mu(B_{A \cap F} \setminus H) = 0$		
		for each $A \in \A$ and $F \in \Sigma^f$}
	\Explain{
		So, the $\tilde \mu(A \setminus H) = 0$ for all $A \in \A$
	}
	\Explain{
		Now assume $G \in \tilde \Sigma $ is such that $\tilde \mu(A \setminus G) = 0$ for all $A \in \A$}
	\Explain{
		Assume $F \in \Sigma^f$
	}
	\Explain{
		Then $F \cap G \in \hat \Sigma$ an there is envelope $E \in \Sigma^f$ such that
		$\hat \mu\Big( (F \cap G) \du E \Big) = 0 $
	}
	\Explain{
		If $A' \in \A'$ such that $A' \subset F$ then there is $C \in \Sigma^f$
		and $A \in \A$ such that $A' =  A \cap C$ 
	}
	\ExplainFurther{ 
		Then 
		$
			\mu(B_{A'} \setminus E) =
			\tilde \mu(B_{A'} \setminus E) =
			\tilde \mu(A' \setminus (F \cap G)) \le 
			\tilde \mu\Big( (A \cap F) \setminus (F \cap G) \Big)  =
			\tilde \mu\Big( (A\setminus G) \cap F \Big) = $}
	\Explain{
		$ =\tilde \mu\Big( A \setminus G \Big) = 0$}
	\Explain{
		So $ \mu\Big( (H \cap F) \setminus E \Big) = 0 $, otherwise $E \cup F^\c$ will violate the property of $H$
		being essential supremum}
	\Explain{But this means that 
		$
			\hat\mu\Big(  (H  \setminus G) \cap F  \Big) =
			\hat \mu\Big( (H \cap F)\setminus (G \cap F)   \Big) =
			\hat \mu\Big( (H \cap F) \setminus E \Big) = 
			\mu\Big( (H \cap F) \setminus  E \Big) = 0 
		$
	}
	\Explain{And as $F$ was arbitrary $\tilde \mu (H \setminus G) = 0$}
	\Explain{So $H = \esssup \A$}
	\EndProof
	\\
	\Theorem{LocalizableApproximation}
	{		
		\forall (X,\Sigma,\mu) : \Loc \.
		\forall E \in \tilde \Sigma \.
		\exists F \in \Sigma \.
		\tilde \mu(E \du F) = 0
	}
	\Explain{
		As we saw in the previous proof
		we can selecect $\esssup_{\tilde \mu}$ in $F$
	}
	\Explain{
		So, take $F = \esssup_{\tilde \mu} \{E\}$}
	\Explain{
		Thus, $\tilde \mu(F \setminus E) = 0$}
	\Explain{
		But also $\tilde \mu(E \setminus F) = 0$ as $\tilde \mu(E \setminus E) = \tilde \mu(\emptyset) = 0$
	}
	\Explain{
		So, $\tilde \mu(F \du E) = 0$}
	\EndProof
	\\
	\Theorem{SemifinitenessCondition}
	{
		\forall (X,\Sigma,\mu) \in \MEAS \.
		\Semifinite(X,\Sigma,\mu)
		\iff
		\forall F \in \Sigma \. 
		\mu(F) = \tilde \mu(F)
	}
	\Explain{
		Firstly, assume $(X,\Sigma,\mu)$ is semifinite}
	\Explain{
		Also assume $\mu(F) \neq \tilde \mu(F)$}
	\Explain{
			But then the only possibility is that $\mu(F) = \infty > \tilde \mu(F)$
		}
	\Explain{
		Then there exists $E \subset F$ such $\infty > \mu(E) > \tilde \mu(F)$
		as $\mu$ is semifinite}
	\Exclaim{ 
		But then $\tilde \mu(F) \ge \tilde \mu(E) = \mu(E) > \tilde \mu(F)$, a contradiction to the property
		of trichtomy}
	\Explain{
		Now, let the righthandside be true}
	\Explain{
		Let $E \in \Sigma$ be such that $\mu(E) = \infty$
	}
	\Explain{
		By assumption $\tilde \mu(E) = \infty$, but as $\tilde \mu$ is semifinite, 
		there id $F \in \tilde \Sigma$ such that $F \subset E$ and $0 < \tilde \mu(F) < \infty$}
	\Explain{
		Also, there must be $G \subset F$ such that $G \in \Sigma$ and $\mu(G) = \tilde \mu(F)$
	}
	\Explain{ Thus, $\mu$ is semifinite}
	\EndProof
}\Page{
	\Theorem{SemifiniteExistanceOfIntegrals}
	{
		\NewLine ::		
		\forall (X,\Sigma,\mu) : \Semifinite \.
		\forall f \in \F_\mu \.
		f \in \Integrable(X,\Sigma,\mu) 
		\iff
		f \in \Integrable(X, \tilde \Sigma, \tilde \mu )
	}
	\NoProof
	\\
	\Theorem{SemifinitefIntegralsEq}
	{
		\NewLine ::		
		\forall (X,\Sigma,\mu) : \Semifinite \.
		\forall f \in  \Integrable(X,\Sigma,\mu) \.
		\int f \; d\mu  = \int f \; d \tilde \mu 
	}
	\NoProof
	\\
	\Theorem{AtomCondition}
	{
		\NewLine ::
		\forall (X,\Sigma,\mu) \in \MEAS \.
		\forall A \in \tilde\Sigma \.
		A \in \Atom(X,\tilde\Sigma,\tilde\mu) 
		\iff \NewLine \iff
		\exists B \in \Atom(X,\Sigma,\mu) \.
		\tilde \mu(B \du A) = 0 \And \And \mu(B) < \infty
	}
	\Explain{
		Firstly, assume $A$ is an atom}
	\Explain{
		Then $\tilde \mu(A) < \infty$ as $\tilde \mu$ is semifinite}
	\Explain{
		Then there exists $B \subset A$ such that $\mu(B) = \tilde \mu(A) < \infty$}
	\Explain{
		But then $B$ must be an atom for $\mu$, otherwise $A$ is not an atom}
	\Explain{
		Now assume the righthandside holds}
	\Explain{
		Then $\tilde \mu(A) = \mu(B) < \infty$}
	\Explain{ 
		Assume $E \in \tilde \Sigma$ such that $E \subset A$
	}
	\Explain{
		Then $\tilde \mu(E) \le \tilde \mu(A) < \infty$, so there $F \in \Sigma$
		such that $\tilde \mu(E \du F) = 0$}
	\Explain{
		Then $\tilde \mu(E) \le \tilde \mu(A) < \infty$, so there $F \in \Sigma$
		such that $\tilde \mu(E \du F) = 0$}
	\Explain{
		But $ 
			\tilde \mu(E) = 
			\tilde \mu(E \cap A) =  
			\tilde \mu(B \cap  F) = 
			\mu(B \cap F) 
		$ which must be equall to $0$ or to $\mu(B) = \tilde \mu(A)$}
	\Explain{ So $A$ is an atom}
	\EndProof
	\\
	\Theorem{PurelyAtomicEquivalence}
	{
		\forall (X,\Sigma,\mu) \in \MEAS \.
		\PA(X,\Sigma,\mu) \iff \PA(X,\tilde \Sigma,\tilde \mu)
	}
	\NoProof
	\\
	\Theorem{AtomlessEquivalence}
	{
		\forall (X,\Sigma,\mu) \in \MEAS \.
		\Aless(X,\Sigma,\mu) \iff \Aless(X,\tilde \Sigma,\tilde \mu)
	}
	\NoProof
	\\
	\Theorem{CLDPreservation}
	{
		\NewLine		
		\forall (X,\Sigma,\mu) \in \MEAS \.
		\tilde \mu = \mu 
		\iff
		\CMS \And \LocDet(X,\Sigma,\mu)
	}
	\EndProof
}
\newpage
\subsubsection{Measures with Locally Determined Null Sets}
\Page{
	\DeclareType{\MwLDNS}{?\MEAS}
	\DefineType{(X,\Sigma,\mu)}{\MwLDNS}
	{
		\forall  A \subset X  \. 
		\exists_\mu A \Imply \exists E \in \Sigma^f \. 
		\exists_\mu  A \cap E	
	}
	\\
	\Theorem{StrictlyLocalizableHasLDNS}
	{
		\NewLine ::
		\forall (X,\Sigma,\mu) : \SLoc \.
		\MwLDNS(X,\Sigma,\mu)
	}
	\Explain{ 
		Take $\E$ be a decomposition of $\mu$}
	\Explain{
		If $A$ is such that $\mu^*(A \cap E) = 0$ for every $E \in \Sigma^f$, then
		$\mu^*(A \cap E)$ for every $E \in \E$}
	\Explain{
		So, define $F_E \in \Sigma$ to be such that $A \cap E \subset F_E$ and $\mu(F_E)=0$ for every $E \in \E$}
	\Explain{
		Then $G = \bigcup_{E \in \E} F_E \cap E$ is measurable as $G \cap E = F_E \cap E \in \Sigma$ and 
		$A \subset G$ as $A \cap E \subset F_E \cap E$ for $E \in \E$}
	\Explain{ 
		Also $\mu(G) = \sum_{E \in \E} \mu(G \cap E) = \sum_{E \in \E} \mu(F_E \cap E) 
		\le \sum_{E \in \E} \mu(F_E) = 0$}
	\Explain{
		So $\mu(G) = 0$ and $A$ is null set}
	\EndProof
	\\
	\Theorem{CompleteAndLocallyDeterminedHasLDNS}
	{
		\NewLine ::
		\forall (X,\Sigma,\mu) : \CMS \And \LocDet \. \NewLine \.
		\MwLDNS(X,\Sigma,\mu)
	}
	\Explain{
		If $A$ is such that $\mu^*(A \cap E) = 0$ for every $E \in \Sigma^f$, then
		$A \cap E \in \Sigma$ for every $E \in \Sigma^f$ as $mu$ is complete}
	\Explain{
		So $A \in \Sigma$ itself as $\mu$ is locally determined}
	\ExplainFurther{
		Recall that complete locally determined measure can be determined as supremum,}
	\Explain{
		so
		$
			\mu^*(A) = \mu(A) = \sup \Big\{ \mu(E) | E \in \Sigma^f , E \subset A \Big\} = 0
		$ and $A$ is null set}
	\EndProof
	\\
	\Theorem{LDNSEssSupLemma}
	{
		\NewLine ::		
		\forall (X,\Sigma,\mu) : \MwLDNS \.
		\forall \A \subset \Sigma \.
		\forall H = \esssup \A \.
		\neg \exists_\mu H \setminus \bigcup \A 
	}
	\Explain{ 
		Consider $F \in \Sigma^f$}
	\Explain{
		Then there is a measurable envelope $E$ for $B = F \cap \left(H \setminus \bigcup \A\right)$	
		as $F$ forms a cover for $B$}
	\Explain{
		Then $\mu(A \setminus E^\c) = \mu(A \cap V) = 
		\mu^*\left(  A \cap F \cap \left(H \setminus \bigcup \A\right) \right) =\mu^*(\emptyset) =0$
		for any $A \in \A$	
	}
	\Explain{
		So, by definition of essential supremum 
		$ 0 = \mu(H \setminus  E^\c) = \mu(H \cap E) \ge \mu^*(B)$}
	\Explain{
		Thus, $\neg \exists_\mu H \setminus \bigcup \A $ 
		as $\mu$ has locally determined null sets}
	\EndProof
}\Page{
	\Theorem{LocalizableHasMeasurableEnvelope}
	{
		\NewLine ::		
		\forall (X,\Sigma,\mu) : \Loc \And \MwLDNS \.
		\forall A \subset X \. \NewLine \.
		\exists \ME(X, \Sigma, \mu, A)
	}
	\Explain{
		Define $\E=\Big\{  E \in \Sigma^f : \mu^*(A \cap E) = \mu(E) \Big\}$ and $H = \esssup \E$}
	\Explain{
	}
	\NoProof
	\\
	\Theorem{SigmaFiniteHasMeasurableEnvelopes}
	{
		\forall (X,\mu)  : \sFinite \.
		\forall A \subset X \.
		\exists \ME(X,\mu,A)
	}
	\NoProof
	\\
	\Theorem{MeasurableEnvelopeOfLocalizableSpace}
	{
		\forall (X,\mu)  : \Loc \.
		\forall A \subset X \.
		\exists \ME(X,\tilde \mu,A)
	}
	\NoProof
}
\newpage
\subsubsection{Global Representative}
\Page{
	\Theorem{LocalizableHasGlobalRepresentative}
	{
		\NewLine ::		
		\forall (X,\Sigma,\mu) : \Loc \.
		\forall \E \subset \Sigma \.
		\forall f :\prod_{E \in \E} \BOR\Big( (E,\Sigma|E)\Reals\Big)  \. \NewLine \.
		\forall \aleph : \forall E,F \in \E \.    f_{E|E \cap F} =_{\ae \mu} f_{F|E\cap F} \.
		\exists g \in \BOR\Big( (X,\Sigma), \Reals\Big) \.
		\forall E \in \E \. g_{|E} =_{\ae \mu} f_{E} 
	}
	\NoProof
}
\newpage
\subsubsection{Strictly Localizable Measures}
\Page{
	\Theorem{StrictlyLocalizabilityCriterion}
	{
		\NewLine ::		
		\forall (X,\Sigma,\mu) : \CMS \And \LocDet \. 
		\forall \E : \TYPE{PairwiseDisjoint}(X,\Sigma^f) \. \NewLine \.
		\forall \aleph : \forall F \in \Sigma^f \. \exists E \in \E \. \mu(E \cap F) > 0  \.
		\SLoc(X,\Sigma,\mu) 
	}
	\NoProof
}
\newpage
\subsection{Submeasures}
\subsubsection{General Submeasures}
\Page{
	\DeclareFunc{submeasure}
	{
		\prod (X,\Sigma,\mu) \in \MEAS \.2^X \to \MEAS	
	}
	\DefineNamedFunc{submeasure}{Y}{(Y,\Sigma|Y,\mu|Y)}{(Y,\Sigma|Y,\mu^*_{|\Sigma|Y})}
	\\
	\Theorem{SubmeasureRepresentation}
	{
		\forall (X,\Sigma,\mu) \in \MEAS \.
		\forall Y \subset X \.
		\forall E \in \Sigma|Y \. 
		\exists F \in \Sigma \.
		\mu(E|Y) = \mu(F) 
	}
	\NoProof
	\\
	\Theorem{NullSetPreservation}
	{
		\forall (X,\Sigma,\mu) \in \MEAS \.
		\forall Y \subset X \.
		\forall A \subset Y \.
		A \in \Null_{\mu|Y} \iff A \in \Null_{\mu}
	}
	\NoProof
	\\
	\Theorem{ConullSetPreservation1}
	{
		\forall (X,\Sigma,\mu) \in \MEAS \.
		\forall Y \subset X \.
		\forall A \subset X \.
		\forall_{\mu} A  \Imply 
		\forall_{\mu|Y} A \cap Y
	}
	\NoProof
	\\
	\Theorem{ConullSetPreservation1}
	{
		\forall (X,\Sigma,\mu) \in \MEAS \.
		\forall Y \subset X \.
		\forall A \subset X \.
		\forall_{\mu} A  \Imply 
		\forall_{\mu|Y} A \cap Y
	}
	\NoProof
	\\
	\Theorem{ConullSetPreservation2}
	{
		\forall (X,\Sigma,\mu) \in \MEAS \.
		\forall Y \subset X \.
		\forall A \subset X \.
		\forall_{\mu|Y} A  \Imply 
		\forall_{\mu} A \cup Y^\c
	}
	\NoProof
	\\
	\Theorem{OuterSubmeasure}
	{
		\forall (X,\Sigma,\mu) \in \MEAS \.
		\forall Y \subset X \. 
		(\mu|Y)^* = \mu^*_{|Y}
	}
	\NoProof
	\\
	\Theorem{DoubleSubmeasure}
	{
		\forall (X,\Sigma,\mu) \in \MEAS \.
		\forall Y \subset X \. 
		\forall Z \subset Y \.
		(X,\Sigma|Y|Z,\mu|Y|Z)  = (X,\Sigma|Z,\mu|Z)
	}
	\NoProof
}
\newpage
\subsubsection{Integration}
\Page{
	\DeclareFunc{subsetIntegral}
	{
		\prod (X,\Sigma,\mu) \in \MEAS \.
		\Integrable(X,\Sigma,\mu) \times 2^X \to \EReals
	}
	\DefineNamedFunc{subsetIntegral}{f,Y}{\int_Y f(y) \; d\mu(y) }
	{
		\int_Y f(y) \; d \mu(y|Y)	
	}
	\\
	\Theorem{IntegralExistancePreservation}
	{
		\forall (X,\Sigma,\mu) \in \MEAS \.
		\forall f \in \Integrable(X,\Sigma,\mu) \.
		\forall Y \subset X \.
		f_{|Y} \in \Integrable(X,\Sigma|Y,\mu|Y)
	}
	\Explain{
		If $\sigma_n(x) = \sum^{k_n}_{i=1} \alpha_{n,i} \delta_x(E_{n,i}) $
		is a sequence of somples converging to $f$ from below,
		Then define $F_{i,n} = E_{n,i} \cap Y$ }
	\Explain{
		Construct $\tau_n(X) = \sum^{k_n}_{i=1} \alpha_{n,i} \delta_x(F_{n,i}) = \sigma_{n|Y}(x)$
	}
	\Explain{
		Then $\tau_n \uparrow f_{|Y}$, so $f_{|Y}$ has integral}
	\\
	\Theorem{SubsetIntegralInequality}{
		\forall (X,\Sigma, \mu) \in \MEAS \.
		\forall Y \subset X \.	
		\forall f \in \Integrable_+(X,\Sigma, \mu) \.
		\int_Y f \leq \int_X f
	}
	\Explain{Obvious for simple functions, then the result follows}
	\EndProof
	\\
	\Theorem{IntegrabilityPreservation}{
		\forall (X,\Sigma, \mu) \in \MEAS \.
		\forall Y \subset X \.	
		\forall f \in  L^1(X,\Sigma, \mu) \.
		f \in L^1(X,\Sigma|Y,\mu|Y)
	}
	\Explain{Follows from previous inequality}
	\EndProof
	\\
	\Theorem{EnvelopingExtenstionExists}
	{
		\NewLine		
		\forall (X,\Sigma,\mu) \in \MEAS \.
		\forall Y \subset X \.
		\forall f \in L^1(Y,\Sigma|Y,\mu|Y) \.
		\exists \tilde f \in L^1(X,\Sigma,Y) \.
		\forall F \in \Sigma \.
		\int_F \tilde f = \int_{Y \cap F} f
	}
	\NoProof
	\\
	\Theorem{SubsetIntegralEqCondition}
	{
		\NewLine ::		
		\forall (X,\Sigma,\mu) \in \MEAS \.
		\forall f \in \Integrable(X,\Sigma,\mu) \.
		\forall Y \subset X \
		\Big( \Thick(X,\Sigma,E,Y) \Big| f_{X\setminus Y} =_{\ae} 0 \Big)
		\Imply
		\int_Y f = \int_X f
	}
	\NoProof
	\\
	\Theorem{IntegralEqByMeasurableEnvelopes}
	{
		\NewLine ::		
		\forall (X,\Sigma,\mu) \in \MEAS \.
		\forall Y \subset X \.
		\forall E : \ME(X,\Sigma,\mu)
		\forall f \in \Integrable(E,\Sigma|E,\mu|E) \.
		\int_Y f = \int_E f
	}
}
\newpage
\subsubsection{Caratheodory Extension}
\Page{
	\Theorem{CaratheodoryExtensionSubsets}
	{
		\NewLine ::	
		\forall X \in \SET \.
		\forall Y \subset X \.
		\forall \theta : \OM(X) \.
		\Sigma_\theta|Y \subset \Sigma_{\theta|Y}
	}
	\NoProof
	\\
	\Theorem{CaratheodoryExtensionInequality}
	{
		\forall X \in \SET \.
		\forall Y \subset X \.
		\forall \theta : \OM(X) \.
		 \NewLine \.
		\forall E \in \Sigma_\theta|Y \.
		\theta_{|\Sigma_\theta}(E|Y) \le (\theta|Y)_{|\Sigma_{\theta|Y}}(E)
	}
	\NoProof
	\\
	\Theorem{CaratheodoryExtensionEq}
	{
		\forall X \in \SET \.
		\forall Y \subset X \.
		\forall \theta : \OM(X) \.
	     \NewLine \.
		\forall E \in \Sigma_\theta|Y \cap \Sigma \.
		\theta_{|\Sigma_\theta}(E|Y) = (\theta|Y)_{|\Sigma_{\theta|Y}}(E)
	}
	\NoProof
}
\newpage
\subsubsection{Lower and Upper Integrals}
\Page{
	\Theorem{UpperIntegralIneq1}
	{
		\forall (X,\Sigma,\mu) \in \MEAS \.
		\forall Y \subset X \.
		\forall f \in \F_\mu \.
		\forall \aleph : f \ge_{\ae \mu} 0 \.
		\overline{\int}_Y f \le \overline{\int}_X f
	}
	\Explain{ 
		If $\overline{\int}_X f = \infty$ the the result is obvious}
	\Explain{ 
		Otherwise there is integrable $g$ such that $g \ge_{\ae \mu} f$ and $\overline{\int}_X f = \int_X g$}
	\Explain{
		But then $\overline{\int}_Y f \le \int_Y g \le \int_X g = \overline{\int}_X f$}
	\Explain{
		Here we used $\aleph$ to prove second inequality
	}
	\EndProof
	\\
	\Theorem{UpperIntegralIneq1}
	{
		\forall (X,\Sigma,\mu) \in \MEAS \.
		\forall Y : \Thick(X) \.
		\forall f \in \F_\mu \.
		\overline{\int}_Y f \le \overline{\int}_X f
	}
	\Explain{
		Replace $\aleph$ by thickness for second inequality
	}
	\EndProof
	\\
	\Theorem{LowerIntegralIneq}
	{
		\forall (X,\Sigma,\mu) \in \MEAS \.
		\forall Y : \TYPE{Thick}(X) \.
		\forall f \in \F_\mu \.
		\underline{\int}_X f \le \underline{\int}_Y f
	}
}
\newpage
\subsubsection{Direct Sums}
\Page{
	\DeclareFunc{directSum}{\prod I \in \SET \. (I \to \MEAS) \to \MEAS}
	\DefineNamedFunc{directSum}{(X,\Sigma,\mu)}{\coprod_{i \in I} (X_i,\Sigma_i,\mu_i)}
	{
		\left( 
			\bigsqcup_{i \in I} X_i, 
			\left\{A  \subset \bigsqcup_{i \in I} X_i : \forall i \in I \. A \cap X_i \in \Sigma_i \right\},
			E \mapsto \sum_{i \in I} \mu_i(E \cap X_i ) 
		\right)	
	}
	\\
	\Theorem{MeasurableCoproduct}
	{
		\NewLine :: 		
		\forall I \in \SET \.
		\forall (X, \Sigma, \mu) : I \to \MEAS \.
		\forall f : \prod_{i \in \I} \BOR_{\mu_i}(X_i) \.
		\coprod_{i \in I} f_i \in \BOR\left( \prod_{i \in I} (X_i,\mu_i) \right)
	}
	\Explain{ 
		Assume $B$ is a real Borel set}
	\Explain{ 
		Then $\left(\coprod_{i \in I} f_i\right)^{-1}(B) = \bigsqcup_{i \in I} f_i^{-1}(B)$}
	\Explain{
		So $X_i \cap \left(\coprod_{i \in I} f_i\right)^{-1}(B) = f_i^{-1}(B) \in \Sigma_i $}
	\Explain{
		But this means that $\left(\coprod_{i \in I} f_i\right)^{-1}(B)$ is measurable for the whole direct sum}
	\EndProof
	\\
	\Theorem{CoproductIntegral}
	{
		\NewLine :: 		
		\forall I \in \SET \.
		\forall (X, \Sigma, \mu) : I \to \MEAS \.
		\forall f : \prod_{i \in \I} \Integrable_+(X_i,\Sigma,\mu_i) \.
		\int \coprod_{i \in I} f_i = \sum_{i \in I} \int f_i 
	}
	\Explain{
		This result is obvious for indicators and, hence,  simple functions}
	\Explain{
		Then use sdandard formula for Lebesgue's Integral and monotonic convergence theorem
	}
	\EndProof
}
\newpage
\subsubsection{Lattices and Ideals}
\newpage
\subsection{The Principle of Exhaustion}
\subsubsection{Subject}
\Page{
	\Theorem{Construction}
	{
		\forall (X,\Sigma,\mu) \in \MEAS \.
		\forall \E \subset \Sigma \.
		\forall \alpha : \Big(\forall E : \Nat \uparrow \E \.
		\lim_{n \to \infty} \mu(E_n) < \infty \Big) \.
		\forall \beth : \E \neq \emptyset \.
		 \NewLine \.
		\exists F : \Nat \uparrow \E \.
		\forall E \in \E \.
		\Big( 
			\exists n \in \Nat \. \forall G \in \E \. E \cup F_n \not \subset G \Big)
		\Big|
		\Big(
			\lim_{n \to \infty} \mu(E \setminus F_n) =  0
		\Big)  
	}
	\SayIn{F_0}{\Elim \beth}{\E}
	\AssumeIn{n}{\Nat}
	\Say{\F_n}{\{ E \in \E : F_{n-1} \subset E \}}{?\E}
	\Say{[1]}{\Elim \F_n \Elim \TYPE{Reflexive}(\E,\subset)}{\F_n \neq \emptyset}
	\Say{u_n}{\sup_{E \in \F_n} \mu(E) }{\EReals_{+}}
	\Say{\Big(F_n,[2]\Big)}{\Elim u_n \Elim \sup\Big( \min(n,u_n-2^{-n})\Big)}
	{
		\sum F_n \in \F_n \. \mu(F_n) \ge \min(n,u_n-2^{-n})
	}
	\Conclude{[n.*]}{\Elim F_n \Elim \F_n}{F_{n-1} \subset F_n}
	\Derive{\Big(\F,u,F,[1]\Big)}{\Intro \prod}
	{
		\sum \F : \Nat \downarrow ?\E \.
		\sum u \Nat \downarrow \EReals \.
		\sum F : \Nat \uparrow \E \. \NewLine \.
		\forall n \in \Nat \.
		F_n \in \F_n  \And 
		u_n = 	\sup_{E \in \F_n} \mu(E) \And
		\mu(F_n) \ge \min(n,u_n)		
	}
	\Say{[2]}{\THM{MonotonicSup}[1.2]}
	{
		\TYPE{Decreasing}(\Nat,\EReals_+,u)
	}
	\Say{[3]}{\THM{BoundedMonotonicConvergence}[2]}
	{
		\TYPE{Converging}(\EReals_+,u)
	}
	\SayIn{t}{\lim_{n\to\infty} u_n}{\EReals}
	\Say{[4]}{\Lambda n \in \Nat \. \Elim t [1.3](n) [1.2][2]}
	{
		\forall n \in \Nat \. 
		\min(n,t - 2^{-n}) \le \min(n,u_n -2^{-n}) \le \mu(F_n) \le u_n
	}
	\Say{[5]}{\THM{LimIneq}[4]}{t \le \lim_{n \to \infty} \mu(F_n) \le t}
	\Say{[6]}{\THM{DoubleIneqLemma}[5]}{\lim_{n \to \infty} \mu(F_n) = t}
	\Say{[7]}{\THM{LowerContinuity}(X,\Sigma,\mu)[6]}{\mu\left( \bigcup^\infty_{n=1} F_n \right) = t}
	\Say{[8]}{\aleph [7]}{t < \infty}	
	\AssumeIn{E}{\E}
	\Assume{[9]}{\forall n \in \Nat \. \exists G \in \E \. E \cup F_n \subset G}
	\Say{[10]}{ \Lambda n \in \Nat \. \Elim F_n \Elim \F_n [1.2](n)  }
	{\forall n \in \Nat \. \mu(E \cup F_n) \le u_n }
	\Say{[11]}{\THM{LowerContnuity}(X,\Sigma,\mu)[10]\THM{LimitIneq}\Intro t}
	{
		\mu\left(E \cup \bigcup^\infty_{n=1} F_n\right) \le t
	}
	\Conclude{[9.*]}{\THM{UpperContinuity}(X,\Sigma,\mu)\THM{DifferenceFormula}[7][8][11]}
	{
		\NewLine ::		
		\lim_{n\to \infty} \mu(E \setminus F_n) =		
		\mu\left( E \setminus \bigcup^\infty_{n=1} F_n \right) =
		\mu\left(E \cup \bigcup^\infty_{n=1} F_n  \right) - \mu\left(\bigcup^\infty_{n=1} F_n \right) =
		t - t = 0
	}
	\Derive{[9]}{\Intro \Imply}
	{
		\Big(\forall n \in \Nat \. \exists G \in \E \. E \cup F_n \subset G\Big)
		\Imply
		\lim_{n\to \infty} \mu(E \setminus F_n) =	 0
	}
	\EndProof
}
\Page{
		\Theorem{EssSupExists}
	{
		\forall (X,\Sigma,\mu) \in \MEAS \.
		\forall \E  : \TYPE{UpwardsDirected}( \Sigma ) \.
		\forall \alpha : \Big(\forall E : \Nat \uparrow \E \.
		\lim_{n \to \infty} \mu(E_n) < \infty \Big) \.
		 \NewLine \.
		\forall \beth : \E \neq \emptyset \.
		\exists F : \Nat \uparrow \E \.
		\bigcup^\infty_{n=1} F_n = \esssup \E
	}
	\Explain{
		Take $F$ as in previous theorem}
	\Explain{
		Then there is always exist $G \in \E$ such that $F_n \cap E \subset G$ 
		for any $n \in \Nat$ and $E \in \E$}
	\Explain{ 
		So, by the previous theorem $\mu\left(E \setminus \bigcup^\infty_{n=1}  \right) = 0$ for any $E \in \E$}
	\Explain{
		Now choose $G$ to be such that $\mu(E \setminus G) = 0$ for any $E \in \E$
	}
	\Explain{
		Then $\mu(F_n \setminus G) =0$ for any $n$}
	\Explain{
		But this means that by lower continuity 
		$\mu\left( \bigcup^\infty_{n=1} F_n \setminus G\right) = \lim_{n \to \infty} \mu(F_n \setminus G) = 0$}
	\Explain{
		So, indeed $\bigcup^\infty_{n=1} F_n = \esssup \E$}
	\EndProof
}
\newpage
\subsubsection{$\sigma$-Finite Measures}
\Page{
	\Theorem{SigmaFiniteEqDef}
	{
		\NewLine		
		\forall (X,\Sigma,\mu) : \Semifinite \.
		\sFinite(X,\Sigma,\mu)
		\iff \NewLine (1)\iff
		\Big(\exists f \in L^1(X,\Sigma,\mu) \. f > 0 \Big)
		(2)\iff \NewLine \iff
		\Big(\mu=0\Big|\exists P : \Probability(X,\Sigma,\mu) \. \Null_P = \Null_\mu \Big) 
		(3)\iff \NewLine \iff
		\forall 
			\Big(
				\E \subset \Sigma \. 
				\forall \aleph : \E \neq \emptyset \.
				\exists F : \Nat \uparrow \E \.
				\forall E \in \E \.
				(\forall n \in \Nat \. F_n \subset E)
				\Imply  \lim_{n \to \infty} \mu(E \setminus F_n) = 0 
			\Big)
		(4)\iff \NewLine \iff
		\Big(
				\E : \TYPE{UpwardDirected}(\Sigma) \. 
				\forall \aleph : \E \neq \emptyset \.
				\exists F : \Nat \uparrow \E \.
				\forall E \in \E \.
				\lim_{n \to \infty} \mu(E \setminus F_n) = 0 
			\Big)
		(5)\iff \NewLine \iff
		\left(
			\forall \E \subset \Sigma \.
			\exists \E' : \Countable(\E) \.
			\forall E \in \E \.
			\mu\left( E \setminus \bigcup \E' \right) = 0
		\right)
		(6)\iff \NewLine \iff
		\Big(
			\forall D : \TYPE{PairwiseDisjoint}(\Sigma \setminus \Null_\mu) \.
			\Countable(\Sigma, D) 
		\Big)
		(7)\iff \NewLine \iff
		\Big(
			\forall D : \TYPE{PairwiseDisjoint}(\Sigma^f \setminus \Null_\mu) \.
			\Countable(\Sigma, D) 
		\Big)(8)
	}
	\Explain{
		$(1) \Imply (2):$  Let $F$ be a finite measure partition of $\mu$}
	\Explain{
		For $x \in F_n$ define $f(x) = (2^n\mu(F_n))^{-1}$ if $\mu(F_n) > 0$, 
		otherwise set $f(x) = 1$}
	\Explain{
		Then $f$ is measurable and by direct product formula 
		$\int f \le \sum^\infty_{n=1} 2^{-n} = 1$	
	}
	\Explain{
		$(2) \Imply (3):$	Let $f$ be $\mu$-integrable and strictly positive}
	\Explain{
		We want to show that if $\mu(E) > 0$, then $\int_E f > 0$}
	\Explain{
		Note that $E =  \bigcup^\infty_{n=1} E \cap f^{-1}(n^{-1},+\infty)$ as $f>0$}
	\Explain{
		Thus there exists some $t \in \Reals_{++}$ such that 
		$\mu\Big( E \cap f^{-1}(t,+\infty)\Big) > 0$ }
	\Explain{
		But then
		$ \int_E f \ge  t \mu\Big( E \cap f^{-1}(t,+\infty)\Big) > 0 $}
	\Explain{
		So set $P(E) = \frac{\int_E f}{\int f}$, theb $P$ is a probability and has same null sets as $\mu$}
	\Explain{
		$(3) \Imply (4):$ If $\mu=0$, then the result is trivial}
	\Explain{
		Take $P$ to be an equivalent probability}
	\Explain{
		then, clearly $\lim_{n \to \infty} P(E_n) \le 1$ for any $E:\Nat \uparrow \E$ as $P$ is a probabilty}
	\ExplainFurther{
		So, the principle of exhaustion works so there is $F : \Nat \uparrow \E$ such that}
	\Explain{
				$\forall E \in \E \.
				(\forall n \in \Nat \. F_n \subset E)
				\Imply  P\left(E \setminus \bigcup^\infty_{n=1} F_n\right) = 0 $}
	\Explain{
		But as $\mu$ and $P$ share null sets the result follows}
	\Explain{
		$(4) \Imply (5):$ this works as with principle of exhaustion
	}
}\Page{
	\Explain{
		$(5) \Imply (6):$ contruc $\E' = 
			\left\{ \bigcup^n_{k=1} E_n \Bigg| n \in \Nat, E : \{1,\ldots,n\} \to \E   \right\}$}
	\Explain{
		Then $\E'$ is upwards directed and there is $F:\Nat \uparrow \E'$ such that
		$\mu\left( E \setminus \bigcup^\infty_{n=1} \E_0 \right)$for all $E \in \E \subset \E'$ }
	\ExplainFurther{
		But for evey $n \in \Nat$ there is number $m_n \in \Nat$ and 
		a finite sequence 	of sets $G_n : \{1,\ldots,m_n\} \to \E$}
	\Explain{
		such that $F_n = \bigcup^{m_n}_{k=1} G_{n,k}$,
		so construct countable set $\E_0 = \bigcup^\infty_{n=1} \im G_n \subset \E$}
	\Explain{
		Then $\bigcup \E_0 = \bigcup^\infty_{n=1} F_n$ and the result follows
	}
	\Explain{
		$(6)\Imply(7):$ Let $\E$ be a set of pairwise disjoint elements of $\Sigma \setminus \Null_\mu$}
	\Explain{
		Then there is a countable $\E_0 \subset \E$ such that 
		$\mu\left( E \setminus \bigcup \E_0 \right) = 0$ for all $E \in \E$}
	\Explain{
		If there is a $E \in \E \setminus \E_0$, then 
		$\mu\left( E \setminus \bigcup \E_0 \right) = \mu(E) > 0$
		as $\E$ has pairwise disjoint elements}
	\Explain{
		But this is a contradiction}
	\Explain{
		$(7)\Imply (8)$: obvious}
	\Explain{
		$(8)\Imply (1)$: Firstly we need to show that there is a partition of $X$
		into sets of finite positive measure}	
	\Explain{
		Let $\mathfrak{D}$ be the set of all disjoint families of $\Sigma^f \setminus \Null_\mu$ 
	}
	\Explain{
		Then by Zorn's lemma there is a maximal element $\mathcal{D} \in \mathfrak{\D}$}
	\Explain{
		By assumption $\D$  must be countable, so $\bigcup \D \in \Sigma$ 
	}
	\Explain{
		If there is $x \in X$ such that $x \not \in \bigcup \D$ then there is a finite measure set $F$
		as $\mu$ is semifinite with $x \in F$}
	\Explain{
		Take $F' = F \cap \left(\bigcup \D\right)^\c$, then still $x \in F'$ and $F'$ is disjoint from $\D$}
	\Explain{
		So, $\{F'\} \cup \D \in \mathfrak{D}$, which contradicts the macimality of $\D$}
	\EndProof
	\\
	\Theorem{SigmaFinitePrincipleOfExhaustion}
	{
		\NewLine ::
		\forall (X,\Sigma,\mu) : \sFinite \.
		\forall \E : \TYPE{NonEmpty}(\Sigma) \.
		\exists F : \Nat \uparrow \E \. \NewLine \.
		\forall E \in \E \.
		\Big( 
			\exists n \in \Nat \. \forall G \in \E \. E \cup F_n \not \subset G \Big)
		\Big|
		\Big(
			\lim_{n \to \infty} \mu(E \setminus F_n) =  0
		\Big)  
	}
	\NoProof
	\\
	\Theorem{SigmaFiniteEssSupExists}
	{
		\NewLine ::
		\forall (X,\Sigma,\mu) : \sFinite \.
		\forall \E : \TYPE{UpwardsDirected} \And \TYPE{NonEmpty}(\Sigma) \.
		\exists F : \Nat \uparrow \E \. 
		\bigcup^\infty_{n=1} F_n = \esssup \E  
	}
	\NoProof
}
\newpage
\subsubsection{Atomless Measures}
\Page{
	\Theorem{ValueChoice}
	{
		\forall (X,\Sigma,\mu) : \Aless \.
		\forall	E \in \Sigma^f \.
		\forall \alpha \in \Big(0,\mu(E)\Big) \.
		\exists  F \in \Sigma \.
		F \subset E  \And \mu(F) = \alpha
	}
	\Explain{
		As $\mu$ is atomless it is always possible to substract 
		$F \subset E$ such that $F \in \Sigma$ and $0 < 2\mu(F) \le  \mu(E) $}
	\Explain{
		So, by induction there always some $F \subset E$ such that 
		$F \in \Sigma$ and $0 < \mu(F) \le 2^{-n} \mu(E)$}
	\Explain{
		So it must be possible to define a sequence of sets $F_n$ such that
		$|\mu(F_n) - t| \le 2^{-n}\mu(E)$for all $n \in \Nat$	}
	\Explain{
		Note, that $F_n$ can be selected to be inccreasing, so
		$G = \bigcup^\infty_{n=1} F_n \subset E$}
	\Explain{
		So, by the lower continuity $\mu\left( G  \right) = \lim_{n \to \infty} \mu(F_n)  =  t$}
	\EndProof
	\\
	\Theorem{NeglidgiblePointByFiniteMeasure}
	{
		\forall (X,\Sigma,\mu) : \Aless \.
		\forall x \in X \.
		\mu^*\{x\} < \infty
		\Imply
		\mu^*\{x\} = 0	
	}
	\Explain{ There is $E \in \Sigma$ such that $x \in E$ and $\mu(E) \le 2\mu^*\{x\}$ }
	\Explain{ Then $E$ can be split into two parts of measure $\frac{1}{2}\mu(E) < \mu^*\{x\}$}
	\Explain{ So $x$ can't be in any o this parts, a contradiction}
	\EndProof
	\\
	\Theorem{NeglidgiblePointByLocalDetermetion}
	{
		\NewLine ::		
		\forall (X,\Sigma,\mu) : \Aless \And \MwLDNS \.
		\forall x \in X \.
		\mu^*\{x\} = 0	
	}
	\Explain{
		Let $E\in\Sigma^f$}
	\Explain{
		Then $E \cap \{x\}$ either equall to $\emptyset$ or to $\{x\}$
		}
	\Explain{
		But if $E \cap \{x\} = \{x\}$ thne $x \in \mu$ and by 
		previous theorem $\mu^*\{x\}=0$
	}
	\Explain{
		So ti is locally determined that $\mu^*\{x\} = 0$
	}
	\EndProof
	\\
	\Theorem{NeglidgiblePointByLoclizability}
	{
		\NewLine ::		
		\forall (X,\Sigma,\mu) : \Aless \And \Loc \.
		\forall x \in X \.
		\mu^*\{x\} = 0	
	}
	\Explain{See Fremlin 215E}
	\EndProof
}
\newpage
\section{Radon-Nikodym Theory}
\subsection{Additive Functionals}
\subsubsection{Subject}
\Page{
	\DeclareType{\AF}{\prod X \in \SET \. \prod \A : \Alg(X) \. A \to \Reals}
	\DefineType{\alpha}{\AF}{\forall A,B : \TYPE{DisjointPair}(\A) \. \alpha(A \cup B) = \alpha(A) + \alpha(B)  }
	\\
	\Theorem{EmptyZero}
	{
		\forall X \in \SET \.
		\forall \A : \Alg(X) \.
		\forall \alpha : \AF(X,\A) \.
		\alpha(\emptyset) = 0
	}
	\Explain{ 
		Use the fact that $\emptyset \cap \emptyset = \emptyset$, so $(\emptyset,\emptyset)$ 
		is a disjoint  pair}
	\Explain{
		Then $\alpha(\emptyset) = \alpha(\emptyset \cup \emptyset) = 2\alpha(\emptyset)$	
	}
	\Explain{
		This means $\alpha(\emptyset) = 0$
	}
	\EndProof
	\\
	\Theorem{IteratedSplitting}
	{
		\forall X \in \SET \.
		\forall \A : \Alg(X) \.
		\forall \alpha : \AF(X,\A) \. \NewLine \.
		\forall n \in \Int_+ \.
		\forall A : \TYPE{DisjointFamily}\Big(\{1,\ldots,n\},\A\Big)
		\alpha\left(\bigcup^n_{k=1} A_k \right) = \sum^n_{k=1} \alpha(A_k)
	}
	\Explain{ Simple proof by induction}
	\EndProof
	\\
	\Theorem{Difference1}
	{
		\forall X \in \SET \.
		\forall \A : \Alg(X) \.
		\forall \alpha : \AF(X,\A) \. \NewLine \.
		\forall A,B \in \A \.
		\forall \aleph : A \subset B \.
		\alpha(B) = \alpha(A) + \alpha(B \setminus A)
	}
	\Explain{ Follows from definition}
	\EndProof
	\\
	\Theorem{Difference2}
	{
		\forall X \in \SET \.
		\forall \A : \Alg(X) \.
		\forall \alpha : \AF(X,\A) \. \NewLine \.
		\forall A,B \in \A \.
		\alpha(B \cup A) = \alpha(A) + \alpha(B \setminus A)
	}
	\Explain{ Follows from definition}
	\EndProof
	\\
	\DeclareType{\CAF}{\prod (X,\Sigma) \in \BOR \. ?\AF(X,\Sigma)}
	\DefineType{\alpha}{\CAF}{
		\forall A : \TYPE{DisjointPair}(\Nat, \A) \.
		\alpha\left( \bigcup^\infty_{n=1} A_n \right) = \sum^\infty_{n=1} \alpha(A_n)
	}
}\Page{
	\Theorem{LowerContinuity}
	{
		\forall  (X,\Sigma,\alpha) : \CAF \.
		\forall E : \Nat \uparrow \Sigma \. \NewLine \.
		\alpha\left( \cup^\infty_{n=1} E_n \right) = 
		\alpha(E_1) + \sum^\infty_{n=1} \alpha (E_{n+1} \setminus E_n) 
	}
	\NoProof
	\\
	\Theorem{UpperContinuity}
	{
		\forall  (X,\Sigma,\alpha) : \CAF \.
		\forall E : \Nat \downarrow \Sigma \. \NewLine \.
		\alpha\left( \cap^\infty_{n=1} E_n \right) = 
		\alpha(E_1) - \sum^\infty_{n=1} \alpha (E_{n} \setminus E_{n+1}) 
	}
	\NoProof
	\\
	\DeclareFunc{functorCAF}{\Cov( \BOR, \VS{\Reals} )}
	\DefineNamedFunc{functorCAF}{X,\Sigma}{\caf(X,\Sigma)}{\CAF\Big( X,\Sigma \Big)}
	\DefineNamedFunc{functorCAF}{(X,\Sigma),(Y,T),f}{\caf_{(X,\Sigma),(Y,T)}(f)}{f_*}
	\\
	\DeclareFunc{functorAF}{\Cov( \mathsf{SETALG}, \VS{\Reals} )}
	\DefineNamedFunc{functorAF}{X,\A}{\af(X,\A)}{\AF\Big( X,\A \Big)}
	\DefineNamedFunc{functorAF}{(X,\A),(Y,\B),f}{\af_{(X,\A),(Y,\B)}(f)}{f_*}
	\\
	\Theorem{DeMoivreFormula}
	{	
		\forall (X,\A) \in \mathsf{SETALG}  \.
		\forall \alpha \in \af(X,\Sigma) \.
		\forall n \in \Int_+ \.
		\forall A : \{1,\ldots,n\} \to \A \. \NewLine
		\alpha\left( \bigcup^n_{i=1} A_i \right)
		+ 
			\sum^{\lfloor n/2 \rfloor}_{k=1} 
			\sum_{I \subset \{1,\ldots,n\},|I|=2k} 
			 \alpha\left( \bigcap_{i \in I} A_i \right)
		=
		\sum^{\lfloor n/2 \rfloor}_{k=0} 
			\sum_{I \subset \{1,\ldots,n\},|I|=2k + 1} 
		\alpha\left( \bigcap_{i \in I} A_i \right)	
	}
	\Explain{ The proof for measures uses only finite additivity, so it also fits here}
	\EndProof
}\Page{
	\Theorem{CountablyAdditiveAltDef}
	{
		\forall (X,\Sigma) \in \BOR \.
		\forall \alpha \in \af(X,\Sigma) \.
		\alpha \in \caf(X,\Sigma)  (1) \.
		\iff \NewLine \iff
		\forall E : \Nat \downarrow \Sigma \.
		\bigcap^\infty_{n=1} E_n = \emptyset \Imply
		\lim_{n \to \infty} \alpha(E_n) = 0 (2)
		\iff \NewLine \iff
		\forall E : \Nat \to \Sigma \.
		\bigcup^\infty_{m=1} \bigcap^\infty_{n=m} E_n = \emptyset \Imply 
		\lim_{n \to \infty} \alpha(E_n) = 0 (3)
		\iff \NewLine \iff
		\forall E : \Nat \to \Sigma \.
		\bigcup^\infty_{m=1} \bigcap^\infty_{n=m} E_n = \bigcap^\infty_{m=1} \bigcup^\infty_{n=m} E_n (4) 
		\Imply
		 \lim_{n \to \infty} \alpha(E_n) = 
		 \alpha\left(\bigcup^\infty_{m=1} \bigcap^\infty_{n=m} E_n \right)
	}
	\Explain{
		$(1)\Imply (2):$
		Use the fact that $E_1 = \bigcap^\infty_{n=1} E_n \sqcup \bigsqcup^\infty_{n=1} (E_n \setminus E_{n-1})$}
	\Explain{
		So, $\lim_{n \to \infty} \alpha(E_n) = \alpha(E_1) - \sum^\infty_{n=1} \alpha(E_n \setminus E_{n+1}) 
		= 		\alpha(E_1) - \alpha\left(\bigcap^\infty_{n=1} E_n\right)  
			-\sum^\infty_{n=1} \alpha(E_n \setminus E_{n+1}) = \alpha(E_1) - \alpha(E_1) = 0$}
	\Explain{
		$(2) \Imply (3):$ Use the fact that $F_m = \bigcup^\infty_{n=m} E_n$ is a decreasing sequence}
	\ExplainFurther{
		$(3) \Imply (4):$ The condition on sequence $E$ means that $E$ is convergent in boolean algebra $\Sigma$}
	\Explain{
		with respect to its $\sup\hyph\inf$ topology (see Vladimirov)}
	\Explain{ 
		So take $L = \lim_{n \to \infty} E_n \in \Sigma$}
	\Explain{ 
		Then $\lim_{n \to \infty} L \setminus E_n = \emptyset$ and $\lim_{n \to \infty} L^\c \cap E_n = \emptyset$
		as $(\setminus)$ and $(\cap)$ are order-continuous}
	\Explain{
		So $ 
			0 = 
			\lim_{n \to \infty} \alpha(L \setminus E_n) =
			\lim_{n \to \infty} \alpha( L \cup E_n)  - \alpha(E_n)
		$
	}
	\ExplainFurther{
		Thus 
		$
			\lim_{n \to \infty} \alpha(E_n) =    
			\lim_{n \to \infty} \alpha( L \cup E_n) =
			\lim_{n \to \infty} \alpha\Big( (L \cup E_n) \cap L \Big) + \alpha\Big((L\cup E_n) \cap L^\c\Big)=$} 
	\Explain{ 
			$=
			\lim_{n \to \infty} \alpha(L) + \alpha(L^\c \cap E_n) = 
			\alpha(L)$
		}
	\Explain{
		$(4)\Imply(1):$ Let $E_n$ be a disjoint sequence in $\Sigma$}
	\Explain{
		Let $F_n = \bigcup^n_{m=1} E_m$}
	\Explain{
		Then $F_n$ is convergent in sence of order topology
		and $\lim_{n \to \infty} F_n = \bigcup^\infty_{n=1} E_n$ }
	\Explain{
		So, by hypothesis 
		$\sum^\infty_{n=1} \alpha(E_n) = \lim_{n \to \infty} \alpha(F_n) = 
		\alpha\Big(\lim_{n \to \infty} F_n\Big) = \alpha\left( \bigcup^\infty_{n=1} F_n \right)$}
	\EndProof
}
\newpage
\subsubsection{Finite-Cofinite Example}
\Page{
	\DeclareFunc{finiteCofiniteAlgebra}{\prod_{X \in \SET} \Alg(X)}
	\DefineNamedFunc{finiteCofiniteAlgebra}{}{\F(X)}{
		 \Finite(X,\bullet) | \Finite\Big(X,\bullet^\c\Big) 	
	}
	\\
	\DeclareFunc{evenOddCounting}{\AF\Big(\Nat,\F(\Nat)\Big)}
	\DefineNamedFunc{evenOddCounting}{A}{\#' A}
	{
		\lim_{n \to \infty} \Big| \big\{ k \in \{1,\ldots,n\} \big| 2k \in A  \big\} \Big|  
		- \Big| \big\{ k \in \{1,\ldots,n\} \big| 2k + 1 \in A  \big\} \Big| 
	}
	\\
	\Theorem{EvenOddCountingIsUnbounded}
	{
		\im \#' = \Int
	}
	\Explain{
		We can use sets containing first $n$ odd or even numbers and only them}
	\EndProof
}
\newpage
\subsubsection{Hahn-Jordan decomposition}
\Page{
	\Theorem{BoundedCAF}
	{
		\forall (X,\Sigma) \in \BOR \.
		\forall \alpha \in \caf(X,\Sigma) \.
		\TYPE{Bounded}(\Sigma,\Reals,\alpha)
	}
	\Explain{
		Assume contra-positive}
	\Explain{
		Then there is a sequence of sets $E : \Nat \to \Sigma $
		such that $\lim_{n\to \infty} \alpha(E_n) = +\infty$ or $-\infty$}
	\Explain{
		Without loss of generality let $\lim_{n\to \infty} \alpha(E_n) = +\infty$}
	\Explain{
		Then we can assert that $\alpha(E_n)$ is strictly increasing}
	\Explain{
			Set  $F_{n,I} =\bigcap_{i \in I} E_i \setminus \bigcup_{j \in I^\c} E_j$
			for $I \subset \{1,\ldots,n\} $}
	\Explain{
		Then $F_n$ is disjoint for each $n \in \Nat$}
	\Explain{
		Select $\I_n=\arg\max_{\I \subset 2^{2^n}} \sum_{I \in \I} \alpha(F_{n,I})$
		and set $G_n = \bigcup_{I \in \I} F_{n,I}$ 
	}
	\Explain{
		For these sets $\alpha(G_n) = \sum_{I \in \I} \alpha(F_{n,I}) \ge \alpha(E_n) \to +\infty$
	}
	\Explain{
		Also the sequence $G_n$ is decreasing and in fact $\alpha(G_n)$ is increasing}
	\Explain{
		But by upper continuity
		$
			\alpha\left(\bigcap^\infty_{n=1} G_n\right) = \alpha(G_1) - \sum^\infty_{n=1} 
			\alpha(G_n \setminus G_{n+1}) \ge \alpha(G_n) \to \infty			
		$		}
	\Explain{
		So, $\alpha\left(\bigcap^\infty_{n=1} G_n\right) = + \infty$ but this is impossible
	}
	\EndProof
	\\
	\Theorem{HahnDecomposition}
	{
		\NewLine ::		
		\forall (X,\Sigma) \in \BOR \.
		\forall \alpha \in \caf \.
		\exists E \in \Sigma \.
		\Big(\forall  H \subset E  \. \alpha(H) \ge 0\Big)
		\And
		\Big(\forall H \subset E^\c \. \alpha(H) \le 0 \Big)
	}
	\Explain{
		By previous result $\alpha$ is bounded, so take $t = \sup_{E \in \Sigma} \alpha(E)$
	}
	\ExplainFurther{
		In fact there must be $E \in \Sigma$ with $\alpha(E) = t$ as we can construct a monotonic 
		sequence with increasing value,}\Explain{ as was shown above}
	\Explain{
		If $H \in \Sigma$ theb $\alpha(H \setminus E) \le 0$}
	\Explain{
		Otherwise, we would have an inequality	$\alpha(E \cup H) > \alpha(E)$,
		which contradicts the maximality}
	\Explain{
		So $H \subset E^\c$ imply $\alpha(H) \le 0$
	}
	\Explain{
		Simmilarly, if measurable $H \subset E$ and $\alpha(H) < 0$,
		then $\alpha(E \setminus H) > \alpha(E)$,
		which is impossible}
	\EndProof
	\\
	\Theorem{JordanDecomposition}
	{
		\NewLine ::		
		\forall (X,\Sigma) \in \BOR \.
		\forall \alpha \in \caf \.
		\exists \mu_+, \mu_- : \Finite(X,\Sigma) \.
		\alpha = \mu_+ - \mu_-
	}
	\Explain{ 
		Let $E$ be as in Hahn's decomposition}
	\Explain{
		Then define $\mu_+(H) = \alpha(H \cap E)$ 
		and $\mu_-(H) = -\alpha(H \cap E^\c)$}
	\EndProof
}
\newpage
\subsubsection{Bounded Additive Functionals}
\Page{
	\DeclareFunc{boundedAdditiveFunctionals}{\Cov\Big(\mathsf{SETALG} , \VS{\Reals}\Big) }
	\DefineNamedFunc{boundedAdditiveFunctionals}{(X,\A)}{\baf(X,\A)}
	{
		\Big\{ \nu \in \af(X,\A) : \exists b \in \Reals : \forall E \in \Sigma \. \big|\nu(E)\big| \le b \Big\}	
	}
	\DefineNamedFunc{boundedAdditiveFunctionals}{(X,\A),(Y,\B),f}{\baf_{(X,\A),(Y,\B)}(f)}{f_*}
	\\
	\DeclareFunc{positivePart}{\prod (X,\A) : \mathsf{SETALG} \. \baf(X,\A) \to \baf_+(X,\A) }
	\DefineNamedFunc{positivePart}{\nu}{\nu_+}
	{\Lambda A \in \A \. \sup \Big\{ \nu(E) \Big| E \in \A, E \subset A  \Big\}}
	\Explain{ 
		As $\nu$ is bounded the value is defined and in fact non less then $0$}
	\Explain{
		Assume $n\in \Nat$,$A : \{1,\ldots,n\} \to \A$ is disjoint}
	\ExplainFurther{
		Then 
		$
			\nu_+\left( \bigcup^n_{i=1} A_i \right) =
			\sup \left\{ \nu(E) \Bigg| E \in \A, E \subset \bigcap^n_{i=1}A_i  \right\} =
			\sup \left\{ \sum^n_{i=1} \nu(E \cap A_i) \Bigg| E \in \A, E \subset \bigcap^n_{i=1}A_i  \right\}
			=$}
	\Explain{
		$
		=\sum^n_{i=1} \sup \Big\{ \nu(E) | E \in \A,E \subset A_i  \Big\} = \sum^n_{i=1} \nu_+(A_i)
		$, so $\nu_+$ is additive functional}
	\Explain{ 
		Also, clearly $\Big|\nu_+(A)\Big| \le b $ if $b$ is a bound for $\nu$,
		so $\nu$ is bounded}
	\\
	\DeclareFunc{negativePart}{\prod (X,\A) : \mathsf{SETALG} \. \baf(X,\A) \to \baf_+(X,\A) }
	\DefineNamedFunc{negativePart}{\nu}{\nu_-}
	{(-\nu)_+}
	\\
	\Theorem{
		NegativePositivePartDecomposition
	}	
	{
		\forall (X,A) : \mathsf{SETALG} \.
		\forall \nu \in \baf(X,A) \.
		\nu = \nu_+ - \nu_-
	}
	\Explain{
		Assume the contrapositive}
	\Explain{
		Then, there exists $A \in \A$ such that $\nu(A) \neq \nu_+(A) - \nu_-(A)$ }
	\Explain{
		From trichtomy principle it follows that either
		$\nu(A)> \nu_+(A) - \nu_-(A)$ or $\nu(A) <  \nu_+(A) - \nu_-(A)$}
	\Explain{
		Without loss of generality assume that $\nu(A)> \nu_+(A) - \nu_-(A)$ }
	\Explain{
		Then, $ \nu(A) + \nu_-(A) >  \nu_+(A) \ge 0 $}
	\Explain{
		Take $E : \Nat \to \A$ to be such a sequence of sets that 
		$E_n \subset A$ and $\nu(E_n) \uparrow -\nu_-(A)$}
	\Explain{
		Then $
			\nu_+(A) < \nu(A) + \nu_-(A) = 
			\lim_{n \to \infty} \nu(A \setminus E_n) \le 
			\lim_{n \to \infty} \nu_+(A) = \nu_+(A $
	}
	\Exclaim{
		But this is a contradiction}
	\EndProof
	\\
	\DeclareFunc{Variation}{\prod (X,\A) : \mathsf{SETALG} \. \baf(X,\A) \to \baf_+(X,\A) }
	\DefineNamedFunc{Variation}{\nu}{|\nu|}
	{\nu_+ + \nu_-}
	\\
	\Theorem{CountableAdditivityPreservation}
	{
		\forall(X,\Sigma) \in \BOR \.
		\forall \nu \in \caf(X,\Sigma) \.
		\nu_+,\nu_-,|\nu| \in  \caf(X,\Sigma)
	}
	\Explain{
		Same arguments as above but with countable sequences}
	\EndProof
}\Page{
	\DeclareFunc{meetBA}
	{
		\prod (X,\A) : \mathsf{SETALG} \.
		\baf^2(X,\A) \to \baf 
	}
	\DefineNamedFunc{meetBA}{\nu,\eta}{\nu \wedge \eta}
	{
		\Lambda A \in \A \. \inf \Big\{ \nu(E) + \eta(A \setminus E) \Big| E \in \A, E \subset A  \Big\}	
	}
	\\
	\DeclareFunc{joinBA}
	{
		\prod (X,\A) : \mathsf{SETALG} \.
		\baf^2(X,\A) \to \baf 
	}
	\DefineNamedFunc{joinBA}{\nu,\eta}{\nu \vee \eta}
	{
		\Lambda A \in \A \. \sup \Big\{ \nu(E) + \eta(A \setminus E) \Big| E \in \A, E \subset A  \Big\}	
	}
	\\
	\Theorem{Lattice}
	{
		\forall (X,\A) : \TYPE{SETALG} \.
		\Big( \baf(X,\A), \vee,\wedge) \in \mathsf{LATT}
	}
	\Explain{ 
		Clearly, $\nu \wedge \eta \le \nu$ and $\nu \wedge \eta \le \eta$}
	\Explain{
		Assume $\xi \in \baf(X,\A)$ such that $\xi \le \nu$ and $\xi \le \eta$}
	\Explain{
		Then 
		$ \xi(A) = \xi(E) + \xi(A \setminus E) \le \nu(E) + \eta(A \setminus E) $
		for any $A,E \in \A$ with $E \subset A$}
	\Explain{
		So  $\xi(A) \le \nu \wedge \eta(A)$, thus $\xi \le \nu \wedge \eta$ as $A$ was arbitrary 
	}
	\Explain{
		The same strategy works with $\nu \vee \eta$}
	\EndProof
	\\
	\Theorem{LatticeSum}
	{
		\forall (X,\A) \in \mathsf{SETALG} \.
		\forall \nu,\eta \in \baf(X,\A) \.
		\nu \vee \eta + \nu \wedge \eta = \nu + \eta
	}
	\NoProof
	\\
	\Theorem{PositivePartsExpression}
	{
		\forall (X,\A) \in \mathsf{SETALG} \.
		\forall \nu \in \baf(X,\A) \.
		\nu_+ = \nu \vee 0
	}
	\NoProof
	\\
	\Theorem{NegativePartsExpression}
	{
		\forall (X,\A) \in \mathsf{SETALG} \.
		\forall \nu \in \baf(X,\A) \.
		\nu_- = \nu \wedge 0
	}
	\NoProof
	\\
	\Theorem{VariationExpression}
	{
		\forall (X,\A) \in \mathsf{SETALG} \.
		\forall \nu \in \baf(X,\A) \.
		|\nu| = \nu \vee (-\nu) = \nu_- \vee  \nu_+
	}
	\NoProof
	\\
	\Theorem{MeetExpression}
	{
		\forall (X,\A) \in \mathsf{SETALG} \.
		\forall \nu,\eta \in \baf(X,\A) \.
		\nu \wedge \eta = \nu - (\nu - \eta)_+
	}
	\NoProof
	\\
	\Theorem{JoinExpression}
	{
		\forall (X,\A) \in \mathsf{SETALG} \.
		\forall \nu,\eta \in \baf(X,\A) \.
		\nu \vee \eta = \nu + (\nu - \eta)_+
	}
	\NoProof
}
\Page{
	\Theorem{LatticeOperationsPreservesCA}	
	{
		\forall (X,\Sigma) \in \BOR \.
		\forall \nu,\eta \in \caf(X,\Sigma) \.
		\nu \wedge \eta,  \nu \vee \eta \in \caf(X,\Sigma)
	}
	\NoProof
	\\
	\DeclareFunc{countablyAdditivePart}
	{
		\prod (X,\Sigma) \in \BOR \.
		\baf(X,\Sigma) \to \caf(X,\Sigma)
	}
	\DefineNamedFunc{countablyAdditivePart}
	{\nu}{\mathrm{ca}(\nu)}
	{
		\Lambda E \in \Sigma  \. \inf_F \sup \nu(F_n)
		\quad		
		\where	 
		\quad		
		F : \Nat \uparrow \Sigma \And E = \bigcup^\infty_{n=1} F_n 
	}
	\\
	\Theorem{CountablyAdditiveBound}
	{
		\forall (X,\Sigma) \in \BOR \.
		\forall  \nu \in \baf(X,\Sigma) \.
		\forall \eta \in \caf(X,\Sigma) \.
		\eta \le \nu \Imply \eta \le \mathrm{ca}(\nu)
	}
	\NoProof
	\\
	\Theorem{CountablyAdditiveEquation}
	{
		\forall (X,\Sigma) \in \BOR \.
		\forall  \nu \in \baf(X,\Sigma) \.
		 \nu \wedge \Big(\nu - \mathrm{ca}(\nu) \Big) = 0
	}
	\NoProof
	\\
	\DeclareFunc{finitelyAdditivePart}
	{
		 \prod (X,\Sigma) \in \BOR \.
		\baf(X,\Sigma) \to \baf(X,\Sigma)
	}
	\DefineNamedFunc{purelyFinitelyAdditivePart}
	{  \nu }{\mathrm{pfa}(\nu)}{\nu - \mathrm{ca}(\nu)}
	\\
	\Theorem{PurelyFinitelyAdditivePartBound}
	{
		\forall (X,\Sigma) \in \BOR \.
		\forall  \nu \in \baf(X,\Sigma) \.
		\forall \eta \in \caf(X,\Sigma) \. \NewLine \.
		0 \le \eta \le \Big|\mathrm{pfa}(\nu)\Big| 
		\Imply  \eta = 0 
	}
	\NoProof
	\\
	\DeclareFunc{totalVariation}
	{
		\forall (X,\Sigma) \in \BOR \.
		\TYPE{Norm}\Big(\baf(X,\Sigma)\Big)
	}
	\DefineNamedFunc{totalVariation}{\nu}{\|\nu\|}{|\nu|(X)}
	\\
	\Theorem{BAIsBanach}
	{
		\forall (X,\Sigma) \in \BOR \.
		\baf(X,\Sigma) \in \Reals\hyph\mathsf{BAN}
	}
	\NoProof
}
\newpage
\subsection{Subject}
\subsubsection{Absolute Continuity}
\Page{
	\DeclareType{AbsolutelyContinuous}
	{
		\prod (X,\Sigma,\mu) \in \MEAS \.
		?\af(X,\Sigma)
	}
	\DefineNamedType{\nu}{AbsoluteltContinuous}{\nu \ll \mu}
	{
		\forall \varepsilon \in \Reals_{++} \.
		\exists \delta \in \Reals_{++} \.
		\forall E \in \Sigma \. 
		\mu(E) \le \delta \Imply \Big|\nu(E)\Big| \le \varepsilon
	}
	\\
	\DeclareType{TrulyContinuous}
	{
		\prod (X,\Sigma,\mu) \in \MEAS \.
		?\af(X,\Sigma)
	}
	\DefineType{\nu}{TrulyContinuous}
	{
		\forall \varepsilon \in \Reals_{++} \.
		\exists \delta \in \Reals_{++} \.
		\exists E \in \Sigma \. 
		 \mu(E) < \infty \And 
		\forall F \in \Sigma \. \NewLine \.
		\mu(F \cap E) \le \delta		 \Imply
		\Big|\nu(E )\Big| \le \varepsilon
	}
	\\	
	\DeclareType{Singular}
	{
		\prod (X,\Sigma,\mu) \in \MEAS \.
		?\af(X,\Sigma)
	}
	\DefineType{\nu}{Singular}
	{
		\exists E \in \Null_\mu \.
		\forall F \in \Sigma \.
		F \subset E^\c \Imply \nu(F) = 0
	}
	\\
	\Theorem{CAFAbsoluteContinuity}
	{
				\forall (X,\Sigma,\mu) \in \MEAS \.
				\forall \nu \in \caf(X,\Sigma) \.
				\nu \ll \mu
				\iff
				\forall E \in \Null_\mu \. \nu(E) = 0	
	}
	\Explain{
		$(\Imply):$ This is obvious}
	\Explain{
		$(\Leftarrow):$ Assume that $\nu$ is not absolutely continuous}
	\Explain{
		Then there exists $\varepsilon > 0$ and a sequence $E_n$ 
		such that $|\nu|(E_n) \ge \varepsilon$ and $\mu(E_n) \le 2^{-n}$
	}
	\Explain{
		Define a decreasing sequence $F_n = \bigcap^\infty_{m=n} E_m$
	}
	\Explain{
		Then $\mu\left( \bigcup^\infty_{n=1} F_n \right) =\lim_{n \to \infty} \mu(F_n) = 0$ and 
		$|\nu|\left( \bigcup^\infty_{n=1} F_n \right) = \lim_{n \to \infty} |\nu|(F_n) \ge \varepsilon$
	}
	\Exclaim{
		But by assumption  $\nu\left( \bigcup^\infty_{n=1} F_n \right) = 0$, a contradiction}
	\EndProof
}\Page{
	\Theorem{TrulyContinuousCondition}
	{
		\NewLine ::		
		\forall (X,\Sigma,\mu) \in \MEAS \.
		\forall \nu \in \af(X,\Sigma) \.
		\TC(X,\Sigma,\mu,\nu)
		\iff \NewLine \iff
		\nu \in \caf(X,\Sigma) \And \nu \ll \mu 
		\And \forall  E \in \Sigma \. 
		\nu(E) \neq 0 \Imply
		\exists F \in \Sigma \.
		\mu(F) \le \infty \And 
		\nu(E \cap F) \neq 0 
	}
	\Explain{
		$(\Rightarrow):$Firstly, assume that $\nu$ is truly continuous for $\mu$}
	\ExplainFurther{
		If $\varepsilon \in \Reals_{++}$, then there is $E \in \Sigma$
		and $\delta \in \Reals_{++}$ such that $\mu(E) < \infty$,} 
	\Explain{
		and 
		forall $F \in \Sigma$ such that $\Big|\nu(E \cap F)\Big|  \le \varepsilon$ 
		if $\mu(F) \le \delta$}
	\Explain{
		So, if $\mu(F) \le \delta$, then  $\mu(F \cap E) \le delta$ by monotonicity and $\Big|\nu(F)\Big|\le \varepsilon$}
	\Explain{
		Thus $\nu \ll \mu$}
	\Explain{
		Now assume $E \in \Sigma$ such that $\nu(E) \neq 0$}
	\Explain{
		Set $\varepsilon = \Big|\nu(E)\Big|/2 > 0$}
	\ExplainFurther{
		Then there is $F \in \Sigma$
		and $\delta \in \Reals_{++}$ such that $\mu(F) < \infty$,} 
	\Explain{
		and 
		forall $G \in \Sigma$ such that $\Big|\nu(G)\Big|  \le \varepsilon$ 
		if $\mu(F \cap G) \le \delta$}
	\Explain{
		But $ \Big|\nu(E)\Big| > \varepsilon$ by construction, so $\mu(F \cap G) > \delta > 0$}
	\Explain{
		Now, let $E:\Nat\downarrow \Sigma$ be such that $\bigcap^\infty_{n=1} E_n = \emptyset$}
	\Explain{
		Then $\lim_{n \to \infty} \mu(E_n) = 0$ by upper continuity
	}
	\Explain{
		But $\nu \ll \mu$,
		so $\lim_{n \to \infty} \Big|\nu(E_n)\Big| = 0$ and, moreover, 
		$\lim_{n \to \infty} \nu(E_n) = 0$}
	\Explain{
		Thus, $\nu$ is countably additive}
	\Explain{
		$(\Leftarrow):$ As $\nu$ is countably additive we may use $|\nu|$}
	\Explain{
		Set $t = \sup_{E \in \Sigma^f} |\nu|(E) \le |\nu|(X) < \infty$}
	\Explain{
		Then there is a sequence of sets $E : \Nat \to \Sigma^f$
		such that $t = \lim_{n \to \infty} |\nu|(E_n)$}
	\Explain{
		Assume $G \in \Sigma$ is disjoint from $F$
	}
	\Explain{
		Then if $0 < |\nu|(G)$ and $\mu(G) < \infty$ then $\lim_{n \to \infty} \nu(E_n \cup G) > t$,
	 	which is a contradiction}
	 \Explain{
	 	if $\mu(G) = \infty$ and $|\nu|(G) > 0$ then there is an $H \in \Sigma$ such that
	 	$\mu(H) < \infty$ and $|\nu|(G \cap H) \ge \Big|\nu(G \cap H) \Big| > 0$}
	 \Explain{
	 	So contradiction as above still can be produced,
	 thus $|\nu|(G) = 0$ }
	\Explain{
		Set $F_n = \bigcup^n_{k=1} E_n$
	}
	\Explain{
		Let $\varepsilon \ge 0$}
	\Explain{
		Then there exists $n$ such that $\nu(F_n) \ge t - \frac{\varepsilon}{2}$}
	\Explain{
		Also there is $\delta$ such that $\mu(H) \le \delta$ imply
		that $\Big|\nu(H)\Big| \le \frac{\varepsilon}{2}$
	 	for all $H \in \Sigma$,as $\nu \ll \mu$}
	\Explain{
		Assume $H \in \Sigma$ is such that $\mu(H \cap F_n) \le \delta$}
	\Explain{
		Then 
		$
			\Big|\nu(H)\Big| \le \Big|\nu(H \cap F^\c_n)\Big| + \Big|\nu(H \cap F_n)\Big| \le 
			|\nu|( H \cap F^\c_n) + \frac{\varepsilon}{2} \le \varepsilon
		$}
	\Explain{
		Thus, $\nu$ is truly continuous with respect to $\mu$
	}
	\EndProof
}\Page{
	\Theorem{SigmaFiniteTrulyContinuousCondition}
	{
		\forall (X,\Sigma,\mu) : \sFinite \.
		\forall \nu \in \af(X,\Sigma) \. \NewLine \.
		\TC(X,\Sigma,\mu,\nu)
		\iff
		\nu \ll \mu \And \nu \in \caf(X,\Sigma)	
	}
	\Explain{
		$(\Rightarrow):$ this is obvious}
	\Explain{
		$(\Leftarrow):$ assume $E \in \Sigma$ such that $\nu(E) \neq 0$}
	\Explain{
		Then $\mu(E) \neq 0$}
	\Explain{
		Also take $F : \Nat \to \Sigma$ to be a finite partition of $X$ for $\mu$}
	\Explain{
		Then where must be some $n$ such that $\nu(F_n \cap E) \neq 0$
		as $\nu$ is countably additive}
	\Explain{
		Thus, $\nu$ is truly continuous
	}
	\EndProof
	\\	
	\Theorem{FiniteTrulyContinuousCondition}
	{
		\forall (X,\Sigma,\mu) : \sFinite \.
		\forall \nu \in \af(X,\Sigma) \. \NewLine \.
		\TC(X,\Sigma,\mu,\nu)
		\iff
		\nu \ll \mu 
	}
	\Explain{
		$(\Rightarrow):$ this is obvious}
	\Explain{
		$(\Leftarrow):$  Take $E=X$ in definiton of truly continuous}
	\EndProof
	\\
	\DeclareFunc{absContFunctor}
	{
		\Cov(\MEAS_0,\VS{\Reals})
	}
	\DefineNamedFunc{absContFunctor}{X,\Sigma,\mu}{\ac(X,\Sigma,\mu)}
	{ \Big\{  \nu \in \caf(X,\Sigma) : \nu \ll \mu  \Big\} }
	\DefineNamedFunc{absContFunctor}{(X,\Sigma,\mu),(Y,T,\mu'),f}{\ac_{(X,\Sigma,\mu),(X,T,\mu')}(f)}
	{f_* }
	\\
	\DeclareFunc{truelyContinuous}
	{
		\MEAS \to \VS{\Reals}
	}
	\DefineNamedFunc{truelyContinuous}{X,\Sigma,\mu}{\tc(X,\Sigma,\mu)}
	{ \TC(X,\Sigma,\mu) }
}
\newpage
\subsubsection{The indefinite integral}
\Page{
	\DeclareFunc{indefiniteIntergeal}{
		\prod (X,\Sigma,\mu) \in \MEAS \.
		L^1(X,\Sigma,\mu) \Arrow{\VS{\Reals}} \caf(X,\Sigma,\mu)
	}
	\DefineNamedFunc{indefiniteIntegral}{f}{fd\mu}
	{
		\Lambda E \in \Sigma \. \int_E f\;d\mu	
	}
	\\
	\Theorem{IndefiniteIntegralIsTrulyContinuous}
	{
		\NewLine ::		
		\forall (X,\Sigma,\mu) \in \MEAS \.
		\forall f \in L^1(X,\Sigma,\mu) \.
		fd\mu \in \tc(X,\Sigma,\mu)
	}
	\Explain{ 
	      take some  $\varepsilon > 0$}
	\Explain{
		Then there is a simple function $\sigma(x) = \sum^n_{k=1} \alpha_k \delta_x(F_k)$
	    such that  $\int | f - \sigma | \le \frac{\varepsilon}{2}$}
	\Explain{
		Let $E = \bigcup^n_{k=1} F_k$, so $\mu(E) \le \sum^n_{k=1} \mu(F_k) < \infty$}
	\Explain{
		If $\alpha \neq 0$ take $\delta = \frac{\varepsilon}{2\max |\alpha_k|}$ otherwise
		$\delta$ can be arbitrary
	}
	\Explain{
		Take $G \in \Sigma$ to be such that $\mu( G \cap E) \le \delta$}
	\Explain{
		Then
		$
			\left|\int_G fd\mu\right| \le 
			\left|\int_{G \cap E} \sigma d \mu \right| + \frac{\varepsilon}{2} \le \varepsilon
		$
	}
	\EndProof
	\\
	\Theorem{IndefiniteIntegralIsAbsolutelyContinuous}
	{
		\NewLine ::		
		\forall (X,\Sigma,\mu) \in \MEAS \.
		\forall f \in L^1(X,\Sigma,\mu) \.
		fd\mu \in \ac(X,\Sigma,\mu)
	}
	\NoProof
}
\newpage
\subsubsection{Subject}
\Page{
	\Theorem{RadonNikodymLemma1}
	{
		\NewLine ::		
		\forall (X,\Sigma,\mu) \in \MEAS \.
		\forall \nu \in \tc_{++}(X,\Sigma,\mu) \.
		\exists \sigma \in \Simple(X,\Sigma,\mu) \.
		0 < \int \sigma  \And  
		\forall E \in \Sigma \. \int_E \sigma \le \nu(E)  
	}
	\Explain{
		We know that $\nu(X) > 0$}
	\Explain{
		Let $\varepsilon = \frac{1}{3} \nu(X) > 0$}
	\Explain{
		So there is $E$ with $\mu(E) < \infty$	and $\delta > 0$ 
		such that $\mu(E \cap F) \le \delta$ imply $\nu(F) \le \varepsilon$ 
		for all $F \in \Sigma$}
	\Explain{
		Then $\nu(E^\c \cap E) = \nu(\emptyset) = 0$, so $\nu(E^\c) \le \varepsilon$}
	\Explain{
		This means that $\nu(E) \ge 2 \varepsilon$}
	\Explain{
		Thus $\mu(E) > \delta > 0$}
	\Explain{
		Let $\alpha = \frac{\varepsilon}{\mu(E)}$ and $\nu' = \nu - \alpha\mu(\bullet|E)$}
	\Explain{
		Then $\nu'(E) \ge 2 \varepsilon - \varepsilon > 0 $
	}
	\Explain{
		Take $G$ to be support for $\nu'_+$ and define 
		$\sigma(x) = \alpha \delta_x(G \cap E)$}
	\Explain{
		Then $\nu(G \cap E) \ge \nu'(G \cap E) \ge \nu'(E) > 0 $,
		so $\mu(G \cap E)>0$ and so $\int \sigma > 0$}
	\ExplainFurther{
		On the other hand 
		$
			\nu(F) \ge
			\nu(F \cap G) \ge 
			\alpha \mu(F \cap G \cap E) = \int_E f		
		$
		as $\nu'(F \cap G) \ge 0$}
	\Explain{
		for any $F \in \Sigma$
	}
	\EndProof
	\\
	\DeclareFunc{subordinateFunctions}
	{
		\prod (X,\Sigma,\mu) \in \MEAS \.
		\tc_{++}(X,\Sigma,\mu) \to ?\Simple_+(X,\Sigma,\mu)
	}
	\DefineNamedFunc{subordinateFunctions}{\nu}{S_\nu}
	{
		\left\{ \sigma \in \Simple_+(X,\Sigma,\mu) : \forall E \in \Sigma \. \int_E \sigma \le \nu(E)  \right\}
	}
	\\
	\Theorem{SubordinateFunctionsAreMaxClosed}
	{
		\forall (X,\Sigma,\mu) \in \MEAS \.
		\forall \nu \in \tc_{++}(X,\Sigma,\mu) \.
		\forall f,g \in S_\nu \. f \vee g \in S_\nu
	}
	\Explain{
		This follows form finite additivity of integrals and from structure of simple functions}
	\EndProof
}\Page{
	\Theorem{RadonNikodymTHM}
	{
		\forall (X,\Sigma,\mu) \in \MEAS \.
		\forall \nu \in \tc(X,\Sigma,\mu) \.
		\exists f \in L^1(X,\Sigma,\mu) \.
		\nu = fd\mu
	}
	\Explain{ 
		At first assume $0 \neq \nu = \nu_+$}
	\Explain{
		Let  $\gamma = \sup_{\sigma \in S_\nu} \int \sigma > 0$}
	\Explain{
		Then there is $f : \Nat \to S_\nu$ such that $\gamma = \lim_{n \to \infty} \int f$}
	\Explain{
		Set $g_n = \bigvee^n_{k=1} f_k$, then also  $\gamma = \lim_{n \to \infty} \int f$
		and $g$ is stricly increasing}
	\Explain{
		By B. Levi's theorem  there is alimit $h = \lim_{n \to \infty} g_n = \lim_{n \to \infty} f_n  $
		such that $\int h = \gamma$}
	\Explain{
		Also $\int_E h = \lim_{n \to \infty} \int_E f_n  \le \nu(E) $ for any $E \in \Sigma$}
	\Explain{
		Assume there is $E \in \Sigma$ such that 
		$\int_E h < \nu(E)$}
	\Explain{
		Define $\nu' = \nu - hd\mu \in \tc(X,\Sigma,\mu)$
	}
	\Explain{
		Also, note that $\nu' \neq 0$ by assumption}
	\Explain{
		Moreover, by lemma there is a separating simple function $\sigma$ such that
		$\sigma d\mu \le \nu'$ and $\int \sigma > 0$}
	\Explain{
		Then there is $n \in \Nat$ such that $ \int (f_n + \sigma)  \int f_n + \int \sigma > \gamma$
	}
	\Explain{
		But then $\int_E (f_n + \sigma) \le \int_E  h +  \nu'(E) = \nu(E)$
	    for any $E \in \Sigma$}
	\Explain{
		So $f_n + \sigma \in S_\nu$ and this contradicts maximality of $\gamma$}
	\Explain{
		Thus $\nu = f d \mu $}
	\Explain{
		For the general case use decomposition $\nu = \nu_+ - \nu_-$
	}
	\EndProof
	\\
	\DeclareFunc{derivariveOfRadon}
	{
		\prod (X,\Sigma,\mu) \in \MEAS \.
		\tc(X,\Sigma,\mu) \Arrow{\VS{\Reals}} L^1(X,\Sigma,\mu)
	}
	\DefineNamedFunc{derivariveOfRadon}{\nu}{\frac{d \nu}{d \mu}}
	{
		\THM{RadonNicodymTHM}(X,\Sigma,\mu,nu)
	}
	\\
	\Theorem{RadonNikodymTHM2}
	{
		\forall (X,\Sigma,\mu) \in \MEAS \.
		\forall \nu \in \Sigma \to \Reals \.
		\nu \in \ac(X,\Sigma,\mu) \iff
		\exists f \in L^1(X,\Sigma,\mu) \.
		\nu = fd\mu
	}
	\NoProof
	\\
	\Theorem{RadonNikodymTHM3}
	{
		\forall (X,\Sigma,\mu) : \Finite \.
		\forall \nu \in \Sigma \to \Reals \.
		\nu \in \af(X,\Sigma,\mu) \And\nu \ll \mu \A  \iff \NewLine \iff
		\exists f \in L^1(X,\Sigma,\mu) \.
		\nu = fd\mu
	}
	\NoProof
}
\newpage
\subsubsection{Lebesgue Decomposition}
\Page{
	\Theorem{LebesgueDecomposition}
	{
		\NewLine ::		
		\forall (X,\Sigma,\mu) \in \MEAS \.
		\forall \nu \in \caf(X,\Sigma,\mu) \.
		\exists! \nu' \in \ac(X,\Sigma,\mu) \.
		\exists! \nu'' : \TYPE{Singular}(X,\Sigma,\mu) \.
		\nu = \nu' + \nu'' 
	}
	\NoProof
	\\
	\Theorem{LebesgueDecomposition}
	{
		\NewLine ::		
		\forall (X,\Sigma,\mu) \in \MEAS \.
		\forall \nu \in \caf(X,\Sigma,\mu) \.
		\exists! \nu' \in \tc(X,\Sigma,\mu) \.
		\exists! \nu'' : \TYPE{Singular}(X,\Sigma,\mu) \. \NewLine \.
		\exists! \nu''' \in  \ac(X,\Sigma,\mu) \.
		\nu = \nu' + \nu'' + \nu'''
		\And  \forall E \in \Sigma^f \. \nu'''(E) = 0
	}
	\NoProof
}
\newpage
\subsection{Conditioning}
\subsubsection{Conditional Integrals}
\Page{
	\Theorem{ConditionalIntegrability}
	{
		\NewLine ::		
		\forall (X,\Sigma,\mu) \in \MEAS \. 
		\forall T \subset_\sigma \Sigma \.
		\forall f : X \to \Reals \.
		f \in L^1(X,T,\mu|T)
		\iff \NewLine \iff
		f \in L^1(X,\Sigma,\mu)
		\And \dom f \in \Null'_{\mu|T} 
		\And f \in \BOR_{\mu|T}^*(X,T)
	}
	\NoProof
	\\
	\Theorem{ConditionalIntegralEqual}
	{
		\NewLine ::		
		\forall (X,\Sigma,\mu) \in \MEAS \. 
		\forall T \subset_\sigma \Sigma \.
		f \in L^1(X,T,\mu|T)	 \.
		\int f \; d\mu = \int f \; d(\mu|T)
	}
	\NoProof
}
\newpage
\subsubsection{Conditional Expectation}
\Page{
	\DeclareType{\CE}
	{
		\NewLine ::		
		\prod (X,\Sigma,\mu) \in \MEAS \. 
		\TYPE{SequentiallyCompleteSubalgebra}(X,\Sigma)
		\to
		L^1(X,\Sigma,\mu)
		\to
		?L^1(X,T,\mu|T)
	}
	\DefineType{g}{\CE}{
		\Lambda T \subset_\sigma \Sigma \. 
		\Lambda f \in L^1(X,\Sigma,\mu) \. 
		\forall E \in T \. \int_E f d\mu = \int_E g d(\mu|T) 
	}
	\\
	\Theorem{ConditionalExpectationAdditivity}
	{
		\NewLine ::		
		\forall (X,\Sigma,\mu) \in \MEAS \.
		\forall T \subset_\sigma \Sigma \.
		\forall f,f' \in L^1(X,\Sigma,\mu) \. \NewLine \.
		\forall g : \CE(X,\Sigma,\mu,T,f) \.
		\forall g' : \CE(X,\Sigma,\mu,T,f') \. \NewLine \.
		\CE(X,\Sigma,\mu,T,f +f',g+g')	
	}
	\Explain{By additivity of integral}
	\EndProof
	\\
	\Theorem{ConditionalExpectationHomogenity}
	{
		\NewLine ::		
		\forall (X,\Sigma,\mu) \in \MEAS \.
		\forall T \subset_\sigma \Sigma \.
		\forall f \in L^1(X,\Sigma,\mu) \. \NewLine \.
		\forall g : \CE(X,\Sigma,\mu,T,f) \.
		\forall \alpha \in \Reals \. \NewLine \.
		\CE(X,\Sigma,\mu,T,\alpha f, \alpha g)	
	}
	\Explain{By homogenity of integral}
	\EndProof
	\\
	\Theorem{ConditionalExpectationIneq}
	{
		\NewLine ::		
		\forall (X,\Sigma,\mu) \in \MEAS \.
		\forall T \subset_\sigma \Sigma \.
		\forall f,f' \in L^1(X,\Sigma,\mu) \. \NewLine \.
		\forall g : \CE(X,\Sigma,\mu,T,f) \.
		\forall g' : \CE(X,\Sigma,\mu,T,f') \. \NewLine \.
		f \le_{\ae \mu} f' \Imply g \le_{\ae (\mu|T)} g'
	}
	\Explain{ Let $E \in T$}
	\Explain{
		Then $\int g \; d(\mu|T) =  \int f \; d\mu  \le   \int f' \; d\mu = \int g' \; d(\mu|T) $
	}
	\Explain{
		So $g \le_{\ae (\mu|T)} g'$}
	\EndProof
}\Page{
	\Theorem{MonotonicConvergenceTHM}
	{
		\NewLine ::		
		\forall (X,\Sigma,\mu) \in \MEAS \.
		\forall T \subset_\sigma \Sigma \.
		\forall f : \Nat \uparrow L^1(X,\Sigma,\mu) \. \NewLine \.
		\forall F \in L^1(X,\Sigma,\mu) \.
		\forall g : \prod^\infty_{n=1} \CE(X,\Sigma,\mu,T,f_n) \.
		\forall \aleph : F =_{\ae} \lim_{n \to \infty} f_n  \. \NewLine \.
		\CE(X,\Sigma,\mu,T,F,\lim_{n \to \infty} g_n)
	}
	\Explain{
		By previous result $g$ is monotonic}
	\Explain{
		Also
		$
			\lim_{n \to \infty} \int g_n \;d(\mu|T) =
			\lim_{n \to \infty} \int f \;d \mu = 
			\int F \; d\mu < \infty
		$
	}
	\Explain{
		So, by B.  Levy  $\lim_{n \to \infty} g_n$
		exists almost everywhere $\mu|T$ and 
		$
			\int_E \lim_{n \to \infty} g \; d(\mu|T) =
			\int F \; d \mu
		$}
	\EndProof
	\\
	\Theorem{DominatedConvergenceTHM}
	{
		\NewLine ::		
		\forall (X,\Sigma,\mu) \in \MEAS \.
		\forall T \subset_\sigma \Sigma \.
		\forall f : \Nat \to  L^1(X,\Sigma,\mu) \. \NewLine \.
		\forall F \in L^1(X,\Sigma,\mu) \.
		\forall h \in L^1(X,\Sigma,\mu) \.
		\forall g : \prod^\infty_{n=1} \CE(X,\Sigma,\mu,T,f_n) \. \NewLine \.
		\forall \aleph : F =_{\ae} \lim_{n \to \infty} f_n  \.
		\forall \beth : \forall n \in \Nat \. |f_n| \le_{\ae} h \. 
		\CE(X,\Sigma,\mu,T,F,\lim_{n \to \infty} g_n)
	}
	\Explain{
		Same proof as above but with dominated convegence theorem}
	\EndProof
	\\
	\Theorem{Restriction}
	{
		\NewLine ::		
		\forall (X,\Sigma,\mu) \in \MEAS \.
		\forall T \subset_\sigma \Sigma \.
		\forall f \in L^1(X,\Sigma,\mu) \. \NewLine \.
		\forall g : \CE(X,\Sigma,\mu,T,f) \.
		\forall E \in T \. \NewLine \.
		\CE\Big(X,\Sigma,\mu,T,f\delta(E), g\delta(E)\Big)	
	}
	\Explain{ Assume $F \in T$}
	\Explain{  
		Then 
		$
			\int_F g\delta(E) \; d(\mu|T) 
			\int_{E \cap F} g \; d(\mu|T) =
			\int_{E \cap F} f \; d\mu =
			\int_F f\delta(E) \; d\mu 
		$	
	}
	\EndProof
	\\
	\Theorem{Product}
	{
		\NewLine ::		
		\forall (X,\Sigma,\mu) \in \MEAS \.
		\forall T \subset_\sigma \Sigma \.
		\forall f \in L^1(X,\Sigma,\mu) \. \NewLine \.
		\forall g : \CE(X,\Sigma,\mu,T,f) \.
		\forall h \in \BOR_{\mu|T}^*(X,T) \And \TYPE{Bounded} \. \NewLine \.
		\CE\Big(X,\Sigma,\mu,T,fh, gh\Big)	
	}
	\Explain{ 
		If $h$ is trivial it works trivially}
	\Explain{
		Otherwise, represent $h$ as a limit of simple functions $\sigma_n$}
 	\Explain{
 		If $h$ is bounded by $b$ when we may assume that $\Big|\sigma_n(x)\Big | \le b $}
 	\Explain{
 		Then $\sigma_n f$ is bounded by $b|f|$ which is integrable}
 	\Explain{
 		So by dominated convergence $gh$ is a conditional expectation of $fh$}
 	\EndProof
}
\newpage
\subsubsection{Jensen Inequality}
\Page{
	\Theorem{ConvexIsMeasurable}
	{
		\NewLine ::
		\forall  \phi : \TYPE{Convex} \.
		\phi \in \BOR(\Reals,\Reals)
	}
	\Explain{
		It is posiible to represent $\phi = \sup_{q \in \Rats} \phi_q$,
		where $\phi_q = \phi(q) + \alpha_q(x-q)$ for $\alpha_q \in \Reals$}
	\Explain{
		When each $\phi_q$ is measurable as it is affine
	}
	\Explain{
		So $\phi$ is measureable as supremum of convex functions}
	\\	
	\Theorem{JensenInequality}
	{
		\NewLine ::		
		\forall (X,\Sigma,\mu) \in \MEAS \.
		\forall \phi : \TYPE{Convex} \.
		\forall f,g \in \BOR_\mu^*(X,\Sigma,) \.
		\forall \aleph : f \ge_{\ae} 0 \.
		\forall \beth : \int f = 1 \. \NewLine \.
		\forall \gimel : fg  \in L^1(X,\Sigma,\mu) \. 
		\phi( \int  fg ) \le \int \phi(f) g
	}
	\Explain{
		This follows from elementary convex analysis}
	\EndProof
}
\newpage
\subsection{Transforamtions}
\subsubsection{Measure Preserving Maps}
\subsubsection{Sums}
\subsubsection{Order}
\subsection{Change of Variable in the Integral}
\newpage
\section{Products of Measures}
\subsection{Product Measure Theorem }
\Page{
\DeclareFunc{SigmaAlgebraProduct}{\SA{A} \to \SA{B} \to \Set(A \times B)}
\DefineNamedFunc{SigmaAlgebraProduct}{\mathcal{A},\B}{\mathcal{A} \times \B}
{\{ a \times b | a \in \mathcal{A}, b \in \mathcal{B} \}}
\\
\DeclareFunc{BorProduct}{\mathsf{BOR} \to \mathsf{BOR} \to \mathsf{BOR}}
\DefineNamedFunc{BorProduct}{(A,\mathcal{A}),(B,\B)}{(A,\mathcal{A}) \times (B,\B)}
{(A \times B, \sigma(\mathcal{A} \times B))}
\\
\DeclareType{Uniformly \; \sigma \hyph Finite}{ ?(
X \to \TYPE{Measure}(Y) ) }
\DefineType{\mu}{Uniformly \; \sigma \hyph Finite}
{ \exists b : \Nat \to \F_Y : \exists k : \Nat \to \Reals_+ : \bigcup^\infty_{n = 1} b_n = 
Y : \NewLine
 :
\forall x \in X \. \forall n \in \Nat \. \mu(x,b_n) \le k_n }
\\
\DeclareType{SlicingMeasure}{\prod X \in \mathsf{MEAS} \. 
\prod Y \in \BOR \.
X \to \TYPE{Measure}(Y) }
\DefineType{\mu}{SlicingMeasure}
 {\forall b \in \F_Y \. \Lambda x \in X \. \mu(x,b) : \TYPE{Measurable}(F_{\BOR}X)}
 \\
 \Theorem{RectangularAlgrebraTHM}{
 \ForEach{ X, Y}{ \BOR }{ 
 \ForEach{ G}{ \TYPE{MonotoneClass}(X \times Y) : \F_X \times \F_Y \subset G}{  \sigma(\F_X \times \F_Y) \subset G} 
 }}
 \Assume{x \times y}{\F_X \times \F_Y }
 \Say{(1)}{\THM{ProductComplement}(x \times y)}{ (x \times y) = x^\c \times y \cap x \times y^\c \cap x^\c \times y^\c }
 \Conclude{(2)}{\ByDef\TYPE{MonotoneClass}(1,\ByDef(G))}{ (x \times y)^\c }
 \Derive{(\F_X \times \F_Y, 1)}{(\F_X \times \F_Y,\ByDef \TYPE{ComplementClosed}(\cdot)}{\TYPE{ComplementClosed}(G)}
 \Conclude{ (2) }{\THM{MonotoneClassTHM}(1)}{\sigma(\F_X \times F_Y) \subset G}
 \EndProof
 \\ 
 \Theorem{MeasurableSection}{
 \ForEach{ X,Y}{\BOR}{ 
 \ForEach{A}{\F_{X \times Y  }}{ 
 \ForEach{x}{X}{  \mathrm{section}(A,x) \in \F_Y}  
 }}}
 \Say{B}{\{ A \in \F_{X \times Y} : \mathrm{section}(A,x) \in \F_Y \}}{\SA{X \times Y}}
 \Say{(I)}{\ByDef{B}}{ \{ a \times b | a \in F_X , b \in F_Y \}
   \subset  B}
 \Conclude{(II)}{\ByDef(\F_X \times F_Y)(\ByDef(\sigma)(I))}{\F_{X \times Y} \subset B  \leadsto
  \F_{X \times Y} = B ;; 
 }
 \EndProof
}
 \newpage
\Page{
  \Theorem{MeasurableSlicing}{
 \ForEach{ S}{\TYPE{SlicingMeasure}(X,U)}{ 
 \ForEach{A}{ \F_{X \times Y} }{ \. \NewLine
   \Lambda x \in X \. S(x, \mathrm{section}(A,x)) : \TYPE{Measurable}(F_{\BOR}X)      
 }}}
 \Say{B}{ \{ A \in \F_{X \times Y} :  \Lambda x \in X \. S(x, \mathrm{section}(A,x)) : \TYPE{Measurable}(F_{\BOR}X)   \}}{
  \NewLine : 
 \Set (F_{\BOR}X \times Y)}
 \Assume{ b }{\Nat \to B}
 \Assume{ \beta }{ \F_{X \times Y} : b \uparrow \beta }
 \Say{(1)}{\THM{SectionIsMonotonic}(b,\beta)}
 {  \ForEach{x}{X}{  \mathrm{section}(x,b_n) \uparrow \mathrm{section}(x,\beta) }}
 \Say{(2)}{\THM{MeasureUpperContinuity}(\Lambda x \in X \. S(x,b),(1))}
 { \Lambda x \in X \. S(x, b_n) \uparrow \Lambda x \in X \. S(x, \beta)  }
 \Say{(3)}{\THM{MonotoneConvergenceTHM}(2)}
 { (x \in X \. S(x, \beta) : \TYPE{Measurable}(F_\BOR X) ) }
 \Conclude{(4)}{\ByDef(B)(3)}{ \beta \in B;}
 \Derive{(1\star)}{\THM{UniversalIntroduction}(\cdot)}
 {\ForEach{b}{  \Nat \to B}{\ForEach{\beta}{\F_{X \times Y} : b \uparrow \beta}}
 { \beta \in B } }
 \Assume{ b }{\Nat \to B}
 \Assume{ \beta }{ \F_{X \times Y} : b \downarrow \beta }
 \Say{(1)}{\THM{SectionIsMonotonic}(b,\beta)}
 {  \ForEach{x}{X}{  \mathrm{section}(x,b_n) \downarrow \mathrm{section}(x,\beta) }}
 \Say{(2)}{\THM{MeasureLowerContinuity}(\Lambda x \in X \. S(x,b),(1))}
 { \Lambda x \in X \. S(x, b_n) \downarrow \Lambda x \in X \. S(x, \beta)  }
 \Say{(3)}{\THM{MonotoneConvergenceTHM}(2)}
 { (x \in X \. S(x, \beta) : \TYPE{Measurable}(F_\BOR X) ) }
 \Conclude{(4)}{\ByDef(B)(3)}{ \beta \in B;}
 \Derive{(2\star)}{\THM{UniversalIntroduction}(\cdot)}
 {\ForEach{b}{  \Nat \to B}{\ForEach{\beta}{\F_{X \times Y} : b \downarrow \beta}}
 { \beta \in B } }
 \Say{(1)}{\ByDef\TYPE{MonotoneClass}(1\star,2\star)}
 { B : \TYPE{MonotoneClass}(X \times Y)}
 \Assume{a}{  \F_X}
 \Assume{b}{ \F_Y}
 \Say{(2)}{ \ByDef\mathrm{section}(a \times b) }
 { \Lambda x \in X \. S(x,\mathrm{section}(x,a \times b)) = \Lambda x \in X \. S(x,b)} 
 \Say{(3)}{ \ByDef\TYPE{SlicingMeasure}(S)(b)(2)}
 { ( \Lambda x \in X \. S(x,\mathrm{section}(x,a \times b)) : \TYPE{Measurable}( F_{\BOR}X) )}
 \Conclude{(4)}{\ByDef B (3)}{a \times b \in B}
 \Derive{(2)}{ \ByDef \F_X \times \F_Y(\cdot)}{\F_X \times \F_Y \subset B }
 \Say{(3)}{\THM{RectangularAlgebraTHM}(X,Y,B)(2)}{\mathrm{Alg}(F_X \times \F_Y) \subset B}
 \Say{(4)}{\THM{MonotoneClassTHM}(1,3)}{\sigma(\F_X \times \F_Y) \subset B}
 \Conclude{(5)}{\THM{SetEqIntroduction}(4,\ByDef B)}{ \F_{X \times Y} = B ;}
 \EndProof 
}
\newpage
\Page{
 \Theorem{ProductMeasureTheorem}{
 \ForEach{X}{\mathsf{MEAS}}{
 \ForEach{Y}{\mathsf{BOR}}{
 \ForEach{S}{\TYPE{SlicingMeasure}(X,Y)}{
  \NewLine \.
 \Exist{!\gamma}{\TYPE{Measure}(F_{\mathsf{BOR}} X \times Y)
  : 
  \ForEach{A}{ \F_{F_{\mathsf{BOR}} X \times Y} }{
  \gamma(A) = \int_X S(x ,\mathrm{section}(A,x)) \mathrm{d}\mu_X  
 }}}}}}
\Say{\gamma}{ \Lambda A \in \F_{ F_\BOR X \times Y} \. \int_X S( x, \mathrm{section}(A,x) ) \, \mathrm{d}\mu_X(x) }{\F_{F_\BOR X \times Y} \to \EReals_+}
\Assume{ A }{\TYPE{Disjoint}(\Nat,\F_{F_\BOR X \times Y})}
\Say{(1)}{ \ByDef\TYPE{Measure}(S(x, \cdot))} {\int_X S( x, \mathrm{section}\Act{ \bigcap^\infty_{n=1} A_n,x} ) \, \mathrm{d}\mu_X(x) =
\int_X \sum^n_{i = 1} S( x, \mathrm{section}(A_n,x) ) \, \mathrm{d}\mu_X(x)}
\Say{(2)}{\THM{IntegralSum}(2)}{\int_X S( x, \mathrm{section}\Act{ \bigcap^\infty_{n=1} A_n,x} ) \, \mathrm{d}\mu_X(x) =
\sum^\infty_{n=1} \int_X  S( x, \mathrm{section}(A_n,x) ) \, \mathrm{d}\mu_X(x)}
\Conclude{(3)}{\ByDef \gamma (2)}{ \gamma\Act{\bigcup^\infty_{n = 1} A_n} = \sum^\infty_{n = 1} \gamma(A_n)}
\DeriveConclude{(1)}{ \ByDef^{-1} \TYPE{Measure}(\cdot) }{ (\gamma : \TYPE{Measure}(\F_{F_\BOR X \times Y}))}
\EndProof
\\
\DeclareFunc{productMeasure}{ \TYPE{SlicingMeasure}(X,Y) \to \TYPE{Measure}(\F_{X \times Y})}
\DefineFunc{productMeasure}{S}{\THM{ProductMeasureTHM}(S)}
\\
\Theorem{ProductProbabilityTheorem}{
\ForEach{X}{\TYPE{ProbabilitySpace}}{
\ForEach{Y}{\BOR}{
\ForEach{P}{ \TYPE{SlicingMeasure} : \NewLine \ForEach{x}{X}{ S(x, Y) = 1}  }{
\FUNC{productMeasure}(P) : \TYPE{Probability}(X \times Y ) 
}}}}
\Say{\mathbb{P}}{\FUNC{productMeasure(P)}}{\TYPE{Measure}(\F_{X \times Y})}
\Say{(1)}{\THM{EqE}(\ByDef\FUNC{section}(\ByDef\TYPE{SlicingMeasure}(P),X \times Y))}{\NewLine : \int_X P(x | \FUNC{section}(X \times Y,x)) \, \mathrm{d}\mu_X(x) 
 = \int_X P(x  | Y) \, \mathrm{d}\mu_X(x)  }
\Say{(2)}{(1)\THM{EqE}(\ByDef(P))}{ \int_X P(x | \FUNC{section}(X \times Y,x)) \, \mathrm{d}\mu_X(x) 
 = \int_X \, \mathrm{d}\mu_X(x)}
\Say{(3)}{(2)\ByDef \TYPE{Probability}(\mu_X)}{\int_X P(x | \FUNC{section}(X \times Y,x)) \, \mathrm{d}\mu_X(x) = 1 }
\Say{(4)}{\ByDef{\mathbb{P}(X \times Y}(3)}{\mathbb{P}(X \times Y) = 1}
\Conclude{(*)}{\ByDef^{-1}\TYPE{Probability}}{(\mathbb{P} : \TYPE{Probability}(X \times Y))}
\EndProof
}
 \newpage
\Page{
\Theorem{ProductSFTHM}
{ 
\forall S : \TYPE{SlicingMeasure} \And \TYPE{Uniformly}  \, \SF{X,Y} : (\mu_X : \SF{X}) \. \NewLine   
 \. \FUNC{productMeasure}(S) : \SF{X \times Y} 
}
\Say{(B,b)}{\ByDef(\TYPE{Uniformly} \; \SF{X \times Y})(S)}{ \sum B : \Nat \to \F_Y : \bigcup^\infty_{n = 1} B_n = Y \.\NewLine \. \Nat \to \Reals_+ : 
 \ForEach{x}{X}{ \ForEach{n}{\Nat}{  S(x,B_n) \le b_n }}}
\Say{A}{\ByDef \SF{X}(\mu_X)}{ \Nat \to \F_X : \bigcup^\infty_{n = 1} A_n = X : \mu_X(A) < \infty }
\Say{(1)}{\THM{ProductPartition}(A,B)}{\bigcup_{n =1}^\infty \bigcup^\infty_{m = 1} A_n \times B_m = X \times Y }
\Say{\gamma}{\FUNC{productMeasure}(S)}{\TYPE{Measure}(X \times Y)}
\Assume{n,m}{\Nat}
\Conclude{(2)}{\ByDef\gamma(\A_n \times B_n) \THM{IntIneq}(\ByDef b_m)\THM{MeasureAsIntegral}(\mu_X,A_n)\ByDef(A_n) : \NewLine }
{\gamma(A_n \times B_m) = \int_{A_n} S(x, B_m) \, \mathrm{d} \mu_X(x) \le \int_{A_n} b_m \mathrm{d} \mu_X = b_m \mu_X(A) < \infty} 
\Derive{(2)}{ \THM{UI} }{\ForEach{n,m}{\Nat}{\gamma(A_n \times B_m) < \infty}}
\Conclude{(*)}{ \ByDef^{-1} \SF{X \times Y}(\gamma,A \times B,1,2)}{(\gamma : \SF{X \times U})}
\EndProof
\\
\DeclareFunc{productOfMeasures}{ \MEAS \to \MEAS  \to \MEAS}
\DefineNamedFunc{productOfMeasures}{(X,\F,\mu),(Y,\mathcal{G},\nu) }{ \mu \times \nu }{ \Act{X \times Y, \sigma(\F \times \mathcal{G}), A \mapsto \int_X \nu(\mathrm{section}(A,x)) \, \mathrm{d}\mu(x)} }
\\
\Theorem{ClassicalPMTHM}{
\ForEach{X,Y}{\MEAS}{
\ForEach{A \times B }{ F_X \times F_Y}{ \mu_X \times \mu_Y (A \times B ) = \mu_X( A  )\mu_Y( B) }
}}
\Conclude{(*)}{\ByDef\FUNC{productOfMeasure}(X,Y)\THM{ProductSection}(A,B)\THM{IntegralHomogenity}(\mu_Y(B) \NewLine \THM{MeasureAsIntegral}(\mu_X,A) }
{  \mu_X \times \mu_Y(A \times B) = \int_X \mu_Y(\FUNC{section}(A \times B), x )\, \mathrm{d}\mu_X(x) =
\NewLine
= \int_X \mu_Y(B)I_A \, \mathrm{d}\mu_X 
 = \mu_Y(B) \int_X I_A \mathrm{d}\mu_X  = \mu_Y(B) \mu_X(A) 
}
\EndProof
\\
\Theorem{MeasureProductCommute}{
\ForEach{X,Y}{\MEAS}{ \mu_X \times \mu_Y = \mu_Y \times \mu_X \circ \FUNC{swap} }
}
\Assume{A \times B}{\F_X \times \F_Y}
\Say{ (1) }{\THM{ClassicalPMTHM}(X,Y)(A \times B)}{ \mu_X \times \mu_Y(A \times B) = \mu_X(A)\mu_Y(B) }
\Say{(2)}{\THM{ClassicalPMTHM}(Y,X)(B \times A)}{ \mu_Y \times \mu_X(B \times A) = \mu_Y(B)\mu_X(A) }
\Conclude{(3)}{(1)(2)}{ \mu_X \times \mu_Y (A \times B) = \mu_Y \times \mu_X (B \times A)}
\DeriveConclude{(*)}{\THM{SwapIntro}(\cdot)}{\mu_X \times \mu_Y = \mu_Y \times \mu_X \circ \FUNC{swap}}
\EndProof
}
\newpage
\subsection{Fubbini Theorem }
\Page{
\Theorem{MeasrableOnProduct}{
\ForEach{X,Y}{\BOR}{
\ForEach{f}{\TYPE{Masurable}(X \times Y) }{
\ForEach{x}{X}{
\Lambda y : Y \. f(x,y) : \TYPE{Measurable}(Y)
}}}}
\Assume{A}{\B \EReals}
\Say{(1)}{ \THM{InversePointProduct}(f,x,A)}{f^{-1}(x,\cdot)(A) = \FUNC{section}(f^{-1}(  A ),x) }
\Say{(2)}{ \ByDef \TYPE{Measurable}(X \times Y)(f)(A)  }{ f^{-1}(A) : F_{X \times Y}}
\Conclude{(3)}{ (1)\THM{MeasurableSection}(x,f^{-1}(A))}{  f^{-1}(x,\cdot)   }
\DeriveConclude{(*)}{\ByDef^{-1}\THM{Measurable}(X)(\cdot)}{\Lambda y : Y \. f(x,y) : \TYPE{Measurable}(Y)}
\EndProof
\\
& Y : \BOR
\\
& X : \MEAS
\\
& S : \TYPE{SlicingMeasure}(X,Y) \And \TYPE{Uniformly}\SF{X,Y}
\\ 
& \nu = \FUNC{productMeasure}(S)
\\
\Theorem{FubiniI}{ 
\ForEach{f}{\TYPE{Measurable}(X \times Y) : f > 0}{
\ForEach{A}{\F_{X \times Y} \. \NewLine}{
\Lambda x : X \. \int_{\FUNC{section}(A,x)} f(x,y)  \, \mathrm{d} S(x, y) : \TYPE{Measurable}(X)
}}}
\Assume{B}{\F_Y}
\Assume{\phi}{\TYPE{Simple}(X \times Y)}
\Say{(n,b,c)}{\ByDef \TYPE{Simple}(X \times Y)}{ \Nat \times n \to \F_{X \times Y} \times n \to \Reals_{++} : \phi = \sum^n_{i=1}c_iI_{b_i} }
\Say{(1)}{\ByDef(n,b,c) \to \int_B \phi \, \mathrm{d}S}{\int_B \phi \, \mathrm{d}S = \sum^n_{i = 1} c_i S(x,  \FUNC{section}( X \times B \cup b_i,x)) }
\Conclude{(2)}{(1)\THM{MeasrableSlicing}(S, X \times B \cup b)}{ \int \phi \, \mathrm{d}S : \TYPE{Measurable}(X)}
\Derive{(1)}{  UI(\cdot)}{\forall \phi : \TYPE{Simple}(X \times Y) \. \int_B f \,  \mathrm{d}S : \TYPE{Measurable}(X)}
\Say{\phi}{ \THM{SimpleApprox}(f) }{ \Nat \to \TYPE{Simple}(X \times Y) : \phi_n \uparrow f   }
\Conclude{(2)}{ \THM{MonotoneConvergence} \left( \int_B \phi \, \mathrm{d} S , \int_B f \, \mathrm{d} S \right)}{ \int_B \phi \, \mathrm{d} S : \TYPE{Measurable}(X)}
\Derive{(1)}{ \ByDef^{-1}\TYPE{SlicingMeasure}(\cdot) }{ fS : \TYPE{SlicingMeasure}(X \times Y) }
\Conclude{(2)}{\THM{MeasurableSlicing}(fS)}{ \Lambda x \in X \. \int_{A_x} f(x,y) \, \mathrm{d}S(x, y) : \TYPE{Measurable}(X) }
\EndProof
}
\newpage
\Page{
\Theorem{FubiniII}{
\ForEach{f}{\TYPE{Measurable}(X \times Y) : f \ge 0  }{
\ForEach{A}{\F_{X \times Y }}{\int_X \int_{A_x} f(x,y) \, \mathrm{d} S(x,y) \, \mathrm{d} \mu(x) = \int_A f \, \mathrm{d} \nu(S)}}}
\Assume{B}{\F_Y}
\Conclude{(1)}{\ByDef\FUNC{Indicator}(B)\ByDef\FUNC{productMeasure}\ByDef\FUNC{Indicator}(B) = \NewLine}
{ \int_X \int_{A_x} I_B \, \mathrm{d} S \, \mathrm{d} \mu = \int_X \int_{A_x \cap B_x} \, \mathrm{d}S \, \mathrm{d}\mu = \nu(A \cap B) = \int_A I_B \, \mathrm{d} \nu }
\Derive{(1)}{\THM{UI}(\cdot)}{ \ForEach{B}{\F_{X \times Y}}{ \int_X \int_{A_x} I_B \, \mathrm{d} S  \, \mathrm{d}\mu = \int_{A} I_B  \, \mathrm{d} \nu  }}
\Assume{\phi}{\TYPE{Simple}(X \times Y)}
\Say{(n,b,c)}{\ByDef \TYPE{Simple}(X \times Y)}{ \Nat \times n \to \F_{X \times Y} \times n \to \Reals_{++} : \phi = \sum^n_{i=1}c_iI_{b_i} }
\Conclude{(2)}{\ldots}{ \int_X \int_{A_x} \phi \, \mathrm{d} S \mathrm{d} \mu 
= \int_{X} \int_{A_x} \sum^n_{i=1}c_i I_{b_i} \, \mathrm{d} S \mathrm{d} \mu
=  \sum^n_{i = 1} c_i \int_X \int_{A_x}  I_{b_i} \, \mathrm{d} S \mathrm{d} \mu = \NewLine
=  \sum^n_{i = 1} c_i \int_A I_{b_i} \, \mathrm{ d } \nu
= \int_A \sum^n_{i = 1} I_{b_i} \, \mathrm{ d } \nu
= \int_A \phi \mathrm{ d } \nu
 }
\Derive{(2)}{\THM{UI}(\cdot}{ \ForEach{\phi}{\TYPE{Simple}(X \times Y)}{ \int_X \int_{A_x} \phi \, \mathrm{d} S \, \mathrm{d} \mu = \int_A \phi \mathrm{ d } \nu } }
\Say{\phi}{\THM{SimpleApproximation}(f)}{ \Nat \to \TYPE{Simple}(X \times Y) : \phi \uparrow f }
\Conclude{ (3)}{\ldots}{
\int_X \int_{A_x} f \,  \mathrm{d} S \, \mathrm{d} \mu  
= \int_X \int_{A_x} \lim_{n \to \infty} \phi_n \, \mathrm{d} S \, \mathrm{d} \mu
= \lim_{n \to \infty} \int_X \int_{A_x} \phi_n \, \mathrm{d} S \, \mathrm{d} \mu
= \lim_{n \to \infty} \int_A \phi_n \, \mathrm{d} \nu = \NewLine
= \int_{A} \lim_{n \to \infty} \phi_n \, \mathrm{d} \nu
= \int_{A} f \, \mathrm{d} \nu
}
\EndProof
\\
\Theorem{ToneliI}{
\ForEach{f}{\TYPE{IntegralExists}(X \times Y, \nu)}{
 \Lambda x \in X \.  \Lambda y \in Y \. f(x,y) : \TYPE{IntegralExists}(Y,S(x)) \AE{\mu}
}}
\Say{(1)}{ \THM{FubiniII}(f_+,X \times Y) }
{ \int_X \int_Y f_+ \, \mathrm{d} S \, \mathrm{d}\mu 
 = \int_{X \times Y} f_+ \mathrm{d}{\nu}  
 }
\Say{(2)}{ \THM{FuniniII}(f_-,X \times Y)}
{
 \int_X \int_Y f_- \, \mathrm{d} S \, \mathrm{d} \mu
 = \int_{X \times Y} f_- \mathrm{d}{\nu}
}
\Say{(3)}{ \ByDef\TYPE{IntegralExists}(\mu)\ByDef\FUNC{Integrate}(f,\nu)((1),(2))\ByDef\FUNC{Integrate}(f,S(x))  : \NewLine }
{ \TYPE{Error} \neq \int_{X \times Y} f  \,\mathrm{d} \nu  
 = \int_{X \times Y} f_+ \, \mathrm{d} \nu
  - \int_{X \times Y} f_- \, \mathrm{d} \nu =
 \int_X \int_Y f_+ \, \mathrm{d} S \, \mathrm{d} \mu
 - \int_X \int_Y f_- \, \mathrm{d} S \, \mathrm{d} \mu =
\int_X \int_Y f \, \mathrm{d} S \, \mathrm{d} \mu
}
\Say{(4)}{\THM{IntegralEq}\left( \mu  , \int_Y f \, \mathrm{d} S, \TYPE{Error} \right)}
{ \int_Y f \mathrm{d} \, S  \neq \TYPE{Error}  \, \AE{\mu} }
\Conclude{(*)}{\ByDef^{-1} \TYPE{IntegralExists}(4)}{ ( f : \TYPE{IntegralExists}(Y,S) \, \AE{\mu} ) }
\EndProof
}
\newpage
\Page{
\Theorem{ToneliII}{
\ForEach{f}{\TYPE{Integrable}(X \times Y, \nu)}{
 \Lambda x \in X \.  \Lambda y \in Y \. f(x,y) : \TYPE{Integrable}(Y,S(x)) \AE{\mu}
}}
\Say{(1)}{ \THM{FubiniII}(f_+,X \times Y) }
{ \int_X \int_Y f_+ \, \mathrm{d} S \, \mathrm{d}\mu 
 = \int_{X \times Y} f_+ \mathrm{d}{\nu}  
 }
\Say{(2)}{ \THM{FuniniII}(f_-,X \times Y)}
{
 \int_X \int_Y f_- \, \mathrm{d} S \, \mathrm{d} \mu
 = \int_{X \times Y} f_- \mathrm{d}{\nu}
}
\Say{(3)}{ \ByDef\TYPE{IntegralExists}(\mu)\ByDef\FUNC{Integrate}(f,\nu)((1),(2))\ByDef\FUNC{Integrate}(f,S(x))  : \NewLine }
{ \infty >  \int_{X \times Y} |f|  \,\mathrm{d} \nu  
 = \int_{X \times Y} f_+ \, \mathrm{d} \nu
  + \int_{X \times Y} f_- \, \mathrm{d} \nu =
 \int_X \int_Y f_+ \, \mathrm{d} S \, \mathrm{d} \mu
 + \int_X \int_Y f_- \, \mathrm{d} S \, \mathrm{d} \mu =
\int_X \int_Y f \, \mathrm{d} S \, \mathrm{d} \mu
}
\Say{(4)}{\THM{IntegralIneq}\left( \mu  , \int_Y f \, \mathrm{d} S, \infty \right)}
{ \int_Y |f| \,  \mathrm{d}  S  < \infty  \, \AE{\mu} }
\Conclude{(*)}{\ByDef^{-1} \TYPE{Integrable}(4)}{ ( f : \TYPE{Integrable}(Y,S) \, \AE{\mu} ) }
\EndProof
\\
\Theorem{Toneli0}{
\ForEach{f}{\TYPE{IntegralExists}(X \times Y, \nu)  \. \NewLine }
{ \exists \phi : \TYPE{IntegralExists}(X \times Y, \nu) :  \int_Y \phi  \, \mathrm{d}S : \TYPE{Measurable}(X) : \phi =_\mu  f
}}
\Say{(1)}{ \THM{ToneliI}(f)}{ f : \TYPE{Integrable}(Y,S) \, \AE{\mu} }
\Say{\phi}{ \Lambda (a,b) \in X \times Y \. \If \int_Y f(a,y) \, \mathrm{d}S(x,y) = \TYPE{Error} \Then 0 \Else f(a,b)  }
{ \TYPE{Integralexists}}
\Say{(2)}{\THM{FubiniI}(\phi_+)}{ \int_Y f_+ \, \mathrm{d}S : \TYPE{Measurable}(X) }
\Say{(3)}{ \THM{FubiniI}(\phi_-)}{ \int_Y f_- \, \mathrm{d}S : \TYPE{Measurable}(X)}
\Say{(4)}{ \THM{AdditiveIntegral}(\phi_+, -\phi_-) }{ \int_Y \phi_+ \, \mathrm{d}S - \int_Y \phi_- \, \mathrm{d}S =   \int_Y \phi \mathrm{d}S }
\Say{(*)}{\THM{ContinousPreserveMeasureable}(2,3,4)}{ \int_Y \phi \, \mathrm{d}S : \TYPE{Measurable}(X)}
\EndProof
\\
\Theorem{FubiniToneli}{
\ForEach{f}{\TYPE{Measurable}(X \times Y)  : \int_{X \times Y} |f| \, \mathrm{d}\nu < \infty \NewLine }
{ \int_{X \times Y} f \, \mathrm{d} \nu = \int_X \int_Y f \mathrm{d}S \,  \mathrm{d}\mu  }}
\\
\Theorem{ClassicalFubini}{
 \ForEach{\nu}{\TYPE{Measure}(Y)}{
 \ForEach{ f }{\TYPE{IntegralExists}(X \times Y, \mu \times \nu) \. \NewLine}{
  \int_{X \times Y} f \, \mathrm{d}\mu \times \nu =
   \int_X \int_Y f \, \mathrm{d}\mu \, \mathrm{d} \nu =
   \int_Y \int_X f \, \mathrm{d}\nu  \,\mathrm{d} \mu
}}} 
}
\newpage
\subsection{Iterated Integrals}
\Page{
\DeclareType{MeasureSystem}{ \prod n \in \Nat \. \prod X : n \to \BOR \. ?( \prod m : n \. \prod^{m -1}_{i = 1} X_i \to  \TYPE{Measure}(X_m)   )   }
\DefineType{ P  }{ \TYPE{MeasureSystem} }{ \forall  m  \in n \. \forall A \in \F_{X_m} \. P( \cdot, A) : \TYPE{Measurable}(X_m)  }
\\
\DeclareFunc{iteratedMeasure}{ \prod n \in \Nat \. \prod X : n \to \BOR \. \TYPE{MeasureSystem}(X) \to \EReals_+  }
\DefineNamedFunc{iteratedMeasure}{ P }{ \int_X \, \mathrm{d}P }{ \int_{X_1} \int_{X_{|\overline{2,n}}}\, \mathrm{d}P_x \, \mathrm{d}P_1(x) }
\\
& n : \Nat
\\ \\
& X : n \to \BOR
\\ \\
& P : \TYPE{MeasureSystem}(X)
\\ \\
\Theorem{IteratedMPTHM}{ 
 \exists \mu : \TYPE{Measure}\left( \prod^n_{i = 1} X_i \right) : \forall A : \prod m : n \. \F_{X_m} \. \mu \left( \prod^n_{i = 1} A_i  \right) = \int_A \, \mathrm{d} P
}        
& \text{ Use MPTHM repeadetly  } \\
\EndProof
\\
\DeclareFunc{ iteratedProductMeasure }{ \TYPE{MeasureSystem}(X) \to \MEAS }
\DefineNamedFunc{ \FUNC{iteratedProductMeasure } }{P}{ \left( \prod^n_{i = 1} X_i , P\right) }{ \left( \prod^n_{i = 1} X_i, \THM{IteratedMPTHM}(P)    \right) }
\\
\DeclareType{ Uniformly \, \SF{\cdot} \, System  }{?\TYPE{MeasureSystem}(X)}
\DefineType{ P  }{ Uniformly \, \SF{X} \, System }{  \forall m : n \. P_m : \TYPE{Uniformly} \, \SF{ \prod^{m - 1}_{i = 1} X_i }  }
\\
\Theorem{ SFSystem}{ P : \TYPE{uniformly \, \SF \, System} \Rightarrow  \left( \prod^n_{i = 1} X_i, P \right) : \SF{  \prod^n_{i = 1} X_i } }
\\
\DeclareFunc{ iteratedIntegral}{ \TYPE{IntegralExists}\left( \prod^n_{i = 1} X_i, P \right) \to \EReals  }
\DefineNamedFunc{ iteratedIntegral }{ f }{ \int_X f \, \mathrm{d}P }{ \int_{X_1} \int_{X_{|\overline{2,n}}} f_x \, \mathrm{d}P_x \, \mathrm{d}P_1(x)  }
}
\newpage
\Page{
 \DeclareType{ProbabilitySystem}{ ? \TYPE{MeasureSystem}(X) }
 \DefineType{P}{ProbabilitySystem}{ \forall m : n \. \forall x \in \prod^{m -1}_{i = 1} X_i \. P(X, \cdot)  : \TYPE{Probability}(X_i) }
\\
& P : \TYPE{ProbabilitySystem}(X)   
\\
\Theorem{IteratedPPTHM}{ ( \prod^n_{i = 1} X_i, P) : \TYPE{Probability}\left( \prod^n_{i = 1} X_i   \right)    }
} 
\newpage
\subsection{Infinite Products}
\Page{
\DeclareType{Cylinder}{ \prod X : \Nat \to \Set \. \prod n \in \Nat \. 
?\left(\prod^n_{i=1} X_{i}\right) \to ?\prod^\infty_{i=1}  X_i }
\DefineType{ C }{ \TYPE{Cylinder}(A) }{ \pi_{1,\ldots,n} C = A }
\\ 
\DeclareType{MeasurableCylinder}{ \prod X : \Nat \to \BOR \.\prod n \in \Nat \.  \F_{\prod^n_{i=1}X_i} \to ?\prod^\infty_{i=1} X_i }
\DefineType{ C : \TYPE{MeasurableCylinder}(A) }{  C : \TYPE{Cylinder}(A) }
\\ 
\DeclareFunc{infiniteBorProduct}{ (\Nat \to \BOR) \to \BOR }
\DefineNamedFunc{InfiniteBorProduct}{X_i, \F_i}{\prod^n_{i=1} (X_i,\F_i)}{ \left( \prod^\infty_{i = 1} X_i, \sigma( \TYPE{MeasurableCylinder}(X) ) \right)}
\\ 
\DeclareFunc{cylinder}{ \prod X : \Nat \to \Set \. \prod n \in \Nat \.  \prod A \subset \prod^n_{i = 1} X_i \to  \TYPE{Cylinder}(X,n,A) }
\DefineFunc{cylinder}{ A  }{ A \times \prod^\infty_{i = n + 1} X_i }
\\ 
\DeclareType{DiscreteRandomProcess}{ \prod  X :  \Nat \to \BOR \. ?\left( \prod n : \Nat \.    \prod^n_{i = 1} X_i \to 
\TYPE{Probability}  X_{n + 1}  \right)  } \\
\DefineType{ P} { \DRP }
{ \NewLine \iff \forall n \in \Nat \.  \forall A \in \F_{ X_n } \. 
 \Lambda x \in \prod^{ n - 1}_{i = 1} \.  P(x,A) : \TYPE{Measureble} \prod^{n - 1}_{i = 1 }X_i }
\\
\Theorem{InfiniteProductTheoremI}{ 
\ForEach{ X}{ \Nat \to \BOR }{ 
\ForEach{P}{\DRP(X)}{
\ForEach{n}{\Nat}{  \NewLine \.
\Lambda B \in \F_{ \prod^n_{i=1} X_i  } \.  \int_{X} I_{B} \mathrm{d}P_{|n} : \TYPE{Probability}\left( \prod^n_{i = 1 } X_i \right) 
} } } }
\Say{(1)}{\ByDef^{-1}( \ByDef \DRP  (P) )}{ \left( P_{|n} : \TYPE{ProbabilitySystem}( X_{|n}) \right)}
\Conclude{(*)}{\THM{IteretadPPTHM}( P_{|n} ) }{ \left( \left( \prod^n_{i = 1} X_i, P_{|n}  \right)  : \TYPE{Probability}\left(  \prod^n_{i = 1} X_i   \right) \right) } 
\EndProof
\\
& X : \Nat \to \BOR 
\\ \\
& P : \DRP(X) 
}
\newpage
\Page{
\DeclareFunc{finiteTimeProbability}{ \prod n \in \Nat \. \TYPE{Probability}\left ( \prod^n_{i = 1} X  \right )  }
\DefineNamedFunc{finiteTimeProbability}{ t  }{ P_t }{ \THM{InfiniteProductTheoremI}(X, P, t )  }
\\
\Theorem{InfiniteProductTheoremII}{
\exists \mathbb{P} : \TYPE{Probabilty}\left(  \prod^\infty_{i = 1} X_i  \right) : 
\ForEach{ t }{ \Nat }{ \ForEach{ B }{ \prod^n_{i = 1} X_i  }{
 \mathbb{P}( \FUNC{cylinder}(B)  ) =  P_t(B)   
} } } 
& \ldots \\
\EndProof 
\Theorem{ClassicalIPTHM}{
\ForEach{ (X,\F,P) }{ \Nat \to \TYPE{ProbabilitySpace}  }
{ \prod^\infty_{i = 1} P_i : \TYPE{Probability}\left(  \prod^\infty_{i = 1} (X_i,\F_i) \right)  
}}
& \ldots \\
\EndProof
\\
\DeclareType{GeneralCylinder}{   \prod T : \Set \.\prod X : T \to \Set \. \prod \tau : \TYPE{Finite}(T) \. ?\prod_{t \in \tau } X_t \to ??\prod_{t \in T} X_t }
\DefineType{C}{\TYPE{GeneralCylinder}(A)}{  C = A \times \prod_{t \in \tau^c} }    
\\
& T : \Set \\
\\
& X : T \to \BOR \\
\\
\DeclareType{GeneralMeasurableCylinder}{  \prod \tau : \TYPE{Finite}(T) \. \F_{\prod_{t \in \tau} X_t} \to ??\prod_{t \in T} X_t }
\DefineType{C}{ \TYPE{GeneralMeasurableCylinder}(A)}{ \TYPE{MeasurableCylinder}(A)  }
\\
\DeclareFunc{generalBorProduct}{  (T \to  \BOR ) \to \BOR    }
\DefineNamedFunc{generalBorProduct}{ (X, \F) }{ \prod_{t \in T} (X_t, \F_t)  }{  ( \prod_{t \in T} X_t, \sigma \left( \TYPE{GeneralMeasurableCylinder}(X, \F) \right)   )  }
\\
\DeclareFunc{generalCylinder}{ \prod \tau : \TYPE{Finite}(T) \. \F_{\prod_{t \in \tau} X_t } \to \F_{\prod_{t \in T} X_t }}
\DefineFunc{generalCylinder}{ B  }{ B \times \prod_{t \in \tau^\c} X_t}
\\
\DeclareType{KolmogorovConsistent}{ ?( \prod \tau : \TYPE{Finite}(T) \. \TYPE{ProbabilitySystem}\left( \prod_{t \in \tau}  \right) )  }
\DefineType{ P }{KolmogorovConsistent}{ \forall \tau : \TYPE{Finite}(T) \. \forall \theta \subset \tau \. \pi_\theta( P_\tau) = P_\theta  }
}
\newpage
\Page{
\Theorem{ KolmogorovExtension }{ 
\ForEach{X}{T \to \TYPE{Polish}}{
\ForEach{P}{ \TYPE{KolmogorovConsistent}(X, \B X ) }{ \NewLine \. 
\exists \mathbb{P} :\TYPE{Probability}\left( \prod_{ t \in T}  (X_t, \B X_t)  \right)
 : 
\ForEach{\tau}{\TYPE{Finite}(T)}{ 
 \pi_\tau \mathbb{P} = P_\tau
}}}}
\Say{ \F_0 }{  \TYPE{GeneralMeasurableCylinder}(X, \B X)  }{\Set}
\Say{ \mathbb{P}}{ \Lambda A \times \prod_{t \in \tau^\c } X_t : \TYPE{GeneralMeasurableCylinder}(X, \B X)(\tau) \. P_\tau(A)  }{ \NewLine :
\TYPE{ GeneralMeasurableCylinder}(X, \B X) \to [0,1] }
\Assume{A}{\TYPE{DisjointElems}(\F_0)}
\Say{\tau}{ \bigcup^n_{i = 1} \tau_A }{ \TYPE{Finite}(T) }
\Say{ B }{ \ByDef\TYPE{GeneralMeasurableCylinder}(A) }{ n \to \F_{\prod t \ in \tau} : \forall i \in n \. A_i = B_i \times \prod_{t \in \tau} X_i }
\Say{  (1)  }{  \ByDef B \ByDef \mathbb{P}}{  \mathbb{P} \left(  \bigcup^n_{i = 1} A_i  \right) 
 = P_\tau \left(  \bigcup^n_{i = 1} B_i \right) }
\Say{ (2)  }{ (1)\ByDef\TYPE{Measure}( P_\tau)   }{ \mathbb{P} \left( \bigcup^n_{i = 1} A_i \right) =  \sum^n_{i = 1} P_\tau ( B_i ) }
\Conclude{ (3) }{ (2)\ByDef \mathbb{P} }{  \mathbb{P} \left( \bigcup^n_{i = 1} A_i \right) = \sum^n_{i = 1} \mathbb{P}(A_i) }
\Derive{  (1)   }{  (\cdot)    }{  \mathbb{P} : \TYPE{FinitelyAdditive}  }
\Assume{ A  }{ \Nat \to \F_0 : A \downarrow \emptyset }
\Assume{ \epsilon }{ \Reals_{++}  : \forall n : \Nat \. \mathbb{P}(A_n) > \epsilon }
\Say{ ( \tau, B ) }{ \ByDef \F_0 (A) }{ \Nat \to \sum \tau : \TYPE{ Finite  }(T) \. \TYPE{GeneralMeasurableCylinder}(X, \B X, )   }
\Say{   C    }{ \THM{PolishISTight}(X)(P)(B)(\Lambda k \in \Nat \. \frac{\epsilon}{2^{k +1}}) }{ \prod n \in \Nat \. \TYPE{Compact}\left( \prod_{t \in \tau} X_i \right)
 : \forall n \Nat \. P_{\tau_n} }
\Say{ \alpha  }{ \FUNC{ generalMeasurableCylinder}( C  ) }{ \prod n \in \Nat \. \TYPE{GeneralMeasurableCylinder}( \tau_n  ) }
\Say{ (2)  }{ \ByDef( D )(1)\ByDef( \alpha )\ByDef{\alpha}(\ByDef C  )  }{
 \forall n \in \Nat \. \mathbb{P} \left(  A_n \setminus D_n  \right)     
 =  \mathbb{P} \left(   A_n \cap \bigcup^n_{ i = 1  } \alpha_i^\c \right)  
 \le \sum^n_{i = 1} \mathbb{P}( A_i  \cap  \alpha_i ) = \NewLine =
\sum^n_{i = 1} P_{ \tau_i}( B_i \setminus C_i  ) <
\sum^n_{i = 1} \frac{ \epsilon }{2^{n + i }} 
\le \epsilon/2 }
\Say{ (3) }{ \THM{IntersectionIsSubset}(\ByDef(D)) }{ \forall n \in \Nat \. D_n \subset A_n  }
\Say{ (4) }{ \THM{SubsetDifference}((3))(2) }{ \forall n \in \Nat \. \mathbb{P}(D_n) > \mathbb{P}(P_n) - \frac{\epsilon}{ 2 }}
\Say{ (5) }{ \ByDef\TYPE{Probability}(4,\ByDef(\epsilon) ) }{  \forall n :  \Nat \. D_n \neq \emptyset  }
\Say{ x }{ \ByDef\TYPE{NonEmpty}(D,5) }{ \prod n : \Nat \. D_n}
\Assume{n}{\Nat}
\Say{(6)}{ \ByDef D_n(  \ByDef x) }{ \forall m : \Nat : m \ge n \. \pi_{\tau_n} x_n \in C_n}
\Say{(7)}{ \THM{PolishIsSeqCompact}\left( \prod_{t \in \tau_n} X_t, C_n \right) }{ ( C_n : \TYPE{SeqCompact}) }
}
\newpage
\Page{
\Say{(m,y)}{ \ByDef\TYPE{SeqCompact}(C_n, \pi_{\tau_n} x )   }{ \TYPE{Subseqer} \times C_n : \lim_{n \to \infty} x_{m_n} = y }
\Conclude{y_n}{y}{C_n}
\Derive{y}{[\cdot]}{\prod n \in \Nat \. C_n}
\Say{(6)}{\ByDef y  }{  \forall n : \Nat \. \forall m : \Nat : m > n \. \pi_{\tau_m}(y_n) = y_m  }
\Say{ Y }{\FUNC{restore}(y,6)}{ \bigcap^\infty_{n =1 } D_n  }
\Say{ (7) }{ \ByDef\TYPE{NonEmpty}\left( \bigcap^\infty_{n = 1} D_n, Y \right)    }{ \bigcap^\infty_{n = 1} D_n \neq \emptyset }
\Say{ (8) }{ \THM{SubsetIntersection}(D,A)}{ \bigcap^\infty_{n = 1} D_n \subset \bigcap^\infty_{n = 1} A_n   }
\Say{(9)}{ \THM{EmptySubset}(8, \ByDef A) }{ \bigcap^\infty_{n = 1} D_n = \emptyset }
\Conclude{(10)}{(7)(9)}{ \bot }
\DeriveConclude{ (2)  }{ \ByDef\TYPE{Convergent}(\Reals_+)( \mathbb{P}(A_n )}{ \lim_{n \to \infty} \mathbb{P}(A_n) = 0 }
\Derive{ (2) }{ \ByDef\TYPE{CountablyAdditive}(\mathbb{P}) }{ \left( \mathbb{P} : \TYPE{CountablyAdditive}\left( \prod_{t \in T} X_t, \F_0 \right) \right) }
\Conclude{ Q  }{\THM{CaratheodoryExtension}(\mathbb{P}) }{ \TYPE{Probability} \left( \prod_{t \in T} (X_t, \B X_t) \right)
: \forall \tau : \TYPE{Finite}(T) \. \pi_{\tau} Q = P_\tau
}
\EndProof
\\
& X : T \to \TYPE{Polish} \\
\\
\DeclareFunc{ RandomFieldLaw  }{ \TYPE{KolmogorovConsistent}(X, \B X) \to  \TYPE{Probability} \prod_{t \in T} X_t  }
\DefineNamedFunc{ RandomFieldLaw  }{ P }{[P]}{ \THM{ KolmogorovExtension  }(P) }
}
\newpage
\section{Finitly Additive Functionals}
\newpage
\section*{Sources:}
\begin{enumerate}
\item K. P. S. Bashkra Rao ---Theory of Charges: A Study of Finitely Additive Measures 1975
\item  R. Ash     --- Probability and Measure Theory 2000
\item  В. И. Богачев --- Основы Теории меры (том 1) 2006 
\item  D. H. Fremlin --- Measure Theory (11,12,13,21,23,25) 2016
\item P. Bouafia, T. De Pauw ---  Localizable Locally Determined Measurable spaces with Neglidgibles 2020
\end{enumerate}
\end{document}

