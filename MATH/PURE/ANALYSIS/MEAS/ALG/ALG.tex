\documentclass[12pt]{scrartcl}
\usepackage{mathtools}
\usepackage[T2A]{fontenc}
\usepackage[utf8]{inputenc}
\usepackage{amsmath}
\usepackage{amsfonts}
\usepackage{hyperref}
\usepackage{amssymb}
\usepackage{ wasysym }
\usepackage{ upgreek }
\usepackage[dvipsnames]{xcolor}
\usepackage[a4paper,top=5mm, bottom=20mm, left=10mm, right=2mm]{geometry}
\renewcommand\pagemark{{\usekomafont{pagenumber}\thepage\ }}
%Markup
\newcommand{\TYPE}[1]{\textcolor{NavyBlue}{\mathtt{#1}}}
\newcommand{\FUNC}[1]{\textcolor{Cerulean}{\mathtt{#1}}}
\newcommand{\LOGIC}[1]{\textcolor{Blue}{\mathtt{#1}}}
\newcommand{\THM}[1]{\textcolor{Maroon}{\mathtt{#1}}}
%META
\renewcommand{\.}{\; . \;}
\newcommand{\de}{: \kern 0.1pc =}
\newcommand{\extract}{\LOGIC{Extract}}
\newcommand{\where}{\LOGIC{where}}
\newcommand{\If}{\LOGIC{if} \;}
\newcommand{\Then}{ \; \LOGIC{then} \;}
\newcommand{\Else}{\; \LOGIC{else} \;}
\newcommand{\IsNot}{\; ! \;}
\newcommand{\Is}{ \; : \;}
\newcommand{\DefAs}{\; :: \;}
\newcommand{\Act}[1]{\left( #1 \right)}
\newcommand{\Example}{\LOGIC{Example} \; }
\newcommand{\Theorem}[2]{& \THM{#1} \, :: \, #2 \\ & \Proof = \\ } 
\newcommand{\DeclareType}[2]{& \TYPE{#1} \, :: \, #2 \\} 
\newcommand{\DefineType}[3]{& #1 : \TYPE{#2} \iff #3 \\} 
\newcommand{\DefineNamedType}[4]{& #1 : \TYPE{#2} \iff #3 \iff #4 \\} 
\newcommand{\DeclareFunc}[2]{& \FUNC{#1} \, :: \, #2 \\}  
\newcommand{\DefineFunc}[3]{&  \FUNC{#1}\Act{#2} \de #3 \\} 
\newcommand{\DefineNamedFunc}[4]{&  \FUNC{#1}\Act{#2} = #3 \de #4 \\} 
\newcommand{\NewLine}{\\ & \kern 1pc}
\newcommand{\Page}[1]{ \begin{align*} #1 \end{align*}   }
\newcommand{ \bd }{ \ByDef }
\newcommand{\NoProof}{ & \ldots \\ \EndProof}
%LOGIC
\renewcommand{\And}{\; \& \;}
\newcommand{\ForEach}[3]{\forall #1 : #2 \. #3 }
\newcommand{\Exist}[2]{\exists #1 : #2}
\newcommand{\Imply}{\Rightarrow} 
\newcommand{\Intro}{\LOGIC{I}}
\newcommand{\Elim}{\LOGIC{E}}
%TYPE THEORY
\newcommand{\Type}{\TYPE{Type}}
%\DeclareMathOperator*{\dom}{dom}
%%STD
\newcommand{\Int}{\mathbb{Z} }
\newcommand{\NNInt}{\mathbb{Z}_{+} }
\newcommand{\Reals}{\mathbb{R} }
\newcommand{\Complex}{\mathbb{C}}
\newcommand{\Rats}{\mathbb{Q} }
\newcommand{\Sphere}{\mathbb{S}}
\newcommand{\Ball}{\mathbb{B}}
\newcommand{\Nat}{\mathbb{N} }
\newcommand{\EReals}{\stackrel{\mathclap{\infty}}{\mathbb{R}}}
\newcommand{\ERealsn}[1]{\stackrel{\mathclap{\infty}}{\mathbb{R}}^{#1}}
\DeclareMathOperator*{\centr}{center}
\DeclareMathOperator*{\argmin}{arg\,min}
\DeclareMathOperator*{\id}{id}
\DeclareMathOperator*{\im}{Im}
\DeclareMathOperator*{\supp}{supp}
\newcommand{\EqClass}[1]{\TYPE{EqClass}\left( #1 \right)}
\newcommand{\End}{\mathrm{End}}
\newcommand{\Aut}{\mathrm{Aut}}
\mathchardef\hyph="2D
\newcommand{\ToInj}{\hookrightarrow}
\newcommand{\ToMono}{\xhookrightarrow}
\newcommand{\ToSurj}{\twoheadrightarrow}
\newcommand{\ToEpi}{\xtwoheadrightarrow}
\newcommand{\ToBij}{\leftrightarrow}
\newcommand{\ToIso}{\xleftrightarrow}
\newcommand{\Arrow}{\xrightarrow}
\newcommand{\Set}{\TYPE{Set}}
\newcommand{\du}{\; \triangle \;}
\renewcommand{\c}{\complement}
\renewcommand{\i}{\mathbf{i}}
\newcommand{\Eqmod}[3]{#1 = #2 \quad \mathrm{mod} \quad #3}
%%ProofWritting
\newcommand{\Say}[3]{& #1 \de #2 : #3, \\}
\newcommand{\SayIn}[3]{& #1 \de #2 \in #3, \\}
\newcommand{\Conclude}[3]{& #1 \de #2 : #3; \\}
\newcommand{\Derive}[3]{& \leadsto #1 \de #2 : #3, \\}
\newcommand{\DeriveConclude}[3]{& \leadsto #1 \de #2 : #3 ; \\} 
\newcommand{\Assume}[2]{& \LOGIC{Assume} \; #1 : #2, \\}
\newcommand{\AssumeIn}[2]{& \LOGIC{Assume} \; #1 \in #2, \\}
\newcommand{\As}{\; \LOGIC{as } \;} 
\newcommand{\ByDef}{\LOGIC{E}}
\newcommand{\QED}{\; \square}
\newcommand{\EndProof}{& \QED \\}
\newcommand{\Proof}{\LOGIC{Proof} \; }
\newcommand{\Explain}[1]{& \text{#1.} \\}
\newcommand{\ExplainFurther}[1]{& \text{#1} \\}
\newcommand{\Exclaim}[1]{& \text{#1!} \\}
%SetTheory
\newcommand{\NonEmpty}{\TYPE{NonEmpty}}
\newcommand{\Finite}{\TYPE{Finite}}
\newcommand{\Countable}{\TYPE{Countable}}
\newcommand{\Uncountable}{\TYPE{Uncountable}}
\newcommand{\Ideal}{\TYPE{Ideal}}
\newcommand{\Inj}{\TYPE{Injective}}
\newcommand{\Surj}{\TYPE{Surjective}}
\newcommand{\Bij}{\TYPE{Bijective}}
\newcommand{\SIdeal}{\TYPE{\sigma\hyph \Ideal}}
\newcommand{\SA}{\TYPE{\sigma \hyph Algebra}}
\newcommand{\Eq}{\TYPE{Equivalence}}
%CategoryTheory
%Types
\newcommand{\Cov}{\TYPE{Covariant}}
\newcommand{\Contra}{\TYPE{Contravariant}}
\newcommand{\NT}{\TYPE{NaturalTransform}}
\newcommand{\UMP}{\TYPE{UnversalMappingProperty}}
\newcommand{\CMP}{\TYPE{CouniversalMappingProperty}}
\newcommand{\paral}{\rightrightarrows}
%functions
\newcommand{\op}{\mathrm{op}}
\newcommand{\obj}{\mathrm{obj}}
\DeclareMathOperator*{\dom}{dom}
\DeclareMathOperator*{\codom}{codom}
\DeclareMathOperator*{\colim}{colim}
%variable
\renewcommand{\C}{\mathcal{C}}
\newcommand{\A}{\mathcal{A}}
\newcommand{\B}{\mathcal{B}}
\newcommand{\D}{\mathcal{D}}
\newcommand{\I}{\mathcal{I}}
\newcommand{\J}{\mathcal{J}}
\newcommand{\R}{\mathcal{R}}
%Cats
\newcommand{\CAT}{\mathsf{CAT}}
\newcommand{\SET}{\mathsf{SET}}
\newcommand{\PARALLEL}{\bullet \paral \bullet}
\newcommand{\WEDGE}{\bullet \to \bullet \leftarrow \bullet}
\newcommand{\VEE}{\bullet \leftarrow \bullet \to \bullet}
%OrderTheory
%Types
\newcommand{\Poset}{\TYPE{Poset}}
\newcommand{\Toset}{\TYPE{Toset}}
\newcommand{\Pres}{\TYPE{PreorderedSet}}
\newcommand{\WF}{\TYPE{WellFounded}}
\newcommand{\WO}{\TYPE{WellOrdered}}
\newcommand{\II}{\TYPE{InitialInterval}}
\newcommand{\UB}{\TYPE{UpperBound}}
\newcommand{\LUB}{\TYPE{LowerUpperBound}}
\newcommand{\LB}{\TYPE{LowerBound}}
\newcommand{\ULB}{\TYPE{UpperLoweBound}}
%Cats
\newcommand{\POSET}{\mathsf{POSET}}
\newcommand{\ORD}{\mathsf{ORD}}
%Symbols
\renewcommand{\P}{\mathsf{P}}
%\newcommand{\F}{\mathsf{F}}
%\newcommand{\U}{\mathsf{U}}
%Algebra
%Groups
%Types
\newcommand{\Group}{\TYPE{Group}}
\newcommand{\Abel}{\TYPE{Abelean}}
\newcommand{\Sgrp}{\subset_{\mathsf{GRP}}}
\newcommand{\Nrml}{\vartriangleleft}
\newcommand{\FG}{\TYPE{FiniteGroup}}
\newcommand{\Stab}{\mathrm{Stab}}
\newcommand{\FGA}{\TYPE{FinitelyGeneratedAbelean}}
\newcommand{\DN}{\TYPE{DirectedNormality}}
\newcommand{\ActsOn}{\curvearrowright}
%Func
\DeclareMathOperator{\tor}{tor}
\DeclareMathOperator{\ord}{ord}
\DeclareMathOperator{\bool}{bool}
\DeclareMathOperator{\rank}{rank}
%Cats
\newcommand{\GRP}{\mathsf{GRP}}
\newcommand{\ABEL}{\mathsf{ABEL}}
%Boolean Algebra
%TYPE
\newcommand{\Bool}{\mathbb{B}}
\newcommand{\Alg}{\TYPE{Algebra}}
\newcommand{\BR}{\TYPE{BooleanRing}}
\newcommand{\BA}{\TYPE{BooleanAlgebra}}
\newcommand{\PD}{\TYPE{PairwiseDisjointElements}}
\newcommand{\PoU}{\TYPE{PartitionOfUnity}}
\renewcommand{\SS}{\TYPE{StoneSpace}}
\newcommand{\TK}{\mathcal{TK}}
\newcommand{\BL}{\TYPE{BooleanLattice}}
\newcommand{\Fix}{\mathrm{Fix}}
\newcommand{\OC}{\TYPE{OrderClosed}}
\newcommand{\SOC}{\TYPE{SequentiallyOrderClosed}}
\newcommand{\oC}{\TYPE{OrderContinuous}}
\newcommand{\sC}{\TYPE{\sigma\hyph Continuous}}
\newcommand{\OD}{\TYPE{OrderDense}}
\newcommand{\REing}{\TYPE{RegularEmbedding}}
\newcommand{\REed}{\TYPE{RegularEmbeded}}
\newcommand{\REable}{\TYPE{RegularEmbedable}}
\newcommand{\OComplete}{\TYPE{OrderDedekindComplete}}
\newcommand{\TAlgebra}{\TYPE{\tau\hyph Algebra}}
\newcommand{\OCompletes}{\TYPE{OrderDedekindCompleteSubset}}
\newcommand{\SComplete}{\TYPE{\sigma\hyph DedekindComplete}}
\newcommand{\SCompletes}{\TYPE{\sigma\hyph DedekindCompleteSubset}}
\newcommand{\LS}{\mathcal{LS}}
\newcommand{\POpen}{\TYPE{PseudoOpen}}
\newcommand{\od}{\mathbf{OD}}
\newcommand{\mgr}{\mathbf{MGR}}
\newcommand{\nd}{\mathbf{ND}}
\newcommand{\CCC}{\TYPE{WithCountableChainCondition}}
\newcommand{\CSI}{\TYPE{\omega_1\hyph SaturatedIdeal}}
\newcommand{\WD}{\TYPE{(\sigma,\infty)\hyph WeaklyDistributive}}
\newcommand{\Aless}{\TYPE{Atomless}}
\newcommand{\PA}{\TYPE{PurelyAtomic}}
\newcommand{\Homog}{\TYPE{Homogeneous}}
%\newcommand{\FS}{\TYPE{FullSubgroup}}
%\newcommand{\CFS}{\TYPE{CountablyFullSubgroup}}
%\newcommand{\EI}{\TYPE{ExchangingInvolution}}
%\newcommand{\SwS}{\TYPE{SubgroupWithSeparators}}
%\newcommand{\SwmI}{\TYPE{SubgroupWithManyInvolutions}}
%FUNC
\DeclareMathOperator{\upr}{upr}
\DeclareMathOperator{\Atom}{Atom}
%\DeclareMathOperator{\Supp}{Supp}
%\newcommand{\genFS}[1]{\left\langle #1 \right\rangle_\mathrm{F}}
%\newcommand{\genCFS}[1]{\left\langle #1 \right\rangle_\mathrm{CF}}
%\DeclareMathOperator{\Sep}{Sep}
%\DeclareMathOperator{\Tr}{Tr}
%CATS
\newcommand{\BOL}{\mathsf{BOL}}
\newcommand{\BOOL}{\mathsf{BOOL}}
%SYMBOL
\newcommand{\Z}{\mathsf{Z}}
%Topology
%General Topology
%Types
\newcommand{\Top}{\TYPE{Topology}}
\newcommand{\Homeo}{\TYPE{Homeomorphism}}
\newcommand{\TS}{\TYPE{TopologicalSpace}} 
\newcommand{\NbhdBase}{\TYPE{NeighborhoodBase}}
\newcommand{\LF}{\TYPE{LocallyFinite}}
\newcommand{\PN}{\TYPE{PerfectlyNormal}}
\newcommand{\CR}{\TYPE{CompletelyRegular}}
\newcommand{\OM}{\TYPE{OpenMap}}
\newcommand{\Filter}{\TYPE{Filter}}
\newcommand{\Filterbase}{\TYPE{Filterbase}}
\newcommand{\CFilterbase}{\TYPE{ConvergentFilterbase}}
\newcommand{\Dense}{\TYPE{Dense}}
\newcommand{\Separable}{\TYPE{Separable}}
\newcommand{\ND}{\TYPE{NowhereDense}}
\newcommand{\Open}{\TYPE{Open}}
\newcommand{\Net}{\TYPE{Net}}
\newcommand{\Closed}{\TYPE{Closed}}
\newcommand{\Clopen}{\TYPE{Clopen}}
\newcommand{\Nbhd}{\TYPE{Neighborhood}}
\newcommand{\Compact}{\TYPE{Compact}}
\newcommand{\Compacts}{\TYPE{CompactSubset}}
\newcommand{\OpenC}{\TYPE{OpenCover}}
\newcommand{\Cluster}{\TYPE{Cluster}}
\newcommand{\Convergent}{\TYPE{Convergent}}
%\newcommand{\LC}{\TYPE{LocallyCompact}}
\newcommand{\Locally}{\TYPE{Locally}}
\newcommand{\Bair}{\TYPE{BaireSpace}}
%\newcommand{\Meager}{\TYPE{Meager}}
\newcommand{\Connected}{\TYPE{Connected}}
\newcommand{\ED}{\TYPE{ExtemellyDisconnected}}
%FUNC
\DeclareMathOperator*{\intx}{int}
\DeclareMathOperator*{\cl}{cl} 
\DeclareMathOperator*{\boundary}{\partial} 
\DeclareMathOperator{\combo}{\triangledown} 
%\DeclareMathOperator{\diag}{\triangle} 
\DeclareMathOperator{\rem}{rem}
%CATS
\newcommand{\TOP}{\mathsf{TOP}}
\newcommand{\HC}{\mathsf{HC}}
\newcommand{\CG}{\mathsf{CG}}
%Symbols
\newcommand{\T}{\mathcal{T}}
\newcommand{\N}{\mathcal{N}}
\renewcommand{\U}{\mathcal{U}}
\renewcommand{\O}{\mathcal{O}}
\renewcommand{\d}{\mathrm{d}}
%\newcommand{\F}{\mathcal{F}}
\newcommand{\X}{\mathcal{X}}
%\newcommand{\d}{\mathrm{d}}
%Metric Topology
\newcommand{\Bounded}{\TYPE{Bounded}}
%FUNC
\DeclareMathOperator{\diam}{diam}
\newcommand{\Cell}{\mathbb{B}}
\newcommand{\Disc}{\mathbb{D}}
\newcommand{\Lip}[1]{#1\hyph\mathrm{Lip}}
%CATS
\newcommand{\Semiiso}{\mathsf{SMS}_{\circ \to \cdot}}
\newcommand{\Iso}{{\mathsf{MS}_{\circ \to \cdot}}}
\newcommand{\SMS}{\mathsf{SMS}}
\newcommand{\MS}{\mathsf{MS}}
\newcommand{\UNI}{\mathsf{UNI}}
\newcommand{\UNIS}{\mathsf{UNIS}}
\newcommand{\TG}{\mathsf{TG}}
\newcommand{\CSeq}{\TYPE{CauchySequence}}
\newcommand{\Complete}{\TYPE{Complete}}
%Descriptive Set Theory
%TYPE
%\newcommand{\Bool}{\mathbb{B}}
\newcommand{\IS}{\TYPE{InitialSegement}}
\newcommand{\FS}[1]{{#1}{}^*}
\newcommand{\Ext}{\TYPE{Extension}}
\newcommand{\Tree}{\TYPE{Tree}}
\newcommand{\Pruned}{\TYPE{Pruned}}
\newcommand{\PTM}{\TYPE{ProperTreeMorphism}}
\newcommand{\LTM}{\TYPE{LipschitzTreeMorphism}}
\newcommand{\Polish}{\TYPE{Polish}}
\newcommand{\IIPG}{\TYPE{InfiniteIterativeTwoPlayersGame}}
\newcommand{\FPS}{\TYPE{FirstPlayerStrategy}}
\newcommand{\SPS}{\TYPE{SecondPlayerStrategy}}
\newcommand{\FPWS}{\TYPE{FirstPlayerWinningStrategy}}
\newcommand{\SPWS}{\TYPE{SecondPlayerWinningStrategy}}
\newcommand{\CS}{\TYPE{ChoquetSpace}}
\newcommand{\SCS}{\TYPE{StrongChoquetSpace}}
\newcommand{\BP}{\mathbf{BP}}
\newcommand{\MGR}{\mathbf{MGR}}
\newcommand{\cat}{\mathbf{CAT}}
\newcommand{\BM}{\TYPE{BairMeasurable}}
\newcommand{\CGSA}{\TYPE{CountablyGeneratedSigmaAlgebra}}
\newcommand{\MC}{\TYPE{MonotonicClass}}
\newcommand{\PSA}{\TYPE{PointSeparatingAlgebra}}
\newcommand{\SBS}{\TYPE{StandardBorelSpace}}
\newcommand{\IH}{\TYPE{InducedHomomorphism}}
%FUNC
\DeclareMathOperator{\len}{len}
\newcommand{\inits}[2]{{#1}_{|\left[1,\ldots,#2\right]}}
\DeclareMathOperator{\lb}{lb}
\DeclareMathOperator{\WFpart}{WF}
\DeclareMathOperator{\Tr}{Tr}
\DeclareMathOperator{\PTr}{PTr}
\DeclareMathOperator*{\Tll}{{T\;\underline{lim}}}
\DeclareMathOperator*{\Tul}{{T\;\overline{lim}}}
\DeclareMathOperator*{\Tl}{{T\;lim}}
\DeclareMathOperator{\rankcb}{rank_{CB}}
\DeclareMathOperator{\lp}{lp}
\newcommand{\alg}{\mathsf{A}}
%CATS
\newcommand{\TREE}{\mathsf{TREE}}
\newcommand{\FSF}{\mathsf{FS}}
\newcommand{\CRONE}{\mathsf{CRONE}}
\newcommand{\BODY}{\mathsf{BODY}}
\newcommand{\BOR}{\mathsf{BOR}}
\newcommand{\bor}{\mathsf{B}}
\newcommand{\Effros}{\mathsf{EFF}}
%symbols
\newcommand{\K}{\mathsf{K}}
\renewcommand{\H}{\mathrm{H}}
\renewcommand{\L}{\mathcal{L}}
\renewcommand{\P}{\mathcal{P}}
\renewcommand{\S}{\mathcal{S}}
%LINEAR
%Linear Algebra
%Types
\newcommand{\Basis}{\TYPE{Basis}} % Basis of the linear space
\newcommand{\submod}[1]{\subset_{\LMOD{#1}}}% submodule as a subset
\newcommand{\subvec}[1]{\subset_{\VS{#1}}}% vector subspace as a subset
\newcommand{\FGM}{\TYPE{FinitelyGeneratedModule}}% Finitely generated module
\newcommand{\LI}{\TYPE{LinearlyIndependent}}
\newcommand{\LIS}{\TYPE{LinearlyIndependentSet}}
\newcommand{\FM}{\TYPE{FreeModule}}
\newcommand{\IBP}{\TYPE{InvariantBasisProperty}}
\newcommand{\UTM}{\TYPE{UpperTriangularMatrix}}
%\newcommand{\LTM}{\TYPE{LowerTriangularMatrix}}
\newcommand{\Diag}{\TYPE{DiagonalMatrix}}
\newcommand{\FP }{\TYPE{FinitelyPresented}}
\newcommand{\GL}{\mathbf{GL}}% General Linear Group
\newcommand{\SL}{\mathbf{SL}}% Special Linear group
\newcommand{\SO}{\mathbf{SO}}
\newcommand{\SU}{\mathbf{SU}}
\newcommand{\prsubvec}[1]{\subsetneq_{\VS{#1}}}	% poper vector subspace as a subset
\newcommand{\LC}{\TYPE{LinearComplement}} 
%\newcommand{\IS}{\TYPE{InvariantSubspace}}
\newcommand{\RP}{\TYPE{ReducingPair}}
\newcommand{\RCF}{\TYPE{RationalCanonicalForm}}
\newcommand{\JCF}{\TYPE{JordanCanonicalForm}}
\newcommand{\Diagble}{\TYPE{Diagonalizable}}
\newcommand{\UT}{\TYPE{UpperTriangulizable}}
\newcommand{\LT}{\TYPE{LowerTriangulizable}}
\newcommand{\IPS}{\TYPE{InnerProductSpace}}
\newcommand{\OBasis}{\TYPE{OrthonormalBasis}}
\newcommand{\FDIPS}{\TYPE{FiniteDimensionalInnerProductSpace}}
\newcommand{\NO}{\TYPE{NormalOperator}}
\newcommand{\NM}{\TYPE{NormalMatrix}}
%\newcommand{\SA}{\TYPE{SelfAdjoint}}
\newcommand{\SSA}{\TYPE{SkewSelfAdjoint}}
\newcommand{\PI}{\TYPE{Pseudoinverse}}
\newcommand{\OVS}{\TYPE{OrthogonalVectorSpace}}
\newcommand{\SVS}{\TYPE{SymplecticVectorSpace}}
\newcommand{\MVS}{\TYPE{MetricVectorSpace}}
\newcommand{\FDMVS}{\TYPE{FiniteDimensionalMetricVectorSpace}}
\newcommand{\Sp}{\mathbf{Sp}}
%Func
\DeclareMathOperator{\Span}{span} % spann by subset
\DeclareMathOperator{\Ann}{Ann}   % annihilator
\DeclareMathOperator{\Ass}{Ass}   % associated primes:
\DeclareMathOperator{\adj}{adj}   % an adjoint matrix
\DeclareMathOperator{\tr}{tr}     % trace
\DeclareMathOperator{\codim}{codim} % codimension
%\DeclareMathOperator{\Cell}{\mathbf{C}} % a componion matrix
\DeclareMathOperator{\JC}{\mathbf{J}}  % a Jordan cell
\DeclareMathOperator{\bigboxplus}{\scalerel*{\boxplus}{\sum}} % a direct sum of operators in the sence of the reducing a pair
\DeclareMathOperator{\Spec}{Spec} % Spectre
\DeclareMathOperator{\bigbot}{\scalerel*{\bot}{\sum}} % an othogonal direct sum
\DeclareMathOperator{\GS}{\mathbf{GS}} %Gramm-Smmidt process
\DeclareMathOperator{\NGS}{\mathbf{NGS}} %Normalized Gramm-Smmidt process
\DeclareMathOperator{\WI}{\mathrm{WI}} %Witt Index
%Cats
\newcommand{\VS}[1]{#1\hyph\mathsf{VS}} % a category of vector spaces (Field)
\newcommand{\FDVS}[1]{#1\hyph\mathsf{FDVS}} % a category of finite-dimensional vector spaces (Field)
\newcommand{\LALGE}[1]{#1\hyph\mathsf{ALGE}}
\newcommand{\LMOD}[1]{#1\hyph\mathsf{MOD}} % a category of the left modules (Ring)
\newcommand{\RMOD}[1]{\mathsf{MOD}\hyph#1} % a category of the right modules (Ring)
\newcommand{\LLMAP}[1]{#1\hyph\mathsf{LMAP}} % a cagory of based linear maps with the left scalar multiplication (Ring)
\newcommand{\LMAT}[1]{#1\hyph\mathsf{MAT}}  % a category of based matrices with the left scalar multiplication (Ring)
\newcommand{\NMAT}[1]{#1\hyph\mathbb{N}} % a category of finite matrices (Field)
%Symbols
\renewcommand{\L}{\mathcal{L}}
%\renewcommand{\O}{\mathbf{O}}
%\renewcommand{\S}{\mathbf{S}}
%%Measure theorty
%Types
\newcommand{\Measure}{\TYPE{Measure}}
%\newcommand{\MS}{\TYPE{MeasureSpace}}
\newcommand{\CMS}{\TYPE{CompleteMeasureSpace}}
\newcommand{\Null}{\mathcal{N}}
\renewcommand{\ae}{\mathrm{a.e.}}
\renewcommand{\OM}{\TYPE{OuterMeasure}}
\newcommand{\IM}{\TYPE{InnerMeasure}}
\newcommand{\Thick}{\TYPE{Thick}}
\newcommand{\Integrable}{\mathsf{I}}
\newcommand{\ME}{\TYPE{MeasurableEnvelope}}
\newcommand{\Probability}{\TYPE{Probability}}
\newcommand{\sFinite}{\TYPE{\sigma \hyph  Finite}}
\newcommand{\Semifinite}{\TYPE{Semifinite}}
\newcommand{\Decomposition}{\TYPE{Decomposition}}
\newcommand{\SLoc}{\TYPE{StrictlyLocalizable}}
\newcommand{\Loc}{\TYPE{Localizable}}
\newcommand{\LocDet}{\TYPE{LocallyDetermined}}
\newcommand{\PtSupp}{\TYPE{PointSupported}}
\newcommand{\SF}[1]{\TYPE{\sigma \hyph  Finite}\left( #1 \right) }
\newcommand{\DRP}{\TYPE{DiscreteRandomProcces}}
\newcommand{\MwLDNS}{\TYPE{MeasureWithLocallyDeterminedNullSets}}
\newcommand{\AF}{\TYPE{AdditiveFunctional}} 
\newcommand{\CAF}{\TYPE{CountablyAdditiveFunctional}}
\newcommand{\TC}{\TYPE{TrulyContinuous}} 
\newcommand{\CE}{\TYPE{ConditionalExpectation}}
%Functions and Operators
\DeclareMathOperator{\esssup}{ess\sup}
%Symbols
\newcommand{\F}{\mathcal{F}}
\newcommand{\E}{\mathcal{E}}
%CATS
\newcommand{\MEAS}{\mathsf{MEAS}}
\newcommand{\Simple}{\mathsf{S}}
\newcommand{\caf}{\mathsf{ca}}
\newcommand{\af}{\mathsf{a}}
\newcommand{\baf}{\mathsf{ba}}
\newcommand{\ac}{\mathsf{ac}}
\newcommand{\tc}{\mathsf{tc}}
%%Measure Algebra
%Types
\newcommand{\MA}{\TYPE{MeasureAlgebra}}
\newcommand{\SI}{\TYPE{SummableIncrements}}
\newcommand{\MPH}{\TYPE{MeasurePreservingHomomorphism}}
\newcommand{\Lp}[1]{\mathbf{L}^{#1}}
\newcommand{\SInd}{\TYPE{StochasticalyIndependent}}
\newcommand{\PAlg}{\TYPE{ProbabilityAlgebra}}
\newcommand{\PAless}{\TYPE{ProperlyAtomless}}
%Categroy
\newcommand{\ma}{\mathsf{MA}}
\newcommand{\pa}{\mathsf{PA}}
\newcommand{\Caf}{\tau\hyph\mathsf{ca}}
\author{Uncultured Tramp} 
\title{Algebraic Measure Theory}
\begin{document}
\maketitle
\thispagestyle{empty}
\newpage
\thispagestyle{empty}
\tableofcontents
\newpage
\section*{Intro}
\newpage
\pagenumbering{arabic}
\section{Measure Algebras}
\subsection{Subject}
\subsubsection{Definition and Basic Property}
\Page{
	\DeclareType{MeasureAlgebra}
	{
		? \sum A : \SComplete \. A \to \EReals_+
	}
	\DefineType{(A,\mu)}{MeasureAlgebra}
	{
			\forall a \in A \. \mu(a) = 0 \iff a = 0
			\And \NewLine \And
			\forall a : \PD(\Nat,A) \. 
			\mu\left(\bigvee^\infty_{n=1}  a_n \right) = \sum^\infty_{n=1} \mu(a_n)
	}
	\\
	\DeclareFunc{measureAlgebraCategory}{\CAT}
	\DefineNamedFunc{measureAlgebraCategory}{}{\ma}
	{
		\Big( \MA,  \BOOL, \circ, \id \Big)	
	}
	\\
	\Theorem{MeasureMonotonicity}
	{
		\forall (A,\mu) : \MA \. 
		\forall a,b \in A \. 
		a \le b \Imply  \mu(a) \le \mu(b)
	}
	\Explain{ Write $\mu(b) = \mu(a) + \mu(b\setminus a) \ge \mu(a)$}
	\EndProof
	\\
	\Theorem{MeasureStrictMonotonicity}
	{
		\forall (A,\mu) : \MA \. 
		\forall a,b \in A \. 
		a > b \Imply  \mu(a) > \mu(b)
	}
	\Explain{ Definition of measure algebra implies that $\mu(b \setminus a) > 0$ }
	\Explain{ Write $\mu(b) = \mu(a) + \mu(b\setminus a) > \mu(a)$}
	\EndProof
	\\
	\Theorem{MinkovskyIneq}
	{
		\forall (A,\mu) : \MA \.
		\forall a,b \in A \. 
		\mu(a \vee b) \le \mu(a) + \mu(b)
	}
	\Explain{
		Write
		$
			\mu(a) + \mu(b) =
		    \mu(a \setminus ab) + \mu(ab) + \mu(b \setminus ab) + \mu(ab) 
		    \ge   mu(a \setminus ab) + \mu(ab) + \mu(b \setminus ab = 
		    \mu(a \vee b)
		$
	}
	\EndProof
	\\
	\Theorem{MonotonicSupremumAsLimit}
	{
		\forall (A,\mu) : \MA \.
		\forall a : \Nat \uparrow A \. 
		\mu\left( \bigvee^\infty_{n=1} a_n\right)  = \lim_{n \to \infty} \mu(a_n)
	}
	\Explain{ Construct disjoint sequence $b_n = a_n \setminus \bigvee^{n-1}_{k=1} a_k$}
	\Explain{ Then by construction 
			$
				\mu\left( \bigvee^\infty_{n=1} a_n \right) = 
				\mu\left(  \bigvee^\infty_{n=1} b_n \right) =
				\sum^\infty_{n=1} \mu(b_n) = 
				\lim_{n \to \infty} \sum^n_{k=1} \mu(b_n) =
				\lim_{n \to \infty} \mu\left(\bigvee^n_{k=1} b_k\right) =
				\lim_{n \to \infty} \mu(a_n)
			$}
	\EndProof
}\Page{
	\Theorem{SupremumIneq}
	{
		\forall (A,\mu) : \MA \.
		\forall a : \Nat \to A \. 
		\mu\left( \bigvee^\infty_{n=1} a_n\right) \le \sum^\infty_{n=1} \mu(a_n)
	}
	\Explain{ Construct increasing sequence $b_n = \bigvee^n_{k=1} a_n$}
	\Explain{
		Then by construction
		$
			\mu\left( \bigvee^\infty_{n=1} a_n \right) = 
			\mu\left(  \bigvee^\infty_{n=1} b_n \right) =
			\lim_{n \to \infty} \mu( b_n ) =
			\lim_{n \to \infty} \mu\left( \bigvee^n_{k=1} a_k \right)  \le 
			\lim_{n \to \infty} \sum^n_{k=1} \mu(a_k) =
			\sum^\infty_{n=1} \mu(a_n)  
		$
	}
	\EndProof
	\\
	\Theorem{MonotonicInfimumAsLimit}
	{
		\NewLine ::		
		\forall (A,\mu) : \MA \.
		\forall a : \Nat \downarrow A \.
		\forall \aleph : \inf_{n \in \Nat} \mu(a_n) < \infty \.
		\mu\left( \bigwedge^\infty_{n=1} a_n\right)  = \lim_{n \to \infty} \mu(a_n)
	}
	\Explain{ Without loss of generality assume that $\mu(a_1) < \infty$}
	\Explain{ Then construc the increasing sequence $b_n = a_1 \setminus a_n$} 
	\ExplainFurther{ Then $
			\mu(a_1) - \mu\left(\bigwedge^\infty_{n=1} a_n  \right) =
			\mu\left( a_1 \setminus \bigwedge^\infty_{n=1} a_n \right)
			=\mu\left( \bigvee^\infty_{n=1} a_1 \setminus a_n\right) 
			=\mu\left( \bigvee^\infty_{n=1} b_n\right)  = \lim_{n \to \infty} \mu(b_n) =$}
	\Explain{$  =
			\lim_{n \to \infty} \mu\left( a_1 \setminus a_n \right)=\lim_{n \to \infty} \mu(a_1) - \mu(a_n) = \mu(a_1) - \lim_{n \to \infty} \mu(a_n)$}
	\Explain{
		So basic algebraic manipulations
		$
			\mu\left(\bigwedge^\infty_{n=1} a_n  \right) = \lim_{n \to \infty} \mu(a_n)
		$		
	}
	\EndProof
	\\
	\Theorem{SupremumExistance}
	{
		\NewLine ::		
		\forall (A,\mu) : \MA \.
		\forall C : \TYPE{UpwardsDirected}(A) \.
		\forall \aleph : \sup_{c \in C} \mu(c) < \infty \.
		\exists a \in A : a = \sup C
	}
	\Explain{1 Assume $\gamma = \sup_{c \in C} \mu(c)$}
	\Explain{2 Then there exists a sequrnce of $a:\Nat \to C$ such that $\mu(a_n )\ge \gamma - 2^{-n}$}
	\Explain{3 As $C$ is upwards closed, it is possible to find $c:\Nat \to C$ 
		such that $c_{n+1} \ge a_{n} \vee c_n$}
	\Explain{4 Then $c$ is monotonic-nondecreasing and so it has 
		$\mu\left(\bigvee^\infty_{n=1} c_n\right) = \lim_{n \to \infty} \mu(c_n) = \gamma$}
	\Explain{4.1 Note that $\gamma \ge \mu(c_n) \ge \gamma - 2^{-n}$}
	\Explain{5 let $d = \bigvee^\infty_{n=1} c_n$}
	\Explain{6 $d \ge f$ for everty $f \in C$}
	\Explain{6.1 Assume this is false}
	\Explain{6.2 Then $f \setminus d \neq 0$ and so $\alpha = \mu(f\setminus d) > 0$}
	\Explain{6.3 Then there exists $n$ such that $\gamma - \mu(c_n) < \alpha$}
	\Explain{6.4 As $C$ is upwards derected there is $g \in C$
		such that $g \ge f \vee c_n$}
	\Explain{6.5 But
		$\mu(g) \ge \mu(f \vee c_n) = \mu(c_n) + \mu(f \setminus c_n) \ge 
			\mu(c_n) + \mu(f \setminus d) > \gamma$ which is impossible}                                                     
	\Explain{7 If there is another $f$ with the property (6), then 
		$d = \bigvee^\infty_{n=1} c_n \le f$ as $c_n \le f$ for each $n\in\Nat$}
	\EndProof
}\Page{
	\Theorem{UpperContinuity}
	{
		\NewLine ::		
		\forall (A,\mu) : \MA \.
		\forall C : \TYPE{UpwardsDirected}(A) \.
		\forall \aleph : \exists a \in A : a = \sup C \.
		\sup_{c \in C} \mu(c) = \mu\left( \sup C \right)
	}
	\Explain{ Case $\sup_{c \in C} \mu(c) = \infty$ is trivial}
	\Explain{ Finite case follows from the cconstruction in the previous theorem}
	\EndProof
	\\
	\Theorem{DisjointUpperContinuity}
	{
		\NewLine ::		
		\forall (A,\mu) : \MA \.
		\forall C : \PD(A) \.
		\forall \aleph : \exists a \in A : a = \sup C \. \NewLine \.
		\mu\left( \sup C \right)= \sum_{c \in C} \mu(c)
	}
	\Explain{Construct a new set $D = \left\{ \bigvee^\infty_{n=1} c_k \bigg|  c : \Nat \to C  \right\}$}
	\Explain{ Then $D$ is upwards directed and $\sup C = \sup D$}
	\Explain{ But this is evedent that 
		$\mu\left(\sup D\right) = \sup_{d \in D} \mu(d) = 
		\sup_{c : \Nat \to C} \mu\left( \bigvee_{n=1} c_n \right)  =
		\sup_{n \in \Nat, c : \{1,\ldots,n\} \to C} \sum^n_{k=1} \mu(c_k)  =
		\sum_{c \in C} \mu(c)$}
	\EndProof
	\\
	\Theorem{InfimumExistance}
	{
		\NewLine ::		
		\forall (A,\mu) : \MA \.
		\forall C : \TYPE{DownwaedDirected}(A) \.
		\forall \aleph : \inf_{c \in C} \mu(c) < \infty \.
		\exists a \in A : a = \inf C
	}
	\Explain{ 1 There exists some $a \in C$ such that $\mu(a) < \infty$}
	\Explain{ 2 Construct another set $D = a \setminus C$}
	\Explain{ 3 Then $D$ is upwards directed and $\sup_{d \in D} \mu(d) \le \mu(a) < \infty$}
	\Explain{ 4 So there is $d = \sup d$ }
	\Explain{ 5 Define $f = a \setminus d$ }
	\Explain{ 6 $f \le c$ for any $c \in C$ as $a \setminus f \ge a \setminus c$}
	\Explain{ 7 if some $g$ has property (6) then $a \setminus g \ge d$ and so $g \le f$}
	\EndProof
	\\
	\Theorem{LowerContinuity}
	{
		\NewLine ::		
		\forall (A,\mu) : \MA \.
		\forall C : \TYPE{DownwardsDirected}(A) \.
		\forall \aleph : \exists a \in A : a = \inf C \. \NewLine \.
		\forall \beth :  \inf_{c \in C} \mu(c) < \infty \.
		\inf_{c \in C} \mu(c) = \mu\left( \inf C \right)
	}
	\Explain{ Use the construction in the previous theorem}
	\EndProof
}
\newpage
\subsubsection{Measure Algebras Generated by Measure Spaces}
\Page{
	\DeclareFunc{measureAlgebra}
	{
		\MEAS \to \MA
	}
	\DefineNamedFunc{measureAlgebra}{X,\Sigma,\mu}{(A_\mu,\bar \mu)}
	{\left( \frac{\Sigma}{\Sigma \cap \Null_\mu},[E] \mapsto \mu(E)\right)}
	\Explain{ This is obviously well defined as $[E] = [F]$ iff $\mu(E \du F) = 0$}
	\\
	\DeclareFunc{canononicalProjection}
	{
		\forall (X,\Sigma,\mu) \in \MEAS \. 
		\sigma\hyph\BOOL(\Sigma,A_\mu) 
	}
	\DefineNamedFunc{canonicalProjection}{E}{\pi_\mu(E)}{[E]}
	\Explain{ 1 The algebraic properites are obvious as $\Sigma \cap \Null_\mu$ is an ideal}
	\Explain{ 2 In order to prove sigma-continuity assume $E : \Nat \to \Sigma$}
	\Explain{ 2.1 Let $Z : \Nat \to \Sigma \cap \Null_\mu$}
	\Explain{ 2.2 Then $F_Z = \bigvee^\infty_{n=1} (E_n \du Z_n) = 
		\left(\bigvee^\infty_{n=1} E_n\right) \du \left( \bigvee^\infty_{n=1} Z_n\right)$}
	\Explain{ 2.3 Note that 
	$\mu\left( \bigvee^\infty_{n=1} Z_n\right) \le \sum^\infty_{n=1} \mu(Z_n)=0$}
	\Explain{ 2.4 So $ \bigvee^\infty_{n=1} Z_n \in \Sigma \cap \Null_\mu$ as $\mu \ge 0$}
	\Explain{ 2.5 Thus $[F_Z] = \left[\bigcap^\infty_{n=1} E_n\right]$ for any selection of $Z$}
	\Explain{ 2.6 This means that 
			$\pi_\mu\left(\bigcap^\infty_{n=1} E_n\right) = \bigvee^\infty_{n=1} \pi_\mu(E_n)$}
	\EndProof
	\\
	\Theorem{MeasureAlgebraMonotonicity}
	{
		\forall (X,\Sigma,\mu) \in \MEAS \.
		\forall T \subset_\sigma \Sigma \.
		\pi_\mu(T) \subset_\sigma A_\mu
	}
	\Explain{ 1 Clearly $B = \pi_\mu(T) \subset A_\mu$}
	\Explain{ 2 Also as $T$ is $\sigma$-algebra and $\pi-\mu$ is a $\sigma$-continuous homomorphism 
		$B$ is again}
	\EndProof
	\\
	\Theorem{MeasureAlgebraInverseMonotonicity}
	{
		\forall (X,\Sigma,\mu) \in \MEAS \.
		\forall B \subset_\sigma A_\mu \.
		\pi_\mu^{-1}(B) \subset_\sigma \Sigma
	}
	\Explain{ 1 Clearly $T = \pi_\mu^{-1}(B) \subset \Sigma$}
	\Explain{ 2 Assume $F$ is a set constructed by applying $\sigma$-algebra operations to setes $E_1,E_2,\ldots \in T$ }
	\Explain{ 3 Then $\pi_\mu(F)$ can be constructed by applying same operations to $\pi(E_1),\pi(E_2),\ldots$}
	\Explain{ 4 This implies that $\pi_\mu(F) \in B$ and reciprorary $F \in T$}
	\Explain{ 5 Thus $T$ is a $\sigma$-algebra}
	\EndProof
}
\newpage
\subsubsection{Stone Representation Theorem}
\Page{
	\Theorem{StoneRepresentationTheorem}
	{
		\forall (A,\mu) : \MA \. \exists (X,\Sigma,\nu) \in \MEAS \. (A,\mu) = (A_\nu,\bar \nu)
	}
	\ExplainFurther{
		1 By Loomis-Sikorski representation there exists a set $X$ with a sigma-algebra $\Sigma$ and}
	\Explain{ 	sigma-ideal $I$ such that $\frac{\Sigma}{I} \cong_\BOOL  A$
	}
	\Explain{ 2
		Then there is a canonical projetion $\pi_I \in \BOOL(\Sigma,A)$	
	}
	\Explain{ 3 Define $\nu = \pi_I \mu$}
	\Explain{ 4 $\nu$ is measure on $\Sigma$}
	\Explain{ 4.1 $\nu(\emptyset) = \mu(0) = 0$}
	\Explain{ 4.2 Assume $E$ is a disjoint sequence in $\Sigma$}
	\Explain{ 4.3 Then $\pi_I(E_n)\pi_I(E_m) = \pi_i(E_n \cap E_m) = \pi_i(\emptyset) =  0$,
	 so $\pi_I(E)$ is disjoint in $A$}
	\Explain{ 4.4 Thus, 
		$\nu\left( \bigcup^\infty_{n=1} E_n \right) = 
		\pi_I\mu\left( \bigcup^\infty_{n=1} E_n \right) =  
		\mu\left( \bigvee^\infty_{n=1} \pi_I(E_n)\right) =
		\sum^\infty_{n=1} \pi_I\mu(E_n) = 
		\sum^\infty_{n=1} \nu(E_n) $ }
	\Explain{ 5 Also by consytuction $ \Null_\nu \cap \Sigma = I$, so $ (A,\mu) = (A_\nu,\bar \nu)$} 
	\EndProof
	\\
	\DeclareFunc{spaceOfStone}{\MA \to \MEAS}
	\DefineNamedFunc{SpaceOfStone}{A,\mu}{(Z_A,\dot \Sigma_\mu,\dot \mu)}
	{
			\THM{StoneRepresentationTheorem}(A,\mu)	
	}
}
\newpage
\subsubsection{Ideals}
\Page{
	\Theorem{PrincipleIdealRestriction}
	{
		\forall (A,\mu) : \MA \.
		\forall a \in A \.
		\MA\Big( (a), \mu_{|(a)} \Big) 
	}
	\Explain{This is obvious}
	\EndProof
	\\
	\Theorem{measureQuotient}
	{
		\NewLine ::		
		\forall (A,\mu) : \MA \.
		\forall I : \Ideal(A) \.
		\forall [a] \in \frac{A}{I} \.
		\exists \gamma \in \EReals_{++} \.
		\gamma = \min \{ \mu(b) | b \in A, \pi_I(b) = [a] \}	
	}
	\Explain{ 1 $\gamma = \inf \{ \mu(b) | b \in A, \pi_I(b) = [a] \}$ exists as a set
	 is bounded by below by $0$}
	\Explain{ 2 If $\gamma = \infty$ then the result is obvious}
	\Explain{ 3 Otherwise there is a decreasing sequence $b : \Nat \to A$ 
		such that $\pi_I(b_n) = [a]$ for any $n$ and 
		$\lim_{n \to \infty} \mu(b_n) = \gamma$}
	\Explain{ 4 Then  $c = \bigwedge^\infty_{n=1} b_n$
		is such that $\mu(c) = \gamma$ and $\pi_I(c) = a$}
	\Explain{ 4.1 Clearly 
		$
		\pi_I\left(\bigwedge^\infty_{n=1} b_n \right) = 
		\bigwedge^\infty_{n=1} \pi_I(b_n) =
		\bigwedge^\infty_{n=1} [a] = [a]		
		$}
	\Explain{ 5 So the infimum is atteined}
	\EndProof
	\\
	\DeclareFunc{measureQuotient}
	{  \prod (A,\mu) : \MA \. 
		\prod I : \Ideal(A) \.
		\frac{A}{I} \to \Reals_{++}
	}
	\DefineNamedFunc{measureQuotient}
	{a}{\mu_{I}(a)}
	{
		\min \{ \mu(b) | b \in A, \pi_I(b) = a \}	
	}
	\\
	\DeclareFunc{finiteElementsIdeal}{\prod (A,\mu) : \MA \. \Ideal(A)}
	\DefineNamedFunc{finiteElementsIdeal}{}{A^f}{\{a \in A | \mu(a) < \infty \}}
}\Page{
	\Theorem{MeasureIdealQuotient}
	{
		\forall (A,\mu) : \MA \.
		\forall I : \Ideal(A) \.		
		\MA\left( \frac{A}{I}, \mu_{I} \right) 
	}
	\Explain{ 1 Clearly $\mu_I(0) = 0$}
	\Explain{ 2 Assume that $[a] \neq 0$}
	\Explain{ 2.1 Then there exists $b \in A$ such that $\pi_I(a) = [a]$ and $\mu(b) = \mu_I[a]$ }
	\Explain{ 2.2 As $[a] \neq 0$, then $b \neq 0$, and henceforth $\mu(b) \neq 0$}
	\Explain{ 2.3 Thus, $\mu_I[a] \neq 0$}
	\Explain{ 3 Assume $[a] : \Nat \to \frac{A}{I}$ is disjoint}
	\Explain{ 3.1 It is possible to select representatives $b_n$ for each $[a_n]$ 
		such that $\mu(b_n) = \mu_I[a_n]$}
	\Explain{ 3.2 Then $b_n b_m \in I$ if $n \neq m$}	
	\Explain{ 3.3 Construct a new sequence 
		$c_n = b_n + \sum^{n-1}_{k=1} b_n b_k$ is a disjoint represintative sequance for $[a_n]$}
	\Explain{ 3.3.1 In fact $c = b$}
	\Explain{ 3.4 $\bigvee^\infty_{n=1} c_n$ is the minimal representative 
		of $\bigvee^\infty_{n=1} [a_n]$ }
	\Explain{ 3.4.1  Assume $d$ is a representative for $\bigvee^\infty_{n=1} a_n$}
	\Explain{ 3.4.2 If $\mu(d) < \mu\left(\bigvee^\infty_{n=1} c_n\right)$ 
		then we may construct $c_n \wedge d$ which is smaller then $c$}
	\Explain{ 3.4.3  But this is a contradiction}
	\Explain{ 3.5 So 
		$\mu_I\left(\bigvee^\infty_{n=1} [a_n]\right) =  
		\mu\left( \bigvee^\infty_{n=1} c_n \right) = 
		\sum^\infty_{n=1} \mu(c_n) =
		\sum^\infty_{n=1} \mu_I[a_n]
		$}	
	\EndProof
}
\newpage
\subsubsection{Measure Properties}
\Page{
	\DeclareType{ProbabilityAlgebra}{?\MA}
	\DefineType{(A,\pi)}{ProbabilityAlgebra}{\pi(e) = 1}
	\\
	\DeclareType{Finite\MA}{?\MA}
	\DefineType{(A,\mu)}{Finite\MA}{\mu(e) < \infty}
	\\
	\DeclareType{\sFinite\MA}{?\MA}
	\DefineType{(A,\mu)}{\sFinite\MA}
	{
		\exists a : \Nat \to A \. \forall n \in \Nat \. \mu(a_n) < \infty \And \bigvee^\infty_{n=1} a_n = e
	}
	\\
	\DeclareType{\Semifinite\MA}{?\MA}
	\DefineType{(A,\mu)}{\Semifinite\MA}
	{
		\forall a \in A \. \mu(a) = \infty \Imply \exists b \in A \. b  < a \And 0 < \mu(b) < \infty
	}
	\\
	\Conclude{\Loc\MA}{ \OComplete \And \Semifinite\MA }{\Type}
	\\
	\Theorem{ProbabilityConstruction}
	{
		\forall (X,\Sigma,\mu) \in \MEAS \.
		\TYPE{Probability}(X,\Sigma,\mu)
		\iff 	
		\TYPE{ProbabilityAlgebra}(A_\mu,\bar \mu)
	}
	\Explain{ This is obvious}
	\EndProof
	\\
	\Theorem{FiniteConstruction}
	{
		\forall (X,\Sigma,\mu) \in \MEAS \.
		\Finite(X,\Sigma,\mu)
		\iff 	
		\Finite\MA(A_\mu,\bar \mu)
	}
	\Explain{ This is obvious}
	\EndProof
	\\
	\Theorem{SigmaFiniteConstruction}
	{
		\forall (X,\Sigma,\mu) \in \MEAS \.
		\sFinite(X,\Sigma,\mu)
		\iff 	
		\sFinite\MA(A_\mu,\bar \mu)
	}
	\Explain{ This is obvious}
	\EndProof
	\\
	\Theorem{SemifiniteConstruction}
	{
		\NewLine ::
		\forall (X,\Sigma,\mu) \in \MEAS \.
		\Semifinite(X,\Sigma,\mu)
		\iff 	
		\Semifinite\MA(A_\mu,\bar \mu)
	}
	\Explain{ This is obvious}
	\EndProof
	\\
	\Theorem{LocalizableConstruction}
	{
		\NewLine ::		
		\forall (X,\Sigma,\mu) \in \MEAS \.
		\Loc(X,\Sigma,\mu)
		\iff 	
		\Loc\MA(A_\mu,\bar \mu)
	}
	\Explain{ This is obvious}
	\EndProof
}\Page{
	\Theorem{AtomInConstruction}
	{
		\NewLine ::		
		\forall (X,\Sigma,\mu) \in \MEAS \.
		\forall  E \in \Sigma \.
		E \in \Atom(X,\Sigma,\mu)
		\iff
		[E] \in \Atom(A_\mu,\bar\mu)
	}
	\Explain{ This is obvious}
	\EndProof
	\\
	\Theorem{AtomlessConstruction}
	{
		\NewLine ::		
		\forall (X,\Sigma,\mu) \in \MEAS \.
		\forall  E \in \Sigma \.
		E \in \Aless(X,\Sigma,\mu)
		\iff
		[E] \in \Aless(A_\mu,\bar\mu)
	}
	\Explain{ This is obvious}
	\EndProof
	\\
	\Theorem{PurelyAtomicConstruction}
	{
		\NewLine ::		
		\forall (X,\Sigma,\mu) \in \MEAS \.
		\forall  E \in \Sigma \.
		E \in \PA(X,\Sigma,\mu)
		\iff
		[E] \in \PA(A_\mu,\bar\mu)
	}
	\Explain{ This is obvious}
	\EndProof
	\\
	\Theorem{FinitenessPropertiesIerarchy}
	{
		\NewLine  :: 
		\forall (A,\mu) : \MA \.
		\TYPE{PobabilityAlgebra}(A,\mu)
		\Imply
		\Finite\MA(A,\mu)
		\Imply \NewLine \Imply
		\sFinite\MA(A,\mu)
		\Imply
		\Loc\MA(A,\mu)
		\Imply
		\Semifinite(A,\mu)				
	}
	\Explain{1 Most implications here are obvious 
		expect the one deriving Localizability from $\sigma$-finiteness}
	\Explain{2 So assume that $(A,\mu)$ is $\sigma$-finite } 
	\Explain{2.1 Then the corresponding Stone space $(\Z A, \Sigma_\mu, \bar \mu )$ is $\sigma$-finite}
	\Explain{2.2  But then $(\Z A, \Sigma_\mu, \bar \mu )$ is localizable }
	\Explain{2.3 So $(A,\mu)$ is also localizable}
	\EndProof
	\\
	\Theorem{MeasureAlgebraOfCompletion}
	{
		\forall (X,\Sigma, \mu) \in \MEAS \. 
		A_\mu \cong_\BOOL A_{\hat \mu}
	}
	\Explain{This is basically follows from definitions}
	\EndProof
	\\
	\Theorem{MeasureAlgebraOfLocallyDeterminedCompletion}
	{
		\NewLine ::		
		\forall (X,\Sigma, \mu) \in \MEAS \. 
		\exists  A_\mu \Arrow{\phi} A_{\bar \mu} : \BOOL \.
		\forall a \in A_{\bar \mu} \. \hat {\bar \mu}(a) < \infty \Imply \exists b \in A_{\mu} \. \phi(b) = a
		\And \NewLine \And
 		\forall b \in A_{\mu} \. \hat \mu(b) < \infty \Imply  \hat {\bar \mu}(\phi(b)) = \hat \mu (b)
	}
	\NoProof
	\\
	\DeclareFunc{localDeterminationMorphism}
	{
		\prod (X,\Sigma, \mu) \in \MEAS \. 
		 \BOOL(A_\mu, A_{\bar \mu})
	}
	\DefineNamedFunc{localDeterminationMorphism}{}{\phi_\mu}
	{\THM{MeasureAlgebraOfLocallyDeterminedCompletion}}
}
\Page{
	\Theorem{localDeterminationMorhismInjectivity}
	{
		\NewLine ::		
		\forall (X,\Sigma,\mu) \in \MEAS \.
		\Semifinite(X,\Sigma,\mu)
		\iff
		\Inj(A_\mu,A_{\bar \mu},\phi_\mu)
	}
	\NoProof
	\\
	\Theorem{localDeterminationMorhismBijectivity}
	{
		\NewLine ::		
		\forall (X,\Sigma,\mu) \in \MEAS \.
		\Loc(X,\Sigma,\mu)
		\iff
		\Bij(A_\mu,A_{\bar \mu},\phi_\mu)
	}
	\NoProof
	\\
	\Theorem{SemifinitenessCriterion}
	{
		\forall (A,\mu) : \MA \. \NewLine \.
		\Semifinite\MA(A,\mu)
		\iff
		\exists P : \PoU(A) \. \forall p \in P \. \mu(p) < \infty
	}
	\Explain{ 1 $(\Rightarrow)$ assume first that $(A,\mu)$ is semifinite}
	\Explain{ 1.1 Then $A^f$ is order dense in $A$}
	\Explain{ 1.2 By order density theorem there is a desired partition of unity}
	\Explain{ 2 $(\Leftarrow)$  Let $P$ be the partition of unity}
	\Explain{ 2.1 Assume $a \in A$ is such that $\mu(a) = \infty$}
	\Explain{ 2.2 Then there exists $p \in P$ such that $pa \neq 0$}
	\Explain{ 2.3 Note that this means that $\mu(pa) > 0$}
	\Explain{ 2.4 Also it is clear that $\mu(pa) \le \mu(p) < \infty$}
	\EndProof
	\\
	\Theorem{SemifiniteneSupElementExpression}
	{
		\NewLine ::
		\forall (A,\mu) : \Semifinite\MA(A,\mu) \.
		\forall a \in A \. a = \bigvee \{ b \in A : b \le a, \mu(b) < \infty   \}
	}
	\Explain{ This follows from the previous theorem}
	\EndProof
	\\
	\Theorem{SemifiniteneSupMeasureComputation}
	{
		\NewLine ::
		\forall (A,\mu) : \Semifinite\MA(A,\mu) \.
		\forall a \in A \. \mu(a) = \bigvee \{ \mu(b) \in A : b \le a, \mu(b) < \infty   \}
	}
	\Explain{ This follows from the previous theorem}
	\EndProof
}
\newpage
\subsubsection{Connections with other Boolean Properties}
\Page{
	\Theorem{SemifiniteIsWeaklyDistributive}
	{
		\NewLine ::	
		\forall (A,\mu) : \Semifinite\MA(A,\mu) \. 
		\WD(A,\mu)
	}
	\Explain{ 1 Assume 
		$X : \Nat \to 2^A$ is a sequence of downwards selected sets with 
		$\inf X_n =  0$ for every $n \in \Nat$}
	\Explain{ 2 Let $C = \{ a \in A : \forall n \in \Nat \. \exists  x \in X_n \. a \ge x \}$}
	\Explain{ 3 Assume $d \in A$ is such that $d \neq 0$}
	\Explain{ 4 Then there is an element $d' \le d$ such that $0 < \mu(d') < 0$}
	\Explain{ 5 $\inf_{x \in X} d'x = 0$ for each $n \in N$}
	\Explain{ 6 Select a sequence $x : \prod^\infty_{n=1} X_n$ suc that 
		$\mu(d'x_n) \le 2^{-n-2} \mu(d')$}
	\Explain{ 7 Define $c = \sup_{n=1} a_n \in C$}
	\Explain{ 8 Then $\mu(d'c) \le \sum^\infty_{n=0} \mu(c x_n) < \mu(d')$} 
	\Explain{ 9 This means that $d \not \le c$}
	\Explain{ 10 And as $d$  was arbitrary $\inf C = 0$}
	\EndProof
	\\
	\Theorem{SemifiniteIffCCC}
	{
		\forall (A,\mu ) : \Semifinite\MA(A,\mu) \. \NewLine \.
		\sFinite\MA(A,\mu) \iff \CCC(A)
	}
	\Explain{ 1 $(\Leftarrow)$ assume that $A$ has ccc}
	\Explain{ 1.1 Then there is a partition of unitity $P$ in $A$ consisting of finite elements 
		as $A$ is semifinite}
	\Explain{ 1.2 But as $A$ has ccc $P$ must be atmost countable}
	\Explain{ 1.3 This proves that $A$ is $\sigma$-finite}
	\Explain{ 2 $(\Rightarrow)$ assume that $(A,\mu)$ is $\sigma$-finite }
	\Explain{ 2.1 Then there exists a countable partition of unity $P$ of $A$ 
		wirh finite elements}
	\Explain{ 2.2 If $A$ is not ccc, then there exists an uncountable refinement $Q$ of $A$ 
		with finite elements}
	\ExplainFurther{ 2.3 Then by pigionhole principle there exists $p \in P$}
	\Explain{ \quad \quad such that  set $Q' = \{ q \in Q :  q \subset p \}$ such that $Q'$ is uncountable}
	\ExplainFurther{ 2.4 as for $\mu(q) > 0$ for any $q \in Q'$ 
		by pigionhole principle there exists some $n\in \Int$}
	\Explain{ \quad \quad such that there are an infinite number 
		of $q \in Q'$ with $\mu(q) \in [2^{-n-1},2^{-n}]$}
	\Explain{ 2.5 So $\mu(p) \ge \sum_{q \in Q'} \mu(q) = \infty$, but this is a contradiction}
	\EndProof
}\Page{
	\Theorem{SemifiniteIffProbabilityRenormalizationExists}
	{
		\NewLine ::		
		\forall (A,\mu) : \Semifinite\MA(A,\mu) \. A \neq \{0\} \Imply \NewLine \Imply
		\exists \pi : A \to \EReals_+ \.  \TYPE{ProbabilityAlgebra}(A,\pi)
	}
	\Explain{ 1 Corresponding Stone space is $\sigma$-finite}
	\Explain{ 2 So there exists a proper renormalization of $\bar \mu$ to a probability $\pi$
		with the same sets of measure zero}
	\Explain{ 3 Then the measure algebra of $(\Z A,\pi)$ is a probability algebra and 
		$A_\pi \cong_\BOOL A$}
	\EndProof
}
\newpage
\subsubsection{Subspace Measures and Indefinite Integrals}
\Page{
	\Theorem{MeasurableEnvelopePrincipleIdealIsomorphism}
	{
		\NewLine ::		
		\forall (X,\Sigma,\mu) \in \MEAS \.
		\forall Y \subset X \.
		\forall E : \ME(X,\Sigma,\mu,Y) \.  
		(A_{\mu|Y},\widehat{\mu|Y}) \cong_{\mathsf{MA}}  \Big( ([E]), \hat \mu_{|([E])} \Big)
	}
	\Explain{ This result is technically convoluted but actually is pretty intuituve}
	\EndProof
	\\
	\Theorem{PrincipleIdealIsomorphism}
	{
		\NewLine ::		
		\forall (X,\Sigma,\mu) \in \MEAS \.
		\forall E \in \Sigma \.  
		(A_{\mu|E},\widehat{\mu|E}) \cong_{\mathsf{MA}}  \Big( ([E]), \hat \mu_{|([E])} \Big)
	}
	\Explain{ A straightforward application of  a previous theorem}
	\EndProof
	\\
	\Theorem{ThickEquivalence}
	{
		\NewLine ::		
		\forall (X,\Sigma,\mu) \in \MEAS \.
		\forall Y : \Thick(X,\Sigma,\mu) \.  
		(A_{\mu|E},\widehat{\mu|E}) \cong_{\mathsf{MA}}  ( X, \hat \mu )
	}
	\Explain{ A straightforward application of a previous theorem}
	\EndProof
	\\
	\Theorem{IndefiniteIntegralPrincipleIdealIsomorphism}
	{
		\NewLine ::		
		\forall (X,\Sigma,\mu) \in \MEAS \.
		\forall f \in \Integrable_+(X,\Sigma,\mu)  \.
		\exists E \in \Sigma \.
		A_{f\;d\mu} \cong_\BOOL ([E])
	}
	\Explain{ We may assume that $\supp f$ has a measurable envelope $E$}
	\Explain{ Then the result is obvious as $\Null_{\mu} \subset \Null_{f\;d\mu}$}
	\EndProof
}
\newpage
\subsubsection{Simple Products}
\Page{
	 \DeclareFunc{simpleProduct}
	 {
	 	\prod_{I \in \SET} (I \to \MA) \to \MA
	 }
	 \DefineNamedFunc{simpleProduct}{A,\mu}{\prod_{i \in I} \left( A_i,  \mu_i \right)}
	 {
		\left( \prod_{i \in I} A_i, \sum_{i \in I} \mu_i \right)	 
	 }
	 \Explain{ Obviously $\sum_{i \in I} \mu_i (0) = \sum_{i \in I} 0 = 0$}
	 \Explain{ Also assume $a : \Nat \to \prod_{i \in I} A_i$ is disjoint}
	 \Explain{ Then  
			$\sum_{i \in I} \mu_i\left( \bigvee^\infty_{n=1} a_n \right) =
			 \sum_{i \in I} \sum^\infty_{n=1} \mu_i( a_{n,i} ) =
			 \sum^\infty_{n=1} \sum_{i \in I} \mu_i( a_{n,i} ) =
			 \sum^\infty_{n=1} \sum_{i \in I} \mu_i (a_n)  			
			$}
	\EndProof
	\\
	\Theorem{PrincipleIdealsInMeasureAlgebras}
	{		
		\NewLine ::		
		\forall I \in \SET \.
		\forall (A,\mu) : I \to \MA \.
		(A_i,\mu_i) \cong_{\mathsf{MA}} \left( (e_i), \left(\sum_{i \in I} \mu_i\right)_{|(e_i)}  \right)
	}
	\Explain{This is pretty ovious}
	\EndProof
	\\
	\\
	\Theorem{SimpleProductCoproductCorrespondance}
	{		
		\NewLine ::		
		\forall I \in \SET \.
		\forall (X,\Sigma,\mu) : I \to \MEAS \.
		\prod_{i \in I} (A_{\mu_i}, \hat \mu_i) \cong  
		\FUNC{measureAlgebra} \coprod_{i \in I} (X_i,\Sigma_i,\mu_i)
	}
	\Explain{ Obvious by Stone Theory}
	\EndProof
	\\
	\Theorem{SimpleProductOfSemifinite}
	{
		\NewLine ::
		\forall I \in \SET \.
		\forall (A,\mu) : I \to \Semifinite\MA \.
		\Semifinite\MA\left( \prod_{i \in I} (A,\mu) \right)
	}
	\Explain{ Assume $a$ has infinite measure in $(A,\mu)$}
	\Explain{ Then there exists $i \in I$ such that $a_i \neq 0$}
	\Explain{ As $(A_i,\mu_i)$ is semifinite there is $b \le a_i$ such that $0 < \mu_i(b) < \infty $}
	\Explain{ Then $be_i \le a$ and $0 < \sum_{j \in I} \mu_j (be_i) = \mu_i(b) < \infty$ }
	\EndProof
	\\	
}\Page{
	\Theorem{SimpleProductOfLocalizable}
	{
		\NewLine ::
		\forall I \in \SET \.
		\forall (A,\mu) : I \to \Loc\MA \.
		\Loc\MA\left( \prod_{i \in I} (A,\mu) \right)
	}
	\Explain{ Let $J$ be a set and $a : J \to \prod_{i \in I} (A_i,\mu_i)$ }
	\Explain{ Then $\sup_{j \in J} a_j = (\sup_{j \in J} a_{j,i} )_{i \in I}$}
	\EndProof
	\\
	\Theorem{PoUProductRepresentation}
	{
		\NewLine ::
		\forall (A,\mu) : \MA \.
		\forall (e_n)^\infty_{n=1} : \PoU(A) \.
		(A,\mu) \cong_{\mathsf{MA}} \prod^\infty_{n=1} \Big((e_n), \mu_{|(e_m)}\Big)
	}
	\Explain{This is pretty obvious}
	\EndProof
	\\
	\Theorem{PoUProductRepresentation}
	{
		\NewLine ::
		\forall (A,\mu) : \Loc\MA \.
		\exists I \in \SET \.		
		\exists  (B,\nu)  : I \to \Finite\MA \. \NewLine \.  
		(A,\mu) \cong_{\mathsf{MA}} \prod_{i \in I} (B_i,\nu_i) 
	}
	\Explain{ It is possible to select a partition of unity $P$ of $A$ consisting of finite elements}
	\Explain{ Then by previous theorem $(A,\mu) \cong \prod_{p \in P} \Big( (p), \mu_{|(p)} \Big)$}
	\Explain{ And each $\Big( (p), \mu_{|(p)} \Big)$ are obviously finite}
	\EndProof
	\\
	\Theorem{LocalizableMeasureAlgebrasHasLocallyDeterminedRepresentations}
	{
		\NewLine ::
		\forall (A,\mu) : \Loc\MA \.
		\exists  (X,\Sigma,\nu)  : \LocDet \.
		(A,\mu) \cong_{\mathsf{MA}} (A_\nu,\hat \nu) 
	}
	\Explain{ Represent $(A,\mu) \cong_{\mathsf{MA}} \prod_{i \in I} (B_i,\nu_i) $}
	\Explain{ Then Stone's spaces $\Z \; B_i$ correspond to finite measure spaces}
	\Explain{ And Stone's space of product correspond to a disjoint union of $\Z \; B_i$}
	\Explain{ But such spaces are trivially locally determined}
	\EndProof
}
\newpage
\subsubsection{Strictly Localizable Spaces}
\Page{
	\Theorem{StrictlyLocalizableSpacePoU}
	{
		\NewLine ::		
		\forall (X,\Sigma,\mu) : \SLoc \.
		\forall P : \PoU(A_\mu) \. \NewLine \.
		\exists  E  : P \to \Sigma \.
		\forall p \in P \. [E_p] = p
		\And
		\TYPE{Decomposition}(X,\Sigma,\mu,\im E)
	}
	\NoProof
}
\newpage
\subsubsection{Subalgebras}
\Page{
	\Theorem{SubalgebaMeasureAlgebra}
	{
		\forall (A,\mu) : \MA \.
		\forall B \subset_\sigma A \.
		\MA(B,\mu_{|B})
	}
	\Explain{This is obvious}
	\EndProof
	\\
	\Theorem{SubalgebaFinifteMeasureAlgebra}
	{
		\NewLine ::		
		\forall (A,\mu) : \Finite\MA \.
		\forall B \subset_\sigma A \.
		\Finite\MA(B,\mu_{|B})
	}
	\Explain{This is obvious}
	\EndProof
	\\
	\Theorem{SigmaFiniteSubalgebraMeasureAlgebra}
	{
		\NewLine ::		
		\forall (A,\mu) : \sFinite\MA \.
		\forall B \subset_\sigma A \. \NewLine \.
		\Semifinite\MA(B,\mu_{|B}) \Imply
		\sFinite\MA(B,\mu_{|B})
	}
	\Explain{ 1 The set $B^f$ is order-dense in $B$}
	\Explain{ 2 But then $B^f$ is also order-dense in $A$}
	\Explain{ 3 Select a finite-measured countable partition of unity $P$ in $A$}
	\Explain{  4 If $B$ is not $\sigma$-finite, then there is a subordinate uncountale partition of unity $Q$}	
	\Explain{ 5 Then there would exist 
		a uncountable refinement of $P$ subordinate to $Q$}
	\Explain{ 6 Then $P$ must contain an infinite element, but this is imposible!}
	\Explain{ 7 So $Q$ must be countable, and so $(B,\mu_{|B})$ must be countable}
	\EndProof
	\\
	\Theorem{FinifteMeasureAlgebraBySubalgebra}
	{
		\NewLine ::		
		\forall (A,\mu) : \MA \.
		\forall B \subset_\sigma A \.
		\Finite\MA(B,\mu_{|B}) \Imply \Finite\MA(A,\mu)
	}
	\Explain{This is obvious}
	\EndProof
	\\
	\Theorem{ProbabilityAlgebraBySubalgebra}
	{
		\NewLine ::		
		\forall (A,\mu) : \MA \.
		\forall B \subset_\sigma A \. \NewLine \.
		\TYPE{ProbabilityAlgebra}(B,\mu_{|B}) \Imply \TYPE{ProbabilityAlgebra}(A,\mu)
	}
	\Explain{This is obvious}
	\EndProof
}\Page{
	\Theorem{SigmaFiniteAlgebraBySubalgebra}
	{
		\NewLine ::		
		\forall (A,\mu) : \MA \.
		\forall B \subset_\sigma A \. \NewLine \.
		\sFinite(B,\mu_{|B}) \Imply \sFinite(A,\mu)
	}
	\Explain{This is obvious}
	\EndProof
}
\newpage
\subsubsection{Localization}
\Page{
	\Theorem{MeasureAlgebraCompletion}
	{
		\NewLine ::		
		\forall (A,\mu) : \Semifinite\MA \.
		\exists! \hat \mu : \tau(A) \to \EReals_{++} \. \NewLine \.
		\hat \mu_{|A} = \mu  \And \Loc\MA(\tau(A),\hat \mu) 
	}
	\Explain{ 1 Define $\hat \mu(t) =  \sup \{ \mu(a) | a \in A, a \le t \}$}
	\Explain{ 2 As $A$ is order dense in $\tau(A)$, it holds that $\hat\mu(a) = 0\iff a = 0$ 
		for any $a \in \tau(A)$}
	\Explain{ 3 If $t : \Nat \to \tau(A)$ is disjoint then 
		$\hat \mu\left( \bigvee_{n=1}^\infty t_n\right) = \sum^\infty_{n=1} \hat \mu(t_n) $
	}
	\Explain{ 3.1 Write $S = 
		\{ a \in A : \exists c : \Nat \to A \.  a = \lim_{n \to \infty} c_n \And c \le t  \}$}
	\Explain{ 3.2 Then there is $s = \sup S \in \tau(A)$}
	\Explain{ 3.3 We write 
		$
			\hat \mu(s) = \sup_{c \le t } \mu\left( \bigvee^\infty_{n=1} c_n \right) =
			\sup_{c \le t} \sum^\infty_{n=1} \mu(c_n) = 
			\sum^\infty_{n=1} \sup_{c \le t_n} \mu(c) =
			\sum^\infty_{n=1} \hat \mu(t_n)
		$ }
	\Explain{ 4 Obviously $(\tau(A),\hat \mu)$ is semifinite and order-complete, and hence Localizable}
	\EndProof
	\\
	\DeclareFunc{localization}
	{
		\Semifinite\MA \to \Loc\MA
	}
	\DefineNamedFunc{localization}{A,\mu}{\Big(\tau(A),\tau(\mu)\Big)}
	{
		\THM{MeasureAlgebraCompletion}	
	}
	\\
	\Theorem{LocalizationFiniteEmbedding}
	{
		\NewLine ::		
		\forall (A,\mu) : \Semifinite\MA \.
		\iota_\tau(A^f) =  \tau^f(A) 
	}
	\Explain{ 1 Assume $t \in \tau(A)$ such that $\hat \mu(t) < \infty$}
	\Explain{ 2 Note, $\hat \mu(t) = \sup_{a \le t} \mu(a)$ }
	\Explain{ 3 So we may select an increasing $a : \Nat \to A$
		such that $\lim_{n \to \infty} \mu(a_n) = \hat \mu(t)$}
	\Explain{ 4 Then $b = \bigvee^\infty_{n=1} a_n \in A$ and $\hat \mu(b) = \mu(b) = \hat \mu(t)$}
	\Explain{ 5 So $\mu(t \setminus b) = 0$, and  so $t = b \in A$ as clearly $b < t$}
	\EndProof
}
\newpage
\subsubsection{Stone Spaces}
\Page{
	\Theorem{LocallalizableMeasureAlgebraHasStrictlyLocalizableStoneSpace}
	{
		\NewLine ::
		\forall (A,\mu) : \Loc\MA \.
		\SLoc (\Z\; A, \Sigma_\mu, \bar \mu)  	
	}
	\Explain{ 1 We already proved that $\bar \mu$ is locally determined}
	\Explain{ 2 As $(A,\mu)$ is semifinite there is a partition of  unity $P$ 
		consisting of finite elements}
	\Explain{ 3 Use Stone representation $S_A(P)$ to construct a corresponding set in $\Z\; A$}
	\Explain{ 4 Assume $E \in \Sigma_\mu$ such that $\bar \mu(E) > 0$}
	\Explain{ 5 By definition of Stone's Space there is a clopen set $F \in \Z \; A$ 
		such that $E \du F$ is meager}
	\Explain{ 6 And there is a Stone representation $a \in A$ such that $F = S_A(a)$}
	\Explain{ 7 Then $\mu(a) = \nu(S_A(a)) = \nu(E) > 0$}
	\Explain{ 8 So, there exists $p \in P$ such that $ap \neq 0$}
	\Explain{ 9 Ths means that $\nu(E \cap S_A(p)) > 0$}
	\Explain{ 10 As  $E$ was arbitrary this means that $S_A(P)$ 
		provides a strict localization for $\bar \mu$}
	\EndProof
	\\
	\Theorem{MeagerSetsAreNowhereDense}
	{
		\NewLine ::		
		\forall (A,\mu) : \Semifinite\MA \.		
		\forall  M  \in \mgr(\Z\; A)  \. 
		 \ND(\Z\;A, M)
	}
	\Explain{ 1 As it was shown $A$ is $\WD$ boolean algebra}
	\Explain{ 2 And this is a property of $\WD$ boolean algebra}
	\EndProof
	\\
	\Theorem{StoneSpaceMeasurableExpression}
	{
		\NewLine ::		
		\forall (A,\mu) : \Semifinite\MA \.		
		\forall  E \in \Sigma_\mu  \.  \NewLine \.
		\exists U : \Clopen(\Z\;A) \.
		\exists F : \ND(\Z\;A)	\.	
		E = U \cap F
	}
	\Explain{ 1 This is clear from the previous theorem}
	\EndProof
	\\
	\Theorem{StoneSpaceMeasureComputation}
	{
		\NewLine ::		
		\forall (A,\mu) : \Semifinite\MA \.		
		\forall  E \in \Sigma_\mu  \.  \NewLine \.
		\bar \mu(E) = \sup \Big\{ \mu(U) \Big| U : \Clopen(\Z\;A) , U \subset E   \Big\}
	}
	\Explain{ 1 This is clear from the previous theorem}
	\EndProof
	\\
	\Theorem{StoneSpaceCLDIsStrictlyLocalizable}
	{
		\NewLine ::
		\forall (A,\mu)  : \Semifinite\MA \.
		\SLoc(\Z \;A, \bar \Sigma_\mu, \bar{\bar \mu}) 
	}
	\NoProof
}\Page{
	\Theorem{StoneSpaceCLDZeroSets}
	{
		\NewLine ::
		\forall (A,\mu)  : \Semifinite\MA \.
		\Null_{\bar{\bar \mu}} = \Null_{\bar \mu} 
	}
	\NoProof
	\\
	\Theorem{FiniteStoneSpaceMeasureComputation}
	{
		\NewLine ::		
		\forall (A,\mu) : \Finite\MA \.		
		\forall  E \in \Sigma_\mu  \.  \NewLine \.
		\bar \mu(E) = \inf \Big\{ \mu(U) \Big| U : \Clopen(\Z\;A), E \subset U \Big\}
	}
	\Explain{ 1 This is clear from the previous theorem}
	\EndProof
}
\newpage
\subsubsection{Purely Infinite Elements}
\Page{
	\DeclareFunc{purelyInfiniteElements}
	{
		\prod (A,\mu) : \MA \. \SIdeal(A)
	}
	\DefineNamedFunc{purelyInfiniteElements}{}{I_\infty(\mu}
	{
		\{ a \in A : \forall b  \in A \. b \le a \And \mu(b) < \infty \Imply b = 0  \}
	}
	\\
	\DeclareFunc{semifiniteMeasure}{\prod (A,\mu) : \MA \. \frac{A}{I_\infty(\mu)} \to \EReals_+}
	\DefineNamedFunc{semifiniteMeasure}{[a]}{\mu_{\mathrm{sf}}}
	{\sup \{ \mu(b) | b \in A : b \le a \And \mu(b) < \infty\}}
	\Explain{ If $[a] = [b]$, then $a \du b \in I_\infty(\mu)$}
	\Explain{ So $\mu_{\mathrm{sf}}$ is well-defined} 
	\\
	\Theorem{SemifiniteMeasureIsMeasure}
	{
		\NewLine ::		
		\forall (A,\mu) : \MA \. \Semifinite\MA\left( \frac{A}{I}, \mu_{\mathrm{sf}} \right)	
	}
	\Explain{ 1 If $\mu_{\mathrm{sf}}[a] = 0$, then clearly $a \in I_\infty$}
	\Explain{ 2 Assume $[a] : \Nat \to A$ is disjoint}
	\Explain{ 2.1 Then $a_n a_m \in I_\infty$  if $n \neq m$}
	\Explain{ 2.2 Select increasing $b : \Nat \to A^f$ such that $b_n \le \bigvee^\infty_{k=1} a_k$
		and $\lim_{n \to \infty} \mu(b_n) = \mu_{\mathrm{sf}} \left[\bigvee^\infty_{k=1} a_k  \right] 
			= \mu_{\mathrm{sf}} \bigvee^\infty_{k=1} [a_k]$}
	\Explain{ 2.3  By (2.1) we mat assert that $ab_n$ is disjoint and then $\bigvee^\infty_{k=1} a_kb_n = b_n$
	 for any $n \in \Nat$ }
	\Explain{ 2.4 So $\mu(b) = \sum^\infty_{k=1}  \mu(a_k b_n) $}
	\ExplainFurther{ 2.5 By taking limits and using monotonic convergence theorem} 
	\Explain{ $\quad \quad \sum^\infty_{k=1} \mu_{\mathrm{sf}}[a_k] = 
		\sum^\infty_{k=1}  \lim_{n \to \infty} \mu(a_k b_n) = 
		\lim_{n \to \infty} \mu(b_n) = 
		\mu_{\mathrm{sf}} \bigvee^\infty_{k=1} [a_k]$}	
	\Explain{ 3 Clearly  $\mu_{\mathrm{sf}}[a] < \mu(a)$}	
	\Explain{ 3.1 If $\mu_{\mathrm{sf}}[a] = \infty$, then $a \not \in I_\infty$}
	\Explain{ 3.2 So it is possible to select $b \in A$ such that $b \le a$ and $0 < \mu(b) \le a$}
	\Explain{ 3.3 $0 < \mu_{\mathrm{sf}}[b] \le \mu(b) < \infty$}
	\Explain{3.4 This proves that $\left( \frac{A}{I}, \mu_{\mathrm{sf}} \right)$ is semifinite}
	\EndProof
}
\newpage
\subsection{Topology}
\subsubsection{Subject}
\Page{
	\DeclareFunc{measureAlgebraAsTopologicalSpace}
	{
		\MA \to \TOP
	}
	\DefineNamedFunc{measureAlgebraAsTopologicalSpace}{(A,\mu)}{(A,\mu)}
	{
		\NewLine \de 		
		\Big( A, \mathcal{W}\big(A^f \times A^f,
			\Reals, \Lambda a \in A^f \. \Lambda b \in A^f \. \Lambda c \in A \. \mu(ac + ab )\big)\Big) 
	}
	\\
	\DeclareFunc{measureAlgebraAsUniformlSpace}
	{
		\MA \to \UNI
	}
	\DefineNamedFunc{measureAlgebraAsUniformSpace}{(A,\mu)}{(A,\mu)}
	{
		\NewLine \de 		
		\Big( A, \mathcal{I}\big(A^f \times A^f,
			\Reals, \Lambda a \in A^f \. \Lambda b \in A^f \. \Lambda c \in A \. \mu(ac \du ab )\big)\Big) 
	}
	\\
	\DeclareFunc{metricOfFrechetNikodym}
	{
		\prod (A,\mu) : \MA \.  \TYPE{Metric}(A^f) 
	}
	\DefineNamedFunc{metricOfFrechetNikodym}{}{\rho_\mu}
	{
		\Lambda a,b \in A^f \. \mu(a \du b)
	}
	\\
	\Theorem{BooleanOperationsAreUniformlyContinuous}
	{
		\NewLine ::		
		\forall (A,\mu) : \MA \.  (*),(\setminus),(\vee),(\wedge) \in \UNI(A \times A, A)
	}
	\Explain{ 1 Let $\circ$ stay for any binary operation above}
	\Explain{ 2 Select $c,d \in A$}
	\Explain{ 3 Then 
		$
			\mu\Big(a(c \circ d) +  b\Big) \le 
			\mu\Big( a(c \vee d) + b \Big) \le 
			\mu( ac + d) + \mu(ad + b)
		$  }
	\Explain{ 4 So $\mu$ is bounded by the sum of uniform functions and is uniformly continuous}
	\EndProof
	\\
	\Theorem{FiniteElementsAreDense}
	{
		\NewLine ::		
		\forall (A,\mu) : \MA \.  \Dense(A,A^f)
	}
	\Explain{ 1 Select $c \in A$}
	\Explain{ 2 Then $c$ has a base of neighborhoods of form 
		$U = \{ u \in A : \mu(au + ac) \le r\}$ with $a \in A^f, r \in \Reals_{++}$}
	\Explain{ 3 But then $ac \in U$ and $ac \in A^f$}
	\EndProof
	\\
	\Theorem{FiniteMeasureAlgebraHasUniformlyContinuousMeasure}
	{
			\NewLine
			\forall (A,\mu) : \Finite\MA \. \mu \in \UNI(A,\Reals_{++})
	}
	\Explain{ This is pretty obvious as $\mu = \rho_\mu(0,a)$}
	\EndProof
}\Page{
	\Theorem{FiniteMeasureAlgebraHasUniformlyContinuousMeasure}
	{
			\NewLine
			\forall (A,\mu) : \Finite\MA \. \mu \in \UNI(A,\Reals_{++})
	}
	\Explain{ This is pretty obvious as $\mu = \rho_\mu(0,a)$}
	\EndProof
	\\
	\Theorem{SemifinitMeasureAlgebraHasLowerSemicontinuousMeasure}
	{
			\NewLine
			\forall (A,\mu) : \Semifinite\MA \. \mu \in \TYPE{LowerSemicontinuous}(A,\EReals_{++})
	}
	\Explain{ 1 Assume $a \in A$ and $\alpha \in \Reals_{+}$ such that $\mu(a) > \alpha$ }
	\Explain{ 2 As $A$ is semifinite there exists $b \le a$ such that $\infty > \mu(b) > \alpha$}
	\Explain{ 3 Assume $c \in A$ is such that $\mu(b + cb) < \mu(b) - \alpha$ }
	\Explain{ 4 Then 
		$\mu(c) \ge \mu(cb) = 
		\mu(b)  - \mu(b(a\setminus c)) =  
		\mu(b) - \mu(b + cb) > \alpha $}
	\EndProof
	\\
	\Theorem{MeasureAlgebraHasUniformlyContinuousFinitisedMeasure}
	{
			\NewLine
			\forall (A,\mu) : \MA \. 
			\forall a \in A^f \.			
			(\Lambda c \in A \. \mu(ac)) \in \UNI(A,\Reals_{++})
	}
	\Explain{  This is simmilar to the case of finite measure space}
	\EndProof
	\\
	\DeclareFunc{finiteElementMetric}
	{
		\prod A : \MA \. A^f \to \TYPE{Semimetric}(A)
	}
	\DefineNamedFunc{finiteElementMetric}{a}{\rho_a}
	{
		\Lambda x,y \in A \. \mu(ax + ay)	
	}
	\\
	\Theorem{MeasurAlgebraProductTopology}
	{
		\NewLine ::
		\forall I \in \SET \.
		\forall (A,\mu) : I \to \MA \. 
		\prod_{i \in I} (A,\mu) =_\TOP \left(\prod_{i \in I} A_i, \sum_{i \in I} \mu_i \right)
	}
	\NoProof
}
\newpage
\subsubsection{Relations with an Order Structure}
\Page{
	\DeclareFunc{upwardDirectedFilter}
	{
		\NewLine ::		
		\prod (A,\mu) : \MA \. 
		 \TYPE{NonEmpty} \And \TYPE{UpwardsDirected}(A) \to \TYPE{CauchyFilerbase}(A)
	}
	\DefineNamedFunc{upwardDirectedFilter}{D}{\F(\uparrow D)}
	{
		\Big\{ \{ c \in D : d \le c \} \Big| d \in D \Big\} 	
	}
	\Explain{ 1 Write $F_d = \{ c \in D : d \le c \}$}
	\Explain{ 2 $\F(\uparrow D)$ is a filter}
	\Explain{ 2.1 As $D$ is non empty, $\F(\uparrow D)$ is also non-empty}
	\Explain{ 2.2  $d \in F_d$, so $F_d \neq \emptyset$ 
		and henceforth $\emptyset \not \in \F(\uparrow D)$}
	\Explain{ 2.3 Assume $F_d,F_f \in \F(\uparrow D)$    }
	\Explain{ 2.3.1  Then there is an element $g \in D$ such that $g \ge f \vee d$}
	\Explain{ 2.3.2  Note, that $F_g \subset F_d \cap F_f$ and $F_g \in  \F(\uparrow D)$ }
	\Explain{ 3 $\F(\uparrow D)$ is Cauchy}
	\Explain{ 3.1 Assume $U$ is some measure connector for $(A,\mu)$}
	\Explain{ 3.2 then there is an element $a \in A^f$ and $r \in \Reals_{++}$
		such that $\{ (f,g) \in A \times A : \mu(af  +  ag) < r \} \subset U$}
	\Explain{ 3.3 The set $\{ \mu(ad) | d \in D  \}$ is bounded  by $\mu(a)$,
		so supremum is attained}
	\Explain{ 3.4 So there is $f \in D$, so $\mu(ad) < \mu(af) + r/2$ for any $d \in D$   }
	\Explain{ 3.5 Assume $g,h \in F_f$ }
	\Explain{ 3.5 Then 
		$
			\mu(ag + ah) \le \mu(ag \setminus af) + \mu(ah \setminus af) =
			\Big(\mu(ag) - \mu(af)\Big) + \Big(\mu(ah) - \mu(af)\Big)  < r 
		$}
	\Explain{ 3.6 Thus, $(g,h) \in U$ and $F_f \times F_f \subset U$}
	\EndProof
	\\
	\Theorem{UpwardsDirectedSup}
	{
		\NewLine ::		
		\forall (A,\mu) : \Semifinite\MA \. 
		\forall D :  \TYPE{UpwardsDirected}(A) \to \TYPE{CauchyFilerbase}(A)  \.
		\forall a \in A	\. \NewLine \.
		a = \sup D \Imply a = \lim \F(\uparrow D)  
	}
	\Explain{ 1 Assume $a = \sup D$}
	\Explain{ 2 Assume $U$ is an uniformity fo $(A,\mu)$}
	\Explain{ 3 then there is an element $c \in A^f$ and $r \in \Reals_{++}$
		such that $\{ g \in A \times A : \mu(ca  +  cg) < r \} \subset U(a)$}
	\Explain{ 4 Consider set $M = \{ \mu(cd) | d \in D  \}$}
	\Explain{ 5 Then $\sup M = \mu(ca)$}
	\Explain{ 6 So there is $d \in D$ such that $\mu(ca + cd) < r$}
	\Explain{ 7 But $ d \le f \le a $for any $f \in F_d$}
	\Explain{ 8 Thus $\mu(cf + cd) < r$ and $F_d \subset U(a)$}
	\Explain{ 9 Thus, $da = \lim \F(\uparrow D)$}
	\EndProof
}\Page{
	\Theorem{UpwardsDirectedLimit}
	{
		\NewLine ::		
		\forall (A,\mu) : \Semifinite\MA \. 
		\forall D : \TYPE{NonEmpty} \And \TYPE{UpwardsDirected}(A)  \.
		\forall a \in A	\. \NewLine \.
		a = \sup D \Imply a  \in \cl_{A} D
	}
	\NoProof
	\\
	\Theorem{UpwardsDirectedFilterLimit}
	{
		\NewLine ::		
		\forall (A,\mu) : \Semifinite\MA \. 
		\forall D : \TYPE{NonEmpty} \And \TYPE{UpwardsDirected}(A)  \.
		\forall a \in A	\. \NewLine \.
		a = \lim \F(\uparrow D) \iff a = \sup D  
	}
	\Explain{ 1 $(\Rightarrow) \quad a = \lim \F(\uparrow D)$ }
	\Explain{ 1.1 Then for any connector $U$ of $(A,\mu)$
		There is some $F \in \F(\uparrow F)$ such that $F \subset U(a)$}
	\Explain{ 1.2 Assume $d \in D$ }
	\Explain{ 1.3 Assume $d \not \le a$}
	\Explain{ 1.4 Then there is $f \in A$ such that $f \le d \setminus a$ 
		and $0 < \mu(f) < \infty$}
	\Explain{ 1.5 Thus $\mu(fh + fa) \ge \mu(f)$ for every $h \in F_s$}
	\Explain{ 1.6 But $G \cap F_d \neq \emptyset$ for  any  $G \in \F(\uparrow D)$ 
		so this contradicts $(1.1)$}
	\EndProof
	\\
	\DeclareFunc{lowerDirectedFilter}
	{
		\NewLine ::		
		\prod (A,\mu) : \MA \. 
		 \TYPE{NonEmpty} \And \TYPE{LowerDirected}(A) \to \TYPE{CauchyFilerbase}(A)
	}
	\DefineNamedFunc{loweDirectedFilter}{D}{\F(\uparrow D)}
	{
		\Big\{ \{ c \in D : d \ge c \} \Big| d \in D \Big\} 	
	}
	\\
	\Theorem{LowerDirectedInf}
	{
		\NewLine ::		
		\forall (A,\mu) : \Semifinite\MA \. 
		\forall D :  \TYPE{NonEmpty} \And \TYPE{LowerDirected}(A)   \.
		\forall a \in A	\. \NewLine \.
		a = \inf D \Imply a = \lim \F(\uparrow D)  
	}
	\Explain{ By duality}
	\EndProof	
	\\
	\Theorem{UpwardsDirectedLimit}
	{
		\NewLine ::		
		\forall (A,\mu) : \Semifinite\MA \. 
		\forall D :  \TYPE{NonEmpty} \And \TYPE{LowerDirected}(A)  \.
		\forall a \in A	\. \NewLine \.
		a = \inf D \Imply a  \in \cl_{A} D
	}
	\Explain{ By duality}
	\EndProof
	\\
	\Theorem{UpwardsDirectedFilterLimit}
	{
		\NewLine ::		
		\forall (A,\mu) : \Semifinite\MA \. 
		\forall D : \TYPE{NonEmpty} \And \TYPE{LowerDirected}(A)  \.
		\forall a \in A	\. \NewLine \.
		a = \lim \F(\uparrow D) \iff a = \inf D  
	}
	\Explain{ By duality}
	\EndProof	
}\Page{
		\Theorem{ClosedSetsAreOrderClosed}{ 
		\forall (A,\mu) : \MA \. 
		\forall F : \Closed(A) \.
		\OC(A,F)
	}	
	\Explain{Follows from previous theorems in this chapter}
	\EndProof
	\\
	\Theorem{DenseSetsAreOrderDense}{ 
		\forall (A,\mu) : \MA \. 
		\forall \Dense(A,D) \.
		\OD(A,D) \.	
	}	
	\Explain{Follows from previous theorems in this chapter}
	\EndProof
	\\
	\Theorem{ClosedRays}
	{
		\forall (A,\mu) : \Semifinite\MA \.
		\forall a \in A \.		
		\Closed\Big( A, \{ c \in A : c \le a  \} \And \{ c \in A : c \ge a \} \Big)
	}
	\Explain{ 1 Let $F = \{ c \in A : c \le a \}$ }
	\Explain{ 2 Assume $d \in F^\c$}
	\Explain{ 3 Then $d \setminus a \neq 0$}
	\Explain{ 4 As $A$ is semifinite there is an $g \in A^f$ 
		such that $g \le d \setminus a$ and $0 < \mu(g)$}
	\Explain{ 5 $\rho_g(d,f) \ge \mu(g)$ fo any $f \in F^\c$}
	\Explain{ 6 And this means that $F^\c$ and $F$ is closed}
	\EndProof
	\\
	\Theorem{SupremumConvergence}
	{
		\forall A : \MA \.
		\forall a : \Nat \uparrow A \.
		\forall s \in A \.
		s = \sup_{n=1} a_n \Imply  s = \lim_{n=1} a_n
	}
	\Explain{ This is obvious now}
	\EndProof
	\\
	\Theorem{InfimumConvergence}
	{
		\forall A : \MA \.
		\forall a : \Nat \downarrow A \.
		\forall s \in A \.
		s = \inf_{n=1} a_n \Imply  s = \lim_{n=1} a_n
	}
	\Explain{ This is obvious now}
	\EndProof
	\\
	\DeclareType{\SI}{
		\prod A : \MA \. ?(\Nat \to A)	
	}
	\DefineType{a}{\SI}{\forall n \in \Nat \. \sum^\infty_{n=1} \mu(a_n + a_{n+1}) < \infty}
}\Page{
	\Theorem{SummableIncrementsLimSupLimInfEq}
	{
		\NewLine ::		
		\forall A : \MA \.
		\forall a : \SI(A) \. 
		\inf_{n=1} \sup_{m=n} a_n = \sup_{n=1} \inf_{m=n} a_n	
	}
	\Explain{ 1 Let $\alpha_n =   \mu(a_n + a_{n+1}),\beta_n = \sum^\infty_{m=n} \alpha_n$}
	\Explain{ 2 As $a$ has summable increments this means $\beta \downarrow 0$}
	\Explain{ 3 Let $b_n = \sup_{m \ge n} a_m + a_{m + 1} = \bigvee^\infty_{m=n} a_m + a_{m+1}$}
	\Explain{ 4 Then $\mu(b_n) \le \sum^\infty_{m=n} \mu(c_m + c_{m+1} ) = \beta_n $}
	\Explain{ 5 Assume $m \le n$}
	\Explain{ 6 And also $a_m + a_n \le \sup_{m\le k \le n} a_k + a_{k+1} \le b_n$}
	\Explain{ 7 So $a_n \setminus b_n \le a_m \le a_n \vee b_n$}
	\Explain{ 8 Thus $a_n \setminus b_n \le \inf_{k \ge m} a_k \le \sup_{k \ge m} a_k \le a_n \vee b_n$}
	\Explain{ 9 By taking limits in $m$ one gets
		$a_n \setminus b_n \le \inf_{m=1} \sup_{k=n} a_k \le \sup_{m=1} \inf_{k=m} a_k \le 
			a_n \vee b_n$}
	\Explain{ 10 $a_n + \inf_{m=1} \sup_{k=m} a_k \le b_n $}
	\Explain{ 11 $a_n + \sup_{m=1} \inf_{k=m} a_k \le b_n $}
	\Explain{ 12 From (10) and (11) 
		$\inf_{m=1} \sup_{k=m} a_k \setminus \sup_{m=1} \inf_{k=m} a_k  \le b_n$}
	\Explain{ 13 But $\lim_{n \to \infty}  b_n = 0$}
	\Explain{ 14 So $\inf_{m=1} \sup_{k=m} a_k  = \sup_{m=1} \inf_{k=m} a_k $}
	\EndProof
	\\
	\Theorem{SummableIncrementsLim}
	{
		\NewLine ::		
		\forall A : \MA \.
		\forall a : \SI(A) \. 
		\forall x \in A \. \NewLine \.
		x = \lim_{n \to \infty} a_n \Imply
		\inf_{n=1} \sup_{m=n} a_n = x = \sup_{n=1} \inf_{m=n} a_n	
	}
	\Explain{ This follows from the previous proof}
	\EndProof
}
\newpage
\subsubsection{Classification Theorems}
\Page{
	\Theorem{SemifiniteIffHausdorff}
	{
		\forall (A,\mu) : \MA \.
		\Semifinite\MA(A,\mu) \iff \TYPE{T2}(A)
	}
	\Explain{1 $(\Rightarrow)$ assume that $(A,\mu)$ is semifinite}
	\Explain{1.1 Take $x,y \in A$ such that $x \neq y$}
	\Explain{1.2 Then  $x +  y \neq 0$ so there is $a \in A^f$ such that $\mu(a) > 0$ and $a < x + y$ }
	\Explain{1.3 So $\rho_a(x,y) = \mu(a) > 0$}
	\Explain{1.4 And cells of form $\Cell_{\rho_a}(x,\mu(a)/2)$ and $\Cell_{\rho_a}(y,\mu(a)/2)$
		produce the separation}
    \Explain{2 $(\Leftarrow)$ assume that $A$ is Hausdorff in the topology of $(A,\mu)$}
    \Explain{2.1 Assume $x \in A$ such that $\mu(x) = \infty$}
    \Explain{2.2 Then $x \neq 0$}
    \Explain{2.3 Assume $a \in A^f$}
    \Explain{2.4 If $xa = 0$ then $\rho_a(x,0) = 0$}
    \Explain{2.5 So, as $A$ is Hausdorff there mustb some $a \in A^f$ such that $xa \neq 0$}
    \Explain{2.6 But this means that $(A,\mu)$ is semifinite}
    \EndProof
    \\
    \Theorem{SigmaFiniteIffMetrizable}
	{
		\NewLine ::		
		\forall (A,\mu) : \MA \.
		\sFinite\MA(A,\mu) \iff \TYPE{Metrizable}(A)
	}
	\Explain{1 $(\Rightarrow)$ assume that $(A,\mu)$ is $\sigma$-finite}
	\Explain{1.1 Then there is a countable partition of unity $a$ with finite elements}
	\Explain{1.2 define $\sigma : A^2 \to \Reals_{++}$ as 
		$\sigma(x,y) = \sum^\infty_{n=1} 2^{-n} \frac{\rho_{a_n}(x,y)}{\mu(a_n)}$}
	\Explain{1.3 Then $\sigma$ is a metic for $A$ }
	\Explain{1.4 So the topology of $(A,\mu)$ is metrizable}  
	\Explain{2 $(\Leftarrow)$ assume that $(A,\mu)$ is metrizable}
	\Explain{2.1 Let $\sigma$ be an metrizing metric}
	\ExplainFurther{2.2 Then there exists a system of elements 
		$k : \Nat \to \Nat, a : \prod^\infty_{n=1} \{1,\ldots,k_n\} \to A^f$ and 
		$\delta : \Nat \to \Reals_{++}$ }
	\Explain{ $\quad \quad$ such that  $\rho_{a_{n,i}}(b,e)$ 
		for any $1 \le i \le k_n$ imply that $\sigma(b,e) < 2^{-n}$ for any $b \in A$}
	\Explain{ 2.3 Then $e = \bigvee^\infty_{n=1} \bigvee^{k_n}_{i=1} a_{n,i}$}
	\Explain{ 2.4 So $(A,\mu)$ is $\sigma$-finite} 
    \EndProof
 }\Page{
    \Theorem{LocalizableIffComplete}
	{
		\NewLine ::		
		\forall (A,\mu) : \MA \.
		\Loc\MA(A,\mu) \iff \TYPE{T2}\And \TYPE{Complete}(A)
	}
	\Explain{ 1 $(\Rightarrow)$ Assume $(A,\mu)$ is localizable measure algebra}
	\Explain{ 1.2 Then $A$ is Hausdorff as $(A,\mu)$ is semifinite}
	\Explain{ 1.3  Assume $\F$ is a Cauchy filter in $A$}
	\Explain{ 1.4 Take $a \in A^f$ }
	\Explain{ 1.5 Then there is $d_a \le a$ and a cauchy sequence $c_a$ subordinate to $\F$ 
			such that 	$\lim_{n \to \infty} \rho_a(d_a,c_{a,n}) = 0$	
		}
	\Explain{ 1.5.1 select a sequence $F_a : \Nat \to \F$
		such that $\rho_a(x,y) \le 2^{-n}$ for $x,y \in F_{a,n}$ and $n \in \Nat$}
	\Explain{ 1.5.2 Then select a sequence $c_{a,n} \in \bigcap^n_{k=1} F_{a,k}$}
	\Explain{ 1.5.3 Then $\rho(c_{a,n},c_{a,n+1}) \le 2^{-n}$}
	\Explain{ 1.5.4 Construct $d_a = \liminf ac_a$}
	\Explain{ 1.5.5 Then $\lim_{n \to \infty} \rho_a(d_a,c_{an}) = 
		\lim_{n \to \infty} \mu( d_a + ac_a  ) = 0$}
	\Explain{ 1.6 Assume $a,b \in A^f$ are such that $a \le b$}
	\Explain{ 1.7 Then  $d_a  = ad_b$}
	\Explain{ 1.7.1  $F_{n,a} \cap F_{n,b} \neq \emptyset$}
	\Explain{ 1.7.2 So select $f \in F_{n,a} \cap F_{n,b}$}
	\ExplainFurther{ 1.7.3 Then 
		$
			\rho_a(d_a,d_b) \le 
				\rho_a(d_a,c_{a,n}) + 
				\rho_a(c_{a,n}, f) +
				\rho_a(f, c_{b,n}) +
				\rho_a(c_{b,n},d_b) \le$}
	\Explain{$\quad\quad
				\le \rho_a(d_a,c_{a,n}) + 
				 2^{-n} +
				2^{-n} +
				\rho_a(c_{b,n},d_b) \to 0
		$ as $n \to \infty$}
	\Explain{ 1.8 Let $f = \bigvee_{a \in A^f} d_a$}
	\Explain{ 1.9 Then $\lim \F = f$}
	\Explain{ 1.9.1  $ ad_a = af $ for any $a \in A^f$ }
	\Explain{ 1.9.2 and there is a $\F$ subordinate Cauchy sequence $c_a$
		such that $\rho_a(f,c_a) = \rho_a(d_a,c_a) \to 0$}
	\Explain{ 1.9.3 Then there is $n \in \Nat$ such that $\rho_a(d_a,c_{a,n}) + 2^{-n} < \varepsilon$}
	\Explain{ 1.9.4 Take any $g \in F_{a,n}$}	
	\Explain{ 1.9.5 But $\rho_a(f,g) \le \rho_a(f,c_{a,n}) + \rho_{c_{a,n}} \le 
			 \rho_a(d_a,c_{a,n}) + 2^{-n} < \varepsilon$}
	\Explain{ 1.9.6 This $F_{a,n} \subset \Cell_{\rho_a}(f,\varepsilon)$ }
	\Explain{ 2 $(\Leftarrow)$ now Assume that $A$ is Hausdorff and complete}
	\Explain{ 2.1 As $A$ is Hausdorff $(A,\mu)$ must be semifinite}
	\Explain{ 2.2 As $A$ is complete $(A,\mu)$ is order complete and hence localizable}
	\Explain{ 2.2.1 Think about order filters $\F(\uparrow D)$ and $\F(\downarrow D)$}
	\EndProof
	\\
	\Theorem{LessRelationIsClosed}
	{
		\forall (A,\mu) : \Semifinite\MA \.
		\Closed\Big( A^2, \{ (a,b) \in A^2 : a \le b \} \Big) 
	}
	\Explain{ 1 As $(A,\mu)$ is a semifinite measure algebra $A$ must be Hausdorff}
	\Explain{ 2 So singleton $\{0\}$ is closed}
	\Explain{ 3 Then $\{ (a,b) \in A^2 : a \le b \} = (\setminus)^{-1}\{0\}$ is closed}
	\EndProof
}
\newpage
\subsubsection{Closed Subalgebras}
\Page{
	\Theorem{ClosedSubalgebraTHM}
	{
		\NewLine ::		
		\forall (A,\mu) : \Loc\MA \.
		\forall B \subset_{\mathsf{RING}} A \.
		\Closed(A,B) \iff \OC(A,B) 
	}
	\Explain{ 1 $(\Rightarrow)$ follows from the general theory}
	\Explain{ 2 $(\Leftarrow)$ Assume now that $B$ is order-closed}
	\Explain{ 2.1 Assume $g \in {\cl}_A B$}
	\Explain{ 2.2 Assume $a \in A^f$ and $n \in \Nat$} 
	\Explain{ 2.3 Then there exists a sequence $c_a : \Nat \to B$ 
		such that $\rho_a(c_{a,n},g) < 2^{-n}$}
	\Explain{ 2.4 Note, $\sum^\infty_{n=1} \mu(ac_{a,n} + ac_{a,n + 1}) 
		\le \sum^\infty_{n=1} \mu(ac_{a,n} +ag) + \mu(ag + ac_{a,n + 1}) < 
		 \sum^\infty_{n=1} 2^{-n} + 2^{-n-1} = \frac{3}{2} $ }
	\Explain{ 2.5 So, sequence $ac_a$ has summable increments  }
	\Explain{ 2.6 Define $d_a = \liminf c_{a}$}
	\Explain{ 2.7 As $ac_a$ has finite increments $\lim_{n \to \infty} \rho_a(c_{a,n},d_n) = 0$}
	\Explain{ 2.8 Furthermore, $\rho_a(d_a,g) = 0$, so $ag = d_a$}
	\Explain{ 2.9 As $B$ is order-closed $d_a \in B$ for each $a \in A^f$}
	\Explain{ 2.10 Set $d'_a = \inf \{ d_b : b \in A^f, a \le b  \} \in B$}
	\Explain{ 2.11  $d'_a a  = \bigwedge_{a \le b} d_b a = \bigwedge_{a \le b} d_b b a  = 
		  \bigwedge_{a \le b}  g b a = g a$}
	\Explain{ 2.12 Let $D = \{ d_a' | a \in A \} $}
	\Explain{ 2.13 Clearly $D$ is upwards directed as $d'_a \vee d'_b = d'_{a \wedge b}$}
	\Explain{ 2.14 Then  $\sup D = \{ ad_a' | a \in A \} = \{ ag | a \in A \} = g$
		as $(A,\mu)$ is semifinite}
	\Explain{ 2.15 so $g \in B$ as $B$ is order-closed}
	\Explain{ 2.16 Thus $B$ is closed}
	\EndProof
	\\
	\Theorem{SubalgebraClosure}
	{
		\forall (A,\mu) : \Loc\MA \.
		\forall B \subset_{\mathsf{RING}} A \.
		\overline{B} =  \tau(B)
	}
	\Explain{ 1 Note that $\overline{B}$ is a subgroup of $A$}
	\Explain{ 2 Also it must be order-closed as $\overline{B}$ is closed}
	\Explain{ 3 Also $\tau(B)$ is an order-closed subalgebra, and hence a closed subalgebra}
	\ExplainFurther{ 4 So both objects can be realized as intersections 
		of closed subalgebras containing $B$,}
	\Explain{ $\quad\quad$	and hence they are equal}
	\EndProof      
	\\
	\DeclareType{ClosedMeasureSubalgebra}{\prod (A,\mu) : \MA \. \TYPE{Subalgebra}(A)}
	\DefineNamedType{B}{ClosedMeasureSubalgebra}{B \subset_{\ma} A}
	{\Closed(A,B)}
}\Page{
	\Theorem{OrderClosedExtension}
	{
		\NewLine ::		
		\forall (A,\mu) : \Loc\MA \.
		\forall B \subset_{\ma} A \.
		\forall a \in A \. 
		\langle B \cup \{a\} \rangle_{\BOOL} \subset_{\ma} A
	}
	\Explain{ This follows from order-closed subalgebra extension theorem for boolean algebras}
	\EndProof
	\\
	\Theorem{SigmaFiniteSigmaSubalgebraIsClosed}
	{
		\forall (X,\Sigma,\mu) : \sFinite \.
		\forall T \subset_\sigma \Sigma \.
		\pi_\mu(T) \subset_{\ma} A_\mu
	}
	\Explain{ 1 As $(X,\Sigma,\mu)$ is $\sigma$-finite $A_\mu$ is also $\sigma$-finite}
	\Explain{ 2 So $A_mu$ is actually metrizable with a metric $\sigma$}
	\Explain{ 3 In a metric space set is closed iff it is sequence-closed}
	\Explain{ 4 Consider a sequence $a : \Nat \to \pi_\mu(T)$ with a limit $x$}
	\Explain{ 5 Then there is a sequence $E : \Nat \to T$ such that $a = [E]$}
	\Explain{ 7 Then $\lim \sup E = \lim \inf E \in T$, but also $[\lim \sup E] = x$ }
	\Explain{ 8 Thus $x \in \pi_\mu(T)$}
	\EndProof
	\\
	\Theorem{SigmaFiniteSigmaSubalgebraIsClosed2}
	{
		\forall (X,\Sigma,\mu) : \sFinite \.
		\forall B \subset_{\ma} A_\mu \.
		\pi_\mu^{-1}(B) \subset_\sigma A_\mu
	}
	\Explain{Inverse argument}
	\EndProof
	\\
	\Theorem{OrderClosedSetsAreClosedInLocalizableAlgebra}
	{
		\NewLine :: 		
		\forall (A,\mu) : \Loc\MA \.
		\forall  C : \OC(A) \.
		\Closed(A,C)
	}
	\Explain{ 1 Same proof as with closed algebras}
	\EndProof
	\\
	\Theorem{SubalgebraClosureIsSubalgebra}
	{
		\NewLine :: 		
		\forall (A,\mu) : \MA \.
		\forall  B \subset_{\mathsf{RING}} A \.
		\overline{B} \subset_{\mathsf{RING}} A
	}
	\Explain{ 1 $B$ is a topological subgroup of $A$}
	\Explain{ 2 So by general theory of topological groups $\overline{B}$ is a subgroup of $A$ again}
	\Explain{ 3 So $\overline{B}$ is closed under operation $(+)$}
	\Explain{ 4 Also $\overline{B}$ is closed and hence order-closed}
	\Explain{ 5 But then it is closed under operations $(\vee),(\wedge)$}
	\Explain{ 6 And being closed under operations $(\vee),(\wedge),(+)$ is enough to be a
		boolean algebra}
	\EndProof
}
\newpage
\subsubsection{Metric Space of Finite Elements}
\Page{
	\Theorem{BooleanOperationsAreUniformlyContinuous}
	{
		\NewLine ::		
		\forall (A,\mu) : \MA \.  (*),(\setminus),(\vee),(\wedge) \in \UNI(A^f \times A^f, A^f)
	}
	\Explain{ This is obvious}
	\EndProof
	\\
	\Theorem{MeasureIs1Lip}
	{
		\NewLine ::		
		\forall (A,\mu) : \MA \.  \mu_{|A^f} \in \Lip{1}(A^f) 
	}
	\Explain{ This is obvious}
	\EndProof
	\\
	\Theorem{FiniteElementsAreComplete}
	{
		\NewLine ::		
		\forall (A,\mu) : \MA \.  \Complete(A^f) 
	}
	\Explain{ 1 Assume $a$ is a cauchy sequence in $A^f$}
	\Explain{ 2 without loss of generality we may assume that $a$ has summable differences }
	\Explain{ 2.1 Just select a subsequence}
	\Explain{ 3 Define $x = \lim \inf a \in A$}
	\Explain{ 4 Then $\lim_{n \to \infty} a_n = x $}
	\Explain{ 5 So, there is some $n \in \Nat$ such that  $\mu(x \setminus a_n) < \infty$ }
	\Explain{ 6 Thus $\mu(x) < \infty$ and $x \in A^f$ }
	\EndProof
}
\newpage
\subsubsection{Relation with Convergence In Measure}
\Page{
	\DeclareFunc{indicatorFunctionRepresentation}
	{
		\prod (X,\Sigma,\mu) \in \MEAS \. A_\mu \to \Lp{0}(X,\Sigma,\mu)
	}
	\DefineNamedFunc{indicatorFunctionRepresentation}{a}{\chi_a}
	{
		[ \chi_E  ] \quad \where \quad a = [E]
	}
	\\
	\Explain{1 This is well defined}
	\Explain{2 Assume that $a = [E] = [F]$ for some $E,F \in \Sigma$}
	\Explain{3 Then $\mu(E \du F) = 0$}
	\Explain{4 Hence, $\chi_E = _\mu \chi_F$ and $[\chi_E] = [\chi_F]$}
	\EndProof
	\\
	\Theorem{IndicatorFunctionRepresentationIsHomeo}
	{
		\NewLine ::		
		\forall (X,\Sigma,\mu) \in \MEAS \.
		\TYPE{Homeomorphism}\Big(A_\mu,\chi_{A_\mu},\chi_\bullet\Big)
	}
	\Explain{1 Here we always assume that $\Lp{0}(X,\Sigma,\mu)$ is equiped with 
		a convergenve in measure topology}
	\Explain{2  Clearly $\chi_\bullet$ is injective}
	\Explain{2.1 Assume $\chi_a = \chi_b$}
	\Explain{2.2 Then there is common representative $E \in \Sigma$ such that $a = [E] = b$}
	\Explain{3 Also $\chi_\bullet$ is trivially sirjective}
	\Explain{4 $\chi_\bullet$ is homeomorphism}
	\ExplainFurther{4.1 This can be seen by direct corespondence between semimetrics
		$\rho_a$} 
	\Explain{4.2 and $\rho_E = \inf_{t \in \Reals_{++}} t + \mu\Big\{ x \in E : |f(x) - g(x)| > t  \Big\}$}
	\Explain{ 4.3 where corespondence is between finite $a \in A^f_\mu$ and $E \in \Sigma^f$ such that 
		$a = [E]$}
	\EndProof
	\\
	\Theorem{FiniteIndicatorEmbeddingL1Isometri}
	{	
		\NewLine ::
		\forall (X,\Sigma,\mu) \in \MEAS \.
		\TYPE{Isometry}\Big(A_\mu,\chi_{A_\mu},\chi_\bullet\Big)
	}
	\Explain{ This is obvious as difference of indicators are measure of difference of sets}
	\EndProof
}
\newpage
\subsubsection{Localization}
\Page{
	\Theorem{LocalizationIsCompletion}
	{
		\forall (A,\mu) : \Semifinite\MA \.
		\TYPE{Completion}(A,\tau(A),\iota_\tau) 
	}
	\Explain{1 $\iota_\tau(A)$ is order dense in $\tau(A)$}
	\Explain{2 So its order-closure is $\tau(A)$}
	\Explain{3 $\tau(A)$ is localizable and $\iota_\tau(A)$ is a subalgebra, 
		so the closure of $\iota_\tau(A)$ is equal to the order closure}
	\EndProof
}
\newpage
\subsubsection{Metric Space of Pobability Subalgebras}
\Page{
	\DeclareFunc{metricSpaceOfProbabilitySubalgebra}
	{
		\TYPE{ProbabilityAlgebra} \to \Complete\TYPE{MetricSpace}
	}
	\DefineNamedFunc{metricSpaceOfProbabilitySubalgebra}{A,\pi}
	{ \mathsf{FB}(A,\pi)}
	{
		\NewLine \de
		\left(  
			\Closed \And \TYPE{Subring}(A),
			\Lambda B,C \subset_{\ma} A \.
			\max\Big( \sup_{b \in B} \inf_{c \in C} \rho_\pi(b,c),
			 \sup_{c \in C} \inf_{b \in B} \rho_\pi(b,c) \Big)
		\right)	
	}
	\ExplainFurther{ 1 Note, that indicator representation maps such closed subalgebras into  
		closed uniformly integrable } 
	\Explain{$\quad\quad$ subsets of $\Lp{1}(\Z\;A,\Sigma_\pi,\bar \pi)$}
	\ExplainFurther{ 2 Then there is a natural isometric inclusion 
		$\chi\Big(\mathbf{FB}(A,\pi)\Big) \subset \mathbf{F}\Big( \Lp{1}(\Z\;A,\Sigma_\pi,\bar \pi) \Big)$,}
	\Explain{ $\quad\quad$  which can be equiped with a Hausdorff metric $d$}
	\Explain{ 3 Now consider an boolean binary operation $\circ$}
	\Explain{ 4 Assume $C : \Nat \to \mathsf{FB}(A,\pi)$  is a converging sequence with a limit $L$}
	\Explain{ 5 Then clearly $e,0 \in L$ as $e,0 \in C_n$ for every $n \in \Nat$}
	\Explain{ 6 Now assume $x,y \in L$}
	\Explain{ 7 Then there exists a sequences $u,v : \prod^\infty_{n=1} C_n$
		such that $x = \lim_{n \to \infty} u_n$ and $y = \lim_{n \to \infty}  v_n$}
	\Explain{ 8 But Then $u_n \circ v_n \in C_n$ and $x \circ y = 
		\lim_{n \to \infty} u_n \circ \lim_{n \to \infty} v_n = \lim_{n \to \infty} u_n \circ v_n \in L$  }
	\Explain{ 9 So $L \in \mathsf{FB}(A,\pi)$}
	\Explain{ 10 As $C$ and $L$ were arbitraty $\chi\Big(\mathbf{FB}(A,\pi)\Big)$
		must be a closed subset of $\mathsf{F}\Big(\Lp{1}(\Z\;A,\Sigma_\pi,\bar \pi)\Big)$}	
	\Explain{ But $\mathsf{F}\Big(\Lp{1}(\Z\;A,\Sigma_\pi,\bar \pi)\Big)$ 
		as complete $\Lp{1}(\Z\;A,\Sigma_\pi,\bar \pi)$ is complete, so 
		$\mathsf{FB}(A,\pi)$ is complete}
	\EndProof
}
\newpage
\subsubsection{Topology of the Lebesgue Algebra}
\Page{
	\DeclareFunc{algebraOfLebesgue}{\sFinite}
	\DefineNamedFunc{algebraOfLebesgue}{}{\Lambda}{\B(\Reals)_\lambda}
	\\
	\Theorem{LebesgueAlgebraIsSeparable}
	{
		\TYPE{Separable}(\Lambda)
	}
	\Explain{1 consider $\A$ to be an algebra generated by open intervals with rational endpoints}
	\Explain{2 Then $|\A| = \aleph_0$ as $\Rats$ are countable}
	\Explain{3 As $\Lambda$ is localizable 
		$\Lambda = \pi_\lambda\Big( \B(\Reals)\Big) = \pi_\lambda\Big( \tau_{\B(\Reals)}(\A) \Big) 
		= \tau\Big( \pi_\lambda(\A) \Big) = \overline{\pi_\lambda(\A)}$}
	\Explain{4 So $\Lambda$ is separable}
	\EndProof
}
\newpage
\subsection{Category}
\subsubsection{Measure Algebra Functor}
\Page{
	\Theorem{NullIdealPreservingMapToHomomorphism}
	{
		\NewLine ::		
		\forall (X,\Sigma,\mu), (Y,T,\nu) \in \MEAS \.
		\forall D : \Thick(X,\Sigma,\mu) \. 
		\forall f : D \to Y \. \NewLine \.
		\forall \aleph : \forall E \in T \.  f^{-1}(E) \in (\hat \Sigma | D) \.  
		\forall \beth :  \forall E \in \Null_\nu \cap T \, f^{-1}(E) \in \Null_\mu \. \NewLine \.
		\exists \phi \in \ma \And \SOC(A_\nu,A_mu) \. 
		\forall E \in  T \. \forall F \in \Sigma  \.    \NewLine \.
		\phi[E] = [F] \iff   f^{-1}(F) \du (E \cap D) \in \Null_\mu 
	}
	\Explain{ 1  Define $\phi[E] = \Big[f^{-1}(E)\Big]$}
	\Explain{ 2  $\phi$ is well defined}
	\Explain{ 2.1 Assume $E,F \in T$ are such that $[E] = [F]$}
	\Explain{ 2.2 Then $\nu(E \du F) = 0$}
	\Explain{ 2.3 So $\mu\Big( f^{-1}(E \du F) \Big) = 0$}
	\Explain{ 2.4 Write 
		$\phi[E] = \Big[f^{-1}(E)\Big] = \Big[f^{-1}(E \du F \du F)\Big] = 
		\Big[f^{-1}(F)\Big] + \Big[ f^{-1}(E \du F)\Big] = 
		\Big[f^{-1}(F)\Big] = \phi[F]$ 
	}
	\Explain{3 $\phi$ is a boolean morphism}
	\Explain{3.1 $\phi(1) = \Big[ f^{-1}(X)\Big] = \Big[ f^{-1}(Y)\Big] = 1$}
	\Explain{3.2 The rest is obvious from properties of $f^{-1}:2^Y \to 2^D$}
	\Explain{3.3 As measures are $\sigma$-additive the $\sigma$-continuity follows by simmilar arguments}
	\Explain{4 The final property is also obvious by construction}
	\EndProof
	\\	
	\DeclareFunc{measureAlgebraFunctor}
	{
		\Contra(\BOR_0,\MA)
	}
	\DefineNamedFunc{measureAlgebraFunctor}{(X,\Sigma,\mu)}
	{\ma(X,\Sigma,\mu)}{(A_\mu,\hat\mu)}
	\DefineNamedFunc{measureAlgebraFunctor}{X,Y,f}
	{\ma_{X,Y}(f)}{\THM{NullIdealPreservingMapToHomomorphism}}
}
\newpage
\subsubsection{Stone Space Functor}
\Page{
	\DeclareFunc{spaceOfStoneFunctor}
	{
		\Contra(\MA, \BOR_0)
	}
	\DefineNamedFunc{spaceOfStoneFunctor}{(A,\mu)}
	{\Z(A,\mu))}{(\Z A, \Sigma_\mu, \bar \mu)}
	\DefineNamedFunc{spaceOfStoneFunctor}{X,Y,f}
	{\Z_{X,Y}(f)}{\Z_{X,Y}(f)}
	\Explain{1 Assume $E$ is  nowhere dense in $\Z X$}
	\Explain{1.2 Then $\Big(\Z_{X,Y}(f)\Big)^{-1}(E)$ is nowhere dense in $\Z Y$}
	\Explain{1.3 But this means that $\Big(\Z_{X,Y}(f)\Big)^{-1}(E)$ is meager and has measure zero}
	\Explain{2 Now assume $E$ has $\bar \mu$-measure zero}
	\Explain{2.1 Then $E$ must be meager}
	\Explain{2.2 So write $E = \bigcap^\infty_{n=1} N_n$, where each $N$ is nowhere dense}
	\Explain{2.3 By elementary set theory 
		$\Big(\Z_{X,Y}(f)\Big)^{-1}(E) = \bigcup^\infty_{n=1}\Big(\Z_{X,Y}(f)\Big)^{-1}(N_n)$}
	\Explain{2.3 As each $\Big(\Z_{X,Y}(f)\Big)^{-1}(N_n)$ has measure $0$, 
		$\Big(\Z_{X,Y}(f)\Big)^{-1}(E)$ also has measure $0$}
	\EndProof
}
\newpage
\subsubsection{Order Continuous Homomorphism}
\Page{
	\Theorem{OrderContinuousByCodomain}
	{
		\NewLine ::		
		\forall (A,\mu) \in \ma \.
		\forall (B,\nu)  : \Semifinite\MA \.
		\forall \phi \in \ma\Big((A,\mu),(B,\nu)\Big) \. \NewLine \.
		\phi \in \TOP(A,B) \Imply \oC(A,B,\phi)
	}
	\Explain{1 Assume $D$ is downwards directed subset of $A$ such that $\inf D = 0$}
	\Explain{2 Then $0 \in \overline{D}$}
	\Explain{3 As $\phi$ is continuous $0 \in \overline{\phi(D)}$}
	\Explain{4 $\inf \phi(D) = 0$}
	\Explain{4.1 Assume $\inf \phi(D) = b > 0$}
	\Explain{4.2 As $\nu$ is semifinite, where is $c \in B^f$ such that $c \le b$ 
		and $\nu(b) > 0$}
	\Explain{4.3 Then $\rho_c(\phi(a),0) = \nu(\phi(a)c) = \nu(c) > 0$ for any $a \in D$}
	\Exclaim{4.4 So $0 \not \in \overline{\phi(D)}$, a contradiction}
	\Explain{5 Then $\phi$ must be order-continuous}
	\EndProof
}\Page{
	\Theorem{ContinuousByDomain}
	{
		\NewLine ::		
		\forall (A,\mu) : \Semifinite\MA \.
		\forall (B,\nu)  \in \ma \.
		\forall \phi \in \ma\Big((A,\mu),(B,\nu)\Big) \. \NewLine \.
		\oC(A,B,\phi) \Imply \phi \in \TOP(A,B)
	}
	\Explain{1 It is enough to prove that $\phi$ is continuous at zero}
	\Explain{2 Assume $b \in B^f$ and $\varepsilon \in \Reals_{++}$}
	\Explain{2.1 Assume that for any $a \in A^f$ and $\delta \in \Reals_{++}$
		where is some $c \in A$ such that $\rho_a(c,0) < \delta$ but $\rho_b(\phi(c),0) \ge \varepsilon$}
	\ExplainFurther{2.1.1  Then it is possible to construct a system of elements $c : A^f \times \Nat \to A$}
	\Explain{ $\quad \quad$ such that $\rho_a(c_{a,n},0) < 2^{-n}$ and 
		$\rho_b(\phi(c_{a,n}),0) \ge \varepsilon$}
	\Explain{ 2.1.2  Set $d_a = \liminf c_a$}
	\Explain{2.1.3 Then $\rho_a(d_a,0) = 0$}
	\Explain{2.1.4 Thus, $d_a a = 0$}
	\Explain{2.1.5 As $\phi$ is order continuous $\phi(d_a) = \limsup \phi(c_a)$}
	\Explain{2.1.6 So, $\rho_b(\phi(d_a),0) \ge \varepsilon$ }
	\Explain{2.1.7 This implies that $\rho_b(\phi(\bar a),0) \ge \varepsilon$}
	\Explain{2.1.8 Now consider set $D = \{ \bar a | a \in A^f \}$}
	\Explain{2.1.8.1 Then $D$ is downwards directed}
	\Explain{2.1.8.1.1 If $c,d \in A^f$ then $c \vee d \in A^f$ also}
	\Explain{2.1.8.1.2 So by De Muavre  law if $\bar c, \bar d \in D$, then 
	$\bar a \wedge \bar b = \overline{a \vee b} \in D$}
	\Explain{2.1.8.2 As $\mu$ is semifinite $\inf D = 0$}
	\Explain{2.1.8.2.1 There is dense subset consisting of elements of $A^f$}
	\Explain{2.1.9 So $0 \in \overline{D}$}
	\Exclaim{2.1.10 But (2.1.9) is in contradiction with (2.1.7)}
	\Explain{2.2 So we showed that there is always some $\delta$ and $a \in A^f$
		such that $\rho_b(\phi(c),0) < \varepsilon$ for any $c \in \Cell_{a}(0,\delta)$}
	\Explain{3 But as $b$ and $\varepsilon$ were arbitrary, the homomorphism $\phi$ 
		must be continuous}
	\EndProof
	\\
	\Theorem{ContinuoutyEquivalence}
	{
		\NewLine ::		
		\forall (A,\mu), (B,\nu) : \Semifinite\MA \.
		\forall \phi \in \ma\Big((A,\mu),(B,\nu)\Big) \. \NewLine \.
		\oC(A,B,\phi) \iff \phi \in \TOP(A,B)
	}
	\Explain{Combine two previous results}
	\EndProof
	\\
	\Theorem{UniformEquivalencse}
	{
		\NewLine ::		
		\forall A \in \BOOL \.
		\forall \mu,\nu : \Semifinite\MA(A) \.
		\U_\nu = \U_\mu
	}
	\Explain{1 Identity mapping is always order-continuous}
	\Explain{2 But by previous theorem it must be a homeomorphism}
	\Explain{3 A homomorphism whis is also a homeomorphism must be a unimorphism}
	\EndProof
}
\newpage
\subsubsection{Measure Preserving Homomorphism}
\Page{
	\DeclareType{\MPH}{\prod (A,\mu),(B,\nu) \in \ma \. ?\BOOL(A,B)}
	\DefineType{\phi}{\MPH}{\forall a \in A \. \mu(a) = \nu\Big( \phi(a) \Big)}
	\\
	\DeclareFunc{measurePreservingMeasureAlgebraCategory}{\mathsf{LSCAT}}
	\DefineNamedFunc{measurePreservingMeasureAlgebraCategory}{}{\ma_\#}
	{\Big( \ma, \MPH, \circ, \id\Big)}
	\\
	\Theorem{MPHIsInjective}
	{
		\forall (A,\mu),(B,\nu) \in \ma \.
		\forall \phi \in \ma_\#\Big( (A,\mu), (B,\nu) \Big) \.
		\forall \phi \in \Inj(A,B)
	}
	\Explain{ 1 If $\phi$ is not injective then it has nontrivial kernel}
	\Explain{ 2 Select $a \in \ker \phi$ such that $a \neq 0$}
	\Exclaim{ 3 Then $\mu(a) > 0$ but $\nu(\phi(a)) = \nu(0) = 0$, a contradiciton}
	\EndProof
	\\
	\Theorem{MPHFiniteness}
	{
		\NewLine ::		
		\forall (A,\mu),(B,\nu) \in \ma \.
		\forall \phi \in \ma_\#\Big( (A,\mu), (B,\nu) \Big) \. \NewLine \.
		\Finite\MA(A,\mu) \iff \Finite\MA(B,\nu)
	}
	\Explain{ 1 For any boolean homomorphism $\phi(e_A) = e_B$}
	\Explain{ 2 So finiteness follows by measure preservation}
	\EndProof
	\\  
	\\
	\Theorem{FiniteMPHIsContinuous}
	{
		\NewLine ::		
		\forall (A,\mu),(B,\nu) : \Finite\MA \.
		\forall \phi \in \ma_\#\Big( (A,\mu), (B,\nu) \Big) \. \phi \in \TOP(A,B)
	}
	\Explain{ $\phi$ is an isometry with respect to natural metrics $\rho_\mu$ and $\rho_\mu$}
	\EndProof
	\\
	\Theorem{FiniteMPHIsOrderContinuous}
	{
		\NewLine ::		
		\forall (A,\mu),(B,\nu) : \Finite\MA \.
		\forall \phi \in \ma_\#\Big( (A,\mu), (B,\nu) \Big) \. \NewLine \.
		\oC(A,B,\phi)
	}
	\Explain{This follows from the previous chapter and previous theorem}
	\EndProof
}\Page{
	\Theorem{SigmaFiniteMPH1}
	{
		\forall (A,\mu) : \Semifinite\MA \.
		\forall (B,\nu) : \sFinite\MA \. \NewLine \.
		\forall \phi \in \ma_\#\Big( (A,\mu), (B,\nu) \Big) \. 
		\sFinite(A,\mu)
	}
	\Explain{ 1 As $\mu$ is semifenit there is a partition of unity of finite elements $D$}
	\Explain{ 2 $|\phi(D)| = |D|$ as $\phi$ is injective}
	\Explain{ 3 $\phi(D)$ is disjoint}
	\Explain{ 3 As $\nu$ is $\sigma$-finite $\phi(D)$ can be embeded into 
		a countable partition of unity, so $|D| \le \aleph_0$ }
	\Explain{ 4 This means that $\mu$ is $\sigma$-finite}
	\EndProof 
	\\
	\Theorem{SigmaFiniteMPH2}
	{
		\forall (A,\mu) : \sFinite\MA \.
		\forall (B,\nu)  \in \ma \. \NewLine \.
		\forall \phi \in \ma_\#\Big( (A,\mu), (B,\nu) \Big) \And \sC(A,B) \. 
		\sFinite\MA(B,\nu)
	}
	\Explain{ 1 There is countable partition of unity $P$ consisting of finite measure elements in $A$}
	\Explain{ 2 Then $\phi(P)$ is a countable disjoint subset of $B$ consisting of finite measure elements}
	\Explain{ 3 But $\sup \phi(P) = \phi(\sup P) = \phi(e_A) = e_B$}
	\Explain{ 4 Thus, $\phi(P)$ is also a countable partition of unity in $(B,\nu)$
		consisting of finite measure elements}
	\Explain{ 5 So $(B,\nu)$ is $\sigma$-finite}
	\EndProof
	\\
	\Theorem{SemifiniteMPH}
	{
		\forall (A,\mu) : \Semifinite\MA \.
		\forall (B,\nu)  \in \ma \. \NewLine \.
		\forall \phi \in \ma_\#\Big( (A,\mu), (B,\nu) \Big) \And \oC(A,B) \. 
		\Semifinite\MA(B,\nu)
	}
	\Explain{ 1 There is  partition of unity $P$ consisting of finite measure elements in $A$}
	\Explain{ 2 Then $\phi(P)$ is a disjoint subset of $B$ consisting of finite measure elements}
	\Explain{ 3 But $\sup \phi(P) = \phi(\sup P) = \phi(e_A) = e_B$}
	\Explain{ 4 Thus, $\phi(P)$ is also a partition of unity in $(B,\nu)$
		consisting of finite measure elements}
	\Explain{ 5 So $(B,\nu)$ is semifinite}
	\EndProof
	\\
	\Theorem{AtomlessMPH}
	{
		\forall (A,\mu) : \Semifinite\MA \And \Aless \.
		\forall (B,\nu)  \in \ma \. \NewLine \.
		\forall \phi \in \ma_\#\Big( (A,\mu), (B,\nu) \Big) \And \oC(A,B) \. 
		\Aless(B)
	}
	\Explain{ 1 There is  partition of unity $P$ consisting of finite measure elements in $A$}
	\Explain{ 2 Then $\phi(P)$ is a disjoint subset of $B$ consisting of finite measure elements}
	\Explain{ 3 But $\sup \phi(P) = \phi(\sup P) = \phi(e_A) = e_B$}
	\Explain{ 4 Thus, $\phi(P)$ is also a partition of unity in $(B,\nu)$
		consisting of finite measure elements}
	\Explain{ 5 Now assume $b$ is an atom in $B$}
	\Explain{ 5.1 Then there is an element $a \in P$ such that $\phi(a)b \neq 0$}
	\Explain{ 5.2 But as $b$ is an atom this means that $b = \phi(a)$}
	\Explain{ 5.3 $A$ is atomless so there are some $c$ such that $0 < c < a$}
	\Explain{ 5.4 So $0 < \phi(c) < \phi(a) = b$}
	\Exclaim{ 5.5 But this means that $b$ is not an atom, a contradiction!}
	\EndProof
}\Page{
	\Theorem{PurelyAtomicMPH}{
		\NewLine ::
		\forall (A,\mu) : \Semifinite\MA \.
		\forall (B,\mu) : \PA\MA \. \NewLine \.
		\forall \phi \in \ma_\#\Big( (A,\mu), (B,\nu) \Big) \.
		\PA(A) 
	}
	\Explain{ 1 Assume $a \in A$ is such that $a \neq 0$}
	\Explain{ 1.1 Assume that $a$ do not contain any atoms}
	\Explain{ 1.2 As $A$ is semifinite there is a $c \in A^f$ such that $0 < c \le a$}
	\ExplainFurther{ 1.3 Then there exist a sequnce of partitions $d : \Bool^* \to A^f$ such that}
	\ExplainFurther{$\quad \quad$ such that $c = \bigvee^{2^n}_{t \in \Bool^n} d_t$ and 
		$d_t \neq 0$ for any $t \in \Bool^*$and $d_t d_s = d_s$ iff $t \sqsubset s$ 
		and $d_t d_s = 0$ iff $|s|=|t|$ and $t \neq s$}
	\Explain{ $\quad \quad$ and $\mu(d_t) \to 0$ as $|t| \to \infty$ }
	\Explain{ 1.3 Then $\phi(d)$ has all same properties }
	\Explain{ 1.4 Moreover $\nu(\phi(c)) = \nu\phi\left(  \bigvee^{2^n}_{t \in \Bool^n} d_t \right)=
	\sum^{2^n}_{t \in \Bool^n} \nu(\phi(d_t))$}
	\Explain{ 1.5 So $\phi(c) = \bigvee^{2^n}_{t \in \Bool^n} \phi(d_t)$ as $\phi(d_t)$ must be disjoint}
	\Explain{ 1.6 So $\phi(c)$ can't contain atoms}
	\Exclaim{ 1.7 But $B$ is purely atomic, so we have a contradiction}
	\EndProof
	\\
	\Theorem{GeneratedSigmaSubalgebraImage}
	{
		\NewLine  ::	
		\forall (A,\mu),(B,\nu) : \Finite\MA \.
		\forall \phi \in \ma_\#\Big( (A,\mu), (B,\nu) \Big) \.
		\forall C \subset A \.
		\phi\langle C \rangle_\sigma = \langle \phi(C) \rangle_\sigma
	}
	\Explain{This follows from previous theorems about finite measure algebras}
	\EndProof
	\\
	\Theorem{MeasurePreservingMeasureAlgebra}
	{
		\NewLine  ::	
		\forall (X,\Sigma,\mu),(Y,T,\nu)  \in \MEAS \.
		\forall  f \in \MEAS^\#\Big((X,\Sigma,\mu),(Y,T,\nu)\Big) \.
		\ma_{X,Y}(f) \in \ma_\#(T_\nu,\Sigma_\mu) 
	}
	\Explain{This is obvious}
	\EndProof
	\\
	\Theorem{MeasurePreservingZeroSpace}
	{
		\NewLine  ::	
		\forall (A\mu),(B,\nu)  \in \MA \.
		\forall  f \in \ma_\#\Big((A,\mu),(B,\nu)\Big) \. \NewLine \.
		\Z_{A,B}(f) \in \MEAS^\#\Big(
			(\Z\;B,\Sigma_\nu,\bar \nu), 
			(\Z\;A,\Sigma_\mu,\bar \mu)
		\Big) 
	}
	\Explain{This is obvious}
	\EndProof
}\Page{
	\Theorem{MeasurePresevingHomomorphismExtensionFromSubalgebra}{
		\NewLine ::
		\forall (A,\mu),(B,\nu) : \Finite\MA \.
		\forall  C \subset_\BOOL A \.
		\forall \aleph : \Dense(A,C) \. \NewLine \.
		\forall \phi \in \MA_\#(C,B) \.
		\exists \Phi \in \MA_\#(A,B) \.
		\Phi_{|C} = \phi 	
	}
	\Explain{ 1 obviously $\phi$ is an isometry}
	\Explain{ 2 So there exists a uniqui iometry extrnsion $\Phi$ of $\phi$ by $\aleph$}
	\Explain{ 3 $\Phi$ is a homomorphism}
	\Explain{ 3.1 This holds as boolean operations are continuous and $\phi$ is also continuous}
	\Explain{ 3.2 Let $\circ$ be some binary boolean operation and $u,v \in A$}
	\Explain{ 3.3 Then there are sequences $x,y : \Nat \to C$ such that $u = \lim x$ and $v =\lim y$}
	\Explain{ 3.4 $\Phi(v) \circ \Phi(u) =  \lim_{n \to \infty} \phi(x_n) \circ \phi(y_n) = 
		\lim_{n \to \infty} \phi(x_n \circ y_n) = \Phi(v \circ u)$ }
	\Explain{4 $\Phi$ is measure preserving}
	\Explain{4.1 Assume $a \in A$}
	\Explain{4.2 just note $\nu(\Phi(a)) = \rho_\nu(\Phi(a),0) = \rho_\nu(\Phi(a),\Phi(0)) = 
		\rho_\mu(a,0) = \mu(a)$}
	\EndProof
	\\
	\Theorem{MeasurePresevingHomomorphismExtensionFromSubset}{
		\NewLine ::
		\forall (A,\mu),(B,\nu) : \Finite\MA \.
		\forall  C \subset  B \.
		\forall f : C \to A \. \NewLine \.
		\forall \aleph : \forall c : \Nat \to C \. \nu(\inf f(c)) = \mu(\inf c) \.
		\exists \Phi \in \ma_\#\Big(\langle C \rangle_\ma, B\Big) \.
		\Phi_{|C} = f 	
	}
	\NoProof
}
\newpage
\subsubsection{Example}
\Page{
	\Explain{ Let $A = 2^\Nat$ with $\mu = \#$}
	\Explain{ The elements of $A$ can be identified with sequences $\Nat \to \BOOL$}
	\Explain{ Let $\phi(a)$ be defined as right shift padded by $0$ if $a$ is finite}
	\Explain{ Let $\phi(a)$ be defined as right shift padded by $1$ if $a$ is cofinite}
	\Explain{ Otherwis let $\phi(a) = a$}
	\Explain{ Then as finite sets form an and $0 + 0 = 0$ and $0 \wedge t = 0$
		it is clear $\phi$ that preserves their structure}
	\ExplainFurther{ Also as cofinite sets are their complement and $ 1 + 1 = 0$ and $1 \wedge t = t$}
	\Explain{ $\quad \quad$  it is clear that $\phi$ is an algebra morphism}
	\Explain{ Clearly $\phi$ preserves cardinality}	
	\Explain{On the other hand consider a sequence $f_n = \{2,\ldots,2n\}$}
	\Explain{ Then $\bigvee^\infty_{n = 1} f_n = 2\Nat$}
	\Explain{ But $2\Nat = \phi(2\Nat) = \phi\left( \bigvee^\infty_{n = 1} f_n \right) \neq 
		\bigvee^\infty_n=1 \phi(f_n) = 2\Nat + 1$}
	\EndProof
}
\newpage
\subsubsection{Tensor Products}
\Page{
	\DeclareFunc{measureAlgebraTensorProduct}
	{
		\prod I : \Finite \. (I \to \ma) \to \ma
	}
	\DefineNamedFunc{measureAlgebraTensorProduct}{A,\mu}{
		\left(\bigotimes_{i \in I} A_i, \prod_{i \in I} \mu_i\right)}
		{
					\ma\left( \bigotimes_{i \in \I} \Z(A_i,\mu_i) \right)
		}
		\\
	\DeclareFunc{measureAlgebraTensorProductEmbedding}
	{
		\NewLine ::		
		\prod I : \Finite \. 
		\prod (A,\mu) : I \to \ma \.
		\prod_{i \in I} \oC\left( A_i , \bigotimes_{j \in I} A_j \right)
	} 
	\DefineNamedFunc{measureAlgebraTensorProductEmbedding}{}{\iota_i}
	{
		\ma_{\Z(A_i,\mu_i),\bigotimes_{i \in I} \Z(A_i,\mu_i)}( \pi_i   )
	}
	\Explain{ 1 $\iota_i$ is well defined}
	\Explain{ 1.1 Assume $E \in  \sigma_{\mu_i}$ 
		is such that $\bar \mu_i(E) = 0$}
	\Explain{ 1.2 Then   
		$\bigotimes_{j \in I} \bar \mu_j \Big(\pi_i^{-1}(E) \Big) = 
		   \bigotimes_{j \in I} \bar \mu_j \prod_{k \in I} \Big(\widehat{\Z A_i}(E)\Big)_k =
			\sup  \left\{ \prod_{j \in I} \bar \mu_j(F) \bigg| 
				F : \prod_{j \in I} \Sigma_{\mu_i}, F_i \subset   E \right\} = 0		
		$}
	\Explain{ 1.3 So $\pi_i \in \BOR_0\left( \bigotimes_{j \in I} \Z(A_j,\mu_j) , \Z(A_i,\mu_i)\right)$}
	\Explain{ 2 $\iota_i$ is order-continuous}
	\Explain{ 2.1 Assume $D \subset A_i$ is downwards closed with $\inf D = 0$}
	\Explain{ 2.2 Also assume $0 \neq u = \inf \iota_i(D)$}
	\Explain{ 2.3 Then $\prod_{i \in I} \mu_i (u) > 0 $ }
	\Explain{ 2.4 By definition there is $E : \prod_{i \in I} \Sigma_{\bar \mu_i}^f$
			 and $F \in \bigotimes_{j \in I} \Sigma_{\bar \mu_j}$ such that $u = [F]$	
			 and $\bigotimes_{i \in I} \bar \mu_i \left( F \cap \prod_{j \in I} E_j \right) > 0$}
	\Explain{ 2.5 But $\inf_{d \in D} d [E_i]  = 0$, so $\inf_{d \in D} \mu_i\Big(d[E_i]\Big) = 0$}
	\Explain{ 2.6 So there exists $d \in D$ such that 
	$\mu_i\Big(d[E_i]\Big)\prod_{j \in \{i\}^\c} \bar \mu_j(E_j) 
	< \bigotimes_{i \in I}  \bar \mu_i \left( F \cap \prod_{j \in I} E_j \right)$}
	\Explain{ 2.7 Also there is $G \in \Sigma$ such that $d = [G]$}
	\Explain{ 2.8 Thus, $\bigotimes_{i \in I} \bar 
		\mu_i\left( F \setminus \prod_{j \in I} \Big(\widehat{E_i}(G)\Big)_j \right) = 0$}
	\ExplainFurther{
		2.9 Then
		$ 
		\bigotimes_{i \in I} \bar \mu_i \left( F \cap \prod_{j \in I} E_j \right) \le
		\bigotimes_{j \in J} \bar \mu_i\left(  \prod_{j \in I}\Big(\widehat{E_i}(G \cap E_i)\Big)_j \right) =
		\bar \mu_i( G \cap E_i) \prod_{j \in \{i\}^\c} \mu_j(E_j) = $}
	\Explain{	
		$\mu_i\Big( d [E_i]\Big )	\prod_{j \in \{i\}^\c} \mu_j(E_j) 
		$
	}
	\Exclaim{ 2.10 A contradiction with (2.5)}
	\EndProof
}\Page{
	\DeclareFunc{measureAlgebraTensorRepresentation}
	{
		\NewLine ::		
		\prod I : \Finite \. 
		\prod (A,\mu) : I \to \ma \.
		\BOOL \left( \bigotimes_{i \in I} A_i, \bigotimes_{i \in I} (A_i,\mu_i) \right)
	} 
	\DefineNamedFunc{measureAlgebraTensorRepresentation}{}{\Psi_{A,\mu}}
	{
		    \FUNC{tensor}\left( 
		    	\Lambda [E] \in \prod_{i \in I} A_i \. \left[ \prod_{i \in I} E_i \right]
		    \right)
	}
	\\
	\Theorem{TensorRepresentationsAreDense}
	{
		\forall I : \Finite \.
		\forall (A,\mu) : I \to \ma \.
		\Dense\left( \bigotimes_{i \in i} (A,\mu_i), \Psi_{A,\mu}\left( \bigotimes_{i \in I} A_i \right) \right)
	}
	\Explain{ 1 Assume $s \in \bigotimes_{i \in I} (A_i,\mu_i)$ and $f \in \left(\bigotimes_{i \in I} (A_i,\mu_i)\right)^f$ and $\varepsilon \in \Reals_{++}$}
	\Explain{ 2 Then there is $S,F \in \bigotimes_{i \in I} \Z(A_i,\mu_i)$
		such that $s = [E]$ and $f = [F]$}
	\Explain{ 3 We show that there is $t \in \Psi_{A,\mu}\left( \bigotimes_{i \in I} A_i \right)$
		such that $\rho_f(t,s) < \varepsilon$}
	\ExplainFurther{ 3.1 As $sf$ is finite there must exist a  natural number $n$ and a  
		system $E : \{1,\ldots,n\} \to \prod_{i \in I} \Sigma_\mu$}
	\Explain{
		such that $\bigotimes_{i \in I} \hat \mu_i 
		\left( S \cap F \du \bigcup^n_{k=1} \prod_{i \in I} E_i \right) < \varepsilon$}
	\Explain{ 3.2 But then $\rho_f\left(s, \bigvee^n_{k=1} 
		\Psi_{A,\mu}\left(\bigotimes_{i \in I} [E_i]\right)\right)  < \varepsilon$}
	\EndProof
	\\
	\Explain{Write just $\bigotimes_{i \in I} a_i$ for $\Psi_{A,\mu}\left(\bigotimes_{i \in I} a_i\right)$}
	\\
	\Theorem{TensorMeasureComputation}
	{
		\forall I : \Finite \.
		\forall (A,\mu) : I \to \ma \.
		\forall t \in \bigotimes_{i \in I} (A,\mu) \. \NewLine \.
		\prod_{i \in I} \mu_i (t) =
		\sup \left\{ \prod_{i \in I} \mu_i\left(t \bigotimes_{i \in I} a_i \right) 
		\bigg|
			a \in \prod_{i \in I} A_i^f 	
		\right\}
	}
	\Explain{ This follos by the definition of the cld product}
	\EndProof
}\Page{
	\Theorem{TensorRepresentationComputation}
	{
		\forall I : \Finite \.
		\forall (A,\mu) : I \to \Semifinite\MA \. \NewLine \.
		\forall a \in \prod_{i \in I} A_i \.
		\prod_{i \in I} \mu_i \left( \bigotimes_{i \in I} a_i \right) = \prod_{i \in I} \mu(a_i)
	}
	\Explain{ This is pretty obvious}
	\EndProof
	\\
	\Theorem{TensorRepresentationUniqueness}
	{
		\NewLine \.		
		\forall I : \Finite
		\forall (A,\mu) : I \to \Semifinite\MA \. \NewLine \.
		\Inj\left(  \bigotimes_{i \in I} A_i, \bigotimes_{i \in I} (A_i,\mu_i), \Psi_{A,\mu} \right)
	}
	\Explain{ This follows from the previous result}
	\EndProof
	\\
	\Theorem{MeasureSpaceCLDProductUniversalProperty}
	{
		\NewLine ::		
		\forall I : \Finite \.
		\forall (X,\Sigma,\mu) : I \to  \Semifinite \.
		\forall (A,\nu) : \Loc\MA \. \NewLine \.
		\forall \phi : \prod_{i \in I} \oC \And \BOOL\left( \ma(X_i,\Sigma_i,\mu_i),  A \right) \.
		\NewLine \.
		\forall \aleph : \forall x \in \prod_{i \in I} \ma(X_i,\Sigma_i,\mu_i) \.
		\nu\left( \bigwedge_{i \in I} \phi_i(x_i) \right) = \prod_{i \in I} \mu_i(x_i) \. \NewLine \.
		\exists! \psi : \MPH\left( \ma\left( \bigotimes_{i \in I} (X,\Sigma,\mu) \right), (A,\nu)\right) \.
		\psi\left( \bigotimes_{i \in I} x_i \right) = \bigwedge_{i \in I} \phi_i(x_i)
	}
	\NoProof
	\\
	\Theorem{LocalizableTensorProductUniversalProperty}
	{
		\NewLine ::
		\forall I : \Finite \.
		\forall (A,\mu) : I \to \Semifinite\MA \.
		\forall (B,\eta) :  \Loc\MA \. \NewLine \.
		\forall \phi : \prod_{i \in I} \oC \And \BOOL(A_i,B) \.
		\forall \aleph : \forall a : \prod_{i \in I}(A_i)  \. 
		\eta\left( \bigvee_{i \in I} \phi_i(a_i) \right) = \prod_{i \in I} \mu_i(a_i) \. \NewLine \.
		\exists! \psi : \MPH \And \oC\left(\bigotimes_{i \in I} (A,\mu_i), B\right) \.  \iota \psi = \phi
	}
	\NoProof
}
\newpage
\subsubsection{Independent Process Algebra}
\Page{
	\DeclareFunc{independentProcessAlgebra}
	{
		\prod_{I \in \SET}  (I \to \TYPE{ProbabilityAlgrbra} ) \to \TYPE{ProbabilityAlgebra}             
	}
	\DefineNamedFunc{randomProcessAlgebra}{A, p}{
		\left(\bigotimes_{i \in I} A_i, \prod_{i \in I} p_i\right)}
		{
					\ma\left( \bigotimes_{i \in \I} \Z(A_i, p_i) \right)
		}
		\\
	\DeclareFunc{independentAlgebraTensorProductEmbedding}
	{
		\NewLine ::		
		\prod_{I \in \SET}
		\prod (A,\mu) : I \to \ma \.
		\prod_{i \in I} \oC\left( A_i , \bigotimes_{j \in I} A_j \right)
	} 
	\DefineNamedFunc{independetAlgebraTensorProductEmbedding}{}{\iota_i}
	{
		\ma_{\Z(A_i,p_i),\bigotimes_{i \in I} \Z(A_i,p_i)}( \pi_i   )
	}
	\\
	\Theorem{independentProcessUniversalProperty}
	{
		\NewLine ::
		\forall I \in \SET \.
		\forall (A, p) : I \to \TYPE{ProbabilityAlgebra} \.
		\forall (B, q) :   \TYPE{ProbabilityAlgebra} \. \NewLine \.
		\forall \phi : \prod_{i \in I} \oC \And \BOOL\left(A_i, \bigotimes_{i \in I} (A_i,p_i) \right) \.
		\NewLine \. 
		\forall \aleph : \forall J : \Finite(I) \. \forall a : \prod_{j \in J}(A_j)  \. 
		\eta\left( \bigvee_{j \in J} \phi_j(a_j) \right) = \prod_{j \in J} \mu_j(a_j) \. \NewLine \.
		\exists! \psi : \MPH \And \oC\left(\bigotimes_{i \in I} (A_i,\mu_i), B\right) \.  \iota \psi = \phi
	}
	\NoProof
	\\
	\DeclareFunc{measureAlgebraTensorRepresentation}
	{
		\NewLine ::		
		\prod I \in \SET \. 
		\prod (A,p) : I \to \TYPE{ProbabilityAlgebra} \.
		\BOOL \left( \bigotimes_{i \in I} A_i, \bigotimes_{i \in I} (A_i,p_i) \right)
	} 
	\DefineNamedFunc{measureAlgebraTensorRepresentation}{}{\Psi_{A,\mu}}
	{
		    \FUNC{tensor}\left( 
		    	\Lambda [E] \in \prod_{i \in I} A_i \. \left[ \prod_{i \in I} E_i \right]
		    \right)
	}
	\\
	\Theorem{TensorRepresentationsAreDense}
	{
		\NewLine ::		
		\forall I \in \SET \.
		\forall (A,p) : I \to \TYPE{ProbabilityAlgebra} \.
		\Dense\left( \bigotimes_{i \in i} (A,p_i), \Psi_{A,\mu}\left( \bigotimes_{i \in I} A_i \right) \right)
	}
	\NoProof
}
\newpage
\subsubsection{independent Subalgebras}
\Page{
	\DeclareType{StochasticalyIndependent}
	{
		\prod (A,p) :  \TYPE{ProbabilityAlgebra} \.
		\prod I \in \SET \.
		?(I \to \TYPE{Subring}(A))
	}
	\DefineType{C}{StochasticalyIndependent}
	{
		\forall J : \TYPE{Finite}(I) \. 
		\forall c : \prod_{j \in J} A_j \.
		p\left( \bigvee_{j \in J} c_j \right) = \prod_{j \in J} p(c_j)
	}
	\\
	\Theorem{StochasticalyIndependentGeneration}
	{
		\NewLine ::		
		\forall (A,p) : \TYPE{ProbabilityAlgebra} \.
		\forall I \in \SET \.
		\forall C : \SInd(A,p,I) \. \NewLine \.
		\forall \aleph : \forall i \in I \. C_i \subset_\ma (A,p) \.
		\bigotimes_{i \in I} (C_i, p) \cong_\ma \left\langle \bigcup_{i \in I} C_i \right\rangle_\ma 
		\subset_\ma (A,p)
	}
	\Explain{ This is obvious}
	\EndProof
	\\
	\Theorem{StochasticalyIndependentInProcessAlgebra}
	{
		\NewLine ::		
		\forall I \in \SET \.
		\forall (A,p) : I \to \TYPE{ProbabilityAlgebra} \.
		 \SInd\left(\bigotimes_{i \in I} (A_i,p_i),I, (A,p)\right) }
	\Explain{ This is obvious}
	\EndProof
}
\newpage
\subsubsection{Coordinate Determination}
\Page{
	\DeclareFunc{coordinateSubalgebra}{\prod_{I \in \SET}  (I \to \PAlg) \to ?I \to \PAlg}
	\DefineNamedFunc{coordinateSubalgebra}{(C,p),J}{C_J}{\bigvee_{j \in J} \iota_j(C_j)}
	\\
	\Theorem{ProcessAlgebraRepresentation}
	{
		\NewLine ::		
		\forall I \in \SET \.
		\forall (C,p) : I \to \PAlg \.
		\forall J \subset I \.
		C_J \cong_{\ma} \bigotimes_{j \in J} (C_j,p_j)
	}
	\Explain{ This is obvious}
	\EndProof
	\\
	\Theorem{CoordinateDeterminationExists}
	{
		\NewLine ::		
		\forall i \in \SET
		\forall (C,p) : I \to \PAlg \. 
		\forall c \in C \. \exists! \min\Big\{ J : \TYPE{Countable}(I) \Big| c \in C_J \Big\}
	}
	\Explain{ 1 Let $\J = \Big\{ J : \TYPE{Countable}(I) \Big| c \in C_J \Big\}$ }	
	\Explain{ 2 $\J \neq \emptyset$}	
	\Explain{ 2.1 Note that $\bigotimes_{i \in I} C_i$ is dense in $\bigotimes_{i \in I} (C_i,p_i)$}
	\ExplainFurther{ 2.2 So there exists a sequence of natural numbers $n : \Nat \to \Nat$,}
	\ExplainFurther{ $\quad \quad$ 
		a system of finite subsets $i \in \prod^\infty_{k=1} \{1,\ldots,n_l\}\times\{1,\ldots,n_k\}\to I$
		and $t \in \prod^\infty_{k=1} \prod^k_{l=1} \prod^{n_k}_{h=1} C_{i_{k,l,h}}$}
	\Explain{  $\quad \quad$ 
			such that $c = \lim_{k \to \infty} \sum^{k}_{l=1} \bigotimes_{h=1}^{n_k} t_{k,t,h}$,
			where are all missing slots are filled by $e$}
	\Explain{
		2.3 Then $J = \im i \in \J$, so $\J \neq \emptyset$}
	\Explain{ 3 $\J$ has a minimal element}
	\Explain{
		3.1 Assume $\C$ is a chain in $\J$}
	\Explain{
		3.2 Then $c \in C_J$ for any $J \in \C$}
	\Explain{
		3.3  So $c \in \bigcap_{J \in \C} C_J = C_{\bigcap_{J \in \C} J}$
	}
	\Explain{ 3.3.1 Here we used the fact that $\C$ is decreasing}
	\Explain{ 3.3.2 $C_J$ Form a sequence of decreasing closed subalgebras}
	\Explain{ 3.4 So $\bigcap_{J \in \C} J \in \J$ and the lower bound is atained}
	\Explain{ 4 The minimum Is unique}
	\Explain{ 4.1 Assume that $I,J \in \J$}
	\Explain{ 4.2 Then $c \in C_I \cap C_J$ }
	\NoProof
	\\
}\Page{
	\DeclareFunc{coordinateDetermination}
	{
		\prod_{I \in \SET}  \prod (C,p) : I \to \PAlg \.  \bigotimes_{i \in I}  (C_i,p_i) \to \TYPE{Countable}(I)
	}
	\DefineNamedFunc{coordinateDetermination}{c}{J_c}
	{
			\THM{CoordinateDeterminationExists}
	}
	\\
	\Theorem{MidElementCoordinatesDetermination}
	{
		\NewLine ::		
		\forall I \in \SET \.
		\forall (C,p) : I \to \PA \.
		\forall a,c \in \bigotimes_{I} (C_i,p_i) \.
		\forall \aleph : a \le c \.
		\exists b \in C_{J_a \cap J_c} \.
		a \le b \le c
	}
	\Exclaim{This follows from Fubbini Theorem}
	\NoProof
	\\
	\Theorem{MidElementCoordinatesDetermination}
	{
		\NewLine ::		
		\forall I \in \SET \.
		\forall (C,p) : I \to \PA \.
		\forall a,c \in \bigotimes_{I} (C_i,p_i) \.
		\forall \aleph : a \le c \.
		\exists b \in C_{J_a \cap J_c} \.
		a \le b \le c
	}
	\Exclaim{This follows from Fubbini Theorem}
	\NoProof
	\\
	\Theorem{CoordinatesDetermination}
	{
		\NewLine ::		
		\forall I \in \SET \.
		\forall (C,p) : I \to \PA \.
		\forall \J : ??I \.
		\bigcap C_\J = C_{\bigcap \J}
	}
	\Explain{Part of the previous Theorem}
	\NoProof
	\\
	\ExplainFurther{
		Note: It may be interesting to prove this results independently of abstract measure theory,}
	\Explain{
		and then prove Fubbini therorem and related results from coordinate Determination}
}
\newpage
\subsection{Radon-Nikodym Parallels}
\newpage
\subsubsection{Finitely Additive Functionals}
\Page{
	\DeclareFunc{finitelyAdditiveFunctionals}
	{
		\Contra(\BOOL,\VS{\Reals})
	}
	\DefineNamedFunc{finitlyAdditiveFunctionals}{A}{\af(A)}
	{
		\NewLine \de	    
	    \Big\{ f : A \to \Reals : \forall (a,b) : \TYPE{DisjointPair}(A) \. f(a \vee b) = f(a) + f(b) \Big\}	
	}
	\DefineNamedFunc{finitelyAdditiveFunctionals}{A,B,\phi}{\af_{A,B}(\phi)}
	{
		\phi_*
	}
	\\
	\DeclareFunc{boundedAdditiveFunctionals}
	{
		\Contra(\BOOL,\VS{\Reals})
	}
	\DefineNamedFunc{boundedAdditiveFunctionals}{A}{\baf(A)}
	{	    
	    \Big\{ f \in \af(A) : \exists r \in \Reals_+ \. \forall a \in a \. |f(a)| < r  \ \Big\}	
	}
	\DefineNamedFunc{boundedAdditiveFunctionals}{A,B,\phi}{\baf_{A,B}(\phi)}
	{
		\phi_*
	}
	\\
	\Theorem{Zero}
	{
		\forall A \in \BOOL \.
		\forall f \in \af(A) \.
		f(0) = 0
	}
	\Explain{ 1 $(0,0)$ is a disjoint pair as $0 \cdot 0 = 0$}
	\Explain{ 2 So $f(0) = f(0 \vee 0) = f(0) + f(0)$}
	\Explain{ 3 Which can be rewritten as $f(0) = 0$}
	\EndProof		
	\\
	\Theorem{Restriction}
	{
		\forall A \in \BOOL \.
		\forall f \in \af(A) \.
		\forall a \in A \.
		\Lambda c \in A \. f(ac)  \in \af(A)
	}
	\Explain{ 1 Defin $g(c) = f(ab)$}
	\Explain{ 2 Assume $(c,d)$ is a disjoint pair}
	\Explain{ 3 Then $(ac)(ad) = acd = 0$}
	\Explain{ 4 $(ac,ad)$ is a disjoint pair also}	
	\Explain{ 5 So $g(c \vee d) = f\Big(a(c \vee d)\Big) = f\Big(ac \vee ad\Big) =
		f(ac) + f(ad) = g(c) + g(d)$}
	\Explain{ 6 Thus, $g \in \af(A)$}
	\EndProof		
	\\
	\Theorem{PositiveIffMonotonic}
	{
		\forall A \in \BOOL \.
		\forall f \in \af(A) \.
		f \ge 0 \iff \TYPE{Monotonic}(A,\Reals,f)
	}
	\Explain{1 Assume $f > 0$}
	\Explain{1.1 Asume $a,b \in A$ is such that $a > b$ }
	\Explain{1.2 Then $f(a) = f(ab \vee b \setminus a) = f(ab) + f(b \setminus a) = 
		f(a) + f(b \setminus a) \ge f(a)$}
	\Explain{2 Assume that $f$ is monotonic}
	\Explain{2.1 Assume $a \in A$}
	\Explain{2.2 Note, that $f(0) = 0$}
	\Explain{2.3 So, as $a \ge 0$ then $f(a) \ge 0$}
	\EndProof
}\Page{
	\Theorem{JordanDecomposition}
	{
		\NewLine ::		
		\forall A \in \BOOL \.
		\forall f \in \af(A) \.
		f \in \baf(A) \iff 
		\exists g,h \in \af(A) \.		g,h \ge 0 \And f = g - h
	}
	\Explain{ 1 ($\Rightarrow$) Assume $f$ is bounded}
	\Explain{ 1.1 Define $g(a) = \sup \{ f(c) | c \in A, c \le a   \} $ }
	\Explain{ 1.2 $g$ is finitely addive}
	\Explain{ 1.2.1 Assume $a,b \in A$ are such that $ab = 0$}
	\ExplainFurther{ 1.2.2 Then 
			$g(a \vee b) = \sup \{  f(c) | c \in A, c \le a \vee b   \}=	
			\sup \Big\{  f\Big(c(a \vee b)\Big) \Big| c \in A, c \le a \vee b \Big\} =$}
   \ExplainFurther{ $\quad \quad
			= \sup \{  f(ca \vee cb) | c \in A, c \le a \vee b   \} =
			\sup \{ f(ca) + f(cb) | c \in A, c \le a \vee b   \} =$}
	\Explain{ $\quad\quad		
			=\sup \{ f(c) + f(d)  | c,d \in A, c \le a, d \le b   \} =
			\sup \{ f(c) | c \in A, c \le a\} + \{ f(c) | c \in A, c \le b\} =
			g(a) + g(b)
			$}
	\Explain{ 1.3 Then $h$ can be defined in a simmilar manner but for $-f$}
	\Explain{ 1.4 $f = g - h$}	
	\ExplainFurther{ 1.4.1 $g(a) - h(a) = 
		\sup \{ f(c) | c \in A, c \le a\} - \sup \{ -f(c) | c \in A, c \le a\} =$}
	\Explain{ $\quad \quad 
		=\sup \{ f(c) | c \in A, c \le a\} + \inf \{ f(c) | c \in A, c \le a\}		
		$ }
   \Explain{ 1.4.2 Then we may select some $c : \Nat \to (a)$  
   	such that $g(a) = \lim_{n \to \infty} f(c_n)$}
   	\Explain{ 1.4.3 Then $-h(a) = \lim_{n \to \infty} f(a \setminus c_n)$ from $(1.4.1)$}
   	\Explain{ 1.4.4 Thus 
   	$g(a) - h(a) = 
   		\lim_{n \to \infty} f(c_n)  + \lim_{n \to \infty} f(a \setminus c_n) =
   	 \lim_{n \to \infty} f(c_n) + f(a \setminus c_n) =
   	 \lim_{n \to \infty} f(a) = f(a)	$}
   	\Explain{ 2 ($\Leftarrow$)  Assume there are $g,h \in \af_+(A)$ such that $f = g - h$}
   	\Explain{ 2.1 Assume $a : \Nat \to A$ is a disjoint sequence}
   	\Explain{ 2.2 Then $\sum^\infty_{n=1} g(a_n) = 
		\lim_{k \to \infty} \sum^k_{n=1} g(a_n) =
		\lim_{k \to \infty} g\left( \bigvee^k_{n=1} a_n \right) \le g(a) < \infty$}
	\Explain{ 2.3 $g$ is bounded}	
	\Explain{ 2.3.1 Assume now that $g$ is unbounded}
	\Explain{ 2.3.2 Then there exists a sequence $c$ such that $\lim_{n \to \infty} g(c_n)=0$}
	\Explain{ 2.3.3 Define $a_n = c_n \setminus \bigvee^{n-1}_{k=1} a_k $}
	\Explain{ 2.3.4 Then $\sum^\infty_{n=1} g(a_n) = 
		\lim_{n \to \infty} g\left( \bigvee^n_{k=1} c_k\right) \ge 
		\lim_{n \to \infty} g(c_k) = \infty$ }
	\Exclaim{ 2.3.5 But this contradicts $(2.2)$}
	\Explain{ 2.4 The same is true about $h$}
	\Explain{ 2.5 So $f$ is bounded as linear comination of bounded functionals}
	\EndProof
	\\
	\DeclareFunc{decompositionOfJordan}
	{
		\prod{A \in \BOOL} \baf(A) \to \baf_+^2(A)
	}
	\DefineNamedFunc{decompositionOfJordan}{f}{(f_+,f_-)}{\THM{JordanDecomposition}(A,f)}
}\Page{
	\DeclareFunc{cilindersElements}
	{
		\prod I \in \SET \. (I \to \BOOL) \to \TYPE{Monoid}
	}
	\DefineNamedFunc{cilindersElements}{A}{C(I,A)}
	{
		\left\{ \bigwedge_{j \in J} \iota_j(a_j)\bigg| J : \Finite(I),  a \in \prod_{j \in J} A_j \right\}
	}
	\\
	\Theorem{CoproductExtension}
	{
		\forall I \in \SET \.
		\forall A : I \to \BOOL \.
		\forall \theta : C(I,A) \to \Reals \. \NewLine \.
		\forall \aleph : \forall c \in C(I,A) \. \forall i \in I \. \forall a \in A_i \.
		\theta(c) = \theta\Big( c \iota_i(a) \Big) + \theta\Big( c \overline{\iota_i(a)} \Big) \.
		\exists f \in \af\left( \bigotimes_{i \in I} A_i \right) \. f_{|C} = \theta
	}
	\NoProof
}
\newpage
\subsubsection{Properly Atomless Functionals}
\Page{
	\DeclareType{\PAless}{\prod_{A \in \BOOL} ?\af(A)}
	\DefineType{f}{\PAless}{ \NewLine
		\iff 
		\forall \varepsilon \in \Reals_{++} \.
		\exists P : \PoU(A) \. 
		|P| < \infty \And 
		\forall p \in P \. 
		\forall a \in (p) \.
		|f(a)| \le \varepsilon  
	}
	\\
	\Theorem{VectorSubspace}
	{
		\forall A \in \BOOL \.
		\PAless(A) \subvec{\Reals} \baf(A)
	}
	\Explain{1 Assume $f$ is properly Atomless}
	\Explain{2 Then $f$ is bounded}
	\Explain{2.1 There is a finite partition of unity $P$ such that 
		$|f(a)| < 1$ for any $p \in P$ and $a \in (p)$}
	\Explain{2.2 Then 
		$|f(a)| = \left|f\left(a \bigvee P \right)\right| 
			=\left| \sum_{p \in P} f(ap) \right| \le \sum_{p \in P} |f(ap)| \le |P| < \infty		
		$}
	\Explain{3 Then $\alpha P$ may use simmilar partitions as $P$ fo $\frac{\varepsilon}{|\alpha|}$}
	\Explain{4 And a sum $f + g$ may use intermeshes of $f$ and $g$}
	\EndProof
}\Page{
	\Theorem{ContinuousPartitioningTHM1}
	{
		\NewLine ::
		\forall A : \SA \.
		\forall I \in \SET \.
		\forall f : I \to \af_+(A) \. \NewLine \.
		\forall \aleph : \forall a \in A \. 
		\exists \alpha \in \left[ \frac{1}{3},\frac{2}{3} \right] \.
		\exists a' \in (a) \.
		\forall i \in I \.
		\alpha f_i(a) = f_i(a') \. \NewLine \.
		\forall a \in A \. 
		\exists  u : [0,1] \uparrow (a) \.
		u_0 = 0 \And u_1 = a \And 
		\forall \tau \in [0,1] \.
		\forall i \in I \.
		f_i(u_\tau) =  \tau f_i(a) 
	}
	\Explain{1 Assume that there is $k \in I$ such that $f_k(a) > 0$ }
	\Explain{1.1 Otherwise set $u_1 = a$ and $u_\tau = 0$}
	\Explain{2 Define $\gamma_i = \frac{f_i(a)}{f_k(a)}$}
	\ExplainFurther{3 Define sets $D : \Int_+ \to 2^{(a)}$ recursevely in a such way that 
	   $D$ is increasing, and each $D_n$ is finite and ordered}
	\Explain{ $\quad \quad$ with $a,0 \in D_n$ and $f_i(d) = \gamma_i f_k(d)$ for every $i \in I$}
	\Explain{3.1 Let $D_0 = \{0,a\}$  }
	\Explain{3.2  Then assume $m = |D_n|$ and let $d : \{1,\ldots,m\} \to D_n$ be an   
		an order-preserving enumeration}
	\ExplainFurther{3.3 Then by $\aleph$ there is 
		$c : \{1,\ldots,m-1\} \to (a)$ such that $c_l \le d_{l+1} \setminus d_l$m}
	\Explain{$\quad \quad$ And  a sequence 
		$\alpha : \{1,\ldots,m-1\} \to \left[\frac{1}{3},\frac{2}{3}\right]$
		such that $f_i(c_l) = \alpha_l f_i(d_{l+1}\setminus d_l)$ for any $i \in I$}
	\Explain{3.4 Define $D_{n+1} = D_n \cup \Big\{ d_l \vee c_l \Big| l \in \{1,\ldots,m-1\}   \Big\}$}
	\Explain{3.5 The it is obvious that $D_{n+1}$ is finite and ordered}
	\ExplainFurther{3.6 So we constructed and increasing $D$ with a property
		$f_i(d_{l+1}\setminus d_l) \le \left( \frac{2}{3} \right)^n  f_i(a) $}
	\Explain{ $\quad \quad$ for any $i \in I$ and $d$ being enumeration of $D_n$ as above}
	\Explain{ 4 Set $C = \bigcup_{n=1}^\infty D_n$}
	\Explain{ 5 Then $C$ is countale totally ordered set with $0,a \in C$ and
		$f_k(C)$ is dense in $[0,f_k(a)]$}
	\Explain{ 6 Define $u_\tau = \sup \{ c \in C, f_k(c) \le \tau f_k(a)\}$ }
	\Explain{ 6.1 This supremum has to exists}
	\Explain{ 6.2 As $f_k(C)$ is dense in $[0,f_k(a)]$ there is
		a sequence $c : \Nat \to C$ with $\lim_{n \to \infty} c_n = \tau f_k(a)$ }
	\Explain{ 6.3 Without loss of generality we may assume that $c$ is non-decreasing}
	\Explain{ 6.4 And we may define $u_\tau = \bigvee^\infty_{n=1} c_n$}
	\Explain{ 7 Then $u_0 = 0$ and $u_1 = a$ and $f_i(u_\tau) = \tau f_i(a)$ for any $i \in I$}
	\EndProof
}\Page{
	\Theorem{ContinuousPartitioningTHM2}
	{
		\NewLine ::
		\forall A : \SA \.
		\forall n \in \Nat \.
		\forall f : \{1,\ldots,n\} \to \PAless(A) \. \NewLine \.
		\forall \aleph :  \forall i \in \{1,\ldots,n\} \. 0 \le f_i \le f_1 \.
		\NewLine \.
		\forall a \in A \. 
		\exists  u : [0,1] \uparrow (a) \.
		u_0 = 0 \And u_1 = a \And 
		\forall \tau \in [0,1] \.
		\forall i \in \{1,\ldots,n\} \.
		f_i(u_\tau) =  \tau f_i(a) 
	}
	\Explain{ 1 We prove that conditions of Previous theorem are satisfied with $I = \{1,\ldots,n\}$}
	\Explain{ 2 At first consider the case $I = \{1\}$}
	\Explain{ 2.1 We seek to prove 
			$a \in A$ 
			there is an $\alpha \in \left[\frac{1}{3}, \frac{2}{3} \right]$ and $a' \in (a)$ such that 
			$f_1(a') = \alpha f_1(a)$}
	\Explain{ 2.2 Then there is finite partition of unity $P$ 
		such that $|f_1(c)| < \frac{1}{3} f_1(a)$ for any $p \in P$ and for any $c \le p$}
	\Explain{ 2.3 Then it must be possible to sample $Q \subset P$ 
		in such a way that $a' = a\bigvee_{p \in Q} p$ and 
		$\frac{f_1(a')}{f_1(a)} \in \left[\frac{1}{3}, \frac{2}{3} \right]$}
	\Explain{ 2.3.1 We know that $a = a \bigvee_{p \in P} p$ 
	 and $|f_1(ap)| \le \frac{f_1(a)}{3}$}
	 \Explain{ 2.3.2 So if $f_1(ap) <  \frac{f_1(a)}{3}$ for some $p \in P$
		there is also some $q \in P$ such that $f_1(a(p\vee q)) \le \frac{2f_1(a)}{3}$}
	\Explain{ 2.3.3 This Process must stop as 
		$f_1(a) = f_1\left(a\bigvee_{p \in P} p\right) = \sum_{p \in P} f_1(ap)$
	}
	\Explain{ 3 We follow by induction}
	\Explain{ 3.1 Assume the theorem holds for all  $i \in I$ with $i < m$ and we have corresponding $u$
		for $\{1,\ldots,m-1\}$}
	\Explain{ 3.2  $|f_m(u_t) - f_m(u_s)| = 
			f_m(u_t \setminus u_s) \le f_0(u_t\setminus u_s)= (t - s) f_0(a)
		$ for $0 \le s \le t \le 1$}
	\Explain{ 3.3  So $\phi(t) = f_m(u_t)$ is continuous}
	\Explain{ 3.4  and the function $\psi : \left[0,\frac{1}{2}\right] \to \Reals_+$ 
		defined by $\psi(t) = f_m(u_{t + \frac{1}{2}}) - f_m(u_t)$
		is continuous}
	\Explain{ 3.5 Note that $\psi(0) + \psi\left(\frac{1}{2}\right) = f_m(a)$ }
	\Explain{ 3.6 So by the intermidiate value theorem there must be some 
	$t \in \left[0,\frac{1}{2}\right]$ such that $\psi(t) = \frac{1}{2}f_m(a)$}
	\Explain{ 3.7 Define $u' = u_{t + \frac{1}{2}} \setminus u_t$}
	\Explain{ 3.8 Then $f_i(u') = \frac{1}{2} f_i(a)$ for all $i \in \{1,\ldots,m\}$}
	\Explain{ 3.9 But this means that that the assertion holds fo $\{1,\ldots,m\}$ 
		And we can use the previous theorem}
	\EndProof
}
\newpage
\subsubsection{Liapounoff's Convexity Theorem}
\Page{
	\\
	\DeclareFunc{vectorValuedFinitelyAdditiveFunctionals}
	{
		\prod V : \Reals\hyph\mathsf{BAN} \. \Contra(\BOOL,\VS{\Reals})
	}
	\DefineNamedFunc{finitlyAdditiveFunctionals}{A}{\af(A,V)}
	{
		\NewLine \de	    
	    \Big\{ f : A \to V : \forall (a,b) : \TYPE{DisjointPair}(A) \. f(a \vee b) = f(a) + f(b) \Big\}	
	}
	\DefineNamedFunc{finitelyAdditiveFunctionals}{A,B,\phi}{\af_{A,B}(\phi)}
	{
		\phi_*
	}
	\\
	\DeclareFunc{vectorValuedBoundedAdditiveFunctionals}
	{
		\prod V : \Reals\hyph\mathsf{BAN} \. \Contra(\BOOL,\VS{\Reals})
	}
	\DefineNamedFunc{boundedAdditiveFunctionals}{A}{\baf(A,V)}
	{	    
	    \Big\{ f \in \af(A) : \exists r \in \Reals_+ \. \forall a \in a \. \|f(a)\| < r  \ \Big\}	
	}
	\DefineNamedFunc{boundedAdditiveFunctionals}{A,B,\phi}{\baf_{A,B}(\phi)}
	{
		\phi_*
	}
	\\
	\DeclareType{\PAless}{\prod_{A \in \BOOL} \prod_{V \in \Reals\hyph\mathsf{BAN}} ?\af(A)}
	\DefineType{f}{\PAless}{ \NewLine
		\iff 
		\forall \varepsilon \in \Reals_{++} \.
		\exists P : \PoU(A) \. 
		|P| < \infty \And 
		\forall p \in P \. 
		\forall a \in (p) \.
		\|f(a)\| \le \varepsilon  
	}
	\\
	\Theorem{LiapounoffsConvexityTHM}
	{
		\NewLine ::		
		\forall A : \SA \.
		\forall n \in \Nat \.	
		\forall f : \PAless(A,\Reals^n) \. 
		\TYPE{Convex}(\Reals^n,f(A))
	}
	\Explain{ This is an application of continuous decomposition theorems}
	\NoProof
	\\
	\Explain{ Note this is an additional problem then this theorem hold for infinite-dimensional vector spaces}
}
\newpage
\subsubsection{Countably Additive Functionals}
\Page{
	\DeclareFunc{countablyAdditiveFunctionals}
	{
		\Contra(\BOOL_\sigma, \VS{\Reals})
	}
	\DefineNamedFunc{countablyAdditiveFunctionals}{A}{\caf(A)}
	{
		\NewLine \de
		\left\{ 
			f \in \af(A) : \forall a : \TYPE{DisjointSequence}(A) 
			\exists \bigvee^\infty_{n=1} a_n \Imply
			f\left(\bigvee^\infty_{n=1} a_n\right) = \sum^\infty_{n=1} f(a_n)  
		\right\} 
	}
	\DefineNamedFunc{countablyAdditiveFunctionals}{A,B,\varphi}{\caf_{A,B}(\varphi)}
	{
		\varphi_*
	}
	\\
	\Theorem{IncreasingExpression}
	{
		\NewLine ::		
		\forall A \in \BOOL \.
		\forall f \in \caf(A) \.
		\forall a : \Nat \uparrow A \.
		\exists \bigvee^\infty_{n=1} a_n 
		\Imply
		f\left( \bigvee^\infty_{n=1} a_n \right) = \lim_{n \to \infty} f(a_n)
	}
	\Explain{ 1 Note that $a_{n} \setminus a_{n-1}$ is a disjoint sequence
		with $a_0 = 0$}
	\Explain{ 2 Then 
		$
		f\left( \bigvee^\infty_{n=1} a_n \right)  =
		f\left( \bigvee^\infty_{n=1} a_n \setminus a_{n-1} \right) = 
		\sum^\infty_{n=1} f(a_n \setminus a_{n-1})  =
		\lim_{n \to \infty} \sum^n_{k=1} f(a_k \setminus a_{k-1}) =
		\lim_{n \to \infty} f(a_n)
		$}
	\EndProof
	\\
	\Theorem{DecreasingExpression}
	{
		\NewLine ::		
		\forall A \in \BOOL \.
		\forall f \in \caf(A) \.
		\forall a : \Nat \downarrow A \.
		\exists \bigwedge^\infty_{n=1} a_n 
		\Imply
		f\left( \bigwedge^\infty_{n=1} a_n \right) = \lim_{n \to \infty} f(a_n)
	}
	\Explain{ 1 Note that $a_{1} \setminus a_{n}$ is increasing}
	\ExplainFurther{ 2 Then 
		$
		f\left(\bigwedge^\infty_{n=1} a_n \right)  =
		f\left(a_1 \setminus \bigvee^\infty_{n=1} (a_1 \setminus a_n) \right)= 
		f(a_1) -  f\left(\bigvee^\infty_{n=1} (a_1 \setminus a_n)\right) = 
		f(a_1) - \lim_{n \to \infty} f(a_1 \setminus a_n) = $}
	\Explain{ $ \quad \quad =
		f(a_1) - \lim_{n \to \infty} f(a_1) - f(a_n) = 
		\lim_{n \to \infty} f(a_n) $}
	\EndProof
	\\	
	\Theorem{Restriction}
	{
		\forall A \in \BOOL \.
		\forall f \in \caf(A) \.
		\forall a \in A \.
		\Lambda c \in A \. f(ac)  \in \caf(A)
	}
	\NoProof
} \Page{
	\Theorem{CAFByLimits}
	{
		\forall A \in \BOOL \. 
		\forall f \in \af(A) \.
		\forall \aleph : \forall a : \Nat \downarrow A \. 
		\bigwedge_{n=1}^\infty a_n = 0 \Imply \lim_{n\to\infty} f(a_n) = 0 \.
		f \in \caf(A)
	}
	\Explain{ 1 Assume $a : \Nat \to A$ is a disjoint sequence with $\bigvee^\infty_{n=1} a_n$
		existing}
	\Explain{ 2 Then $\bigwedge^\infty_{n=1} \bigvee^\infty_{m=n} a_n = 0 $}
	\Explain{ 3 So, $\lim_{n \to \infty} f\left(  \bigvee^\infty_{m=n} a_n\right) = 0$ by $\aleph$}
	\Explain{ 4 Then for any $m \in \Nat$ there is a rewrite 
		$
			f\left( \bigvee^\infty_{n=1} a_n \right) = 
			\sum^m_{k=1} f(a_k) + f\left( \bigvee^\infty_{n=m + 1} a_n \right)
		$ 
	}
	\Explain{ 5 Taking a limit $m \to \infty$ produces the desired result
		$
			f\left( \bigvee^\infty_{n=1} a_n \right) = \sum^\infty_{n=1} f(a_n)
		$   }
	\EndProof
	\\
	\Theorem{DominatedCAF}
	{
		\forall A \in \BOOL \.
		\forall f \in \af(A) \.
		\forall g \in \caf(A) \.
		|f| \le g \Imply f \in \caf(A)
	}
	\Explain{ 1 Assume $a : \Nat \downarrow A$ such that $\bigwedge^\infty_{n=1} a_n = 0$}
	\Explain{ 2 $g \in \caf(A)$ imply that $\lim_{n \to \infty} g(a_n) = 0$}
	\Explain{ 3 But then domination Imply that $\lim_{n \to \infty} f(a_n) = 0$}
	\Explain{ 4 By previous theorem this means that $f \in \caf(A)$}
	\EndProof
	\\
	\Theorem{JordanDecomposition}
	{
		\forall A \in \BOOL \.  
		\forall f \in \caf(A) \.
		f  \in \baf(A) \iff
		\exists g,h \in \caf_+(A) \. f = g - h
	}
	\Explain{ 1 $(\Rightarrow)$ Assume $f$ is bounded}
	\Explain{ 1.1 Then $f = f_+ - f_-$ by simple Jordan's decomposition for finitely additive functionals}
	\Explain{ 1.2 We may write $f_+(a) = \sup \{f(c) | c \in A, c \le a\}$ }
	\Explain{ 1.3 Assume $a : \Nat \to A$ is a disjoint sequence
	  such that $\bigvee^\infty_{n=1} a_n$ exists}
	\ExplainFurther{ 1.4 Then $f_+\left(\bigvee^\infty_{n=1} a_n\right) = 
			\sup \left\{f(c) \bigg| c \in A, c \le \bigvee_{n=1}^\infty a\right\} =
			\sup \left\{f\left(c\bigvee^\infty_{n=1} a_n\right) 
				\bigg| c \in A, c \le \bigvee_{n=1}^\infty a\right\} =$}
	\ExplainFurther{
	$  \quad \quad= \sup \left\{f\left(\bigvee^\infty_{n=1} ca_n\right) \bigg| c \in A, c \le \bigvee_{n=1}^\infty a\right\} = \sup \left\{ \sum^\infty_{n=1} f(ca_n) \bigg| c \in A, c \le \bigvee_{n=1}^\infty a\right\} =$}
 	\ExplainFurther{
 			$\quad \quad=\sup \left\{ \sum^\infty_{n=1} f(c_n) \bigg| 
 			c : \Nat \to A, \forall n \in \Nat \. c _n\le a_n \right\} =
 			\sum^\infty_{n=1} \sup \{f(c) | c \in A, c \le a_n\} =
 			\sum^\infty_{n=1} f_+(a_n)
 			$ 
	 }
}
\Page{
	 \Explain{ 1.4.1 the sum $\sum^\infty_{n=1} f(c_n)$ must exist
	 	as $\sum^\infty_{n=1} |f(c_n)| \le \sum^\infty_{n=1} f_+(c_n) $}
	 \Explain{ 1.4.2 And if $\sum^\infty_{n=1} f_+(c_n) $ diverges then the 
	 	sequence $\phi_n = f_+\left( \bigvee^n_{k= 1} c_k\right)$ must be unbounded}
	 \Exclaim{ 1.4.3 But $f_+$ must bounded by basic Jordan decomposition theorem, a contradiciton}
	 \Explain{ 1.4.4 So $\sum^\infty_{n=1} f(c_n)$ exists as absolutely converging series}
	 \Explain{ 2 $(\Leftarrow)$ This direction follows from basic Jordan Decomposition}
	 \EndProof
	\\	 
	 \Theorem{HahnDecomposition1}
	{
		\forall A \in \SA \.  
		\caf(A) \subvec{\Reals} \baf(A) 
	}
	\Explain{ 1 Assume $f \in \baf(A)$ }
	\Explain{ 2  Let  $\gamma = \sup_{a \in A} f(a)$}
	\Explain{ 3 Then there is a disjoint sequence $a : \Nat \uparrow A$
		such that $\gamma = \lim_{n=1} f(a_n) = f\left( \bigvee^\infty_{n=1}  a_n \right) < \infty$}
	\Explain{ 3.1 Clearly there is a sequence $c : \Nat \to A$ 
		such that $\lim_{n=1} f\left( c_n \right) = \gamma$
	}
	\Explain{ 3.2 Without loss of generality it may be assumed that $f(c_n) > 0$}
	\Explain{ 3.2.1 Otherwise $\sup_{a \in A} f(a) = f(0) = 0$}
	\Explain{ 3.3 So $\gamma = 
		\lim_{n \to \infty} f(c_n) \le 
		\lim_{n \to \infty} \sum^n_{k=1} f(c_k)  =
		\lim_{n \to \infty} f\left( \bigvee^n_{k=1} c_k \right) \le \gamma$  }
	\Explain{ 3.4 So $\gamma = \lim_{n \to \infty} f\left( \bigvee^n_{k=1} c_k \right)$}	
	\Explain{ 3.5 Just define $a_n = \bigvee^n_{k=1} c_k$}
	\EndProof	
	\\
	\Theorem{HahnDecomposition2}
	{
		\forall A \in \SA \.  
		\forall f \in \caf(A) \.
		\exists d \in A \.
		\forall c \in A \.
		f(c\bar d) \le 0 \And f(cd) \ge 0
	}
	\Explain{ 1 Let $\gamma$ and $a$ be as above}
	\Explain{ 2 Let $d = \bigvee^\infty_{n=1} a_n$}
	\Explain{ 3 Assume $c \in (\bar d)$ }
	\Explain{ 4 If $f(c) \ge 0$ then $f(c \vee d) > \gamma$ but this is impossible}
	\Explain{ 5 Otherwise if $c \le d$ and $f(c) < 0$ then $f(d \setminus c) > \gamma$
		which is impossible}
	\EndProof
}
\newpage
\subsubsection{Completely Additive Functionals}
\Page{
	\DeclareFunc{completelyAdditiveFunctionals}
	{
		\Contra(\BOOL_\tau, \VS{\Reals})
	}
	\DefineNamedFunc{completelyAdditiveFunctionals}{A}{\Caf(A)}
	{
		\NewLine \de
		\left\{ 
			f \in \af(A) : \forall D : \TYPE{DownwardsDirected}(A) \.
			\bigvee_{d \in D} d = 0 \Imply \inf_{d \in D}  |f(d)| = 0 
		\right\} 
	}
	\\
	\Theorem{CountaleAdditivity}
	{
		\forall A \in \BOOL \. \Caf(A) \subvec{\Reals} \caf(A)
	}
	\Explain{ 1 Assume $f \in \Caf(A)$}
	\Explain{ 2 Also Assume $a : \Nat \to A$ is a decreasing with $\bigwedge^\infty_{n=1} a_n = 0$}
	\Explain{ 3 Then $\lim_{n =1} f(a_1) = 0$}
	\Explain{ 4 Hence $f$ is countably additive}
	\EndProof
	\\
	\Theorem{InfimumLocalization}
	{
		\NewLine ::		
		\forall A \in \BOOL \.
		\forall f \in \Caf(A) \.
		\forall \varepsilon \in \Reals_{++} \.
		\forall D : \TYPE{DownwardsDirected}(A) \.
		\forall \aleph : \inf D =  0 \. \NewLine \.
		\exists d \in D \.
		\forall c \in (d) \.
		|f(c)| < \varepsilon
	}
	\Explain{ 1 Assume otherwise}	
	\Explain{ 2 Let $C = \{ c \in A : |f(c)| \ge \varepsilon, \exists d : d \le c \}$}
	\Explain{ 3 Every member of $A$ includes some membebr of $C$}	
	\Explain{ 3.1 Assume $d \in D$ }
	\Explain{ 3.2 Then by $(1)$ there is $c \in A$ such that $c \le d$ and $|f(c)| \ge \varepsilon$}
	\Explain{ 3.3 Let $D'_d = \{ d'\setminus c |  d' \in D, d' \le d \}$}
	\Explain{ 3.4 Then $D'_d$ is downwards directed and $\lim D'_d = 0$}
	\Explain{ 3.5 So there is a $d'$ such that $|f(d'\setminus c)| <  |f(c)| - \varepsilon $}
	\Explain{ 3.6 Let $c' = d' \vee c$}
	\Explain{ 3.7 Then $c' \le d$}
	\Explain{ 3.8 Also $|f(c')| = | f(d' \setminus c)  + f(c)| \ge |f(c)| - |f(d' \setminus c)|  \ge \varepsilon$}
	\Explain{ 3.9 So $c' \in C$}
	\Explain{ 4 Since every member of $C$ includes a member $A$ 
		it must be the case that $C$ is downwards directed and $\lim C = 0$}
	\Explain{ 5 On the other hand $\lim_{c \in C} |f(c)| \ge \varepsilon$}
	\Explain{ 6 And this contradicts the fact of $f \in \Caf(A)$}
	\EndProof
	\\
	\Theorem{Continuity}
	{
		\forall A \in \BOOL \. 
		\forall f \in \Caf_+(A) \. 
		\oC(A,\Reals, f) 
	}
	\NoProof
	\\
}\Page{
	\Theorem{Restriction}
	{
		\forall A \in \BOOL \.
		\forall f \in \Caf(A) \.
		\forall a \in A \.
		\Lambda c \in A \. f(ac)  \in \Caf(A)
	}
	\NoProof
	\\	
	\Theorem{DominatedCAF}
	{
		\forall A \in \BOOL \.
		\forall f \in \af(A) \.
		\forall g \in \Caf(A) \.
		|f| \le g \Imply f \in \Caf(A)
	}
	\NoProof
	\\
	\Theorem{CCCUpgrade}
	{
		\forall A : \CCC \.
		\caf(A) = \Caf(A)
	}
	\Explain{ 1 Take $f \in \caf(A)$}
	\Explain{ 2 Assume $D$ is downwards directed in $A$ with $\inf D = 0$}
	\Explain{ 3 Then there is a countable $C \subset D$ with $\inf C = 0$
		as $A$ is CCC}
	\Explain{ 4 Let $c$ be an enumeration of $C$ with $\lim_{n \to \infty} c_n = 0$}
	\Explain{ 5 Then it is possible to construct a sequence $d \in D$ such that 
		$d_n \le \bigvee^n_{k=1} c_k$}
	\Explain{ 6 Thus $ \inf f(  D ) \le \lim_{n \to infty} f(d_n) = 0$ }
	\Explain{ 7 and so $f \in \Caf(A)$}
	\EndProof
	\\
	\Theorem{BoundedBySup}
	{
		\forall A \in \BOOL \.
		\forall f \in \af(A) \.
		\forall \aleph : \forall d : \TYPE{DisjointSequence}(A) \.  
		\sup_{m \in \Nat} |f(d_m)|  < \infty \.
		f \in \baf(A) 
	}
	\Explain{1 Assume $f$ is not bounded}
	\Explain{2 Then we can construct recursevely a  countale partition of unity 
		such that $\sup |f(p)| > \infty$}
	\Explain{2.1 Select $p_{0,1} = e$ }
	\Explain{2.2 On the step $n$ there can be we seek elemen $a$ with 
		$|f(a)| \ge n + |f(p_{n,n})|$}
	\Explain{2.3 Then we can assert that $a \le p_{n,n}$}
	\Explain{2.3.1 Define $p_{n+1,k} = p_{n,k}$ fo $k < n$} 
	\Explain{2.3.2 Define $p_{n+1,n} = p_{n,n} \setminus a$ and $p_{n+1,n+1} = a$}
	\Explain{2.3.3 Then $\bigvee^{n+1}_{k=1} p_{n+1,k} = 
	\bigvee^{n-1}_{k=1} p_{n,k}  \vee (p_{n,n} \setminus a) \vee (a) =
	\bigvee^n_{k=1} p_{n,k}	
	$}
	\Explain{2.3.4 Also $|f(p_{n+1,n})| \ge n > n-1$}
	\Explain{2.3.5 Then either $\sup |f(p_{n+1,n})| = \infty$ or $\sup |f(p_{n+1,n+1})| = \infty$}
	\Explain{2.3.6 In the first case swap $p_{n+1,n}$ and $p_{n+1,n+1}$}
	\Explain{2.4 As every element $p_{\bullet,k}$  of the fixed index $k$ changes atmost 2 times 
		we can construct an infinite disjoint sequence $d$}
	\Explain{ 2.5 Then $|f(d_n)| \ge n-1$, so $\sup_{m \in \Nat} |f(d_m)|  = \infty$}
	\Explain{ 3 This contradicts $(\aleph)$}
	\EndProof	
}\Page{
	\Theorem{JordanDecomposition1}
	{
		\forall A \in \BOOL \.
		\Caf(A) \subvec{\Reals} \baf(A)
	}
	\Explain{ 1 Assume that $d$ is a disjoint sequence in $A$}
	\Explain{ 2 Define $D = \{ a \in A : \exists N \in \Nat \. \forall n \in \Nat \. n \ge N \Imply   
		 d_n \le a \}$}
	\Explain{ 3 Then $D$ is downwards directed}
	\Explain{ 4 Also as $d_n^\c \in D$ and $d$ is disjoint it follows that $\bigwedge D = 0$}
	\Explain{ 5 So it follows that there is $a \in D$ such 
		that $|f(c)| \le 1$ for all $c \le a$}
	\Explain{ 6 But this means that $|f(d_n)| \le 1$ for a cofinite set of indexes}
	\Explain{ 7 Thus, $d$ is bounded}
	\Explain{ 8 So, as $d$ was arbitrary, by the previous theorem $f$ is also bounded}
	\EndProof
\\
	\Theorem{JordanDecomposition2}
	{
		\forall A \in \BOOL \.
		\forall f \in \Caf(A) \. f_+,f_- \in \Caf(A)
	}
	\Explain{ 1 Write $f_+(a) = \sup \{ f(c) | c \in A, c \le a\}$}
	\Explain{ 2 Assume $D$ is a downwards directed set with $\bigwedge D = 0$}
	\Explain{ 3 Also assume $\varepsilon \in \Reals_{++}$}
	\Explain{ 4 We know that there is $d \in D$ such that $|f(c)| \le \varepsilon$ for all $c \le d$}
	\Explain{ 5 So $\inf_{u \in D} f_+(u) \le f_+(d) \le \varepsilon$}
	\Explain{ 6 Thus, $\inf_{d \in D} f_+(d) = 0$ and $f_+ \in \Caf(A)$}
	\Explain{ 7 The same argument holds for $f_-$}	
	\EndProof
}\Page{
	\Theorem{UnitySummability}
	{
		\NewLine ::		
		\forall A \in \BOOL \.
		\forall f : A \to \Reals \.
		f \in \Caf(A) \iff
		\forall P : \PoU(A) \.
		f(e) = \sum_{p \in P} f(p)
	}
	\Explain{ 1 $(\Rightarrow)$ Assume $f \in \Caf(A)$}
	\Explain{ 1.1 Transfinite induction on $|J|$ with trivial base $f(e) = f(e)$}
	\Explain{ 1.1.1 Assume that the result holds for some non-limit ordinal $\kappa$}
	\Explain{ 1.1.2 Consider an ordering $p$ of $P$ with cardinality equivalent to $\kappa + 1$}
	\Explain{ 1.1.3 Let $f' \in \Caf(A)$ be a restiction of $f$ to $p_{\kappa + 1}^\c$}
	\Explain{ 1.1.4 Also define $Q$ to be qual to $P$ but with $p_{\kappa +1}$ and $p_{\kappa}$
		replaced by $p_\kappa \vee p_{\kappa + 1}$}
	\ExplainFurther{ 1.1.5 Then by induction hypothesis}
	\Explain{ 
		$ \quad \quad f(e) = f'(e)  + f(p_{\kappa + 1}) =  f(p_{\kappa + 1}) + \sum_{q \in Q} f'(q) = 
		f(p_{\kappa + 1}) + \sum_{\tau \le \kappa} f(p_\tau) = \sum_{q \in P} f(q)$}
	\Explain{ 1.2 Now let $\kappa$ be a limit cardinal and that induction hypothesis holds for 
		all $\tau  < \kappa$}
	\ExplainFurther{ 1.2.1 Note that $\sum_{p \in P} f(p)$
		converges unconditionally to $f(e)$ iff for any $\varepsilon \in \Reals_{++}$
		there is finite $F \subset P$}
	\Explain{ $\quad \quad$ such that $\left| f(e) - \sum_{p \in G} f(p)\right| \le \varepsilon$ 
		for any finite $G$ with $F \subset G$}
	\Explain{ 1.2.2 Consider a set 
		$D = \left\{ e \setminus \bigvee_{p \in F} p \bigg| F : \Finite(P) \right\}$}
	\Explain{ 1.2.3 Then, as $P$ is a partition of unity $\bigwedge D = 0$}
	\Explain{ 1.2.4 Also $D$ is downwards directed as $\bar a \wedge \bar b = \overline{a \vee b}$}
	\Explain{ 1.2.5 So there is $d \in D$ such that $|f(c)| \le \varepsilon$ for all $c \le d$}
	\Explain{ 1.2.6 Represent $d = e \setminus \bigvee_{ p \in F} p$ for some finite $F \subset P$}
	\Explain{ 1.2.7 Take some finite $G \subset P$ such that $F \subset G$}
	\Explain{ 1.2.8 Then $\left|f(e) - \sum_{p \in G} f(p) \right| = 
		\left| f(e \setminus \bigvee_{p \in G} p \right| < \varepsilon$ as 
		$e \setminus \bigvee_{p \in G} p \le e \setminus \bigvee_{p \in F} p  $ }
	\Explain{ 2 $(\Leftarrow)$ Now consider the case then the second condition holds}
	\Explain{ 2.1 for any disjoint $D \subset A$ with $\bigvee D = a$ it holds that
		$f(a) = \sum_{d \in D} f(d)$}
	\Explain{ 2.1.1 Consider a  a partiotion of unity $P = D \cup \{\bar a\}$}	
	\Explain{ 2.1.2 Then $ \sum_{p \in P} f(p) = f(e) = f(a) + f(\bar a) $}
	\Explain{ 2.1.3 By substraction $f(\bar a)$ one gets $\sum_{d \in D} f(d) = f(a)$}
	\Explain{ 2.2 $f \in \af(A)$}
	\Explain{ 2.2.1 Consider $a,c \in A$ such that $ac = 0$}
	\Explain{ 2.2.2 Then $f(a \vee c) = f(a) + f(c)$ by $(2.1)$}
}\Page{
	\Explain{ 2.3 $f \in \baf(A)$ }
	\Explain{ 2.3.1 Assume $d : \Nat \to A$ is disjoint}
	\Explain{ 2.3.2 Let $\mathcal{D}$ be a set of all disjoint sets $D$ with $\im d \subset D$}
	\Explain{ 2.3.3 Then By Zorn Lemma there is an upper bound $P$ which must be a partition of unity}
	\Explain{ 2.3.4 Then $f(e) = \sum_{p \in P} f(p)$}
	\Explain{ 2.3.5 But this means the $\lim_{n \to \infty} f(d) = 0$}
	\Explain{ 2.3.6 As $d$ was arbitrary $f(d)$ is bounded}
	\Explain{ 2.4 Now it is possible to write $f = f_+ - f_-$}
	\Explain{ 2.5 Then $\sup_{d \in D} f_+(d) = f(a)$ for a disjoint set $D$ with $\bigwedge D =  a$}
	\Explain{ 2.5.1 Assume $D$ is such disjoint set}
	\Explain{ 2.5.2 Then $f(b) = \sum_{d \in D} f(bd) \le \sum_{d \in D} f_+(d)$ for any $b \le a$}
	\Explain{ 2.5.3 So by taking supremum $f_+(a) \le \sum_{d \in D } f_+(d)$ }	
	\Explain{ 2.5.4 But also $\sum_{d \in D } f_+(d) = 
		\sup \left\{ \sum_{d \in F} f(d) \bigg|  F : \Finite(D)  \right\}  =
		\sup \left\{ f\left( \bigvee F \right)  \bigg|  F : \Finite(D)  \right\} \le f_+(a)
		$   }
	\Explain{ So $\sum_{d \in D } f_+(d) = f_+(a)$}
	\Explain{ 2.6 $f_+ \in \Caf(A)$  }
	\Explain{ 2.6.1 Assume $D$ is a downwards directed set with $\bigwedge D = 0$  }
	\Explain{ 2.6.2 Let $C = \{a \in A : \exists d \in D : da = 0 \}$ }
	\Explain{ 2.6.3  Then $C$ is order dense}
	\Explain{ 2.6.4 So it is possible to extract a partition of Unity $P \subset C$}	
	\Explain{ 2.6.5  $ \sum_{p \in P} f_+(p) = f_+(e)$ by (2.5)}
	\Explain{ 2.6.6 So for any $\varepsilon \in \Reals_{++}$
		 there is some finite $F \subset C$ such that 
		 $ f_+\left( e \setminus \bigvee F \right) = f_+(e) - \sum_{p \in F} f_+(p) < \varepsilon $}	
	\Explain{ 2.6.7  By construction of $C$ there is a $d \in D$ such that $d \le e \setminus \bigvee F$} 
	\Explain{ 2.6.8 Therefore $f_+(d) < \varepsilon$}
	\Explain{ 2.6.9 $\inf_{d \in D} f_+(d) = 0$ as $\varepsilon$ was arbitrary}
	\Explain{ 2.7 $f_- \in \Caf(A)$  by simmilar arguments}
	\Explain{ 2.8 So $f \in \Caf(A)$}	
	\EndProof
	\\
	\Theorem{Summability}
	{
		\NewLine ::		
		\forall A \in \BOOL \.
		\forall f  \in \Caf(A) \.
		\forall D : \TYPE{Disjoint}(A) \.
		\forall a \in A \
		a = \bigvee D \Imply
		f(a) = \sum_{d \in D} f(d)
	}
	\Explain{ This is a part of the previous theorem}
	\EndProof
}\Page{
	\Theorem{StrictHahnDecomposition}
	{
		\NewLine ::		
		\forall A \in \BOOL \. 
		\forall f \in \Caf(A) \.
		\exists! q \in A \.
		\forall c \in C \.   0 < c \le q \Imply f(c) > 0 \And c \le \bar(q) \Imply f(c) \le 0
	}
	\Explain{ 1 Define $C_+ = \{ a \in A : 0 < c \le a \Imply f(c) > 0  \}$ 
		and $C_- = \{ a \in A : c \le a \Imply f(c) \le 0 \}$}
	\Explain{ 2 $C_+ \cup C_-$ is order dense}
	\ExplainFurther{ 2.1 By ordinary Hahn decomposition there is $a' \in A$
		such that $f(c) \ge 0$ for all $c \le a'$}
	\Explain{ $\quad \quad$ and $f(c) \le 0$ for all $c \le \overline{a'}$}
	\Explain{ 2.2  $ a\overline{a'} \in C_- $ for any $a \in A$ such that $a\ neq 0$}
	\Explain{ 2.3 In case $a\overline{a'} = 0$ it must be the case that $a \le a'$}
	\Explain{ 2.4 If $a \not \in C_+$ there must be some $d \le a$ such that $f(d) \le 0$}
	\Explain{ 2.5 But $d \le a'$, so $f(d) = 0$ and $f(c) = 0$ for any $c \le d$}
	\Explain{ 2.6 So $d \in C_-$ and $ad \neq 0$}
	\Explain{ 3 So there is a partition of unity $P \subset C_+ \cup C_- $}
	\Explain{ 4 $P \cap C_+$ is countable}
	\Explain{ 4.1 The series $\sum_{p \in P} f(p)$ must be absolutely convergent}
	\Explain{ 4.2 So any subseries of $\sum_{p \in P} f(p)$ must be strictly convergent}
	\Explain{ 4.3 This includes $\sum_{p \in P \cap C_+} f(p)$}
	\Explain{ 4.4 But $f(p) > 0$ any element $P \cap C_+$, so there can be 
		atmost countable number of such elements}
	\Explain{ 5 Element $q = \bigvee (P \cap C_+)$ exists}
	\Explain{ 6 Clearly $f(a) = 
		f\left( \bigvee_{p \in P \cap C_+} ap \right) = 
		\sum_{p \in p \cap C_+} f(ap) > 0$ for any $a \le q$ such that $a \neq 0$}
	\Explain{ 7 Then $f(a) \le 0$ if $a \le \bar{q}$}
	\Explain{ 8 $q$ is unique}
	\Explain{ 8.1 Assume $p$ has same properties as $q$}
	\Explain{ 8.2 But then $f(p\setminus q) \le 0$ and $f(q \setminus p) = 0$ 
		meaning that $p \setminus q = q \setminus p = 0$}
	\Explain{ 8.3 Thus $p = q$}
	\EndProof
	\\
	\DeclareFunc{saturation}
	{
		\prod_{A \in \BOOL} \Caf^2(A) \to A
	}
	\DefineNamedFunc{saturation}{f,g}{[f > g]_A}{\THM{StrictHahnDecomposition}(A, f-g)}	
}
\newpage
\subsubsection{Absolutely Continuous Additive Functionals}
\Page{
	\DeclareType{AbsolutelyContinuousAdditiveFunctional}
	{
		\NewLine ::		
		\prod (A,\mu) \in \ma \.  ?\af(A) 
	}
	\DefineNamedType{f}{AbsolutelyContinuousAdditiveFunctional}{f \in \ac(A,\mu)}
	{
		\NewLine \iff		
		\forall \varepsilon \in \Reals_{++} \.
		\exists \delta \in \Reals_{++} \.
		\forall a \in A \. \mu(a) \le \delta \Imply |f(a)| \le \varepsilon 
	}
	\\
	\Theorem{ContinuousIsCompletelyAdditive}
	{
		\forall (A,\mu) \in \ma \.
		\forall f \in \af(A) \.
		f \in C_0(A) \Imply f \in \Caf(A)
	}
	\Explain{ 1 Assume $D$ is a downwards directed in $A$ with $\bigwedge D = 0$}
	\Explain{ 2 Then $\lim_{d \in D} d = 0$ in a measure topology of $A$}
	\Explain{ 3 So by continuity $\lim_{d \in D} |f(d)| = 0$ so $\inf_{d \in D} |f(d)| = 0$ also}
	\Explain{ 4 But this means that $f \in \Caf(A)$}
	\EndProof
	\\
	\Theorem{CountablyAdditive}
	{
		\forall (A,\mu) \in \ma \.
		\caf(A) \subset \ac(A,\mu) 
	}
	\Explain{ 1 Take $f \in \caf(A) \setminus \ac(A,\mu)$}
	\ExplainFurther{ 2 Then there exists $\varepsilon \in \Reals_{++}$
		such that for all $\delta \in \Reals_{++}$ there is an 
		element $a \in A$} 
	\Explain{ $\quad\quad$ with $\mu(a) \le \delta$ anf $|f(a)| \ge \varepsilon$}
	\Explain{ 3 Select a sequence $a : \Nat \to A$ 
		with $|f(a_n)| \ge \varepsilon$ and $\mu(a_n) \le 2^{-n}$}
	\Explain{ 4  Define a decreasing sequence $c_n = \bigvee_{k=n}^\infty a_k$}
	\Explain{ 5  Then $\mu(c_n) = 2^{1-n} \to 0$}
	\Explain{ 6  So $\lim_{n \to \infty} c_n = 0$ and $\bigvee^\infty_{n=1} c_n = 0$}
	\Explain{ 7  Thus, $\inf_{n \in \Nat} |f(c_n)| = 0$}
	\Explain{ 8 on the other hand $f(c_n) \ge \varepsilon$ which leads to a contradiction}
	\EndProof
	\\
	\DeclareType{QuasiSemifinite}{ 
		\forall (A,\mu) \in \ma \.  ?(A \to \Reals)	
	}
	\DefineType{\varphi}{QuasiSemifinite}
	{
		\forall a  \in A \. \varphi(a) \neq 0 \Imply
		\exists c \in A^f \. \varphi(ac) \neq 0
	}
}\Page{
	\Theorem{ContinuousIsQuasiSemifinite}
	{
		\NewLine :: 		
		\forall (A,\mu) \in \ma \.
		\forall f \in \af(A) \. \NewLine \.
		f \in C_0(A) \Imply  f \in \caf \And \TYPE{QuasiSemifinite}(A) 
	}
	\Explain{ 1 We know that $f \in \Caf(A)$, so $f \in \caf(A)$}
	\Explain{ 2 Assume $a \in A$ such that $f(a) \neq 0$}
	\Explain{ 3 Then $a \neq 0$ and $\mu(a) \neq 0$}
	\Explain{ 4 Assume $\mu(a) = \infty$} 
	\Explain{ 5 If $\mu(c) = \infty$ for any $c \in A$ such that $c \le a$ and $c \neq 0$
	  Then $\lim a = 0$}
	 \Explain{ 6 And so $f(a) = \lim f(a) = 0$, which is impossible}
	 \Explain{ 7 Therefore,  $\{ 0\} \subsetneq C =  A^f \cap (a)$}
	 \Explain{ 8 Let $D = \{ d \in A : d \le a \And  \forall u \in A \. 0 < u \le d \Imply \mu(u) = \infty \}$}
	 \Explain{ 9 Then $C \cup D$ is dense in $(a)$}
	 \Explain{ 10 Let $P \subset C \cup D$ be a partition of unity}
	 \Explain{ 11 Note that $f(d) = 0$ by arguments simmilar to $(5)$ and $(6)$
	 	for any $d \in D$}
	 \Explain{ 12 Thus, $0 \neq f(a) = \sum_{p \in P} f(p) = \sum_{p \in P \cap C} f(p)$}
	 \Explain{ 13 Therefore, there exists $c \in C$ such that $f(ac) = f(c) \neq 0$}
	 \EndProof
}\Page{
	 \Theorem{SigmaAdditiveAndQuasiSemifiniteIsUniformlyContinuous}
	 {
		\NewLine ::	 	
	 	\forall (A,\mu) \in \ma \.
	 	\forall f \in \af(A) \. 
	 	f \in \caf \And \TYPE{QuasiSemifinite}(A) 
	 	\Imply
	 	f \in \UNI\Big((A,\mu),\Reals \Big)
	 }
	 \Explain{ 1 $f$ is bounded, so there is a Jordan decomposition $f = f_+ - f_-$}
	 \Explain{ 2 Define $g = f_+ + f_-$ and $\gamma = \sup \{ g(a) | a \in A^f \}$}
	 \Explain{ 3 Then there is a sequence of elements $a : \Nat \to A^f$ 
	 	such that $\gamma = \lim_{n \to \infty} g(a_n)$}
	 \Explain{ 4 Let $a^* = \bigvee^\infty_{n=1} a_n$}
	 \Explain{ 5 If $d \in A$ and $a^*d = 0$ then $f(d) = 0$ }
	 \Explain{ 5.1 Assume $d \in A$ is such that $a^*d = 0$ and $c \in A^f$}
	 \Explain{ 5.2 Then $|f(cd)| \le g(cd) \le g(c \setminus a_n) = 
	  g(a_n \vee c) - g(a_n) \le \gamma - g(a_n)$}
	 \Explain{ 5.3 By taking the limit we see that $|f(cd)| = 0$
	 	and hence $f(cd) = 0$}
	 \Explain{ 5.4 As $c$ was arbitrary as $f$ is quasi-semifinite $f(d)=0$}
	 \Explain{ 6 Construct the sequence $c^*_n = \bigvee^\infty_{k=n} a_n$}
	 \Explain{ 7 Then $\lim_{n \to \infty} g(a^* \setminus c^*_n) = 0$}
	 \Explain{ 8 As $f$ is countably additive it must be absolutely continuous}
	 \Explain{ 9 Assume $\varepsilon \in \Reals_{++}$,
	 	then there is $\delta$ such that $|f(a)| \le \varepsilon$ having $\mu(a) < \delta$
	 	for all $a \in A$}
	\Explain{ 10 Assume $n \in \Nat$ is such that $|g(a^* \setminus c^*_n)| < \varepsilon$}
	\Explain{ 11 Then  
		$|f(a)| \le | f(ac^*_n)| + |f(a(a^* \setminus c^*_n))| + |f(a \setminus a^*)|
		\le | f(ac^*_n)|  + g(a^*\setminus c^*_n) \le  | f(ac^*_n)|  + \varepsilon$
		for any $a \in A$}
	\Explain{ 12 Assume $a,c \in A$ such that $\mu((b+c)c^*_n) < \delta$ }	
	\Explain{ 13 Then
		$
			|f(a) - f(c)|  \le |f(a \setminus c)|   + |f(c \setminus a)| \le 
			| f((a \setminus c)c^*_n)| +   | f((a \setminus d)c^*_n)| + 2\varepsilon  
			\le 4\varepsilon 		
		$}
	\Explain{ 14 But this means that $f$ is uniformly continuous}
	\EndProof
	\\
	\Theorem{AdditiveFunctionalContinuity}
	{
		\forall (A,\mu) \in \ma \.
		\forall f \in \af(A) \.
		f \in C_0(A) \iff f \in \UNI(A,\Reals)
	}
	\Explain{This follows from the previous theorems}
	\EndProof
}\Page{
	\Theorem{SemifiniteAdditiveFunctionalContinuity}
	{
		\NewLine ::		
		\forall (A,\mu) : \Semifinite \.
		\forall f \in \af(A) \.
		f \in \Caf(A) \iff f \in \UNI(A,\Reals)
	}
	\Explain{ 1 $(\Leftarrow)$ is obvious}
	\Explain{ 1.1 $f$ is continuous in zero an hence completely additive}
	\Explain{ 2 $(\Rightarrow)$ consider $f \in \Caf(A)$}
	\Explain{ 2.1 I will show that $f$ is quasi-Semifinite}
	\Explain{ 2.1.1 Assume $a \in A$ is such that $f(a) \neq 0$}
	\Explain{ 2.1.2 Then $a \neq 0$}
	\Explain{ 2.1.3  $\{0\} \neq C = (a) \cap A^f$ is dense in $(a)$
		as $\mu$ is semifinite}
	\Explain{ 2.1.4 Let $P \subset C$ be a partiotion of unity for $(a)$}
	\Explain{ 2.1.5 Then $0 \neq f(a) = \sum_{p \in P} f(p)$}
	\Explain{ 2.1.6 So there must be $c \in C$ such that $f(c) \neq 0$}
	\Explain{ 2.2 As $f$ is also countably additive it must be uniformly continuous}
	\EndProof
	\\
	\Theorem{SigmaFiniteAdditiveFunctionalContinuity}
	{
		\NewLine ::		
		\forall (A,\mu) : \sFinite \.
		\forall f \in \af(A) \.
		f \in \Caf(A) \iff f \in \UNI(A,\Reals) \iff f \in \caf(A)
	}
	\Explain{1 $\sigma$-finite measure algebras are CCC}
	\Explain{2 So every countably additive functional must be completely additive}
	\EndProof
	\\
	\Theorem{FiniteAdditiveFunctionalContinuity}
	{
		\NewLine ::		
		\forall (A,\mu) : \Finite \.
		\forall f \in \af(A) \. \NewLine \.
		f \in \Caf(A) \iff f \in \UNI(A,\Reals) \iff f \in \caf(A) \iff f \in \ac(A,\mu)
	}
	\Explain{ 1 Assume $f$ is absolutely continuous with repsect to $\mu$}
	\Explain{ 2 Also assume $D$ is downwards directed in $A$ with $\bigvee D = 0$}	
	\Explain{ 3 So $\inf_{d \in D} \mu(d) = 0$}
	\Explain{ 3.1 This argument requires $\mu$ to be finite}
	\Explain{ 4 By absolute continuity $\inf_{d \in D} |f(d)| = 0$}
	\Explain{ 5 As $D$ was arbitraty this means that $f$ is completely continuous }
	\EndProof
	\\
	\DeclareFunc{zeroIdealRespectingAdditiveFunctionals}
	{
		\prod (X,\Sigma,\mu) \in \MEAS \. \TYPE{VectorSubspace}(\af(X,\Sigma,\mu))	
	}
	\DefineNamedFunc{zeroIdealRespectingAddivieFunctionals}{}
	{\af_0(X,\Sigma,\mu)}{\{ f \in \af(X,\Sigma,\mu) : \forall Z \in \Null_\mu \. f(Z) = 0 \}}
}\Page{
	\DeclareFunc{canonicalAdditiveFunctionalsIsomorphism}
	{
		\prod (X,\Sigma,\mu) \in \MEAS \.   \NewLine \.
		\TYPE{Isomorphism}\Big( \VS{\Reals}, \af\big( \ma(X,\Sigma,\mu) \big), \af_0(X,\Sigma,\mu) \Big)	
	}
	\DefineNamedFunc{canonicalAdditiveFunctionalsIsomorphism}{f}
	{\varphi(f)}{  f \circ \pi_{\Null_\mu} }
	\Explain{ 1 The Mapping $\varphi$ is cleary injective}
	\Explain{ 2 It is also bijective}
	\Explain{ 2.1 Assume $f \in \af_0(X,\Sigma,\mu) $ }
	\Explain{ 2.2 Then there is an auxilary functional $\bar f$ 
		defined by $\bar f [E] = E$}
	\Explain{ 2.3 This is well defined as $f$ respects the ideal od zero sets}
	\Explain{ 2.4 Then obviously $\varphi(\bar f) = f$}
	\EndProof
	\\
	\Theorem{Isomorphism}
	{
		\forall (X,\Sigma,\mu) \in \MEAS \.
		\forall f \in \af\big(\ma(X,\Sigma,\mu) \big) \.
		f \in \ac\big(\ma(X,\Sigma,\mu) \big) \iff
		\varphi(f) \in \ac(X,\Sigma,\mu) 
	}
	\Explain{This is obvious }
	\EndProof
	\\
	\Theorem{IsomorphismCountablyAdditive}
	{
		\NewLine ::		
		\forall (X,\Sigma,\mu) \in \MEAS \.
		\forall f \in \af\big(\ma(X,\Sigma,\mu) \big) \.
		f \in \caf\big(\ma(X,\Sigma,\mu) \big) \iff
		\varphi(f) \in \caf(X,\Sigma,\mu) 
	}
	\Explain{This is obvious }
	\EndProof
	\\
	\Theorem{IsomorphismCountablyAdditive}
	{
		\NewLine ::		
		\forall (X,\Sigma,\mu) \in \MEAS \.
		\forall f \in \af\big(\ma(X,\Sigma,\mu) \big) \.
		f \in \caf\big(\ma(X,\Sigma,\mu) \big) \iff
		\varphi(f) \in \ac(X,\Sigma,\mu) \cap \caf(X,\Sigma,\mu) 
	}
	\Explain{This is obvious}
	\EndProof
	\\
	\Theorem{IsomorphismTruelyContinuous}
	{
		\NewLine ::		
		\forall (X,\Sigma,\mu) \in \MEAS \.
		\forall f \in \af\big(\ma(X,\Sigma,\mu) \big) \.
		f \in C\big(\ma(X,\Sigma,\mu) \big) \iff
		\varphi(f) \in \tc(X,\Sigma,\mu) 
	}
	\Explain{This is obvious }
	\EndProof
	\\
	\Theorem{SemifiniteIsomorphismTruelyContinuous}
	{
		\NewLine ::		
		\forall (X,\Sigma,\mu) : \Semifinite \.
		\forall f \in \af\big(\ma(X,\Sigma,\mu) \big) \.
		f \in \Caf\big(\ma(X,\Sigma,\mu) \big) \iff
		\varphi(f) \in \tc(X,\Sigma,\mu) 
	}
	\Explain{This is obvious }
	\EndProof
}
\newpage
\subsubsection{Radon-Nikodym's Isomorphism}
\Page{
	\DeclareFunc{isomorphismOfRadonNikodym}
	{
		\prod (X,\Sigma,\mu) : \Semifinite \.   \NewLine \.
		\TYPE{Isomorphism}\Big( \mathsf{OVS}, 
		\mathbf{L}^1(X,\Sigma,\mu),  \Caf\big( \ma(X,\Sigma,\mu) \big) \Big)	
	}
	\DefineNamedFunc{isomorphismOfRadonNikodym}{[f]}
	{\rho\nu[f]}{ \Lambda [E] \in \Sigma_\mu \.  \int_E f \; d\mu  }
	\ExplainFurther{ 1 The expression $\int_E f \; d\mu$ above is clearly well defined 
		for an integrable $f$ as $[E]$}
	\Explain{ $\quad\quad$ is defined up to a set of the measure zero}
	\ExplainFurther{ 2 $[f]$ Is also defined up to function $g$ equal to $0$ almost everywhere $\mu$} 
	\Explain{ $\quad\quad$ so the whole operator $\rho\nu$ is well defined}
	\Explain{ 3 $\rho \nu$  is invertible}
	\Explain{ 3.1 Assume $f \in \Caf(\Sigma_\mu,\bar \mu)$}
	\Explain{ 3.2 Then $\varphi(f)$ is truely continuous additive functional on $(X,\Sigma,\mu)$  
		as this space is semifinite}
	\ExplainFurther{ 3.3 So by classical Radon-Nikodym's theorem
		there is $\frac{d \varphi(f)}{d \mu} \in L^1(X,\Sigma,\mu)$}
	\Explain{$\quad \quad$ such that 
	$\varphi(f)(E) = \int_E  \frac{d \varphi(f)}{d \mu} \; d\mu$ for any $E \in \Sigma$}
	\Explain{ 3.4 So define 
	$(\rho \nu)^{-1}(f) = \left[ \frac{d \varphi(f)}{d \mu} \right] \in \mathbf{L}^1(X,\Sigma,\mu)$   }	
	\Explain{ 4 The linearity and order preservation is pretty obvious for 
		$\rho\nu$}
	\EndProof
	\\
	\ExplainFurther{Question: Is this a natural equivalence of functors?}
}
\newpage
\subsubsection{Standard Extension}
\Page{
	\Theorem{CompleteleAdditiveRestriction}
	{
		\NewLine ::		
		\forall (A,\mu) : \MA \.
		\forall a \in A^f \.
		(\Lambda c \in A \. \mu(ac)) \in \Caf(A) 
	}
	\Explain{1 As $\mu$ is additive its restriction is also additive}
	\Explain{2 Assume $D$ is a downwards directed in $A$ with $\bigwedge D = 0$}
	\Explain{3 Then $aD$ is still downwards directed in $A^f$ and $\bigwedge aD = 0 $}
	\Explain{4 Note, that by choice of $a$ the restriction is finitely additive}
	\Explain{5 Then $\inf_{d \in D} \mu(ad) = \inf_{d \in aD} \mu(d) = 0$}
	\EndProof
}\Page{
	\Theorem{StandardExtensionLemma}
	{
		\NewLine ::		
		\forall (A,\mu) \in \Finite\MA \.
		\forall (C,\nu) \subset_\ma (A,\mu) \.
		\exists! \caf(C,\nu) \Arrow{R} \caf(A,\mu) : \mathsf{OVS} \. \NewLine \.
		\forall f \in \caf(C,\nu) \.
		\forall \alpha \in \Reals \.
		R(f)_{|C} = f \And
		[R(f) > \alpha \mu]   =   [f > \alpha \nu]
	}
	\Explain{ 1 Represent $(A,\mu)$ as a measure algebra of the
		measure space $(X,\Sigma,\hat \mu)$}
	\Explain{ 2 Then $(C,\nu)$ can be seen as a measure algebra of the  
		measure space $(X, T, \hat \nu)$  with $T \subset \Sigma$ 
		and $\hat \nu = \hat \mu_{|T}$}
	\Explain{ 3 $\varphi(f) \in \tc(X, T ,\hat \nu)$  for any $f \in \caf(C,\nu)$ }
	\Explain{ 4 So there is a Radon-Nikodym presentation $\phi = \rho\nu^{-1}(f)$
		such that $f[E] = \int_E \phi \; d\hat\nu$ for any $E \in T$}
	\Explain{ 5 Define $R(f)[E] = \int_E \phi \; d\hat\mu$ 
		for any $E \in \Sigma$}
	\Explain{ 6 $[R(f) > \alpha \mu] \in C$}
	\Explain{ 6.1 Define level sets $H_\alpha = \{ x \in X : \phi(x) > \alpha \} \in T$}
	\Explain{ 6.2 $\int_E \phi \; d\mu > \alpha \hat \mu(E) $
		 if $E \subset H_\alpha$ and $\hat \mu(E) > 0$ for any $E \in \Sigma$}
	\Explain{ 6.3 $\int_E \phi \; d\mu \le \alpha \hat \mu(E) $
		 if $E \cap H_\alpha = \emptyset$ for any $E \in \Sigma$}
	\Explain{ 6.4 This can be rewritten in terms of measur algebras}
	\Explain{ 6.5 $R(f)(a) > \alpha \mu(a)$ if $a \le [H_\alpha]$ and $a \neq 0$  
		for any $a \in A$}
	\Explain{ 6.6 And $R(f)(a) \le \alpha \mu(a)$ if $a[H_\alpha] = 0$ 
		for any $a \in A$}
	\Explain{ 6.7 Thus, $[R(f) > \alpha \mu] = [H_\alpha] \in C$}	
	\Explain{ 7 Clearly $R(f)_{|C} = f$}
	\Explain{ 8 Note, that $\mu_{|C} = \nu$}
	\Explain{ 9 Therefore, $[R(f) > \alpha \mu]   =   [f > \alpha \nu]$
		for all $\alpha \in \Reals$}
	\Explain{ 10 $R$ is uinquily detrermined}
	\Explain{ 10.1 Assume $g$ has all required properties}
	\Explain{ 10.2 Then $[g > \alpha \mu] = [f > \alpha \nu] = [R(f) > \alpha \mu]$
		for all $\alpha \in \Reals$}
	\Explain{ 10.3 Then there is a measurable function $\gamma : X \to \Reals$ 
		such that $\int_E \gamma \;d \mu = g[E]$ for any $E \in \Sigma$}
	\Explain{ 10.4 But then level sets of $\gamma$ equal to level sets of $\phi$
		modulo $\mu$-measure zero}
	\Explain{ 10.5 So $\gamma = \phi$ $\mu$-almost everywhere}
	\Explain{ 10.6 Thus $R(f) = g$}
	\EndProof
	\\
	\DeclareFunc{standardExtension}
	{
		\prod (A,\mu)  : \Finite\MA \. \prod (C,\nu) \subset_{\ma} (A,\mu) \.
		\mathsf{OVS}\Big( \caf(C,\nu), \caf(A,\mu) \Big)
	}
	\DefineNamedFunc{standardExtension}{f}{R(f)}
	{\THM{StandardExtensionLemma}\Big( (A,\mu), (C,\nu), f \Big)}
	\\
	\Theorem{Measure}
	{
		\forall (A,\mu)  : \Finite\MA \. \forall (C,\nu) \subset_{\ma} (A,\mu) \.
		R(\nu) = \mu
	}
	\Explain{In this case $\phi(x) = 1$}
	\EndProof
}\Page{
	\Theorem{PartitionOfUnity}
	{
		\NewLine ::		
		\forall (A,\mu)  : \Finite\MA \. \forall (C,\nu) \subset_{\ma} (A,\mu) \.
		\forall f : \Nat \to \caf_+(C,\nu) \. \NewLine \.
		\forall \aleph : \forall c \in C \. \nu(c) = \sum^\infty_{n=1} f_n(c) \.
		\forall a \in A \. \mu(a) = \sum^\infty_{n=1} R(f_n)(a)
	}
	\Explain{ 1 Let $\phi : \Nat \to L^1(X,T,\hat\nu)$ be functional representations for
		$f_n$ as in the previous theorem } 
	\Explain{ 2 Define $\gamma_n  = \sum^n_{k=0} \phi_k$}
	\Explain{ 3 Then $\gamma$ is an increasing sequence such that 
		$\lim_{n \to \infty} \gamma_n = 1$ almost everywhere $\mu$}
	\Explain{ 4 But then  
		$\sum^\infty_{n=1} R(f_n)[E] = 
		 \sum^\infty_{n=1} \int_E \phi_n \; d\hat \mu  =
		 \lim_{n \to \infty} \int_E \gamma_n \; d \hat \mu =
		 \int_E \lim_{n \to \infty} \gamma_n \; d\hat \mu =
		 \int_E  d\hat \mu =  \mu [E]  
		$}
	\Explain{4.1  Here we used monotonic convergence theorem}
	\EndProof
}
\newpage
\subsection{Category of Probility Algebras}
\subsubsection{Reduced Products}
\Page{
	\Theorem{ReducedProductExists}
	{
		\NewLine ::		
		\forall I \in \SET \.
		\forall (A,p) \to \PAlg \.
		\forall \F : \TYPE{Ultrafilter}(I) \. \NewLine \. 
		\exists P \in \af\left( \frac{\prod_{i \in I}A_i}{J} \right) \.  
	 	\forall a \in \prod_{i \in I} A_i \.
	 	P[a] = \lim_{i \to \F} p_i(a_i)  
		\NewLine
		\quad \quad \where \quad  J = \left\{ a_i \in \prod_{i \in I} : \lim_{i \to \F} p_i(a_i)  = 0   \right\}
	}
	\Explain{1 Clearly $J$ is ideal}
	\Explain{1.1 If $a,b \in J$ then $a + b \in J$}
	\Explain{1.1.1 $ \lim_{i \to \F} p_i(a_i +b_i) \le 
		\lim_{i \to \F} p_i(a_i) + p_i(b_i)  = 
		\lim_{i \to \F} p_i(a_i) + \lim_{i \to \F} p_i(b_i) = 0		
		 $  }
	\Explain{1.1.2 So $ \lim_{i \to \F} p_i(a_i +b_i) = 0$ as each $p_i \ge 0$}
	\Explain{1.2 if $a \in J$ and $b \in \prod_{i \in I} A_i$  then $ab \in J$}
	\Explain{1.2.1 $ \lim_{i \to \F} p_i(a_i b_i) \le 
		\lim_{i \to \F} p_i(a_i) = 0		
		 $  }
	\Explain{1.2.2 So $ \lim_{i \to \F} p_i(a_i +b_i) = 0$ as each $p_i \ge 0$}
	\Explain{2 $P[a] = \lim_{i \to \F} p_i(a_i)$ is well defined}
	\Explain{2.1 The system $\lim_{i \to \F} p_i(a_i)$
		must be convergent $p_i(a_i) \in [0,1]$ which is compact and 
		$\F$ is an ultrafilter}
	\Explain{2.2 Clearly $[a]$ is defined up to a $j \in J$}
	\Explain{2.3 And $\lim_{i \in \F} p_i(j_i)  = 0$}
	\ExplainFurther{2.4 Thus, $ 
			\lim_{i \in \F} p_i(a_i) = 			
			\lim_{i \in \F} p_i(a_i) - \lim_{i \in \F} p_i(j_i) =
			\lim_{i \in \F} p_i(a_i) -  p_i(j_i) \le 
			\lim_{i \in \F} p_i(a_i + j_i) \le$}
	\Explain{ $ \quad \quad \le \lim_{i \in \F} p_i(a_i) +  p_i(j_i) =
			\lim_{i \in \F} p_i(a_i) + \lim_{i \in \F} p_i(j_i) =  \lim_{i \in \F} p_i(a_i)  $}
	\Explain{ 2.5 Showing that $\lim_{i \in \F} p_i(a_i + j_i) = \lim_{i \in \F} p_i(a_i) $ }
	\EndProof
}\Page{
	\DeclareFunc{reducedProduct}
	{
		\prod_{I \in \SET} (I \to \PAlg) \to \TYPE{Ultrafilter}(I) \to \PAlg
	}
	\DefineNamedFunc{reducedProduct}{(A,p),\F}{
		\left( \prod_{i\in I} (A_i,p_i)|\F, p_\F \right)}
	{
		\left( \frac{\prod_{i \in I} A_i}{J},
			\THM{ReducedProductExists}\Big( I, (A,\mu), \F \Big) \right) \NewLine \quad \quad \where \quad
		 J = \left\{ a_i \in \prod_{i \in I} : \lim_{i \to \F} p_i(a_i)  = 0   \right\}
	}
	\Explain{ 1 Clearly $ \prod_{i\in I} (A_i,p_i)|\F$ is and algebra and $p_\F$ is non-negative additive}
	\Explain{ 2 $p_\F$ is countably additive}
	\Explain{ 2.1 Assume $[a] : \Nat \to \frac{\prod_{i \in I} A_i}{J}$ is disjoint}
	\Explain{ 2.2 Then $\lim{i \to \F} p_i(a_{n,i}a_{m,i}) = 0$ 
		for each $n,m \in \Nat$ such that $n \neq m$}
	\Explain{ 2.3 Define $b_{n,i} = \bigvee_{k=1}^n a_{k,i}$}
	\Explain{ 2.4 Then $[b_n] = \bigvee^n_{k=1} [a_k]$}
	\Explain{ 2.5 Define $\gamma = \sum^\infty_{n=0} p_\F[a_n]  =  
		\sup_{n \in \Nat} p_\F[b_n] = \sup_{n \in \Nat} \lim_{i \to \F}  p_i(b_{n,i})$}
	\Explain{ 2.6 Define $R_n = \{ i \in I : p_i(b_{n,i}) \le \gamma + 2^{-n} \}$}
	\Explain{ 2.7 Then $R_n$ is non-decreasing in $I$ and $R_1 = I$ }
	\Explain{ 2.8 Each $R_n \in \F$}
	\Explain{ 2.8.1 Not that by 2.5 there is an $F \in \F$ such that $F \subset R_n$ for any $n \in \Nat$}
	\Explain{ 2.8.2 But $\F$ is upwards closed}
	\Explain{ 2.9 Define $c \in \prod_{i \in I} A_i$ 
		by setting $c_i =  b_{n-1,i}$ if $i \in R_{n-1} \setminus R_{n}$
		and $c_i = \bigvee^\infty_{n=1} b_{n,i}$ otherwise}
	\Explain{ 2.10 Then $[a_n] \le [c]$}
	\Explain{ 2.10.1   $a_n,i \le c $ on $R_n \in \F$}
	\Explain{ 2.10.2 But this means that 
		$p_\F([a_n] \setminus [c]) = \lim_{i \to \F} p_i(a_{n,i} \setminus c_i) = 0$}
	\Explain{ 2.11 Also $p_\F[c] \le \gamma$}
	\Explain{ 2.11.1 This follows from the definition of $R_n$}
	\Explain{ 2.12 Thus, by the standard argument $c = \bigvee^\infty_{n=1} a_n$}
	\Explain{ 3 $\prod_{i \in I} (A_i,\mu_i) | \F$ is $\sigma$-algebra}
	\Explain{ 3.1 Proof of (2) tells as how to construct a supremum}
	\EndProof 
	\\
	\DeclareFunc{reducedPower}
	{
		\prod_{I \in \SET} \PAlg \to \TYPE{Ultrafilter}(I) \to \PAlg
	}
	\DefineNamedFunc{reducedProduct}{(A,p),\F}{
		( (A,p)^I|\F, p_\F )}
	{
		\left( \prod_{i \in I}(A,p |\F, p_\F \right)
	}
}\Page{
	\Theorem{Morphisms}
	{
		\NewLine :: 		
		\forall I \in \SET \.
		\forall \F : \TYPE{Ultrafilter}(I) \.
		\forall (A,p),(B,q)  : I \to \PAlg \. \NewLine \.
		\forall \phi \in \prod_{i \in I} \ma_\#\Big((A_i,p_i),(B_i,q_i)\Big) \.
		\exists! \Phi \in \ma_\#\left( 
			\left( \prod_{i \in I} A_i | \F, p_\F\right),\left( \prod_{i \in I} B_i |\F, q_\F \right) \right) \.
		\NewLine \.
		\forall a \in \prod_{i \in I} A_i \. 
		\Phi[a] = \Big[( \phi_i(a_i) )_{i \in I} \Big]
	}
	\Explain{ 1 $\Phi[a]$ is well defined by relation above}
	\Explain{ 1.1 $a$ is determined modulo 
		$j \in J_A = \left\{ a_i \in \prod_{i \in I} A_i : \lim_{i \to \F} p_i(a_i)  = 0   \right\}$}
	\ExplainFurther{ 1.2 As $\phi$ are measure preserving 
	$\prod_{i \in I}\phi_i(j) \in J_B = 
	\left\{ b_i \in \prod_{i \in I} B_i : \lim_{i \to \F} q_i(a_i)  = 0   \right\}$}
	\Explain{ $\quad \quad$ having  $\lim_{i \to \F} q_i( \phi_i(j_i)) = \lim_{i \to \F} p_i(j_i) = 0$ }
	\Explain{2 Obviously $\Phi$ is unique}
	\EndProof
	\\
	\DeclareFunc{morphismReducedProduct}
	{
		\prod_{I \in \SET} \prod \F : \TYPE{Ultrafilter}(I) \. 
		\prod  (A,p),(B,q)  : I \to \PAlg \. \NewLine \.
		\prod_{i \in I} \ma_\#\Big((A_i,p_i),(B_i,q_i)\Big) 
		\to
		 \ma_\#\left( 
		 \left( \prod_{i \in I} A_i | \F, p_\F\right),\left( \prod_{i \in I} B_i |\F, q_\F \right) \right)
	}
	\DefineNamedFunc{morphismReducedProduct}{\phi}{ \phi_\F}
	{
		\Lambda [a] \in \prod_{i \in I} (A_i, p_i) | \F \. \Big[  (\phi_i(a_i))_{i \in I}  \Big] 
	}	
	\\
	\Theorem{Functoriality}
	{
		\NewLine :: 		
		\forall I \in \SET \.
		\forall \F : \TYPE{Ultrafilter}(I) \.
		\forall (A,p),(B,q),(C,u)  : I \to \PAlg \. \NewLine \.
		\forall \phi \in \prod_{i \in I} \ma_\#\Big((A_i,p_i),(B_i,q_i)\Big) \.
		\forall \psi \in  \prod_{i \in I} \ma_\#\Big((B_i,p_i),(C_i,u_i)\Big) \.
		\phi_\F \psi_\F = (\phi \psi)_\F
	}
	\Explain{This is obvious by the expression}
	\EndProof	
}
\newpage
\subsubsection{Filtered Colimits}
\Page{
	\DeclareFunc{measurePreservingMeasureAlgebraCategory}{\mathsf{LSCAT}}
	\DefineNamedFunc{measurePreservingMeasureAlgebraCategory}{}{\pa_\#}
	{\NewLine \de \Big( \PAlg, \MPH, \circ, \id\Big)}
	\\
	\DeclareType{FilteredDiagram}{\prod_{\C \in \CAT} 
		?\sum I : \TYPE{Preorder} \. \sum X : I \to \C \. \sum \prod_{(i,j) \in \prec_I} \C(X_i,X_j) 
		\And \TYPE{Ultrafilter}(I)     }
	\DefineType{(I,X,\phi,\F)}{FilteredDiagram}
	{
		\forall (i,j),(j,k) \in \prec_I \. \phi_{i,j}\phi_{j,k} =	\phi_{i,k}	
		\forall i \in I \.  \{ j \in I : j \preceq i   \} \in \F	
	}
	\\
	\Theorem{CofilteredCoconeConstruction}
	{
		\NewLine ::		
		\forall  (I,(A,p),\phi,\F) : \TYPE{FilteredDiagram}(\pa_\#) \.
		\exists \TYPE{Cocone}\Big(\pa_\#,(A,p), \phi \Big)
	}
	\Explain{ 1  The limit is $(B,P) = \prod_{i \in I} (A_i,p_i) | \F$}
	\Explain{ 2 There are $\psi_i \in \pa_\#\Big( (A_i,p_i), (B,P)\Big)$ 
		such that $\psi_i = \phi_{i,j} \psi_j$}
	\Explain{ 2.1 Assume $a \in A_i$}
	\Explain{ 2.2 Take  $a' \in \prod_{j \in I} A_j$ such that 
		$a'_j = \phi_{j,i}(a)$ if $i \prec j$ and otherwise $a'_j = x_j$ 
		is arbitrary}
	\Explain{ 2.3 Then $[a']$ do not depend on the choice of $x$
		by the property of $\F$}
	\Explain{ 2.3.1 Consider a structure $u \in \prod_{j \in I} A_j$
		such that $u_j = 0$ then $i  \preceq  j$}
	\Explain{ 2.3.2 Assume $j \in I$}
	\Explain{ 2.3.3 Then $\{ k \in I : j \preceq k \},\{ k \in I : i \preceq k \} \in \F$}
	\Explain{ 2.3.4 As $\F$ an ultrafilter 
		$\{ k \in I : j \preceq k \} \cap \{ k \in I : i \preceq k \} \neq \emptyset$ }
	\Explain{ 2.3.5 Thus $\lim_{i \to \F} p_j(u_j) = 0$}
	\Explain{ 2.4 Define $\psi_i(a) = [a']$}
	\Explain{ 2.5 Then  $\psi_i(a)$ is a measure preserving homomorphism as $phi_{i,j}$ is}
	\Explain{ 2.6 And Property $\psi_i = \phi_{i,j} \psi_j$ follows by construction
		and properties of cofiltered diagrams}
	\EndProof
	\\
	\DeclareFunc{cofilteredCoconeConstruction}
	{
		\NewLine ::		
		\prod  (I,(A,p),\phi,\F) : \TYPE{CofilteredDiagram}(\pa_\#) \.
		\sum \TYPE{Cocone}\Big(\pa_\#,(A,p), \phi \Big)
	}
	\DefineNamedFunc{cofilteredCoconeConstruction}{}{  \lim_{i,j \to \F} \Big( (A_i,p_i), \phi_{i,j}\Big) }
	{ \NewLine \de \THM{CofilteredCoconeConstruction}\Big(I,(A,p), \phi, \F \Big)  }
}\Page{
	\Theorem{CofiltratedCoconeElementsRepresentation}
	{
		\NewLine ::		
		\forall  (I,(A,p),\phi,\F) : \TYPE{FilteredDiagram}(\pa_\#) \.
		\forall  a \in \prod_{i \in I} A_i \. \NewLine \.
		[a]_B \le \bigvee_{i \in I} \psi_i(a_i) \quad \where \quad
		\Big( (B,P),\psi \Big) =  \lim_{i,j \to \F} \Big( (A_i,p_i), \phi_{i,j}\Big)
	}
	\Explain{ This is intuitively clear bu requires an eloborate proof}
	\Explain{ Assume we have operation $b|i$ for $b$ such that $[a] | i = a_i$}
	\Explain{ Then $[a] | i = a_i \le a_i \vee \bigvee_{j \preceq i} \phi_{i,j}(a_i) = 
		\bigvee_{i \in I} \psi_i(a_i) | i$}
	\NoProof
}\Page{
	\Theorem{DirectedLimits}{\TYPE{HasАilteredСolimits}(\pa_\#)}
	\Explain{ 1 Consider a filtered diagram $(I,(A,p),\phi,\F)$}
	\Explain{ 2 Let $\Big((B,P),\psi\Big) = \lim_{i,j \to \F} \Big( (A_i,p_i), \phi_{i,j} \Big) $}
	\Explain{ 3 Let $C = \bigcup_{i \in I} \psi_i(A_i)$}
	\Explain{ 5 Assume $(D,Q)$ ia another probability algebra and 
		$\xi_i \in \ma_\#\Big((A_i,p_i) ,(D,Q)\Big)$ 
		is such that $\phi_{i,j} \xi_i = \xi_j$}	
	\Explain{ 6 Define mapping $eta : C \to D$ as 
		$eta(\psi_i(a)) = \xi_i(a)$}
	\Explain{ 6.1 Locally $\eta$ is defined for each $A_i$
		as measure preserving map $\xi_i$ must be injective}
	\Explain{ 6.2 And in the case $\psi_i(a) = \psi_j(b)$ for $i \neq j$
		this is still the case that $\xi_i(a) = \xi_j(b)$}
	\Explain{ 6.2.1 There are $k \in I$ such that $i \prec k$ and $j \prec k$}
	\Explain{ 6.2.2 So there ae $\phi_{k,i}$ and $\phi_{k,j}$}
	\Explain{ 6.2.3 Also $\phi_{i,k} \xi_k = \xi_i$ and $\phi_{j,k} \xi_k = \xi_j$}
	\Explain{ 6.2.4 Measure preserving $\xi_k$ is injective, 
		so it must be the case that $\phi_{i,k}(a) = \phi_{i,k}(b)$}
	\Explain{ 6.2.5 But $\phi_{i,k} \xi_k = \xi_i$ and $\phi_{j,k}\xi_k = \xi_j$}
	\Explain{ 6.2.6 Thus, $\xi_i(a) = \xi_k(\phi_{i,k}(a)) = \xi_k(\phi_{j,k}(b)) = \xi_j(b)$}
	\Explain{ 6.3 So the mapping $\eta$ is well defined}
	\Explain{ 7 $\inf_{c \in C} Q(\eta(c)) = \inf_{c \in C} P(c)$}
	\Explain{ 7.1 For each $c \in C$ where is $i \in I$ and $a \in A_i$ such that $c = \phi_i(a)$}
	\Explain{ 7.2 Then by construction  $\eta(c) = \xi_i(a)$}
	\Explain{ 7.3 But both $\xi_i$ and $\phi_i$ are measure preserving so
		  $Q(\eta(c)) = Q(\xi_i(a)) = p_i(a) = P(\psi_i(a))$}
	\Explain{ 8 Then we can apply a subset extension theorem to $\eta$
		and get $H \in \ma\Big( (\langle C \rangle_\ma, P_{|\ldots}),(D,Q)\Big)$}
	\Explain{ 9 Set  $(\langle C \rangle_\ma, P_{|\ldots})$ to be our colimit}
	\Explain{ 10 From construction it follows that $\phi_i H = \xi_i$}
	\Explain{ 11 If $H'$ i any other measure preserving morphism with this property 
		it mus be the case that $H_{|C} = H'_{|C}$.}
	\Explain{ 12 But $\langle C \rangle_\ma$ is the domain, so  $H = H'$}
	\EndProof
	\\
	\Theorem{ProjectiveLimits}{\TYPE{HasFilteredLimits}(\pa_\#)}
	\Explain{ Consider a filtered diagram $(I,(A,p),\phi,\F)$}
	\Explain{ The limit can be defined as
		$\left\{ a \in \prod_{i \in I} A_i : \forall (i,j) \in (\prec_I) \. a_j = \phi_{i,j}(a_i)  \right\}$}	
	\EndProof
} 
\newpage
\subsubsection{Commuting endomorphism}
\Page{
	\Theorem{Augmentation}{
		\forall (A,p) \in \PAlg \.
		\forall \Phi \subset_{\mathsf{MONO}} \End_{\pa_\#}(A,p) \.
		\forall \aleph : \forall \phi,\phi' \in \Phi \. \phi \phi' = \phi' \phi \. \NewLine \.
		\exists (B,q) \in \PAlg \.
		\exists \eta  \in \pa_\#\Big( (A,p), (B,q)\Big) \. 
		\exists  \Phi \Arrow{\mathcal{H}} \Aut_{\pa_\#}(B,q) : \mathsf{MONO} \. \NewLine
		\forall \phi \in \Phi \.   \phi \eta = \eta \mathcal{H}(\phi)
	}
	\Explain{ 1 Introduce preorder on $\Phi$ by
		$\phi \preceq \psi$ if there is $\theta \in \Phi$ such that $\psi = \theta \phi$} 
	\Explain{ 2 As these endomorphisms are injective there can be atmost one such $\theta$}
	\Explain{ 3 So we may define structure $\theta_{\phi,\psi}$}
	\Explain{ 4 Then construction $(A,\theta)$ is a diagram}
	\Explain{ 5 If $\phi,\psi \in \Phi$, then $\phi \prec \psi \phi$ and $\psi \prec \phi \psi = \psi \phi$}
	\Explain{ 6 So by ultrafilter Lemma we can define an ultrafilter $\F$ on $\Phi$ 
		making $(\Phi,A,\theta,\F)$ into filtered diagram}
	\Explain{ 7 Define $(B,q) = (A,p)^\Phi|\F$}
	\Explain{ 8 Define $\eta(a) =   \Big[\phi(a)\Big]_{\phi \in \Phi} \in \pa_\#\Big( (A,p), (B,q)\Big)$}
	\Explain{ 9 Define $\mathcal{H}(\phi)[a] = [\phi(a_\psi)]_{\psi \in \Phi }$}
	\Explain{ 9.1 $\mathcal{H}(\phi)$ is well defined as $\phi$ is measure preserving}
	\Explain{ 9.2        
		$ \phi \eta a   = 
		  [\phi \psi(a)]_{\psi \in \Phi} =  
		  [\psi \phi(a)]_{\psi \in \Phi} = 
		  \eta \mathcal{H}(\phi) [a] 		
		$	
	}
	\Explain{ 9.3 Each $\mathcal{H}(\phi)$ is invertible}
	\Explain{ 9.3.1 Define $\alpha_\phi(a) = (\theta_{\psi,\phi}(a))_{\phi \prec \psi}$}
	\Explain{ 9.3.2 Each $\alpha_\phi$ is well defined by the argument simmilar to the previous chapter}
	\Explain{ 9.3.3 Also $\alpha_\phi \mathcal{H}(\psi) = \alpha_{\phi \psi} 
		= \theta_{\phi,\phi\psi} \alpha_\phi = \psi \alpha_\phi$ }
	\Explain{	 9.3.4 Also $ \phi \alpha_{\phi \psi} = \alpha_\psi$ }
	\Explain{ 9.3.5  $\alpha_\phi(A) \cup \alpha_\psi(A) =
		\alpha_{\phi\psi}(\psi(A)) \cup \alpha_{\phi\psi}(\phi(A)) \subset
		\alpha_{\phi\psi}(A)		
		$  for all $\phi,\psi \in \Phi$}
	\Explain{ 9.3.6 So $D = \bigcup_{\phi \in \Phi} \alpha_\phi(A)$ is a subalgebra of $B$} 
	\Explain{ 9.3.7 Define probability algebra $C = \overline{D}$ with probability $P|C$}
	\Explain{ 9.3.8 Then $\eta = \alpha_{\id}$ has its image contained in $C$}
	\Explain{ 9.3.9 $\mathcal{H}(\psi)(D) = D$ for any $\psi \in \Phi$}	
	\Explain{ 9.3.9.1 $\mathcal{H}(\psi)(D) =  
		\bigcup_{\phi \in \Phi} \alpha_\phi\mathcal{H}(\psi)(A) = 
		\bigcup_{\phi \in \Phi} \psi \alpha_\phi (A) \subset D$}
	\Explain{ 9.3.9.2 Assume $d \in D$}
	\Explain{ 9.3.9.3 Then there is $\psi \in \Phi$ such that $a = \alpha_\psi(a)$}	
	\Explain{ 9.3.9.4 Define 
		$b = (\theta_{\psi,\xi} a)_{\psi \preceq \xi},$
		and
		$b' = (\theta_{\psi \phi, \xi} a)_{ \psi \phi  \preceq \xi} $
	}
	\Explain{ 9.3.9.5 Then $\phi \psi b' = ( \xi(a))_{\xi \in \Phi} $ and $\psi b = (\xi( a))_{\xi \in \Phi} $ }
	\Explain{ 9.3.9.6 So by injectivity and commutativity $\phi b' = b$}	
	\Explain{ 9.3.9.7 Then  
		$ \mathcal{H}(\phi)[b'] = [b] = a $} 	
	\Explain{ 9.3.10 Thus, each $\mathcal{H}(\phi)$ is surjective}
	\Explain{ 9.3.11  As $\mathcal{H}(\phi)$ is injective as a measure preserving map it must be 
			an isomorphism}
	\EndProof
}
\newpage
\section{Maharam's Theory}
\subsection{Types}
\subsubsection{Relative Atoms}
\subsubsection{Subject}
\subsection{Classification Theorem}
\subsection{Closed Subalgebras}
\subsection{Classification of Products}
\subsection{Von Neuman's Lifting Theorem}
\section{Abstract Ergodic Theory}
\section{Measurable Algebras}
\newpage
\section*{Sources}
\begin{enumerate}
\item  O. Vladimirskaya --- Classes of Banach spaces connected with the Lyapunov convexity 1999
\item  D. H. Fremlin --- Measure Theory Volume 3 (32,33,34,37,38,39) 2016
\end{enumerate}
\end{document}

