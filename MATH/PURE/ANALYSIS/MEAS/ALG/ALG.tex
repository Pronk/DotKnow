\documentclass[12pt]{scrartcl}
\usepackage{mathtools}
\usepackage[T2A]{fontenc}
\usepackage[utf8]{inputenc}
\usepackage{amsmath}
\usepackage{amsfonts}
\usepackage{hyperref}
\usepackage{amssymb}
\usepackage{ wasysym }
\usepackage{ upgreek }
\usepackage[dvipsnames]{xcolor}
\usepackage[a4paper,top=5mm, bottom=20mm, left=10mm, right=2mm]{geometry}
\renewcommand\pagemark{{\usekomafont{pagenumber}\thepage\ }}
%Markup
\newcommand{\TYPE}[1]{\textcolor{NavyBlue}{\mathtt{#1}}}
\newcommand{\FUNC}[1]{\textcolor{Cerulean}{\mathtt{#1}}}
\newcommand{\LOGIC}[1]{\textcolor{Blue}{\mathtt{#1}}}
\newcommand{\THM}[1]{\textcolor{Maroon}{\mathtt{#1}}}
%META
\renewcommand{\.}{\; . \;}
\newcommand{\de}{: \kern 0.1pc =}
\newcommand{\extract}{\LOGIC{Extract}}
\newcommand{\where}{\LOGIC{where}}
\newcommand{\If}{\LOGIC{if} \;}
\newcommand{\Then}{ \; \LOGIC{then} \;}
\newcommand{\Else}{\; \LOGIC{else} \;}
\newcommand{\IsNot}{\; ! \;}
\newcommand{\Is}{ \; : \;}
\newcommand{\DefAs}{\; :: \;}
\newcommand{\Act}[1]{\left( #1 \right)}
\newcommand{\Example}{\LOGIC{Example} \; }
\newcommand{\Theorem}[2]{& \THM{#1} \, :: \, #2 \\ & \Proof = \\ } 
\newcommand{\DeclareType}[2]{& \TYPE{#1} \, :: \, #2 \\} 
\newcommand{\DefineType}[3]{& #1 : \TYPE{#2} \iff #3 \\} 
\newcommand{\DefineNamedType}[4]{& #1 : \TYPE{#2} \iff #3 \iff #4 \\} 
\newcommand{\DeclareFunc}[2]{& \FUNC{#1} \, :: \, #2 \\}  
\newcommand{\DefineFunc}[3]{&  \FUNC{#1}\Act{#2} \de #3 \\} 
\newcommand{\DefineNamedFunc}[4]{&  \FUNC{#1}\Act{#2} = #3 \de #4 \\} 
\newcommand{\NewLine}{\\ & \kern 1pc}
\newcommand{\Page}[1]{ \begin{align*} #1 \end{align*}   }
\newcommand{ \bd }{ \ByDef }
\newcommand{\NoProof}{ & \ldots \\ \EndProof}
%LOGIC
\renewcommand{\And}{\; \& \;}
\newcommand{\ForEach}[3]{\forall #1 : #2 \. #3 }
\newcommand{\Exist}[2]{\exists #1 : #2}
\newcommand{\Imply}{\Rightarrow} 
\newcommand{\Intro}{\LOGIC{I}}
\newcommand{\Elim}{\LOGIC{E}}
%TYPE THEORY
\newcommand{\Type}{\TYPE{Type}}
%\DeclareMathOperator*{\dom}{dom}
%%STD
\newcommand{\Int}{\mathbb{Z} }
\newcommand{\NNInt}{\mathbb{Z}_{+} }
\newcommand{\Reals}{\mathbb{R} }
\newcommand{\Complex}{\mathbb{C}}
\newcommand{\Rats}{\mathbb{Q} }
\newcommand{\Sphere}{\mathbb{S}}
\newcommand{\Ball}{\mathbb{B}}
\newcommand{\Nat}{\mathbb{N} }
\newcommand{\EReals}{\stackrel{\mathclap{\infty}}{\mathbb{R}}}
\newcommand{\ERealsn}[1]{\stackrel{\mathclap{\infty}}{\mathbb{R}}^{#1}}
\DeclareMathOperator*{\centr}{center}
\DeclareMathOperator*{\argmin}{arg\,min}
\DeclareMathOperator*{\id}{id}
\DeclareMathOperator*{\im}{Im}
\DeclareMathOperator*{\supp}{supp}
\newcommand{\EqClass}[1]{\TYPE{EqClass}\left( #1 \right)}
\newcommand{\End}{\mathrm{End}}
\newcommand{\Aut}{\mathrm{Aut}}
\mathchardef\hyph="2D
\newcommand{\ToInj}{\hookrightarrow}
\newcommand{\ToMono}{\xhookrightarrow}
\newcommand{\ToSurj}{\twoheadrightarrow}
\newcommand{\ToEpi}{\xtwoheadrightarrow}
\newcommand{\ToBij}{\leftrightarrow}
\newcommand{\ToIso}{\xleftrightarrow}
\newcommand{\Arrow}{\xrightarrow}
\newcommand{\Set}{\TYPE{Set}}
\newcommand{\du}{\; \triangle \;}
\renewcommand{\c}{\complement}
\renewcommand{\i}{\mathbf{i}}
\newcommand{\Eqmod}[3]{#1 = #2 \quad \mathrm{mod} \quad #3}
%%ProofWritting
\newcommand{\Say}[3]{& #1 \de #2 : #3, \\}
\newcommand{\SayIn}[3]{& #1 \de #2 \in #3, \\}
\newcommand{\Conclude}[3]{& #1 \de #2 : #3; \\}
\newcommand{\Derive}[3]{& \leadsto #1 \de #2 : #3, \\}
\newcommand{\DeriveConclude}[3]{& \leadsto #1 \de #2 : #3 ; \\} 
\newcommand{\Assume}[2]{& \LOGIC{Assume} \; #1 : #2, \\}
\newcommand{\AssumeIn}[2]{& \LOGIC{Assume} \; #1 \in #2, \\}
\newcommand{\As}{\; \LOGIC{as } \;} 
\newcommand{\ByDef}{\LOGIC{E}}
\newcommand{\QED}{\; \square}
\newcommand{\EndProof}{& \QED \\}
\newcommand{\Proof}{\LOGIC{Proof} \; }
\newcommand{\Explain}[1]{& \text{#1.} \\}
\newcommand{\ExplainFurther}[1]{& \text{#1} \\}
\newcommand{\Exclaim}[1]{& \text{#1!} \\}
%SetTheory
\newcommand{\NonEmpty}{\TYPE{NonEmpty}}
\newcommand{\Finite}{\TYPE{Finite}}
\newcommand{\Countable}{\TYPE{Countable}}
\newcommand{\Uncountable}{\TYPE{Uncountable}}
\newcommand{\Ideal}{\TYPE{Ideal}}
\newcommand{\Inj}{\TYPE{Injective}}
\newcommand{\Surj}{\TYPE{Surjective}}
\newcommand{\Bij}{\TYPE{Bijective}}
\newcommand{\SIdeal}{\TYPE{\sigma\hyph \Ideal}}
\newcommand{\SA}{\TYPE{\sigma \hyph Algebra}}
\newcommand{\Eq}{\TYPE{Equivalence}}
%CategoryTheory
%Types
\newcommand{\Cov}{\TYPE{Covariant}}
\newcommand{\Contra}{\TYPE{Contravariant}}
\newcommand{\NT}{\TYPE{NaturalTransform}}
\newcommand{\UMP}{\TYPE{UnversalMappingProperty}}
\newcommand{\CMP}{\TYPE{CouniversalMappingProperty}}
\newcommand{\paral}{\rightrightarrows}
%functions
\newcommand{\op}{\mathrm{op}}
\newcommand{\obj}{\mathrm{obj}}
\DeclareMathOperator*{\dom}{dom}
\DeclareMathOperator*{\codom}{codom}
\DeclareMathOperator*{\colim}{colim}
%variable
\renewcommand{\C}{\mathcal{C}}
\newcommand{\A}{\mathcal{A}}
\newcommand{\B}{\mathcal{B}}
\newcommand{\D}{\mathcal{D}}
\newcommand{\I}{\mathcal{I}}
\newcommand{\J}{\mathcal{J}}
\newcommand{\R}{\mathcal{R}}
%Cats
\newcommand{\CAT}{\mathsf{CAT}}
\newcommand{\SET}{\mathsf{SET}}
\newcommand{\PARALLEL}{\bullet \paral \bullet}
\newcommand{\WEDGE}{\bullet \to \bullet \leftarrow \bullet}
\newcommand{\VEE}{\bullet \leftarrow \bullet \to \bullet}
%OrderTheory
%Types
\newcommand{\Poset}{\TYPE{Poset}}
\newcommand{\Toset}{\TYPE{Toset}}
\newcommand{\Pres}{\TYPE{PreorderedSet}}
\newcommand{\WF}{\TYPE{WellFounded}}
\newcommand{\WO}{\TYPE{WellOrdered}}
\newcommand{\II}{\TYPE{InitialInterval}}
\newcommand{\UB}{\TYPE{UpperBound}}
\newcommand{\LUB}{\TYPE{LowerUpperBound}}
\newcommand{\LB}{\TYPE{LowerBound}}
\newcommand{\ULB}{\TYPE{UpperLoweBound}}
%Cats
\newcommand{\POSET}{\mathsf{POSET}}
\newcommand{\ORD}{\mathsf{ORD}}
%Symbols
\renewcommand{\P}{\mathsf{P}}
%\newcommand{\F}{\mathsf{F}}
%\newcommand{\U}{\mathsf{U}}
%Algebra
%Groups
%Types
\newcommand{\Group}{\TYPE{Group}}
\newcommand{\Abel}{\TYPE{Abelean}}
\newcommand{\Sgrp}{\subset_{\mathsf{GRP}}}
\newcommand{\Nrml}{\vartriangleleft}
\newcommand{\FG}{\TYPE{FiniteGroup}}
\newcommand{\Stab}{\mathrm{Stab}}
\newcommand{\FGA}{\TYPE{FinitelyGeneratedAbelean}}
\newcommand{\DN}{\TYPE{DirectedNormality}}
\newcommand{\ActsOn}{\curvearrowright}
%Func
\DeclareMathOperator{\tor}{tor}
\DeclareMathOperator{\ord}{ord}
\DeclareMathOperator{\bool}{bool}
\DeclareMathOperator{\rank}{rank}
%Cats
\newcommand{\GRP}{\mathsf{GRP}}
\newcommand{\ABEL}{\mathsf{ABEL}}
%Boolean Algebra
%TYPE
\newcommand{\Bool}{\mathbb{B}}
\newcommand{\Alg}{\TYPE{Algebra}}
\newcommand{\BR}{\TYPE{BooleanRing}}
\newcommand{\BA}{\TYPE{BooleanAlgebra}}
\newcommand{\PD}{\TYPE{PairwiseDisjointElements}}
\newcommand{\PoU}{\TYPE{PartitionOfUnity}}
\renewcommand{\SS}{\TYPE{StoneSpace}}
\newcommand{\TK}{\mathcal{TK}}
\newcommand{\BL}{\TYPE{BooleanLattice}}
\newcommand{\Fix}{\mathrm{Fix}}
\newcommand{\OC}{\TYPE{OrderClosed}}
\newcommand{\SOC}{\TYPE{SequentiallyOrderClosed}}
\newcommand{\oC}{\TYPE{OrderContinuous}}
\newcommand{\sC}{\TYPE{\sigma\hyph Continuous}}
\newcommand{\OD}{\TYPE{OrderDense}}
\newcommand{\REing}{\TYPE{RegularEmbedding}}
\newcommand{\REed}{\TYPE{RegularEmbeded}}
\newcommand{\REable}{\TYPE{RegularEmbedable}}
\newcommand{\OComplete}{\TYPE{OrderDedekindComplete}}
\newcommand{\TAlgebra}{\TYPE{\tau\hyph Algebra}}
\newcommand{\OCompletes}{\TYPE{OrderDedekindCompleteSubset}}
\newcommand{\SComplete}{\TYPE{\sigma\hyph DedekindComplete}}
\newcommand{\SCompletes}{\TYPE{\sigma\hyph DedekindCompleteSubset}}
\newcommand{\LS}{\mathcal{LS}}
\newcommand{\POpen}{\TYPE{PseudoOpen}}
\newcommand{\od}{\mathbf{OD}}
\newcommand{\mgr}{\mathbf{MGR}}
\newcommand{\nd}{\mathbf{ND}}
\newcommand{\CCC}{\TYPE{WithCountableChainCondition}}
\newcommand{\CSI}{\TYPE{\omega_1\hyph SaturatedIdeal}}
\newcommand{\WD}{\TYPE{(\sigma,\infty)\hyph WeaklyDistributive}}
\newcommand{\Aless}{\TYPE{Atomless}}
\newcommand{\PA}{\TYPE{PurelyAtomic}}
\newcommand{\Homog}{\TYPE{Homogeneous}}
%\newcommand{\FS}{\TYPE{FullSubgroup}}
%\newcommand{\CFS}{\TYPE{CountablyFullSubgroup}}
%\newcommand{\EI}{\TYPE{ExchangingInvolution}}
%\newcommand{\SwS}{\TYPE{SubgroupWithSeparators}}
%\newcommand{\SwmI}{\TYPE{SubgroupWithManyInvolutions}}
%FUNC
\DeclareMathOperator{\upr}{upr}
\DeclareMathOperator{\Atom}{Atom}
%\DeclareMathOperator{\Supp}{Supp}
%\newcommand{\genFS}[1]{\left\langle #1 \right\rangle_\mathrm{F}}
%\newcommand{\genCFS}[1]{\left\langle #1 \right\rangle_\mathrm{CF}}
%\DeclareMathOperator{\Sep}{Sep}
%\DeclareMathOperator{\Tr}{Tr}
%CATS
\newcommand{\BOL}{\mathsf{BOL}}
\newcommand{\BOOL}{\mathsf{BOOL}}
%SYMBOL
\newcommand{\Z}{\mathsf{Z}}
%Topology
%General Topology
%Types
\newcommand{\Top}{\TYPE{Topology}}
\newcommand{\Homeo}{\TYPE{Homeomorphism}}
\newcommand{\TS}{\TYPE{TopologicalSpace}} 
\newcommand{\NbhdBase}{\TYPE{NeighborhoodBase}}
\newcommand{\LF}{\TYPE{LocallyFinite}}
\newcommand{\PN}{\TYPE{PerfectlyNormal}}
\newcommand{\CR}{\TYPE{CompletelyRegular}}
\newcommand{\OM}{\TYPE{OpenMap}}
\newcommand{\Filter}{\TYPE{Filter}}
\newcommand{\Filterbase}{\TYPE{Filterbase}}
\newcommand{\CFilterbase}{\TYPE{ConvergentFilterbase}}
\newcommand{\Dense}{\TYPE{Dense}}
\newcommand{\Separable}{\TYPE{Separable}}
\newcommand{\ND}{\TYPE{NowhereDense}}
\newcommand{\Open}{\TYPE{Open}}
\newcommand{\Net}{\TYPE{Net}}
\newcommand{\Closed}{\TYPE{Closed}}
\newcommand{\Clopen}{\TYPE{Clopen}}
\newcommand{\Nbhd}{\TYPE{Neighborhood}}
\newcommand{\Compact}{\TYPE{Compact}}
\newcommand{\Compacts}{\TYPE{CompactSubset}}
\newcommand{\OpenC}{\TYPE{OpenCover}}
\newcommand{\Cluster}{\TYPE{Cluster}}
\newcommand{\Convergent}{\TYPE{Convergent}}
%\newcommand{\LC}{\TYPE{LocallyCompact}}
\newcommand{\Locally}{\TYPE{Locally}}
\newcommand{\Bair}{\TYPE{BaireSpace}}
%\newcommand{\Meager}{\TYPE{Meager}}
\newcommand{\Connected}{\TYPE{Connected}}
\newcommand{\ED}{\TYPE{ExtemellyDisconnected}}
%FUNC
\DeclareMathOperator*{\intx}{int}
\DeclareMathOperator*{\cl}{cl} 
\DeclareMathOperator*{\boundary}{\partial} 
\DeclareMathOperator{\combo}{\triangledown} 
%\DeclareMathOperator{\diag}{\triangle} 
\DeclareMathOperator{\rem}{rem}
%CATS
\newcommand{\TOP}{\mathsf{TOP}}
\newcommand{\HC}{\mathsf{HC}}
\newcommand{\CG}{\mathsf{CG}}
%Symbols
\newcommand{\T}{\mathcal{T}}
\newcommand{\N}{\mathcal{N}}
\renewcommand{\U}{\mathcal{U}}
\renewcommand{\O}{\mathcal{O}}
\renewcommand{\d}{\mathrm{d}}
%\newcommand{\F}{\mathcal{F}}
\newcommand{\X}{\mathcal{X}}
%\newcommand{\d}{\mathrm{d}}
%Metric Topology
\newcommand{\Bounded}{\TYPE{Bounded}}
%FUNC
\DeclareMathOperator{\diam}{diam}
\newcommand{\Cell}{\mathbb{B}}
\newcommand{\Disc}{\mathbb{D}}
\newcommand{\Lip}[1]{#1\hyph\mathrm{Lip}}
%CATS
\newcommand{\Semiiso}{\mathsf{SMS}_{\circ \to \cdot}}
\newcommand{\Iso}{{\mathsf{MS}_{\circ \to \cdot}}}
\newcommand{\SMS}{\mathsf{SMS}}
\newcommand{\MS}{\mathsf{MS}}
\newcommand{\UNI}{\mathsf{UNI}}
\newcommand{\UNIS}{\mathsf{UNIS}}
\newcommand{\TG}{\mathsf{TG}}
\newcommand{\CSeq}{\TYPE{CauchySequence}}
\newcommand{\Complete}{\TYPE{Complete}}
%Descriptive Set Theory
%TYPE
%\newcommand{\Bool}{\mathbb{B}}
\newcommand{\IS}{\TYPE{InitialSegement}}
\newcommand{\FS}[1]{{#1}{}^*}
\newcommand{\Ext}{\TYPE{Extension}}
\newcommand{\Tree}{\TYPE{Tree}}
\newcommand{\Pruned}{\TYPE{Pruned}}
\newcommand{\PTM}{\TYPE{ProperTreeMorphism}}
\newcommand{\LTM}{\TYPE{LipschitzTreeMorphism}}
\newcommand{\Polish}{\TYPE{Polish}}
\newcommand{\IIPG}{\TYPE{InfiniteIterativeTwoPlayersGame}}
\newcommand{\FPS}{\TYPE{FirstPlayerStrategy}}
\newcommand{\SPS}{\TYPE{SecondPlayerStrategy}}
\newcommand{\FPWS}{\TYPE{FirstPlayerWinningStrategy}}
\newcommand{\SPWS}{\TYPE{SecondPlayerWinningStrategy}}
\newcommand{\CS}{\TYPE{ChoquetSpace}}
\newcommand{\SCS}{\TYPE{StrongChoquetSpace}}
\newcommand{\BP}{\mathbf{BP}}
\newcommand{\MGR}{\mathbf{MGR}}
\newcommand{\cat}{\mathbf{CAT}}
\newcommand{\BM}{\TYPE{BairMeasurable}}
\newcommand{\CGSA}{\TYPE{CountablyGeneratedSigmaAlgebra}}
\newcommand{\MC}{\TYPE{MonotonicClass}}
\newcommand{\PSA}{\TYPE{PointSeparatingAlgebra}}
\newcommand{\SBS}{\TYPE{StandardBorelSpace}}
\newcommand{\IH}{\TYPE{InducedHomomorphism}}
%FUNC
\DeclareMathOperator{\len}{len}
\newcommand{\inits}[2]{{#1}_{|\left[1,\ldots,#2\right]}}
\DeclareMathOperator{\lb}{lb}
\DeclareMathOperator{\WFpart}{WF}
\DeclareMathOperator{\Tr}{Tr}
\DeclareMathOperator{\PTr}{PTr}
\DeclareMathOperator*{\Tll}{{T\;\underline{lim}}}
\DeclareMathOperator*{\Tul}{{T\;\overline{lim}}}
\DeclareMathOperator*{\Tl}{{T\;lim}}
\DeclareMathOperator{\rankcb}{rank_{CB}}
\DeclareMathOperator{\lp}{lp}
\newcommand{\alg}{\mathsf{A}}
%CATS
\newcommand{\TREE}{\mathsf{TREE}}
\newcommand{\FSF}{\mathsf{FS}}
\newcommand{\CRONE}{\mathsf{CRONE}}
\newcommand{\BODY}{\mathsf{BODY}}
\newcommand{\BOR}{\mathsf{BOR}}
\newcommand{\bor}{\mathsf{B}}
\newcommand{\Effros}{\mathsf{EFF}}
%symbols
\newcommand{\K}{\mathsf{K}}
\renewcommand{\H}{\mathrm{H}}
\renewcommand{\L}{\mathcal{L}}
\renewcommand{\P}{\mathcal{P}}
\renewcommand{\S}{\mathcal{S}}
%LINEAR
%Linear Algebra
%Types
\newcommand{\Basis}{\TYPE{Basis}} % Basis of the linear space
\newcommand{\submod}[1]{\subset_{\LMOD{#1}}}% submodule as a subset
\newcommand{\subvec}[1]{\subset_{\VS{#1}}}% vector subspace as a subset
\newcommand{\FGM}{\TYPE{FinitelyGeneratedModule}}% Finitely generated module
\newcommand{\LI}{\TYPE{LinearlyIndependent}}
\newcommand{\LIS}{\TYPE{LinearlyIndependentSet}}
\newcommand{\FM}{\TYPE{FreeModule}}
\newcommand{\IBP}{\TYPE{InvariantBasisProperty}}
\newcommand{\UTM}{\TYPE{UpperTriangularMatrix}}
%\newcommand{\LTM}{\TYPE{LowerTriangularMatrix}}
\newcommand{\Diag}{\TYPE{DiagonalMatrix}}
\newcommand{\FP }{\TYPE{FinitelyPresented}}
\newcommand{\GL}{\mathbf{GL}}% General Linear Group
\newcommand{\SL}{\mathbf{SL}}% Special Linear group
\newcommand{\SO}{\mathbf{SO}}
\newcommand{\SU}{\mathbf{SU}}
\newcommand{\prsubvec}[1]{\subsetneq_{\VS{#1}}}	% poper vector subspace as a subset
\newcommand{\LC}{\TYPE{LinearComplement}} 
%\newcommand{\IS}{\TYPE{InvariantSubspace}}
\newcommand{\RP}{\TYPE{ReducingPair}}
\newcommand{\RCF}{\TYPE{RationalCanonicalForm}}
\newcommand{\JCF}{\TYPE{JordanCanonicalForm}}
\newcommand{\Diagble}{\TYPE{Diagonalizable}}
\newcommand{\UT}{\TYPE{UpperTriangulizable}}
\newcommand{\LT}{\TYPE{LowerTriangulizable}}
\newcommand{\IPS}{\TYPE{InnerProductSpace}}
\newcommand{\OBasis}{\TYPE{OrthonormalBasis}}
\newcommand{\FDIPS}{\TYPE{FiniteDimensionalInnerProductSpace}}
\newcommand{\NO}{\TYPE{NormalOperator}}
\newcommand{\NM}{\TYPE{NormalMatrix}}
%\newcommand{\SA}{\TYPE{SelfAdjoint}}
\newcommand{\SSA}{\TYPE{SkewSelfAdjoint}}
\newcommand{\PI}{\TYPE{Pseudoinverse}}
\newcommand{\OVS}{\TYPE{OrthogonalVectorSpace}}
\newcommand{\SVS}{\TYPE{SymplecticVectorSpace}}
\newcommand{\MVS}{\TYPE{MetricVectorSpace}}
\newcommand{\FDMVS}{\TYPE{FiniteDimensionalMetricVectorSpace}}
\newcommand{\Sp}{\mathbf{Sp}}
%Func
\DeclareMathOperator{\Span}{span} % spann by subset
\DeclareMathOperator{\Ann}{Ann}   % annihilator
\DeclareMathOperator{\Ass}{Ass}   % associated primes:
\DeclareMathOperator{\adj}{adj}   % an adjoint matrix
\DeclareMathOperator{\tr}{tr}     % trace
\DeclareMathOperator{\codim}{codim} % codimension
%\DeclareMathOperator{\Cell}{\mathbf{C}} % a componion matrix
\DeclareMathOperator{\JC}{\mathbf{J}}  % a Jordan cell
\DeclareMathOperator{\bigboxplus}{\scalerel*{\boxplus}{\sum}} % a direct sum of operators in the sence of the reducing a pair
\DeclareMathOperator{\Spec}{Spec} % Spectre
\DeclareMathOperator{\bigbot}{\scalerel*{\bot}{\sum}} % an othogonal direct sum
\DeclareMathOperator{\GS}{\mathbf{GS}} %Gramm-Smmidt process
\DeclareMathOperator{\NGS}{\mathbf{NGS}} %Normalized Gramm-Smmidt process
\DeclareMathOperator{\WI}{\mathrm{WI}} %Witt Index
%Cats
\newcommand{\VS}[1]{#1\hyph\mathsf{VS}} % a category of vector spaces (Field)
\newcommand{\FDVS}[1]{#1\hyph\mathsf{FDVS}} % a category of finite-dimensional vector spaces (Field)
\newcommand{\LALGE}[1]{#1\hyph\mathsf{ALGE}}
\newcommand{\LMOD}[1]{#1\hyph\mathsf{MOD}} % a category of the left modules (Ring)
\newcommand{\RMOD}[1]{\mathsf{MOD}\hyph#1} % a category of the right modules (Ring)
\newcommand{\LLMAP}[1]{#1\hyph\mathsf{LMAP}} % a cagory of based linear maps with the left scalar multiplication (Ring)
\newcommand{\LMAT}[1]{#1\hyph\mathsf{MAT}}  % a category of based matrices with the left scalar multiplication (Ring)
\newcommand{\NMAT}[1]{#1\hyph\mathbb{N}} % a category of finite matrices (Field)
%Symbols
\renewcommand{\L}{\mathcal{L}}
%\renewcommand{\O}{\mathbf{O}}
%\renewcommand{\S}{\mathbf{S}}
%%Measure theorty
%Types
\newcommand{\Measure}{\TYPE{Measure}}
%\newcommand{\MS}{\TYPE{MeasureSpace}}
\newcommand{\CMS}{\TYPE{CompleteMeasureSpace}}
\newcommand{\Null}{\mathcal{N}}
\renewcommand{\ae}{\mathrm{a.e.}}
\renewcommand{\OM}{\TYPE{OuterMeasure}}
\newcommand{\IM}{\TYPE{InnerMeasure}}
\newcommand{\Thick}{\TYPE{Thick}}
\newcommand{\Integrable}{\mathsf{I}}
\newcommand{\ME}{\TYPE{MeasurableEnvelope}}
\newcommand{\Probability}{\TYPE{Probability}}
\newcommand{\sFinite}{\TYPE{\sigma \hyph  Finite}}
\newcommand{\Semifinite}{\TYPE{Semifinite}}
\newcommand{\Decomposition}{\TYPE{Decomposition}}
\newcommand{\SLoc}{\TYPE{StrictlyLocalizable}}
\newcommand{\Loc}{\TYPE{Localizable}}
\newcommand{\LocDet}{\TYPE{LocallyDetermined}}
\newcommand{\PtSupp}{\TYPE{PointSupported}}
\newcommand{\SF}[1]{\TYPE{\sigma \hyph  Finite}\left( #1 \right) }
\newcommand{\DRP}{\TYPE{DiscreteRandomProcces}}
\newcommand{\MwLDNS}{\TYPE{MeasureWithLocallyDeterminedNullSets}}
\newcommand{\AF}{\TYPE{AdditiveFunctional}} 
\newcommand{\CAF}{\TYPE{CountablyAdditiveFunctional}}
\newcommand{\TC}{\TYPE{TrulyContinuous}} 
\newcommand{\CE}{\TYPE{ConditionalExpectation}}
%Functions and Operators
\DeclareMathOperator{\esssup}{ess\sup}
%Symbols
\newcommand{\F}{\mathcal{F}}
\newcommand{\E}{\mathcal{E}}
%CATS
\newcommand{\MEAS}{\mathsf{MEAS}}
\newcommand{\Simple}{\mathsf{S}}
\newcommand{\caf}{\mathsf{ca}}
\newcommand{\af}{\mathsf{a}}
\newcommand{\baf}{\mathsf{ba}}
\newcommand{\ac}{\mathsf{ac}}
\newcommand{\tc}{\mathsf{tc}}
%%Measure Algebra
%Types
\newcommand{\MA}{\TYPE{MeasureAlgebra}}
\newcommand{\SI}{\TYPE{SummableIncrements}}
%Categroy
\newcommand{\ma}{\mathsf{MA}}
\author{Uncultured Tramp} 
\title{Abstract Measure Theory}
\begin{document}
\maketitle
\thispagestyle{empty}
\newpage
\thispagestyle{empty}
\tableofcontents
\newpage
\section*{Intro}
\newpage
\pagenumbering{arabic}
\section{Measure Algebras}
\subsection{Subject}
\subsubsection{Definition and Basic Property}
\Page{
	\DeclareType{MeasureAlgebra}
	{
		? \sum A : \SComplete \. A \to \EReals_+
	}
	\DefineType{(A,\mu)}{MeasureAlgebra}
	{
			\forall a \in A \. \mu(a) = 0 \iff a = 0
			\And \NewLine \And
			\forall a : \PD(\Nat,A) \. 
			\mu\left(\bigvee^\infty_{n=1}  a_n \right) = \sum^\infty_{n=1} \mu(a_n)
	}
	\\
	\Theorem{MeasureMonotonicity}
	{
		\forall (A,\mu) : \MA \. 
		\forall a,b \in A \. 
		a \le b \Imply  \mu(a) \le \mu(b)
	}
	\Explain{ Write $\mu(b) = \mu(a) + \mu(b\setminus a) \ge \mu(a)$}
	\EndProof
	\\
	\Theorem{MeasureStrictMonotonicity}
	{
		\forall (A,\mu) : \MA \. 
		\forall a,b \in A \. 
		a > b \Imply  \mu(a) > \mu(b)
	}
	\Explain{ Definition of measure algebra implies that $\mu(b \setminus a) > 0$ }
	\Explain{ Write $\mu(b) = \mu(a) + \mu(b\setminus a) > \mu(a)$}
	\EndProof
	\\
	\Theorem{MinkovskyIneq}
	{
		\forall (A,\mu) : \MA \.
		\forall a,b \in A \. 
		\mu(a \vee b) \le \mu(a) + \mu(b)
	}
	\Explain{
		Write
		$
			\mu(a) + \mu(b) =
		    \mu(a \setminus ab) + \mu(ab) + \mu(b \setminus ab) + \mu(ab) 
		    \ge   mu(a \setminus ab) + \mu(ab) + \mu(b \setminus ab = 
		    \mu(a \vee b)
		$
	}
	\EndProof
	\\
	\Theorem{MonotonicSupremumAsLimit}
	{
		\forall (A,\mu) : \MA \.
		\forall a : \Nat \uparrow A \. 
		\mu\left( \bigvee^\infty_{n=1} a_n\right)  = \lim_{n \to \infty} \mu(a_n)
	}
	\Explain{ Construct disjoint sequence $b_n = a_n \setminus \bigvee^{n-1}_{k=1} a_k$}
	\Explain{ Then by construction 
			$
				\mu\left( \bigvee^\infty_{n=1} a_n \right) = 
				\mu\left(  \bigvee^\infty_{n=1} b_n \right) =
				\sum^\infty_{n=1} \mu(b_n) = 
				\lim_{n \to \infty} \sum^n_{k=1} \mu(b_n) =
				\lim_{n \to \infty} \mu\left(\bigvee^n_{k=1} b_k\right) =
				\lim_{n \to \infty} \mu(a_n)
			$}
	\EndProof
}\Page{
	\Theorem{SupremumIneq}
	{
		\forall (A,\mu) : \MA \.
		\forall a : \Nat \to A \. 
		\mu\left( \bigvee^\infty_{n=1} a_n\right) \le \sum^\infty_{n=1} \mu(a_n)
	}
	\Explain{ Construct increasing sequence $b_n = \bigvee^n_{k=1} a_n$}
	\Explain{
		Then by construction
		$
			\mu\left( \bigvee^\infty_{n=1} a_n \right) = 
			\mu\left(  \bigvee^\infty_{n=1} b_n \right) =
			\lim_{n \to \infty} \mu( b_n ) =
			\lim_{n \to \infty} \mu\left( \bigvee^n_{k=1} a_k \right)  \le 
			\lim_{n \to \infty} \sum^n_{k=1} \mu(a_k) =
			\sum^\infty_{n=1} \mu(a_n)  
		$
	}
	\EndProof
	\\
	\Theorem{MonotonicInfimumAsLimit}
	{
		\NewLine ::		
		\forall (A,\mu) : \MA \.
		\forall a : \Nat \downarrow A \.
		\forall \aleph : \inf_{n \in \Nat} \mu(a_n) < \infty \.
		\mu\left( \bigwedge^\infty_{n=1} a_n\right)  = \lim_{n \to \infty} \mu(a_n)
	}
	\Explain{ Without loss of generality assume that $\mu(a_1) < \infty$}
	\Explain{ Then construc the increasing sequence $b_n = a_1 \setminus a_n$} 
	\ExplainFurther{ Then $
			\mu(a_1) - \mu\left(\bigwedge^\infty_{n=1} a_n  \right) =
			\mu\left( a_1 \setminus \bigwedge^\infty_{n=1} a_n \right)
			=\mu\left( \bigvee^\infty_{n=1} a_1 \setminus a_n\right) 
			=\mu\left( \bigvee^\infty_{n=1} b_n\right)  = \lim_{n \to \infty} \mu(b_n) =$}
	\Explain{$  =
			\lim_{n \to \infty} \mu\left( a_1 \setminus a_n \right)=\lim_{n \to \infty} \mu(a_1) - \mu(a_n) = \mu(a_1) - \lim_{n \to \infty} \mu(a_n)$}
	\Explain{
		So basic algebraic manipulations
		$
			\mu\left(\bigwedge^\infty_{n=1} a_n  \right) = \lim_{n \to \infty} \mu(a_n)
		$		
	}
	\EndProof
	\\
	\Theorem{SupremumExistance}
	{
		\NewLine ::		
		\forall (A,\mu) : \MA \.
		\forall C : \TYPE{UpwardsDirected}(A) \.
		\forall \aleph : \sup_{c \in C} \mu(c) < \infty \.
		\exists a \in A : a = \sup C
	}
	\Explain{1 Assume $\gamma = \sup_{c \in C} \mu(c)$}
	\Explain{2 Then there exists a sequrnce of $a:\Nat \to C$ such that $\mu(a_n )\ge \gamma - 2^{-n}$}
	\Explain{3 As $C$ is upwards closed, it is possible to find $c:\Nat \to C$ 
		such that $c_{n+1} \ge a_{n} \vee c_n$}
	\Explain{4 Then $c$ is monotonic-nondecreasing and so it has 
		$\mu\left(\bigvee^\infty_{n=1} c_n\right) = \lim_{n \to \infty} \mu(c_n) = \gamma$}
	\Explain{4.1 Note that $\gamma \ge \mu(c_n) \ge \gamma - 2^{-n}$}
	\Explain{5 let $d = \bigvee^\infty_{n=1} c_n$}
	\Explain{6 $d \ge f$ for everty $f \in C$}
	\Explain{6.1 Assume this is false}
	\Explain{6.2 Then $f \setminus d \neq 0$ and so $\alpha = \mu(f\setminus d) > 0$}
	\Explain{6.3 Then there exists $n$ such that $\gamma - \mu(c_n) < \alpha$}
	\Explain{6.4 As $C$ is upwards derected there is $g \in C$
		such that $g \ge f \vee c_n$}
	\Explain{6.5 But
		$\mu(g) \ge \mu(f \vee c_n) = \mu(c_n) + \mu(f \setminus c_n) \ge 
			\mu(c_n) + \mu(f \setminus d) > \gamma$ which is impossible}                                                     
	\Explain{7 If there is another $f$ with the property (6), then 
		$d = \bigvee^\infty_{n=1} c_n \le f$ as $c_n \le f$ for each $n\in\Nat$}
	\EndProof
}\Page{
	\Theorem{UpperContinuity}
	{
		\NewLine ::		
		\forall (A,\mu) : \MA \.
		\forall C : \TYPE{UpwardsDirected}(A) \.
		\forall \aleph : \exists a \in A : a = \sup C \.
		\sup_{c \in C} \mu(c) = \mu\left( \sup C \right)
	}
	\Explain{ Case $\sup_{c \in C} \mu(c) = \infty$ is trivial}
	\Explain{ Finite case follows from the cconstruction in the previous theorem}
	\EndProof
	\\
	\Theorem{DisjointUpperContinuity}
	{
		\NewLine ::		
		\forall (A,\mu) : \MA \.
		\forall C : \PD(A) \.
		\forall \aleph : \exists a \in A : a = \sup C \. \NewLine \.
		\mu\left( \sup C \right)= \sum_{c \in C} \mu(c)
	}
	\Explain{Construct a new set $D = \left\{ \bigvee^\infty_{n=1} c_k \bigg|  c : \Nat \to C  \right\}$}
	\Explain{ Then $D$ is upwards directed and $\sup C = \sup D$}
	\Explain{ But this is evedent that 
		$\mu\left(\sup D\right) = \sup_{d \in D} \mu(d) = 
		\sup_{c : \Nat \to C} \mu\left( \bigvee_{n=1} c_n \right)  =
		\sup_{n \in \Nat, c : \{1,\ldots,n\} \to C} \sum^n_{k=1} \mu(c_k)  =
		\sum_{c \in C} \mu(c)$}
	\EndProof
	\\
	\Theorem{InfimumExistance}
	{
		\NewLine ::		
		\forall (A,\mu) : \MA \.
		\forall C : \TYPE{DownwaedDirected}(A) \.
		\forall \aleph : \inf_{c \in C} \mu(c) < \infty \.
		\exists a \in A : a = \inf C
	}
	\Explain{ 1 There exists some $a \in C$ such that $\mu(a) < \infty$}
	\Explain{ 2 Construct another set $D = a \setminus C$}
	\Explain{ 3 Then $D$ is upwards directed and $\sup_{d \in D} \mu(d) \le \mu(a) < \infty$}
	\Explain{ 4 So there is $d = \sup d$ }
	\Explain{ 5 Define $f = a \setminus d$ }
	\Explain{ 6 $f \le c$ for any $c \in C$ as $a \setminus f \ge a \setminus c$}
	\Explain{ 7 if some $g$ has property (6) then $a \setminus g \ge d$ and so $g \le f$}
	\EndProof
	\\
	\Theorem{LowerContinuity}
	{
		\NewLine ::		
		\forall (A,\mu) : \MA \.
		\forall C : \TYPE{DownwardsDirected}(A) \.
		\forall \aleph : \exists a \in A : a = \inf C \. \NewLine \.
		\forall \beth :  \inf_{c \in C} \mu(c) < \infty \.
		\inf_{c \in C} \mu(c) = \mu\left( \inf C \right)
	}
	\Explain{ Use the construction in the previous theorem}
	\EndProof
}
\newpage
\subsubsection{Measure Algebras Generated by Measure Spaces}
\Page{
	\DeclareFunc{measureAlgebra}
	{
		\MEAS \to \MA
	}
	\DefineNamedFunc{measureAlgebra}{X,\Sigma,\mu}{(A_\mu,\bar \mu)}
	{\left( \frac{\Sigma}{\Sigma \cap \Null_\mu},[E] \mapsto \mu(E)\right)}
	\Explain{ This is obviously well defined as $[E] = [F]$ iff $\mu(E \du F) = 0$}
	\\
	\DeclareFunc{canononicalProjection}
	{
		\forall (X,\Sigma,\mu) \in \MEAS \. 
		\sigma\hyph\BOOL(\Sigma,A_\mu) 
	}
	\DefineNamedFunc{canonicalProjection}{E}{\pi_\mu(E)}{[E]}
	\Explain{ 1 The algebraic properites are obvious as $\Sigma \cap \Null_\mu$ is an ideal}
	\Explain{ 2 In order to prove sigma-continuity assume $E : \Nat \to \Sigma$}
	\Explain{ 2.1 Let $Z : \Nat \to \Sigma \cap \Null_\mu$}
	\Explain{ 2.2 Then $F_Z = \bigvee^\infty_{n=1} (E_n \du Z_n) = 
		\left(\bigvee^\infty_{n=1} E_n\right) \du \left( \bigvee^\infty_{n=1} Z_n\right)$}
	\Explain{ 2.3 Note that 
	$\mu\left( \bigvee^\infty_{n=1} Z_n\right) \le \sum^\infty_{n=1} \mu(Z_n)=0$}
	\Explain{ 2.4 So $ \bigvee^\infty_{n=1} Z_n \in \Sigma \cap \Null_\mu$ as $\mu \ge 0$}
	\Explain{ 2.5 Thus $[F_Z] = \left[\bigcap^\infty_{n=1} E_n\right]$ for any selection of $Z$}
	\Explain{ 2.6 This means that 
			$\pi_\mu\left(\bigcap^\infty_{n=1} E_n\right) = \bigvee^\infty_{n=1} \pi_\mu(E_n)$}
	\EndProof
	\\
	\Theorem{MeasureAlgebraMonotonicity}
	{
		\forall (X,\Sigma,\mu) \in \MEAS \.
		\forall T \subset_\sigma \Sigma \.
		\pi_\mu(T) \subset_\sigma A_\mu
	}
	\Explain{ 1 Clearly $B = \pi_\mu(T) \subset A_\mu$}
	\Explain{ 2 Also as $T$ is $\sigma$-algebra and $\pi-\mu$ is a $\sigma$-continuous homomorphism 
		$B$ is again}
	\EndProof
	\\
	\Theorem{MeasureAlgebraInverseMonotonicity}
	{
		\forall (X,\Sigma,\mu) \in \MEAS \.
		\forall B \subset_\sigma A_\mu \.
		\pi_\mu^{-1}(B) \subset_\sigma \Sigma
	}
	\Explain{ 1 Clearly $T = \pi_\mu^{-1}(B) \subset \Sigma$}
	\Explain{ 2 Assume $F$ is a set constructed by applying $\sigma$-algebra operations to setes $E_1,E_2,\ldots \in T$ }
	\Explain{ 3 Then $\pi_\mu(F)$ can be constructed by applying same operations to $\pi(E_1),\pi(E_2),\ldots$}
	\Explain{ 4 This implies that $\pi_\mu(F) \in B$ and reciprorary $F \in T$}
	\Explain{ 5 Thus $T$ is a $\sigma$-algebra}
	\EndProof
}
\newpage
\subsubsection{Stone Representation Theorem}
\Page{
	\Theorem{StoneRepresentationTheorem}
	{
		\forall (A,\mu) : \MA \. \exists (X,\Sigma,\nu) \in \MEAS \. (A,\mu) = (A_\nu,\bar \nu)
	}
	\ExplainFurther{
		1 By Loomis-Sikorski representation there exists a set $X$ with a sigma-algebra $\Sigma$ and}
	\Explain{ 	sigma-ideal $I$ such that $\frac{\Sigma}{I} \cong_\BOOL  A$
	}
	\Explain{ 2
		Then there is a canonical projetion $\pi_I \in \BOOL(\Sigma,A)$	
	}
	\Explain{ 3 Define $\nu = \pi_I \mu$}
	\Explain{ 4 $\nu$ is measure on $\Sigma$}
	\Explain{ 4.1 $\nu(\emptyset) = \mu(0) = 0$}
	\Explain{ 4.2 Assume $E$ is a disjoint sequence in $\Sigma$}
	\Explain{ 4.3 Then $\pi_I(E_n)\pi_I(E_m) = \pi_i(E_n \cap E_m) = \pi_i(\emptyset) =  0$,
	 so $\pi_I(E)$ is disjoint in $A$}
	\Explain{ 4.4 Thus, 
		$\nu\left( \bigcup^\infty_{n=1} E_n \right) = 
		\pi_I\mu\left( \bigcup^\infty_{n=1} E_n \right) =  
		\mu\left( \bigvee^\infty_{n=1} \pi_I(E_n)\right) =
		\sum^\infty_{n=1} \pi_I\mu(E_n) = 
		\sum^\infty_{n=1} \nu(E_n) $ }
	\Explain{ 5 Also by consytuction $ \Null_\nu \cap \Sigma = I$, so $ (A,\mu) = (A_\nu,\bar \nu)$} 
	\EndProof
	\\
	\DeclareFunc{spaceOfStone}{\MA \to \MEAS}
	\DefineNamedFunc{SpaceOfStone}{A,\mu}{(Z_A,\dot \Sigma_\mu,\dot \mu)}
	{
			\THM{StoneRepresentationTheorem}(A,\mu)	
	}
}
\newpage
\subsubsection{Ideals}
\Page{
	\Theorem{PrincipleIdealRestriction}
	{
		\forall (A,\mu) : \MA \.
		\forall a \in A \.
		\MA\Big( (a), \mu_{|(a)} \Big) 
	}
	\Explain{This is obvious}
	\EndProof
	\\
	\Theorem{measureQuotient}
	{
		\NewLine ::		
		\forall (A,\mu) : \MA \.
		\forall I : \Ideal(A) \.
		\forall [a] \in \frac{A}{I} \.
		\exists \gamma \in \EReals_{++} \.
		\gamma = \min \{ \mu(b) | b \in A, \pi_I(b) = [a] \}	
	}
	\Explain{ 1 $\gamma = \inf \{ \mu(b) | b \in A, \pi_I(b) = [a] \}$ exists as a set
	 is bounded by below by $0$}
	\Explain{ 2 If $\gamma = \infty$ then the result is obvious}
	\Explain{ 3 Otherwise there is a decreasing sequence $b : \Nat \to A$ 
		such that $\pi_I(b_n) = [a]$ for any $n$ and 
		$\lim_{n \to \infty} \mu(b_n) = \gamma$}
	\Explain{ 4 Then  $c = \bigwedge^\infty_{n=1} b_n$
		is such that $\mu(c) = \gamma$ and $\pi_I(c) = a$}
	\Explain{ 4.1 Clearly 
		$
		\pi_I\left(\bigwedge^\infty_{n=1} b_n \right) = 
		\bigwedge^\infty_{n=1} \pi_I(b_n) =
		\bigwedge^\infty_{n=1} [a] = [a]		
		$}
	\Explain{ 5 So the infimum is atteined}
	\EndProof
	\\
	\DeclareFunc{measureQuotient}
	{  \prod (A,\mu) : \MA \. 
		\prod I : \Ideal(A) \.
		\frac{A}{I} \to \Reals_{++}
	}
	\DefineNamedFunc{measureQuotient}
	{a}{\mu_{I}(a)}
	{
		\min \{ \mu(b) | b \in A, \pi_I(b) = a \}	
	}
	\\
	\DeclareFunc{finiteElementsIdeal}{\prod (A,\mu) : \MA \. \Ideal(A)}
	\DefineNamedFunc{finiteElementsIdeal}{}{A^f}{\{a \in A | \mu(a) < \infty \}}
}\Page{
	\Theorem{MeasureIdealQuotient}
	{
		\forall (A,\mu) : \MA \.
		\forall I : \Ideal(A) \.		
		\MA\left( \frac{A}{I}, \mu_{I} \right) 
	}
	\Explain{ 1 Clearly $\mu_I(0) = 0$}
	\Explain{ 2 Assume that $[a] \neq 0$}
	\Explain{ 2.1 Then there exists $b \in A$ such that $\pi_I(a) = [a]$ and $\mu(b) = \mu_I[a]$ }
	\Explain{ 2.2 As $[a] \neq 0$, then $b \neq 0$, and henceforth $\mu(b) \neq 0$}
	\Explain{ 2.3 Thus, $\mu_I[a] \neq 0$}
	\Explain{ 3 Assume $[a] : \Nat \to \frac{A}{I}$ is disjoint}
	\Explain{ 3.1 It is possible to select representatives $b_n$ for each $[a_n]$ 
		such that $\mu(b_n) = \mu_I[a_n]$}
	\Explain{ 3.2 Then $b_n b_m \in I$ if $n \neq m$}	
	\Explain{ 3.3 Construct a new sequence 
		$c_n = b_n + \sum^{n-1}_{k=1} b_n b_k$ is a disjoint represintative sequance for $[a_n]$}
	\Explain{ 3.3.1 In fact $c = b$}
	\Explain{ 3.4 $\bigvee^\infty_{n=1} c_n$ is the minimal representative 
		of $\bigvee^\infty_{n=1} [a_n]$ }
	\Explain{ 3.4.1  Assume $d$ is a representative for $\bigvee^\infty_{n=1} a_n$}
	\Explain{ 3.4.2 If $\mu(d) < \mu\left(\bigvee^\infty_{n=1} c_n\right)$ 
		then we may construct $c_n \wedge d$ which is smaller then $c$}
	\Explain{ 3.4.3  But this is a contradiction}
	\Explain{ 3.5 So 
		$\mu_I\left(\bigvee^\infty_{n=1} [a_n]\right) =  
		\mu\left( \bigvee^\infty_{n=1} c_n \right) = 
		\sum^\infty_{n=1} \mu(c_n) =
		\sum^\infty_{n=1} \mu_I[a_n]
		$}	
	\EndProof
}
\newpage
\subsubsection{Measure Properties}
\Page{
	\DeclareType{ProbabilityAlgebra}{?\MA}
	\DefineType{(A,\pi)}{ProbabilityAlgebra}{\pi(e) = 1}
	\\
	\DeclareType{Finite\MA}{?\MA}
	\DefineType{(A,\mu)}{Finite\MA}{\mu(e) < \infty}
	\\
	\DeclareType{\sFinite\MA}{?\MA}
	\DefineType{(A,\mu)}{\sFinite\MA}
	{
		\exists a : \Nat \to A \. \forall n \in \Nat \. \mu(a_n) < \infty \And \bigvee^\infty_{n=1} a_n = e
	}
	\\
	\DeclareType{\Semifinite\MA}{?\MA}
	\DefineType{(A,\mu)}{\Semifinite\MA}
	{
		\forall a \in A \. \mu(a) = \infty \Imply \exists b \in A \. b  < a \And 0 < \mu(b) < \infty
	}
	\\
	\Conclude{\Loc\MA}{ \OComplete \And \Semifinite\MA }{\Type}
	\\
	\Theorem{ProbabilityConstruction}
	{
		\forall (X,\Sigma,\mu) \in \MEAS \.
		\TYPE{Probability}(X,\Sigma,\mu)
		\iff 	
		\TYPE{ProbabilityAlgebra}(A_\mu,\bar \mu)
	}
	\Explain{ This is obvious}
	\EndProof
	\\
	\Theorem{FiniteConstruction}
	{
		\forall (X,\Sigma,\mu) \in \MEAS \.
		\Finite(X,\Sigma,\mu)
		\iff 	
		\Finite\MA(A_\mu,\bar \mu)
	}
	\Explain{ This is obvious}
	\EndProof
	\\
	\Theorem{SigmaFiniteConstruction}
	{
		\forall (X,\Sigma,\mu) \in \MEAS \.
		\sFinite(X,\Sigma,\mu)
		\iff 	
		\sFinite\MA(A_\mu,\bar \mu)
	}
	\Explain{ This is obvious}
	\EndProof
	\\
	\Theorem{SemifiniteConstruction}
	{
		\NewLine ::
		\forall (X,\Sigma,\mu) \in \MEAS \.
		\Semifinite(X,\Sigma,\mu)
		\iff 	
		\Semifinite\MA(A_\mu,\bar \mu)
	}
	\Explain{ This is obvious}
	\EndProof
	\\
	\Theorem{LocalizableConstruction}
	{
		\NewLine ::		
		\forall (X,\Sigma,\mu) \in \MEAS \.
		\Loc(X,\Sigma,\mu)
		\iff 	
		\Loc\MA(A_\mu,\bar \mu)
	}
	\Explain{ This is obvious}
	\EndProof
}\Page{
	\Theorem{AtomInConstruction}
	{
		\NewLine ::		
		\forall (X,\Sigma,\mu) \in \MEAS \.
		\forall  E \in \Sigma \.
		E \in \Atom(X,\Sigma,\mu)
		\iff
		[E] \in \Atom(A_\mu,\bar\mu)
	}
	\Explain{ This is obvious}
	\EndProof
	\\
	\Theorem{AtomlessConstruction}
	{
		\NewLine ::		
		\forall (X,\Sigma,\mu) \in \MEAS \.
		\forall  E \in \Sigma \.
		E \in \Aless(X,\Sigma,\mu)
		\iff
		[E] \in \Aless(A_\mu,\bar\mu)
	}
	\Explain{ This is obvious}
	\EndProof
	\\
	\Theorem{PurelyAtomicConstruction}
	{
		\NewLine ::		
		\forall (X,\Sigma,\mu) \in \MEAS \.
		\forall  E \in \Sigma \.
		E \in \PA(X,\Sigma,\mu)
		\iff
		[E] \in \PA(A_\mu,\bar\mu)
	}
	\Explain{ This is obvious}
	\EndProof
	\\
	\Theorem{FinitenessPropertiesIerarchy}
	{
		\NewLine  :: 
		\forall (A,\mu) : \MA \.
		\TYPE{PobabilityAlgebra}(A,\mu)
		\Imply
		\Finite\MA(A,\mu)
		\Imply \NewLine \Imply
		\sFinite\MA(A,\mu)
		\Imply
		\Loc\MA(A,\mu)
		\Imply
		\Semifinite(A,\mu)				
	}
	\Explain{1 Most implications here are obvious 
		expect the one deriving Localizability from $\sigma$-finiteness}
	\Explain{2 So assume that $(A,\mu)$ is $\sigma$-finite } 
	\Explain{2.1 Then the corresponding Stone space $(\Z A, \Sigma_\mu, \bar \mu )$ is $\sigma$-finite}
	\Explain{2.2  But then $(\Z A, \Sigma_\mu, \bar \mu )$ is localizable }
	\Explain{2.3 So $(A,\mu)$ is also localizable}
	\EndProof
	\\
	\Theorem{MeasureAlgebraOfCompletion}
	{
		\forall (X,\Sigma, \mu) \in \MEAS \. 
		A_\mu \cong_\BOOL A_{\hat \mu}
	}
	\Explain{This is basically follows from definitions}
	\EndProof
	\\
	\Theorem{MeasureAlgebraOfLocallyDeterminedCompletion}
	{
		\NewLine ::		
		\forall (X,\Sigma, \mu) \in \MEAS \. 
		\exists  A_\mu \Arrow{\phi} A_{\bar \mu} : \BOOL \.
		\forall a \in A_{\bar \mu} \. \hat {\bar \mu}(a) < \infty \Imply \exists b \in A_{\mu} \. \phi(b) = a
		\And \NewLine \And
 		\forall b \in A_{\mu} \. \hat \mu(b) < \infty \Imply  \hat {\bar \mu}(\phi(b)) = \hat \mu (b)
	}
	\NoProof
	\\
	\DeclareFunc{localDeterminationMorphism}
	{
		\prod (X,\Sigma, \mu) \in \MEAS \. 
		 \BOOL(A_\mu, A_{\bar \mu})
	}
	\DefineNamedFunc{localDeterminationMorphism}{}{\phi_\mu}
	{\THM{MeasureAlgebraOfLocallyDeterminedCompletion}}
}
\Page{
	\Theorem{localDeterminationMorhismInjectivity}
	{
		\NewLine ::		
		\forall (X,\Sigma,\mu) \in \MEAS \.
		\Semifinite(X,\Sigma,\mu)
		\iff
		\Inj(A_\mu,A_{\bar \mu},\phi_\mu)
	}
	\NoProof
	\\
	\Theorem{localDeterminationMorhismBijectivity}
	{
		\NewLine ::		
		\forall (X,\Sigma,\mu) \in \MEAS \.
		\Loc(X,\Sigma,\mu)
		\iff
		\Bij(A_\mu,A_{\bar \mu},\phi_\mu)
	}
	\NoProof
	\\
	\Theorem{SemifinitenessCriterion}
	{
		\forall (A,\mu) : \MA \. \NewLine \.
		\Semifinite\MA(A,\mu)
		\iff
		\exists P : \PoU(A) \. \forall p \in P \. \mu(p) < \infty
	}
	\Explain{ 1 $(\Rightarrow)$ assume first that $(A,\mu)$ is semifinite}
	\Explain{ 1.1 Then $A^f$ is order dense in $A$}
	\Explain{ 1.2 By order density theorem there is a desired partition of unity}
	\Explain{ 2 $(\Leftarrow)$  Let $P$ be the partition of unity}
	\Explain{ 2.1 Assume $a \in A$ is such that $\mu(a) = \infty$}
	\Explain{ 2.2 Then there exists $p \in P$ such that $pa \neq 0$}
	\Explain{ 2.3 Note that this means that $\mu(pa) > 0$}
	\Explain{ 2.4 Also it is clear that $\mu(pa) \le \mu(p) < \infty$}
	\EndProof
	\\
	\Theorem{SemifiniteneSupElementExpression}
	{
		\NewLine ::
		\forall (A,\mu) : \Semifinite\MA(A,\mu) \.
		\forall a \in A \. a = \bigvee \{ b \in A : b \le a, \mu(b) < \infty   \}
	}
	\Explain{ This follows from the previous theorem}
	\EndProof
	\\
	\Theorem{SemifiniteneSupMeasureComputation}
	{
		\NewLine ::
		\forall (A,\mu) : \Semifinite\MA(A,\mu) \.
		\forall a \in A \. \mu(a) = \bigvee \{ \mu(b) \in A : b \le a, \mu(b) < \infty   \}
	}
	\Explain{ This follows from the previous theorem}
	\EndProof
}
\newpage
\subsubsection{Connections with other Boolean Properties}
\Page{
	\Theorem{SemifiniteIsWeaklyDistributive}
	{
		\NewLine ::	
		\forall (A,\mu) : \Semifinite\MA(A,\mu) \. 
		\WD(A,\mu)
	}
	\Explain{ 1 Assume 
		$X : \Nat \to 2^A$ is a sequence of downwards selected sets with 
		$\inf X_n =  0$ for every $n \in \Nat$}
	\Explain{ 2 Let $C = \{ a \in A : \forall n \in \Nat \. \exists  x \in X_n \. a \ge x \}$}
	\Explain{ 3 Assume $d \in A$ is such that $d \neq 0$}
	\Explain{ 4 Then there is an element $d' \le d$ such that $0 < \mu(d') < 0$}
	\Explain{ 5 $\inf_{x \in X} d'x = 0$ for each $n \in N$}
	\Explain{ 6 Select a sequence $x : \prod^\infty_{n=1} X_n$ suc that 
		$\mu(d'x_n) \le 2^{-n-2} \mu(d')$}
	\Explain{ 7 Define $c = \sup_{n=1} a_n \in C$}
	\Explain{ 8 Then $\mu(d'c) \le \sum^\infty_{n=0} \mu(c x_n) < \mu(d')$} 
	\Explain{ 9 This means that $d \not \le c$}
	\Explain{ 10 And as $d$  was arbitrary $\inf C = 0$}
	\EndProof
	\\
	\Theorem{SemifiniteIffCCC}
	{
		\forall (A,\mu ) : \Semifinite\MA(A,\mu) \. \NewLine \.
		\sFinite\MA(A,\mu) \iff \CCC(A)
	}
	\Explain{ 1 $(\Leftarrow)$ assume that $A$ has ccc}
	\Explain{ 1.1 Then there is a partition of unitity $P$ in $A$ consisting of finite elements 
		as $A$ is semifinite}
	\Explain{ 1.2 But as $A$ has ccc $P$ must be atmost countable}
	\Explain{ 1.3 This proves that $A$ is $\sigma$-finite}
	\Explain{ 2 $(\Rightarrow)$ assume that $(A,\mu)$ is $\sigma$-finite }
	\Explain{ 2.1 Then there exists a countable partition of unity $P$ of $A$ 
		wirh finite elements}
	\Explain{ 2.2 If $A$ is not ccc, then there exists an uncountable refinement $Q$ of $A$ 
		with finite elements}
	\ExplainFurther{ 2.3 Then by pigionhole principle there exists $p \in P$}
	\Explain{ \quad \quad such that  set $Q' = \{ q \in Q :  q \subset p \}$ such that $Q'$ is uncountable}
	\ExplainFurther{ 2.4 as for $\mu(q) > 0$ for any $q \in Q'$ 
		by pigionhole principle there exists some $n\in \Int$}
	\Explain{ \quad \quad such that there are an infinite number 
		of $q \in Q'$ with $\mu(q) \in [2^{-n-1},2^{-n}]$}
	\Explain{ 2.5 So $\mu(p) \ge \sum_{q \in Q'} \mu(q) = \infty$, but this is a contradiction}
	\EndProof
}\Page{
	\Theorem{SemifiniteIffProbabilityRenormalizationExists}
	{
		\NewLine ::		
		\forall (A,\mu) : \Semifinite\MA(A,\mu) \. A \neq \{0\} \Imply \NewLine \Imply
		\exists \pi : A \to \EReals_+ \.  \TYPE{ProbabilityAlgebra}(A,\pi)
	}
	\Explain{ 1 Corresponding Stone space is $\sigma$-finite}
	\Explain{ 2 So there exists a proper renormalization of $\bar \mu$ to a probability $\pi$
		with the same sets of measure zero}
	\Explain{ 3 Then the measure algebra of $(\Z A,\pi)$ is a probability algebra and 
		$A_\pi \cong_\BOOL A$}
	\EndProof
}
\newpage
\subsubsection{Subspace Measures and Indefinite Integrals}
\Page{
	\Theorem{MeasurableEnvelopePrincipleIdealIsomorphism}
	{
		\NewLine ::		
		\forall (X,\Sigma,\mu) \in \MEAS \.
		\forall Y \subset X \.
		\forall E : \ME(X,\Sigma,\mu,Y) \.  
		(A_{\mu|Y},\widehat{\mu|Y}) \cong_{\mathsf{MA}}  \Big( ([E]), \hat \mu_{|([E])} \Big)
	}
	\Explain{ This result is technically convoluted but actually is pretty intuituve}
	\EndProof
	\\
	\Theorem{PrincipleIdealIsomorphism}
	{
		\NewLine ::		
		\forall (X,\Sigma,\mu) \in \MEAS \.
		\forall E \in \Sigma \.  
		(A_{\mu|E},\widehat{\mu|E}) \cong_{\mathsf{MA}}  \Big( ([E]), \hat \mu_{|([E])} \Big)
	}
	\Explain{ A straightforward application of  a previous theorem}
	\EndProof
	\\
	\Theorem{ThickEquivalence}
	{
		\NewLine ::		
		\forall (X,\Sigma,\mu) \in \MEAS \.
		\forall Y : \Thick(X,\Sigma,\mu) \.  
		(A_{\mu|E},\widehat{\mu|E}) \cong_{\mathsf{MA}}  ( X, \hat \mu )
	}
	\Explain{ A straightforward application of a previous theorem}
	\EndProof
	\\
	\Theorem{IndefiniteIntegralPrincipleIdealIsomorphism}
	{
		\NewLine ::		
		\forall (X,\Sigma,\mu) \in \MEAS \.
		\forall f \in \Integrable_+(X,\Sigma,\mu)  \.
		\exists E \in \Sigma \.
		A_{f\;d\mu} \cong_\BOOL ([E])
	}
	\Explain{ We may assume that $\supp f$ has a measurable envelope $E$}
	\Explain{ Then the result is obvious as $\Null_{\mu} \subset \Null_{f\;d\mu}$}
	\EndProof
}
\newpage
\subsubsection{Simple Products}
\Page{
	 \DeclareFunc{simpleProduct}
	 {
	 	\prod_{I \in \SET} (I \to \MA) \to \MA
	 }
	 \DefineNamedFunc{simpleProduct}{A,\mu}{\prod_{i \in I} \left( A_i,  \mu_i \right)}
	 {
		\left( \prod_{i \in I} A_i, \sum_{i \in I} \mu_i \right)	 
	 }
	 \Explain{ Obviously $\sum_{i \in I} \mu_i (0) = \sum_{i \in I} 0 = 0$}
	 \Explain{ Also assume $a : \Nat \to \prod_{i \in I} A_i$ is disjoint}
	 \Explain{ Then  
			$\sum_{i \in I} \mu_i\left( \bigvee^\infty_{n=1} a_n \right) =
			 \sum_{i \in I} \sum^\infty_{n=1} \mu_i( a_{n,i} ) =
			 \sum^\infty_{n=1} \sum_{i \in I} \mu_i( a_{n,i} ) =
			 \sum^\infty_{n=1} \sum_{i \in I} \mu_i (a_n)  			
			$}
	\EndProof
	\\
	\Theorem{PrincipleIdealsInMeasureAlgebras}
	{		
		\NewLine ::		
		\forall I \in \SET \.
		\forall (A,\mu) : I \to \MA \.
		(A_i,\mu_i) \cong_{\mathsf{MA}} \left( (e_i), \left(\sum_{i \in I} \mu_i\right)_{|(e_i)}  \right)
	}
	\Explain{This is pretty ovious}
	\EndProof
	\\
	\\
	\Theorem{SimpleProductCoproductCorrespondance}
	{		
		\NewLine ::		
		\forall I \in \SET \.
		\forall (X,\Sigma,\mu) : I \to \MEAS \.
		\prod_{i \in I} (A_{\mu_i}, \hat \mu_i) \cong  
		\FUNC{measureAlgebra} \coprod_{i \in I} (X_i,\Sigma_i,\mu_i)
	}
	\Explain{ Obvious by Stone Theory}
	\EndProof
	\\
	\Theorem{SimpleProductOfSemifinite}
	{
		\NewLine ::
		\forall I \in \SET \.
		\forall (A,\mu) : I \to \Semifinite\MA \.
		\Semifinite\MA\left( \prod_{i \in I} (A,\mu) \right)
	}
	\Explain{ Assume $a$ has infinite measure in $(A,\mu)$}
	\Explain{ Then there exists $i \in I$ such that $a_i \neq 0$}
	\Explain{ As $(A_i,\mu_i)$ is semifinite there is $b \le a_i$ such that $0 < \mu_i(b) < \infty $}
	\Explain{ Then $be_i \le a$ and $0 < \sum_{j \in I} \mu_j (be_i) = \mu_i(b) < \infty$ }
	\EndProof
	\\	
}\Page{
	\Theorem{SimpleProductOfLocalizable}
	{
		\NewLine ::
		\forall I \in \SET \.
		\forall (A,\mu) : I \to \Loc\MA \.
		\Loc\MA\left( \prod_{i \in I} (A,\mu) \right)
	}
	\Explain{ Let $J$ be a set and $a : J \to \prod_{i \in I} (A_i,\mu_i)$ }
	\Explain{ Then $\sup_{j \in J} a_j = (\sup_{j \in J} a_{j,i} )_{i \in I}$}
	\EndProof
	\\
	\Theorem{PoUProductRepresentation}
	{
		\NewLine ::
		\forall (A,\mu) : \MA \.
		\forall (e_n)^\infty_{n=1} : \PoU(A) \.
		(A,\mu) \cong_{\mathsf{MA}} \prod^\infty_{n=1} \Big((e_n), \mu_{|(e_m)}\Big)
	}
	\Explain{This is pretty obvious}
	\EndProof
	\\
	\Theorem{PoUProductRepresentation}
	{
		\NewLine ::
		\forall (A,\mu) : \Loc\MA \.
		\exists I \in \SET \.		
		\exists  (B,\nu)  : I \to \Finite\MA \. \NewLine \.  
		(A,\mu) \cong_{\mathsf{MA}} \prod_{i \in I} (B_i,\nu_i) 
	}
	\Explain{ It is possible to select a partition of unity $P$ of $A$ consisting of finite elements}
	\Explain{ Then by previous theorem $(A,\mu) \cong \prod_{p \in P} \Big( (p), \mu_{|(p)} \Big)$}
	\Explain{ And each $\Big( (p), \mu_{|(p)} \Big)$ are obviously finite}
	\EndProof
	\\
	\Theorem{LocalizableMeasureAlgebrasHasLocallyDeterminedRepresentations}
	{
		\NewLine ::
		\forall (A,\mu) : \Loc\MA \.
		\exists  (X,\Sigma,\nu)  : \LocDet \.
		(A,\mu) \cong_{\mathsf{MA}} (A_\nu,\hat \nu) 
	}
	\Explain{ Represent $(A,\mu) \cong_{\mathsf{MA}} \prod_{i \in I} (B_i,\nu_i) $}
	\Explain{ Then Stone's spaces $\Z \; B_i$ correspond to finite measure spaces}
	\Explain{ And Stone's space of product correspond to a disjoint union of $\Z \; B_i$}
	\Explain{ But such spaces are trivially locally determined}
	\EndProof
}
\newpage
\subsubsection{Strictly Localizable Spaces}
\Page{
	\Theorem{StrictlyLocalizableSpacePoU}
	{
		\NewLine ::		
		\forall (X,\Sigma,\mu) : \SLoc \.
		\forall P : \PoU(A_\mu) \. \NewLine \.
		\exists  E  : P \to \Sigma \.
		\forall p \in P \. [E_p] = p
		\And
		\TYPE{Decomposition}(X,\Sigma,\mu,\im E)
	}
	\NoProof
}
\newpage
\subsubsection{Subalgebras}
\Page{
	\Theorem{SubalgebaMeasureAlgebra}
	{
		\forall (A,\mu) : \MA \.
		\forall B \subset_\sigma A \.
		\MA(B,\mu_{|B})
	}
	\Explain{This is obvious}
	\EndProof
	\\
	\Theorem{SubalgebaFinifteMeasureAlgebra}
	{
		\NewLine ::		
		\forall (A,\mu) : \Finite\MA \.
		\forall B \subset_\sigma A \.
		\Finite\MA(B,\mu_{|B})
	}
	\Explain{This is obvious}
	\EndProof
	\\
	\Theorem{SigmaFiniteSubalgebraMeasureAlgebra}
	{
		\NewLine ::		
		\forall (A,\mu) : \sFinite\MA \.
		\forall B \subset_\sigma A \. \NewLine \.
		\Semifinite\MA(B,\mu_{|B}) \Imply
		\sFinite\MA(B,\mu_{|B})
	}
	\Explain{ 1 The set $B^f$ is order-dense in $B$}
	\Explain{ 2 But then $B^f$ is also order-dense in $A$}
	\Explain{ 3 Select a finite-measured countable partition of unity $P$ in $A$}
	\Explain{  4 If $B$ is not $\sigma$-finite, then there is a subordinate uncountale partition of unity $Q$}	
	\Explain{ 5 Then there would exist 
		a uncountable refinement of $P$ subordinate to $Q$}
	\Explain{ 6 Then $P$ must contain an infinite element, but this is imposible!}
	\Explain{ 7 So $Q$ must be countable, and so $(B,\mu_{|B})$ must be countable}
	\EndProof
	\\
	\Theorem{FinifteMeasureAlgebraBySubalgebra}
	{
		\NewLine ::		
		\forall (A,\mu) : \MA \.
		\forall B \subset_\sigma A \.
		\Finite\MA(B,\mu_{|B}) \Imply \Finite\MA(A,\mu)
	}
	\Explain{This is obvious}
	\EndProof
	\\
	\Theorem{ProbabilityAlgebraBySubalgebra}
	{
		\NewLine ::		
		\forall (A,\mu) : \MA \.
		\forall B \subset_\sigma A \. \NewLine \.
		\TYPE{ProbabilityAlgebra}(B,\mu_{|B}) \Imply \TYPE{ProbabilityAlgebra}(A,\mu)
	}
	\Explain{This is obvious}
	\EndProof
}\Page{
	\Theorem{SigmaFiniteAlgebraBySubalgebra}
	{
		\NewLine ::		
		\forall (A,\mu) : \MA \.
		\forall B \subset_\sigma A \. \NewLine \.
		\sFinite(B,\mu_{|B}) \Imply \sFinite(A,\mu)
	}
	\Explain{This is obvious}
	\EndProof
}
\newpage
\subsubsection{Localization}
\Page{
	\Theorem{MeasureAlgebraCompletion}
	{
		\NewLine ::		
		\forall (A,\mu) : \Semifinite\MA \.
		\exists! \hat \mu : \tau(A) \to \EReals_{++} \. \NewLine \.
		\hat \mu_{|A} = \mu  \And \Loc\MA(\tau(A),\hat \mu) 
	}
	\Explain{ 1 Define $\hat \mu(t) =  \sup \{ \mu(a) | a \in A, a \le t \}$}
	\Explain{ 2 As $A$ is order dense in $\tau(A)$, it holds that $\hat\mu(a) = 0\iff a = 0$ 
		for any $a \in \tau(A)$}
	\Explain{ 3 If $t : \Nat \to \tau(A)$ is disjoint then 
		$\hat \mu\left( \bigvee_{n=1}^\infty t_n\right) = \sum^\infty_{n=1} \hat \mu(t_n) $
	}
	\Explain{ 3.1 Write $S = 
		\{ a \in A : \exists c : \Nat \to A \.  a = \lim_{n \to \infty} c_n \And c \le t  \}$}
	\Explain{ 3.2 Then there is $s = \sup S \in \tau(A)$}
	\Explain{ 3.3 We write 
		$
			\hat \mu(s) = \sup_{c \le t } \mu\left( \bigvee^\infty_{n=1} c_n \right) =
			\sup_{c \le t} \sum^\infty_{n=1} \mu(c_n) = 
			\sum^\infty_{n=1} \sup_{c \le t_n} \mu(c) =
			\sum^\infty_{n=1} \hat \mu(t_n)
		$ }
	\Explain{ 4 Obviously $(\tau(A),\hat \mu)$ is semifinite and order-complete, and hence Localizable}
	\EndProof
	\\
	\DeclareFunc{localization}
	{
		\Semifinite\MA \to \Loc\MA
	}
	\DefineNamedFunc{localization}{A,\mu}{\Big(\tau(A),\tau(\mu)\Big)}
	{
		\THM{MeasureAlgebraCompletion}	
	}
	\\
	\Theorem{LocalizationFiniteEmbedding}
	{
		\NewLine ::		
		\forall (A,\mu) : \Semifinite\MA \.
		\iota_\tau(A^f) =  \tau^f(A) 
	}
	\Explain{ 1 Assume $t \in \tau(A)$ such that $\hat \mu(t) < \infty$}
	\Explain{ 2 Note, $\hat \mu(t) = \sup_{a \le t} \mu(a)$ }
	\Explain{ 3 So we may select an increasing $a : \Nat \to A$
		such that $\lim_{n \to \infty} \mu(a_n) = \hat \mu(t)$}
	\Explain{ 4 Then $b = \bigvee^\infty_{n=1} a_n \in A$ and $\hat \mu(b) = \mu(b) = \hat \mu(t)$}
	\Explain{ 5 So $\mu(t \setminus b) = 0$, and  so $t = b \in A$ as clearly $b < t$}
	\EndProof
}
\newpage
\subsubsection{Stone Spaces}
\Page{
	\Theorem{LocallalizableMeasureAlgebraHasStrictlyLocalizableStoneSpace}
	{
		\NewLine ::
		\forall (A,\mu) : \Loc\MA \.
		\SLoc (\Z\; A, \Sigma_\mu, \bar \mu)  	
	}
	\Explain{ 1 We already proved that $\bar \mu$ is locally determined}
	\Explain{ 2 As $(A,\mu)$ is semifinite there is a partition of  unity $P$ 
		consisting of finite elements}
	\Explain{ 3 Use Stone representation $S_A(P)$ to construct a corresponding set in $\Z\; A$}
	\Explain{ 4 Assume $E \in \Sigma_\mu$ such that $\bar \mu(E) > 0$}
	\Explain{ 5 By definition of Stone's Space there is a clopen set $F \in \Z \; A$ 
		such that $E \du F$ is meager}
	\Explain{ 6 And there is a Stone representation $a \in A$ such that $F = S_A(a)$}
	\Explain{ 7 Then $\mu(a) = \nu(S_A(a)) = \nu(E) > 0$}
	\Explain{ 8 So, there exists $p \in P$ such that $ap \neq 0$}
	\Explain{ 9 Ths means that $\nu(E \cap S_A(p)) > 0$}
	\Explain{ 10 As  $E$ was arbitrary this means that $S_A(P)$ 
		provides a strict localization for $\bar \mu$}
	\EndProof
	\\
	\Theorem{MeagerSetsAreNowhereDense}
	{
		\NewLine ::		
		\forall (A,\mu) : \Semifinite\MA \.		
		\forall  M  \in \mgr(\Z\; A)  \. 
		 \ND(\Z\;A, M)
	}
	\Explain{ 1 As it was shown $A$ is $\WD$ boolean algebra}
	\Explain{ 2 And this is a property of $\WD$ boolean algebra}
	\EndProof
	\\
	\Theorem{StoneSpaceMeasurableExpression}
	{
		\NewLine ::		
		\forall (A,\mu) : \Semifinite\MA \.		
		\forall  E \in \Sigma_\mu  \.  \NewLine \.
		\exists U : \Clopen(\Z\;A) \.
		\exists F : \ND(\Z\;A)	\.	
		E = U \cap F
	}
	\Explain{ 1 This is clear from the previous theorem}
	\EndProof
	\\
	\Theorem{StoneSpaceMeasureComputation}
	{
		\NewLine ::		
		\forall (A,\mu) : \Semifinite\MA \.		
		\forall  E \in \Sigma_\mu  \.  \NewLine \.
		\bar \mu(E) = \sup \Big\{ \mu(U) \Big| U : \Clopen(\Z\;A) , U \subset E   \Big\}
	}
	\Explain{ 1 This is clear from the previous theorem}
	\EndProof
	\\
	\Theorem{StoneSpaceCLDIsStrictlyLocalizable}
	{
		\NewLine ::
		\forall (A,\mu)  : \Semifinite\MA \.
		\SLoc(\Z \;A, \bar \Sigma_\mu, \bar{\bar \mu}) 
	}
	\NoProof
}\Page{
	\Theorem{StoneSpaceCLDZeroSets}
	{
		\NewLine ::
		\forall (A,\mu)  : \Semifinite\MA \.
		\Null_{\bar{\bar \mu}} = \Null_{\bar \mu} 
	}
	\NoProof
	\\
	\Theorem{FiniteStoneSpaceMeasureComputation}
	{
		\NewLine ::		
		\forall (A,\mu) : \Finite\MA \.		
		\forall  E \in \Sigma_\mu  \.  \NewLine \.
		\bar \mu(E) = \inf \Big\{ \mu(U) \Big| U : \Clopen(\Z\;A), E \subset U \Big\}
	}
	\Explain{ 1 This is clear from the previous theorem}
	\EndProof
}
\newpage
\subsubsection{Purely Infinite Elements}
\Page{
	\DeclareFunc{purelyInfiniteElements}
	{
		\prod (A,\mu) : \MA \. \SIdeal(A)
	}
	\DefineNamedFunc{purelyInfiniteElements}{}{I_\infty(\mu}
	{
		\{ a \in A : \forall b  \in A \. b \le a \And \mu(b) < \infty \Imply b = 0  \}
	}
	\\
	\DeclareFunc{semifiniteMeasure}{\prod (A,\mu) : \MA \. \frac{A}{I_\infty(\mu)} \to \EReals_+}
	\DefineNamedFunc{semifiniteMeasure}{[a]}{\mu_{\mathrm{sf}}}
	{\sup \{ \mu(b) | b \in A : b \le a \And \mu(b) < \infty\}}
	\Explain{ If $[a] = [b]$, then $a \du b \in I_\infty(\mu)$}
	\Explain{ So $\mu_{\mathrm{sf}}$ is well-defined} 
	\\
	\Theorem{SemifiniteMeasureIsMeasure}
	{
		\NewLine ::		
		\forall (A,\mu) : \MA \. \Semifinite\MA\left( \frac{A}{I}, \mu_{\mathrm{sf}} \right)	
	}
	\Explain{ 1 If $\mu_{\mathrm{sf}}[a] = 0$, then clearly $a \in I_\infty$}
	\Explain{ 2 Assume $[a] : \Nat \to A$ is disjoint}
	\Explain{ 2.1 Then $a_n a_m \in I_\infty$  if $n \neq m$}
	\Explain{ 2.2 Select increasing $b : \Nat \to A^f$ such that $b_n \le \bigvee^\infty_{k=1} a_k$
		and $\lim_{n \to \infty} \mu(b_n) = \mu_{\mathrm{sf}} \left[\bigvee^\infty_{k=1} a_k  \right] 
			= \mu_{\mathrm{sf}} \bigvee^\infty_{k=1} [a_k]$}
	\Explain{ 2.3  By (2.1) we mat assert that $ab_n$ is disjoint and then $\bigvee^\infty_{k=1} a_kb_n = b_n$
	 for any $n \in \Nat$ }
	\Explain{ 2.4 So $\mu(b) = \sum^\infty_{k=1}  \mu(a_k b_n) $}
	\ExplainFurther{ 2.5 By taking limits and using monotonic convergence theorem} 
	\Explain{ $\quad \quad \sum^\infty_{k=1} \mu_{\mathrm{sf}}[a_k] = 
		\sum^\infty_{k=1}  \lim_{n \to \infty} \mu(a_k b_n) = 
		\lim_{n \to \infty} \mu(b_n) = 
		\mu_{\mathrm{sf}} \bigvee^\infty_{k=1} [a_k]$}	
	\Explain{ 3 Clearly  $\mu_{\mathrm{sf}}[a] < \mu(a)$}	
	\Explain{ 3.1 If $\mu_{\mathrm{sf}}[a] = \infty$, then $a \not \in I_\infty$}
	\Explain{ 3.2 So it is possible to select $b \in A$ such that $b \le a$ and $0 < \mu(b) \le a$}
	\Explain{ 3.3 $0 < \mu_{\mathrm{sf}}[b] \le \mu(b) < \infty$}
	\Explain{3.4 This proves that $\left( \frac{A}{I}, \mu_{\mathrm{sf}} \right)$ is semifinite}
	\EndProof
}
\newpage
\subsection{Topology}
\subsubsection{Subject}
\Page{
	\DeclareFunc{measureAlgebraAsTopologicalSpace}
	{
		\MA \to \TOP
	}
	\DefineNamedFunc{measureAlgebraAsTopologicalSpace}{(A,\mu)}{(A,\mu)}
	{
		\NewLine \de 		
		\Big( A, \mathcal{W}\big(A^f \times A^f,
			\Reals, \Lambda a \in A^f \. \Lambda b \in A^f \. \Lambda c \in A \. \mu(ac + ab )\big)\Big) 
	}
	\\
	\DeclareFunc{measureAlgebraAsUniformlSpace}
	{
		\MA \to \UNI
	}
	\DefineNamedFunc{measureAlgebraAsUniformSpace}{(A,\mu)}{(A,\mu)}
	{
		\NewLine \de 		
		\Big( A, \mathcal{I}\big(A^f \times A^f,
			\Reals, \Lambda a \in A^f \. \Lambda b \in A^f \. \Lambda c \in A \. \mu(ac \du ab )\big)\Big) 
	}
	\\
	\DeclareFunc{metricOfFrechetNikodym}
	{
		\prod (A,\mu) : \MA \.  \TYPE{Metric}(A^f) 
	}
	\DefineNamedFunc{metricOfFrechetNikodym}{}{\rho_\mu}
	{
		\Lambda a,b \in A^f \. \mu(a \du b)
	}
	\\
	\Theorem{BooleanOperationsAreUniformlyContinuous}
	{
		\NewLine ::		
		\forall (A,\mu) : \MA \.  (*),(\setminus),(\vee),(\wedge) \in \UNI(A \times A, A)
	}
	\Explain{ 1 Let $\circ$ stay for any binary operation above}
	\Explain{ 2 Select $c,d \in A$}
	\Explain{ 3 Then 
		$
			\mu\Big(a(c \circ d) +  b\Big) \le 
			\mu\Big( a(c \vee d) + b \Big) \le 
			\mu( ac + d) + \mu(ad + b)
		$  }
	\Explain{ 4 So $\mu$ is bounded by the sum of uniform functions and is uniformly continuous}
	\EndProof
	\\
	\Theorem{FiniteElementsAreDense}
	{
		\NewLine ::		
		\forall (A,\mu) : \MA \.  \Dense(A,A^f)
	}
	\Explain{ 1 Select $c \in A$}
	\Explain{ 2 Then $c$ has a base of neighborhoods of form 
		$U = \{ u \in A : \mu(au + ac) \le r\}$ with $a \in A^f, r \in \Reals_{++}$}
	\Explain{ 3 But then $ac \in U$ and $ac \in A^f$}
	\EndProof
	\\
	\Theorem{FiniteMeasureAlgebraHasUniformlyContinuousMeasure}
	{
			\NewLine
			\forall (A,\mu) : \Finite\MA \. \mu \in \UNI(A,\Reals_{++})
	}
	\Explain{ This is pretty obvious as $\mu = \rho_\mu(0,a)$}
	\EndProof
}\Page{
	\Theorem{FiniteMeasureAlgebraHasUniformlyContinuousMeasure}
	{
			\NewLine
			\forall (A,\mu) : \Finite\MA \. \mu \in \UNI(A,\Reals_{++})
	}
	\Explain{ This is pretty obvious as $\mu = \rho_\mu(0,a)$}
	\EndProof
	\\
	\Theorem{SemifinitMeasureAlgebraHasLowerSemicontinuousMeasure}
	{
			\NewLine
			\forall (A,\mu) : \Semifinite\MA \. \mu \in \TYPE{LowerSemicontinuous}(A,\EReals_{++})
	}
	\Explain{ 1 Assume $a \in A$ and $\alpha \in \Reals_{+}$ such that $\mu(a) > \alpha$ }
	\Explain{ 2 As $A$ is semifinite there exists $b \le a$ such that $\infty > \mu(b) > \alpha$}
	\Explain{ 3 Assume $c \in A$ is such that $\mu(b + cb) < \mu(b) - \alpha$ }
	\Explain{ 4 Then 
		$\mu(c) \ge \mu(cb) = 
		\mu(b)  - \mu(b(a\setminus c)) =  
		\mu(b) - \mu(b + cb) > \alpha $}
	\EndProof
	\\
	\Theorem{MeasureAlgebraHasUniformlyContinuousFinitisedMeasure}
	{
			\NewLine
			\forall (A,\mu) : \MA \. 
			\forall a \in A^f \.			
			(\Lambda c \in A \. \mu(ac)) \in \UNI(A,\Reals_{++})
	}
	\Explain{  This is simmilar to the case of finite measure space}
	\EndProof
	\\
	\DeclareFunc{finiteElementMetric}
	{
		\prod A : \MA \. A^f \to \TYPE{Semimetric}(A)
	}
	\DefineNamedFunc{finiteElementMetric}{a}{\rho_a}
	{
		\Lambda x,y \in A \. \mu(ax + ay)	
	}
	\\
	\Theorem{MeasurAlgebraProductTopology}
	{
		\NewLine ::
		\forall I \in \SET \.
		\forall (A,\mu) : I \to \MA \. 
		\prod_{i \in I} (A,\mu) =_\TOP \left(\prod_{i \in I} A_i, \sum_{i \in I} \mu_i \right)
	}
	\NoProof
}
\newpage
\subsubsection{Relations with an Order Structure}
\Page{
	\DeclareFunc{upwardDirectedFilter}
	{
		\NewLine ::		
		\prod (A,\mu) : \MA \. 
		 \TYPE{NonEmpty} \And \TYPE{UpwardsDirected}(A) \to \TYPE{CauchyFilerbase}(A)
	}
	\DefineNamedFunc{upwardDirectedFilter}{D}{\F(\uparrow D)}
	{
		\Big\{ \{ c \in D : d \le c \} \Big| d \in D \Big\} 	
	}
	\Explain{ 1 Write $F_d = \{ c \in D : d \le c \}$}
	\Explain{ 2 $\F(\uparrow D)$ is a filter}
	\Explain{ 2.1 As $D$ is non empty, $\F(\uparrow D)$ is also non-empty}
	\Explain{ 2.2  $d \in F_d$, so $F_d \neq \emptyset$ 
		and henceforth $\emptyset \not \in \F(\uparrow D)$}
	\Explain{ 2.3 Assume $F_d,F_f \in \F(\uparrow D)$    }
	\Explain{ 2.3.1  Then there is an element $g \in D$ such that $g \ge f \vee d$}
	\Explain{ 2.3.2  Note, that $F_g \subset F_d \cap F_f$ and $F_g \in  \F(\uparrow D)$ }
	\Explain{ 3 $\F(\uparrow D)$ is Cauchy}
	\Explain{ 3.1 Assume $U$ is some measure connector for $(A,\mu)$}
	\Explain{ 3.2 then there is an element $a \in A^f$ and $r \in \Reals_{++}$
		such that $\{ (f,g) \in A \times A : \mu(af  +  ag) < r \} \subset U$}
	\Explain{ 3.3 The set $\{ \mu(ad) | d \in D  \}$ is bounded  by $\mu(a)$,
		so supremum is attained}
	\Explain{ 3.4 So there is $f \in D$, so $\mu(ad) < \mu(af) + r/2$ for any $d \in D$   }
	\Explain{ 3.5 Assume $g,h \in F_f$ }
	\Explain{ 3.5 Then 
		$
			\mu(ag + ah) \le \mu(ag \setminus af) + \mu(ah \setminus af) =
			\Big(\mu(ag) - \mu(af)\Big) + \Big(\mu(ah) - \mu(af)\Big)  < r 
		$}
	\Explain{ 3.6 Thus, $(g,h) \in U$ and $F_f \times F_f \subset U$}
	\EndProof
	\\
	\Theorem{UpwardsDirectedSup}
	{
		\NewLine ::		
		\forall (A,\mu) : \Semifinite\MA \. 
		\forall D :  \TYPE{UpwardsDirected}(A) \to \TYPE{CauchyFilerbase}(A)  \.
		\forall a \in A	\. \NewLine \.
		a = \sup D \Imply a = \lim \F(\uparrow D)  
	}
	\Explain{ 1 Assume $a = \sup D$}
	\Explain{ 2 Assume $U$ is an uniformity fo $(A,\mu)$}
	\Explain{ 3 then there is an element $c \in A^f$ and $r \in \Reals_{++}$
		such that $\{ g \in A \times A : \mu(ca  +  cg) < r \} \subset U(a)$}
	\Explain{ 4 Consider set $M = \{ \mu(cd) | d \in D  \}$}
	\Explain{ 5 Then $\sup M = \mu(ca)$}
	\Explain{ 6 So there is $d \in D$ such that $\mu(ca + cd) < r$}
	\Explain{ 7 But $ d \le f \le a $for any $f \in F_d$}
	\Explain{ 8 Thus $\mu(cf + cd) < r$ and $F_d \subset U(a)$}
	\Explain{ 9 Thus, $da = \lim \F(\uparrow D)$}
	\EndProof
}\Page{
	\Theorem{UpwardsDirectedLimit}
	{
		\NewLine ::		
		\forall (A,\mu) : \Semifinite\MA \. 
		\forall D : \TYPE{NonEmpty} \And \TYPE{UpwardsDirected}(A)  \.
		\forall a \in A	\. \NewLine \.
		a = \sup D \Imply a  \in \cl_{A} D
	}
	\NoProof
	\\
	\Theorem{UpwardsDirectedFilterLimit}
	{
		\NewLine ::		
		\forall (A,\mu) : \Semifinite\MA \. 
		\forall D : \TYPE{NonEmpty} \And \TYPE{UpwardsDirected}(A)  \.
		\forall a \in A	\. \NewLine \.
		a = \lim \F(\uparrow D) \iff a = \sup D  
	}
	\Explain{ 1 $(\Rightarrow) \quad a = \lim \F(\uparrow D)$ }
	\Explain{ 1.1 Then for any connector $U$ of $(A,\mu)$
		There is some $F \in \F(\uparrow F)$ such that $F \subset U(a)$}
	\Explain{ 1.2 Assume $d \in D$ }
	\Explain{ 1.3 Assume $d \not \le a$}
	\Explain{ 1.4 Then there is $f \in A$ such that $f \le d \setminus a$ 
		and $0 < \mu(f) < \infty$}
	\Explain{ 1.5 Thus $\mu(fh + fa) \ge \mu(f)$ for every $h \in F_s$}
	\Explain{ 1.6 But $G \cap F_d \neq \emptyset$ for  any  $G \in \F(\uparrow D)$ 
		so this contradicts $(1.1)$}
	\EndProof
	\\
	\DeclareFunc{lowerDirectedFilter}
	{
		\NewLine ::		
		\prod (A,\mu) : \MA \. 
		 \TYPE{NonEmpty} \And \TYPE{LowerDirected}(A) \to \TYPE{CauchyFilerbase}(A)
	}
	\DefineNamedFunc{loweDirectedFilter}{D}{\F(\uparrow D)}
	{
		\Big\{ \{ c \in D : d \ge c \} \Big| d \in D \Big\} 	
	}
	\\
	\Theorem{LowerDirectedInf}
	{
		\NewLine ::		
		\forall (A,\mu) : \Semifinite\MA \. 
		\forall D :  \TYPE{NonEmpty} \And \TYPE{LowerDirected}(A)   \.
		\forall a \in A	\. \NewLine \.
		a = \inf D \Imply a = \lim \F(\uparrow D)  
	}
	\Explain{ By duality}
	\EndProof	
	\\
	\Theorem{UpwardsDirectedLimit}
	{
		\NewLine ::		
		\forall (A,\mu) : \Semifinite\MA \. 
		\forall D :  \TYPE{NonEmpty} \And \TYPE{LowerDirected}(A)  \.
		\forall a \in A	\. \NewLine \.
		a = \inf D \Imply a  \in \cl_{A} D
	}
	\Explain{ By duality}
	\EndProof
	\\
	\Theorem{UpwardsDirectedFilterLimit}
	{
		\NewLine ::		
		\forall (A,\mu) : \Semifinite\MA \. 
		\forall D : \TYPE{NonEmpty} \And \TYPE{LowerDirected}(A)  \.
		\forall a \in A	\. \NewLine \.
		a = \lim \F(\uparrow D) \iff a = \inf D  
	}
	\Explain{ By duality}
	\EndProof	
}\Page{
		\Theorem{ClosedSetsAreOrderClosed}{ 
		\forall (A,\mu) : \MA \. 
		\forall F : \Closed(A) \.
		\oC(A,F)
	}	
	\Explain{Follows from previous theorems in this chapter}
	\EndProof
	\\
	\Theorem{DenseSetsAreOrderDense}{ 
		\forall (A,\mu) : \MA \. 
		\forall F : \Dense(A) \.
		\OD(A,F)
	}	
	\Explain{Follows from previous theorems in this chapter}
	\EndProof
	\\
	\Theorem{ClosedRays}
	{
		\forall (A,\mu) : \Semifinite\MA \.
		\forall a \in A \.		
		\Closed\Big( A, \{ c \in A : c \le a  \} \And \{ c \in A : c \ge a \} \Big)
	}
	\Explain{ 1 Let $F = \{ c \in A : c \le a \}$ }
	\Explain{ 2 Assume $d \in F^\c$}
	\Explain{ 3 Then $d \setminus a \neq 0$}
	\Explain{ 4 As $A$ is semifinite there is an $g \in A^f$ 
		such that $g \le d \setminus a$ and $0 < \mu(g)$}
	\Explain{ 5 $\rho_g(d,f) \ge \mu(g)$ fo any $f \in F^\c$}
	\Explain{ 6 And this means that $F^\c$ and $F$ is closed}
	\EndProof
	\\
	\Theorem{SupremumConvergence}
	{
		\forall A : \MA \.
		\forall a : \Nat \uparrow A \.
		\forall s \in A \.
		s = \sup_{n=1} a_n \Imply  s = \lim_{n=1} a_n
	}
	\Explain{ This is obvious now}
	\EndProof
	\\
	\Theorem{InfimumConvergence}
	{
		\forall A : \MA \.
		\forall a : \Nat \downarrow A \.
		\forall s \in A \.
		s = \inf_{n=1} a_n \Imply  s = \lim_{n=1} a_n
	}
	\Explain{ This is obvious now}
	\EndProof
	\\
	\DeclareType{\SI}{
		\prod A : \MA \. ?(\Nat \to A)	
	}
	\DefineType{a}{\SI}{\forall n \in \Nat \. \sum^\infty_{n=1} \mu(a_n + a_{n+1}) < \infty}
}\Page{
	\Theorem{SummableIncrementsLimSupLimInfEq}
	{
		\NewLine ::		
		\forall A : \MA \.
		\forall a : \SI(A) \. 
		\inf_{n=1} \sup_{m=n} a_n = \sup_{n=1} \inf_{m=n} a_n	
	}
	\Explain{ 1 Let $\alpha_n =   \mu(a_n + a_{n+1}),\beta_n = \sum^\infty_{m=n} \alpha_n$}
	\Explain{ 2 As $a$ has summable increments this means $\beta \downarrow 0$}
	\Explain{ 3 Let $b_n = \sup_{m \ge n} a_m + a_{m + 1} = \bigvee^\infty_{m=n} a_m + a_{m+1}$}
	\Explain{ 4 Then $\mu(b_n) \le \sum^\infty_{m=n} \mu(c_m + c_{m+1} ) = \beta_n $}
	\Explain{ 5 Assume $m \le n$}
	\Explain{ 6 And also $a_m + a_n \le \sup_{m\le k \le n} a_k + a_{k+1} \le b_n$}
	\Explain{ 7 So $a_n \setminus b_n \le a_m \le a_n \vee b_n$}
	\Explain{ 8 Thus $a_n \setminus b_n \le \inf_{k \ge m} a_k \le \sup_{k \ge m} a_k \le a_n \vee b_n$}
	\Explain{ 9 By taking limits in $m$ one gets
		$a_n \setminus b_n \le \inf_{m=1} \sup_{k=n} a_k \le \sup_{m=1} \inf_{k=m} a_k \le 
			a_n \vee b_n$}
	\Explain{ 10 $a_n + \inf_{m=1} \sup_{k=m} a_k \le b_n $}
	\Explain{ 11 $a_n + \sup_{m=1} \inf_{k=m} a_k \le b_n $}
	\Explain{ 12 From (10) and (11) 
		$\inf_{m=1} \sup_{k=m} a_k \setminus \sup_{m=1} \inf_{k=m} a_k  \le b_n$}
	\Explain{ 13 But $\lim_{n \to \infty}  b_n = 0$}
	\Explain{ 14 So $\inf_{m=1} \sup_{k=m} a_k  = \sup_{m=1} \inf_{k=m} a_k $}
	\EndProof
	\\
	\Theorem{SummableIncrementsLim}
	{
		\NewLine ::		
		\forall A : \MA \.
		\forall a : \SI(A) \. 
		\forall x \in A \. \NewLine \.
		x = \lim_{n \to \infty} a_n \Imply
		\inf_{n=1} \sup_{m=n} a_n = x = \sup_{n=1} \inf_{m=n} a_n	
	}
	\Explain{ This follows from the previous proof}
	\EndProof
}
\newpage
\subsubsection{Classification Theorems}
\Page{
	\Theorem{SemifiniteIffHausdorff}
	{
		\forall (A,\mu) : \MA \.
		\Semifinite\MA(A,\mu) \iff \TYPE{T2}(A)
	}
	\Explain{1 $(\Rightarrow)$ assume that $(A,\mu)$ is semifinite}
	\Explain{1.1 Take $x,y \in A$ such that $x \neq y$}
	\Explain{1.2 Then  $x +  y \neq 0$ so there is $a \in A^f$ such that $\mu(a) > 0$ and $a < x + y$ }
	\Explain{1.3 So $\rho_a(x,y) = \mu(a) > 0$}
	\Explain{1.4 And cells of form $\Cell_{\rho_a}(x,\mu(a)/2)$ and $\Cell_{\rho_a}(y,\mu(a)/2)$
		produce the separation}
    \Explain{2 $(\Leftarrow)$ assume that $A$ is Hausdorff in the topology of $(A,\mu)$}
    \Explain{2.1 Assume $x \in A$ such that $\mu(x) = \infty$}
    \Explain{2.2 Then $x \neq 0$}
    \Explain{2.3 Assume $a \in A^f$}
    \Explain{2.4 If $xa = 0$ then $\rho_a(x,0) = 0$}
    \Explain{2.5 So, as $A$ is Hausdorff there mustb some $a \in A^f$ such that $xa \neq 0$}
    \Explain{2.6 But this means that $(A,\mu)$ is semifinite}
    \EndProof
    \\
    \Theorem{SigmaFiniteIffMetrizable}
	{
		\NewLine ::		
		\forall (A,\mu) : \MA \.
		\sFinite\MA(A,\mu) \iff \TYPE{Metrizable}(A)
	}
	\Explain{1 $(\Rightarrow)$ assume that $(A,\mu)$ is $\sigma$-finite}
	\Explain{1.1 Then there is a countable partition of unity $a$ with finite elements}
	\Explain{1.2 define $\sigma : A^2 \to \Reals_{++}$ as 
		$\sigma(x,y) = \sum^\infty_{n=1} 2^{-n} \frac{\rho_{a_n}(x,y)}{\mu(a_n)}$}
	\Explain{1.3 Then $\sigma$ is a metic for $A$ }
	\Explain{1.4 So the topology of $(A,\mu)$ is metrizable}  
	\Explain{2 $(\Leftarrow)$ assume that $(A,\mu)$ is metrizable}
	\Explain{2.1 Let $\sigma$ be an metrizing metric}
	\ExplainFurther{2.2 Then there exists a system of elements 
		$k : \Nat \to \Nat, a : \prod^\infty_{n=1} \{1,\ldots,k_n\} \to A^f$ and 
		$\delta : \Nat \to \Reals_{++}$ }
	\Explain{ $\quad \quad$ such that  $\rho_{a_{n,i}}(b,e)$ 
		for any $1 \le i \le k_n$ imply that $\sigma(b,e) < 2^{-n}$ for any $b \in A$}
	\Explain{ 2.3 Then $e = \bigvee^\infty_{n=1} \bigvee^{k_n}_{i=1} a_{n,i}$}
	\Explain{ 2.4 So $(A,\mu)$ is $\sigma$-finite} 
    \EndProof
 }\Page{
    \Theorem{LocalizableIffComplete}
	{
		\NewLine ::		
		\forall (A,\mu) : \MA \.
		\Loc\MA(A,\mu) \iff \TYPE{T2}\And \TYPE{Complete}(A)
	}
	\Explain{ 1 $(\Rightarrow)$ Assume $(A,\mu)$ is localizable measure algebra}
	\Explain{ 1.2 Then $A$ is Hausdorff as $(A,\mu)$ is semifinite}
	\Explain{ 1.3  Assume $\F$ is a Cauchy filter in $A$}
	\Explain{ 1.4 Take $a \in A^f$ }
	\Explain{ 1.5 Then there is $d_a \le a$ and a cauchy sequence $c_a$ subordinate to $\F$ 
			such that 	$\lim_{n \to \infty} \rho_a(d_a,c_{a,n}) = 0$	
		}
	\Explain{ 1.5.1 select a sequence $F_a : \Nat \to \F$
		such that $\rho_a(x,y) \le 2^{-n}$ for $x,y \in F_{a,n}$ and $n \in \Nat$}
	\Explain{ 1.5.2 Then select a sequence $c_{a,n} \in \bigcap^n_{k=1} F_{a,k}$}
	\Explain{ 1.5.3 Then $\rho(c_{a,n},c_{a,n+1}) \le 2^{-n}$}
	\Explain{ 1.5.4 Construct $d_a = \liminf ac_a$}
	\Explain{ 1.5.5 Then $\lim_{n \to \infty} \rho_a(d_a,c_{an}) = 
		\lim_{n \to \infty} \mu( d_a + ac_a  ) = 0$}
	\Explain{ 1.6 Assume $a,b \in A^f$ are such that $a \le b$}
	\Explain{ 1.7 Then  $d_a  = ad_b$}
	\Explain{ 1.7.1  $F_{n,a} \cap F_{n,b} \neq \emptyset$}
	\Explain{ 1.7.2 So select $f \in F_{n,a} \cap F_{n,b}$}
	\ExplainFurther{ 1.7.3 Then 
		$
			\rho_a(d_a,d_b) \le 
				\rho_a(d_a,c_{a,n}) + 
				\rho_a(c_{a,n}, f) +
				\rho_a(f, c_{b,n}) +
				\rho_a(c_{b,n},d_b) \le$}
	\Explain{$\quad\quad
				\le \rho_a(d_a,c_{a,n}) + 
				 2^{-n} +
				2^{-n} +
				\rho_a(c_{b,n},d_b) \to 0
		$ as $n \to \infty$}
	\Explain{ 1.8 Let $f = \bigvee_{a \in A^f} d_a$}
	\Explain{ 1.9 Then $\lim \F = f$}
	\Explain{ 1.9.1  $ ad_a = af $ for any $a \in A^f$ }
	\Explain{ 1.9.2 and there is a $\F$ subordinate Cauchy sequence $c_a$
		such that $\rho_a(f,c_a) = \rho_a(d_a,c_a) \to 0$}
	\Explain{ 1.9.3 Then there is $n \in \Nat$ such that $\rho_a(d_a,c_{a,n}) + 2^{-n} < \varepsilon$}
	\Explain{ 1.9.4 Take any $g \in F_{a,n}$}	
	\Explain{ 1.9.5 But $\rho_a(f,g) \le \rho_a(f,c_{a,n}) + \rho_{c_{a,n}} \le 
			 \rho_a(d_a,c_{a,n}) + 2^{-n} < \varepsilon$}
	\Explain{ 1.9.6 This $F_{a,n} \subset \Cell_{\rho_a}(f,\varepsilon)$ }
	\Explain{ 2 $(\Leftarrow)$ now Assume that $A$ is Hausdorff and complete}
	\Explain{ 2.1 As $A$ is Hausdorff $(A,\mu)$ must be semifinite}
	\Explain{ 2.2 As $A$ is complete $(A,\mu)$ is order complete and hence localizable}
	\Explain{ 2.2.1 Think about order filters $\F(\uparrow D)$ and $\F(\downarrow D)$}
	\EndProof
}
\newpage
\subsubsection{Closed Subalgebras}
\Page{
	\Theorem{ClosedSubalgebraTHM}
	{
		\NewLine ::		
		\forall (A,\mu) : \Loc\MA \.
		\forall B \subset_{\mathsf{RING}} A \.
		\Closed(A,B) \iff \OC(A,B) 
	}
	\Explain{ 1 $(\Rightarrow)$ follows from the general theory}
	\Explain{ 2 $(\Leftarrow)$ Assume now that $B$ is order-closed}
	\Explain{ 2.1 Assume $g \in {\cl}_A B$}
	\Explain{ 2.2 Assume $a \in A^f$ and $n \in \Nat$} 
	\Explain{ 2.3 Then there exists a sequence $c_a : \Nat \to B$ 
		such that $\rho_a(c_{a,n},g) < 2^{-n}$}
	\Explain{ 2.4 Note, $\sum^\infty_{n=1} \mu(ac_{a,n} + ac_{a,n + 1}) 
		\le \sum^\infty_{n=1} \mu(ac_{a,n} +ag) + \mu(ag + ac_{a,n + 1}) < 
		 \sum^\infty_{n=1} 2^{-n} + 2^{-n-1} = \frac{3}{2} $ }
	\Explain{ 2.5 So, sequence $ac_a$ has summable increments  }
	\Explain{ 2.6 Define $d_a = \liminf c_{a}$}
	\Explain{ 2.7 As $ac_a$ has finite increments $\lim_{n \to \infty} \rho_a(c_{a,n},d_n) = 0$}
	\Explain{ 2.8 Furthermore, $\rho_a(d_a,g) = 0$, so $ag = d_a$}
	\Explain{ 2.9 As $B$ is order-closed $d_a \in B$ for each $a \in A^f$}
	\Explain{ 2.10 Set $d'_a = \inf \{ d_b : b \in A^f, a \le b  \} \in B$}
	\Explain{ 2.11  $d'_a a  = \bigwedge_{a \le b} d_b a = \bigwedge_{a \le b} d_b b a  = 
		  \bigwedge_{a \le b}  g b a = g a$}
	\Explain{ 2.12 Let $D = \{ d_a' | a \in A \} $}
	\Explain{ 2.13 Clearly $D$ is upwards directed as $d'_a \vee d'_b = d'_{a \wedge b}$}
	\Explain{ 2.14 Then  $\sup D = \{ ad_a' | a \in A \} = \{ ag | a \in A \} = g$
		as $(A,\mu)$ is semifinite}
	\Explain{ 2.15 so $g \in B$ as $B$ is order-closed}
	\Explain{ 2.16 Thus $B$ is closed}
	\EndProof
	\\
	\Theorem{SubalgebraClosure}
	{
		\forall (A,\mu) : \Loc\MA \.
		\forall B \subset_{\mathsf{RING}} A \.
		\overline{B} =  \tau(B)
	}
	\Explain{ 1 Note that $\overline{B}$ is a subgroup of $A$}
	\Explain{ 2 Also it must be order-closed as $\overline{B}$ is closed}
	\Explain{ 3 Also $\tau(B)$ is an order-closed subalgebra, and hence a closed subalgebra}
	\ExplainFurther{ 4 So both objects can be realized as intersections 
		of closed subalgebras containing $B$,}
	\Explain{ $\quad\quad$	and hence they are equal}
	\EndProof      
	\\
	\DeclareType{ClosedMeasureSubalgebra}{\prod (A,\mu) : \MA \. \TYPE{Subalgebra}(A)}
	\DefineNamedType{B}{ClosedMeasureSubalgebra}{B \subset_{\ma} A}
	{\Closed(A,B)}
}\Page{
	\Theorem{OrderClosedExtension}
	{
		\NewLine ::		
		\forall (A,\mu) : \Loc\MA \.
		\forall B \subset_{\ma} A \.
		\forall a \in A \. 
		\langle B \cup \{a\} \rangle_{\BOOL} \subset_{\ma} A
	}
	\Explain{ This follows from order-closed subalgebra extension theorem for boolean algebras}
	\EndProof
}
\newpage
\subsubsection{Metric Space of Finite Elements}
\Page{
	\Theorem{BooleanOperationsAreUniformlyContinuous}
	{
		\NewLine ::		
		\forall (A,\mu) : \MA \.  (*),(\setminus),(\vee),(\wedge) \in \UNI(A^f \times A^f, A^f)
	}
	\Explain{ This is obvious}
	\EndProof
	\\
	\Theorem{MeasureIs1Lip}
	{
		\NewLine ::		
		\forall (A,\mu) : \MA \.  \mu_{|A^f} \in \Lip{1}(A^f) 
	}
	\Explain{ This is obvious}
	\EndProof
	\\
	\Theorem{FiniteElementsAreComplete}
	{
		\NewLine ::		
		\forall (A,\mu) : \MA \.  \Complete(A^f) 
	}
	\Explain{ 1 Assume $a$ is a cauchy sequence in $A^f$}
	\Explain{ 2 without loss of generality we may assume that $a$ has summable differences }
	\Explain{ 2.1 Just select a subsequence}
	\Explain{ 3 Define $x = \lim \inf a \in A$}
	\Explain{ 4 Then $\lim_{n \to \infty} a_n = x $}
	\Explain{ 5 So, there is some $n \in \Nat$ such that  $\mu(x \setminus a_n) < \infty$ }
	\Explain{ 6 Thus $\mu(x) < \infty$ and $x \in A^f$ }
	\EndProof
}
\newpage
\subsubsection{Relation with Convergence In Measure}
\subsection{Category}
\subsection{Radon-Nikodym Parallels}
\section{Maharam's Theory}
\section{Abstract Ergodic Theory}
\section{Measurable Algebras}
\newpage
\section*{Sources:}
\begin{enumerate}
\item  D. H. Fremlin --- Measure Theory (32,33,34) 2016
\end{enumerate}
\end{document}

