\documentclass[12pt]{scrartcl}
\usepackage[T2A]{fontenc}
\usepackage[utf8]{inputenc}
\usepackage{mathtools}
\usepackage{amsmath}
\usepackage{amsfonts}
\usepackage{hyperref}
\usepackage{amssymb}
\usepackage{ wasysym }
\usepackage{accents}
\usepackage{extpfeil}
\usepackage{graphicx}
\usepackage{scalerel}
\usepackage[dvipsnames]{xcolor}
\usepackage[a4paper,top=5mm, bottom=5mm, left=10mm, right=2mm]{geometry}
%Markup
\newcommand{\TYPE}[1]{\textcolor{NavyBlue}{\mathtt{#1}}}
\newcommand{\FUNC}[1]{\textcolor{Cerulean}{\mathtt{#1}}}
\newcommand{\LOGIC}[1]{\textcolor{Blue}{\mathtt{#1}}}
\newcommand{\THM}[1]{\textcolor{Maroon}{\mathtt{#1}}}
%META
\renewcommand{\.}{\; . \;}
\newcommand{\de}{: \kern 0.1pc =}
\newcommand{\where}{\LOGIC{where}}
\newcommand{\If}{\LOGIC{if} \;}
\newcommand{\Then}{ \; \LOGIC{then} \;}
\newcommand{\Else}{\; \LOGIC{else} \;}
\newcommand{\Act}[1]{\left( #1 \right)}
\newcommand{\Theorem}[2]{& \THM{#1} \, :: \, #2 \\ & \Proof = \\ } 
\newcommand{\DeclareType}[2]{& \TYPE{#1} \, :: \, #2 \\} 
\newcommand{\DefineType}[3]{& #1 : \TYPE{#2} \iff #3 \\} 
\newcommand{\DefineNamedType}[4]{& #1 : \TYPE{#2} \iff #3 \iff #4 \\} 
\newcommand{\DeclareFunc}[2]{& \FUNC{#1} \, :: \, #2 \\}  
\newcommand{\DefineFunc}[3]{&  \FUNC{#1}\Act{#2} \de #3 \\} 
\newcommand{\DefineNamedFunc}[4]{&  \FUNC{#1}\Act{#2} = #3 \de #4 \\} 
\newcommand{\NewLine}{\\ & \kern 1pc}
\newcommand{\Page}[1]{ \begin{align*} #1 \end{align*}   }
\newcommand{ \bd }{ \ByDef }
\newcommand{\NoProof}{ & \ldots \\ \EndProof}
\newcommand{\Explain}[1]{& \text{#1.} \\}
\newcommand{\ExplainFurther}[1]{& \text{#1} \\}
\newcommand{\Exclaim}[1]{& \text{#1!} \\}
%LOGIC
\renewcommand{\And}{\; \& \;}
\newcommand{\ForEach}[3]{\forall #1 : #2 \. #3 }
\newcommand{\Exist}[2]{\exists #1 : #2}
\newcommand{\Imply}{\Rightarrow} 
%TYPE THEORY
\newcommand{\DFunc}[3]{\prod #1 : #2 \. #3 }
\newcommand{\DPair}[3]{\sum #1 : #2 \. #3}
\newcommand{\Type}{\TYPE{Type}}
\newcommand{\Class}{\TYPE{Kind}}
%%STD
\newcommand{\Int}{\mathbb{Z} }
\newcommand{\NNInt}{\mathbb{Z}_{+} }
\newcommand{\Reals}{\mathbb{R} }
\newcommand{\Complex}{\mathbb{C}}
\newcommand{\Rats}{\mathbb{Q} }
\newcommand{\Nat}{\mathbb{N} }
\newcommand{\EReals}{\stackrel{\mathclap{\infty}}{\mathbb{R}}}
\newcommand{\ERealsn}[1]{\stackrel{\mathclap{\infty}}{\mathbb{R}}^{#1}}
\DeclareMathOperator*{\centr}{center}
\DeclareMathOperator*{\argmin}{arg\,min}
\renewcommand{\i}{\mathrm{i}}
%%Set Theory
\DeclareMathOperator*{\id}{id}
\DeclareMathOperator*{\im}{Im}
\DeclareMathOperator*{\supp}{supp}
\newcommand{\Eq}{\TYPE{Equivalence}}
\newcommand{\EqClass}[1]{\TYPE{EquivalenceClass}\left( #1 \right)}
\newcommand{\End}{\mathrm{End}}
\newcommand{\Aut}{\mathrm{Aut}}
\newcommand{\Func}[2]{\TYPE{Functor}\left( #1, #2 \right)}
\mathchardef\hyph="2D
\newcommand{\Surj}{\TYPE{Surjective}}
\newcommand{\Inj}{\TYPE{Injective}}
\newcommand{\ToInj}{\hookrightarrow}
\newcommand{\ToMono}{\xhookrightarrow}
\newcommand{\ToSurj}{\twoheadrightarrow}
\newcommand{\ToEpi}{\xtwoheadrightarrow}
\newcommand{\ToBij}{\leftrightarrow}
\newcommand{\ToIso}{\xleftrightarrow}
\newcommand{\Arrow}{\xrightarrow}
\newcommand{\Set}{\TYPE{Set}}
\newcommand{\Ideal}{\TYPE{Ideal}}
\newcommand{\du}{\; \triangle \;}
\newcommand{\Finites}{\TYPE{FiniteSubset}}
\renewcommand{\c}{\complement}
\newcommand{\Cover}{\TYPE{Cover}}
\newcommand{\Ultrafilter}{\TYPE{Ultrafilter}}
\newcommand{\Finite}{\TYPE{Finite}}
\newcommand{\SA}{\TYPE{\sigma \hyph Algebra}}
%%ProofWritting
\newcommand{\Say}[3]{& #1 \de #2 : #3, \\}
\newcommand{\SayIn}[3]{& #1 \de #2 \in #3, \\}
\newcommand{\Conclude}[3]{& #1 \de #2 : #3; \\}
\newcommand{\ConcludeIn}[3]{& #1 \de #2 \in #3; \\}
\newcommand{\Derive}[3]{& \leadsto #1 \de #2 : #3, \\}
\newcommand{\DeriveIn}[3]{& \leadsto #1 \de #2 \in #3, \\}
\newcommand{\DeriveConclude}[3]{& \leadsto #1 \de #2 : #3 ; \\} 
\newcommand{\DeriveConcludeIn}[3]{& \leadsto #1 \de #2 \in #3 ; \\} 
\newcommand{\Assume}[2]{& \LOGIC{Assume} \; #1 : #2, \\}
\newcommand{\AssumeIn}[2]{& \LOGIC{Assume} \; #1 \in #2, \\}
\newcommand{\As}{\; \LOGIC{as } \;} 
\newcommand{\Intro}{\LOGIC{I}}
\newcommand{\Elim}{\LOGIC{E}}
\newcommand{\QED}{\; \square}
\newcommand{\EndProof}{& \QED \\}
\newcommand{\ByDef}{\eth} 
\newcommand{\ByConstr}{\jmath}  
\newcommand{\Alt}{\LOGIC{Alternative} \;}
\newcommand{\CL}{\LOGIC{Close} \;}
\newcommand{\More}{\LOGIC{Another} \;}
\newcommand{\Proof}{\LOGIC{Proof} \; }
%CategoryTheory
%Types
\newcommand{\Cov}{\TYPE{Covariant}}
\newcommand{\Contra}{\TYPE{Contravariant}}
\newcommand{\NT}{\TYPE{NaturalTransform}}
\newcommand{\UMP}{\TYPE{UnversalMappingProperty}}
\newcommand{\CMP}{\TYPE{CouniversalMappingProperty}}
\newcommand{\paral}{\rightrightarrows}
%functions
\newcommand{\op}{\mathrm{op}}
\newcommand{\obj}{\mathrm{obj}}
\DeclareMathOperator*{\dom}{dom}
\DeclareMathOperator*{\codom}{codom}
\DeclareMathOperator*{\colim}{colim}
%variable
\renewcommand{\C}{\mathcal{C}}
\newcommand{\A}{\mathcal{A}}
\newcommand{\B}{\mathcal{B}}
\newcommand{\D}{\mathcal{D}}
\newcommand{\I}{\mathcal{I}}
\newcommand{\J}{\mathcal{J}}
\newcommand{\R}{\mathrm{R}}
%Cats
\newcommand{\CAT}{\mathsf{CAT}}
\newcommand{\SET}{\mathsf{SET}}
\newcommand{\PARALLEL}{\bullet \paral \bullet}
\newcommand{\WEDGE}{\bullet \to \bullet \leftarrow \bullet}
\newcommand{\VEE}{\bullet \leftarrow \bullet \to \bullet}
%Topology
%General Topology
%Types
\newcommand{\Top}{\TYPE{Topology}}
\newcommand{\Homeo}{\TYPE{Homeomorphism}}
\newcommand{\TS}{\TYPE{TopologicalSpace}} 
\newcommand{\NbhdBase}{\TYPE{NeighborhoodBase}}
\newcommand{\LF}{\TYPE{LocallyFinite}}
\newcommand{\PN}{\TYPE{PerfectlyNormal}}
\newcommand{\CR}{\TYPE{CompletelyRegular}}
\newcommand{\OM}{\TYPE{OpenMap}}
\newcommand{\Filter}{\TYPE{Filter}}
\newcommand{\Filterbase}{\TYPE{Filterbase}}
\newcommand{\CFilterbase}{\TYPE{ConvergentFilterbase}}
\newcommand{\Dense}{\TYPE{Dense}}
\newcommand{\Separable}{\TYPE{Separable}}
\newcommand{\ND}{\TYPE{NowhereDense}}
\newcommand{\Open}{\TYPE{Open}}
\newcommand{\Net}{\TYPE{Net}}
\newcommand{\Closed}{\TYPE{Closed}}
\newcommand{\Clopen}{\TYPE{Clopen}}
\newcommand{\Nbhd}{\TYPE{Neighborhood}}
\newcommand{\Compact}{\TYPE{Compact}}
\newcommand{\Compacts}{\TYPE{CompactSubset}}
\newcommand{\OpenC}{\TYPE{OpenCover}}
\newcommand{\Cluster}{\TYPE{Cluster}}
\newcommand{\Convergent}{\TYPE{Convergent}}
%\newcommand{\LC}{\TYPE{LocallyCompact}}
\newcommand{\Locally}{\TYPE{Locally}}
\newcommand{\Bair}{\TYPE{BaireSpace}}
\newcommand{\Meager}{\TYPE{Meager}}
\newcommand{\Connected}{\TYPE{Connected}}
%FUNC
\DeclareMathOperator*{\intx}{int}
\DeclareMathOperator*{\cl}{cl} 
\DeclareMathOperator*{\boundary}{\partial} 
\DeclareMathOperator{\combo}{\triangledown} 
%\DeclareMathOperator{\diag}{\triangle} 
\DeclareMathOperator{\rem}{rem}
%CATS
\newcommand{\TOP}{\mathsf{TOP}}
\newcommand{\HC}{\mathsf{HC}}
\newcommand{\CG}{\mathsf{CG}}
%Symbols
\newcommand{\T}{\mathcal{T}}
\newcommand{\N}{\mathcal{N}}
\renewcommand{\U}{\mathcal{U}}
\renewcommand{\O}{\mathcal{O}}
\renewcommand{\d}{\mathrm{d}}
\newcommand{\F}{\mathcal{F}}
\newcommand{\X}{\mathcal{X}}
%\newcommand{\d}{\mathrm{d}}
%Metic Topology
%FUNC
\DeclareMathOperator{\diam}{diam}
\newcommand{\Cell}{\mathbb{B}}
\newcommand{\Disc}{\mathbb{D}}
%CATS
\newcommand{\Semiiso}{\mathsf{SMS}_{\circ \to \cdot}}
\newcommand{\Iso}{{\mathsf{MS}_{\circ \to \cdot}}}
\newcommand{\SMS}{\mathsf{SMS}}
\newcommand{\MS}{\mathsf{MS}}
\newcommand{\UNI}{\mathsf{UNI}}
\newcommand{\UNIS}{\mathsf{UNIS}}
\newcommand{\TG}{\mathsf{TG}}
\newcommand{\CSeq}{\TYPE{CauchySequence}}
\newcommand{\Complete}{\TYPE{Complete}}
%Descriptive Set Theory
%TYPE
%Descriptive Set Theory
%TYPE
%\newcommand{\Bool}{\mathbb{B}}
%\newcommand{\IS}{\TYPE{InitialSegement}}
\newcommand{\FS}[1]{{#1}{}^*}
\newcommand{\Ext}{\TYPE{Extension}}
\newcommand{\Tree}{\TYPE{Tree}}
\newcommand{\Pruned}{\TYPE{Pruned}}
\newcommand{\PTM}{\TYPE{ProperTreeMorphism}}
%\newcommand{\LTM}{\TYPE{LipschitzTreeMorphism}}
\newcommand{\Polish}{\TYPE{Polish}}
\newcommand{\IIPG}{\TYPE{InfiniteIterativeTwoPlayersGame}}
\newcommand{\FPS}{\TYPE{FirstPlayerStrategy}}
\newcommand{\SPS}{\TYPE{SecondPlayerStrategy}}
\newcommand{\FPWS}{\TYPE{FirstPlayerWinningStrategy}}
\newcommand{\SPWS}{\TYPE{SecondPlayerWinningStrategy}}
\newcommand{\CS}{\TYPE{ChoquetSpace}}
\newcommand{\SCS}{\TYPE{StrongChoquetSpace}}
\newcommand{\BP}{\mathbf{BP}}
\newcommand{\MGR}{\mathbf{MGR}}
\newcommand{\cat}{\mathbf{CAT}}
\newcommand{\BM}{\TYPE{BairMeasurable}}
\newcommand{\CGSA}{\TYPE{CountablyGeneratedSigmaAlgebra}}
\newcommand{\MC}{\TYPE{MonotonicClass}}
\newcommand{\PSA}{\TYPE{PointSeparatingAlgebra}}
\newcommand{\SBS}{\TYPE{StandardBorelSpace}}
%FUNC
\DeclareMathOperator{\len}{len}
\newcommand{\inits}[2]{{#1}_{|\left[1,\ldots,#2\right]}}
\DeclareMathOperator{\lb}{lb}
\DeclareMathOperator{\WFpart}{WF}
\DeclareMathOperator{\Tr}{Tr}
\DeclareMathOperator{\PTr}{PTr}
\DeclareMathOperator*{\Tll}{{T\;\underline{lim}}}
\DeclareMathOperator*{\Tul}{{T\;\overline{lim}}}
\DeclareMathOperator*{\Tl}{{T\;lim}}
\DeclareMathOperator{\rankcb}{rank_{CB}}
\DeclareMathOperator{\lp}{lp}
\newcommand{\alg}{\mathsf{A}}
%CATS
\newcommand{\TREE}{\mathsf{TREE}}
\newcommand{\FSF}{\mathsf{FS}}
\newcommand{\CRONE}{\mathsf{CRONE}}
\newcommand{\BODY}{\mathsf{BODY}}
\newcommand{\BOR}{\mathsf{BOR}}
\newcommand{\bor}{\mathsf{B}}
\newcommand{\Effros}{\mathsf{EFF}}
%symbols
\newcommand{\K}{\mathsf{K}}
\renewcommand{\H}{\mathrm{H}}
\renewcommand{\L}{\mathcal{L}}
\renewcommand{\P}{\mathcal{P}}
\renewcommand{\S}{\mathcal{S}}
%Algebra
%Groups
%Types
\newcommand{\Group}{\TYPE{Group}}
\newcommand{\Abel}{\TYPE{Abelean}}
\newcommand{\Sgrp}{\subset_{\mathsf{GRP}}}
\newcommand{\Nrml}{\vartriangleleft}
\newcommand{\FG}{\TYPE{FiniteGroup}}
\newcommand{\Stab}{\mathrm{Stab}}
\newcommand{\FGA}{\TYPE{FinitelyGeneratedAbelean}}
\newcommand{\DN}{\TYPE{DirectedNormality}}
\newcommand{\Sphere}{\mathbb{S}}
\newcommand{\Torus}{\mathbb{T}}
%Func
\newcommand{\ActOn}{\curvearrowright}
\DeclareMathOperator{\tor}{tor}
\DeclareMathOperator{\bool}{bool}
\DeclareMathOperator{\rank}{rank}
%Cats
\newcommand{\GRP}{\mathsf{GRP}}
\newcommand{\ABEL}{\mathsf{ABEL}}
%LINEAR
%LINEAR
%Linear Algebra
%Types
\newcommand{\Basis}{\TYPE{Basis}} % Basis of the linear space
\newcommand{\submod}[1]{\subset_{\LMOD{#1}}}% submodule as a subset
\newcommand{\subvec}[1]{\subset_{\VS{#1}}}% vector subspace as a subset
\newcommand{\FGM}{\TYPE{FinitelyGeneratedModule}}% Finitely generated module
\newcommand{\LI}{\TYPE{LinearlyIndependent}}
\newcommand{\LIS}{\TYPE{LinearlyIndependentSet}}
\newcommand{\FM}{\TYPE{FreeModule}}
\newcommand{\IBP}{\TYPE{InvariantBasisProperty}}
\newcommand{\UTM}{\TYPE{UpperTriangularMatrix}}
\newcommand{\LTM}{\TYPE{LowerTriangularMatrix}}
\newcommand{\Diag}{\TYPE{DiagonalMatrix}}
\newcommand{\FP }{\TYPE{FinitelyPresented}}
\newcommand{\GL}{\mathbf{GL}}% General Linear Group
\newcommand{\SL}{\mathbf{SL}}% Special Linear group
\newcommand{\SO}{\mathbf{SO}}
\newcommand{\SU}{\mathbf{SU}}
\newcommand{\prsubvec}[1]{\subsetneq_{\VS{#1}}}	% poper vector subspace as a subset
\newcommand{\LC}{\TYPE{LinearComplement}} 
\newcommand{\IS}{\TYPE{InvariantSubspace}}
\newcommand{\RP}{\TYPE{ReducingPair}}
\newcommand{\RCF}{\TYPE{RationalCanonicalForm}}
\newcommand{\JCF}{\TYPE{JordanCanonicalForm}}
\newcommand{\Diagble}{\TYPE{Diagonalizable}}
\newcommand{\UT}{\TYPE{UpperTriangulizable}}
\newcommand{\LT}{\TYPE{LowerTriangulizable}}
%\newcommand{\IPS}{\TYPE{InnerProductSpace}}
\newcommand{\OBasis}{\TYPE{OrthonormalBasis}}
\newcommand{\FDIPS}{\TYPE{FiniteDimensionalInnerProductSpace}}
\newcommand{\NO}{\TYPE{NormalOperator}}
\newcommand{\NM}{\TYPE{NormalMatrix}}
%\newcommand{\SA}{\TYPE{SelfAdjoint}}
%\newcommand{\SSA}{\TYPE{SkewSelfAdjoint}}
%\newcommand{\PI}{\TYPE{Pseudoinverse}}
%\newcommand{\OVS}{\TYPE{OrthogonalVectorSpace}}
%\newcommand{\SVS}{\TYPE{SymplecticVectorSpace}}
%\newcommand{\MVS}{\TYPE{MetricVectorSpace}}
%\newcommand{\FDMVS}{\TYPE{FiniteDimensionalMetricVectorSpace}}
%\newcommand{\Sp}{\mathbf{Sp}}
%Func
\DeclareMathOperator{\Span}{span} % spann by subset
\DeclareMathOperator{\Ann}{Ann}   % annihilator
\DeclareMathOperator{\Ass}{Ass}   % associated primes
\DeclareMathOperator{\diag}{diag} % diagonal
\DeclareMathOperator{\adj}{adj}   % an adjoint matrix
\DeclareMathOperator{\tr}{tr}     % trace
\DeclareMathOperator{\codim}{codim} % codimension
%\DeclareMathOperator{\Cell}{\mathbf{C}} % a componion matrix
\DeclareMathOperator{\JC}{\mathbf{J}}  % a Jordan cell
\DeclareMathOperator{\bigboxplus}{\scalerel*{\boxplus}{\sum}} % a direct sum of operators in the sence of the reducing a pair
\DeclareMathOperator{\Spec}{Spec} % Spectre
\DeclareMathOperator{\bigbot}{\scalerel*{\bot}{\sum}} % an othogonal direct sum
\DeclareMathOperator{\GS}{\mathbf{GS}} %Gramm-Smmidt process
\DeclareMathOperator{\NGS}{\mathbf{NGS}} %Normalized Gramm-Smmidt process
\DeclareMathOperator{\WI}{\mathrm{WI}} %Witt Index
%Cats
\newcommand{\VS}[1]{#1\hyph\mathsf{VS}} % a category of vector spaces (Field)
\newcommand{\FDVS}[1]{#1\hyph\mathsf{FDVS}} % a category of finite-dimensional vector spaces (Field)
\newcommand{\LMOD}[1]{#1\hyph\mathsf{MOD}} % a category of the left modules (Ring)
\newcommand{\RMOD}[1]{\mathsf{MOD}\hyph#1} % a category of the right modules (Ring)
\newcommand{\LLMAP}[1]{#1\hyph\mathsf{LMAP}} % a cagory of based linear maps with the left scalar multiplication (Ring)
\newcommand{\LMAT}[1]{#1\hyph\mathsf{MAT}}  % a category of based matrices with the left scalar multiplication (Ring)
\newcommand{\NMAT}[1]{#1\hyph\mathbb{N}} % a category of finite matrices (Field)
%Symbols
\renewcommand{\L}{\mathcal{L}}
\renewcommand{\O}{\mathbf{O}}
%\newcommand{\U}{\mathbf{U}}
\renewcommand{\S}{\mathbf{S}}
%FIELDS
\newcommand{\Field}{\TYPE{Field}}
\newcommand{\ACF}{\TYPE{AlgebraicallyClosedField}}
%Functional Analysis
%TYPES
\newcommand{\PLF}{\TYPE{PositiveLinearFunctional}}
\newcommand{\NS}{\TYPE{NormedSpace}}
\newcommand{\SNS}{\TYPE{SeminormedSpace}}
\newcommand{\Banach}{\TYPE{Banach}}
\newcommand{\IPS}{\TYPE{InnerProductSpace}}
\newcommand{\Hilbert}{\TYPE{Hilbert}}
\newcommand{\TopC}{\TYPE{TopologicalyCompletable}}
\newcommand{\AbsC}{\TYPE{AbsolutelyConvergent}}
\newcommand{\PNS}{\TYPE{PolynormedSpace}}
\newcommand{\CNS}{\TYPE{CountablyNormedSpace}}
%CATS
\newcommand{\NORM}{\textsf{NORM}}
\newcommand{\NORMI}{{\textsf{NORM}_{\circ \to \cdot}}}
\newcommand{\BAN}{\textsf{BAN}}
\newcommand{\BANI}{{\textsf{BAN}_{\circ \to \cdot}}}
\newcommand{\HIL}[1]{#1\hyph\textsf{HIL}}
\newcommand{\HILI}{{\textsf{HIL}_{\circ \to \cdot}}}
%Simbols
\newcommand{\w}{\mathbf{w}}
%Convex
%
\newcommand{\Convex}{\TYPE{Convex}}
\newcommand{\CB}{\TYPE{ConvexBody}}
\newcommand{\CC}{\TYPE{ConvexCone}}
\newcommand{\Mink}{\TYPE{MinkowskySpace}}
\newcommand{\Euc}{\TYPE{EucledeanSpace}}
\newcommand{\Cone}{\TYPE{Cone}}
\newcommand{\Wedge}{\TYPE{Wedge}}
\newcommand{\PVS}{\TYPE{PreorderedVectorSpace}}
\newcommand{\OVS}{\TYPE{OrderedVectorSpace}}
\newcommand{\AVS}{\TYPE{ArchemedeanVectorSpace}}
\newcommand{\RS}{\TYPE{RieszSpace}}
%FUNC
\newcommand{\rint}{{\mathrm{rel}  \intx}}
\DeclareMathOperator{\lina}{lina}
\DeclareMathOperator{\lin}{lin}
\DeclareMathOperator{\core}{core}
\DeclareMathOperator{\conv}{conv}
\DeclareMathOperator{\cone}{cone}
\DeclareMathOperator{\cconv}{\overline{conv}}
\DeclareMathOperator{\CCONV}{\mathsf{CCONV}}
\renewcommand{\H}{\mathrm{H}}
%TOPALG
%\TYPES
\newcommand{\Connector}{\TYPE{Connector}}
\newcommand{\CConnector}{\TYPE{ClosedConnector}}
\newcommand{\Unif}{\TYPE{Uniformity}}
\newcommand{\US}{\TYPE{UniformSpace}}
\newcommand{\UB}{\TYPE{UniformBase}}
\newcommand{\Sym}{\TYPE{SymmetricConnector}}
\newcommand{\SB}{\TYPE{SymmetricBase}}
\newcommand{\BofU}{\TYPE{BaseOfUniformity}}
\newcommand{\UNbhd}{\TYPE{UniformNeighborhood}}
\newcommand{\UC}{\TYPE{UniformlyContinuous}}
\newcommand{\UHomeo}{\TYPE{Unimorphism}}
\newcommand{\UniCov}{\TYPE{UniformCover}}
\newcommand{\CF}{\TYPE{CauchyFilterbase}}
\newcommand{\CUS}{\TYPE{CompleteUniformSpace}}
\newcommand{\SCUS}{\TYPE{SequenceCompleteUniformSpace}}
\newcommand{\Small}{\TYPE{Small}}
\newcommand{\TB}{\TYPE{TotallyBounded}}
\newcommand{\MUS}{\TYPE{MetrizableUniformSpace}}
\newcommand{\Completion}{\TYPE{Completion}}
\newcommand{\SCompletion}{\TYPE{SeparableCompletion}}
\newcommand{\pt}{\mathrm{pt}}
\newcommand{\EqC}{\TYPE{Equicontinuous}}
\newcommand{\UEqC}{\TYPE{UnifomlyEquicontinuous}}
\newcommand{\LIM}{\TYPE{LeftInvariantMetric}}
\newcommand{\RIM}{\TYPE{RightInvariantMetric}}
\newcommand{\TIM}{\TYPE{TwosidedInvariantMetric}}
\renewcommand{\SS}{\TYPE{SymmetricSet}}
\newcommand{\veemetric}{\TYPE{\vee\hyph Semimetric}}
\newcommand{\TGC}{\TYPE{TopologicalGroupCompletion}}
\newcommand{\BG}{\TYPE{BaireGroup}}
\newcommand{\PG}{\TYPE{PolishGroup}}
\newcommand{\Borg}{\TYPE{BorelGroup}}
\newcommand{\cli}{\TYPE{cli}}
\newcommand{\Selector}{\TYPE{Selector}}
\newcommand{\Transversal}{\TYPE{Transversal}}
\newcommand{\SBG}{\TYPE{StandardBorelGroup}}
\newcommand{\Polishable}{\TYPE{Polishable}}
\newcommand{\SIN}{\TYPE{SIN}}
%FUNC
\newcommand{\inv}{\mathrm{inv}}
\DeclareMathOperator{\perm}{\mathrm{perm}}
%\CAT
\newcommand{\TGRP}{\mathsf{TGRP}}
\newcommand{\PGRP}{\mathsf{PGRP}}
%\Symbol
\newcommand{\V}{\mathcal{V}}
\newcommand{\W}{\mathcal{W}}
\renewcommand{\L}{\mathcal{L}}
\renewcommand{\R}{\mathcal{R}}
\renewcommand{\S}{\mathcal{S}}
\newcommand{\OComplete}{\TYPE{OrderDedekindComplete}}
%TopologicalVectorSpaces
%TYPES
\newcommand{\TopField}{\TYPE{TopologicalField}}
\newcommand{\TopVS}{\TYPE{TopologicalVectorSpace}}
\newcommand{\Vect}{\TYPE{Vector}}
\newcommand{\AVF}{\TYPE{AbsoluteValueField}}
\newcommand{\LConv}{\TYPE{LocallyConvexSpace}}
\newcommand{\TC}{\TYPE{TopologicalComplement}}
\newcommand{\MaxS}{\TYPE{MaximalSubspace}}
\newcommand{\AKC}{\TYPE{AbsolutelyKConvex}}
\newcommand{\KC}{\TYPE{KConvex}}
\newcommand{\LKConv}{\TYPE{LocallyKConvexSpace}}
\newcommand{\CCompact}{\TYPE{CCompact}}
%Cats
\newcommand{\TVS}[1]{{#1}\hyph\mathsf{TVS}}
\newcommand{\HTVS}[1]{{#1}\hyph\mathsf{HTVS}}
\newcommand{\LCS}[1]{{#1}\hyph\mathsf{LCS}}
\newcommand{\LCHS}[1]{{#1}\hyph\mathsf{LCHS}}
%FUNC
\DeclareMathOperator{\bal}{\mathrm{bal}}
\newcommand{\kconv}{K\hyph\mathrm{conv}\;}
\title{Topological Vector Spaces 2}
\author{Uncultured Tramp}
\begin{document}
\maketitle
\newpage
\tableofcontents
\newpage
\section{Abstract Topological Vector Spaces}
\subsection{Locally Convex Spaces}
\subsubsection{Intro and Definition}
\Page{
	\DeclareType{\TopVS}{\prod k : \TopField \. ? \sum_{V \in \VS{k}} \Top(V)}
	\DefineType{(V,\tau)}{\TopVS}{\
		\cdot_V \in \TOP\Big( k \times (V,\tau),(V,\tau)\Big) \And 
		+_V \in \TOP\Big( (V,\tau) \times (V,\tau),(V,\tau)\Big)
	}
	\\
	& k :: \TopField ;\\
	\\
	\Conclude{\Vect\Top}{\Lambda V \in \VS{k} \. \TopVS(V)}
	{\prod_{V \in \VS{k}} V \. ? \Top(V)} 
	\\
	\DeclareFunc{categoryOfTopologicalVectorSpaces}
	{
		\TopField \to \CAT
	}
	\DefineNamedFunc{categoryOfTopologicalVectorSpaces}{k}{\TVS{k}}
	{
		\NewLine \de		
		(\TopVS(k),\VS{k} \cap \TOP,\circ,\id)
	}
	\\
	\DeclareFunc{categoryOfTopologicalVectorSpaces}
	{
		\TopField \to \CAT
	}
	\DefineNamedFunc{categoryOfHausdorffTopologicalVectorSpaces}{k}{\HTVS{k}}
	{
		\NewLine \de		
		(\TopVS(k) \And \TYPE{T2},\VS{k} \cap \TOP,\circ,\id)
	}
	\\
	\DeclareFunc{asTopologicalGroup}
	{
		\TVS{k} \to \TGRP
	}
	\DefineNamedFunc{asTopologicalGroup}{V}{V}{V}
	\\
	\\
	\DeclareFunc{asVectorSpace}
	{
		\TVS{k} \to \VS{k}
	}
	\DefineNamedFunc{asVectorSpace}{V}{V}{V}
	\\
}
\newpage
\subsubsection{Absorbent and Balanced Sets}
\Page{
	& k :: \AVF(\Reals) ;\\
	\\
	\DeclareType{Balanced}{ \prod_{V : \TVS{k} } ??V }
	\DefineType{B}{Balanced}{\Disc_k(0,1)B \subset B}
	\\
	\DeclareType{Absorbent}{\prod k : \AVF(\Reals) \. \prod_{V : \TVS{k} } ??V }
	\DefineType{A}{Absorbent}{
		\forall v \in V \. 
		\exists \rho \in \Reals_{++} \. 
		\forall \alpha \in \Disc_k(0,\rho) \.
		\alpha v \in A
	}
	\\
	\Theorem{VectorSubspaceIsBalanced}
	{
		\forall V \in \TVS{k} \.
		\forall U \subset_{\VS{k}} V \.
		\TYPE{Balanced}(V,U)
	}
	\Explain{ Obvious}
	\EndProof
	\\
	\Theorem{AbsorbentVectorSubspaceIswhole}
	{
		\forall V \in \TVS{k} \.
		\forall U \subset_{\VS{k}} V \.
		\TYPE{Absorbent}(V,U) \Imply V
	}
	\Explain{ Take $v \in V$}
	\Explain{ By definition of absorbent there is $\alpha \in k_*$ such that $\alpha v \in U$}
	\Explain{ But then  $ v = \alpha^{-1} \alpha v \in U$}
	\Explain{ So, $U = V$}
	\EndProof
	\\
	\Theorem{BalancedSetsAreDedikindComplete}
	{
		\forall V \in \TVS{k} \. \OComplete\Big( \TYPE{Balanced}(V)\Big)
	}
	\Explain{Assume $\beta$ is a set of balanced sets in $V$}
	\Explain{ If $v \in \bigcup \beta$, then there is a $B \in \beta$ such that $v \in B$}
	\Explain{ 
	And by definition of balanced $\alpha v \in B \subset \bigcup \beta $ 
	for any $\alpha \in \Cell_k(0,1)$}
	\Explain{
		So $\bigcup \beta$ is Balanced}
	\Explain{ if $v \in \bigcap \beta$, then $v \in B$ for any $B \in \beta$}
	\Explain{ 
	And by definition of balanced $\alpha v \in B \subset \bigcup \beta $ 
	for any $\alpha \in \Cell_k(0,1)$ and for all $B \in \beta$}
	\Explain{
		So $\bigcap \beta$ is Balanced}
	\EndProof
	\\
	\Theorem{AbsorbentAreClosedUnderUnions}
	{
		\forall V \in \TVS{k} \. \forall \alpha : ?\TYPE{Absorbent}(V) \. 
		\TYPE{Absorbent}\left(V,\bigcup\alpha \right)
	}
	\Explain{  This is obvious}
	\EndProof
}\Page{
	\Theorem{AbsorbentAreClosedUnderFiniteIntersections}
	{
		\NewLine ::		
		\forall V \in \TVS{k} \. \forall \alpha : \Finite\Big(\TYPE{Absorbent}(V)\Big) \. 
		\TYPE{Absorbent}\left(V,\bigcap\alpha \right)
	}
	\Explain{  Say $n = |\alpha|$}
	\Explain{ if $n = 0$, then $\bigcap \alpha = V$ which is always absorbent}
	\Explain{
		otherwisr represent $\alpha = \{A_1,\ldots, A_n\}$ and assume $v \in V$}
	\Explain{
		Select a finite sequence $\rho:\{1,\ldots,n\} \to \Reals_{++}$,
		with $\rho_i$ absorbing $v$ for $A_i$}
	\Explain{
	Let $\sigma = \min \{\rho_1,\ldots,\rho_n\}$}
	\Explain{
		Then $\sigma$ is absorbing for every $A_i$, so it is absorbing for $\bigcap \alpha$}
	\EndProof
	\Explain{
		In case of infinite intersiction the minimum may not exit}
	\\
	\DeclareFunc{balancedHull}{\prod_{V : \TVS{k}} 2^V \to \TYPE{Balanced}(V)}
	\DefineNamedFunc{balancedHull}{A}{\bal A}
	{
		\bigcap \Big\{ B : \TYPE{Balanced}(V),  A \subset B \Big\}
	}
	\\
	\Theorem{BalancedHullProductExpression}
	{
		\forall_{V \in \TVS{k}}  \forall A \subset V \. \bal A = \Cell_k(0,1) A
	}
	\Explain{
		Clearly $\Cell_k(0,1)A$ is balanced}
	\Explain{
		Assume that $B$ is a balanced set such that $A \subset B$
	}
	\Explain{
		Then $\Cell_k(0,1) A \subset \Cell_k(0,1) B \subset B$ as
		$B$ as balanced}
	\Explain{
		This proves the result as balanced hull of $A$ may beviewed as the smallest
		balanced set containing $A$}
	\EndProof 
	\\
	\DeclareFunc{balancedCore}{\prod_{V : \TVS{k}} 2^V \to \TYPE{Balanced}(V)}
	\DefineNamedFunc{balancedCore}{A}{ A^{\bal}}
	{
		\bigcup \Big\{ B : \TYPE{Balanced}(V),  B \subset A \Big\}
	}
	\\
	\Theorem{BalancedCoreAsIntersction}
	{
		\forall_{V \in \TVS{k}}  \forall A \subset V \. \bal A = \bigcap_{\alpha \in \Cell_k^\c(0,1)} \alpha A
	}
	\Explain{Firstly, I show that $B = \bigcap_{\alpha \in \Cell_k^\c(0,1)} \alpha A$ is balanced}
	\Explain{
		Assume $v \in B$}
	\Explain{
		Then,  $v \in \alpha A $ for all $\alpha \in \Cell_k^\c(0,1)$}
	\Explain{
		Thus $\Cell_k(0,1)v \subset A$}
	\Explain{
		By definition $A^{\bal}$ as a union this means, that $v \in A^{\bal}$, so 
		$B \subset A^{\bal}$}
	\Explain{
			Assume now that $v \in A^{\bal}$}
	\Explain{
			Then $\Cell_k(0,1) v \subset \Cell_k(0,1)A^{\bal} \subset A^{\bal} \subset A$
			As $A^{\bal}$ is a union of subsets}
	\Explain{
		But this mean that $v \in B$
	, so $A = B$}
	\EndProof
}\Page{
	\Theorem{ClosedBalancedCoreIsOpen}
	{
		\forall V : \TVS{k} \. 
		\forall F : \Closed(V) \.
		\Closed(V,F^{\bal})  
	}
	\Explain{ Multiplication by non-zero scalar is a homeomorphism}	
	\Explain{ So result follows from intersection representation as $\alpha F$ will be closed}
	\EndProof
	\\
	\Theorem{LinearMapsBalancedToBalanced}
	{
		\NewLine ::		
		\forall V,W : \TVS{k} \.
		\forall T \in \VS{k}(V,W) \.
		\forall B : \TYPE{Balanced}(V) \.
		\TYPE{Balanced}\Big(W, T(B) \Big)
	}
	\Explain{ 
		Assume $w \in T(B)$ and $\alpha \in \Disc_k(0,1)$}
	\Explain{  
		Then there is $v \in B$ such that $T(v) = w$}
	\Explain{
		as $B$ is balanced $\alpha v \in B$}
	\Explain{
		Thus $\alpha w = \alpha T(v) = T(\alpha v) \in T(B)$
	}
	\Explain{
		This proves that $T(B)$ is balanced}
	\EndProof
	\\
	\Theorem{LinearSurjectiveMapsAbsorbentToAbsorbent}
	{
		\NewLine ::		
		\forall V,W : \TVS{k} \.
		\forall T \in \VS{k} \And \Surj(V,W) \.
		\forall A : \TYPE{Absorbent}(V) \.
		\TYPE{Absorbent}\Big(W, T(A) \Big)
	}
	\Explain{ 
		Assume $w \in W$}
	\Explain{
		Then there is $v \in V$ such that $T(v) = w$ as $T$ is surjective}
	\Explain{  
		Then there exists $\rho \in \Reals_{++}$ such that $\Disc(0,\rho)v \subset A$
		as $A$ is absorbent}
	\Explain{
		Take $\alpha \in \Disc(0,\rho)$}
	\Explain{
		Then $\alpha w = \alpha T(v) = T(\alpha v) \in T(A)$}
	\Explain{
		This proves that $T(A)$ is absorbent}
	\EndProof
	\\
	\Theorem{BalancedPreimageIsBalanced}{
		\NewLine ::		
		\forall V,W : \TVS{k} \.
		\forall T \in \VS{k}(V,W) \.
		\forall B : \TYPE{Balanced}(W) \.
		\TYPE{Balanced}\Big(V, T^{-1}(B) \Big)
	}
	\Explain{
		Take $v \in T^{-1}(B)$ and $\alpha \in \Disc_k(0,1)$}
	\Explain{
		Then $T(v) \in B$, but also $ T(\alpha v) = \alpha T(v) \in B$ as
		$B$ is balanced }
	\Explain{
		But this means that $\alpha v \in T^{-1}(B)$}
	\EndProof
	\\
	\Theorem{BalancedPreimageIsBalanced}{
		\NewLine ::		
		\forall V,W : \TVS{k} \.
		\forall T \in \VS{k}(V,W) \.
		\forall A: \TYPE{Absorbent}(W) \.
		\TYPE{Absorbent}\Big(V, T^{-1}(A) \Big)
	}
	\Explain{
		Take $v \in V$}
	\Explain{
		Then there is $\rho \in \Reals_{++}$ such that 
		$ T(\alpha v) =  \alpha T(v) \in A$ for any $\alpha \in \Disc_k(0,\rho)$ 
		as $A$ is absorbent}
	\Explain{
		But this means that $\alpha v \in T^{-1}(A)$
	}
	\EndProof
}
\newpage
\subsubsection{Topology and Convexity}
\Page{
	\Conclude{\TYPE{Disc}}{ \Lambda V \in \TVS{k} \. \Convex \And \TYPE{Balanced}(V) }
	{
		\prod_{V \in \TVS{k}} ??V
	}
	\\
	\Theorem{DiscCharacterization}
	{
		\NewLine ::		
		\forall V \in \TVS{k} \.
		\forall D \subset V \.
		\TYPE{Disc}(V,D)
		\iff
		\forall v,w \in D \.
		\forall \alpha, \beta \in k \.
		|\alpha| + |\beta| \le 1
		\Imply
		\alpha v + \beta w \in D 
	}
	\Explain{ Firstly, assume that $D$ is a Disc}
	\Explain{ 
		Take $v,w \in D$ and $\alpha, \beta \in k$ such that $|\alpha| + |\beta| \le 1$}
	\Explain{
		$\alpha v, \beta w \in D$ as $D$ is balanced}
	\Explain{
		So if $\alpha = 0$ or $\beta = 0$ then $\alpha v + \beta w = \alpha v \in V$ or
		$\alpha v + \beta w  = \beta w \in V$}
	\Explain{
		Otherwise, $|\alpha| + |\beta| \neq 0$ and  $ \frac{|\alpha|}{|\alpha| + |\beta|} +  
			\frac{|\beta|}{|\alpha| + |\beta|} = 1$}
	\Explain{
		Also, $\frac{|\alpha| + |\beta|}{|\alpha|} \alpha v, 
		\frac{|\alpha| + |\beta|}{|\beta|} \beta w \in D$
		as $|\alpha| + |\beta| \le 1$ and $D$ is absorbent}
	\Explain{
		Then 
		$\alpha v + \beta w = 
		\frac{|\alpha|}{|\alpha| + |\beta|} \frac{|\alpha| + |\beta|}{|\alpha|} \alpha v +  
		\frac{|\beta|}{|\alpha| + |\beta|}	\frac{|\alpha| + |\beta|}{|\beta|} \beta w \in D
		$ as $D$ is convex}
	\Explain{
		Now assume that the condition holds}
	\Explain{
		Then convexity and being balanced is obvious}
	\EndProof
	\\
	\Theorem{DiskedHull}
	{
		\forall V \in \TVS{K} \.
		\forall A \subset V \.
		\bigcap \Big\{ D : \TYPE{Disc}(V), A \subset D  \Big\} = 
		\conv \bal A
	}
	\Explain{
		Firstly we need to show that $\conv \bal A$ is balanced}
	\Explain{
		Assume $v \in \conv \bal A$ and $\alpha \in \Disc_k(0,1)$}
	\Explain{
		If $\alpha = 0$ then $\alpha v = 0 \in \bal A \subset \conv \bal A$}	
	\Explain{
		Otherwise, if $C$ is convex in $V$, 
		then $ \frac{\alpha}{|\alpha|} C$ is also convex}
	\Explain{
		Also if $\bal A \subset C$ then $ \bal A =  \frac{\alpha}{|\alpha|} \bal A \subset 
		\frac{\alpha}{|\alpha|} C$ as $\bal A$ is balanced}
	\Explain{
		Thus, $\frac{\alpha}{|\alpha|} v \in \conv \bal A $}
	\Explain{
		Also, as it was said $0 \in \bal A \subset \conv \bal A$}
	\Explain{
		So $\alpha v =  \frac{|\alpha|}{|\alpha|}  \alpha v  + \Big(1 - |\alpha|\Big)0 \in \conv \bal A$
		as $\conv \bal A$ is convex}
	\Explain{
		So $\conv \bal A$ is a disk and 
		$B = \bigcap \Big\{ D : \TYPE{Disc}(V), A \subset D  \Big\}   \subset \conv \bal A$}
	\Explain{
		Now assume that $D$ is a disk such that $A \subset D$}
	\Explain{
		Then $\bal A \subset D$ as $D$ is balanced}
	\Explain{
		Furthermore, $\conv \bal A \subset D$ as $D$ is convex}
	\Explain{
	 	Thus $\conv \bal A = B$}
	\EndProof
}\Page{
	\Theorem{TVSIsConnected}
	{
		\forall V \in \TVS{k} \.		
		\Connected(k)
		\Imply
		\Connected(V)
	}
	\Explain{
		Note that $V = \bigcup_{v \in V} kv $}
	\Explain{
		Each $kv$ is connected as continuous image of connected $k$}
	\Explain{
		Then all lines $kv$ intersect at $0$, so  $V$ is connected}
	\EndProof
	\\
	\Theorem{AbsorbentNeighborhoodsOfZero}
	{
		\forall V \in \TVS{k} \.
		\forall U \in \U_V(0) \.
		\TYPE{Absorbent}(V,U)		
	}
	\Explain{ Assume $v \in V$}
	\Explain{
		Then $\lim_{\alpha \to 0} \alpha v = 0$}
	\Explain{
			So, there exists $\rho \in \Reals_{++}$
			such that $\Cell_k(0,\rho) v  \subset U$}
	\Explain{
		Then   $\Disc_k\left(0,\frac{\rho}{2}\right) v \subset  \Cell_k(0,\rho) v  \subset U$}
	\Explain{
		Thus, $U$ is absorbent}
	\EndProof
	\\
	\Theorem{NeighborhoodsOfZeroScaling}
	{
		\forall V \in \TVS{k} \.
		\forall U \in \U_V(0) \.
		\forall \alpha \in k_* \.
		\alpha U \in \U_V(0)
	}
	\Explain{
		$\alpha \cdot \bullet$ is a homeomorphism,
		so $\alpha U$ i	s open
	}
	\Explain{
		Obviously, $0 = \alpha 0 \in \alpha U$ as $0 \in U$}
	\Explain{
		Thus, $U \in \U_V(0)$}
	\EndProof
	\\
	\Theorem{EachNeighborhoodsOfZeroContainsBalancedNeighborhoods}
	{
		\NewLine ::
		\forall V \in \TVS{k} \.
		\forall U \in \U_V(0) \.
		\exists W \in \U_V(0) \.
		W \subset U \And \TYPE{Balanced}(V,W)
	}
	\Explain{
		$(\cdot)^{-1}(U)$ is open in $k \times V$}
	\Explain{
		So there exist $W \in \U_V(0)$ and $\rho \in \Reals_{++}$
		such that $\Cell_k(0,\rho) \times W \subset (\cdot)^{-1}(U) $
		as $0 \in  (\cdot)^{-1}(U)$}
	\Explain{
		This means that $\Cell_k(0,\rho) W \subset U$}
	\Explain{
		Also, note that $\Cell_k(0,\rho) W = \bigcup_{|\alpha| < \rho} \alpha W \in \U_V(0)$}
	\Explain{
		Assume $v \in \Cell_k(0,\rho) W$ and $\alpha \in \Disc_k(0,1)$}
	\Explain{
		Then there is 
		$w \in W$  and $\beta \in \Cell_k(0,\rho)$ such that $v = w\beta$}
	\Explain{
		But $\alpha \beta$ is also in $\Cell_k(0,\rho)$ and
		so $\alpha v = \alpha \beta w \in \Cell_k(0,\rho) W$}
	\Explain{
		Thus, $\Cell_k(0,\rho) W$ is balanced}
	\EndProof
	\\
	\Theorem{ClosedAndBlancedNeighborhoodBase}
	{
		\NewLine ::		
		\forall V \in \TVS{k} \.
		\exists \F : \TYPE{Filterbase}(V,\U_V(0)) \.
		\forall F \in \F \. 
		\Closed \And \TYPE{Balanced}(V,F)
	}
	\Explain{Pretty obvious}
	\EndProof
}\Page{
	\DeclareType{\LConv}
	{
		? \TVS{k}
	}
	\DefineType{V}{\LConv}{
		\exists \F : \TYPE{Filterbase}\Big(V,\N_V(0)\Big) \.
		\forall F \in \F \. \Convex(F,\F)
	}
	\\
	\DeclareFunc{categoryOfLocallyConvexSpaces}
	{
		\AVF(\Reals) \to \CAT
	}
	\DefineNamedFunc{categoryOfLocallyConvexSpaces}{k}{\LCS{k}}
	{
		\NewLine \de		
		(\LConv(k),\VS{k} \cap \TOP,\circ,\id)
	}
	\\
	\DeclareFunc{categoryOfTopologicalVectorSpaces}
	{
		\AVF(\Reals) \to \CAT
	}
	\DefineNamedFunc{categoryOfHausdorffTopologicalVectorSpaces}{k}{\LCHS{k}}
	{
		\NewLine \de		
		(\LConv(k) \And \TYPE{T2},\VS{k} \cap \TOP,\circ,\id)
	}
	\\
	\Theorem{NormedSpaceIsLocallyConvex}
	{
		\NORM(k) \subset \LCHS{k}
	}
	\Explain{ Balls in normed spaces are convex}
	\Explain{
		Also they are metric space, hence Hausdorff}
	\EndProof
	\\
	\Theorem{NormedSpaceIsLocallyConvex}
	{
		\NORM(k) \subset \LCHS{k}
	}
	\Explain{ Balls in normed spaces are convex}
	\Explain{
		Also they are metric space, hence Hausdorff}
	\EndProof
	\\
	\Theorem{LCSHasDiscBase}
	{
		\forall V \in \LCS{k} \.
		\exists \F : \TYPE{Filterbase}\Big(V,\N_V(0), \F\Big) \.
		\forall F \in \F \. \TYPE{Disc}(V,F)
	}
	\Explain{
		Take $U \in \N_V(0)$}
	\Explain{
		Then there exists a convex neighborhood $C \in \N_V(0)$
		with $C \subset U$ as $V$ is locally convex}
	\Explain{
		Then there is $B \subset C$ which is a balanced neiborhood 
		which was proved for all topological vector spaces}
	\Explain{
		Then $\conv B \subset C$ is convex and still an neighborhood of zero}
	\Explain{
		But convex hull of the balanced set is balanced,hence $\conv B$ is a disc
	}
	\EndProof
	\\
	\Theorem{LCSHasOpenDiscBase}
	{
		\forall V \in \LCS{k} \.
		\exists \F : \TYPE{Filterbase}\Big(V,\N_V(0), \F\Big) \.
		\forall F \in \F \. \TYPE{Disc} \And \Open(V,F)
	}
	\NoProof
	\\
	\Theorem{LCSHasClosedDiscBase}
	{
		\forall V \in \LCS{k} \.
		\exists \F : \TYPE{Filterbase}\Big(V,\N_V(0), \F\Big) \.
		\forall F \in \F \. \TYPE{Disc} \And \Closed(V,F)
	}
	\NoProof
}\Page{
	 \Theorem{VectorTopologyByAbsorbentAndBalancedSets}
	 {
		\NewLine ::	 	
	 	\forall V \in \VS{k} \.
	 	\forall \F : \TYPE{GroupFilterbase}(V) \.
	 	\forall \aleph : \F \subset \TYPE{Balanced} \And \TYPE{Absorbent}(V) \.
	 	\Big( V  , \langle \F \rangle_{\TGRP} \Big) \in \TVS{k} 
	 }
	 \Explain{
	 	As $F \in \F$ is balanced, then
	 	$F = -F$, so $ \langle \F \rangle_{\TGRP}$ is a 
	  group topology for $(V,+)$}
	 \Explain{
	 	Now assume $F \in \F$ and $\alpha \in k_*$}
	 \Explain{
	 	Then there exists balanced $U \in \langle \F \rangle_{\TGRP}$
	 	such that $0 \in U$ and $2U \subset U + U \subset F$}
	  \Explain{
	 	Then there exists balanced $U \in \langle \F \rangle_{\TGRP}$
	 	such that $0 \in U$ and $2U \subset U + U \subset F$}
	 \Explain{
	 	This can be generalized to th case when
	 	 $U \in \langle \F \rangle_{\TGRP}$ and
	 	 $2^n U \subset  F$}
	 \Explain{
	 		So, we can take such $U$ that $|\alpha| \le 2^n$ and $\alpha U \subset 2^n U \subset F$
	 		for any $\alpha \in k_*$ and $F \in \F$}
	 \Explain{
	 	Now consider $\alpha \in k_*$, $v \in V$ and $F \in \F$}	
	 \Explain{
	 	There exists $U \in \F(0)$ such that $U + U + U \subset F$}
	 \Explain{
	 	As $U$ is absorbent there is $\rho \in (0,1)$ 
	 	such that $\Cell(0,\rho)v \subset U \subset F$}
	 \Explain{
	 	Thus,
	 	$Cell(0,\rho)(v + U) = \Cell(0,\rho)v + \Cell(0,\rho)U = U + U \subset F$
	 }
	 \Explain{
	 	Now, assume $\alpha \neq 0$
	 }
	 \Explain{
	 	There is $U' \in \F$ such that $\alpha U' \subset U$
	 }
	 \Explain{
	 	Then there is also a $W \in \F$ such that $W \subset U' \cap U$
	 }
	 \Explain{
	 	Thus,
	 	$
	 		\Cell(\alpha,\rho)(v + W)  = 
	 		\alpha v +   \alpha W  + 	\Cell(0,\rho)(v + W) \subset
	 		\alpha v + U + U + U \subset \alpha v + F
	 	$
	 }
	 \Explain{
	 	This proves that scalar multiplication is continuous}
	 \EndProof
	 \\
	 \Theorem{LocallyConvexTopologyByDiscFilterbase}
	 {
		\NewLine ::	 	
	 	\forall V \in \VS{k} \.
	 	\forall \F : \TYPE{Filterbase}(V) \.
	 	\forall \aleph : \F \subset \TYPE{Disc} \And \TYPE{Absorbent} (V) \. \NewLine \.
	 	\forall \beth : \forall F \in \F \. \exists \alpha \in (0,1/2) \. \alpha F \in \F  \. 
	 	\Big( V  , \langle \F \rangle_{\TGRP} \Big) \in \LCS{k} 
	 }
	 \Explain{
	 	We need to show that $\F$ is a group filterbase}
	 \Explain{	Assume $F \in \F$}
	 \Explain{ By assumption there are $\alpha \in (0,1/2)$
	 	such that $\alpha F \in \F$} 
	 \Explain{
	 	Then, as $\alpha F$ is convex and $F$ is absorbent
	 	$\alpha F + \alpha F = 2 \alpha F \subset F$}
	 \Explain{
	 	Thus, by previous theorem $\Big( V  , \langle \F \rangle_{\TGRP} \Big)$
	 	is a topolofical vector space
	 }
	 \Explain{
	 	And it is locally convex as there is a filterbase consising of disks 
	 	by construction}
	 \EndProof
}
\newpage
\subsubsection{Semimetrization}
\Page{
	\DeclareType{FSeminorm}
	{
		\prod V \in \VS{k} \. ?(V \to \Reals_+)
	}
	\DefineType{\sigma}{FSeminorm}
	{
		\Big(
			\forall \alpha \in \Disc_k(0,1) \. 
			\forall v \in V \.
			\sigma(\alpha v) \le \sigma(v)
		\Big)
		\And  \NewLine \And
		\left(
			\forall v \in V \.
			\lim_{n \to \infty} \sigma\left(\frac{v}{n}\right)
		\right)
		\And 
		\left(
			\forall v,w \in V \.
			\sigma(v + w) \le \sigma(v) + \sigma(w)
		\right)
	}
	\\
	\DeclareType{FNorm}
	{
			\prod V \in \VS{k} \. ?\TYPE{FSeminorm}(V)
	}
	\DefineType{\sigma}{FNorm}
	{
		\forall v \in V \. \sigma(v) = 0 \iff v = 0
	}
	\\
	\Theorem{FSeminormSemimetrization}
	{
		\forall V \in \VS{k} \.
		\forall \sigma : \TYPE{FSeminorm} \.
		\exists \tau : \Vect\Top(V) \.
		\sigma \in C(V,\tau)
	}
	\Explain{ I will show that $\sigma$ is a value}
	\Explain{
		Firstly, note that 
		$\sigma(-v)\le \sigma(v)$
		and
		$\sigma(v) \le \sigma(-v)$, so
		$\sigma(v) = \sigma(-v$
	}
	\Explain{
		Also $\sigma(0) = \sigma\left( \frac{0}{n} \right) \to 0$,
		so $\sigma(0)$}
	\Explain{
		Other properties of value follows trivially by commutativity of $+_V$}
	\Explain{
		Now I show that scalar multiplication is continuous
		in topology defined by semimetric $\rho(v,w) = \sigma(v-w)$}
	\Explain{
		There are neighborgoods of zero defined by relation $\sigma(v) < \varepsilon$}
	\Explain{
		By first property of F-seminorm these balls are ballanced
	}
	\Explain{
		And by second property of F-seminorm these balls are absorbent}
	\Explain{
		So produced topology of $\rho$ is a vector space topplogy}
	\EndProof
	\\
	\Theorem{FNormSemimetrization}
	{
		\forall V \in \VS{k} \.
		\forall \sigma : \TYPE{FNorm} \.
		\exists \tau : \Vect\Top(V) \.
		\sigma \in C(V,\tau) \And \TYPE{T2}(V,\tau)
	}
	\Explain{In this case $\rho$ is a metric, so resulting topology musy be Hausdorff}
	\EndProof
	\\
	\DeclareFunc{subspaceSeminorm}
	{
		\prod V  \in \VS{k} \.
		\prod U \subset_{\VS{k}}  V \.
		\TYPE{FSeminorm}(V) \to \TYPE{FSeminorm}\left(\frac{V}{U}\right)
	}
	\DefineNamedFunc{subspaceSeminorm}{\sigma}{[\sigma]_U}
	{
		\Lambda [v] \in \frac{V}{U} \.  \inf_{u \in U} \sigma(v + u)	
	}
	\\
	\Theorem{SubspaceSemimetrization}{
		\forall  V \in \TVS{k} \And \TYPE{Semimetrizable} \.
		\forall U \subset_{\VS{k}} V \.
		 \TYPE{Semimetrizable}\left( \frac{V}{U} \right)
	}
	\NoProof
}
\newpage
\subsubsection{Completion}
\Page{
	\DeclareType{Completion}{
		\prod_{V \in \TVS{k}} ?\sum_{W \in \TVS{k} } \TYPE{TopologicalEmbedding}(V,W)     
	} 
	\DefineType{(W,\iota)}{Completion}{\TYPE{Complete}(V) \And \Dense\Big(W,\iota(V)\Big)}
	\\
	\Theorem{EveryTVSHasACompletion}
	{
		\forall V \in \TVS{k} \.
		\exists \TYPE{Completion}(V)
	}
	\Explain{As with topological Groups}
	\EndProof
	\\
	\DeclareType{TopologicalVectorSpaceSubset}
	{
		\prod_{V \in \TVS{k}} ??V
	}
	\DefineNamedType{U}{TopologicalVectorSpaceSubset}
	{
		U \subset_{\TVS{k}} V
	}
	{
		U \subset_{\VS{k}} V \And \Closed(V,U)
	}
	\\
	\Theorem{CompleteteQuotient}
	{
		\forall V \in \TVS{k} \.
		\forall U \subset{\TVS{k}} V \.
		\Complete(V) \Imply \Complete\left(\frac{V}{U}\right)
	}
	\Explain{As with topological groups}
	\EndProof
	\\
	\Theorem{BalancedHullOfTotallyBoundedIsTotallyBounded}
	{
		\NewLine ::
		\forall V \in \TVS{k} \. 
		\forall B : \TB(V) \.
		\TB(V, \bal B)
	}
	\Explain{ 
		Embed $B$ in a completion of $\hat V$ of $V$}
	\Explain{
		Then $\cl B$ is a compact in $\hat V$}
	\Explain{
		As $\Disc_k(0,1)$ is comapct in $k$, then
		$\Disc_k(0,1){\cl}_{\hat V} B$ is compact is continuous image
		of compact $\Disc_k(0,1) \times {\cl}_{\hat V} B$}
	\Explain{
		Then $\bal B = \Disc_k(0,1) B $ is totally bounded as 
		a subset of compact $\Disc_k(0,1){\cl}_{\hat V} B$}
	\EndProof
	\\
	\Theorem{BalancedHullOfCompactIsCompacts}
	{
		\NewLine ::
		\forall V \in \TVS{k} \. 
		\forall K : \Compacts(V) \.
		\Compacts(V, \bal  K)
	}
	\Explain{ $\Disc_k(0,1)K$ is compact as am image of compact $\Disc_k(0,1) \times K$}
	\EndProof
}\Page{
	\Theorem{ConvexHullofTotallyBoundedAsTotallyBounded}
	{
		\NewLine ::
		\forall V \in \LCS{k} \.
		\forall B : \TB(V) \.
		\TB(V, \conv B)
	}
	\ExplainFurther{
		In order to show that $\conv B$ is totally bounded
		we need to show that $conv B$ can be covered}
	\Explain{
		by finite number of translates $(U + v_i)^n_{i=1}$		
		for any $U \in \U_V(0)$
	}
	\Explain{
		Select disc $D \in \U_V(0)$ such that $D + D \subset U$}
	\Explain{
		This is possible as $V$ is loclly convex}
	\Explain{
		As $K$ totally bounded there are a finite set of translates such that
		$K \subset (D + v_i)^n_{i=1} \subset \conv \{ v_1, \ldots, v_n \} + D$}
	\Explain{
		As sum of convex sets is convex $\conv K \subset \conv \{ v_1, \ldots, v_n \} + D$
	}
	\ExplainFurther{
		As $\conv \{ v_1, \ldots, v_n \} $ is compact it is possible to select a finite set
		of  $m$ translates $u_i$ of $D$ such that} 
	\Explain{	$\conv K \subset  \bigcup_{i=1}^m (D+u_i)$}
	\Explain{
		So $\conv K$ is totally bounded}
	\EndProof	
	\\
	\Theorem{ConvexHullofTotallyBoundedAsTotallyBounded}
	{
		\NewLine ::
		\forall V \in \LCS{k} \and \Complete \.
		\forall K : \Compacts(V) \.
		\Compacts(V, \conv K)
	}
	\Explain{$\conv K$ is closed}
	\Explain{
		And as it was shown in the previous theorem $\conv K$ is also totally bounded,
	hence compact}
	\EndProof
}
\newpage
\subsubsection{Continuous Decompositions}
\Page{
	\DeclareType{TopologicalComplement}
	{
		\prod V : \TVS{k} \. ?\LC(V)
	}
	\DefineNamedType{(U,W)}{TopologicalComplement}{V =_{\TVS{k}} U \oplus W}
	{
		\NewLine \iff		
		\Homeo\Big(U\oplus W, V, \Lambda(u,w) \in U \oplus W \. u  + w  \Big)  
	}
	\\
	\Theorem{TopologicalComplementsByContinuousProjection}
	{
		\NewLine ::		
		\forall V  \in \TVS{k} \.
		\forall U,W : \LC(V) \. 
		U \oplus W =_{\TVS{k}} V
		\iff
		P_{U,W} \in \End_{\TOP}(V) 		
	}
	\Explain{
		Define $T : U \oplus W \to V$ by $T(u,w) = u + w$}
	\Explain{ 
		$(\Rightarrow):$ Assume that $T$ is a homeomorphism}
	\ExplainFurther{ 
		There is an expression $P_{U,W} = T^{-1} P_1 I_U$,
		where $P_1 : U \oplus W \to U$ is a projection,}
	\Explain{ 
		and $I_U : U \to V$ is a natural embedding}
	\Explain{
		Thus, $P_{U,W}$ is continuous as composition 
		of continuous functions}
	\Explain{
		$(\Leftarrow):$Assume $(\Delta, u_\delta + w_\delta)$ is a net in $V$ 
		converging to $0$
	}
	\Explain{
		Then by continuity 
		$
			0 = P_{U,W}(0) = 
			P_{U,W}(\lim_{\delta \in \Delta} u_\delta + w_\delta)  =
			\lim_{\delta \in \Delta} P_{U,W} (u_\delta + w_\delta) = 
			\lim_{\delta \in \Delta} u_\delta
		$}
	\Explain{
		Also $E - P_{U,W} = P_{W,U}$ is continuous}
	\Explain{
		So by the argument similar to one above $\lim_{\delta \in \Delta} w_\delta = 0$}
	\Explain{
		Thus, $\lim_{\delta \in \Delta} (u_\delta,w_\delta) = 0$ and $T^{-1}$ is  
		continuous meaning that $T$ is homeomorphism}
		\EndProof
	\\
	\Theorem{TopologicalComplementsByIsomorphicQuotient}
	{
		\NewLine ::		
		\forall V  \in \TVS{k} \.
		\forall U,W : \LC(V) \. 
		U \oplus W =_{\TVS{k}} V
		\iff
		\Homeo\left(W,\frac{V}{U}, \pi_{U|W} \right)
	}
	\Explain{ 
		$\pi_U$ is a quotient map, and hence continuous}
	\Explain{
		$(\Rightarrow):$Assume $(\Delta,[U + w_\delta])$ is a net in $\frac{V}{U}$ converging to zero}
	\Explain{
			But this means that $\lim_\delta w_\delta = 0$ 
			and $\lim_\delta \pi_{U|W}^{-1}[U + \w_\delta] = \lim_\delta w_\delta= 0$}
	\Explain{
		So $\pi_{U|W}$ is homeomorphism}
	\Explain{
		$(\Leftarrow):$ write $P_{U,W} = \pi_{U} \pi_{U|W}^{-1} I_W$}
	\Explain{
		This is continuous a as composition of continuous functions}
	\Explain{
		So by the previous theorem $V = U \oplus_{\TVS{k}} W$}
	\EndProof
	\\
	\Theorem{ComplementedImpliesClosed}
	{
		\forall V \in \TVS{k}
		\forall (U,W) : \TC(V) \.
		\Closed(V,U)
	}
	\Explain{ 
		By previous theorem $P_{W,U}$ is continuous}
	\Explain{
		Thus, $U = \ker P_{W,U}$ is closed}
	\EndProof
}\Page{
		\DeclareType{\MaxS}{\prod_{V \in \VS{k}} ?\TYPE{VectorSubspace}(V) }
		\DefineType{U}{\MaxS}{\forall W \subvec{k} V \. U \subsetneq W \Imply W = V }
		\\
		\Theorem{MaximalClosedSubspace}
		{
			\NewLine ::			
			\forall V \in \TVS{k} \.
			\forall U \subvec{k} V \.	 \NewLine \.	
			\MaxS \And \Closed(V,U) 
			\iff
			\exists f \in \TOP(V,k) \. 
			U = \ker f \And f \neq 0
		}
		\Explain{ 
			$(\Rightarrow):$ Assume $U$ is closed and maximal subspace in $V$}
		\Explain{
			As $U$ is maximal it should have a codimension 1}
		\Explain{
			So where exists $v \in U^\c$ such that $V = U \oplus \langle v \rangle$}
		\Explain{
			As $U$ is closed, where exists a balanced open subset $O \in \U_V(0)$
			such that $(O + v) \cap U = \emptyset$}
		\Explain{
			assume $u + \alpha v \in O$ is such that $|\alpha| > 1$ and $u \in U$
		}
		\Explain{
			Then, as $O$ is balanced, $\alpha^{-1} u + v \in O$
		}
		\Explain{
			But, then  $(\alpha^{-1} u + v) - v = \alpha^{-1} u \in (O + v) \cap U$,
			which is a contradiction}
		\Explain{
			Thus, $u + \alpha t \in \sigma O$ implies that $|\alpha| < |\sigma|$}
		\Explain{
			Define $f(u + \alpha v) = \alpha : V \to k$}
		\Explain{ 
			Consider a net $v_\delta = u_\delta + \alpha_\delta v$ converging to zero with $u_\delta$ 
			in $U$}
		\Explain{
			But the previous remark shows that $f(v_\delta) = \alpha_\delta$  converges to zero}
		\\
		\Theorem{SchroederBernsteinTHM}
		{
			\NewLine ::			
			\forall V,V' \in \TVS{k} \.
			\forall \aleph : V \cong_{\TVS{k}} V \oplus V \.
			\forall \beth :  V' \cong_{\TVS{k}} V' \oplus V' \. \NewLine \.
			\forall \gimel :  \TC(V,V') \.
			\forall \daleth : \TC(V',V') \.
			V \cong_{\TVS{k}} V'
		}
		\Explain{
			Write  
			$V \cong V' \oplus U = (V' \oplus V') \oplus U \cong V'  \oplus (V' \oplus U) \cong V' \oplus V$
		}
		\Explain{ 
			Symmetricaly, $V' \cong V' \oplus V$ 
		}
		\Explain{
			Thus, $V \cong V \oplus V' \cong V'$
		}
		\EndProof
}
\newpage
\subsubsection{Finite Dimension Conditions}
\Page{
	\Theorem{OneDimTVS}
	{
		\forall V \in \HTVS{k} \.
		\dim V = 1 \iff V \cong_{\TVS{k}} k
	}
	\Explain{
		As dimension is invarint for linear isomorphism $(\Leftarrow)$ is obvious }
	\Explain{
		$(\Rightarrow):$ As $\dim V = 1$ there is a $v \in V$ such that $v \neq 0$ 
		and $V = kv$}
	\Explain{ 
		Then the map	$T(\alpha v) = \alpha$ is a linear isomorphism
	}
	\Explain{
		fix some $\rho \in \Reals_{++}$}
	\Explain{
		As $V$ is Hausdorff there must exist 
		an open set $U \in \U_V(0)$ such that $\rho v \not \in U$}
	\Explain{
		Furthermore, $U$ must have a balanced subset 
		$W \in \U_V(0)$}
	\Explain{
		As $W$ is balanced, $W \subset \Cell(0,\rho)v$
	}
	\Explain{
		So,  $\alpha_\delta v \to 0 \iff \alpha_\delta \to 0$ 
	}
	\Explain{
		Thus, $T$ must be a homeomorphism}
	\EndProof
	\\
	\Theorem{FinDimIsomorphism}
	{
		\NewLine		
		\forall V \in \HTVS{k} \. \forall n \in \Nat \.
		\dim V = n \iff V \cong_{\TVS{k}} (k^n,\|\bullet\|_\infty)
	}
	\Explain{
		I modify the proof of the previous theorem}
	\Explain{
		By algebraic there must exist a base $\mathbf{e} = (e_1,\ldots, e_n)$ of $V$
	}
	\Explain{
		fix $\rho$ in $\Reals_{++}$}
	\Explain{
		As $V$ is Hausdorff and each $e_i \neq 0$ there 
		$U \subset \U_V(0)$ such $\rho e_i \not \in U$ for any $i \in \{1,\ldots,n\}$}
	\Explain{
		So there exists a blanced subset $W$ of $U$ such that
		$W \subset \Cell_{k^n,\|\bullet\|_\infty}(0,\rho)\cdot \mathbf{e}$}
	\Explain{
		Thus, the mapping $\mathbf{\alpha \cdot e} \mapsto \mathbf{\alpha}$
		is continuous}
	\ExplainFurther{
		Also, if $U \in \U_V(0)$ the set $U$ must be absorbent,}
	\Explain{ so there is a sequence $\rho_1,\ldots,\rho_n \in \Reals_{++}$ such that
		$\Disc_k(0,\rho_i) e_i \subset U$}
	\Explain{
		Let $\sigma = \min(e_1,\ldots,e_n) \in \Reals_{++}$}
	\Explain{
		Then $\Cell_{k^n,\|\bullet\|_\infty}(0,\sigma) \cdot \mathbf{e} \subset U$
	}
	\Explain{
		So, the inverse $\alpha \mapsto \alpha \cdot \mathbf{e}$ is also continuous}
	\EndProof
	\\
	\Theorem{FDimdSubspaceIsClosed}
	{
			\forall V \in \HTVS{k} \.
			\forall U \subvec{k} V \.
			\dim U  < \infty \Imply \Closed(V,U)
	}
	\Explain{ 
		$U$ is Hausdorff as a subset of Hausdorff space}
	\Explain{
		Then $U$ is isomorphic to $\ell^\infty_{k,\dim U}$ which is complete}
	\Explain{
		So, $U$ can be viewed as an uniform embedding of complete space into
		$V$, and hence must be closed}
	\EndProof
}\Page{
	\Theorem{ClosedFDimSum}
	{
		\forall V \in \TVS{k} \.
		\forall U \subset_{\TVS{k}} V \.
		\forall W \subvec{k} V \.
		\dim W < \infty \Imply \Closed(V,U + W)
	}
	\Explain{ 
		As $U$ is closed in $V$ the quotient $\frac{V}{U}$ must be Hausdorff}
	\Explain{ 
		As $\dim P_U(W) \le \dim \dim W$ the image $P_U(W)$ is still finite dimensional}
	\Explain{ 
		So by previous theorem $P_U(W)$ is closed in $\frac{V}{U}$}
	\Explain{ But then the preimage $ U + W  = P^{-1}_U P_U(W)$
		is closed as quotient map  $P_U$ is continuous}
	\EndProof
	\\
	\Theorem{FiniteDimensionalDomain}
	{
		\forall V,U \in \HTVS{k} \.
		\forall T \in \VS{k}(V,U) \. \NewLine \.
		\dim V < \infty  \Imply 
		T \in \TVS{k}(V,U)
	}
	\Explain{
		$\dim T(V) \le \dim V$, thus $T(V)$ must be finite dimensional}
	\Explain{
		Thus both $V$ and $T(V)$ are isomorphic to copies of 
		$l^\infty_k$ with coresponding finite dimensions}
	\Explain{
		And $T$ must be continuous as any mapping between such spaces does}
	\\
	\Theorem{FiniteDimensionalCodomain}
	{
		\forall V,U \in \HTVS{k} \.
		\forall T \in \TVS{k} \And \Surj(V,U) \. \NewLine \.
		\dim U < \infty  \Imply 
		\Open(V,U,T)
	}
	\Explain{
		By isomorphism theorem
		$\frac{V}{\ker T} \cong_{\VS{k}} T(V) = U$ 
	}
	\Explain{
		So $\dim \frac{V}{\ker T} < \infty$
	}
	\Explain{
		Also $\frac{V}{\ker T} $ is Haussdorf as $T$ is continuous
	}
	\Explain{
		So by prvious theorem the isomorphism must 
		$\frac{V}{\ker T} \cong_{\VS{k}} U$ must be continuous}
	\Explain{
		So $U$ is also finite dimensional Hausdorff  this bijection is homeomorphism 
		and so $\frac{V}{\ker T} \cong_{\TVS{k}} U$}
	\Explain{
		Denote this homeomorphism by $S$}
	\Explain{
		Then $T$ factors as $P_{\ker T} S$ and both these maps are open}
	\EndProof
}\Page{
	\Theorem{FDimIffLocallyCompact}
	{
		\forall V \in \HTVS{k} \.
		\dim V < \infty 
		\iff
		\TYPE{LocallyCompact}(V)
	}
	\Explain{$(\Rightarrow): V$ is homeomorphic to $l^{\infty}_{k,\dim V}$
		and this space is locally compact.}
	\Explain{
		This can be easily shown by considering a base of closed cubes}
	\Explain{
		So $V$ is locally compact}
	\Explain{
		$(\Leftarrow):$ now consider the case when $V$ is locally compact}
	\Explain{
		Then there exists a compact balansed neighborhood of zero, say $K$}
	\Explain{
		Take $K$ to be any another open neighborhood and choose
		$W \in \U_V(0)$ such balanced set that $W+W \subset U$
	}
	\Explain{
		As $K$ is compact, it is totally bounded and hence can be covered by
		a finite set of translates $K \subset \bigcup^n_{i=1} W + v_i$}
	\Explain{
		As $W$ is absorbent and balanced there is $\rho \in (1,+\infty)$ 
		such that each $v_i \in \rho U$}
	\Explain{
		Then $K \subset \bigcup^n_{i=1} W + v_i \subset W + \rho W \subset \rho W  + \rho W = 
			\rho( W + W) \subset \rho U		
		$
	}
	\Explain{ 
		Thus, sets of form $2^{-n} K$ form base at zero}
		\Explain{
		As $K$ is totally bounded it can can be covered by
		a finite set of translates $K \subset \bigcup^n_{i=1} \frac{1}{2}K + e_i$}
		\Explain{
			$F  = \Span e$ is finite-dimensional and hence closed}
		\Explain{ 
			 $K \subset \bigcup n_{i=1} \frac{1}{2}K + e_i \subset \frac{1}{2} K + F$}
		\Explain{
			But also 
			$\alpha F = F$
			or any non-zero scalar $\alpha$}
		\Explain{
			So $\frac{1}{2}K \subset \frac{1}{4} K + F$
		}
		\Explain{
			Iterating this relation  and substituting we get the result
			that
			$K \subset \frac{1}{2^n} K + F$ for any $n \in \Nat$}
		\Explain{
			This can be rewriten as $K \subset \bigcap^\infty_{n=1} \frac{1}{2^n} K + F = F$  
		}
		\Explain{
			But $K$ spans whole $V$, and so $V = F$ which is finite dimensional}
		\EndProof
		\\
		\Theorem{FDimCompactConvexHullIsCompact}
		{
			\NewLine ::			
			\forall V \in \TVS{k} \.
			\forall K : \Compacts(V) \.
			\dim V < \infty \Imply \Compacts(V,\conv K) \.
		}
		\Explain{
			Let $n = \dim V$}
		\Explain{
			$\conv K$ consists of convex combination of form  
			$\sum^{2n + 1}_{i=1} \lambda_i x_i$ where $\lambda \ge 0$
			and $\sum^{2n + 1}_{i=1} \lambda_i = 1$ and $x_i \in K$  
		}
		\Explain{
			This condition can be express as $\lambda \in \du_{2n+1} \subset k^{2n+1}$}
		\Explain{
			But $\du_{2n+1} $ is also compact, ans so is $\du_{2n+1} \times K^{2n+1}$
			by Tychonoff's theorem}
		\Explain{
			So $\conv K = (\cdot)(\du_{2n+1} \times K^{2n+1})$ 
			is compact as a continuous image of  a compact}
		\EndProof
}
\newpage
\subsubsection{Case of Ultravalued Field}
\Page{
	& k : \TYPE{UltravaluedField}; \\
	\\
	\DeclareType{\AKC}{\prod_{V : \TVS{k}} ??V}
	\DefineType{A}{\AKC}{
		\Disc_k(0,1)A +\Disc_k(0,1)A = A	
	}
	\\
	\DeclareType{\KC}{\prod_{V : \TVS{k}} ??V}
	\DefineType{V}{\KC}{
		\exists v \in V \. \exists A : \AKC(V) \. C = A + v	
	}
	\\
	\Theorem{AbsolutelyKConvexByZeroContaintment}
	{
		\forall V \in \TVS{k} \.
		\forall C : \KC(V) \.
		0 \in C \Imply \AKC(V,C)
	}
	\Explain{ $C$ must be a translate of absolutely K-Convex set, so write $C = A + v$}
	\Explain{
			As $A$ is absolutely K-Convex, then
			$\alpha(x + v) + \beta(y + v) - v \in C$ for any $x,y \in C$ and $\alpha,\beta \in \Disc_k(0,1)$}
	\Explain{
		Take $\alpha = \beta =1, y=0$}
	\Explain{
		Then the expression above reduces to $x + v \in C$}
	\Explain{
		But this means that $ A \subset C$}
	\Explain{
		On the other hand,  $\alpha(x + v) + \beta(y + v) \in A$
		for any $x,y \in C$ and $\alpha,\beta \in \Disc_k(0,1)$}
	\Explain{
		Taking $\alpha = 1, \beta = -1, y=0$, produces $x \in A$}
	\Explain{
		Thus $C \subset A$ and $C = A$ is absolutely K-convex}
	\EndProof
	\\
	\Theorem{TripleCombinationKConvexityCondition}
	{
		\NewLine ::		
		\forall V \in \TVS{k} \.
		\forall C \subset V \. \NewLine \.
		\KC(V,C)
		\iff
		\forall x,y,z \in C \.
		\forall \alpha, \beta,\gamma \in \Disc_k(0,1) \.
		\alpha + \beta + \gamma = 1
		\Imply
		\alpha x + \beta y + \gamma z \in C				
	}
	\Explain{ 1 $(\Rightarrow):$ assume that $C$ is K-convex}
	\Explain{ 1.1 $C$ must be a translate of absolutely K-Convex set, so write $C = A + v$}
	\Explain{
		1.2 Then
		$
			\alpha x + \beta y + \gamma z  =
			\alpha (x - v) + \beta (y - v) + \gamma (z - v)  + v \in C
		$}
	\Explain{ 2 $(\Leftarrow)$}
	\Explain{ 2.1 If $C = \emptyset$ then it is trivially K-convex, so assume the contrary}
	\Explain{ 2.2 Take $v \in V$ and let $A = C - v$}
	\Explain{ 2.3 $A$ is absolutely K-convex}
	\Explain{ 2.3.1 Assume $x,y \in C$ and $\alpha,\beta \in \Disc_k(0,1)$}
	\Explain{ 2.3.2 $1 - \alpha - \beta \in \Disc_k(0,1)$ }
	\Explain{ 2.3.2.1 $| 1 - \alpha - \beta| \le \max\Big(1,|\alpha|,|\beta|\Big) = 1$ }
	\Explain{ 2.3.3 Then by the hypothesis
		$ \alpha x + \beta y  + (1 - \alpha - \beta) v \in C  $  }
	\Explain{ 2.3.4 Translating by $-v$ gives
		$ \alpha (x - v) +  \beta (y -v) =  \alpha x + \beta y   + (1 - \alpha - \beta) v - v \in A  $  }
	\EndProof
}\Page{
\Theorem{convexCombinationKConvexityCondition}
	{
		\NewLine ::		
		\forall V \in \TVS{k} \.
		\forall \aleph : \mathrm{res}\;\mathrm{char} \; k \neq 2 \.
		\forall C \subset V \. \NewLine \.
		\KC(V,C)
		\iff
		\forall x,y \in C \.
		\forall \alpha \in \Disc_k(0,1) \.
		\alpha x + (1 - \alpha) y + \gamma z \in C				
	}
	\Explain{ 1 $(\Rightarrow)$ This direction is obvious}
	\Explain{ 1.1 The convex combination is a weaker form of triple combination in the previous result}
	\Explain{ 2 $(\Leftarrow)$ }
	\Explain{ 2.1 If $C = \emptyset$ then it is trivially K-convex, so assume the contrary}
	\Explain{ 2.2 Take $v \in V$ and let $A = C - v$}
	\Explain{ 2.3 $A$ is absolutely K-convex}
	\Explain{ 2.3.1  Assume $x,y \in C$ and $\alpha,\beta \in \Disc_k(0,1)$}
	\Explain{
		2.3.2 Rewrite $\alpha (x - v) + \beta (y - v) + v =  
		\frac{1}{2}\big( 2 \alpha x  + (1-2\alpha) v \big)
		+ \frac{1}{2}\big( 2 \beta y  + (1-2\beta) v \big)$}
	\Explain{
		2.3.3 Both $\frac{1}{2}\big( 2 \alpha x  + (1-2\alpha) v \big)$ and
		$\frac{1}{2}\big( 2 beta y  + (1-2\beta) v \big)$ in $C$}
	\Explain{
		2.3.3.1 for ultravalue $|2 \alpha | = |\alpha + \alpha| \le |\alpha| = 1$
	}
	\Explain{
		2.3.3.2 Same holds for $\beta$}
	\Explain{
		2.3.3.3 So the convex combination hypothesis can be applied}
	\Explain{ 2.3.4 clearlly $\frac{1}{2} + \frac{1}{2} = 1$, so
		$ \alpha (x - v) + \beta (y - v)  \in A$
	}
	\Explain{ 2.3.4.1   
		$\left|\frac{1}{2}\right| = 1$ as residual characteristic of the field is not $2$}
	\EndProof
	\\
	\Theorem{AbsolutelyKConvexIntersection}
	{
		\forall V : \TVS{k} \.
		\forall I \in \SET \. \NewLine \.
		\forall A : I \to \AKC(V) \.
		\AKC\left(V, \bigcap_{i \in I} A_i \right)
	}
	\Explain{ Obvious}
	\EndProof
}\Page{
	\Theorem{KConvexIntersection}
	{
		\forall V : \TVS{k} \.
		\forall I \in \SET \. \NewLine \.
		\forall C : I \to \KC(V) \.
		\KC\left(V, \bigcap_{i \in I} C_i \right)
	}
	\Explain{ 1 
		Assume that $\bigcap_{i \in I} C_i \neq \emptyset$}
	\Explain{ 1.1
		Otherwise the condition is trivial}
	\Explain{ 2
		Take any $v \in \bigcap_{i \in I} C_i $}
	\Explain{ 3 
		Then  $\left(\bigcap_{i \in I} C_i\right) -v$ is absolutely K-convex and 
		$\bigcap_{i \in I} C_i$ is K-convex}
	\Explain{ 3.1
		$\left(\bigcap_{i \in I} C_i\right) -v = \bigcap_{i \in I} (C_i - v) $
		as translation by $v$ is bijective}
	\Explain{ 3.2
		Then every $C_i - v$ are K-convex sets, which contain zero, so they are absolutely K-Convex}
	\Explain{ 3.3
		So, the intersection $\bigcap_{i \in I} (C_i - v)$ is also absoluterly K-Convex}
	\EndProof
	\\
	\DeclareFunc{kConvexHull}
	{
		\prod_{V : \TVS{k}} (?V) \to \KC(V)
	}
	\DefineNamedFunc{kConvexHull}{X}{\kconv X}
	{
		\bigcap \Big\{  C : \KC(V)  , X \subset C \Big\}	
	}
	\\
	\Theorem{KConvexHullByLinearCombinations}
	{
		\NewLine ::		
		\forall V \in \TVS{k} \. 
		\forall X \subset V \. \NewLine \.
		\kconv X =  \left\{
			  x_{n+1} + \sum^{n}_{i=1} \alpha_i ( x_i - x_{n+1}) \Big|
			  n \in \Int_+,  \alpha : \{1,\ldots,n\} \to \Disc_k(0,1),
			  x : \{1,\ldots,n + 1\} \to X          
		\right\}}
	\Explain{ 1
		Let $B$ denote the set defined above}
	\Explain{ 2
		$B$ is K-Convex}
	\Explain{ 2.1
		Note, that $x_{n+1}$ in definition can be fixed}
	\Explain{ 2.2
		Then $B - x_{n+1}$ is obviously absolutely K-convex}
	\Explain{ 3
		$X \subset B$
	}
	\Explain{ 3.1
		Just take $n=1,\alpha_1 = 1$}
	\Explain{ 4
		So $\kconv X \subset B$
	}
	\Explain{ 5
		If $C$ is K-convex, then $B \subset C$}
	\Explain{ 5.1
		Some $x_{n+1} \in X$ must also be contained in $C$
	}
	\Explain{ 5.2
		So $C - x_{n+1}$ is absolutely K-convex.
	}
	\Explain{ 5.3
		So by induction $\sum^n_{i=1} \alpha_i (x_i - x_{n+1} ) \in C - x_{n+1}  $
	}
	\Explain{ 6
		Thus, $B \subset \kconv X$, and so $B = \kconv X$
	}
	\EndProof
}\Page{
		\DeclareFunc{kDiskHull}
	{
		\prod_{V : \TVS{k}} (?V) \to \AKC(V)
	}
	\DefineNamedFunc{kDiscHull}{X}{K\hyph\mathrm{disc}\; X}
	{
		\bigcap \Big\{  C : \AKC(V)  , X \subset C \Big\}	
	}
	\\
	\Theorem{AbsolutelyKConvexInterior}
	{
		\forall V : \TVS{k} \.
		\forall A : \AKC(V) \.
		\intx A = \emptyset | \intx A = A
	}
	\Explain{ 1 assume $\intx A \neq \emptyset$}
	\Explain{ 2 Take $v \in \intx A$}
	\Explain{ 3 Without loss of generality assume $v = 0$}
	\Explain{ 3.1 Then $A - v$ is an isomorphic absolutely convex set with $0 \in \intx A $}
	\Explain{ 4 Take any $U \in \U_V(0)$ such that $U \subset \intx A \subset A$}
	\Explain{ 5 Now take arbitrary $v \in A$}
	\Explain{ 6 Then $U + v \subset A$ }
	\Explain{ 6.1 $U + v$ consists of elements $u + v$ with $u \in U \subset A$}
	\Explain{ 6.2 As $v \in A$ also and $A$ is absolutely K-convex it must be the case that $u + v \in A$}
	\Explain{ 7 As translation is a homeomorphism $U + v$ is open and so $v \in \intx A$}
	\EndProof
	\\
	\Theorem{OpenKDiscHull}
	{
		\forall V : \TVS{k} \.
		\forall U : \Open(V) \.
		\Open(V,K\hyph\mathrm{disc}\;U)
	}
	\Explain{ 1 $ K\hyph\mathrm{disc}\;U$ is absolutely K-convex}
	\Explain{ 2 $ U \subset   K\hyph\mathrm{disc}\;U$, so $\intx  K\hyph\mathrm{disc}\;U \neq 0$}
	\Explain{ 3 But this means that $K\hyph\mathrm{disc}\;U$ is open}
	\EndProof
	\\
	\DeclareType{\LKConv}{?\TVS{k}}
	\DefineType{V}{\LKConv}
	{
		\exists \F : \TYPE{Filterbase}\Big(V,\U_V(0)\Big) \. \forall F \in \F \. \KC(V,F)
	}
}\Page{
	\Theorem{NonarchimedeanVSHasZeroTopDim}
	{
		\forall V : \LKConv(k) \And \TYPE{T2} \. \dim_\TOP V = 0
	}
	\Explain{ 1 $V$  has a base of closed K-discs}
	\Explain{ 1.1 Consider $U \in \U_V(0)$}
	\Explain{ 1.2 Then there exists an open K-disic $D$ such that $0 \in D \subset \overline{D} \subset U$}
	\Explain{ 1.3 Then $\overline{D}$ is a K-disk}
	\Explain{ 1.3.1 If $u,v \in \overline{D}$ it means that every their open neighborhood meet $D$ }
	\Explain{ 1.3.2 Assume $\alpha,\beta \in \Disc_k(0,1)$ }
	\Explain{ 1.3.3 Consider an open neighborhood $W$ of $\alpha u + \beta v$}
	\Explain{ 1.3.4 Then there is  an open neighborhood of zero  $O + O \subset W - \alpha u- \beta v $}
	\Explain{ 1.3.5 Consider the case $\alpha \neq 0 \neq \beta$}
	\Explain{ 1.3.6 Then there must be some $u' \in D \cap \frac{1}{\alpha}(O + \alpha u)$}
	\Explain{ 1.3.7 Then there is also $v' \in  D \cap \frac{1}{\beta}(O + \beta v)$}
	\Explain{ 1.3.8 Then $\alpha u' + \beta v' \in D$ as $D$ is absoluterly K-convex}
	\Explain{ 1.3.9 Also $\alpha u' + \beta  v' \in O + O + \alpha u + \beta v \subset W$}
	\Explain{ 1.3.10 As $W$ was arbitrary this means that $\alpha u + \beta v \in \overline{D}$}
	\Explain{ 1.4  $\overline{D} \subset U$}
	\Explain{ 1.4.1 This is true as $V$ is Hausdorff, and Hence regular}
	\Explain{ 2 But then every K-disc in this base is clopen}
	\Explain{ 2.1 To be in base every K-disc $D$ should contain an element of $U_V(0)$}
	\Explain{ 2.2 Hence $D$ has non-empty interior}
	\Explain{ 2.3 But This means that $D$ is open}
	\Explain{ 3 Thus $\dim_\TOP V = 0$}
	\EndProof 
	\\
	\DeclareType{RelativelyKConvex}
	{
		 \prod_{V_\TVS{k}}  \prod_{A \subset V} ??A
	}
	\DefineType{R}{RelativelyKConvex}
	{
			\exists C : \KC(K) \.  R = C \cap A
	} 
	\\
	\DeclareType{KConvexFilterbase}
	{
		 \prod V : \TVS{k} \. \prod_{A \subset V} ?\TYPE{Filterbase}(A) 
	}
	\DefineType{\F}{KConvexFilterbase}
	{
		\forall F \in \F \. \TYPE{RelativelyKConvex}(V,A,F)
	}
	\\
	\DeclareType{CCompact}
	{
		 \prod_{V_\TVS{k}}  ??V
	}
	\DefineType{K}{CCompact}
	{
		\forall \F : \TYPE{KConvexFilterbase}(V,K) \.
		\exists \TYPE{AdherencePoint}\Big(V, \F \Big)	
	} 
	\\
	& |\cdot| \neq  \Lambda \alpha \in k \. [\alpha \neq 0] \\ 
}\Page{
	\Theorem{EveryCompactIsCCompact}
	{
		\forall V : \TVS{k} \. 
		\forall K : \Compact(V,K) \.
		 \CCompact(V,K)
	}
	\Explain{ 1 Assume $\F$ is a K-Convex filterbase on $K$}
	\Explain{ 2 Then associated ultrafilter must have a limit}
	\Explain{ 3 This limit is an adherence point of $\F$}
	\EndProof
	\\
	\Theorem{ClosedSubsetOfCCompact}
	{	
		\forall V : \HTVS{k} \.
		\forall K  : \CCompact(V) \.
		\forall L : \Closed(K) \And \KC(V) \.
		\CCompact(V,L)
	}
	\Explain{ 1 Assume $\F$ is a K-Convex filterbase on $L$}
	\Explain{ 2 Then the $\F$ is also a K-Convex filterbase for $K$}
	\Explain{ 3 Then, there is an adherence poiint $p \in K$ fo $\F'$}
	\Explain{ 4 $p$ is also an adherence point for $\F$}
	\Explain{ 4.1  Take any $U \in \U_V(p)$  }
	\Explain{ 4.2  Then $F \cap K \cap U \neq \emptyset$ for any $F \in \F$ }
	\Explain{ 4.3  Bat  all these $F \subset L$}
	\Explain{ 4.4 Thus $p \in \cl_K L = L$}
	\EndProof
	\\
	\Theorem{MaximalConvexFilterbase}
	{
		\NewLine :: 			
		\forall V : \LKConv(k) \.
		\forall C : \KC(V) \.
		\forall \F \in \max \TYPE{KConvexFilterbase}(V,C) \. \NewLine \.
		\forall p \in \C \.
		\TYPE{AherencePoint}(C,\F, p)
		\iff
		\lim \F = p
	}
	\Explain{ 1  $(\Rightarrow):$ Assume $p$ is an adherence point for $\F$ in $\C$}
	\Explain{ 1.1 Then $\forall F \in \. \forall U \in \U_V(p) \. U \cap F \neq \emptyset$ }
	\Explain{ 1.2 Assume that $U \in \U_C(p)$}
	\Explain{ 1.3 Then there exist a K-convex $D$ and open $W \in \U_C(p)$ 
	such that $W \subset D \subset V$}
	\Explain{ 1.4 Then $\forall F \in \F \.  D \cap F  \neq \emptyset$}
	\Explain{ 1.4.1 $\forall F \in \F \.  W \cap F \neq \emptyset$}
	\Explain{ 1.4.2 $W \subset D$ }
	\Explain{ 1.5 As $\F$ is maximal $D \in \F$}
	\Explain{ 1.6 Thus, $p = \lim \F$}
	\Explain{ 2 $(\Leftarrow):$ Now Assume $p = \lim \F$}
	\Explain{ 2.1 Then  $\forall U \in \U_C(p) \. \exists F \in \F \. F \subset U$}
	\Explain{ 2.2 Take arbitrary $U \in \U_C(p)$ and $F \in \F$}
	\Explain{ 2.3 Then by $(2.1)$ there exits $G \in \F$ such that $G \subset Y$}
	\Explain{ 2.4 As $\F$ is a filterbase $G \cap F \neq \emptyset$}
	\Explain{ 2.5 Thus $F \cap U \neq \emptyset$}
	\Explain{ 2.6 This proves that $p$ is and adherence point for $\F$}
	\EndProof
}\Page{
	\Theorem{KConvexAndCcompactIsClosed}
	{
		\NewLine ::		
		\forall V : \LKConv(k) \.
		\forall K : \CCompact \And \KC(V) \.
		\Closed(V,K)
	}
	\Explain{ 1 Assume $p$ is a Limit point for $K$}
	\Explain{ 2 Then there exists an filter $\F$ in $K$ such that $p = \lim \F$ }
	\Explain{ 2.1 Take $\N_V(p) \cap K$ for example}
	\Explain{ 3 Then $p$ is an adherence point of $\F$}
	\Explain{ 4 construct a K-convex filterbase $\C$ from $\F$}
	\Explain{ 4.1 For example, use the fact that $V$ is locally K-convex}
	\Explain{ 4.2 Let $C$ be the intersections of $K$ and K-convex neighborhoods of $p$}
	\Explain{ 5 Then $p$ is still a limit point of $\C$ in $V$}
	\Explain{ 6 There also must exist an adherence point of $\C$ in $K$, say $q$}
	\Explain{ 7 But as $V$ is Hausdorff and $\C$ has a limit it must be the case $q = p$}
	\Explain{ 8 Thus $K$ has all its limit points and must be closed}
	\EndProof
	\\
	\Theorem{CCompactProduct}
	{
		\forall I \in \Set \.
		\forall V : I \to \TVS{k} \.
		\forall C : \prod_{i \in I} \CCompact(V_i) \.
		\CCompact\left(\prod_{i \in I}V_i,\prod_{i \in I} C_i \right)
	}
	\Explain{
		Same proof as Tychonoff's theorem's proof with filters, but with $k$-convex sets}
	\EndProof
	\\
	\Theorem{CCompactCombination}
	{
		\forall V : \LKConv{k} \.
		\forall n \in \Int_+ \.
		\forall D : \{1,\ldots,n\} \to \AKC \And \CCompact(V) \.
		\CCompact\left(V ,\kconv \bigcup^n_{i=1} D_i\right) 
	}
	\Explain{ 1 I will give a proof by induction}
	\Explain{ 2 $\kconv \bigcup^n_{i=1} D_i = \emptyset$ in case $n= 0$ 
		and  is trivially c-compact}
	\Explain{ 3  $\kconv \bigcup^{n+1}_{i=1} D_i = 
		\kconv \left( D_{n+1} + \bigcup^{n}_{i=1} D_i\right)$ by the result
		expressing K-convex hulls by linear combinations}
	\Explain{ 4 So for the induction step we need to prove case 
		of two с-compacts $D_1$ and $D_2$}
	\Explain{ 5 assume $\F$ is a closed k-convex filterbase on $\kconv D_1 \cup D_2$ }
	\Explain{ 6 Let $\F' = \Big\{ \{ (x,y) \in D_1 \times D_2 : 
				\exists \alpha,\beta \in \Disc_k(0,1) \. \alpha x + \beta y \in F    \}   
				\Big| F \in \F \Big\}$} 
	\Explain{ 7 Then $\F'$ is a k-convex fiterbase on $D_1 \times D_2$}
	\Explain{ 8 $D_1 \times D_2$ is c-compact}
	\Explain{ 9 So there is an adherence point $(x,y)$ of $\F'$}
	\Explain{ 10 Let $C = K\hyph\mathrm{disc}\{ x, y\}$}
	\Explain{ 11 Then $C$ is c-compact K-disc}
	\Explain{ 12 Then $\overline{F} \cap C \neq \emptyset$ fo all $F \in \F$}
	\Explain{ 13 So $\F'' = \{ \overline{F} \cap C | F \in \F  \}$ is a filterbas on $C$}
	\Explain{ 14 So there exists and adherence point $P$ of $\F''$}
	\Explain{ 15 But $p$ is als an adherence point of $\F$ then}
	\EndProof
}\Page{
	\Theorem{CCompactIffSphericallyComplete}
	{
		\CCompact(k) \iff \TYPE{SphericallyComplete}(k)
	}
	\Explain{ 1 $(\Rightarrow):$ Assume that $k$ is c-compact}
	\Explain{ 1.1 Let $B:\Nat \to 2^k$ be a deacrising sequence of closed balls}
	\Explain{ 1.2 Then $\B =\{B_i | i \in \Nat\}$ is a $k$-convex filter}
	\Explain{ 1.3 So there must exist and adherence point $\beta$ of $\B$}
	\Explain{ 1.4 Then $\beta \in B_n$ for every $n \in \Nat$}
	\Explain{ 1.4.1 $B_n \cap U \neq \emptyset$for every $U \in \U_k(\beta)$}
	\Explain{ 1.4.2 This means that $\beta \in \overline{B}_n$}
	\Explain{ 1.4.3 But $B_n = \overline{B}_n$ as $B_n$ is closed}
	\Explain{ 1.5 Which can be rendered as $\beta \in \bigcap^\infty_{n=1} B$}
	\Explain{ 2  $(\Rightarrow):$ Assume that $k$ is sphercally complete}
	\Explain{ 2.1  we claim that every $k$-convex set in $k$ is either 
		$\emptyset$ or a ball}
	\Explain{ 2.1.1 Assume $A$ is an absolutely $k$-convex set such that 
		$\emptyset \neq A \neq k$}
	\Explain{ 2.1.2 Take $\omega \in A^\c$}
	\Explain{ 2.1.3 Then $\omega \neq 0$}
	\Explain{ 2.1.4 Then every $\omega'$ such that $|\omega| \le |\omega'|$ is not in $A$} 
	\Explain{ 2.1.4.1 Assume there is some $\omega' \in A$ such that $|\omega| \le |\omega'|$}
	\Explain{ 2.1.4.2 Then $\left| \frac{\omega}{\omega'}\right| \le 1$}	
	\Explain{ 2.1.4.3 Thus, as $A$ is a k-disc, $\omega = \frac{\omega}{\omega'}\omega' \in A$}
	\Explain{ 2.1.5 So the set $R = \Big\{ |\omega| \Big| \omega \in A^\c\Big\}$
		is bounded from above} 
	\Explain{ 2.1.6 Let $r = \sup R$}
	\Explain{ 2.1.7 Take $\alpha \in A$ and $\beta \in k$ with $|\beta| \le |\alpha|$}
	\Explain{ 2.1.8 Then $\beta \in A$}
	\Explain{ 2.1.9 so $A$ is a ball of radius $r$ open or closed depending on iclusion of $r$ to $R$}
	\Explain{ 2.2 Also note, 
		that in non-archimedian space any balls are either disjoin or contained in one or another}
	\Explain{
		2.3 So any $k$-convex filterbase $\F$ in $k$ can be represented as 
		a decreasing sequence of balls, closed or open}
	\Explain{
		2.4 Construct sequence of closed balls $\B$ by taking closures}
	\Explain{
		2.4.1 radii of balls will form a set $R$ bounded from below by $0$}
	\Explain{
		2.4.2 let $\delta = \inf R$}
	\Explain{
		2.4.3 Then there exists a decreasing sequence of balls $B$ with respective radi $r$
		such that $\lim_{n\to\infty} r_n = \delta$}
	\Explain{
		2.4.3.1 This is true as all elements in the filterbase $\F$ must have non-empty intersecion}
	\Explain{
		2.5 Then there exists $\beta \in \bigcap \B$}
	\Explain{
		2.4.4 Take $\B = \{ B_n | n \in \Nat\}$
	}
	\Explain{
		2.6 $\beta$ is an adherence point of $\F$
	}
	\Explain{
		2.6.1 There is some $B \in \B$ such $\beta \in B \subset \overline{F}$ for very element $F \in \F$}
	\Explain{
		2.6.2 Then $F \cap U \neq \emptyset$ for every $U \in \U_k(\beta)$}
	\EndProof
}
\newpage
\subsubsection{Some Interesting Examples}
\Page{
	& k :: \AVF \\
	\\
	\Theorem{NonLocallyConvexSpace}{
		\exists V : \TVS{k} \.  
		\neg\LConv(V)
	}
	\Explain{ 1 Let $V = L^p(\Reals,\lambda)$ for $p \in (0,1)$}
	\Explain{ 2 Its topology can be metrized by the metroc $\rho(f,g) = \int |f- g|^p$  }
	\Explain{ 2.1 we use inequelity of form 
		$\left( \sum^n_{i=1} \alpha_i \right)^p \le \sum^n_{i=1} \alpha_i$
			for 	$\alpha_i > 0$}
	\Explain{ 3 on the other hand $\conv \Cell_V(0,\sigma) \subset \Cell_V(0,2^{p-1}\sigma)$ }
	\Explain{ 3.1 Assume $f \in \Cell_V(0,\sigma)$}
	\Explain{ 3.2 Define $F(t) = \int_{-\infty}^t |f|^p$}
	\Explain{ 3.3 Then $F$ is a continuos function on $[-\infty,+\infty]$ 
		such that $F(-\infty) = 0$ and $F(+\infty) = \rho(0,f)$ }
	\Explain{ 3.4 By intermidieat value theorem there exists $t \in \Reals$
		such that $F(t) = \frac{\rho(0,f)}{2}$
	}
	\Explain{ 3.5 Let $g(x) = f(x)\delta_x(-\infty,t), h(x) = f(x)\delta_x(t,+\infty)$}
	\Explain{ 3.6 Then  $\rho(g,0) \le \frac{\sigma}{2}$ and  $\rho(h,0) \le \frac{\sigma}{2}$
		and $f = h + g = \frac{2}{2}f + \frac{2}{2}g$}
	\Explain{ 3.7 But $2g,2h \in \Cell_V(0,2^{2p-1}\sigma)$, so $f \in \conv \Cell_V(0,2^{2p-1}\sigma)$ }
	\Explain{ 4 By iterating one gets $\conv \Cell_V(0,\sigma) = V$}
	\Explain{ 5 So there are no non-trivial convex neighborhoods of $0$}
	\EndProof
	\\
	\Theorem{NonCompactConvexHullOfTheCompact}
	{
		\exists V  : \TVS{k} \.
		\exists K : \Compacts(V) \.
		\neg \Compacts(V, \conv K)
	}
	\Explain{ 1 Let $V = \ell^1$ }
	\Explain{ 2 Let $K = \left\{0, \delta_1^\bullet,\ldots, \frac{1}{n} \delta_{n}^\bullet  ,\ldots \right\}$}
	\Explain{ 3 Define $\xi_n = 
		\frac{1}{\sum^n_{i=1} 2^{-i}  } \sum^n_{t=1} \frac{2^{-t}}{t} \delta_{t}^\bullet
			\in \conv K$}
	\Explain{ 4 Then $\zeta = 
		\lim_{n \to \infty} \xi_n =  \sum^\infty_{t=1} \frac{2^{-t}}{t} \delta_{t}^\bullet$
	}
	\Explain{ 5 But then $\zeta_i \neq 0$ for all $i \in \Nat$,
		but this means that $\zeta \not \in \conv K$, so $K$ is not compact}
	\EndProof
}\Page{
	\Theorem{NoncomplimentedClosedSubpaceExist}
	{
		\exists V : \TVS{k} \.
		\exists U \subset_{\TVS{k}} V \.
		\neg  \TC(V,U)
	}
	\Explain{ 1 Let $V = \ell^\infty$ }
	\Explain{ 2 Let $U = c_0$}
	\NoProof
	\\
	& k :: \TYPE{UltravaluedField} \\
	\\
	\Theorem{PathologicalConvexSet}
	{
		\NewLine ::		
		\mathrm{res}\;\mathrm{char}(k) = 2
		\Imply
		\exists V : \TVS{k} \.
		\exists A : \neg \KC(V) \.
		\forall a,b \in A \.
		\forall \lambda \in \Disc_k(0,1) \.
		\lambda a + (1 - \lambda)b \in A
	}
	\Explain{ 1 Let $V = k^3$ and let 
		$A = \Big\{ a \in \Disc_k(0,1) : \exists i \in \{1,2,3\} \. a_i \in \Cell_k(0,1) \Big\}$}
	\Explain{2 $A$ has desired property for convex combinations of two elements}	
	\Explain{ 2.1 Assume $\lambda \in \Disc_k(0,1)$ and $a,b \in A$}
	\Explain{ 2.2 Note, 
		either $|\lambda|=1$ or $|1-\lambda| = 1$}
	\Explain{ 2.2.1 $ 1 = [1] = [1 - \lambda  + \lambda] = [1 - \lambda] + [\lambda]$ 
		in a residue1 field $\mathbb{F}_2$ }
	\Explain{ 2.3 There exists some $i,j \in \{1,2,3\}$ such that $|a_i| < 1$ and $|b_j| < 1$}
	\Explain{ 2.4 So $|\lambda a_i| = |\lambda||a_i| < 1$ and $|(1-\lambda)b_j| = 
		|1-\lambda||b_j| < 1$}
	\Explain{ 2.5 so either $| \lambda a_i + (1-\lambda) b_i | < 1 $ or 
					$| \lambda a_j + (1-\lambda) b_j | < 1 $}
	\Explain{ 3 $A$ is not K-convex}
	\Explain{ 3.1 $(-1,1,1) \not \in A$}
	\Explain{ 3.1.1 $|-1| = |1| = 1$ }
	\Explain{ 3.2 on the othe hand $(-1,1,1) =  -1 \cdot e_1 + 1\cdot e_2 + 1 \cdot e_3 \in \kconv A $ }
}
\newpage
\subsection{Towards Bornology}
\subsection{Hahn-Banach Theory}
\subsection{Duality and Weak Notions}
\subsection{ Vector-Valued Hahn-Banach Theorems}
\subsection{Barreled Spaces}
\subsection{Bornological Spaces}
\subsection{Closed Graph Theory}
\subsection{Reflexivity}
\subsection{Norm Convexity}
\section{Spaces of Distributions}
\section{Ordered Topological Vector Spaces}
\newpage
\section*{Sources}
\begin{enumerate}
\item Horvath H. - Topological Vector Spaces and Distributions (1966)
\item K\"othe G. - Topological Vector Spaces (1969)
\item Trevis F.  - Topological Vector Spaces, Distributions and Kernels (2010)
\item Grothendieck - Topological Vector Spaces (1973)
\item Wilansky A. - Modern Methods in Topological Vector Spaces (1978) 
\item Rudin W.  - Functional Analysis (1991) 
\item Fabian M. et al.   - Functional Analysis and infinite-dimensional geometry (2001)
\item Naricci L. ; Beckenstein E. - Topological Vector Spaces I (2010)
\item Богачев В. ; Смолянов О. ; Cоболев В. И. -  Топологические Векторные пространства (2012)
\item Chernikov A. ; Mennin A. ; -  Combinatorial properties of non-archimedean convex sets  (2021)
\end{enumerate}
\end{document}