\documentclass[12pt]{scrartcl}
\usepackage[T2A]{fontenc}
\usepackage[utf8]{inputenc}
\usepackage{mathtools}
\usepackage{amsmath}
\usepackage{amsfonts}
\usepackage{hyperref}
\usepackage{amssymb}
\usepackage{ wasysym }
\usepackage{accents}
\usepackage{extpfeil}
\usepackage{graphicx}
\usepackage{scalerel}
\usepackage[dvipsnames]{xcolor}
\usepackage[a4paper,top=5mm, bottom=5mm, left=10mm, right=2mm]{geometry}
%Markup
\newcommand{\TYPE}[1]{\textcolor{NavyBlue}{\mathtt{#1}}}
\newcommand{\FUNC}[1]{\textcolor{Cerulean}{\mathtt{#1}}}
\newcommand{\LOGIC}[1]{\textcolor{Blue}{\mathtt{#1}}}
\newcommand{\THM}[1]{\textcolor{Maroon}{\mathtt{#1}}}
%META
\renewcommand{\.}{\; . \;}
\newcommand{\de}{: \kern 0.1pc =}
\newcommand{\where}{\LOGIC{where}}
\newcommand{\If}{\LOGIC{if} \;}
\newcommand{\Then}{ \; \LOGIC{then} \;}
\newcommand{\Else}{\; \LOGIC{else} \;}
\newcommand{\Act}[1]{\left( #1 \right)}
\newcommand{\Theorem}[2]{& \THM{#1} \, :: \, #2 \\ & \Proof = \\ } 
\newcommand{\DeclareType}[2]{& \TYPE{#1} \, :: \, #2 \\} 
\newcommand{\DefineType}[3]{& #1 : \TYPE{#2} \iff #3 \\} 
\newcommand{\DefineNamedType}[4]{& #1 : \TYPE{#2} \iff #3 \iff #4 \\} 
\newcommand{\DeclareFunc}[2]{& \FUNC{#1} \, :: \, #2 \\}  
\newcommand{\DefineFunc}[3]{&  \FUNC{#1}\Act{#2} \de #3 \\} 
\newcommand{\DefineNamedFunc}[4]{&  \FUNC{#1}\Act{#2} = #3 \de #4 \\} 
\newcommand{\NewLine}{\\ & \kern 1pc}
\newcommand{\Page}[1]{ \begin{align*} #1 \end{align*}   }
\newcommand{ \bd }{ \ByDef }
\newcommand{\NoProof}{ & \ldots \\ \EndProof}
\newcommand{\Explain}[1]{& \text{#1.} \\}
\newcommand{\ExplainFurther}[1]{& \text{#1} \\}
\newcommand{\Exclaim}[1]{& \text{#1!} \\}
%LOGIC
\renewcommand{\And}{\; \& \;}
\newcommand{\ForEach}[3]{\forall #1 : #2 \. #3 }
\newcommand{\Exist}[2]{\exists #1 : #2}
\newcommand{\Imply}{\Rightarrow} 
%TYPE THEORY
\newcommand{\DFunc}[3]{\prod #1 : #2 \. #3 }
\newcommand{\DPair}[3]{\sum #1 : #2 \. #3}
\newcommand{\Type}{\TYPE{Type}}
\newcommand{\Class}{\TYPE{Kind}}
%%STD
\newcommand{\Int}{\mathbb{Z} }
\newcommand{\NNInt}{\mathbb{Z}_{+} }
\newcommand{\Reals}{\mathbb{R} }
\newcommand{\Complex}{\mathbb{C}}
\newcommand{\Rats}{\mathbb{Q} }
\newcommand{\Nat}{\mathbb{N} }
\newcommand{\EReals}{\stackrel{\mathclap{\infty}}{\mathbb{R}}}
\newcommand{\ERealsn}[1]{\stackrel{\mathclap{\infty}}{\mathbb{R}}^{#1}}
\DeclareMathOperator*{\centr}{center}
\DeclareMathOperator*{\argmin}{arg\,min}
\renewcommand{\i}{\mathrm{i}}
%%Set Theory
\DeclareMathOperator*{\id}{id}
\DeclareMathOperator*{\im}{Im}
\DeclareMathOperator*{\supp}{supp}
\newcommand{\Eq}{\TYPE{Equivalence}}
\newcommand{\EqClass}[1]{\TYPE{EquivalenceClass}\left( #1 \right)}
\newcommand{\End}{\mathrm{End}}
\newcommand{\Aut}{\mathrm{Aut}}
\newcommand{\Func}[2]{\TYPE{Functor}\left( #1, #2 \right)}
\mathchardef\hyph="2D
\newcommand{\Surj}{\TYPE{Surjective}}
\newcommand{\Inj}{\TYPE{Injective}}
\newcommand{\ToInj}{\hookrightarrow}
\newcommand{\ToMono}{\xhookrightarrow}
\newcommand{\ToSurj}{\twoheadrightarrow}
\newcommand{\ToEpi}{\xtwoheadrightarrow}
\newcommand{\ToBij}{\leftrightarrow}
\newcommand{\ToIso}{\xleftrightarrow}
\newcommand{\Arrow}{\xrightarrow}
\newcommand{\Set}{\TYPE{Set}}
\newcommand{\Ideal}{\TYPE{Ideal}}
\newcommand{\du}{\; \triangle \;}
\newcommand{\Finites}{\TYPE{FiniteSubset}}
\renewcommand{\c}{\complement}
\newcommand{\Cover}{\TYPE{Cover}}
\newcommand{\Ultrafilter}{\TYPE{Ultrafilter}}
\newcommand{\Finite}{\TYPE{Finite}}
\newcommand{\SA}{\TYPE{\sigma \hyph Algebra}}
%%ProofWritting
\newcommand{\Say}[3]{& #1 \de #2 : #3, \\}
\newcommand{\SayIn}[3]{& #1 \de #2 \in #3, \\}
\newcommand{\Conclude}[3]{& #1 \de #2 : #3; \\}
\newcommand{\ConcludeIn}[3]{& #1 \de #2 \in #3; \\}
\newcommand{\Derive}[3]{& \leadsto #1 \de #2 : #3, \\}
\newcommand{\DeriveIn}[3]{& \leadsto #1 \de #2 \in #3, \\}
\newcommand{\DeriveConclude}[3]{& \leadsto #1 \de #2 : #3 ; \\} 
\newcommand{\DeriveConcludeIn}[3]{& \leadsto #1 \de #2 \in #3 ; \\} 
\newcommand{\Assume}[2]{& \LOGIC{Assume} \; #1 : #2, \\}
\newcommand{\AssumeIn}[2]{& \LOGIC{Assume} \; #1 \in #2, \\}
\newcommand{\As}{\; \LOGIC{as } \;} 
\newcommand{\Intro}{\LOGIC{I}}
\newcommand{\Elim}{\LOGIC{E}}
\newcommand{\QED}{\; \square}
\newcommand{\EndProof}{& \QED \\}
\newcommand{\ByDef}{\eth} 
\newcommand{\ByConstr}{\jmath}  
\newcommand{\Alt}{\LOGIC{Alternative} \;}
\newcommand{\CL}{\LOGIC{Close} \;}
\newcommand{\More}{\LOGIC{Another} \;}
\newcommand{\Proof}{\LOGIC{Proof} \; }
%CategoryTheory
%Types
\newcommand{\Cov}{\TYPE{Covariant}}
\newcommand{\Contra}{\TYPE{Contravariant}}
\newcommand{\NT}{\TYPE{NaturalTransform}}
\newcommand{\UMP}{\TYPE{UnversalMappingProperty}}
\newcommand{\CMP}{\TYPE{CouniversalMappingProperty}}
\newcommand{\paral}{\rightrightarrows}
%functions
\newcommand{\op}{\mathrm{op}}
\newcommand{\obj}{\mathrm{obj}}
\DeclareMathOperator*{\dom}{dom}
\DeclareMathOperator*{\codom}{codom}
\DeclareMathOperator*{\colim}{colim}
%variable
\renewcommand{\C}{\mathcal{C}}
\newcommand{\A}{\mathcal{A}}
\newcommand{\B}{\mathcal{B}}
\newcommand{\D}{\mathcal{D}}
\newcommand{\I}{\mathcal{I}}
\newcommand{\J}{\mathcal{J}}
\newcommand{\R}{\mathrm{R}}
%Cats
\newcommand{\CAT}{\mathsf{CAT}}
\newcommand{\SET}{\mathsf{SET}}
\newcommand{\PARALLEL}{\bullet \paral \bullet}
\newcommand{\WEDGE}{\bullet \to \bullet \leftarrow \bullet}
\newcommand{\VEE}{\bullet \leftarrow \bullet \to \bullet}
%Topology
%General Topology
%Types
\newcommand{\Top}{\TYPE{Topology}}
\newcommand{\TS}{\TYPE{TopologicalSpace}} 
\newcommand{\NbhdBase}{\TYPE{NeighborhoodBase}}
\newcommand{\LF}{\TYPE{LocallyFinite}}
\newcommand{\PN}{\TYPE{PerfectlyNormal}}
\newcommand{\CR}{\TYPE{CompletelyRegular}}
\newcommand{\OM}{\TYPE{OpenMap}}
\newcommand{\Filter}{\TYPE{Filter}}
\newcommand{\Filterbase}{\TYPE{Filterbase}}
\newcommand{\CFilterbase}{\TYPE{ConvergentFilterbase}}
\newcommand{\Dense}{\TYPE{Dense}}
\newcommand{\Separable}{\TYPE{Separable}}
\newcommand{\ND}{\TYPE{NowhereDense}}
\newcommand{\Open}{\TYPE{Open}}
\newcommand{\Net}{\TYPE{Net}}
\newcommand{\Closed}{\TYPE{Closed}}
\newcommand{\Clopen}{\TYPE{Clopen}}
\newcommand{\Nbhd}{\TYPE{Neighborhood}}
\newcommand{\Compact}{\TYPE{Compact}}
\newcommand{\Compacts}{\TYPE{CompactSubset}}
\newcommand{\OpenC}{\TYPE{OpenCover}}
\newcommand{\Cluster}{\TYPE{Cluster}}
\newcommand{\Convergent}{\TYPE{Convergent}}
\newcommand{\LC}{\TYPE{LocallyCompact}}
\newcommand{\Bair}{\TYPE{BaireSpace}}
\newcommand{\Meager}{\TYPE{Meager}}
\newcommand{\Connected}{\TYPE{Connected}}
%FUNC
\DeclareMathOperator*{\intx}{int}
\DeclareMathOperator*{\cl}{cl} 
\DeclareMathOperator*{\boundary}{\partial} 
\DeclareMathOperator{\combo}{\triangledown} 
\DeclareMathOperator{\diag}{\triangle} 
\DeclareMathOperator{\rem}{rem}
%CATS
\newcommand{\TOP}{\mathsf{TOP}}
\newcommand{\HC}{\mathsf{HC}}
\newcommand{\CG}{\mathsf{CG}}
%Symbols
\newcommand{\T}{\mathcal{T}}
\renewcommand{\U}{\mathcal{U}}
\renewcommand{\O}{\mathcal{O}}
\renewcommand{\d}{\mathrm{d}}
\newcommand{\F}{\mathcal{F}}
\newcommand{\X}{\mathcal{X}}
%\newcommand{\d}{\mathrm{d}}
%Metic Topology
%FUNC
\DeclareMathOperator{\diam}{diam}
\newcommand{\Cell}{\mathbb{B}}
%CATS
\newcommand{\Semiiso}{\mathsf{SMS}_{\circ \to \cdot}}
\newcommand{\Iso}{{\mathsf{MS}_{\circ \to \cdot}}}
\newcommand{\SMS}{\mathsf{SMS}}
\newcommand{\MS}{\mathsf{MS}}
\newcommand{\UNI}{\mathsf{UNI}}
\newcommand{\UNIS}{\mathsf{UNIS}}
\newcommand{\TG}{\mathsf{TG}}
\newcommand{\CSeq}{\TYPE{CauchySequence}}
\newcommand{\Complete}{\TYPE{Complete}}
%Descriptive Set Theory
%TYPE
%Descriptive Set Theory
%TYPE
%\newcommand{\Bool}{\mathbb{B}}
\newcommand{\IS}{\TYPE{InitialSegement}}
\newcommand{\FS}[1]{{#1}{}^*}
\newcommand{\Ext}{\TYPE{Extension}}
\newcommand{\Tree}{\TYPE{Tree}}
\newcommand{\Pruned}{\TYPE{Pruned}}
\newcommand{\PTM}{\TYPE{ProperTreeMorphism}}
\newcommand{\LTM}{\TYPE{LipschitzTreeMorphism}}
\newcommand{\Polish}{\TYPE{Polish}}
\newcommand{\IIPG}{\TYPE{InfiniteIterativeTwoPlayersGame}}
\newcommand{\FPS}{\TYPE{FirstPlayerStrategy}}
\newcommand{\SPS}{\TYPE{SecondPlayerStrategy}}
\newcommand{\FPWS}{\TYPE{FirstPlayerWinningStrategy}}
\newcommand{\SPWS}{\TYPE{SecondPlayerWinningStrategy}}
\newcommand{\CS}{\TYPE{ChoquetSpace}}
\newcommand{\SCS}{\TYPE{StrongChoquetSpace}}
\newcommand{\BP}{\mathbf{BP}}
\newcommand{\MGR}{\mathbf{MGR}}
\newcommand{\cat}{\mathbf{CAT}}
\newcommand{\BM}{\TYPE{BairMeasurable}}
\newcommand{\CGSA}{\TYPE{CountablyGeneratedSigmaAlgebra}}
\newcommand{\MC}{\TYPE{MonotonicClass}}
\newcommand{\PSA}{\TYPE{PointSeparatingAlgebra}}
\newcommand{\SBS}{\TYPE{StandardBorelSpace}}
%FUNC
\DeclareMathOperator{\len}{len}
\newcommand{\inits}[2]{{#1}_{|\left[1,\ldots,#2\right]}}
\DeclareMathOperator{\lb}{lb}
\DeclareMathOperator{\WFpart}{WF}
\DeclareMathOperator{\Tr}{Tr}
\DeclareMathOperator{\PTr}{PTr}
\DeclareMathOperator*{\Tll}{{T\;\underline{lim}}}
\DeclareMathOperator*{\Tul}{{T\;\overline{lim}}}
\DeclareMathOperator*{\Tl}{{T\;lim}}
\DeclareMathOperator{\rankcb}{rank_{CB}}
\DeclareMathOperator{\lp}{lp}
\newcommand{\alg}{\mathsf{A}}
%CATS
\newcommand{\TREE}{\mathsf{TREE}}
\newcommand{\FSF}{\mathsf{FS}}
\newcommand{\CRONE}{\mathsf{CRONE}}
\newcommand{\BODY}{\mathsf{BODY}}
\newcommand{\BOR}{\mathsf{BOR}}
\newcommand{\bor}{\mathsf{B}}
\newcommand{\Effros}{\mathsf{EFF}}
%symbols
\newcommand{\K}{\mathsf{K}}
\renewcommand{\H}{\mathrm{H}}
\renewcommand{\L}{\mathcal{L}}
\renewcommand{\P}{\mathcal{P}}
\renewcommand{\S}{\mathcal{S}}
%Algebra
%Groups
%Types
\newcommand{\Group}{\TYPE{Group}}
\newcommand{\Abel}{\TYPE{Abelean}}
\newcommand{\Sgrp}{\subset_{\mathsf{GRP}}}
\newcommand{\Nrml}{\vartriangleleft}
\newcommand{\FG}{\TYPE{FiniteGroup}}
\newcommand{\Stab}{\mathrm{Stab}}
\newcommand{\FGA}{\TYPE{FinitelyGeneratedAbelean}}
\newcommand{\DN}{\TYPE{DirectedNormality}}
\newcommand{\Sphere}{\mathbb{S}}
\newcommand{\Torus}{\mathbb{T}}
%Func
\newcommand{\ActOn}{\curvearrowright}
\DeclareMathOperator{\tor}{tor}
\DeclareMathOperator{\bool}{bool}
\DeclareMathOperator{\rank}{rank}
%Cats
\newcommand{\GRP}{\mathsf{GRP}}
\newcommand{\ABEL}{\mathsf{ABEL}}
%LINEAR
%Types
\newcommand{\Basis}{\TYPE{Basis}}
%CatsG
\newcommand{\VS}{\mathsf{VS}}
%FIELDS
\newcommand{\Field}{\mathbb{F}}
%Functional Analysis
%TYPES
\newcommand{\PLF}{\TYPE{PositiveLinearFunctional}}
\newcommand{\NS}{\TYPE{NormedSpace}}
\newcommand{\SNS}{\TYPE{SeminormedSpace}}
\newcommand{\Banach}{\TYPE{Banach}}
\newcommand{\IPS}{\TYPE{InnerProductSpace}}
\newcommand{\Hilbert}{\TYPE{Hilbert}}
\newcommand{\TopC}{\TYPE{TopologicalyCompletable}}
\newcommand{\AbsC}{\TYPE{AbsolutelyConvergent}}
\newcommand{\PNS}{\TYPE{PolynormedSpace}}
\newcommand{\CNS}{\TYPE{CountablyNormedSpace}}
\newcommand{\TVS}{\TYPE{TopologicalVectorSpace}}
%CATS
\newcommand{\NORM}{\textsf{NORM}}
\newcommand{\NORMI}{{\textsf{NORM}_{\circ \to \cdot}}}
\newcommand{\BAN}{\textsf{BAN}}
\newcommand{\BANI}{{\textsf{BAN}_{\circ \to \cdot}}}
\newcommand{\HIL}[1]{#1\hyph\textsf{HIL}}
\newcommand{\HILI}{{\textsf{HIL}_{\circ \to \cdot}}}
%Simbols
\newcommand{\w}{\mathbf{w}}
%TOPALG
%\TYPES
\newcommand{\Connector}{\TYPE{Connector}}
\newcommand{\CConnector}{\TYPE{ClosedConnector}}
\newcommand{\Unif}{\TYPE{Uniformity}}
\newcommand{\US}{\TYPE{UniformSpace}}
\newcommand{\UB}{\TYPE{UniformBase}}
\newcommand{\Sym}{\TYPE{SymmetricConnector}}
\newcommand{\SB}{\TYPE{SymmetricBase}}
\newcommand{\BofU}{\TYPE{BaseOfUniformity}}
\newcommand{\UNbhd}{\TYPE{UniformNeighborhood}}
\newcommand{\UC}{\TYPE{UniformlyContinuous}}
\newcommand{\UHomeo}{\TYPE{Unimorphism}}
\newcommand{\UniCov}{\TYPE{UniformCover}}
\newcommand{\CF}{\TYPE{CauchyFilterbase}}
\newcommand{\CUS}{\TYPE{CompleteUniformSpace}}
\newcommand{\SCUS}{\TYPE{SequenceCompleteUniformSpace}}
\newcommand{\Small}{\TYPE{Small}}
\newcommand{\TB}{\TYPE{TotallyBounded}}
\newcommand{\MUS}{\TYPE{MetrizableUniformSpace}}
\newcommand{\Completion}{\TYPE{Completion}}
\newcommand{\SCompletion}{\TYPE{SeparableCompletion}}
\newcommand{\pt}{\mathrm{pt}}
\newcommand{\EqC}{\TYPE{Equicontinuous}}
\newcommand{\UEqC}{\TYPE{UnifomlyEquicontinuous}}
\newcommand{\LIM}{\TYPE{LeftInvariantMetric}}
\newcommand{\RIM}{\TYPE{RightInvariantMetric}}
\newcommand{\TIM}{\TYPE{TwosidedInvariantMetric}}
\renewcommand{\SS}{\TYPE{SymmetricSet}}
\newcommand{\veemetric}{\TYPE{\vee\hyph Semimetric}}
\newcommand{\TGC}{\TYPE{TopologicalGroupCompletion}}
\newcommand{\BG}{\TYPE{BaireGroup}}
\newcommand{\PG}{\TYPE{PolishGroup}}
\newcommand{\Borg}{\TYPE{BorelGroup}}
\newcommand{\cli}{\TYPE{cli}}
\newcommand{\Selector}{\TYPE{Selector}}
\newcommand{\Transversal}{\TYPE{Transversal}}
\newcommand{\SBG}{\TYPE{StandardBorelGroup}}
\newcommand{\Polishable}{\TYPE{Polishable}}
\newcommand{\SIN}{\TYPE{SIN}}
%FUNC
\newcommand{\inv}{\mathrm{inv}}
\DeclareMathOperator{\perm}{\mathrm{perm}}
%\CAT
\newcommand{\TGRP}{\mathsf{TGRP}}
\newcommand{\PGRP}{\mathsf{PGRP}}
%\Symbol
\newcommand{\V}{\mathcal{V}}
\newcommand{\W}{\mathcal{W}}
\renewcommand{\L}{\mathcal{L}}
\renewcommand{\R}{\mathcal{R}}
\renewcommand{\S}{\mathcal{S}}
\author{Uncultured Tramp} 
\title{Topological Groups}
\begin{document}
\maketitle
\newpage
\tableofcontents
\newpage
\section{Uniform Spaces}
Uniform spaces generalizes many concept of metric topology beyond real numbers.
\subsection{Connector Uniformities}
One tool defining Uniformities are connectors or entourages.
\subsubsection{Connectors}
Connectors can be thought as large binary matrices with ones on diagonals.
\Page{
	\DeclareType{Connector}
	{
		\prod_{X \in \SET} ?(X \to ?X)
	}
	\DefineType{U}{Connector}{\forall x \in X \. x \in U(x)}
	\\
	\DeclareFunc{SemimetricConnector}
	{
		\prod X \in \SMS \. 
		\Reals_{++} \to \TYPE{Connector}(X)
	}
	\DefineNamedFunc{SemimetricConnector}{\varepsilon}
	{\mathbb{B}_\varepsilon}
	{
		\Lambda x \in X \. \mathbb{B}(x,\varepsilon)
	}
	\\
	\DeclareFunc{connectorAsSubset}
	{
		\prod_{X \in \SET} \TYPE{Connector}(X) \to ?X^2 
	}
	\DefineNamedFunc{connectorAsSubset}{U}{U}
	{
		\Big\{ (x,y) \in X^2 \Big|  y \in U(x) \Big\}
	}
	\\
	\Theorem{ConnectorContainsDiagonal}
	{
		\forall X \in \SET \.
		\forall U : \TYPE{Connector}(X) \.
		\Delta(X) \subset U
	}
	\NoProof
	\\
	\DeclareType{SubsetWithDiagonal}
	{
		\prod_{X \in \SET} ??X^2
	}
	\DefineType{V}{SubsetWithDiagonal}{\Delta(X) \subset U}
	\\
	\DeclareFunc{connectorFromDiagonalSubset}
	{
		\prod_{X \in \SET}
		\TYPE{SubsetWithDiagonal}(X) \to \Connector(X)
	}
	\DefineNamedFunc{connectorFromDiagonalSubset}{V}{V}
	{
		\Lambda x \in X \. V_x	
	}
	\\
	\DeclareFunc{transpose}
	{
		\prod_{X \in \SET}  \Connector(X) \to \Connector(X)
	}
	\DefineNamedFunc{transpose}{U}{U^\top}{\Big\{ (y,x)  \Big| (x,y)  \in U   \Big\}}
	\\
	\Theorem{IdepmpotenTransposition}
	{
		\forall X \in \SET \.
		\forall U : \Connector(X) \.
		(U^\top)^\top = U
	}
	\NoProof
}\Page{
	\Theorem{ConnectorContainment}
	{
		\forall X \in \SET \.
		\forall U,V : \Connector(X) \.
		U \subset V \Imply \forall x \in X \. U(x) \subset V(x)
	}
	\NoProof
	\\
	\Theorem{ConnectorTransposeContainment}
	{
		\forall X \in \SET \.
		\forall U,V : \Connector(X) \.
		U \subset V \Imply \forall x \in X \. U^\top \subset V^\top
	}
	\NoProof
	\\
	\Theorem{ConnectorSetAlgebra}
	{
		\forall X \in \SET \.
		\forall U,V : \Connector(X) \.
		\Connector(X,U \cap V \And X \cup V)
	}
	\NoProof
	\\
	\Theorem{TransposeIntersection}
	{
		\forall X \in \SET \.
		\forall U,V : \Connector(X) \.
		(U\cap V)^\top = U^\top \cap V^\top
	}
	\NoProof
	\\
	\Theorem{TransposeComposition}
	{
		\forall X \in \SET \.
		\forall U,V : \Connector(X) \.
		(U \circ V)^\top = V^\top \circ U^\top
	}
	\NoProof
	\\
	\Theorem{SemimetricConnectorComposition}
	{
		\forall X \in \SMS \.
		\forall t,s \in \Reals_{++} \.
		\Cell_t \circ \Cell_s \subset \Cell_{t+s}
	}
	\AssumeIn{(x,z)}{\Cell_t \circ \Cell_s}
	\Say{\Big(y,[1]\Big)}{\Elim \FUNC{composition}(\Cell_t,\Cell_s)}
	{
		\sum_{ y \in X}  (x,y) \in \Cell_t \And (y,z) \in \Cell_s
	}
	\Say{[2]}{\Elim \Cell_t [1.1]}{d(x,y) < t}
	\Say{[3]}{\Elim \Cell_s [1.2]}{d(y,z) < s}
	\Say{[4]}{\THM{TriangleIneq}(X)[2][3]}
	{
		d(x,z) \le d(x,y) + d(y,z) < t + s
	}
	\Conclude{\Big[(x,z).*]}{\Intro \Cell_{t+s}}
	{
		(x,z) \in \Cell_{t+s}
	}
	\DeriveConclude{[*]}{\Intro \subset}{\Cell_t \circ \Cell_s \subset \Cell_{s + t}}
	\EndProof	
	\\
	\Theorem{CompositionContainment}
	{
		\forall X \in \SET \.
		\forall U,V : \Connector(X) \.
		U \subset U\circ V \And U \subset V\circ U
	}
	\Say{[1]}{\THM{ConnectorContainsDiagonal}(X,V)}{\Delta(X) \subset V}
	\Conclude{[*]}{\Elim \FUNC{composition}[1]}
	{
			U \subset U\circ V \And U \subset V\circ U
	}
	\EndProof
	\\
	\Theorem{SemimetricConnectorIsMonotonic}
	{
		\forall X \in \SMS \.
		\Reals_{++} \Arrow{\Cell} ?X^2 : \mathsf{POSET}
	}
}\Page{
	\Theorem{SemimetricConnectorIntersection}
	{
		\forall X \in \SMS \.
		\forall t,s \in \Reals_{++} \.
		\Cell_t \cap \Cell_s = \Cell_{s\wedge t}
	}
	\Say{[1]}{\THM{TosetMeet}(\Reals_{++},t,s)}{s \wedge t = t | s \wedge t = s}
	\Say{[2]}{
		\THM{SemimetricConnectorIsMonotonic}(X)
		\THM{NestedIntersection}(X)
		\Elim =_{\Reals}	
	}
	{
		\NewLine :		
		s \wedge t = t \Imply 
			\Cell_t \cap \Cell_s \ = 
			\Cell_t =			
			\Cell_{t\wedge s}  
		\And 
		s \wedge t = s \Imply 
			\Cell_s \cap \Cell_t = 
			\Cell_s = 
			\Cell_{t \wedge s}
	}
	\Conclude{[*]}{\Elim | [1][2] }
	{
		\Cell_s \cap \Cell_t =  \Cell_{t \wedge s}
	}
	\EndProof
	\\
	\Theorem{ConnectorProductContainment}
	{
		\forall X \in \SET \.
		\forall U,V  : \Connector(X) \.
		\forall x \in X \. 
		U(x) \times V(x) \subset V \circ U^\top 
	}
	\AssumeIn{(a,b)}{U(x) \times V(x)}
	\Say{[1]}{\Elim \Connector(X)\Elim (a,b)}
	{	
			(x,a) \in U \And (x,b) \in V
	}
	\Conclude{\Big[(a,b).*]}{\Elim \circ [1]}
	{
		(a,b) \in V \circ U^\top
	}
	\DeriveConclude{[*]}{\Intro \subset}
	{
		U(x) \times V(x) \subset V \circ U^\top 
	}
	\EndProof
	\\
	\Theorem{ConnectorProductContainment2}
	{
		\NewLine ::		
		\forall X \in \SET \.
		\forall V  : \Sym(X) \. 
		\forall x \in X \.  
		V(x) \times V(x) \subset V \circ V 
	}
	\NoProof
}
\newpage
\subsubsection{Uniform Topology}
Unifirmities generalize concept of cells or balls from metric topology.
They are filters of connectors with nice properties.
Also, they define topology similarly to a distance metric.
\Page{
	\DeclareType{Uniformity}
	{
		\prod_{X \in \SET} ?\TYPE{Filter}(X^2)
	}
	\DefineType{\U}{Uniformity}
	{
		\forall U \in \U \.
		\TYPE{WithDiagonal}(X,U) \And U^\top \in \U \And
		\exists V \in \U \.  V \circ V \subset U 
	}
	\\
	\DeclareType{UniformityBase}
	{
		\prod_{X \in \SET} ?\TYPE{Filterbase}(X^2)
	}
	\DefineType{\B}{UniformityBase}
	{
		\forall U \in \B \.
		\TYPE{WithDiagonal}(X,U) \And \exists V \in \B \.  V \subset U^\top  \And
		\exists V \in \B \.  V \circ V \subset U 
	}
	\\
	\Conclude{\US}{\sum_{X \in \SET} \Unif(X) }{\Type}
	\\
	\Theorem{UniformBaseGeneratesUnifomity}
	{
		\forall X \in \SET \.
		\forall \B : \UB(X) \.
		\Unif\Big(X,\langle \B \rangle\Big)
	}
	\NoProof
	\\
	\Theorem{BasesOfUniformityAreUniformityBases}
	{
		\forall X \in \SET \.
		\forall \U : \Unif(X) \.
		\forall \B : \TYPE{FilterBase}(X) \. \NewLine \.
		(\B \subset \U \And 
		\forall U \in \U \. \exists B \in \B \. B \subset U )
		\Imply U = \langle B \rangle
		\And
		\UB(X,\B)
	}
	\\
	\Theorem{SemimetricFilterbase}
	{
		\forall X \in \SMS \.
		\exists X \Imply
		\UB\Big( X, \{ \Cell_\varepsilon | \varepsilon \in \Reals_{++}  \} \Big)
	}
	\Say{\B}{  \{ \Cell_\varepsilon | \varepsilon \in \Reals_{++}  \}  }{??X^2}
	\Say{[1]}{\THM{SemimetricConnectorInersection}(X)\Intro \B}
	{
		\forall U,V \in \B \. U \cap V \in \B
	}
	\Say{[2]}{\Elim \B [0]\Intro \B}{\exists \B}
	\Say{[3]}{\Intro \TYPE{Filterbase}[1][2]}{\TYPE{Filtrerbase}(X,\B)}
	\Say{[4]}{\Elim \B\Elim \TYPE{Symmetric}(X,\Reals,d)\Intro \B}
	{
		\forall U \in \B \. U^\top \in \B
	}
	\AssumeIn{U}{\B}
	\Say{\Big(t,[5]\Big)}{\Elim \B(U)}{\sum_{t \in \Reals_{++}} (U = \Cell_t)}
	\Conclude{[U.*]}
	{
		\THM{SemimetricConnectorComposition}\left(X,\frac{t}{2},\frac{t}{2}\right)	
	}
	{
		\Cell_{\frac{t}{2}} \circ \Cell_{\frac{t}{2}}
		\subset
		\Cell_{t}		
	}
	\Derive{[5]}{\Elim \B \Intro \forall}
	{
		\forall U \in \B \. \exists V \in \B \. V \circ V \subset U  
	}
	\Conclude{[*]}{\Intro \UB [3] \Elim \B \Elim \Cell \Intro \B [4][5]}
	{
		\UB(X,\B)
	}
	\EndProof
	\\
}\Page{
	\DeclareFunc{uniformTopology}
	{
		\prod_{X \in \SET}
		\Unif(X) \to \TYPE{Topology}(X)
	}
	\DefineNamedFunc{uniformTopology}{\U}{\T_\U}{
		\Big\{
				O \subset X \Big| 
				\forall x \in O \. \exists U \in \U \. U(x) \subset O  
		\Big\}
	}
	\Say{\T}{\FUNC{uniformTopology}(\U)}
	{
		??X	
	}
	\Say{[1]}{\Elim \T \LOGIC{ExNihilo}(X) \Intro \T}
	{
		\emptyset \in X
	}
	\Say{[2]}{\Elim_1 \TYPE{Filter}(X,\U)}{\U \neq \emptyset}
	\Say{[3]}{\Elim \T [2] \THM{UniversumContainsAll}(X)}
	{
		X \in \T
	}
	\Say{[4]}{\Elim \T [1] \THM{UnionContainsSubsets}(X)}
	{
		\forall \O \subset \T \. \bigcup \O \in \T
	}
	\AssumeIn{n}{\Nat}
	\Assume{V}{\{1,\ldots,n\} \to \T}
	\AssumeIn{x}{\bigcap^n_{k=1} V_k}
	\Say{\Big(U,[5]\Big)}{\Elim \T(V)}
	{
		\sum U : \{1,\ldots,n\} \to \U \.
		\forall k \in n \.
		U_k(x) \subset V_k
	}
	\SayIn{W}{\bigcap^n_{k=1} U_k}{\U}
	\Conclude{[n.*]}{\Elim W \THM{SubsetIntersection} \Intro W}
	{
		W(x) \subset \bigcap^n_{k=1} V_k
	}
	\Derive{[5]}
	{
		\Elim \T
	}
	{
		\forall n \in \Nat \.
		\forall V : \{1,\ldots,n\} \to \T \.
		\bigcap^n_{k=1} V_k \in \T
	}
	\Conclude{[*]}
	{
		\Intro \TYPE{Topology} [1][3][4][5]
	}
	{
		\TYPE{Topology}(X)
	}
	\EndProof
	\\
	\DeclareFunc{uniformity}
	{
		\prod (X,\U) : \US \. \Unif(X)
	}
	\DefineNamedFunc{Uniformity}{}{\U_X}{\U}
	\\
	\DeclareFunc{UniAsTop}
	{
		\US \to \TOP
	}
	\DefineNamedFunc{UniAsTop}{X}{X}{\Big(X,\FUNC{uniformTopology}(\U_X)\Big)}
	\\
	\DeclareType{Stronger}{\prod_{X \in \SET} ?\Unif^2(X)}
	\DefineNamedType{(\U,\V)}{Stronger}{\U \ge \V}
	{
		\T_\V \subset \T_\U
	}
	\\
	\DeclareType{EquivalentUniformities}{\prod_{X \in \SET} ?\Unif^2(X)}
	\DefineNamedType{(\U,\V)}{EquivalenUniformities}{\U \cong \V}
	{
		\T_\V  = \T_\U
	}
	\\
	\DeclareType{BaseOfUniformity}
	{
		\prod X : \US \. ? \UB(X)
	}
	\DefineType{\B}{BaseOfUniformity}{ \langle \B \rangle = \U_X }
}
\Page{
	\DeclareFunc{discreteUniformuty}{\prod_{X \in \SET} \Unif(X)}
	\DefineFunc{discreteUniformity}{}{\Big\{ U \in X^2 : \Delta(X) \subset U \Big\}}
	\\	
	\DeclareFunc{codiscreteUniformuty}{\prod_{X \in \SET} \Unif(X)}
	\DefineFunc{codiscreteUniformity}{}{\{ X \times X \}}
	\\
	\Theorem{DiscreteUniformityTopology}
	{ 
		\forall X \in \SET \.
		\T(X,\FUNC{discreteUniformity}(X)) = 2^X	
	}
	\NoProof
	\\
	\Theorem{CodiscreteUniformityTopology}
	{ 
		\forall X \in \SET \.
		\T(X,\FUNC{codiscreteUniformity}(X)) = \{ \emptyset, X  \}	
	}
	\NoProof
	\\
	\DeclareFunc{relativeUniformity}
	{
		\prod X : \US \.  2^X \to \US
	}
	\DefineNamedFunc{relativeUniformity}{A}{A}
	{
		\Big( A,	\{ U \cap A^2 | U \in \U_X \} \Big)
	}
}
\newpage
\subsubsection{Symmetric Connectors}
Symmetric connectors are particularly nice.
Every uniformity has a base of symmetic connectors.
\Page{
	\DeclareType{\Sym}{\prod_{X \in \SET} ?\Connector(X)}
	\DefineType{U}{\Sym}{U^\top = U}
	\\
	\DeclareType{\SB}{\prod_{X \in \SET} ?\UB(X)}
	\DefineType{\B}{\SB}{\forall U \in \B \. \Sym(X,U)}	
	\\
	\Theorem{SymmetricBaseExists}
	{
		\forall X : \US \.
		\exists \B : \SB(X) \.
		\U_X = \langle \B \rangle
	}
	\Say{\B}
	{
		\{
			U \in \U_X : \Sym(X,U)
		\}
	}
	{
		?\U_X
	}
	\Say{[1]}{\Elim \TYPE{Filter}(X,\U_X)}
	{
		\exists \U_X
	}
	\Say{[2]}{\Elim \top \Intro \Sym}
	{
		\forall U \in \U_X \. \Sym\Big(X,U \cap U^\top\Big)
	}
	\Say{[3]}{[1][2]}{\exists \B}
	\Say{[4]}{\Elim \B\Elim \Sym(X) \Elim \cap \Intro \Sym(X) \Intro \B }
	{
		\forall U,V \in \B \. U \cap V \in \B
	}
	\Say{[5]}{\Intro \Filterbase[3][4]}
	{
		\Filterbase(X,\B)
	}
	\Say{[6]}{\Elim \B \Elim \Sym(X) \Intro \B }
	{
		\forall U \in \B \.  U^\top \in \B
	}
	\AssumeIn{U}{\B}
	\Say{\Big(V,[7]\Big)}
	{
		\Elim_3 \Unif(X,\U,U)
	}
	{
		\sum V \in \U \. V \circ V \subset U
	}
	\Say{[8]}{\THM{CompositionContainment}(X,V,V)[7]}
	{
		V \subset V \circ V \subset U
	}
	\Say{[9]}{\Elim \Sym(X,U)[8]}
	{
		V^\top \subset U
	}
	\Say{W}{V \cap V^\top}{\Sym(X)}
	\Say{[10]}{\Elim W \Elim V \Intro W}{W \in \U}
	\Say{[11]}{\Elim W [8][9] \Intro W}{ W \subset U}
	\Conclude{[U.*]}{[11][7] \Elim W \THM{IntersectionSubset} \Intro W}
	{
		W \circ W  \subset U
	}
	\Derive{[7]}
	{
		\Intro \SB [5][6]
	}
	{
		\SB(X,\B)
	}
	\Conclude{[*]}{\Elim \B [2] \THM{GeneratingFilterbase} \Intro \B}
	{
		\U_X = \langle \B \rangle
	}
	\EndProof
}
\newpage
\subsubsection{Neighborhoods}
Every base of uniformities has a corresponding base of neighborhoods.
\Page{
	\DeclareFunc{uniformityAssociatedBase}
	{
		\prod X : \US \.
		\prod_{x \in X}
		\BofU(X) \to 
		\NbhdBase(X,x)  
	}
	\DefineNamedFunc{uniformityAssociatedBase}
	{\B}{\widetilde{\B}_x}
	{
		\Big\{  
			\{ y \in X : \exists V \in \U_X : V(y) \subset U(x) \} \Big| 
		U \in \B  \Big\}
	}
	\AssumeIn{U}{\U_X}
	\Say{G}{\{ y \in X : \exists V \in \U_X : V(y) \subset U(x)  \}}{?X}
	\Say{[1]}{\THM{SelfSubset}\Big( X, U(x) \Big)}
	{
		U(x) \subset U(x)
	}
	\Say{[2]}{\Elim G [1] \Intro G}
	{
		x \in G
	}
	\Say{[3]}{\Elim \Connector \Elim G }
	{
		G \subset U(x)
	}
	\AssumeIn{g}{G}
	\Say{\Big(V,[3]\Big)}{\Elim G(g)}
	{
		\sum V : \Connector \.
		V(g) \subset U(x)
	}
	\Say{\Big(W,[4]\Big)}{\Elim \Unif(X,U)}
	{
		\sum W \in \U_X \.
		W \circ W \subset V 
	}
	\AssumeIn{y}{W(g)}
	\AssumeIn{z}{W(y)}
	\Say{[5]}{\Elim \Connector(W) \Elim y}
	{
		(g,y) \in W
	}
	\Say{[6]}{\Elim \Connector(W) \Elim z}
	{
		(y,z) \in W
	}
	\Say{[7]}{\Intro (W \circ W)[5][6]}{(g,z) \in W \circ W }
	\Conclude{[z.*]}{[7][4][3]}{z \in V(g) \subset U(x)}
	\Derive{[5]}{\Intro \subset }{ W(y) \subset U(x)  }
	\Conclude{[y.*]}{\Elim G [5]}
	{
		 y \in G
	}
	\DeriveConclude{[U.*]}{\Elim \FUNC{uniformTopology}}
	{
		G \in \T(X)
	}
	\Derive{[1]}{
		\Intro \forall \Intro \exists
	}
	{
		\forall U \in \B \.  
		\exists G \in \T(X) \.
		x \in G \subset U
	}
	\AssumeIn{N}{\U(x)}
	\Say{\Big( U,[2]\Big)}
	{
		\Elim \FUNC{uniformTopology}(X)\Elim N
	}
	{
		\sum U \in \U_X \.   x \in U(x) \subset N   
	}
	\Say{\Big( B,[3]\Big)}
	{
		\Elim \UB(\U_X,\B)
	}
	{
		\sum B \in \B \.   x \in  B(x) \subset U(x)   
	}
	\Conclude{\Big(G,[N.*]\Big)}{[1][2][3]}
	{
		\sum G \in \widetilde{\B}_x \. G \subset N 
	}
	\DeriveConclude{[*]}{\Intro \NbhdBase}
	{
		\NbhdBase(X,\widetilde{\B}_x)
	}
	\EndProof
	\\
	\DeclareType{\UNbhd}
	{
		\prod X : \US \. ?X \to ??X	
	}
	\DefineType{B}{\UNbhd}
	{
		\Lambda A \subset X \. 
		\exists U \in \U_X \.
		U(A) \subset B
	}
}\Page{
	\Theorem{UniformNeighborhoodIsANeighborhood}
	{
		\forall X : \US \.
		\forall A \subset X \. \NewLine \.
		\forall N : \UNbhd(X,A) \.
		\TYPE{Neighborhood}(X,A,N)
	}
	\Say{\Big(U,[1]\Big)}{\Elim \UNbhd(X,A,N)}
	{
		\sum U \in \U_X \. U(A) \subset B
	}
	\Say{[2]}{\Elim \U}
	{
		\forall a \in A \. 
		\exists O \in \U(a) \.
		O \subset U(a) \subset U(A) \subset B
	}
	\Say{O}{\bigcup_{a \in A} [2](a)}
	{
		\T(X)
	}
	\Say{[3]}{\Elim O}{ A \subset O \subset B  }
	\Conclude{[*]}{\Intro \TYPE{Neighborhood}[3]}
	{
		\TYPE{Neighborhood}(X,A,N)
	}
	\EndProof
	\\
	\Theorem{EveryCompactNeighborhoodIsUniform}
	{
		\forall X : \US \.
		\forall A : \Compacts(X) \. \NewLine \.
		\forall N : \Nbhd(X,A) \.
		\UNbhd(X,A,N)
	}
	\Say{\Big(O,[1]\Big)}{\Elim \Nbhd(X,A,N)}
	{
		\sum O \in \T(X) \.   A \subset O \subset N
	}
	\Say{\Big( U,[2]\Big)}{\Elim \FUNC{uniformTopology}[1]}
	{
		\prod  U' : A \to \U_X \. 
		\forall a \in A \.  U_a'(a) \subset O
	}
	\Say{\Big( U,[22]\Big)}{\Elim \Unif[2]}
	{
		\prod  U : A \to \U_X \. 
		\forall a \in A \.  U_a \circ U_a(a) \subset O
	}
	\Say{\Big(V,[3]\Big)}{\Elim \widetilde{B} [22]}
	{
		\prod  V : \prod_{a \in A} \U(a) \. 
		\forall a \in A \. V_a \subset U_a(a)
	}
	\Say{[4]}{\Elim V}{\TYPE{OpenCover}(X,A,\im V)}
	\Say{\Big(n,a,[5]\Big)}{\Elim \Compacts\Big(X,A,[4]\Big)}
	{
		\sum^\infty_{n=1} \sum \{1,\ldots,n\} \to A \. \TYPE{OpenCover}(X,A,\im V_a)
	}
	\Say{[6]}{[5][3]}{\TYPE{Cover}\Big(X,A,U_a(a)\Big)}
	\SayIn{W}{\bigcap^n_{k=1} U_{a_k}}{\U_X}
	\AssumeIn{w}{W(A)}
	\Say{\Big(b,[7] \Big)}{\Elim w}
	{
		\sum b \in A \. w \in W(b) 
	}
	\Say{\Big( k, [8]  \Big)}{\Elim \TYPE{Cover}[6](b)}
	{
		\sum^n_{k=1} b \in  U_{a_k}(a_k)  
	}
	\Say{[9]}{\Elim W [7](k)}{ (b,w) \in U_{a_k} }
	\Conclude{[w.*]}{\Elim \Connector(X,U_{a_k})[7][9]}
	{
		w \in U_{a_k} \circ U_{a_k}(a_k)
	}	
	\Derive{[7]}{\Intro \subset [22][2][1]}
	{   
			W(A) \subset \bigcup_{a \in A} U_a\circ U_a(a) \subset N
	}
	\Conclude{[*]}{\Intro \UNbhd [7]}
	{
		\UNbhd(X,A,N)
	}
	\EndProof
	\\
	\Theorem{ConentorAsNeighborhoodOfDiagonal}
	{
		\forall X : \US \.
		\forall U \in \U_X \.
		\Nbhd\Big(X^2,\Delta(X),U\Big)
	}
	\NoProof
}
\newpage
\subsubsection{Closures and regularity}
There is a special way to compute closures in uniform spaces.
Also, any separeted uniform space is regular
\Page{
	\Theorem{ClosureFormula}
	{
		\forall X  : \US \.
		\forall \B : \BofU(X) \.
		\forall A \subset X \.
		\overline{A} = \bigcap_{V \in \B} \bigcup_{a \in A} V(a)
	}
	\AssumeIn{x}{\overline{A}}
	\AssumeIn{V}{\B}
	\Say{[1]}{\THM{SymmetrocBaseExists}(x)}
	{
		\sum W : \Sym(X) \. W \subset V
	}
	\Say{[2]}{\THM{ClosureAltDef}(X,A)\Elim \widetilde{B}_x}{\exists A \cap W(x)}
	\SayIn{y}{\Elim \exists [2]}{W(x)}
	\Say{[3]}{\Elim \Sym(X,W) \Elim y}{x \in W(y)}
	\Say{[4]}{[1][3]}
	{
		x \in V(x)	
	}
	\Conclude{[x.*]}{\THM{UnionSubset}[4]}
	{
		x \in \bigcup_{a \in A} V(a)
	}
	\Derive{[1]}{\Intro \subset}
	{
			\overline{A} \subset \bigcup_{a \in A} V(a)
	}
	\AssumeIn{x}{{\overline{A}}^\c}
	\Say{\Big(U,[2]\Big)}
	{
			\THM{SymmetrocBaseExists}(X)
			\THM{ClosureAltDef}(X,A)\Elim \widetilde{B}_x
	}
	{
		\NewLine :	    
	    \sum U  : \Sym(X) \.  
		U(x) \cap A = \emptyset
	}
	\Say{\Big(V,[3]\Big)}
	{
		\Elim \UB(X,\B,U)
	}
	{
		\sum V \in \B \. V \subset U
	}
	\Assume{[4]}{x \in \bigcup_{a \in A} V(a)}
	\Say{\Big(a,[5])}{\Elim \FUNC{Union}\Elim x}
	{
		\sum_{a \in A} x \in V(a)  
	}
	\Say{[6]}{[3][5]}{x \in U(a)}
	\Say{[7]}{\Elim \Sym(X,U)[6]}{a \in U(x)}
	\Conclude{[8]}{[2][7]}
	{
		\bot	
	}
	\DeriveConclude{[x.*]}{\Elim \bot}{x \not \in \bigcup_{a \in A} V(a)}
	\DeriveConclude{[*]}{[1]\Intro \TYPE{SetEq}}
	{
		\overline{A} = \bigcap_{V \in \B} \bigcup_{a \in A} V(a)
	}
	\EndProof
}\Page{
	\Theorem{EveryUniformSpaceIsRegular}
	{
		\forall X : \US \.
		\TYPE{T0}(X) \Imply \TYPE{T3}(X)
	}
	\AssumeIn{x}{X}
	\AssumeIn{N}{\U(x)}
	\Say{\Big(U,[1]\Big)}
	{
		\Elim \FUNC{uniformTopology}(X)
	}
	{
		\sum U \in \U_X \. U \subset N 
	}
	\Say{\Big(W,[2]\Big)}
	{
		\Elim \US(X,U)
	}
	{
		\sum W \in \U_X \. W \circ W \subset U 
	}
	\Conclude{[x.*]}{\THM{ClosureFormula}(X,V(x))[2][1]}
	{
		\overline{V(x)} \subset \bigcup_{a \in V(x)} V(a)
		\subset  U(x) \subset  N
	}
	\Derive{[1]}{\Intro \forall}
	{
		\forall x \in X \.
		\forall N \in \U(x) \.
		\exists V \in \U(x) \.
		\overline{V} \subset N
	}
	\Conclude{[*]}
	{
		\THM{RegularT0IsT3}[1]
	}
	{
		\TYPE{T3}(X)
	}
	\EndProof
	\\
	\Theorem{UniformT3SpaceIntersectsToDiagonal}
	{
		\forall X : \US \.
		\TYPE{T3}(X) \iff
		\bigcap \U_X = \Delta(X) 
	}
	\NoProof
	\\
	\Theorem{CloppenByConnector}
	{
		\forall X : \US \.
		\forall A \subset X \. 
		\forall U : \Connector(X) \. \NewLine \.
		\bigcup_{a \in A} U(a) \subset A 
		\Imply
		\Clopen(X,A)
	}
	\Say{[1]}
	{
		\Lambda a \in A \. [0]\THM{UnionSubset}{A,U(a)}
	}
	{ 
		\forall a \in A \. U(a) \subset A	
	}
	\Say{[2]}{\Elim \FUNC{uniformTopology}[1]}
	{
		A \in \T(X)
	}
	\Say{[3]}{\THM{ClosureFormula}(X,A)[0]}
	{
		\overline{A} \subset A
	}
	\Say{[4]}{\Elim \FUNC{closure}(X,A)[3]}
	{
		\overline{A} = A
	}
	\Conclude{[*]}{\Intro \Clopen[2][4]}
	{
		\Clopen(X,A)	
	}
	\EndProof
	\\
	\Theorem{ConnectorsClopenAggregation}
	{
		\forall X : \US \.
		\forall U \in \U_X \.
		\forall A \subset \U_X \.
		\Clopen\left(X, \bigcup^\infty_{n=1} \bigcup_{a \in A} U^{\circ n}(a) \right)
	}
	\Say{B}{ \bigcup^\infty_{n=1} \bigcup_{a \in A} U^{\circ n}(a)}{?X}
	\Say{[1]}{
		\Elim B
		\Elim \Connector(X,U)
		\THM{CompositionContainment}(X)
		\Intro B
	}
	{
		\NewLine	:	
		U(B) = 
		U \bigcup^\infty_{n=1} U^{\circ n}(A) =
		\bigcup^\infty_{n=2} U^{\circ n}(A) =
		\bigcup^\infty_{n=1} U^{\circ n}(A) =
		B
	}
	\Conclude{[*]}{\THM{CloppenByConnector}[1]}{\Clopen(X,B)}
	\EndProof
}
\newpage
\subsubsection{Closed Connectors}
There are always a base of closed connectors
\Page{
	\Theorem{ConnectorClosureFormula}
	{
		\forall X : \US \.
		\forall \B : \BofU(X) \.
		\forall U \in \U_X \. \NewLine \. 
		\overline{U} = \bigcap \{ V \circ U \circ V : V \in \B  \}
	}
	\AssumeIn{(a,b)}{\overline{U}^\c}
	\Say{\Big(V,[1]\Big)}{\THM{ClosureAltDef}(X^2,U)\Elim (a,b)}
	{
		\sum V \in  \U(a,b) \.  V \cap U = \emptyset 
	}
	\Say{\Big(W,[2]\Big)}
	{
		\Elim \FUNC{productTopology}(X,V)		
		\Elim \FUNC{uniformTopology}(X)
		\Elim \widetilde{B} 
		\THM{SymmetricBaseExists}(X)
	}
	{
		\NewLine	:	
		\sum W : \Sym(X) \.
		W(a) \times W(b) \subset V
	}
	\Say{\Big(O,[3]\Big)}{\Elim \UB(X,\B,W)}
	{
		\sum O \in \B \. O \subset W
	}
	\Assume{[4]}{O \circ U \circ O(a,b)}
	\Say{\Big(x,y,[5]\Big)}{\Elim \Connector(X,O \And U)[4]}
	{
		\sum x,y \in X \.
		x \in O(a) \And y \in U(x) \And b \in O(y)
	}
	\Say{[6]}{[5.1][3]}{W(a,x)}
	\Say{[7]}{[5.2][3]\Elim \Sym (X,W)}{W(b,y)}
	\Say{[8]}{[5.3]}{U(x,y)}
	\Say{[9]}{\Intro \times [6][7][2]}{(x,y) \subset W(a) \times W(b) \subset V }
	\Conclude{[4.*]}{[9][1]}{\bot}		
	\Derive{[4]}{\Elim \bot}
	{
		\neg O \circ U \circ O (a,b) 
	}
	\Conclude{\Big[(a,b).*\Big]}{\THM{CompositionContainment}[4]}{\neg O(a,b)}
	\Derive{[1]}{\Intro \exists \Intro \forall }
	{
		\forall (a,b) \in \overline{V}^\c \. 
		\exists B \in \B \.
		\neg B \circ U \circ B(a,b) 	
	}
	\AssumeIn{(a,b)}{\overline{U}}
	\AssumeIn{B}{\B}
	\Say{\Big(W,[2]\Big)}
	{
		\THM{SymmetricBaseExists}(X,B)
	}
	{
		\sum W : \Sym(X) \.
		W \subset B
	}
	\Say{[3]}{\Elim \FUNC{productTopology}(X,V)		
		\Elim \FUNC{uniformTopology}(X)
		\Elim \widetilde{B} [2]}
	{
		\exists \Big(W(a) \times W(b)\Big) \cap U
	}
	\SayIn{(x,y)}{\Elim \exists [3]}
	{
		\Big(W(a) \times W(b)\Big) \cap U
	}
	\Conclude{[*]}{\Elim \Connector(X,U\And W)\Elim \Sym(X,W) }
	{
		(a,b) \in   W \circ U \circ W \subset B \circ U \circ B
	}
	\DeriveConclude{[*]}{[1]}
	{
			\overline{U} = \bigcap \{ V \circ U \circ V : V \in \B  \}
	}
	\EndProof
	\\
	\Theorem{UniformityTrisection}
	{
		\forall X : \US \. 
		\forall U \in \U_X \.
		\exists V : \Sym(X) \.
		V \circ V \circ V \subset U	
	}
	\Say{\Big(V,[1]\Big)}{\Elim \Unif(X,U)}
	{
		\sum V \in \U_X \.   V \circ V \subset U
	}
	\Say{\Big(W,[2]\Big)}{\Elim \Unif(X,V)}
	{
		\sum W \in \U_X \.   W \circ W \subset V
	}
	\Say{[3]}{\THM{MonotonicContainMent}(X)[1][2]}
	{
		W \circ W \circ W \subset  
		W \circ W \circ W \circ W \subset
		V \circ V \subset 
		U 
	}
	\Say{\Big(O,[4]\Big)}{\THM{SymmetricBaseExists}(X,W)}
	{
		\sum O  : \Sym(X) \. O \subset W
	}
	\Conclude{[5]}{[3][4]}{O \circ O \circ O \subset U}
	\EndProof
}\Page{
	\DeclareType{\CConnector}{\prod X : \US \. ?\U_X}
	\DefineType{U}{\CConnector}{\Closed(X^2,U)}	
	\\
	\Theorem{ClosedConnectorsBaseExists}
	{
		\forall X : \US \.
		\exists \B : \BofU(X) \. \NewLine \.
		\forall U \in \B \. 
		\CConnector \And \Sym(X,U)
	}
	\Say{\B}{\Big\{ U \in \U_X : \CConnector \And \Sym(X,U)\Big\}}
	{
		?\U_X
	}
	\Say{S}{\Big\{  U \in \U_X : \Sym(X,U) \Big\}}{\BofU(X)}
	\AssumeIn{U}{\U_X}
	\Say{\Big(V,[1]\Big)}{\THM{UniformityTrisection}(X,U)}
	{
		\sum V : \Sym(X) \. 
		V \circ V \circ V \subset U
	}
	\Say{[2]}{\THM{ConnectorClosureFormula}(X,V)[1]}
	{
		\overline{V} \subset V 	\circ V \circ V \subset U
	}
	\Say{[4]}{\THM{CinnectorClosureFormula}(X,V,S)}
	{
		\overline{V} = \bigcap\{  W \circ V \circ W  | W \in S \}
	}
	\Conclude{[U.*]}{\Elim \Sym(X,V)[4]\Intro \Sym }
	{
		\Sym\Big( X,\overline{V}\Big)	
	}
	\DeriveConclude{[*]}{\Intro \B \Intro \BofU}
	{
		\BofU\Big( X, \B \Big)
	}
	\EndProof
	\\
	\Theorem{UniformityTrisection2}
	{
		\forall X : \US \. 
		\forall U \in \U_X \. \NewLine \.
		\exists V : \Sym \And \CConnector(X) \.
		V \circ V \circ V \subset U	
	}
	\NoProof
}
\newpage
\subsubsection{Uniform Convergence}
Uniform convergence of continuous functions preserves continuoty.
\Page{
	\DeclareType{UniformlyConvergent}
	{
		\prod X \in \SET \.
		\prod Y : \US \.
		? \Big(\Net(X \to Y) \times (X \to Y)\Big)
	}
	\DefineNamedType{\Big((\Delta,f),g\Big)}{UniformlyConvergent}
	{f_{\delta \in \Delta} \rightrightarrows g }
	{
		\NewLine \iff		
		\forall U \in \U_Y \. 
		\exists \delta_0 \in \Delta \.
		\forall \delta \ge \delta_0 \.
		\forall x \in X \. 
		\Big(f_\delta(x),g(x)\Big) \in U
	}
	\\
	\Theorem{UniformConvergencePreservesContinuity}
	{
		\forall X \in \TOP \.
		\forall Y : \US \.
		\forall g : X \to Y \. \NewLine \.
		\forall (\Delta,f) : \Net\Big( \TOP(X,Y)\Big) \.
		f_{\delta \in \Delta} \rightrightarrows g
		\Imply
		g \in \TOP(X,Y)
	}
	\AssumeIn{x}{X}
	\AssumeIn{O}{\U(g(x))}
	\Say{\Big(U,[1]\Big)}{\Elim \FUNC{uniformTopology}\Big(Y,O,g(x)\Big)  }
	{
		\sum U \in \U_Y \.  U\Big( g(x) \Big) \subset O
	}
	\Say{\Big(V, [2] \Big)}
	{
			\THM{UniformityTrisection2}(X,U)	
	}
	{
		\NewLine :		
		\sum V : \Sym \And \CConnector(Y) \.
		V \circ V \circ V \subset U
	}
	\Say{\Big(\delta,[3]\Big)}{\Elim [0](V)}
	{
		\sum_{\delta \in \D} \forall t \ge \delta \. 
		\forall u \in X \.		
		 \Big( f_t(u), g(u) \Big) \in V
	}
	\Say{\Big( O', [4] \Big)}{\Elim \widetilde{B}(V,g(x))}
	{
		\sum O' \in \T(Y) \. g(x) \in O' \subset V\Big( g(x) \Big)
	}
	\Say{[5]}{\Elim \TOP(X,Y,f_\delta,O')}
	{
		f_\delta^{-1}(O') \in \T(X)
	}
	\AssumeIn{u}{f_\delta^{-1}(O')}
	\Say{[6]}{\Elim u \Elim \FUNC{preimage}}
	{
		f_\delta(u) \in O' \subset V\Big( g(x) \Big) 	
	}
	\Say{[7]}{[3](\delta,u)}{\Big( f_\delta(u),g(u) \Big) \in V}
	\Conclude{[x.*]}{\Elim \Sym(Y,V)[6][7][2][1]}
	{
		\Big(g(u),g(x)\Big) \in O
	}
	\DeriveConclude{[*]}
	{
		\THM{ContinuityIsLocal}(X,Y)
	}
	{
		g \in \TOP(X,Y)
	}
	\EndProof
}
\newpage
\subsubsection{Pseudo-Uniformities [$\infty$]}
There are more general concepts then uniformities.
Will be written on demand.
\subsubsection{v-Closure [$\infty$]}
This is about columns in connectors.
Will be written on demand.
\subsubsection{Transitive Uniformities[$\infty$]}
Transitive uniformities are very special.
Will be written on demand.
\newpage
\subsection{Coverering Uniformities[$\infty$]}
One can also define uniformities as families of covers.
Will be written on demand.
\newpage
\subsection{Uniform Continuity}
\subsubsection{Uniform Maps}
The notion of uniform map nicely generalizes frim metric spaces.
\Page{
	\DeclareType{\UC}{\prod X,Y : \US \. ?(X \to Y)}
	\DefineType{\varphi}{\UC}{
		\forall U \in \U_Y \. 
		\exists V \in \U_X \.
		(\varphi \times \varphi)(V) \subset U 
	}
	\\
	\Theorem{MetricUniformContinuity}
	{
		\forall X,Y \in \SMS \.
		\forall \varphi : X \to Y \.
		\UC(X,Y,\varphi)
		\iff \NewLine \iff
		\forall \varepsilon \in \Reals_{++} \.
		\exists \delta \in \Reals_{++} \.
		\forall a,b \in X \.
		d(a,b) < \delta \Imply  d\Big( \varphi(a), \varphi(b)\Big) < \varepsilon  
	}
	\NoProof
	\\
	\Theorem{UniformContinuousIsContinuous}
	{
		\forall X,Y : \US \.
		\forall \varphi : \UC(X,Y) \. \NewLine \.
		\varphi \in \TOP(X,Y)
	}
	\AssumeIn{x}{X}
	\AssumeIn{O}{\U\Big( \varphi(x)\Big)}
	\Say{\Big(U,[1]\Big)}{\Elim \FUNC{uniformTopology}\Big(Y, \varphi(x)\Big)}
	{
		\sum U \in \U_Y \. U\Big( \varphi(x) \Big)  \subset O
	}
	\Say{\Big(V,[2]\Big)}{\Elim \UC(X,Y,\varphi,U) }
	{
		\sum V \in \U_X \. (\varphi \times \varphi)(V) \subset U
	}
	\Say{\Big(W,[3]\Big)}{\Elim \widetilde{\B}_x(V)}
	{
		\sum W \in \U(x) \.   W \subset V(x)
	}
	\Conclude{[x.*]}{[1][2][3]}{\varphi(W) \subset O}
	\DeriveConclude{[*]}{\THM{ContinuityIsLocal}}
	{
		\varphi \in \TOP(X,Y)	
	}
	\EndProof	
	\\
	\Theorem{UniformityInclusionByIdentityContinuity}
	{
		\NewLine :
		\forall X \in \SET \.
		\U,\V : \Unif(X)  \.
		\U \le \V
		\iff
		\UC\Big((X,\V),(X,\U),\id\Big)	
	}
	\NoProof
	\\
	\DeclareType{\UHomeo}{\prod X,Y : \US \. ?\TYPE{Homeomorphis}(X,Y)}
	\DefineType{\varphi}{\UHomeo}
	{
		\UC(X,Y,\varphi) \And \UC(Y,X,\varphi^{-1})
	}
}\Page{
	\Theorem{UniformConvergencePreservesContinuity}
	{
		\forall X,Y \in \UNI \.
		\forall g : X \to Y \. \NewLine \.
		\forall (\Delta,f) : \Net\Big( \UNI(X,Y)\Big) \.
		f_{\delta \in \Delta} \rightrightarrows g
		\Imply
		g \in \UNI(X,Y)
	}
	\AssumeIn{V}{\U_Y}
	\Say{\Big(W,[1] \Big)}{\THM{UniformityTrisection}(Y,V)}
	{
		\sum W : \Sym(Y) \. W \circ W \circ W \subset V
	}
	\Say{\Big(\delta,[2]\Big)}{\Elim \TYPE{UniformConvergence}[0](W)}
	{
		\sum_{\delta \in \Delta} 
		\forall x \in X \.
		\forall t > \delta \.
		\Big(f_t(x),g(x)\Big) \in V W
	}
	\Say{\Big(U,[3]\Big)}{\Elim \UC(X,Y,f_\delta,W)}
	{
		\sum U \in \U_X \. f_\delta \times f_\delta(U) \subset W
	}
	\AssumeIn{(a,b)}{U}
	\Say{[4]}{[3](a,b)}{ \Big( f_\delta(a), f_\delta(b) \Big) \in W  }
	\Say{[5]}{[2](a,\delta)}{  \Big( f_\delta(a), g(a) \Big) \in W }
	\Say{[6]}{[3](b,\delta)}{  \Big( f_\delta(b), g(b) \Big) \in W  }
	\Conclude{\Big[(a,b).* \Big]}{[4][5][6] \Elim \Connector(Y,W)[1]}
	{
			\Big(g(a),g(b)\Big) \in  W \circ W \circ W \subset V
	}
	\DeriveConclude{[V.*]}{\Intro \subset}{(g \times g)(U) \subset V}
	\DeriveConclude{[*]}{\Intro \UC}{g \in \UNI(X,Y)}
	\EndProof 
}
\newpage
\subsubsection{Category of Uniform Spaces, Initial and Final Uniformity}
With uniform maps as morphism we have a bicomplete category. 
The notion of final and initial uniformity is very useful for constructing limits.
Note, that embedding in $\TOP$ reflects limits.
\Page{
	\DeclareFunc{UniformCategory}{\CAT}
	\DefineNamedFunc{UniformCategory}{}{\UNI}
	{
		\Big( \US, \UC, \circ, \id \Big)
	}
	\\
	\DeclareFunc{UniformSeparetadCategory}{\CAT}
	\DefineNamedFunc{UniformSeparatedCategory}{}{\UNIS}
	{
		\Big( \US \And \TYPE{T3}, \UC, \circ, \id \Big)
	}
	\\
	\DeclareFunc{initialUniformity}
	{
		\prod_{X,I \in \SET} \prod_{Y:I \to \UNI}  
		\left(\prod_{i \in I}  X \to Y_i  \right) \to \Unif(X) 	
	}
	\DefineNamedFunc{initialUniformity}{\phi}{\I_X(I,Y,\phi)}
	{
		\min \bigg\{ 
			\U : \Unif(X) : 
			\forall i \in I \. 
			\phi_i \in \UNI\Big((X,\U), Y \Big) 
		\bigg\}	
	}
	\\
	\Theorem{InitialUniformityExpression}
	{
		\forall X,I \in \SET \.
		\forall Y : I \to \UNI \.
		\forall \phi : \prod_{i \in I}  X \to Y_i \. \NewLine \.
		\I_X(I,Y,\phi) = 
			\bigg\langle\Big\{ 
				(\phi_i \times \phi_i)^{-1}(V) 
				\Big|
				    i \in I, V \in \U_{Y_i}
				\Big\} 
			\bigg\rangle_\F
	}
	\NoProof
	\\
	\DeclareFunc{productUniformSpace}
	{
		\prod_{I \in \SET} (I \to \UNI) \to \UNI
	}
	\DefineNamedFunc{productUniformSpace}{X}{\prod_{i \in I} X_i}
	{\left( \prod_{i \in I} X_i ,\I(I,X,\pi)\right)}
	\\
	\Theorem{UniformCategoryIsComplete}
	{
		\TYPE{Complete}(\UNI)
	}
	\NoProof
	\\
	\Theorem{LimitsOfUniAgreeWithTop}
	{
		\forall (I,X,\phi) : \TYPE{Diagram}(\UNI) \.
		\lim_{\UNI} (I,X,\phi) \cong_\TOP \lim_{\TOP} (I,X,\phi)
	}
	\NoProof
	\\
	\DeclareFunc{supremumUniformity}
	{
		\prod_{X,I \in \SET} \Big(I \to \Unif(X)\Big) \to \Unif(X)
	}
	\DefineNamedFunc{supremumUniformity}{\U}{\bigvee_{i \in I} \U_i}
	{
		\I_X\Big(I,(X,\U),\id\Big)	
	}
}\Page{
	\Theorem{InitialUniformityUniversalProperty}
	{
		\NewLine ::		
		\forall I,X \in \SET \.
		\forall Y : I \to \UNI \.
		\forall \phi : \left(\prod_{i \in I} X \to Y_i  \right) \. 
		\forall Q \in \UNI \.
		\forall \psi : Q \to X \. \NewLine \.
		\psi \in \UNI\bigg(Q,\Big(X, \I_X(I,Y,\phi) \Big)\bigg) \iff
		\forall i \in I \.  \psi \phi_i \in \UNI(Q,Y_i) 
	}
	\NoProof
	\\
	\DeclareFunc{finalUniformity}
	{
		\prod_{Y,I \in \SET} \prod_{X : I \to \UNI}  
		\left(\prod_{i \in I}  X_i \to Y  \right) \to \Unif(X) 	
	}
	\DefineNamedFunc{finalUniformity}{\phi}{\F_Y(I,X,\phi)}
	{
		\max \bigg\{ 
			\V : \Unif(Y) : 
			\forall i \in I \. 
			\phi_i \in \UNI\Big(X, (Y,\V) \Big) 
		\bigg\}	
	}
	\Say{\mathfrak{V}}{\bigg\{ 
			\V : \Unif(Y) : 
			\forall i \in I \. 
			\phi_i \in \UNI\Big(X_i, (Y,\V) \Big) 
		\bigg\}}{?\Unif(Y)}
	\Say{\U}{\bigvee_{V \in \mathfrak{V}} V}{\Unif(X)}
	\AssumeIn{i}{I}
	\Say{[1]}{\Elim \mathfrak{V}(i)}
	{\forall \V \in \mathfrak{V} \. \phi_i \in \UNI\Big(X_i, (Y,\V) \Big)}
	\Say{[2]}{\Elim \U}
	{
		\U = \I_Y\Big(\mathfrak{V},\Lambda \V \in \mathfrak{V} \. (Y,V), \id \Big)
	}
	\Conclude{[i.*]}{\THM{InitialUniformityUniversalProperty}[1][2]}
	{
		\phi_i \in \UNI\Big(X_i, (Y,\U)\Big)
	}
	\Derive{[1]}{\Intro \mathfrak{V}}{\U \in \mathfrak{V}}
	\Conclude{[*]}{\Elim \U [1]\Intro \F_Y(I,X,\phi)}
	{
		\U = \F_Y(I,X,\phi)	
	}
	\EndProof
	\\
	\Theorem{FinalUniformityUniversalProperty}
	{
		\NewLine ::		
		\forall I,Y \in \SET \.
		\forall X : I \to \UNI \.
		\forall \phi : \left(\prod_{i \in I} X_i \to Y  \right) \. 
		\forall Z \in \UNI \.
		\forall \psi : Y \to Z \. \NewLine \.
		\psi \in \UNI\bigg(\Big(Y, \F_Y(I,X,\phi) \Big), Z\bigg) \iff
		\forall i \in I \.  \phi_i\psi \in \UNI(X_i,Z) 
	}
	\NoProof
	\\
	\Theorem{UniformCategoryIsBicomplete}
	{
		\TYPE{Bicomplete}(\UNI)
	}
	\NoProof
	\\
	\DeclareFunc{quotientUniformity}
	{
		\prod_{X \in \UNI} \TYPE{Equivalence}(X) \to \UNI
	}
	\DefineNamedFunc{quotientUniformity}{\sim}{\frac{X}{\sim}}
	{
		\left( \frac{X}{\sim}, \F(\star,\star \mapsto X,\pi_\sim) \right)
	}
}\Page{
	\DeclareFunc{infimumUniformity}
	{
		\prod_{I,X \in \SET}\Big(I \to \Unif(X)\Big) \to \Unif(X)
	}
	\DefineNamedFunc{infimumUniformity}{\U}{\bigwedge_{i \in \I} \U_i}
	{
		\bigg( X, \F_X\Big(I,(X,\U),\id\Big) \bigg)
	}
	\\
	\Theorem{UniformSeparatedCategoryIsBicomplete}
	{
		\TYPE{Bicomplete}(\UNIS)
	}
	\NoProof
	\\
	\Theorem{MinUniformityBase}
	{
		\NewLine ::		
		\forall X \in \SET \.
		\forall \U,\V : \Unif(X) \.
		\BofU\Big(X,\U \wedge \V, \{ U \circ V | U \in \U, V \in \V \} \Big)
	}
	\NoProof
}
\newpage
\subsubsection{Uniform Covers}
Notion of uniform covers is useful for identifying uniform maps.
\Page{
	\DeclareType{\UniCov}{\prod_{X \in \UNI} ?\Cover(X)}
	\DefineType{\C}{\UniCov}{
		\exists U \in \U_X \. 
		\forall x \in X \. 
		\exists C \in \C \.
		U(x) \subset C 
	}
	\\
	\Theorem{UniformContinuityByPreimages}
	{
		\forall X,Y \in \UNI \.
		\forall \varphi : X \to Y \. \NewLine \.
		\varphi \in \UNI(X,Y) 
		\iff
		\forall V \in \U_Y \. 
		\UniCov\bigg(X,\Big\{ \varphi^{-1}\Big( V(y) \Big) \Big|   y \in Y \Big\} \bigg)
	}
	\Assume{[1]}{\varphi \in \UNI(X,Y)}
	\AssumeIn{V}{\U_Y}
	\Say{\C}{\Big\{ \varphi^{-1}\Big( V(y) \Big) \Big|   y \in Y \Big\}}{??X}
	\Say{\Big(U,[2]\Big)}{\Elim \UC(X,Y,\varphi)[1]}
	{
		\sum U \in \U_X \.  (\varphi \times \varphi)(U) \subset V
	}
	\AssumeIn{x}{X}
	\Say{[3]}{[2](x)}
	{
		\varphi\Big( U(x) \Big) \subset  V\Big( \varphi(x) \Big)
	}
	\Conclude{[x,*]}{ \varphi^{-1}[3]}
	{ 
		U(x) \subset \varphi^{-1}\Big( V\big(\varphi(x)\big) \Big)
	}
	\DeriveConclude{[1.*]}{\Intro \UniCov}{\UniCov(X,\C)}
	\Derive{[1]}{\Intro \Imply}
	{
		\varphi \in \UNI(X,Y) 
		\Imply
		\forall V \in \U_Y \. 
		\UniCov\bigg(X,\Big\{ \varphi^{-1}\Big( V(y) \Big) \Big|   y \in Y \Big\} \bigg)
	}
	\Assume{[2]}
	{
		\forall V \in \U_Y \. 
		\UniCov\bigg(X,\Big\{ \varphi^{-1}\Big( V(y) \Big) \Big|   y \in Y \Big\} \bigg)
	}
	\AssumeIn{V}{\U_Y}
	\Say{\C}{\Big\{ \varphi^{-1}\Big( V(y) \Big) \Big|   y \in Y \Big\}}{\UniCov(X)}
	\Say{\Big(U,[3]\Big)}{\Elim \UniCov(X)}
	{
		\forall x \in X \. 
		\exists C \in \C \.
		U(x) \subset C 
	}
	\Conclude{[V.*]}{\varphi\Big(\Elim \C [3]\Big)\Intro \times}
	{
		(\varphi \times \varphi)(U) \subset V
	}
	\DeriveConclude{[2.*]}{\Intro \UC}{\varphi \in \UNI(X,Y)}
	\DeriveConclude{[*]}{\Intro (\iff) [1]}
	{
		\varphi \in \UNI(X,Y) 
		\iff
		\forall V \in \U_Y \. 
		\UniCov\bigg(X,\Big\{ \varphi^{-1}\Big( V(y) \Big) \Big|   y \in Y \Big\} \bigg)
	}
	\EndProof	
}
\newpage
\subsubsection{Compact Uniform Spaces}
All continuous maps with compact uniform domains are uniformly continuous.
\Page{
	\Theorem{EveryCoverOfCompactSpaceIsUniform}
	{
		\NewLine ::		
		\forall X \in \UNI \.
		\Compact(X) \Imply 
		\forall \O : \OpenC(X) \.
		\UniCov(X,\O)
	}
	\Say{\Big(\O',[1]\Big)}{\Elim \Compact(X,\O)}
	{
		\sum \O' : \Finites(\O) \. \Cover(X,\O')
	}
	\Say{\Big(V,[2]\Big)}{\Elim \FUNC{uniformTopology}[1]}
	{
		\sum V : \prod_{O \in \O'} \prod_{x \in O} \U_X \.
		\forall O \in \O' \. \forall x \in O \.    
		V_{O,x}(x) \subset O 
	}
	\Say{\Big(W,[21]\Big)}{\THM{SymmetricBaseExists}(X,V)}
	{
		\NewLine :		
		\sum W : \prod_{O \in \O'} \prod_{x \in O}  \Sym(X) \.
		\forall O \in \O' \. \forall x \in O \.    
		W_{O,x} \circ W_{O,x} \subset U_{O,x}
	}
	\Say{\Big(O',[3]\Big)}{\Elim \widetilde{\mathcal{B}}(W)[2]}
	{
		\sum O' : \prod_{O \in \O'} \prod_{x \in O}  x \in O'_{O,x} \subset W_{O,x}(x)  
	}
	\Say{[4]}{\Elim \Cover(X,\O') [1][3]}{\OpenC(X,\im O')}
	\Say{\Big( n,x,\theta, [5]\Big)}{\Elim \Compact(X,\im O')}
	{
		\sum_{n=1}^\infty 
		\sum x : \{1,\ldots,n\} \to X \. \NewLine \.
		\sum \theta : \{1,\ldots,n\} \to \O' \.
		\Big(\forall k \in \{1,\ldots,n\} \. x_k \in \theta_k \Big) \And
		\OpenC(X,\im O'_{\theta,x})
	}
	\Say{[6]}{[5][3]}{\Cover(X,\im W_{\theta,x}(x))}
	\SayIn{U}{\bigcap^n_{k=1} W_{\theta_k,x_k}}{\U_X}
	\AssumeIn{y}{X}
	\Say{\Big(k,[7]\Big)}{\Elim \Cover [6]}
	{
		\sum^n_{k=1} y \in W_{\theta_k, x_k}(x_k)  
	}
	\AssumeIn{u}{U(y)}
	\Say{[8]}{\Elim u \Elim U (k)}{u \in W_{\theta_k,x_k}(y) }
	\Conclude{[y.*]}{\Elim \Sym(X,W_{\theta_k,x_k})[21][2]}
	{
		u \in W_{\theta_k,x_k} \circ W_{\theta,x_k}(x_k) \subset 
		V_{\theta_k,x_k} \subset  \theta_k	
	}
	\DeriveConclude{[*]}{\Intro \UniCov}
	{
		\UniCov(X,\O)
	}
	\EndProof
	\\
	\Theorem{CompactUniformContinuity}
	{
		\forall X,Y \in \UNI \.
		\Compact(X) \Imply 
		\TOP(X,Y) = \UNI(X,Y)
	}
	\NoProof
	\\
	\Theorem{CompactUniformityEquivalence}
	{
		\forall X,Y \in \UNI \.
		\Compact(X \And Y)
		\Imply
		(X \cong_\TOP Y \Imply X \cong_\UNI Y)
	}
	\NoProof
}
\newpage
\subsection{Completeness}
As metric spaces, the uniform spaces can be complete.
\subsubsection{Cauchy Filterbase}
Cauchy Filterbases are natural generalizations of Cauchy sequences.
\Page{
	\DeclareType{\CF}{\prod_{X \in \UNI} ?\Filterbase(X)}
	\DefineType{\F}{\CF}{\forall U \in \U_X \. \exists F \in \F \. F \times F \subset U }
	\\
	\Theorem{EveryConvergentFilterbaseIsCauchy}
	{
		\NewLine ::		
		\forall X \in \UNI \.
		\forall \F : \CFilterbase(X) \.
		\CF(X,\F)
	}
	\Say{\Big(x,[1] \Big)}{\Elim \CFilterbase(X,\F)}
	{
		\sum_{x \in X} \lim \F = x
	}
	\AssumeIn{U}{\U_X}
	\Say{(V,[3])}{\THM{SymmetricBaseExists}(X)}
	{
		\sum V : \Sym(X) \. V \circ V \subset X
	}
	\Say{\Big(O,[4]\Big)}{\Elim \widetilde{\B}_x(V)}
	{
		\sum O \in \U(x) \.  O \subset V(x)
	}
	\Say{\Big( F ,[5] \Big)}{\Elim \lim [1](O)}
	{
		\sum F \in \F \.  F \subset O
	}
	\Conclude{[U.*]}{[4][5]\THM{ProductSubset}(X)}
	{F \times F \subset V \circ V \subset U}
	\DeriveConclude{[*]}{\Intro \CF}{\CF(X,\F)}
	\EndProof
	\\
	\Theorem{UniformMapsPreserveCauchyFilters}
	{
		\NewLine ::		
		\forall X,Y \in \UNI \.
		\forall \varphi \in \UNI(X,Y) \.
		\forall \F : \CF(X) \.
		\CF\Big(Y,\varphi(\F)\Big)
	}
	\AssumeIn{V}{\U_Y}
	\Say{\Big(U,[1]\Big)}{\Elim \UC(X,Y,f,V)}
	{
		\sum U \in \U_X \. \varphi \times \varphi(U) \subset V
	}
	\Say{\Big(F, [2]\Big)}{\Elim \CF(X,\F,U)}
	{
		\sum F \in \F \. F \times F \subset U
	}
	\Conclude{[V.*]}{[1][2]}{\varphi(F) \times \varphi(F) \subset V}
	\DeriveConclude{[*]}{\Intro \CF}{\CF\Big(Y,\varphi(\F)\Big)}
	\EndProof
}\Page{
	\Theorem{CauchyClustersAreLimits}
	{
		\forall X \in \UNI \.
		\forall	\F : \CF(X) \.
		\forall x : \Cluster(X,\F) \.
		x \in \lim \F
	}
	\AssumeIn{O}{\U(x)}
	\Say{\Big(U,[1]\Big)}{\Elim \FUNC{uniformTopology}(U,O)}
	{
		\sum U \in \U_X \.   U(x) \subset O
	}
	\Say{\Big(V,[2]\Big)}{\THM{ClosedBaseExists}(X,U)}
	{
		\sum V : \CConnector(X) \. V \subset U
	}
	\Say{\Big( F, [3] \Big)}{\Elim \CF(X,\F,V)}
	{
		\sum F \in \F \.  F \times F \subset V
	}
	\AssumeIn{f}{F}
	\Say{[5]}{\Elim \Cluster(X,\F,x,F)\Elim \CConnector(X,V)[3]}
	{
		(x,f) \in \overline{F} \times F \subset V
	}
	\Conclude{[f.*]}{[2]\Elim \Connector(X,U)[1]}{f \in O}
	\DeriveConclude{[O.*]}{\Intro \subset}{F \subset O}
	\DeriveConclude{[*]}{\Intro \lim}{x \in \lim \F}
	\EndProof
	\\
	\Theorem{SupUniformityCauchyFilterbase}
	{
		\NewLine ::		
		\forall X,I \in \SET \.
		\forall \U : I \to \Unif(X) \. 
		\forall \F : \Filterbase(X) \. \NewLine \.
		\CF\left(\left( X, \bigvee_{i \in I} \U_i \right),\F\right)
		\iff
		\forall i \in I \.		
		\CF\Big((X,\U_i),\F\Big)	
	}
	\Assume{[1]}{\CF\left(\left( X, \bigvee_{i \in I} \U_i \right),\F\right)}
	\Say{[2]}{\Elim \bigvee_{i \in I} \U_i}
	{
		\forall i \in I \. 
		\id \in \UNI\left(\left( X, \bigvee_{i \in I} \U_i \right),(X,\U_i)\right)
	}
	\Conclude{[1.*]}{\THM{UniformMapsPreserveCauchyFilters}[1][2]}
	{
		\forall i \in I \.		
		\CF\Big((X,\U_i),\F\Big)
	}
	\Derive{[1]}{\Intro \Imply}
	{
		\CF\left(\left( X, \bigvee_{i \in I} \U_i \right),\F\right)
		\Imply
		\forall i \in I \.		
		\CF\Big((X,\U_i),\F\Big)
	}
	\Assume{[1]}{\forall i \in I \. \CF\Big((X,\U_i),\F\Big)}
	\AssumeIn{U}{\bigvee_{i \in I}\U_i}
	\Say{\Big(n,i,V,[2]\Big)}{
		\Elim \bigvee_{i \in I}\U_i (U)
		\THM{InitialUnifomityExpression}
	}
	{
		\sum^\infty_{n=0}  \sum i : \{1,\ldots, n\} \to I \. 
		\sum V : \prod_{k=1}^\infty \U_{i_k} \.		
		\bigcap^n_{k=1} V_k \subset U
	}
	\Say{\Big( F, [3]\Big)}{\Lambda k \in \{1,\ldots,n\} \. \Elim \CF(X,\U_{i_k},V_k)}
	{
		\sum F : \{1,\ldots,n\} \to \F \.
		\forall k \in \{1,\ldots,n\} \. 
		F_k \times F_k \subset V_k
	}
	\SayIn{\Big(G,[4]\Big)}{\Elim \Filterbase(X,F)}{
		\sum G \in \F \. G \subset \bigcap_{k=1}^n F_k
	}
	\Conclude{[U.*]}{[4][3]\THM{SubsetIntersect}[2]}{G \times G \subset U }
	\DeriveConclude{[2.*]}{\Intro \CF}
	{
		\CF\left(\left( X, \bigvee_{i \in I} \U_i \right),\F\right)
	}
	\DeriveConclude{[*]}{\Intro \iff [*]}
	{
		\CF\left(\left( X, \bigvee_{i \in I} \U_i \right),\F\right)
		\iff
		\forall i \in I \.		
		\CF\Big((X,\U_i),\F\Big)
	}
	\EndProof
}
\newpage
\subsubsection{Complete Uniform Spaces}
Now complete spaces are those, where all Cauchy filters are converging
\Page{
	\DeclareType{\CUS}{?\UNI}
	\DefineType{X}{\CUS}{\forall \F : \CF(X) \. \CFilterbase(X,\F)}
	\\
	\DeclareType{\CSeq}{\prod_{X \in \UNI} \Nat \to X}
	\DefineType{x}{\CSeq}{
		\CF\Big( X,  
			\big\{ 
				\{ x_n, n \ge m \} 
				\big| 
				m \in \Nat
			\big\}
		\Big)  
	}
	\\
	\DeclareType{\SCUS}{?\UNI}
	\DefineType{X}{\SCUS}{\forall x : \CSeq(X) \. \Convergent(X,x)}
	\\
	\Theorem{IsomorphicCompleteness}
	{
		\forall X,Y \in \UNI \.
		X \cong_\UNI Y 
		\Imply \NewLine \Imply
		\Big( \CUS(X) \iff \CUS(Y) \Big)
	}
	\NoProof
	\\
	\Theorem{IsomorphicSequenceCompleteness}
	{
		\forall X,Y \in \UNI \.
		X \cong_\UNI Y 
		\Imply \NewLine \Imply
		\Big( \SCUS(X) \iff \SCUS(Y) \Big)
	}
	\NoProof
	\\
	\Theorem{LargerEqUniformityIsComplete}
	{
		\forall X : \CUS \.
		\forall \V \ge \U_X \. \NewLine \.
		\V \cong \U_X \Imply  \CUS(X,\V)
	}
	\NoProof
	\\
	\Theorem{LargerEqUniformityIsComplete}
	{
		\forall X : \CUS \.
		\forall \V \ge \U_X \. \NewLine \.
		\V \cong \U_X \Imply  \CUS(X,\V)
	}
	\NoProof
	\\
	\Theorem{LargerEqUniformityIsSeqComplete}
	{
		\forall X : \SCUS \.
		\forall \V \ge \U_X \. \NewLine \.
		\V \cong \U_X \Imply  \SCUS(X,\V)
	}
	\NoProof
}\Page{
	\Theorem{ClosedOfCompleteIsComplete}
	{
		\NewLine ::		
		\forall X : \CUS \.
		\forall A : \Closed(X) \.
		\CUS(A)  
	}
	\Assume{\F}{\CF(A)}
	\Say{\F'}{\{ F \cup B | F \in \F, B \subset X \}  }{\Filter(X)}
	\AssumeIn{U}{\U_X}
	\SayIn{U'}{U \cap A \times A}{\U_A}
	\Say{\Big(F,[1]\Big)}
		{
			\Elim \CF(A,\F)
			\Elim U' 
			\THM{IntersectionIsSubset}(X,U',U,A\times A)}
		{ 
			\sum_{F \in \F}  F \times F \subset U' \subset U		
		}
	\Conclude{[U.*]}{\Elim F \THM{UnionWithEmpty}(X)\Intro \F'}
	{
			F \in \F'
	}
	\Derive{[1]}{\Intro \CF}{\CF\Big(X,\F'\Big)}
	\SayIn{x}{\Elim \CUS(X,\F')}{\lim \F'}
	\Say{[2]}{\Elim \Closed(X,A)\Elim \F' \THM{ClosedFilterLimit}}{x \in A}
	\Conclude{[\F.*]}{\Elim \F' \Elim x [2]}{x \in \lim \F}
	\DeriveConclude{[*]}{\Intro \CUS}{\CUS(A)}
	\EndProof
	\\
	\Theorem{CompleteProductTHM}
	{
		\forall I \in \SET \.
		\forall X : I \to \UNI \.
		\Big( \forall i \in I \. \exists X_i \Big)
		\Imply \NewLine \Imply
		\CUS\left(\prod_{i \in I} X_i\right) 
		\iff
		\forall i \in I \. \CUS(X_i)
	}
	\Assume{[1]}{\CUS\left(\prod_{i \in I} X_i\right) }
	\AssumeIn{i}{I}
	\Say{J}{I \setminus \{i\}}{?I}
	\Say{x}{\LOGIC{Choice}\Big(J,X,[0]\Big)}{\prod_{j \in J} X_j}
	\Assume{\F}{\CF(X_i)}
	\Say{\F'}{
		\left\{ 
			F \times_i  \prod_{j \in J} \{x_j\} \Bigg|  F \in \F \right\}}
	{
		\Filterbase\left( \prod_{i \in I} X_i \right)
	}
	\Say{\phi}{\Lambda u \in X_i \. \Lambda k \in I \.  \If i == k \Then u \Else x_k}
	{
		X_i \to \prod_{i \in I} X_i
	}
	\Say{[1]}{\Elim \phi \Elim \F'}{\F' = \phi_*(\F)}
	\Say{[2]}{\Elim \phi \Elim \pi_i \Intro \id \Elim \CAT(\UNI)}
	{\phi \pi_i = \id \in \UNI(X_i,X_i)}
	\Say{[3]}{\Lambda j \in J \. \Elim \phi \Elim \pi_j \Elim \Unif(\U_{X_j})  }
	{
		\forall j \in J \. \phi \pi_j = x_j \in \UNI(X_i,X_j) 
	}
	\Say{[4]}{
		\Elim \prod_{i \in I} X_i 
		\THM{InitialUniformityUniversalProperty}[2][3]
	}
	{
		\phi \in \UNI\left(X_i, \prod_{i \in I} X_i\right)
	}
	\Say{[5]}{\THM{UniformMapsPreserveCauchyFilters}[1][4]}
	{
		\CF\left( \prod_{i \in I} X_i, \F'\right)
	}
	\SayIn{f}{\lim \F'}{\prod_{i \in I} X_i}
}\Page{
	\AssumeIn{O}{\U(f_i)}
	\SayIn{O'}{O \times_i \prod_{j \in J} X_j}{\U(f)}
	\Say{\Big( F', [6] \Big)}{\Elim \lim [5](O')}
	{
		\sum F' \in \F' \. F' \subset O'
	}
	\Say{F}{\pi_i(F')}{?X}
	\Say{[7]}{\Elim F \Elim \F'}{F \in \F}
	\Conclude{[O.*]}{\Elim F \Elim O' [6]}{F \subset O}
	\DeriveConclude{[1.*]}{\Intro \lim}{f_i \in \lim \F }
	\Derive{[1]}{\Intro \Imply}
	{
		\CUS\left(\prod_{i \in I} X_i\right) 
		\Imply
		\forall i \in I \. \CUS(X_i)
	}
	\Assume{[2]}{\forall i \in I \. \CUS(X_i)}
	\Assume{\F}{\CF\left( \prod_{i \in I} X_i \right)}
	\Say{[3]}
	{
		\Lambda i \in I \.		
		\THM{UniformMapsPreserveCauchyFilters}
		\left(\prod_{i \in I} X_i, X_i,\pi_i,\F\right)
	}
	{
		\NewLine :		
		\forall i \in \I \. \CF\Big(X_i, \pi_i(\F)\Big)
	}
	\SayIn{f}{\Lambda i \in I \. \lim \pi_i(\F) }
	{
		\prod_{i \in I} X_i
	}
	\Assume{O}{\U(f)}
	\Say{\Big(J,E,[4]\Big)}{\Elim \FUNC{productTopology}(I,X,O)}
	{
		\sum J : \Finite(I) \.
		\sum E : \prod_{j \in J} \T(X_j) \. 
		f \in \prod_{j \in J} E_j \times \prod_{j \in J^\c} X_j \subset O
	}
	\Say{\Big(F,[5]\Big)}{\Elim f (E)}
	{
		\sum F \in \prod_{j \in J} \pi_j(\F) \. F_j \subset E_j 
	}
	\Say{[6]}{
		\Lambda j \in J \. 
			\Elim F 
			\Elim \pi_j 
			\Elim \Filter\left( \prod_{i \in I} X_i,\F\right)
	}
	{
		\forall j \in J \. F_j \times_j \prod_{i \in \{j\}^\c} X_i \in \F
	}
	\Say{[7]}{
		 \Elim \Filter\left( \prod_{i \in I} X_i,\F\right)[6]
	}
	{
		\prod_{j \in J} F_j \times \prod_{j \in J^\c} X_j \in \F
	}
	\Conclude{[O.*]}{[7][5][4]}
	{
		\prod_{j \in J} F_j \times \prod_{j \in J^\c} X_j \subset  O 	
	}
	\DeriveConclude{[2.*]}{\Intro \lim}{f \in \lim \F}	
	\DeriveConclude{[*]}{ \Intro \iff [1] }
	{
		\CUS\left(\prod_{i \in I} X_i\right) 
		\iff
		\forall i \in I \. \CUS(X_i)
	}
	\EndProof	
}
\newpage
There is also an extension theorem.
\subsubsection{Extension Theorem}
\Page{
	\Theorem{UCExtensionTheorem}
	{
		\forall X \in \UNI \.
		\forall Y : \CUS \. 
		\forall D  : \Dense(X) \.
		\forall \phi \in \UNI(D,Y) \. \NewLine \.
		\exists \Phi \in \UNI(X,Y) \.
		\Phi_{|D} = \phi
	}
	\AssumeIn{x}{X}
	\Say{\F_x}{ \Big\{ U(x) \cap D \Big| U \in \U_X \Big\}  }{??D}
	\Say{[1]}{\Elim \F_x \Elim \Unif(X)}{ \Filterbase(X,\F_x)}
	\Say{[2]}{\Elim \F_x \Elim \FUNC{uniformTopology}(X) \Elim \Dense(X,D)}
	{
		\lim \F_x = x
	}
	\Say{[3]}{\THM{ConvergenrFilterbaseIsCauchy}[2]}
	{
		\CF(X,\F_x)
	}
	\Say{[4]}{\THM{UniformBasePreservesCauchyFilters}\Big( X,Y, \phi\Big)[3]}
	{
		\CF\Big(Y, \phi(\F_x)\Big)
	}
	\ConcludeIn{\Phi(x)}{\Elim \CUS\Big(Y,\phi(\F_x)\Big)\Elim \CFilterbase}
	{
		  \lim \phi(\F_x)
	}
	\Derive{\Phi}{\Intro(\to)}{X \to Y}
	\Say{[1]}{\Elim \Phi \THM{ContinuousPreserveLimits}}
	{
		\Phi_{|D} = \phi
	}
	\AssumeIn{V}{\U_Y}
	\Say{\Big(W,[2]\Big)}{\THM{UniformityTrisection2}(Y,V)}
	{
		\sum W \in \Sym \And \CConnector(Y) \. \NewLine \.
		W \circ W \circ W \subset  V
	}
	\Say{\Big(U,[3]\Big)}
	{
		\Elim \UNI(D,X,\phi,W)	
	}
	{
		\sum U \in \U_D \.  (\phi \times \phi)(U) \subset W
	}
	\Say{\Big(U',[4]\Big)}{\Elim \FUNC{subsetUniformity}(X,D,U)}
	{
		\sum U' \in \U_X \. U = U' \cap (D \times D)
	}
	\Say{\Big(O,[5]\Big)}{\THM{UniformityTrisection2}(X,U')}
	{
		\sum O  : \Sym \And \CConnector(X) \. \NewLine \.
		O \circ O \circ O \subset U'
	}
	\AssumeIn{(a,b)}{O}
	\SayIn{r}{\Elim \Dense(X,D) \Elim \widetilde{\B}_a(O)}{O(a) \cap D}
	\SayIn{s}{\Elim \Dense(X,D) \Elim \widetilde{\B}_b(O)}{O(b) \cap D}
	\Say{[6]}{ \Elim \Sym(Y,O) [5]}
	{
		(r,s) \in U'
	}
	\Say{[7]}{[6][4][4]}{\Big( \phi(r), \phi(s) \Big) \in U'}
	\Assume{N}{\U\Big( \Phi(a) \Big)}
	\Say{\Big(F,[8]\Big)}{\Elim \Phi(N) }
	{
		\sum F \in \F_a \   \phi(F) \subset N 
	}
	\Say{\Big(I,[9]\Big)}{\Elim \F_a(F)}
	{
		\sum I \in \U_X \.  F = I(a) \cap D
	}
	\SayIn{d}{\Elim \Unif (\U_X) \Elim \Dense(X,D) }
	{
		I(a) \cap O(a) \cap D
	}
	\Say{[10]}{\Elim d [9][8]}
	{
		\phi(d) \in N
	}
	\Say{[11]}{\Elim d \Elim r \Elim \Sym(X,O) [5]}{ (d,r) \in O \circ O \subset U' }
	\Say{[12]}{[11][3][4]}{\Big( \phi(d), \phi(r)  \Big) \in W}
	\Conclude{[13]}{\Intro \exists  [12][10]}{\exists N \cap W\Big(\phi(r)\Big)}
	\Derive{[8]}{\Elim \CConnector(X,O)}
	{
		\Phi(a) \in W\Big( \phi(r) \Big)	
	}
}\Page{
	\Assume{N}{\U\Big( \Phi(b) \Big)}
	\Say{\Big(F,[9]\Big)}{\Elim \Phi(N) }
	{
		\sum F \in \F_b \   \phi(F) \subset N 
	}
	\Say{\Big(I,[10]\Big)}{\Elim \F_b(F)}
	{
		\sum I \in \U_X \.  F = I(b) \cap D
	}
	\SayIn{d}{\Elim \Unif (\U_X) \Elim \Dense(X,D) }
	{
		I(b) \cap O(b) \cap D
	}
	\Say{[11]}{\Elim d [10][9]}
	{
		\phi(d) \in N
	}
	\Say{[12]}{\Elim d \Elim s \Elim \Sym(X,O) [5]}{ (d,s) \in O \circ O \subset U' }
	\Say{[13]}{[12][3][4]}{\Big( \phi(d), \phi(s)  \Big) \in W}
	\Conclude{[14]}{\Intro \exists  [13][10]}{\exists N \cap W\Big(\phi(s)\Big)}
	\Derive{[9]}{\Elim \CConnector(X,O)}
	{
		\Phi(b) \in W\Big( \phi(s) \Big)	
	}
	\Conclude{\Big[(x,y).*\Big]}{\Elim \Sym(Y,W)[7][8][9]}
	{
		\Big(\Phi(a),\Phi(b)\Big) \in W \circ W \circ W \subset V
	}
	\DeriveConclude{[*]}{\Intro \UNI}{\Phi \in \UNI(X,Y)}
	\EndProof
	\\
	\Theorem{UCUniqueExtensionTheorem}
	{
		\NewLine ::		
		\forall X \in \UNI \.
		\forall Y : \CUS \And \TYPE{T3} \. 
		\forall D  : \Dense(X) \. 
		\forall \phi \in \UNI(D,Y) \. \NewLine \.
		\exists! \Phi \in \UNI(X,Y) \.
		\Phi_{|D} = \phi
	}
	\NoProof
	\\
	\Theorem{UCbyUCRestricton}
	{
		\forall X,Y \in \UNI \.
		\forall \varphi \in \TOP(X,Y) \.
		\forall D : \Dense(X) \.
		\varphi_{|D} \in \UNI(D,Y)
		\Imply
		\varphi \in \UNI(X,Y)
	}
	\NoProof
	\\
	\Theorem{UnimorphismExtension}
	{
		\forall X,Y : \CUS \And \TYPE{T3} \.
		\forall A : \Dense(X) \.
		\forall B : \Dense(Y) \.
		A \cong_\UNI B \Imply X \cong_\UNI Y
	}
	\NoProof
}
\newpage
There is also a notions of total boundednes.
Subsets of uniform space are compacts iff they complete and totally bounded.
\subsubsection{Total Boundednes}
\Page{
	\DeclareType{Small}{\prod_{X \in \UNI} \U_X \to ??X}
	\DefineType{A}{Small}{\Lambda U \in \U_X \.  A \times A  \subset U }
	\\
	\Theorem{SmallSymmetricConnector}
	{
		\NewLine ::
		\forall X \in \UNI \. 
		\forall U \in \U_X \.
		\forall V : \Sym(X) \.
		\forall x \in X \.
		V \circ V \subset U \Imply \Small\Big(X,U,V(x)\Big)
	}
	\Say{[1]}{\THM{ConnectrorProductContainment}(X,V,x)}
	{
		V(x) \times V(x) \subset V \circ V
	}
	\Say{[2]}{[1][0]}{V(x) \times V(x) \subset U}
	\Conclude{[*]}{\Intro \Small [2]}{\Small\Big(X,U,V(x)\Big)}
	\EndProof
	\\
	\DeclareType{\TB}{\prod_{X \in \UNI} ??X}
	\DefineType{A}{\TB}{
		\forall U \in \U_X \. 
		\exists n \in \Nat \.
		\exists C : \{1, \ldots, n \} \to \Small(X,U) \.
		A \subset \bigcap^n_{k=1} C_k
	}
	\\
	\Theorem{InTBEveryUltrafilterIsCauchy}
	{
		\forall X : \UNI \.   
		\TB(X,X)
		\Imply \NewLine \Imply
		\forall \F : \Ultrafilter(X) \.
		\CF\Big( X, \F \Big)
	}
	\AssumeIn{U}{\U_X}
	\Say{\Big(n,C,[1]\Big)}{\Elim \TB(X,X,U)}
	{
		\sum^\infty_{n=1} \sum \{1,\ldots,n\} \to \Small(X,U) \. 
		X = \bigcup^n_{k=1} C_k
	}
	\Say{\Big(k,[2]\Big)}{\THM{UltrafilterUnion}[1]}
	{
		\sum^n_{k=1} C_k \in \F 
	}
	\Conclude{[U.*]}{\Elim \Small(X,U,C_k)}{C_k \times C_k \subset U}
	\DeriveConclude{[*]}{\Intro \CF}{\CF(X,\F)}
	\EndProof
}\Page{
	\Theorem{TotallyBoundedClosure}
	{
		\forall X \in \UNI \.
		\forall A : \TB(X) \.
		\TB\Big(X, \overline{A} \Big)
	}
	\AssumeIn{U}{\U_X}
	\Say{\Big(V,[1]\Big)}{\THM{ClosedConnectorsBaseExists}}
	{
		\sum V : \CConnector(X) \. V \subset U
	}
	\Say{\Big(n,C,[2]\Big)}{\Elim \TB(X, A, V )}
	{
		\sum^\infty_{n=1}
		\sum C : \{1,\ldots,n\} \to \Small(X,V) \.
		A \subset \bigcup^n_{k=1} C_k
	}
	\Say{[3]}{\Elim \Small(C)}
	{
		\forall k \in \{1,\ldots,n\} \. C_k \times C_k \subset V
	}
	\Say{[4]}{\Elim \CConnector(X,V)[3][1]}
	{
		\forall k \in \{1,\ldots,n\} \. 
		\overline{C}_k \times \overline{C}_k \subset V \subset U
	}
	\Say{[5]}{\Intro \Small [4]}
	{
		\forall k \in \{1,\ldots,n\} \. 
		\Small\big(X,U,\overline{C}_k\big)
	}
	\Conclude{[U.*]}{\THM{FiniteClosureUnion}[4]}
	{
		\overline{A} \subset \bigcup^n_{k=1} \overline{C}_k
	}
	\DeriveConclude{[*]}{\Intro \TB}{\TB\Big(X,\overline{A} \Big)}
	\EndProof
	\\
	\Theorem{UCPreservesTB}
	{
		\forall X,Y \in \UNI \.
		\forall \varphi \in \UNI(X,Y) \. 
		\forall A : \TB(X) \.
		\TB\Big(Y, f(A) \Big)
	}
	\AssumeIn{V}{\U_X}
	\Say{\Big(U,[1]\Big)}{\Elim \UC(X,Y,\varphi,V)}
	{
		\sum_{U \in \U_X} (\varphi \times \varphi) (U) \subset V
	}
	\Say{\Big(n,C,[2]\Big)}{\Elim \TB(X,A,U)}
	{
		\sum^\infty_{n=1} \sum C : \{1,\ldots,n\} \to \Small(X,U) \.
		A \subset \bigcap^n_{k=1} C_k
	}
	\Say{[3]}{\Elim \Small(C)}{
		\forall k \in \{1,\ldots,n\} \. 
		C_k \times C_k \subset U
	}
	\Say{[4]}{\varphi[3][1]}{
		\forall k \in \{1,\ldots,n\} \. 
		\varphi(C_k) \times \varphi(C_k) \subset V 
	}
	\Say{[5]}{\Intro \Small [4]}{
		\forall k \in \{1,\dots, n \} \.
		\Small\Big(Y,V,\varphi(C_k)\Big) 
	}
	\Conclude{[V.*]}{\THM{UnionMap}\Big(X,Y,\varphi\Big)[2]}
	{
		\varphi(A) \subset \bigcap^n_{k=1} \varphi(C_k)
	}
	\DeriveConclude{[*]}{\Intro \TB}{\TB\Big(Y,\varphi(A)\Big)}
	\EndProof
}\Page{
	\Theorem{TBByUltrafilters}
	{
		\forall X : \UNI \.
		\Big(\forall \F : \Ultrafilter(X) \.
		\CF\Big( X, \F \Big)
		\Imply \NewLine \Imply
		\TB(X,X)
	}
	\Assume{[1]}{\neg \TB(X,X)}
	\Say{\Big(U,[2]\Big)}{\Elim \TB [1]}
	{
		\sum U : \Sym(X) \. \forall A : \Finite(X) \. X \neq U(A)	
	}
	\Say{\Big(x,[3]\Big)}{\Elim U [2]}
	{
		x : \Nat \to X \. 
		\forall n,m \in \Nat \. 
		n \neq m \Imply  (x_n,x_m) \not \in U
	}
	\Say{A}{\Lambda n \in \Nat \. \{ x_m | m \ge n \}}{\Nat \to ?X}
	\Assume{\F}{\Filter(X)}
	\Assume{[4]}{\im A \subset \F}
	\Assume{[5]}{\CF(X,\F)}
	\Say{\Big(F,[6]\Big)}{\Elim \CF(X,\F,U)}{ \sum_{F \in \F} F \times F \subset U }
	\Say{\Big( N, [7]\Big)}{\Elim \Filter(X,\F,F)[4]\Elim A}
	{
		\sum N : \TYPE{Infinite}(\Nat) \. \forall n \in N \. x_n \in F
	}	
	\Conclude{[\F.*]}{[7][3]}{\bot}
	\Derive{[4]}{\Intro\forall}
	{
		\forall \F : \Filter(X) \. \im A \subset \F \Imply \neg \CF(X,\F)
	}
	\Say{\Big(\F,[5]\Big)}{\THM{UltrafilterTHM}(X,\im A)}
	{
		\sum \F : \Ultrafilter(X) \.  \im A \subset \F
	}
	\Say{[6]}{[4][5]}{\neg \CF(X,\F)}
	\Say{[7]}{[0](\F)}{\CF(X,\F)}
	\Conclude{[1.*]}{[6][7]}{\bot}
	\DeriveConclude{[*]}{\Elim \bot}{\TB(X,X)}
	\EndProof		
	\\
	\Theorem{UniCompactenessTHM}
	{
		\forall X \in \UNI \.
		\Compact(X) 
		\iff
		\TB \And \CUS(X)
	}
	\Assume{[1]}{\Compact(X)}
	\Say{[2]}{\THM{CompactnessByUltrafilters}}
	{
		\forall \F : \Ultrafilter(X) \.
		\CFilterbase(X,\F)
	}
	\Say{[3]}{\THM{EveryConvergenFilterbaseIsCauchy}[2]}
	{
		\forall \F : \Ultrafilter(X) \.
		\CF(X,\F)
	}
	\Say{[1.*.1]}{\THM{TBByUltrafilters}[3]}
	{
		\TB(X,X)
	}
	\Assume{\F}{\CF(X)}
	\Say{\Big(x,[4]\Big)}{\THM{inCompactFilterHasCluster}}
	{
		\sum_{x \in X} \Cluster\Big(X,\F,x\Big)
	}
	\Conclude{[\F.*]}{\THM{CuychyClustersAreLimits}}
	{
		x \in \lim \F
	}
	\Derive{[4]}{ \Intro \CFilterbase \Intro \forall}
	{
		\forall \F : \CF(X) \.  \CFilterbase(X,\F)
	}
	\Conclude{[1.*.2]}{\Intro \CUS [4]}{\CUS(X,\F)}
	\Derive{[1]}{\Intro \Imply}
	{
		\Compact(X) 
		\Imply
		\TB \And \CUS(X)
	}
	\Assume{[2]}{\TB \And \CUS(X)}
	\Say{[3]}{\THM{InTBEveryUltrafilterIsCauchy}(X)}
	{
		\forall \F : \Ultrafilter(X) \.
		\CF(X,\F)
	}
	\Say{[4]}{\Elim \CUS(X)[2]}
	{
		\forall \F : \Ultrafilter(X) \.
		\CFilterbase(X,\F)
	}
	\Conclude{[2.*]}{\THM{CompactByUltrafilters}[4]}{\Compact(X)}
	\DeriveConclude{[*]}{\Intro \Imply \Intro \iff [1]}
	{
		\Compact(X) 
		\iff
		\TB \And \CUS(X)
	}
	\EndProof
}
\newpage
\subsubsection{Bounded Sets [$\infty$]}
There is also boundedness, but this property is wierd.
This chapter will be written then there is a demand for it.
\Page{
	\DeclareType{Bounded}{\prod_{X \in \UNI} ??X}
	\DefineType{A}{Bounded}
	{
		\forall U \in \U_X \. 
		\exists n \in \Nat \.
		\exists F : \Finite(X) \.
		A \subset U^{\circ n}(F)
	}
}
\newpage
\subsection{Special Constructions}
Many operation which can be used to construct metric spaces 
from given topological data, can be also extended for uniformities.
\subsubsection{Uniformization}
As asnalogy of metrization there is an uniformization.
\Page{
	\DeclareFunc{ringUniformity}
	{
		\prod_{X \in \TOP} \Unif(X)
	}
	\DefineNamedFunc{ringUniformity}{}{\C_X}
	{
		\I_X\Big( C(X), \Reals, {\id}_{C(X)}\Big)
	}
	\\
	\Theorem{CompletelyRegularUniformization}
	{
		\forall X : \CR \. (X,\C_X) \cong_\TOP X
	}
	\NoProof
	\\
	\Theorem{EveryMetricSpaceAdmitsCompleteStruct}
	{
		\forall X \in \MS \. 
		\exists \U : \Unif(X) \. \NewLine \.
		\CUS(X,\U) \And X \cong_\TOP (X,\U)
	}
	\Say{\U}{\Cell_X \vee \C_X}{\Unif(X)}
	\Assume{\F}{\CF(X,\U)}
	\Say{[1]}{\Elim \U \THM{SupUniformityCauchyFilterbase}(X,\U,\F)}
	{
		\CF(X,\B_X)
	}
	\Say{\Big(f, [2] \Big)}{\Elim \TYPE{Completion}(X,\widehat{X},\F)}
	{
		\sum f \in \widehat{X} \. f = \lim \F
	}
	\SayIn{\phi}{\Lambda x \in X \. d(x,f)}{C(X)}
	\Assume{[3]}{f \not \in X}
	\Say{[4]}{\Elim \U \Elim \phi \THM{continuousInverse}}
	{
		\frac{1}{\phi} \in \UNI(X,\Reals)
	}
	\Say{[5]}{\Elim f }{\lim \frac{1}{\phi(\F)} = \infty}
	\Conclude{[6]}{\THM{UniformMapsPreserveFilters}[5][4]\Elim \CF}
	{
		\bot
	}
	\DeriveConclude{[\F.*]}{\Elim \bot}{ f \in X}
	\DeriveConclude{[*]}{\Intro \CUS}{\CUS(X,\U)}
	\EndProof
	\\
	\Theorem{CompactUniformityIsExists}
	{
		\forall X \in \HC \.
		\exists! \U : \Unif(X) \.
		(X,\U) \cong_\TOP X 
	}
	\NoProof
	\\
	\DeclareFunc{betaUniformity}
	{
		\prod X : \TYPE{Tychonoff} \.
		\Unif(X)
	}
	\DefineNamedFunc{betaUniformity}
	{}{\B_X}{\U_{\beta X} \cap (X \times X) }
	\\
	\DeclareFunc{alphaUniformity}
	{
		\prod X : \TYPE{LocallyComapct} \And \TYPE{T2} \.
		\Unif(X)
	}
	\DefineNamedFunc{alphaUniformity}
	{}{\A_X}{\U_{\omega X} \cap (X \times X) }
}
\Page{
	\Theorem{NormalBetaUniformity}
	{
		\forall X : \TYPE{T4} \And \neg \Compact \. \NewLine \.
		\neg\CUS(X,\B_X) \And \SCUS(X,\B_X)
	}
	\NoProof
	\\
	\DeclareType{MetrizableUniforSpace}
	{
		? \UNI
	}
	\DefineType{X}{MetrizableUniformSpace}
	{\exists d : \TYPE{Metric}(X) \. \U_X = \Cell_{(X,d)}}
	\\
	\DeclareType{SemimetrizableUniforSpace}
	{
		? \UNI
	}
	\DefineType{X}{SemimetrizableUniformSpace}
	{\exists d : \TYPE{Semimetric}(X) \. \U_X = \Cell_{(X,d)}}
	\\
	\Theorem{NormalBetaIsNotMetrizable}
	{
		\forall X : \TYPE{T4} \And \neg \Compact \.
		\neg \MUS(X,\B_X)
	}
	\NoProof	
	\\
	\Theorem{TychonoffBetaIsNotMetrizable}
	{
		\forall X : \TYPE{Tychonoff} \And \neg \Compact \.
		\neg \MUS(X,\B_X)
	}
	\NoProof
	\\
	\Theorem{RingUniformotyCompleteIffRealCompact}
	{
		\forall X : \TYPE{Tychonoff} \.
		\CUS(X,\C_X)
		\iff
		\TYPE{Realcompact}(X)
	}
	\NoProof	
	\\
	\DeclareType{Uniformazable}{?\TOP}
	\DefineType{X}{Uniformazable}{\exists \U \in \Unif(X) \. (X,\U) \cong_\TOP X }
}
\newpage
Gages is a different kind of generalizations of metric spaces.
It uses a set of metrics to define its topology.
Gage spaces are completely regular, but not neccesarly normal.
\subsubsection{Gages}
\Page{
	\Theorem{GageSubbase}
	{
		\forall X \in \SET \.
		\forall \mathfrak{R} : ?\TYPE{Semimetric}(X) \.
		\exists \mathfrak{R} \Imply
		\TYPE{Subbase}\Big( 
			\big\{ \Cell_\rho(x,\varepsilon) 
				\big|
				\rho \in \mathfrak{R},
				x \in X, 
				\varepsilon \in \Reals_{++}
			\big\}  
		\Big)
	}
	\NoProof
	\\
	\DeclareFunc{gageTopology}
	{
		\prod X \in \SET \.
		\TYPE{NonEmpty} \; \TYPE{Semimetric}(X) \to
		\TYPE{Topology}(X)		
	}
	\DefineNamedFunc{gageTopology}{\mathfrak{R}}{\T_{\mathfrak{R}}}
	{
		\Big\langle
			\big\{ \Cell_\rho(x,\varepsilon) 
				\big|
				\rho \in \mathfrak{R},
				x \in X, 
				\varepsilon \in \Reals_{++}
			\big\}  
		\Big\rangle
	}
	\\
	\DeclareFunc{gageUniformity}
	{
		\prod X \in \SET \.
		\TYPE{NonEmpty} \;\TYPE{Semimetric}(X) \to
		\Unif(X)
	}
	\DefineNamedFunc{gageUniformity}{\mathfrak{R}}{\U_{\mathfrak{R}}}
	{
		\bigvee_{\rho \in \mathfrak{R}} \Cell_{(X,\rho)}
	}
	\\
	\Theorem{GageTopology}
	{
		\forall X \in \SET \.
		\forall \mathfrak{R} : ?\TYPE{Semimetric}(X) \.
		(X,\T_{\mathfrak{R}}) \cong_\TOP (X,\U_{\mathfrak{R}})
	}
	\NoProof
	\\
	\DeclareType{GageSpace}{?\TOP}
	\DefineType{X}{GageSpace}
	{
		\exists  \mathfrak{R} : \TYPE{NonEmpty} \; \TYPE{Semimetric}(X) \. 
		X \cong_\TOP (X,\T_{\mathfrak{R}}) 
	}
	\\
	\Theorem{EveryGageSpaceIsUniformizable}
	{
		\forall X : \TYPE{GageSpace}  \.
		\TYPE{Uniformizable}(X)
	}
	\NoProof
	\\
	\Theorem{EveryGageSpaceIsCompletelyRegular}
	{
		\forall X : \TYPE{GageSpace}  \.
		\CR(X)
	}
	\NoProof
}
\newpage
\subsubsection{Metrization}
Uniform spaces can be metrized if they have countable base of uniformity.
It turnes out that gage topologies and uniform topologies are the same thing.
\Page{
	\DeclareFunc{triangulization}
	{
		\prod_{X \in \SET} 
		(X \times X \to \Reals_{+}) \to \TYPE{TriangleIneq}(X) 
	}
	\DefineNamedFunc{triangulization}{f}{d_f}
	{
		\Lambda x,y \in X \.		
		\inf \left\{ 
			\sum^{n-1}_{i=1} f(u_i,u_{i+1})\Bigg| 
			n \in \Nat, 
			u : \{1, \ldots ,n\} \to X,
			u_1 = x, u_n = y  
		\right\} 
	}
	\AssumeIn{x,y,z}{X}
	\Say{A}{
		\left\{ 
			\sum^{n-1}_{i=1} f(u_i,u_{i+1})\Bigg| 
			n \in \Nat, 
			u : \{1, \ldots ,n\} \to X,
			u_1 = x, u_n = y  
		\right\}
	}{?\Reals_+}
	\Say{B}{
		\left\{ 
			\sum^{n-1}_{i=1} f(u_i,u_{i+1})\Bigg| 
			n \in \Nat, 
			u : \{1, \ldots ,n\} \to X,
			u_1 = y, u_n = z  
		\right\}
	}{?\Reals_+}
	\Say{C}{
		\left\{ 
			\sum^{n-1}_{i=1} f(u_i,u_{i+1})\Bigg| 
			n \in \Nat, 
			u : \{1, \ldots ,n\} \to X,
			u_1 = x, u_n = z  
		\right\}
	}{?\Reals_+}
	\AssumeIn{a}{A}
	\AssumeIn{b}{B}
	\Say{\Big(n,u,[1]\Big)}{\Elim A(a)}
	{
			\sum^\infty_{n=1} \sum u : \{1,\ldots,n\} \to X \.
			u_1 = x \And u_n = y \And  a = \sum^{n-1}_{i=1} f(u_i,u_{i+1})
	}
	\Say{\Big(m,v,[2] \Big)}{\Elim B(b)}
	{
		\sum^\infty_{m=1} \sum u : \{1,\ldots,m\} \to X \.
		v_1 = y \And v_n = z \And  b = \sum^{m-1}_{i=1} f(v_i,v_{i+1})
	}
	\Conclude{\Big[(a,b).*\Big]}{[1][2]\Elim C}{
		a + b =
		\sum^{n-1}_{i=1} f(u_i,u_{i+1})
		+
		\sum^{m-1}_{i=1} f(v_i,v_{i+1}) \in C  
	}
	\DeriveConclude{[*]}{\Intro d_f}
	{
		d_f(x,z) \le d_f(x,y) + d_f(y,z)
	}
	\EndProof
	\\
	\DeclareType{CountableUniformSpace}
	{
		? \UNI
	}
	\DefineType{X}{CountableUniformSpace}{
		\exists \B : \BofU(X) \. |\B| \le \aleph_0	
	}
}\Page{
	\Theorem{SemimetrizationOfUniformSpace}
	{
		\forall X : \TYPE{CountableUniformSpace} \. 
		\TYPE{SemimetrizableUniformSpace}(X)
	}
	\Say{\Big(\B,[1]\Big)}{\Elim\TYPE{CountableUniformSpace}(X)}
	{
		\sum \B : \BofU(X) \.  |\B| \le \aleph_0
	}
	\Say{B}{\FUNC{enumerate}(\B)}{\Nat \ToSurj \B }
	\Say{\Big(V,[2]\Big)}{ 
		\FUNC{recursion}
		(
			\Int_+,  X \times X, 
			\Lambda V \in \U_X \. \Lambda n \in \Nat \. 
			\THM{UniformityTrisection}(X,V \cap B_n)
		) 
	}{
		\NewLine :		
		\sum V : \Nat \to \Sym(X) \. 
		\forall n \in \Int_+ \. 
		V_{n+1} \circ V_{n+1} \circ V_{n+1} \subset V_n \cap B_{n+1}	
	}
	\Say{\lambda}{
		\Lambda x,y \in X \.		
		\inf \{ 2^{-n}  | n \in \Int_+, (x,y) \in V_n  \}
	}{
		X \times X \to \Reals_{++}	
	}
	\Say{\rho}{d_\lambda}{\TYPE{TriangleIneq}(X)}
	\Say{[3]}{\Elim \lambda \Elim \rho }{\TYPE{Semimetric}(X,\rho)}
	\AssumeIn{x,y,z,w}{X}
	\Say{r}{\max\Big( \lambda(w,x),\lambda(x,y),\lambda(y,z) \Big)}{\Reals_{++}}
	\Say{\Big(n,[4]\Big)}{\Elim r \Elim \lambda}
	{
		\sum n \in \Int_+ \cup \{\infty\} \.
		2^{-n} = r
	}
	\Say{[5]}{[4]\Elim r}
	{
		\max\Big( \lambda(w,x),\lambda(x,y),\lambda(y,z) \Big) \le 2^{-n}
	}
	\Say{[6]}{\Elim\lambda [5]}
	{
		(w,x),(x,y),(y,z) \in V_n
	}
	\Say{[7]}{\Elim \Sym(X,V_n) [2][6]}
	{
		(w,z) \in V_{n-1}
	}
	\Say{[8]}{\Intro \lambda [7]}
	{
		\lambda(w,z) \le 2^{-n+1}
	}
	\Conclude{\Big[(x,y,z,w).*]}
	{
		[8][4]\Elim r
	}
	{
		\lambda(w,z) \le 
		2\max\Big( \lambda(w,x),\lambda(x,y),\lambda(y,z) \Big)
	}
	\Derive{[4]}{\Intro \forall}
	{
		\forall x,y,z,w \in X \. 
		\lambda(w,z) \le 
		2\max\Big( \lambda(w,x),\lambda(x,y),\lambda(y,z) \Big)
	}
	\Say{[5]}{\Elim \max [4]}
	{
		\forall n \in \Nat \.
		\forall x:\{1,\ldots,n\} \to X \. 
		\lambda(x_1,x_n) \le 2\sum^{n-1}_{i=1} \lambda(x_{i},x_{i+1})
	}
	\Say{[6]}{\Elim \rho [5]}{\rho \le \lambda \le 2\rho}	
	\AssumeIn{U}{\U_X}
	\Say{\Big(n,[7]\Big)}{\Elim \BofU(X,\B)[2]}
	{
		\sum^\infty_{n=1} V_n \subset B_n \subset U
	}
	\Conclude{[U.*]}{[6][7]}{\Cell_{\rho}(2^{-n}) \subset U}
	\Derive{[8]}{\Elim \Unif(X,\Cell_\rho)}{\U_X \subset \Cell_\rho}	
	\Assume{\varepsilon}{\Reals_{++}}
	\Say{\Big(n,[8]\Big)}{\THM{ExponentialLimit}(2,\varepsilon)}
	{
		\sum n \in \Nat \. 2^{-n}  < \varepsilon
	}
	\Conclude{[\varepsilon.*]}{\Elim \rho [8]}
	{
		V_n \subset  \Cell_\rho(\varepsilon)	
	}
	\Derive{[9]}{\Elim \Unif(X,\U_X}{\Cell_\rho \subset \U_X}
	\Conclude{[*]}{\Intro \TYPE{SetEq}[8][9]}
	{
		\Cell_\rho = \U_X
	}
	\EndProof	
	\\
	\DeclareType{UniformGageSpace}
	{
		?\UNI
	}
	\DefineType{X}{UniformGageSpace}
	{
		\exists \mathfrak{R} : \TYPE{NonEmptyp}\;\TYPE{Semimetric}{X} \.
		\U_X = \U_{\mathfrak{R}}	
	}
}\Page{
	\Theorem{EveryUniformSpaceIsAGage}
	{
		\forall X \in \UNI \.
		\TYPE{GageSpace}(X)
	}
	\Say{\mathfrak{U}}
	{
		\{
			\V \in \Unif(X) :
			\TYPE{CountableUniformSpace}(X,\V) \And \V \subset \U_X
		\}
	}{?\Unif(X)}
	\Say{\Big(\rho,[1]\Big)}
	{
		\THM{SemimetrizationOfUniformSpace}
	}
	{
		\rho : \mathfrak{U} \to \TYPE{Semimetric}(X) \.
		\forall \U \in \mathfrak{U} \.  
		\Cell_{\rho_\U} = \U
	}
	\Say{\mathfrak{R}}{\im \rho}
	{
		\TYPE{NonEmpty}\;\TYPE{Semimetric}(X)
	}
	\Say{[2]}
	{
		\Elim \mathfrak{U} 
		\Intro \sup
		[1]
		\Intro  U_\mathfrak{R}
	}
	{
		\U_X = \bigvee \mathfrak{U} = 
		\bigvee_{\rho \in \mathfrak{R}} \Cell_\rho =
		\U_\mathfrak{R}
	}
	\Conclude{[*]}{\Intro \TYPE{GageUniformSpace}[2]}
	{
		\TYPE{GageUniformSpace}(X)	
	}
	\EndProof
	\\
	\Theorem{UniformizableIffCompletelyRegular}
	{
		\forall X \in \TOP \.
		\CR(X) \iff \TYPE{Uniformizable}(X)
	}
	\NoProof
	\\
	\DeclareType{UnimorphicEmbedding}
	{
		\prod X,Y \in \UNI \. ?\UNI(X,Y)
	}
	\DefineType{\varphi}{UnimorphicEmbedding}
	{
		\TYPE{Unimorphism}\Big(X,\varphi(X),\varphi^{|\varphi(X)}\Big)
	}
	\\
	\Theorem{UnimorphicEmbeddingToAProduct}
	{
		\NewLine ::		
		\forall X \in \UNI \. 
		\exists I \in \SET \.
		\exists Y : I \to \SMS \.
		\exists \TYPE{UnimorphicEmbedding}\left(X,\prod_{i \in I} Y_i\right)	
	}
	\Say{\Big(\mathfrak{R},[1]\Big)}
	{
		\THM{EveryUniformSpaceIsAGage}(X)
	}
	{
		\sum \mathfrak{R} : \TYPE{NonEmpy}\;\TYPE{Semimetric}(X) \. 
		\U_X = \U_{\mathfrak{R}}
	}
	\Say{\varphi}{
		\Lambda x \in X \. 
		\Lambda \rho \in \mathfrak{R} \. x
	}
	{
		X \to X^{\mathfrak{R}}
	}
	\Say{[2]}{\Elim \varphi}
	{
		\forall \rho \in \mathfrak{R} \. 
		\varphi \pi_\rho = {\id}_X \in \UNI\Big(X,(X,\rho)\Big)
	}
	\Say{[3]}{\THM{InitialUniformityUniversalProperty}[1]}
	{
		\varphi \in \UNI\left(X, \prod_{\rho \in \mathfrak{R}} (X,\rho) \right)
	}
	\Say{[4]}{\Elim \varphi}{\varphi(X) = \Delta^{\mathfrak{R}}(X)}
	\AssumeIn{n}{\Nat}
	\Assume{\rho}{\{1,\ldots,n\} \to \mathfrak{R}}
	\Assume{\varepsilon}{\{1,\ldots,n\} \to \mathfrak{R}}
	\Say{R}{\im \rho}{\Finite(\mathfrak{R})}	
	\Say{[5]}{\Elim \FUNC{productUniformity}}
	{
		\Big(\Delta^{\mathfrak{R}}(X) \times \Delta^{\mathfrak{R}}(X)\Big)
		\cap		
		\prod^n_{i=1} \Cell_{\rho_i}(\varepsilon_i) \times X^{R^\c}
		\in  \U(\ldots)
	}
	\Conclude{[n.*]}{\Elim \varphi}
	{
		(\varphi^{-1} \times \varphi^{-1}) 
		\Big(\Delta^{\mathfrak{R}}(X) \times  \Delta^{\mathfrak{R}}(X)\Big)
		\cap		
		\prod^n_{i=1} \Cell_{\rho_i}(\varepsilon_i) \times X^{R^\c}
		\subset  \bigcap^n_{i=1} \Cell_{\rho_i}(\varepsilon_i)
	}
	\DeriveConclude{[*]}{\Elim \FUNC{gageUniformity}[1]\Intro \TYPE{UnimorphicEmbedding}}
	{
		\TYPE{UnimorphicEmbedding}
		\left( X, \prod_{\rho \in \mathfrak{R}} (X,\rho), \varphi
		\right)
	}
	\EndProof
}
\newpage
Also there is a completion.
\subsubsection{Completion}
\Page{
	\DeclareType{Completion}{
		\prod_{X \in \UNI} 
		?\sum Y : \CUS \.
		\TYPE{UnimorphicEmbedding}(X,Y)
	}
	\DefineType{\iota}{Completion}{\Dense\Big(Y,\iota(X)\Big)}
	\\
	\DeclareType{\SCompletion}{
		\prod_{X \in \UNI} 
		?\Completion(X)
	}
	\DefineType{(Y,\iota)}{\SCompletion}{Y \in \UNIS}
	\\
	\Theorem{EveryUniformSpaceHasACompltion}
	{
		\forall X \in \UNI \.
		\exists \Completion(X)
	}
	\Say{\Big(I,Y,\varphi\Big)}{\THM{UnimorphicEmbeddingToAProduct}}
	{
		\sum_{I \in \SET} \sum_{Y : I \to \SMS} \sum 
		\varphi : \TYPE{UniomorphicEmbedding}\left( X, \prod_{i \in I} Y_i \right)
	}
	\Say{\Big(\widehat{Y},\iota,[1]\Big)}{
		\Lambda i \in I \.		
		\THM{SemimetricCompletionExists}(Y_i)
	}
	{
		\prod_{i \in I} \Completion(Y_i)
	}
	\Say{Z}{{\cl}_{\left(\prod_{i \in I}\widehat{Y}_i\right)} \varphi(X)}
	{
		\Closed\left( \prod_{i \in I} \widehat{Y}_i \right)
	}
	\Say{[2]}{
		\THM{CompleteProductTHM}(I,\widehat{Y}_i)
		\THM{ClosedOfCompleteIsComplete}
	}
	{
		\CUS(Z)
	}
	\Say{\psi}{\varphi \prod_{i \in I} \iota_i}
	{
		\TYPE{HomeomorphicEmbedding}(X,Z)
	}
	\Say{[3]}{\Elim Z \Intro \Dense}{\Dense\Big(Z, \psi(X)\Big)}
	\Conclude{[*]}{\Intro \Completion}
	{
		\Completion(X,Z,\psi)
	}
	\EndProof
	\\
	\Theorem{UnimorphicEmbeddingToAMetricProduct}
	{
		\NewLine ::		
		\forall X \in \UNIS \. 
		\exists I \in \SET \.
		\exists Y : I \to \MS \.
		\exists \TYPE{UnimorphicEmbedding}\left(X,\prod_{i \in I} Y_i\right)	
	}
	\Say{\Big(I,Y,\varphi\Big)}{\THM{UnimorphicEmbeddingToAProduct}}
	{
		\sum_{I \in \SET} \sum_{Y : I \to \SMS} \sum 
		\varphi : \TYPE{UniomorphicEmbedding}\left( X, \prod_{i \in I} Y_i \right)
	}
	\Say{\Big(Z,\phi\Big)}
	{
		\Lambda i \in I \. \THM{MetricQuotient}
	}
	{
		\prod_{i \in I} \sum_{Z_i \in \MS} \TYPE{Isometry}(Y_i,Z_i)
	}
	\SayIn{\psi}{\varphi \prod_{i \in I}\psi_i}
	{
		\UNI\left( X, \prod_{i \in I} Z_i \right)
	}
	\Say{[1]}{\Elim \psi \Elim \UNIS(X) \Elim \TYPE{UnimorphicEmbedding}(\varphi)}
	{
		\TYPE{Injective}\left( X, \prod_{i\in I} Z_i, \psi \right)
	}
	\Say{[2]}{[1]Elim \TYPE{Isometry}(\phi)}
	{
		\TYPE{IsometricEmbedding}
		\left(
			\varphi(X), 
			\prod_{i \in I} Z_i
			\left(
				\prod_{i \in I} \phi_i
			\right)_{|\varphi(X)}
		\right)
	}
	\Conclude{[*]}{\Elim \psi [2]}
	{
		\TYPE{UnimorphicEmbedding}\left( X, \prod_{i\in I} Z_i, \psi \right)
	}
	\EndProof
}
\Page{
	\Theorem{SeparatedSpaceHasSeparetedComplition}
	{
		\forall X \in \UNIS \.
		\exists \SCompletion(X)
	}
	\NoProof
	\\
	\Theorem{SeparebleCompletionAreUnique}
	{
		\forall X \in \UNIS \.
		\forall (Y,\iota),(Y',\iota') :  \SCompletion(X) \.
		Y \cong_\UNI Y'
	}
	\NoProof
	\\
	\DeclareFunc{separableCompletion}
	{
		\prod X \in \UNIS \.
		\SCompletion(X)
	}
	\DefineNamedFunc{separableCompletion}{}{(\gamma X,\iota_{\gamma X})}
	{
		\THM{SeparedSpaceHasSeparatedComplition}(X)
	}
	\\
	\Theorem{UniformlyContinuousByDenseSubset}
	{
		\NewLine ::
		\forall X \in \UNI \.
		\forall Y \in \UNIS \.
		\forall \varphi \in \TOP(X,Y) \.
		\forall D : \TYPE{Dense}(X) \.
		\varphi_{|D} \in \UNI(D,Y) 
		\Imply
		\varphi \in \UNI(X,Y) 
	}
	\NoProof
	\\
	\Theorem{UniformityEqualityTHM}
	{
		\forall Y \in \SET \.
		\forall X \subset X \.
		\forall \U,\V \in \Unif(X) \. \NewLine
		(Y,\U),(Y,\V) \in \UNIS \And 
		\Dense\Big( (Y,\U) \And (Y,\V), X\Big)
		\And \U \cap X \times X = \V \cap X \times X
		\Imply
		\U = \V
	}
	\NoProof
}
\newpage
\subsubsection{Uniformly Continuous Metric [*]}
Will be written on demand.
\newpage
\subsection{Function Spaces}
Many sets of functions have natural uniformities.
\subsubsection{Pointwise Unifomity}
Functions with their values in an uniform space can be given an uniformity which corresponds to a pointwise convergence. Turns out it corresponds to the product uniformity.
\Page{
	\DeclareFunc{evaluation}
	{
		\prod_{X,Y \in \SET}  X \to (X \to Y) \to Y
	}
	\DefineNamedFunc{evaluation}{x,f}{\epsilon_x(f)}{f(x)}
	\\
	\DeclareFunc{pointwiseUniformSpace}
	{
		\SET \times \UNI \to \UNI
	}
	\DefineNamedFunc{pointwiseUniformSpace}{X,Y}
	{
		X \to_\pt Y
	}
	{
		\Big( 
			X \to Y,
			\I(X,Y, \epsilon )
		\Big)	
	}
	\\
	\Theorem{PointwiseUniformSpaceIsAProduct}
	{
		\forall X \in \SET \.
		\forall Y \in \UNI \.
		(X \to_\pt Y) \cong_\UNI  Y^X 
	}
	\NoProof
	\\
	\Theorem{PointwisePreservesSeparation}
	{
		\forall X \in \SET \.
		\forall Y \in \UNIS \.
		(X \to_\pt Y) \in \UNIS
	}
	\NoProof
	\\
	\Theorem{PointwisePreservesCompleteness}
	{
		\forall X \in \SET \.
		\forall Y : \CUS  \. \NewLine \.
		\CUS(X \to_\pt Y) 
	}
	\NoProof
	\\
	\Theorem{PointwisePreservesSeparatedCompleteness}
	{
		\forall X \in \SET \.
		\forall Y : \SCUS  \. \NewLine \.
		\SCUS(X \to_\pt Y) 
	}
	\NoProof
	\\
	\Theorem{PointwiseFilterConvergence}
	{
		\forall X \in \SET \.
		\forall Y \in \UNI \.
		\forall f : X \to Y \.
		\forall \F : \Filter(X \to_\pt Y) \. \NewLine \.
		f \in \lim \F 
		\iff
		\forall x \in X \. 
		f(x) \in \lim \epsilon_x(\F)
	}
	\NoProof
	\\
	\Theorem{PointwiseCauchyFilters}
	{
		\forall X \in \SET \.
		\forall Y \in \UNI \.
		\forall f : X \to Y \.
		\forall \F : \Filter(X \to_\pt Y) \. \NewLine \.
		\CF(X \to_\pt Y,\F) 
		\iff
		\forall x \in X \. 
		\CF\Big(X \to_\pt Y,\epsilon_x(\F)\Big)
	}
	\NoProof
}
\Page{
	\Theorem{PointwiseCompactness}
	{
		\forall X \in \SET \.
		\forall Y \in \UNI \. 
		\forall K \subset (X \to_\pt Y) \.
		\NewLine \.
		\Compacts(X \to_\pt Y, K)
		\iff
		\Closed(X \to_\pt Y, K)
		\And
		\forall x \in X \.
		\Compacts\Big(Y, \overline{\epsilon_x (K)}\Big)
	}
	\NoProof
	\\
	\Theorem{PointwiseCompleteness}
	{
		\forall X \in \SET \.
		\forall Y \in \UNI \. 
		\forall K \subset (X \to_\pt Y) \.
		\NewLine \.
		\CUS(K)
		\iff
		\Closed(X \to_\pt Y, K)
		\And
		\forall x \in X \.
		\CUS\Big(\overline{\epsilon_x (K)}\Big)
	}
	\NoProof
}
\newpage
\subsubsection{Uniformity of Uniform Convergence}
It is also possible to define uniformity of uniform convergence.
\Page{
	\DeclareFunc{uniformUniformity}
	{
		\prod_{X \in \SET} \prod_{Y \in \UNI} \Unif(X \to Y)
	}
	\DefineNamedFunc{uniformUniformity}{}{\U\U(X,Y)}
	{
		\bigg\{
				\Big\{   
						(f,g) \in (X \to Y)^2  \Big| 
						\forall x \in X \.
						\big(f(x),g(x)\big) \in U 				
				\Big\}
				\bigg|
				U \in \U(Y)        
		\bigg\}
	}
	\\
	\DeclareFunc{uniformConvergenceSpace}
	{
		\SET \to \UNI \to \UNI
	}
	\DefineNamedFunc{uniformConvergenceSpace}{X,Y}{(X \to_\U Y)}
	{
		\Big( X \to Y, \U\U(X,Y)  \Big)
	}
	\\
	\Theorem{PointwiseFilterConvergence}
	{
		\forall X \in \SET \.
		\forall Y \in \UNI \.
		\forall f : X \to Y \.
		\forall \F : \Filter(X \to_\U Y) \. \NewLine \.
		f \in \lim_{\U} \F 
		\iff
		\CF(X \to_\U Y,\F) 
		\And f \in \lim_{\pt} \F
	}
	\Assume{[1]}{f \in \lim_{\U} \F}
	\Say{[2]}{\THM{ConvergentIsCauchy}[1]}{\CF(X \to_\U Y,\F)}
	\AssumeIn{x}{X}
	\Assume{O}{\U\Big(f(x)\Big)}
	\Say{[3]}{\Elim \FUNC{uniformTopology}\Elim \U\U(X,Y)}
	{
		 O^X \in \T(X \to_\U Y)
	}
	\Say{\Big(F,[4]\Big)}{[1][3]}{\sum F \in \F \. F \subset O^X}
	\Conclude{[x.*]}{\epsilon_x [4]}{\epsilon_x(F) \subset O}
	\Derive{[3]}{\Intro \forall}
	{
		\forall x \in X \. f(x) \in \lim \epsilon_x \F			
	}
	\Conclude{[1.*]}{\THM{PointwiseFilterConvergence}[3]}
	{
		f \in \lim_{\pt} \F 
	}
	\Derive{[1]}{\Intro \Imply}
	{
		f \in \lim_{\U} \F 
		\Imply
		\CF(X \to_\U Y,\F) 
		\And f \in \lim_{\pt} \F
	}
	\Assume{[2]}{\CF(X \to_\U Y,\F)}
	\Assume{[3]}{f \in \lim_{\pt} \F }
	\AssumeIn{O}{\U_{X \to_\U Y}(f)}
	\Say{\Big(U,[4]\Big)}{\Elim \FUNC{uniformTopologu}(O)}
	{
		\sum U \in \U\U(X,Y) \. U(f) \subset O	
	}
	\Say{\Big(W,[5]\Big)}{
		\THM{SymmetricConnectorBaseExists}(Y,V)
		\THM{ClosedConnectorBaseExists}(Y,V)
	}
	{
		\NewLine :		
		\sum W : \Sym \And \CConnector(X \to_\U Y) \.
		W \subset U
	}
	\Say{\Big(V,[6]\Big)}{\Elim \U\U(X,Y,U)}
	{
		\sum V 	\in \Sym \And \CConnector(Y) \. 
				\NewLine \.				
				W = 
				\Big\{   
						(f,g) \in (X \to Y)^2  \Big| 
						\forall x \in X \.
						\big(f(x),g(x)\big) \in V 				
				\Big\}
	}
	\Say{\Big(F,[7]\Big)}{\Elim \CF(X \to_\U Y,\F,U)}
	{
		\sum F \in \F \. F \times F \subset W
	}
	\AssumeIn{g}{F}
	\Say{[9]}{[7][8]\Elim \Sym(Y,V)}{\forall x \in X \. F(x) \subset V\Big( g(x) \Big)}
	\AssumeIn{x}{X}
	\Say{[10]}{[3](x)}
	{
		\lim \F(x) = f(x)
	}
	\AssumeIn{E}{\U_Y}
	\Say{\Big(G,[11]\Big)}{\Elim \TYPE{Convergent}[10]}
	{
		\sum G \in \F \.
		G(x) \subset E\Big( f(x)  \Big)
	}
}\Page{
	\Conclude{[E.*]}{[11][9](x)}
	{
		F(x) \cap G(x) \subset V\Big( g(x) \Big) \cap E\Big( f(x) \Big)
	}
	\DeriveConclude{[g.*]}{\Elim \U_Y \Elim \CConnector(Y,V)}
	{
		f(x) \in  V\Big(g(x)\Big)
	}
	\DeriveConclude{[O.*]}{[4][5][6]}{G \subset O}
	\DeriveConclude{[2.*]}{\Intro \lim}{\lim_{\U} \F = f}
	\DeriveConclude{[*]}{\Intro \iff}
	{
			\lim_{\U} \F = f
			\iff
			\CF(X \to_\U Y,\F) 
			\And f \in \lim_{\pt} \F
	}	
	\EndProof
}
\newpage
\subsubsection{Uniform Convergence over S}
For any family $\S$ of subset of the domain, there is possible to define a uniformity for uniform convergence over $\S$. It turns out that both pointwise and uniform convergence are special cases of this.
\Page{
	\DeclareFunc{fUniformity}
	{
		\prod X \in \SET \.
		\prod Y \in \UNI \.
		??X \to \Unif(X \to Y)	
	}
	\DefineNamedFunc{fUniformity}{\mathcal{S}}{\F(X,Y,\mathcal{S})}
	{
			\I\bigg(
				\mathcal{S}, 
				\Lambda S \in \mathcal{S} \. \Big(S \to Y, \U\U(S,Y)\Big),
				\Lambda S \in \mathcal{S} \. \Lambda f : X \to Y \. f_{|S}    
			\bigg)	
	}
	\\
	\Theorem{FiniteFUniformityIsPointwise}
	{
		\forall X \in \SET \.
		\forall Y \in \UNI \. 
		\Big( X \to Y, \F(X,Y,\Finite(Y)) \Big) \cong_\UNI X \to_\pt Y
	}
	\NoProof
	\\
	\Theorem{GlobaFUniformityIsUniform}
	{
		\forall X \in \SET \.
		\forall Y \in \UNI \.
		\Big( X \to Y, \F(X,Y,\{Y\}) \Big) \cong_\UNI X \to_\U Y
	}
	\\
	\DeclareFunc{compactConvergence}
	{
		\TOP \to \UNI \to \UNI	
	}
	\DefineNamedFunc{compactConvergence}{X,Y}{X \to_\mathbb{K} Y}
	{
			\Big( X \to Y, \F\big(X,Y,\Compacts(X)\big)\Big)
	}
	\\
	\DeclareFunc{precompactConvergence}
	{
		\UNI \to \UNI \to \UNI
	}
	\DefineNamedFunc{precompactConvergence}{X,Y}{X \to_\lambda Y}
	{
			\Big( X \to Y, \F\big(X,Y,\TB(X)\big)\Big)
	}
	\\
	\Theorem{ClosedContinousCriterion}
	{
		\forall X \in \TOP \.
		\forall Y \in \UNI \.
		\forall \mathcal{S} : ??X \.\NewLine \.
		\forall x \in X \.
		\exists S \in \mathcal{S} \.
		x \in \intx S
		\Imply
		\Closed\Big( \big(X \to Y,\F(X,Y,\mathcal{S})\big) ,C(X,Y)\Big)
	}
	\AssumeIn{f}{\overline{C(X,Y)}}
	\AssumeIn{x}{X}
	\Say{\Big(S,[2]\Big)}{[0](x)}
	{
		\sum S \in \mathcal{S} \. 
		x \in \intx S
	}
	\Conclude{[x.*]}{\Elim \F(X,Y,\mathcal{S}) \THM{UniformLimitIsContinuous}(f)}
	{
		f_{|S} \in C(S,Y)
	}
	\DeriveConclude{[f.*]}{\THM{ContinuityIsLocal}}
	{
		f \in C(S,Y)
	}
	\DeriveConclude{[*]}{\THM{ClosedByLimits}}
	{
		\Closed\Big( \big(X \to Y,\F(X,Y,\mathcal{S})\big) ,C(X,Y)\Big)
	}
	\EndProof
	\\
	\Theorem{FunctionalTopologyCompleteness}
	{
		\forall X \in \SET \.
		\forall Y \in \UNI \.
		\forall \mathcal{S} : ??X \. \NewLine \.
		\CUS(Y) \And \TYPE{Cover}(X,\mathcal{S}) 
		\iff 
		\CUS\Big( X \to Y,  \F(X,Y,\mathcal{S})\Big)
	}
	\NoProof
}
\newpage
\subsubsection{Equicontinuiuty and Uniform Equicontinuiuty[!]}
Notions of equicontinuity and Arzello-Ascolli theorem also generalizes nicely.
There is no proofs in this chapter. They may be provided on demand.
\Page{
	\DeclareType{EquicontinousAtAPoint}
	{
		\prod X \in \TOP \.
		\prod Y \in \UNI \.
		X 
		\to
		??(X \to Y)		
	}
	\DefineType{F}{EquicontinousAtAPoint}
	{
		\Lambda x \in X \. 
		\forall U \in \U_Y \. 
		\exists O \in \U(x) \.
		\forall  f \in F \.
		f(O) \subset V\Big(f(x)\Big)
	}
	\\
	\DeclareType{UniformlyEquicontinousAtAPoint}
	{
		\prod X,Y \in \UNI \.
		X 
		\to
		??(X \to Y)		
	}
	\DefineType{F}{UniformlyEquicontinousAtAPoint}
	{
		\Lambda x \in X \. 
		\forall U \in \U_Y \. 
		\exists V \in \U_X \.
		\forall  f \in F \.
		f\Big(V(x)\Big) \subset V\Big(f(x)\Big)
	}
	\\
	\DeclareType{Equicontinous}
	{
		\prod X \in \TOP \.
		\prod Y \in \UNI \.
		??(X \to Y)		
	}
	\DefineType{F}{Equicontinous}
	{
		\forall x \in X \.
		\TYPE{EquicontinuousAtAPoint}(X,Y,F,x)
	}
	\\
	\DeclareType{UniformlyEquicontinous}
	{
		\prod X,Y \in \UNI \. ??(X \to Y)		
	}
	\DefineType{F}{UniformlyEquicontinous}
	{
		\forall x \in X \.
		\TYPE{UniformlyEquicontinuousAtAPoint}(X,Y,F,x)
	}
	\\
	\Theorem{UniformlyEquicontinuousAltDef}
	{
		\forall  X,Y \in \UNI \.
		\forall F : ??(X \to Y) \. \NewLine \.
		\UEqC(X,Y,F)
		\iff
		\forall V \in \U_Y \.
		\exists U \in \U_X \.
		\forall f \in F \.
		(f \times f)(U) \subset V
	}
	\NoProof
	\\
	\Theorem{BourbakiJointEquiontinuityTheorem}
	{
		\forall T \in \SET \.
		\forall X \in \TOP \.
		\forall Y \in \UNI \.
		\forall \mathcal{S} : ??T \. \NewLine \.
		\forall f : T \times X \to Y \.
		\Lambda x \in X \. 
		\Lambda t \in T \. f(t,x)
		\in \TOP\bigg( X, \Big(T \to Y, \F(T,Y,\mathcal{S})\Big)   \bigg)
		\iff \NewLine \iff
		\forall S \in \mathcal{S} \.
		\EqC\Big(X,Y \big\{ \Lambda x \in X \. f(t,x)   \big| t \in S  \big\}   \Big)
	}
	\NoProof
	\\
	\Theorem{BourbakiJointUniformEquiontinuityTheorem}
	{
		\forall T \in \SET \.
		\forall X,Y \in \UNI \.
		\forall \mathcal{S} : ??T \. \NewLine \.
		\forall f : T \times X \to Y \.
		\Lambda x \in X \. 
		\Lambda t \in T \. f(t,x)
		\in \UNI\bigg( X, \Big(T \to Y, \F(T,Y,\mathcal{S})\Big)   \bigg)
		\iff \NewLine \iff
		\forall S \in \mathcal{S} \.
		\UEqC\Big(X,Y \big\{ \Lambda x \in X \. f(t,x)   \big| t \in S  \big\}   \Big)
	}
	\NoProof
}\Page{
	\Theorem{EquicontinuityClosureTHM}
	{
		\forall X \in \TOP \.
		\forall Y \in \UNI \. 
		\forall F : ?(X \to Y) \. \NewLine \.
		\EqC(X,Y,F)
		\iff
		\EqC(X,Y,{\cl}_{\pt} F)
	}
	\NoProof
	\\
	\Theorem{UniformEquicontinuityClosureTHM}
	{
		\forall X \in \TOP \.
		\forall Y \in \UNI \. 
		\forall F : ?(X \to Y) \. \NewLine \.
		\UEqC(X,Y,F)
		\iff
		\UEqC(X,Y,{\cl}_{\pt} F)
	}
	\NoProof
	\\
	\Theorem{EquicontinuitySClosureTHM}
	{
		\forall X \in \TOP \.
		\forall Y \in \UNI \. 
		\forall \mathcal{S} : \TYPE{Cover}(X) \.
		\forall F : ?(X \to Y) \. \NewLine \.
		\EqC(X,Y,F)
		\iff
		\EqC(X,Y,{\cl}_{\mathcal{S}} F)
	}
	\NoProof
	\\
	\Theorem{UniformEquicontinuitySClosureTHM}
	{
		\forall X \in \TOP \.
		\forall Y \in \UNI \. 
		\forall \mathcal{S} : \TYPE{Cover}(X) \.
		\forall F : ?(X \to Y) \. \NewLine \.
		\UEqC(X,Y,F)
		\iff
		\UEqC(X,Y,{\cl}_{\mathcal{S}} F)
	}
	\NoProof
	\\
	\Theorem{PointwiseConvergenceIsCompact}
	{
		\NewLine ::		
		\forall X \in \TOP \.
		\forall Y \in \UNI \.
		\forall F : \EqC(X,Y) \.
		(F,\pt) \cong_\UNI (F,\mathbb{K})
	}
	\NoProof
	\\
	\Theorem{OnCompactPpintwiseConvergenseIsUniform}
	{
		\NewLine ::		
		\forall X : \Compact \.
		\forall Y \in \UNI \.
		\forall F : \EqC(X,Y) \. 
		\Big(F,\U\U(X,Y)\Big) \cong_\UNI  (F,\pt)
	}
	\NoProof
	\\
	\Theorem{PointwiseAndCompactClosureAgree}
	{
		\NewLine ::		
		\forall X \in \TOP \.
		\forall Y \in \UNI \.
		\forall F : \EqC(X,Y) \. 
		{\cl}_{\pt} F = {\cl}_{\mathbb{K}} F
	}
	\NoProof
}\Page{
	\Theorem{EquicontinuousSquareTHM}
	{
		\NewLine ::		
		\forall X : \Compact \.
		\forall Y \in \UNI \.
		\forall F : \EqC(X,Y) \. 
		\Big(F^2,\U\U^2(X,Y)\Big) \cong_\UNI  (F^2,\pt^2)
	}
	\NoProof
	\\
	\Theorem{SConvergenceTotallBoundnessImplyEq}
	{
		\forall X \in \TOP \.
		\forall Y \in \UNI \.
		\forall \mathcal{S} : \Cover(X) \. \NewLine \.
		\forall F  : \TB\Big(X \to Y, \F(X,Y,\mathcal{S})\Big) \. 
		\Big( \forall S \in \mathcal{S} \. F_{|S} \subset \TOP(S,Y) \Big)
		\Imply \NewLine \Imply
		\forall S \in \mathcal{S} \.
		\EqC\Big( S,Y, F_{|S} \Big)
	}
	\NoProof
	\\
	\Theorem{SConvergenceTotallBoundnessImplyTB}
	{
		\forall X \in \TOP \.
		\forall Y \in \UNI \.
		\forall \mathcal{S} : \Cover(X) \. \NewLine \.
		\forall F  : \TB\Big(X \to Y, \F(X,Y,\mathcal{S})\Big) \. 
		\Big( \forall S \in \mathcal{S} \. F_{|S} \subset \TOP(S,Y) \Big)
		\Imply \NewLine \Imply
		\forall x \in X \.
		\TB\Big(Y,F(x)\Big)
	}
	\NoProof
	\\
	\Theorem{ArzeloAscolliTHM}
	{
		\forall X \in \TOP \.
		\forall Y \in \UNIS \.
		\forall F \subset C(X,Y) \. \NewLine \.
		\Compacts\Big(X \to_{\mathbb{K}} Y,F\Big)
		\iff
		\Closed\Big(X \to_{\mathbb{K}} Y,F\Big)
		 \And
		\forall x \in X \. \Compacts\Big(Y,\overline{F(x)}\Big)
		\And \NewLine \And
		\forall K : \Compacts(X) \. \EqC(X,Y,F_{|K})
	}
	\NoProof
}
\newpage
\section{Topological Groups Basics}
Groups can be equipid with a topology in such a way that their alebraic and topological structrure interplay.
\subsection{Group Topology}
There are different ways to equip a group with a group topology, 
but all of them have some necessary properties.
\subsubsection{Category of Topological Groups}
Topological groups form a complete category.
\Page{
	\DeclareType{TopologicalGroup}
	{
		?\sum G \in \GRP \. \TYPE{Topology}(G)
	}
	\DefineType{(G,\T)}{TopologicalGroup}
	{
		\NewLine \iff		
		\circ_G \in \TOP\Big((G,\T) \times (G,\T),(G,\T)\Big)
		\And
		\Lambda g \in G \. g^{-1} \in \TOP\Big((G,\T),(G,\T)\Big)
	}
	\\
	\DeclareFunc{topologicalGroupAsGroup}
	{
		\TYPE{TopologicalGroup} \to \GRP
	}
	\DefineNamedFunc{topologicalGroupAsGroup}{G,\T}{(G,\T)}
	{
		G
	}
	\\
	\DeclareFunc{topologicalGroupAsTopologicalSpace}
	{
		\TYPE{TopologicalGroup} \to \TOP
	}
	\DefineNamedFunc{topologicalGroupAsTopologicalSpace}{G,\T}{(G,\T)}
	{
		(G,\T)
	}
	\\
	\DeclareFunc{categoryOfTopologicalGroups}{\CAT}
	\DefineNamedFunc{categoryOfTopologicalGroups}{}{\TGRP}
	{
		( \TYPE{TopologicalGroup}, \TOP \And \GRP, \circ, \id)	
	}
	\\
	\Theorem{HomomorphismWeakTopologyIsGroupTopology}
	{
		\NewLine ::		
		\forall I \in \SET \.
		\forall G \in \GRP \.
		\forall H : I \to \TGRP \. 
		\forall \phi : \prod_{i \in I} \GRP(G,H_i) \.
		\Big(G,\W(I,H,\phi)\Big) \in \TGRP
	}
	\Say{[1]}
	{
		\Lambda i \in I \. 
		\Elim_2 \GRP(G,H_i,\phi_i)
		\Elim \W_G(I,H,\phi)
		\Elim \TGRP(H_i)
		\Elim \CAT(\TOP)
	}
	{
		\NewLine :
		\forall i \in I \.		
		\inv_G \phi_i = \phi_i \inv_{H_i} \in 
		\TOP\bigg(\Big(G,\W(I,H,\phi)\Big),H_i\bigg)
	}
	\Say{[2]}{\THM{WeakTopologyUniversalProperty}[1]}
	{
		\inv_G \in \Aut_\TOP\Big(G,\W(I,H,\phi)\Big)
	}
	\Say{[3]}
	{
		\Lambda i \in I \. 
		\Elim_1 \GRP(G,H_i,\phi_i)
		\Elim \W_G(I,H,\phi)
		\Elim \TGRP(H_i)
		\Elim \CAT(\TOP)
	}
	{
		\NewLine :
		\forall i \in I \.		
		(\cdot_G) (\phi_i \times \phi_i) = (\phi_i \times \phi_i) (\cdot_{H_i}) \in 
		\TOP\bigg(\Big(G,\W(I,H,\phi)\Big)^2,H_i\bigg)
	}
	\Say{[4]}{\THM{WeakTopologyUniversalProperty}[3]}
	{
		\cdot_G \in \TOP\bigg(\Big(G,\W(I,H,\phi)\Big)^2,\Big(G,\W(I,H,\phi)\Big)\bigg)
	}
	\Say{[3]}{\Intro \TGRP [2][4]}
	{
		\Big(G,\W(I,H,\phi)\Big) \in \TGRP
	}
	\EndProof
	\\
	\Theorem{SupTopologyIsGroup}
	{
		\forall G \in \GRP \.
		\forall I \in \SET \.
		\forall \T : \T \to \TGRP(G) \.
		\left(G,\bigvee_{i \in I} \T_i \right) \in \TGRP
	}
	\NoProof
}\Page{
	\Theorem{TopologicalGroupsAreComplete}
	{
		\TYPE{Complete}(\TGRP)
	}
	\NoProof
}
\newpage
\subsubsection{Absolute Values and Invariant Metrics}
Group topology can be determinde by an absolute value function, or by an invariant metric.
Absolute value functions and invariant metrics are the same. 
\Page{
	\DeclareType{AbsoluteValue}
	{
		\prod_{G \in \GRP} (G \to \Reals_{+})
	}
	\DefineType{\alpha}{AbsoluteValue}
	{
		\alpha(e) = 0 
		\And
		\forall g \in G \.
		\alpha(g) = \alpha(g^{-1})
		\And
		\forall g,h \in G \.
		\alpha(gh) \le \alpha(g) + \alpha(h)
		\And
		\NewLine		
		\And
		\forall x : \Nat \to G \.
		\forall g \in G \.
		\lim_{n \to \infty} \alpha(x_n) = 0
		\Imply
		\lim_{n \to \infty} \alpha(gx_ng^{-1}) = 0
	}
	\\
	\DeclareFunc{absoluteValueAsSemimetric}
	{
		\prod_{G \in \GRP} 
		\TYPE{AbsoluteValue}(G)
		\to
		\TYPE{Semimetric}(G)
	}
	\DefineNamedFunc{absoluteValueAsSemimetric}{\alpha}{d_\alpha}
	{
		\Lambda a,b \in G \. \alpha(ab^{-1})
	}
	\Say{[1]}{
		\Lambda g \in G \. 
		\Elim d_\alpha(g,g)
		\THM{InverseMeaning}(G)
		\Elim_1 \TYPE{AbsoluteValue}(G,\alpha)
	}
	{
		\NewLine :		
		\forall g \in G \. 
		d_\alpha(g,g) = 
		\alpha(gg^{-1}) =
		\alpha(e) =
		0
	}
	\Say{[2]}{
		\Lambda g,h \in G \. 
		\Elim d_\alpha(g,h) 
		\THM{ProductInverse}(G)
		\Elim_2 \TYPE{AbsoluteValue}(G,\alpha,g^{-1})
		\Intro d_\alpha(h,g)
	}
	{
		\NewLine :		
		\forall g,h \in G \. 
		d_\alpha(g,h) = 
		\alpha(gh^{-1}) =
		\alpha((hg^{-1})^{-1}) =
		\alpha(hg^{-1})=
		d_\alpha(h,g)
	}
	\Say{[3]}{
		\Lambda f,g,h \in G \. 
		\Elim d_\alpha(g,h) 
		\THM{InverseMeaning}(G,g)
		\Elim_3 \TYPE{AbsoluteValue}(G,\alpha,g^{-1},fg^{-1},gh^{-1})
		\Intro d_\alpha(f,g)
		\Intro d_\alpha(g,h)
	}
	{
		\NewLine :		
		\forall f,g,h \in G \. 
		d_\alpha(f,h) = 
		\alpha(fh^{-1}) =
		\alpha(fg^{-1}gh^{-1}) \le 
		\alpha(fg^{-1}) + \alpha(gh^{-1})=
		d_\alpha(f,g) + d_\alpha(g,h)
	}
	\Conclude{[*]}{\Intro \TYPE{Semimetiric}(G)[1][2][3]}
	{
		\TYPE{Semimetric}(G,d_\alpha)
	}
	\EndProof
	\\
	\DeclareType{\LIM}{\prod_{G \in \GRP} ?\TYPE{Semimetric}(X)}
	\DefineType{\rho}{\LIM}{\forall a,b,g \in G \. d(ag,bg) = d(a,b)}
	\\
	\DeclareType{\RIM}{\prod_{G \in \GRP} ?\TYPE{Semimetric}(X)}
	\DefineType{\rho}{\RIM}{\forall a,b,g \in G \. d(ga,gb) = d(a,b)}
	\\
	\Conclude{\TIM}{\prod_{G \in \GRP} \LIM \And \RIM}{\GRP \to \Type}
	\\
	\Theorem{AbsoluteValueMetricIsRightInvariant}
	{
		\NewLine ::		
		\forall G \in \GRP \.
		\forall \alpha : \TYPE{AbsoluteValue}(G) \. 
		\RIM(G,d_\alpha)	
	}
	\Say{[1]}{
		\Lambda a,b,g \in G \.
		\Elim d_\alpha(ag,bg)
		\THM{InverseProduct}(G)
		\THM{InverseMeaning}(G,g)
		\Intro d_\alpha(ag,bg)
	}{
		\NewLine :		
		\forall a,b,g \in G \.
		d_\alpha(ag,bg) = 
		\alpha(agg^{-1}b^{-1}) = 
		\alpha(ab^{-1}) =
		d_\alpha(a,b)
	}
	\Conclude{[*]}{\Intro \RIM [1]  }
	{
		\RIM(G,d_\alpha)
	}
	\EndProof
	\\
}\Page{
	\Theorem{AbsoluteValueMetrizesATopologicalGroup}
	{
		\forall G \in \TGRP \.
		\forall \alpha : \TYPE{AbsoluteValue}(G) \.
		(G,d_\alpha) \in \TGRP
	}
	\AssumeIn{g}{G}
	\Assume{x}{\Nat \to G}
	\Assume{[1]}{\lim_{n \to \infty} x_n = g}
	\Say{[2]}{\THM{MetricLimit}[1]}{\lim_{n=1} d_\alpha(x_n,g) = 0}
	\Say{[3]}{\Elim d_\alpha }
	{
		0 = \lim_{n \to \infty} \alpha(x_ng^{-1})
	}
	\Say{[4]}{
		\Elim_4 \TYPE{AbsoluteValue}(G,\alpha,g^{-1})[4]
		\THM{InverseProperty}(G,\alpha)
	}
	{
		\NewLine :		
		0 = \lim_{n \to \infty} \alpha(g^{-1}x_ng^{-1}g) =
		\lim_{n \to \infty} \alpha(g^{-1}x_n) = 
		\lim_{n \to \infty} d_\alpha(g^{-1},x_n^{-1})
	}
	\Conclude{[g.*]}{\THM{MetricLimit}[4]}{\lim_{n=1} x_n^{-1} = g^{-1}}
	\Derive{[1]}{\THM{ContinuousByLimits}}
	{
		\Lambda g \in G \. g^{-1} \in \Aut_\TOP(G,d_\alpha)
	}
	\AssumeIn{g,h}{G}
	\Assume{x,y}{\Nat \to G}
	\Assume{[2]}{\lim_{n \to \infty} x_n = g}
	\Assume{[3]}{\lim_{n \to \infty} y_n = h}
	\Say{[4]}{\THM{MetricLimit}[2]\Elim d_\alpha}
	{
		\lim_{n=1} d_\alpha(x_n,g) = 
		\lim_{n=1} \alpha(x_ng^{-1}) = 0		
		0
	}
	\Say{[5]}{\THM{MetricLimit}[3]\Elim d_\alpha}{
		\lim_{n=1} d_\alpha(y_n,h) = 
		\lim_{n=1} \alpha(y_nh^{-1})		
		0
	}
	\Conclude{[6]}
	{
		\Lambda n \in \Nat \.
		\Elim d_\alpha 
		\Elim_2 \TYPE{AbsoluteValue}(G,\alpha,x^{-1}_n)
		\THM{InverseMeaning}
		\Elim_3 \TYPE{AbsoluteValue}(G,\alpha, y_nh^{-1},g^{-1}x_n)
		\NewLine
		\Elim \TYPE{Symmetric}(G,G,d_\alpha)
		\THM{LimitSum}
		[4][5]			
	}
	{
		\lim_{n=1} d_{\alpha}(x_ny_n,gh) =
		\lim_{n=1} \alpha(x_ny_nh^{-1}g^{-1}) =
		\lim_{n=1} \alpha(x^{-1}_nx_ny_nh^{-1}g^{-1}x_n) = \NewLine =
		\lim_{n=1} \alpha(y_nh^{-1}g^{-1}x_n) \le
		\lim_{n=1} \alpha(y_nh^{-1}) + \alpha(g^{-1}x_n) =
		\lim_{n=1} d_\alpha(y_n,h) + \lim_{n=11} d_\alpha(x_n,g) = 
		0
	}
	\Say{[7]}{\THM{NonNegtiveZeroBound}[8]}
	{
		\lim_{n=1} d_{\alpha}(x_ny_n,gh) = 0
	}
	\Conclude{\Big[8.*\Big]}{\THM{MetricLimit}[7]}{\lim_{n \to \infty} x_ny_n = gh}
	\Derive{[2]}{\THM{ContinuousByLimits}}
	{
		\circ \in \TOP\Big( (X,d_\alpha)^2,(X,d_\alpha)\Big)	
	}
	\Conclude{[*]}{\Intro \TGRP [1][2]}
	{
		(X,d_\alpha) \in \TGRP
	}
	\EndProof
}\Page{
	\DeclareFunc{absoluteValueFromRIM}
	{
		\prod_{G \in \TGRP} 
		\RIM(G) \to \TYPE{AbsoluteValue}(F)
	}
	\DefineNamedFunc{absoluteValueFromRIM}{\rho}{\alpha_\rho}
	{
		\Lambda g \in G \. \rho(g,e)
	}
	\Say{[1]}{
		\Elim \alpha_\rho(e)
		\Elim_1 \TYPE{Semimetric}(\rho)  
	}{
		\alpha_\rho(e) = \rho(e,e) = 0
	}
	\Say{[2]}{
		\Lambda g \in G \.
		\Elim \alpha_\rho(g^{-1}) 
		\Elim \RIM(G,\rho)
		\Elim \TYPE{Symmetric}(G,\rho)
		\Intro \alpha_\rho(g^{-1})
	}
	{
		\NewLine :
		\forall g \in G \.		
		\alpha_\rho(g^{-1}) =
		\rho(g^{-1},e) =
		\rho(e,g) =
		\rho(g,e) =
		\alpha_\rho(g) 
	}
	\Say{[3]}{
		\Lambda g,h  \in G \.
		\Elim \alpha_\rho(gh)
		\Elim \RIM(G,\rho,gh,e,h^{-1})
		\Elim \TYPE{TriangleIneq}(G,\rho,g,e,h^{-1})
		\NewLine
		\Elim \RIM(G,\rho,e,h^{-1},h)
		\Intro \alpha_\rho
	}
	{		
		\forall g,h \in G \.
		\alpha_\rho(gh) = 
		\rho(gh,e) =
		\rho(g,h^{-1}) \le \NewLine \le
		\rho(g,e) + \rho(e,h^{-1}) =
		\rho(g,e) + \rho(h,e) = 
		\alpha_\rho(g) + \alpha_\rho(h)
	}
	\Assume{x}{\Nat \to G}
	\Assume{[4]}{\lim_{n \to \infty} \alpha_\rho(x_n) = 0}
	\AssumeIn{g}{G}
	\Say{[5]}{[4]\Elim \alpha_\rho\THM{MetricLimit}(G,\rho)}
	{
		\lim_{n \to \infty} x_n = e
	}
	\Say{[6]}{\Elim \TGRP(F)g[5]g^{-1}}
	{
		\lim_{n \to \infty} gx_ng^{-1} =e
	}
	\Conclude{[x.*]}
	{
		[6]\THM{MetricLimit}(G,\rho) \Intro \alpha_\rho
	}
	{
		\lim_{n \to \infty} \alpha_\rho(gx_ng^{-1}) = 0
	}
	\DeriveConclude{[*]}{\Intro \TYPE{AbsoluteValue}[1,2,3]}
	{
		\TYPE{AbsoluteValue}(G,\alpha_\rho)
	}	
	\EndProof
	\\
	\Theorem{AbsoluteValueGeneratingMetricCondition}
	{
		\NewLine ::		
		\forall G \in \GRP \.
		\forall \alpha : \TYPE{AbsoluteValue}(G) \.
		\TYPE{Metric}(G,d_\alpha)
		\iff 
		\forall g \in G \. 
		g \neq e \Imply  \alpha(g) > 0
	}
	\AssumeIn{a,b}{G}
	\Assume{[1]}{a \neq b}
	\Say{[2]}{\Elim \GRP(G)[1]}{ab^{-1} \neq e }
	\Conclude{\Big[(a,b).*\Big]}{\Elim }
	{
		d_\alpha(a,b) =
		\alpha(ab^{-1}) > 0
	}
	\Derive{[*]}{\Intro \TYPE{Metric}}
	{
		\TYPE{Metric}\Big( G, d \Big)	
	}
	\EndProof
	\\
	\Theorem{LEMByIsometry}
	{
		\forall G \in \GRP \.
		\forall \TYPE{Semimetric}(G,\rho) \. \NewLine \.
		\LIM(G,\rho)
		\iff
		\forall a \in A \.
		\TYPE{Isometry}(G,G,\lambda_a)
	}
	\NoProof
	\\
	\DeclareFunc{metricInversion}
	{
		\prod_{G \in \GRP} \TYPE{Semimetric}(G) \to \TYPE{Semimetric}(G)
	}
	\DefineNamedFunc{metricInversion}{\rho}
	{\rho^{-1}}
	{
		\Lambda g,h \in G \. \rho(g^{-1},h^{-1}) 
	}
	\\
	\Theorem{InvariantInversion}
	{
		\forall G \in \GRP \.
		\forall \rho : \LIM(G) \.
		\RIM(G,\rho^{-1})
	}
	\NoProof
}
\newpage
\subsubsection{Neighbourhoods of Unity}
Topology of a topological group 
is fully determined by the neighborhood system 
of its unity.
\Page{
	\DeclareType{\SS}{\prod_{G \in \GRP} ?G}
	\DefineType{S}{\SS}{\inv(S) = S}
	\\
	\Theorem{UnityHasSymmetricHoodBase}
	{
		\forall G \in \TGRP \.
		\forall U \in \U(e) \.
		\exists V \in \U(e) :
		V \subset U \And \SS(G,V)
	}
	\Say{V}{U \cap \inv(U)}{?G}
	\Say{[1]}{\THM{unityInverse}(G)\Elim V}{e \in V}
	\Say{[2]}{\Elim \TGRP(G)\Elim V}
	{
		V \in \T(G)	
	}
	\Conclude{[*]}{\Elim V}{\inv(V) = V}
	\EndProof
	\\
	\Theorem{InverseContinuityAtUnityCriterion}
	{
		\NewLine :		
		\forall G \in \TOP \And \GRP \.
		\inv_G \in C_e(G,G) 
		\iff
		\forall U \in \U(e) \.
		\inv(U) \in \U(e)
	}
	\NoProof
	\\
	\Theorem{MultContinuityAtUnityCriterion}
	{
		\NewLine :		
		\forall G \in \TOP \And \GRP \.
		(\cdot_G) \in C_{(e,e)}(G^2,G) 
		\iff
		\forall U \in \U(e) \.
		\exists V \in \U(e) :
		VV \subset U
	}
	\Assume{[1]}{(\cdot_G) \in C_{(e,e)}(G^2,G)}
	\AssumeIn{U}{\U(e)}
	\SayIn{W}{(\cdot_G)^{-1}(U)}{\U(e,e)}
	\Say{\Big(A,B,[2]\Big)}{\THM{ProductTopologyBae}(G,G,W)}
	{
		\sum A,B \in \U(e) \.
		A \times B \subset W  
	}
	\SayIn{V}{A \cap B}{\U(e)}
	\Conclude{[1*]}{\Elim V[2]}{VV \subset U}
	\Derive{[1]}{\Intro \Imply}
	{
		(\cdot_G) \in C_{(e,e)}(G^2,G) 
		\Imply
		\forall U \in \U(e) \.
		\exists V \in \U(e) :
		VV \subset U
	}
	\Assume{[2]}
	{
		\forall U \in \U(e) \.
		\exists V \in \U(e) :
		VV \subset U
	}
	\AssumeIn{U}{\U(e)}
	\Say{\Big(V,[3]\Big)}
	{
		[2](U)	
	}
	{
		\sum V \in \U(e) \. VV \subset U
	}
	\Conclude{[U.*]}{\THM{ProductTopologyBase}(G,G,V)}{V \times V \in \U(e,e)}
	\DeriveConclude{[2.*]}{\Intro C_{(e,e)}}{(\cdot_G) \in C_{(e,e)}(G^2,G)} 
	\DeriveConclude{[*]}{\Intro \iff[1]}
	{
		(\cdot_G) \in C_{(e,e)}(G^2,G) 
		\iff
		\forall U \in \U(e) \.
		\exists V \in \U(e) :
		VV \subset U 
	}
	\EndProof
}\Page{
	\Theorem{TopologicalGroupAltDef}
	{
		\forall G \in \GRP \.
		\forall \T : \TYPE{Topology}(G) \.
		(G,\T) \in \TGRP 
		\iff
		\NewLine
		\iff
		\Big(
			\forall g,h \in G \. 
			\forall U \in \U(h) \.
			 gU \in \U(gh) 
		\Big)
		\And
		\Big(
			\forall U \in \U(e) \.
			\forall \inv \; U \in \U(e) \.
		\Big)
		\NewLine \And
		\And
		\Big(
			\forall U \in \U(e) \.
			\exists V \in \U(e) \.
			VV \subset U
		\Big)
		\And
		\Big(
			\forall U \in \U(e) \.
			\forall g \in G \.
			\exists V \in \U(e) \.
			aVa^{-1} \subset U
		\Big)
	}
	\Say{[1]}{\THM{MultContinuityAtUnityCriterion}[0.3]}
	{
		(\cdot_G) \in C_{(e,e)}(G^2,G) 
	}
	\Assume{(\Delta,g)}{\TYPE{Net}(G)}
	\Assume{[2]}{e \in \lim_{\delta \in \Delta} g_\delta }
	\AssumeIn{a}{G}
	\AssumeIn{U}{\U(e)}
	\Say{\Big(V,[3]\Big)}{[0.4](U)}{
		\sum V \in \U(e) \.
		aVa^{-1} \subset U
	}
	\Say{\Big(\delta,[4]\Big)}{\Elim [3](V)}
	{
		\sum \delta \in \Delta \. \forall \sigma \ge \delta \. g_\sigma \in V
	}
	\Conclude{\Big[(\Delta,g).3)]}{[3][4]}
	{
		\forall \sigma \ge \delta \.
		ag_\sigma a^{-1} \subset U
	}
	\Derive{[2]}{\Intro C_e}
	{
		\forall a \in G \. (\Lambda g \in G \. aga^{-1} ) \in C_e(G,G)
	}
	\Assume{(\Delta,x),(\Delta,y)}{\TYPE{Net}(G)}
	\AssumeIn{g,h}{G}
	\Assume{[3]}{g \in \lim_{\delta \in \Delta} x_\delta}
	\Assume{[4]}{h \in \lim_{\delta \in \Delta} y_\delta}
	\Say{[5]}{[0.1][3]}
	{
		e \in  \lim_{\delta \in \Delta} g^{-1}x_\delta
	}
	\Say{[6]}{[0.1][4]}
	{
		e \in \lim_{\delta \in \Delta} y_\delta h^{-1}
	}
	\Say{[7]}{[1][5][6]}
	{
		e \in  \lim_{\delta \in \Delta} g^{-1}x_\delta y_\delta h^{-1}
	}
	\Say{[8]}{[2](g)[7]}
	{
		e \in \lim_{\delta \in \Delta} x_\delta y_\delta h^{-1} h^{-1} g^{-1}
	}
	\Conclude{[\ldots*]}{[0.1][8]}
	{
		hg \in \lim_{\delta \in \Delta} x_\delta y_\delta
	}
	\Derive{[3]}{\THM{ContinuityByNets}}{(\cdot_G) \in \TOP(G^2,G)}
	\AssumeIn{g}{G}
	\AssumeIn{U}{\U(g)}
	\Say{[4]}{[0.1](U,g^{-1})}{g^{-1}(U) \in \U(e)}
	\Say{[5]}{[0.2][4]}{ \inv\Big( g^{-1}(U) \Big) = \inv(U) g \in \U(e)  }
	\Say{[6]}{[0.1][5]}{g^{-1} \inv(U) g \in \U(g^{-1})}
	\Conclude{[g.*]}{[3][6]}{\inv(U) \in \U(g^{-1})}
	\Derive{[4]}{\Intro \TOP}{\inv \in \TOP(G,G)}
	\Conclude{[*]}{\Intro \TGRP [3][4]}{(G,\T) \in \TGRP}
	\EndProof
	\\
	\Theorem{ConjugationIsAutomorphism}
	{
		\forall G \in \TGRP \.
		\forall g \in G \.
		\gamma_g \in \Aut_{\TGRP}(G)
	}
	\NoProof
}
\Page{
	\Theorem{TopologicalGroupAltDef2}
	{
		\forall G \in \GRP \.
		\forall \T : \TYPE{Topology} \.
		G \in \TGRP
		\iff
		\NewLine
		\iff
		\Big(\Lambda g,h \in G \. gh^{-1} \Big)
		\in 
		\TOP\Big( (G,\T)^2, (G,\T)\Big) 
	}
	\Say{\phi}
	{
		\Lambda g,h \in G \. gh^{-1}
	}
	{
		\TOP\Big( (G,\T)^2, (G,\T)\Big) 
	}
	\Say{[1]}{\Elim \phi \Intro \inv}
	{
		\inv = \Lambda g \in G \.  \phi(e,g)
	}
	\Say{[2]}{\Intro \TOP [1]}{\inv \in \TOP\Big((G,\T),(G,\T)\Big)}
	\Say{[3]}{\Elim \phi \Intro \cdot }
	{
		(\cdot_G) =  (\id \times \inv)\phi
	}
	\Say{[4]}{\Intro \TOP [3]}
	{
		(\cdot_G) \in \TOP\Big( (G,\T)^2, (G,\T)\Big)
	}
	\Conclude{[*]}{\Intro \TGRP [2][4]}{(G,\T) \in \TGRP}
	\EndProof
}
\newpage
\subsubsection{Uniformity and Regularity}
Topological groups are uniform spaces and, hence completely regular.
\Page{
	\DeclareFunc{leftGroupConnector}
	{
		\prod_{G \in \TGRP} 
		\U(e) \to \Connector(G) 	
	}
	\DefineNamedFunc{leftGroupConnector}{U}{U_L}
	{
		\{ (a,b) \in G^2 \. a^{-1}b \in U   \}
	}
	\\
	\Theorem{LeftConnectorsIntersection}
	{
		\forall G \in \TGRP \.
		\forall U,V \in \U(e) \.
		(U \cap V)_L = U_L \cap V_L
	}
	\NoProof
	\\
	\Theorem{LeftConnectorTranspose}
	{
		\forall G \in \TGRP \.
		\forall U \in \U(e) \.
		(U_L)^\top = \Big(\inv(U)\Big)_L
	}
	\NoProof
	\\
	\Theorem{LeftConnectorsCompose}
	{
		\forall G \in \TGRP \.
		\forall U,V \in \U(e) \.
		U_L \circ V_L = (VU)_L
	}
	\AssumeIn{(a,c)}{U_L \circ V_L}
	\Say{\Big(b,[2]\Big)}{\Elim (\circ) (U_L \circ V_L)(a,c) }
	{
		\sum b \in G \. (a,b) \in U_L \And (b,c) \in V_L
	}
	\Say{[3]}{\Elim U_L [2]}{a^{-1}b \in U \And b^{-1}c \in V}
	\Say{[4]}{[3.1][3.2]\THM{InverseMeaning}(G,G)}
	{
		a^{-1}c = a^{-1}bb^{-1}c = a^{-1}c \in UV
	}
	\Conclude{\Big[(a,c).*\Big]}{\Intro (VU)_L[4]}
	{
		(a,c) \in (VU)_L	
	}
	\Derive{[1]}{\Intro \subset}{ U_L \circ V_L \subset (VU)_L }
	\AssumeIn{(a,c)}{(VU)_L}
	\Say{[2]}{\Elim (VU)_L (a,c) }
	{
		a^{-1}c \in VU
	}
	\Say{\Big(v,u,[3]\Big)}{\Elim VU [2]}
	{
		\sum v \in V \. \sum u \in U \. a^{-1}c = vu
	}
	\Say{[4]}{[3]u^{-1}}{a^{-1}cu^{-1} = v \in V}
	\Say{[5]}{\Intro V_L [4]}{(a,cu^{-1}) \in V_L}
	\Say{[6]}{v^{-1}[3]}{v^{-1}a^{-1}c = u \in U}
	\Say{[7]}{\Intro U_L [6]}{(av,c) \in U_L}
	\Say{[8]}{[3]\THM{InverseMeaning}(G,u)}
	{
		cu^{-1} = avuu^{-1} = av
	}
	\Conclude{\Big[(a,c).*\Big]}{\Intro U_L \circ V_L [5][7][8]}
	{
		(a,c) \in  U_L \circ V_L
	}
	\DeriveConclude{[*]}{\Intro \TYPE{SetEq}[1]}
	{
		U_L \circ V_L = (VU)_L
	}	
	\EndProof
	\\
	\DeclareFunc{leftGroupUniformity}{\prod_{G \in \TGRP} \Unif(G)}
	\DefineNamedFunc{leftGroupUniformity}{}{\L_G}
	{
		\Big\langle\{U_L  | U \in \U_G(e) \}\Big\rangle_\F
	}
}\Page{
	\Theorem{TopologicalGroupIsUniformizableByLeftUniformities}
	{
		\forall G \in \TGRP \. G \cong_\TOP (G,\L_G)
	}
	\AssumeIn{g}{G}
	\AssumeIn{U}{\U(g)}
	\AssumeIn{u}{U}
	\Say{\varphi}{\Lambda x \in G \. u^{-1}x}{\Aut_\TOP(G)}
	\Say{V}{\varphi(U)}{\U(e)}
	\Say{[1]}{\Elim \GRP(G) \Elim V \Intro V_L}{U = V_L(u)}
	\Conclude{[u.*]}{\THM{EqIsSubset}[1]}{V_L(u) \subset U}
	\DeriveConclude{[U.*]}{\Elim \FUNC{uniformTopology}}
	{
		U \in \U_{\L_G}(g)
	}
	\Derive{[1]}{\Intro \subset }{\U(g) \subset \U_{\L_G}(g)}
	\Assume{U}{\U_{\L_G}(g)}
	\Say{\Big(V,[2]\Big)}{\Elim \L_G}
	{
		\sum V \in \U(e) \. U = V_L(g)
	}
	\Say{[3]}{\Elim  V_L}{U = gV}
	\Conclude{[U.*]}{\Elim \TGRP(G)[3]}{U \in \U(g)}
	\DeriveConclude{[x.*]}{\Intro \TYPE{SetEq}[1]}{\U(g) = \U_{\L_G}(g)}
	\DeriveConclude{[*]}{\THM{TopologyEqByHoods}}
	{
		G \cong_\TOP (G,\L_G)
	}
	\EndProof
	\\
	\DeclareFunc{rightGroupConnector}
	{
		\prod_{G \in \TGRP} 
		\U(e) \to \Connector(G) 	
	}
	\DefineNamedFunc{rightGroupConnector}{U}{U_R}
	{
		\{ (a,b) \in G^2 \. ab^{-1} \in U   \}
	}
	\\
	\Theorem{rightConnectorsIntersection}
	{
		\forall G \in \TGRP \.
		\forall U,V \in \U(e) \.
		(U \cap V)_R = U_R \cap V_R
	}
	\NoProof
	\\
	\Theorem{RightConnectorTranspose}
	{
		\forall G \in \TGRP \.
		\forall U \in \U(e) \.
		(U_R)^\top = \Big(\inv(U)\Big)_R
	}
	\NoProof
	\\
	\Theorem{RightConnectorsCompose}
	{
		\forall G \in \TGRP \.
		\forall U,V \in \U(e) \.
		U_R\circ V_R = (VU)_R
	}
	\NoProof
	\\
	\DeclareFunc{rightGroupUniformity}{\prod_{G \in \TGRP} \Unif(G)}
	\DefineNamedFunc{rightGroupUniformity}{}{\R_G}
	{
		\Big\langle\{U_R  | U \in \U_G(e) \}\Big\rangle_\F
	}
}\Page{
	\Theorem{TopologicalGroupIsUniformizableByRightUniformities}
	{
		\forall G \in \TGRP \. G \cong_\TOP (G,\R_G)
	}
	\NoProof
	\\
	\Theorem{TopologicalGroupsAreCompletelyRegular}
	{
		\forall G \in \TGRP \. \CR(G)
	}
	\NoProof
	\\
	\Theorem{SeparatedTopologicalGroupsAreTychonoff}
	{
		\forall G \in \TGRP \. 
		\TYPE{T0}(G) \iff \TYPE{Tychonoff}(G)
	}
	\NoProof
	\\
	\DeclareFunc{upperTwoSidedUniformity}{\prod_{G \in \TGRP} \Unif(G)}
	\DefineNamedFunc{upperTwoSidedUniformity}{}{\S_G^\vee}{\L_G \vee \R_G}
	\\
	\Theorem{TopologicalGroupIsUniformizableByTwoSidedUniformities}
	{
		\forall G \in \TGRP \. G \cong_\TOP (G,\S_G^\vee)
	}
	\NoProof
	\\
	\DeclareFunc{supConnector}
	{
		\prod G \in \TGRP \. \U(e) \to \Connector(G)
	}
	\DefineNamedFunc{supGroupConnector}{U}{U_\vee}{ U_L \cap U_R}
	\\
	\Theorem{TwoSidedUniformityBase}
	{
		\forall G \in \TGRP \. 
		\BofU\Big(G, \S^\vee_G ,\{ U_\vee  | U : \SS(G) \And \U(e)  \} \Big)
	}
	\AssumeIn{U}{\S^\vee_G}
	\Say{\Big(O,[1]\Big)}{\Elim \S^vee_G \Elim U \Elim \L_G }
	{
		\sum O \in \U(e) \. O_L \subset U 	
	}
	\SayIn{E}{O \cap \inv O' }
	{
		\SS(G) \And \U(e)
	}
	\Conclude{[U.*]}{
		\Elim E_\vee 
		\THM{IntersectionIsSubset}(G \times G, E_L,E_R)
		\Elim E
		\THM{LeftConnectorIntersection}(G)
		\THM{IntersectionIsSubset}(G, O,\inv\; O )
		[1]
	}
	{
		E_\vee = E_L \cap E_R \subset E_L \subset O_L \subset U
	}
	\Derive{[*]}{\Intro \BofU}
	{
		\BofU\Big(G, \S^\vee_G ,\{ U_\vee  | U : \SS(G) \And \U(e)  \} \Big)
	}
	\EndProof
	\\
	\Theorem{ClosureInTopologicalGroup}
	{
		\forall G \in \TGRP \.
		\forall A \subset  G \.
		\overline{A} = \bigcap \Big\{ AU  \Big| U \in \U(e)    \Big\} 
	}
	\NoProof
}\Page{
	\Theorem{ClosureInTopologicalGroup1}
	{
		\forall G \in \TGRP \.
		\forall A \subset  G \.
		\overline{A} = \bigcap \Big\{ UA  \Big| U \in \U(e)    \Big\} 
	}
	\NoProof
	\\
	\Theorem{ClosureInversion}
	{
		\forall G \in \TGRP \.
		\forall A \subset G \. 
		\overline{A^{-1}} =   {\overline{A}}^{-1}
	}
	\Conclude{[*]}{
			\THM{ClosureInTopologicalGroup}(G,A)
			\Elim \Aut_\TOP(G,\inv)
			\Elim \GRP(G) \NewLine
			\Intro \FUNC{image}(\inv)
			\THM{ClosureInTopologicalGroup2}(G,A)
	}
	{
		\NewLine :		
		\overline{A^{-1}} = 
		\bigcap \Big\{ A^{-1}U  \Big| U \in \U(e)    \Big\} = 
		\bigcap \Big\{ A^{-1}U^{-1}  \Big| U \in \U(e)    \Big\} =
		\bigcap \Big\{ (UA)^{-1}  \Big| U \in \U(e)    \Big\} =  \NewLine =
		\left( \bigcap \Big\{ AU \Big| U \in \U(e) \Big\} \right)^{-1} =
		{\overline{A}}^{-1}
	}
	\EndProof
	\\
	\Theorem{ConjugationClosure}
	{
		\forall G \in \TGRP \.
		\forall A \subset G \. 
		\forall g \in G \.
		g\overline{A}g^{-1} = 
		\overline{gAg^{-1}}
	}
	\NoProof
	\\
	\Theorem{ClosureMult}
	{
		\forall G \in \TGRP \.
		\forall A,B \subset G \.
		(\overline{A})(\overline{B}) \subset \overline{AB}
	}
	\NoProof
	\\
	\Theorem{OpenProduct}
	{
		\forall G \in \TGRP \.
		\forall U \in \T(G) \.
		\forall A \subset G \.
		UA,UA \in \T(G)
	}
	\NoProof
	\\
	\Theorem{ClosedProduct}
	{
		\forall G \in \TGRP \.
		\forall x \in G \.
		\forall A : \Closed(G) \.
		\Closed( G  ,xA \And Ax )
	}
	\NoProof
	\\
 	\Theorem{InverseIsUniformlyContinuousLR}
 	{
 		\forall G \in \TGRP \.
 		\mathrm{inv}_G \in \UNI\Big( (G,\L), (G,\R) \Big)
 	}
 	\NoProof
 	\\
 	\Theorem{InverseIsUniformlyContinuousRL}
 	{
 		\forall G \in \TGRP \.
 		\mathrm{inv}_G \in \UNI\Big( (G,\R), (G,\L) \Big)
 	}
 	\NoProof
}\Page{
	\Theorem{ClosureOfSubgroup}
	{
		\forall G \in \TGRP \.
		\forall H \subset_\GRP G \.
		\overline{H} \subset_\GRP G 
	}
	\Say{[1]}{
		\THM{SetMultSubset}\Big(G,\overline{H},(\overline{H})^{-1}\Big)
		\THM{ClosureInversion}\Big(G, H \Big)
		\THM{ClosureNult}(G,H,H^{-1})
		\Elim \TYPE{Subgroup}(H)	
	}
	{
		\NewLine :		
		\overline{H} \subset		
		\overline{H} (\overline{H})^{-1} =
		\overline{H} (\overline{H^{-1}}) \subset
		\overline{HH^{-1}} =
		\overline{H}	
	}
	\Say{[2]}{
		\THM{DoubleIneqLemma}
		\Big(?G,\overline{H},\overline{H} (\overline{H})^{-1}\Big)
		[1]
	}
	{
			\overline{H} = \overline{H} (\overline{H})^{-1}
	}
	\Conclude{[*]}{\THM{SubgroupAltDef}[2]}
	{
		\overline{H} \Sgrp G
	}
	\EndProof
	\\
	\Theorem{ClosureOfNormalSubgroup}
	{
		\forall G \in \TGRP \.
		\forall H \Nrml G \.
		\overline{H} \Nrml G 
	}
	\Say{[1]}
	{
		\Lambda g \in G \. 
		\THM{ConjugationClosure}(G,H,g)
		\Elim \TYPE{NormalSubgroup}(G,H)
	}
	{
		\forall g \in G \. 
		g\overline{H}g^{-1} =
		\overline{gHg^{-1}} = 
		\overline{H} 
	}
	\Conclude{[*]}{\Intro \Nrml [1]}
	{
		\overline{H} \Nrml G
	}
	\EndProof
	\\
	\Theorem{AbelianClosureIsAbelian}
	{
		\forall G \in \TGRP \.
		\forall H \Sgrp G \.
		\TYPE{T2}(G) \And H \in \ABEL
		\Imply
		\overline{H} \in \ABEL
	}
	\SayIn{\varphi}{\Lambda g,h \in G \. ghg^{-1}h^{-1}}
	{
		\TOP(G^2,G)
	}
	\Say{[1]}{\THM{T2HasClosedPoints}(G)\THM{ClosedPreimage}(G^2,G,\varphi,\{e\})}
	{	
		\TYPE{Closed}\Big(G \times G,\varphi^{-1}\{e\}\Big)
	}
	\Say{[2]}{\THM{AbelianHasTrivialCommutator}(G,H) \Elim \varphi}
	{
		H \times H \subset  \varphi^{-1}\{e\}
	}
	\Say{[3]}{[1][2]\Elim \FUNC{Closure}(G\times G) \THM{ClosureProduct}(G)}
	{
		\overline{H} \times \overline{H}
		\subset 
		\overline{H \times H}
		\subset
		\varphi^{-1}\{e\}
		\Elim \varphi 
	}
	\Conclude{[*]}{\THM{AbelianByTrivialCommutot}[3]}
	{
		\overline{H} \in \ABEL	
	}
	\EndProof
	\\
	\Theorem{OpenGroupsAreClopen}
	{
		\forall G \in \TGRP \.
		\forall H \Sgrp G \.
		H \in \T(G) \Imply
		\Clopen(G,H)
	}
	\Say{[1]}{
		\THM{Outproduct}(G,H)
		\Elim \FUNC{setProduct}(G,H^\c, H)
		\Elim \TOP(G) [0]
	}
	{
		H^\c = H^\c H = \bigcap_{x \in H^\c} x H \in \T(G)
	}
	\Say{[2]}{\Intro \Closed[1]}{\Closed(G,H)}
	\Conclude{[*]}{\Intro \Clopen [2][0]}{\Clopen(G,H)}
	\EndProof
	\\
	\Theorem{OpenGroupProduceClopenSets}
	{
		\forall G \in \TGRP \.
		\forall H \Sgrp G \.
		\forall A \subset G \.
		H \in \T(G)
		\Imply
		\Clopen(G,AH \And HA)
	}
	\AssumeIn{x}{(AH)^\c}
	\Say{[1]}{\Elim x }{\forall a \in A \. \forall h \in H \. ah \neq x}
	\AssumeIn{h}{H}
	\Assume{[2]}{xh \in AH}
	\Say{[3]}{[2]h^{-1} \Elim \TYPE{Subgroup}(G,H)}{x \in AHh^{-1} = AH}
	\Conclude{[1.*]}{[1][3]}{\bot}
	\Derive{[1]}{\Elim \bot \THM{OpenProduct}(G,H,(AH)^\c)}
	{
		 (AH)^\c = (AH)^\c H  \in \T(G)
	}
	\Say{[2]}{\Intro \TYPE{Closed}[1]}{\TYPE{Closed}(G,AH)}
	\Say{[3]}{\THM{OpenProduct}(G,H,A)}{AH \in \T(G)}
	\Conclude{[*]}{\Intro \Clopen [2][3]}
	{
		\Clopen(G,AH)
	}
	\EndProof
}
\Page{
	\Theorem{OpenGroupIntersection}
	{
		\forall G \in \TGRP \.
		\bigcap \TYPE{Subgroup} \And \T(G) \Nrml G
	}
	\Say{Z}{\bigcap \TYPE{Subgroup} \And \T(G)}{\TYPE{Subgroup}(G)}
	\AssumeIn{g}{G}
	\AssumeIn{z}{Z}
	\Assume{H}{\TYPE{Subgroup} \And \Open(G)}
	\Say{[1]}{\Elim \TGRP}{\TYPE{Subgroup} \And \Open(G,g^{-1}Hg)}
	\Say{[2]}{\Elim Z \Elim x [1]}{z \in g^{-1}Hg}
	\Conclude{[H.*]}{g[2]g^{-1}}{gzg^{-1} \in H}
	\DeriveConclude{[g.*]}{\Intro z}{gzg^{-1} \in Z}
	\Conclude{[*]}{\Intro \TYPE{NormalSubgroup}}{Z \Nrml G}
	\EndProof
	\\
	\Theorem{ClosedGroupIntersection}
	{
		\forall G \in \TGRP \.
		\bigcap \Big\{ H \Sgrp G : \Closed(G,H) \And H \neq \{e\} \Big\}
		\Nrml
		G
	}
	\Say{\A}{
		\Big\{ H \Sgrp H : \Closed(G,H) \And H \neq \{e\} \Big\}	
	}{?\TYPE{Subgroup}(G)}
	\Say{Z}{\bigcap \A}{\TYPE{Subgroup}(G)}
	\AssumeIn{g}{G}
	\AssumeIn{z}{Z}
	\AssumeIn{H}{\A}
	\Say{[1]}{\Elim \TGRP \THM{BijectionPreservesCardinality}(G)\Elim \A}
	{ g^{-1}Hg \in \A   }
	\Say{[2]}{\Elim Z \Elim x [1]}{z \in g^{-1}Hg}
	\Conclude{[H.*]}{g[2]g^{-1}}{gzg^{-1} \in H}
	\DeriveConclude{[g.*]}{\Intro z}{gzg^{-1} \in Z}
	\Conclude{[*]}{\Intro \TYPE{NormalSubgroup}}{Z \Nrml G}
	\EndProof
	\\
	\Theorem{ClosureAnnihilation}
	{
		\forall G \in \TGRP \.
		\forall A,B \subset G \.
		\forall U \in \T(G) \.
		\overline{A}U\overline{B} = AUB
 	}
 	\NoProof
 	\\
 	\Theorem{DoubleClosureExpession}
 	{
 		\forall G \in \TGRP \.
 		\forall A \subset G \.
 		\forall B : \Compacts(G) \.
 		\overline{AB} = \bigcap_{U \in \U(e)} AUB
 	}
 	\NoProof
}
\newpage
\subsubsection{SIN Groups and Uniform Continuity}
\Page{
	\DeclareFunc{LowerTwoSidedUniformity}{\prod_{G \in \TGRP} \Unif(G)}
	\DefineNamedFunc{LowerTwoSidedUniformity}{}{\S_G^\wedge}
	{ \R_G \wedge \L_G  }
	\\
	\DeclareType{SinGroup}{?\TGRP}
	\DefineNamedType{G}{SinGroup}{\TYPE{SIN}(G)}
	{
		\forall U \in \U_G(e) \.
		\exists V \in \U_G(e) \.
		\forall g \in G \.
		gVg^{-1} = V \subset U
	}
	\\
	\Theorem{SINThm}
	{
		\forall G \in \TGRP \. 
		\TYPE{SIN}(G) \iff \L_G = \R_G
	}
	\Explain{$(\Leftarrow)$ 
		By the hypothesis there is $V \in \U(e)$
		such that $V_L \subset U_R$ for every $U \in \U(e)$}
	\Explain{
		This means that $ gV \subset Ug $ for any $g \in G$}
	\Explain{
		But this can be rewritten as $gVg^{-1} \subset U$}
	\Explain{
		$W = \bigcup_{g \in G} gVg^{-1}$ is invariant}
	\Explain{
		And as $U$ was arbitraty then $G$ is SIN
	}
	\Explain{
	$(\Rightarrow)$ 
	Use simmilar derivation, but in the inverse directions}
	\EndProof
	\\
	\Theorem{AbelianIsSIN}
	{
		\forall G \in \TGRP \. G \in \ABEL \Imply \TYPE{SIN}(G) 
	}
	\NoProof
	\\
	\Theorem{CompactIsSIN}
	{
		\forall G \in \TGRP \. \Compact(G) \Imply \TYPE{SIN}(G) 
	}
	\NoProof
	\\
	\Theorem{DiscreteIsSIN}
	{
		\forall G \in \TGRP \. \TYPE{Discrete}(G) \Imply \TYPE{SIN}(G) 
	}
	\NoProof
	\\
	\Theorem{CodiscreteIsSIN}
	{
		\forall G \in \TGRP \. \TYPE{Codiscrete}(G) \Imply \TYPE{SIN}(G) 
	}
	\NoProof
}\Page{
	\Theorem{UniformlyContinuousMult}
	{
		\forall G \in \TGRP \. 
		 (\cdot_G) \in \UNI\Big( (G,\L)\times(G,\R). (G,\L \wedge \R) \Big)
	}
	\NoProof
	\\
	\Theorem{SINByIdUniformContinuityLeft}
	{
		\forall G \in \TGRP \.
		\SIN(G) \iff {\id}_G \in \UNI\Big( (G,\L), (G,\R) \Big)
	}
	\NoProof
	\\
	\Theorem{SINByInvUniformContinuityRight}
	{
		\forall G \in \TGRP \.
		\SIN(G) \iff {\id}_G \in \UNI\Big( (G,\R), (G,\L) \Big)
	}
	\NoProof
	\\
	\Theorem{SINByInvUniformContinuityLeft}
	{
		\forall G \in \TGRP \.
		\SIN(G) \iff {\mathrm{inv}}_G \in \UNI\Big( (G,\L), (G,\L) \Big)
	}
	\NoProof
	\\
	\Theorem{SINByInvUniformContinuityRight}
	{
		\forall G \in \TGRP \.
		\SIN(G) \iff {\mathrm{inv}}_G \in \UNI\Big( (G,\R), (G,\R) \Big)
	}
	\NoProof
	\\
	\Conclude{\FUNC{typicalUniformity}}
	{
		\Lambda G \in \TGRP \.  \mathbb{U}_G = 
		\L_G | \R_G | \L_G \vee \R_G | \L_G \wedge \R_G
	}{\prod G \in \TGRP \.  \Unif(G) }
	\\
	\Theorem{SinByCommutativeConvergence}
	{
		\NewLine ::		
		\forall G : \TGRP \.
		\Big( \forall x,y : \Nat \to G \. \lim_{n \to \infty} x_ny_n = e \iff \lim_{n \to \infty} y_nx_n = e \Big)
		\iff
		\SIN(G) 
	}
	\Explain{
		$(\Leftarrow):$ 
			Assume that $\lim_{n \to \infty} x_ny_n =e$ and take $U \in \U_G(e)$}
	\Explain{
		 	Then as $G$ is SIN there is neighborhood $V \subset U$ of $e$ 
		 	such that $gVg^{-1} = V$ for every $g \in G$}
	\Explain{
		   There is $N \in \Nat$ such that $x_ny_n \in V$ for
		   all $n \ge N$}
	\Explain{
			This means that $y_nx_n \in y_n V y^{-1}_n = V \subset U$ for all $n \ge N$,
			so $\lim_{n \to \infty} y_n x_n = e$ as $U$ was arbitrary}
	\Explain{ 
		The reverse deduction for $y_n x_n$ is trivially the same}
	\Explain{ 
		$(\Rightarrow):?$ }
	\NoProof
}
\newpage
\subsubsection{Меtrics for Twosided Uniformities}
There is a special form of metrics for twosided uniformities.
\Page{
	\DeclareType{\veemetric}{\prod_{G \in \GRP} ?\TYPE{Semimetric}(G)}
	\DefineNamedType{\rho}{\veemetric}{\vee\hyph\TYPE{Semimetric}}
	{
		\exists \sigma : \LIM(G) \. 
		\forall g,h \in G \. \NewLine
		\rho(g,h) = \sigma(g,h) + \sigma(g^{-1},h^{-1})
	}
	\\
	\Theorem{VeeMetricConstructionIsUnique}
	{
		\NewLine ::		
		\forall G \in \GRP \.
		\forall \rho : \veemetric(G)  \.
		\exists ! \sigma : \LIM(G) \.
		\forall g,h \in G \. \NewLine \.
		\rho(g,h) = \sigma(g,h) + \sigma(g^{-1},h^{-1})
	}
	\Say{\Big(\sigma,[1]\Big)}{\Elim \veemetric(G,\rho)}
	{
		\NewLine :		
		\sum \sigma : \LIM(G) \.
		\forall g, h \in G \. \rho(g,h) = \sigma(g,h) +\sigma(g^{-1},h^{-1})
	}
	\Say{[2]}{
		q\Lambda g \in G \.
		\Elim \alpha_\rho(g)[1]
		\Elim \LIM(G,\sigma)
		\Intro \alpha_\sigma  
	}
	{
		\NewLine :		
		\alpha_\rho(g) = 
		\rho(g,e) = 
		\sigma(g,e) +\sigma(g^{-1},e) = 
		2 \sigma(g,e) = 
		2 \alpha_\sigma(g)
	}
	\Say{[3]}{\Intro(=,\to)[2]}{\alpha_\rho = 2\alpha_\sigma}
	\Conclude{[*]}{d\left( \frac{[3]}{2} \right)}{\sigma = d_{\frac{\alpha_\rho}{2}}}
	\EndProof
	\\
	\Theorem{VeeMetricMetrisesUpperTwoSidedUniformity}
	{
		\forall G \in \GRP \.
		\forall \rho : \veemetric(G) \.
		\S^\vee_{(G,\rho)} = \Cell_\rho
	}
	\NoProof
}
\newpage
\subsubsection{Ellis Theorem}
If group has a locally compact Haussdorff topology then it is only enough to have continuous multiplication to show that the topolgy is a group toplogy!  
\Page{
	\DeclareType{EllisTopology}
	{
		\prod_{G \in \GRP} 
		?\TYPE{Topology}(G)
	}
	\DefineType{\T}{EllisTopology}
	{
		\TYPE{T2}(G,\T)
		\And
		\LC(G,\T)
		\And
		\cdot_G \in \TOP(G^2,G)
	}
	\\
	\Theorem{EllisCompactInversionLemma}
	{
		\NewLine ::		
		\forall G \in \GRP \.
		\forall \T : \TYPE{EllisTopology}(G) \.
		\forall K : \Compacts(G,\T) \.
		\Closed\Big( (G,\T), K^{-1}\Big)
	}
	\AssumeIn{b}{\overline{K^{-1}}}
	\Say{\Big(\F,[1]\Big)}{\THM{ClosureByLimits}(G,K^{-1},b)}
	{
		\sum \F : \Filter(K^{-1}) \.
		b = \lim \F
	}
	\Say{a}{\THM{FilterCompact}(K,\inv_*\;\F)}{\TYPE{Cluster}\Big(K,\inv_*\;\F\Big)}
	\Say{\Big(\mathcal{G},[2]\Big)}{
		\THM{ClusterConvergingFilter}
	}
	{
		\sum \mathcal{G} : \TYPE{Filter}(K) \. a = \lim \mathcal{G} \And
		\inv_* \mathcal{F} \subset \mathcal{G} 
	}
	\Say{[3]}{[2.1][1]}{b = \lim \inv_* \mathcal{G}}
	\Say{\L}{\mathcal{G}\times \inv_* \mathcal{G}}
	{
		\Filterbase(K \times K^{-1})
	}
	\Say{[4]}{\Elim \L [1][3]}
	{
		\lim \L = (a,b)	
	}
	\Say{[5]}{\Elim_3 \TYPE{EllisTopology}(G,\T)[4]}
	{
		 \lim (\cdot_G)(\L) = ab
	}
	\Say{[6]}{\Elim \L \THM{FilterLimit}}
	{
		\TYPE{Cluster}\Big( G,  (\cdot_G)(\L), e \Big) 
	}
	\Say{[7]}{
			\Elim_1 \TYPE{EllisTopology}(G,\T)
			\THM{T2HasUniqueClusters}	
	}
	{
			ab = e
	}
	\Conclude{[b.*]}{a^{-1}[7]}{b \in K^{-1}}
	\Derive{[1]}{\Intro \subset}{\overline{K^{-1}} \subset K^{-1}}
	\Say{[*]}{\Elim \FUNC{Closure}[1]}
	{
		\overline{K^{-1}} = K^{-1}
	}	
	\EndProof
}\Page{
	\Theorem{EllisCountableInversionLemma}
	{
		\NewLine ::		
		\forall G \in \GRP \.
		\forall \T : \TYPE{EllisTopology}(G) \.
		\forall A : \TYPE{Countable}(G) \.
		\forall b \in \overline{A} \.
		b^{-1} \in \overline{A^{-1}}
	}
	\Say{\Big(\F,[1]\Big)}{\THM{ClosureByLimits}(G,K^{-1},b)}
	{
		\sum \F : \Filter(K^{-1}) \. b = \lim \F
	}
	\Say{H}{\langle A \cup \{b\}\rangle_\GRP}{\TYPE{Subgroup}(G)}
	\Say{[2]}{\THM{GeneratedByConutableIsCountable}(G)\Elim H}
	{
		|H| \le \aleph_0
	}
	\AssumeIn{K}{\mathcal{K}(e)}
	\AssumeIn{y}{\overline{H}}
	\Say{[3]}{\Elim_3 \TYPE{EllisTopology}(G,\T)}
	{
		yK \in \mathcal{K}(y)
	}
	\Say{[4]}{\THM{ClosureAltDef}(G,H)\Elim \mathcal{K}[3]}
	{
		\exists yK \cap H 
	}
	\Say{\Big(k,[5]\big)}{\Elim \exists [4]}
	{
		\sum k \in K \. yk \in H
	}
	\Conclude{[y.*]}{[5]k^{-1}}{y \in HK^{-1}}
	\Derive{[3]}{\Intro \subset}{\overline{H} \subset HK^{-1}}
	\Say{[4]}{
		\Lambda g \in H \.		
		\Elim_3 \TYPE{EllisTopology}(G,\T)
		\THM{HomeoClusureEq}\Big(
			(G,\T),
			H,
			\lambda_g
		\Big)
		\Elim \TYPE{Subgroup}(G,H)
	}
	{
		\NewLine :		
		\forall g \in G \.
		g\overline{H} =
		\overline{gH} = \overline{H}
	}
	\Say{[5]}{
		[3]
		\Lambda x \in H \. 
		\THM{InverseMeaning}(G,x)
		[4]
	}
	{
		\NewLine :		
		\overline{H} =
		\bigcup_{x \in H} xK^{-1} \cap \overline{H} =
		\bigcup_{x \in H} x(K^{-1} \cap x^{-1}\overline{H}) =
		\bigcup_{x \in H} x(K^{-1} \cap \overline{H}) 
	}
	\Say{[6]}{\THM{EllisCompactInversionLemma}(G,\mathcal{T},K)}
	{
		\Closed\Big((G,\T), K^{-1} \Big)
	}
	\Say{[7]}{
		\Elim_{1,2} 
		\TYPE{EllisTopology}(G,\T)
		\THM{BaireCategoryTHM}
	}
	{
		\TYPE{Baire}(\overline{H},\T \cap \overline{H})
	}
	\Say{\Big(h,[8]\Big)}{\Elim \TYPE{Baire}(\overline{H},\T \cap \overline{H})[5]}
	{
		\sum h \in H \. \exists^* x \in \overline{H} \. x \in h(K^{-1} \cap \overline{H})
	}
	\Say{[9]}{\Elim \exists^* [8]}
	{
		\neg \TYPE{Dense}\Big(
			\overline{H},
			\overline{H}\setminus h(K^{-1} \cap \overline{H})
		\Big)
	}
	\Say{\Big(U,[10]\Big)}{
		\Elim \TYPE{Dense} [9]	
	}
	{
		\sum U \in \T \.  
		U \subset h(K^{-1} \cap \overline{H})
		\And \exists U \cap \overline{H}
	}
	\SayIn{u}{\THM{ClosureAltDef}\Big((G,\T),H,U\Big)[10]}
	{
		H \cap U
	}
	\Say{[11]}{
			[4] \Elim u	
	}
	{
		bu^{-1}(U \cap \overline{H}) =
		bu^{-1}U \cap \overline{H} \in \U_{\overline{H}}(b)
	}
	\Say{\Big(F,[12]\Big)}
	{
		\Elim \TYPE{FilterConvergece}[1][11][10][4]
		\THM{IntersectionIsSubset}
	}
	{
		\NewLine :		
		F \subset bu^{-1}(U \cap \overline{H})
		\subset bu^{-1}h(K^{-1} \cap \overline{H})
		\subset  bu^{-1}hK^{-1}
	}
	\Say{\Big(F,[12]\Big)}
	{
		\Elim \TYPE{FilterConvergece}[1][11][10][4]
		\THM{IntersectionIsSubset}
	}
	{
		\NewLine :		
		F \subset bu^{-1}(U \cap \overline{H})
		\subset bu^{-1}h(K^{-1} \cap \overline{H})
		\subset  bu^{-1}hK^{-1}
	}
	\Say{[13]}{[12]^{-1}}
	{
		F^{-1} \subset Kb^{-1}uh^{-1}
	}
	\Say{a}{
		\THM{CompactHasCluseter}
		\Big(
			  (A,\T \cap A),  Kb^{-1}uh^{-1}, \F^{-1}
		\Big)
	}{\TYPE{Cluster}\Big(
			  (A,\T \cap A),  Kb^{-1}uh^{-1}, \F^{-1}
		\Big)}
	\Say{[14]}{[13]\Elim a}
	{
		a \in Kb^{-1}uh^{-1} \cap \overline{A^{-1}}
	}
	\Conclude{[*]}{\LOGIC{ByAnalogy}\Big(\THM{EllisCompactInversionLemma} \Big)[14]}
	{
		b^{-1} = a \in 	\overline{A^{-1}}
	}
	\EndProof
}\Page{
	\Theorem{EllisInversCompactnessLemma}
	{
		\NewLine ::		
		\forall G \in \GRP \.
		\forall \T : \TYPE{EllisTopology} \.
		\forall K : \Compacts( G,\T) \.
		\Compacts\Big( (G,\T), K^{-1} \Big)
	}
	\Say{[1]}{\THM{EllisCompactInversionLemma}(G,\T,K)}
	{
		\Closed\Big((G,\T),K^{-1}\Big)
	}
	\SayIn{U}{
		\Elim_2 \TYPE{EllisTopology}(G,\T)
		\Elim \LC(G,e)
	}
	{
		\mathcal{K}(e)
	}
	\Assume{[2]}
	{
		\forall A : \Finite(G) \. K^{-1} \not \subset AU
	}
	\Say{\Big(k,[3] \Big)}{\Elim \Nat [2]}
	{
		\sum k : \Nat \to K^{-1} \. 
		\forall n \in \Nat \. k_{n+1} \not \in \bigcup^n_{i=1} k_iU
	}
	\Say{\Big( b,[4]\Big)}{\THM{CompactHasClusetrs}\Big( (G,\T),K, k^{-1} \Big)}
	{
		\sum  b \in K \. \TYPE{Cluster}\Big( (G,\T), k^{-1},b\Big)
	}
	\Say{\Big(V,[5]\Big)}
	{
		\Elim_1 \TYPE{EllisTopology}(G,\T)
		\THM{ProducTopologyBase}(\T,\T,U)
	}{
		\sum V \in \U(e) \. V^2 \subset U
	}
	\Say{\Big(n,[6]\Big)}{\Elim \TYPE{Cluster}\Big( (G,\T), k^{-1},b,Vb\Big)}
	{
		\sum^\infty_{n=1} k^{-1}_n \in Vb
	}
	\Say{[7]}{k_n[6]b^{-1}}{b^{-1} \in k_n V}
	\Say{A}{\{ k_m^{-1} | m > n \}}{?K}
	\Say{[8]}{\Elim \TYPE{Cluster}\Big( (G,\T), k^{-1},b)\Elim A}
	{  b \in \overline{A} }
	\Say{[9]}{\THM{EllisCountableInversionLemma}[8]}
	{
		b^{-1} \in \overline{A^{-1}}
	}
	\Say{\Big(m,[10]\Big)}{\Elim A [9] \THM{ClosureAltdef}(b^{-1}V)}
	{
		\sum m > n \. k_m \in  b^{-1}V
	}
	\Say{[11]}{[10][7][5]}
	{
		k_m \in k_n V^2 \subset k_n U
	}
	\Conclude{[2.*]}{[2][11]}{\bot}
	\Derive{\Big(A,[2]\Big)}{\Elim \bot}
	{
		\exists A : \Finite(G) \. K^{-1} \subset AU
	}
	\Say{[3]}{\Elim_1 \TYPE{EllisTopology}(G,\T) \THM{FiniteCompactUnion}}
	{
		\Compacts\Big( (G,\T), AU \Big)
	}
	\Say{[*]}{\THM{ClosedSubsetOfCompactIsCompact}[2][3]}
	{
		\Compacts\Big( (G,\T), K^{-1} \Big)
	}
	\EndProof
}\Page{
	\Theorem{EllisTheorem}
	{
		\forall G \in \GRP \.
		\forall \T : \TYPE{EllisTopology}(G) \.
		(G,\T) \in \TGRP
	}
	\AssumeIn{U}{\U(e)}
	\Assume{[1]}
	{
		\forall K \in \mathcal{K}(e) \. K^{-1} \not \subset U
	}
	\Say{\F}
	{\Big\{  K^{-1} \cap U^\c \Big| K \in \mathcal{K}(e)  \Big\}}
	{?\Compacts(K,\T)}
	\Say{[2]}{[1]\Elim \F}{\emptyset \not \in \F}
	\Say{[3]}{\Elim_2 \TYPE{EllisTopology}\Elim \LC(G,\T)\Elim \F}
	{
		\F \neq \emptyset
	}
	\Say{[4]}{\Elim \mathcal{K}(e)\Elim \Aut_{\SET}(G,\inv)\Elim \F}
	{
		\forall A,B \in \F \. A \cap B \in \F
	}
	\Say{[5]}{\Intro \TYPE{Filterbase}[2-4]}
	{
		\TYPE{Filterbase}\Big( (G,\T), \F \Big)
	}
	\Say{[6]}{\THM{CantorIntersectionTHM}[5]}
	{
		\exists \bigcap \F
	}
	\Say{[7]}{
		\Elim_{1,2} \TYPE{EllisTopology}(G,\T)
		\THM{RegularNbhdBaseIntersection}
	}
	{
		\bigcap \mathcal{K}(e) = \{e\}
	}
	\Say{[8]}{\THM{BijectionOfIntersection}(G,G,\inv)[7]}
	{
		\bigcap \inv_* \mathcal{K}(e) = \{e\}
	}
	\Say{[9]}{[8]\Elim \F}{\bigcap \F = \emptyset}
	\Conclude{[1.*]}{[9][6]}{\bot}
	\DeriveConclude{\Big(K,[U.*]\Big)}{\Elim \bot}
	{
		\sum K \in \mathcal{K}(e) \. K^{-1} \subset U
	}
	\Derive{[1]}{\Intro C_e}{\inv \in C_e\Big((G,\T),(G,\T)\Big)}
	\Say{[2]}{[1]\Elim_3 \TYPE{EllisTopology}(G,\T)}
	{
		\inv \in \Aut_\TOP(G,\T)
	}
	\Conclude{[*]}{\Intro \TGRP [2]\Elim_3 \TYPE{EllisTopology}(G,\T)}
	{
		(G,\T) \in \TGRP
	}
	\EndProof
}
\newpage
\subsubsection{Topological Groups with Ultrametrics}
Balls around the unity producesed by ultrametric are subgroups, and hence clopen.
\Page{
	\DeclareType{Ultravalue}
	{
		\prod_{G \in \GRP} ?\TYPE{AbsoluteValue}(G)
	}
	\DefineType{\alpha}{Ultravalue}
	{
			\forall a,b \in A \. \alpha(ab) \le \max\Big(\alpha(a),\alpha(b)\Big)
	}
	\\
	\Theorem{UlravalueProduceUltrametric}
	{
		\forall A \in \ABEL \.
		\forall \alpha : \TYPE{Ultravalue}(A) \.
		\TYPE{Ultrametric}(A,d_\alpha)
	}
	\NoProof
	\\
	\Theorem{UltrametricCellsAreSubgroups}
	{
		\forall A \in \ABEL \.
		\forall \alpha : \TYPE{Ultravalue}(A) \.
		\forall r \in \Reals_{++} \
		\Cell(0,r) \Nrml A
	}
	\AssumeIn{a,b}{\Cell(0,r)}
	\Say{[1]}
	{
			\Elim d_\alpha
			\Elim \TYPE{Ultravalue}(A)
			\Intro d_\alpha
			\Elim a,b \in \Cell(0,r)
	}
	{
		\NewLine :		
		d_\alpha(0,a+b) = 
		\alpha(a + b) \le 
		\max\Big(\alpha(a),\alpha(b)\Big) =
		\max\Big(d_\alpha(0,a),d_\alpha(0,b)\Big) <
		r
	}
	\Conclude{\Big[(a,b).*\Big]}
	{
		\Elim \Cell(0,r)[1]
	}
	{
		a + b \in \Cell(0,r)
	}
	\Derive{[1]}{\Intro \forall}
	{
		\forall a,b \in \Cell(0,r) \.
		a + b \in \Cell(0,r)
	}
	\Say{[2]}{\Elim \TYPE{AbsoluteValue}(A,\alpha)}
	{
		\forall a \in \Cell(0,r) \. a^{-1} \in \Cell(0,r)
	}
	\Conclude{[*]}{\Intro \TYPE{Subgroup}}
	{
		\Cell(0,r) \Sgrp G
	}
	\EndProof
	\\
	\Theorem{UltrametrizableGroupHasBaseOfSubgroups}
	{
		\NewLine ::		
		\forall A \in \ABEL \.
		\forall \alpha : \TYPE{Ultravalue}(A) \.
		\exists \mathcal{N} : \NbhdBase(A,\rho) \.
		\forall N \in \mathcal{N} \. N \Nrml A
	}
	\NoProof
	\\
	\Theorem{UltrametrizableGroupHasClopenBalls}
	{
		\NewLine ::		
		\forall G \in \TGRP \.
		\forall \alpha : \TYPE{Ultravalue}(G) \.
		\forall r \in \Reals \.
		\TYPE{Clopen}(G,\Cell_\alpha(e,r))
	}
	\NoProof
	\\
	\Theorem{UltrametrizableGroupsAreZeroDim}
	{
		\NewLine ::		
		\forall G \in \GRP \.
		\forall \alpha : \TYPE{Ultravalue}(G) \.
		\dim_\TOP (G,d_\alpha) = 0
	}
	\NoProof
}
\newpage
\subsubsection{Some Interesting Examples}
Sometimes our basic expectations fail.
\Page{
	\Theorem{SumOfIntegersIsNotClosed}
	{
		\alpha \in \Reals \setminus \Rats  \.
		\neg \Closed(\Reals,\Int + \alpha\Int)
	}
	\Say{A}{\Int + \alpha \Int}{?\Reals}
	\Say{[1]}{\THM{IrrationalGenDense}(\alpha)\Intro A}
	{
		\Dense(\Reals,A)
	}
	\Assume{[2]}{\Closed(\Reals,A)}
	\Say{[3]}{\Elim \Dense(\Reals,A)[2]}{A = \Reals}
	\Say{\Big(n,m,[4]\Big)}{\Elim A [3]\left( \frac{\alpha}{2}\right)}
	{
		\sum n,m \in \Int \.  \frac{\alpha}{2} = n\alpha + m
	}
	\Say{[5]}{[4] - n\alpha}{ \frac{1 - 2n}{2} \alpha = m  }
	\Say{[6]}{\frac{2}{1-2n}[5]\Intro \Rats}{\alpha = \frac{2m}{1-2n} \in \Rats }
	\Conclude{[2.*]}{\Elim \alpha [6]}{\bot}
	\Derive{[*]}{\Elim \bot}{\neg \Closed(\Reals,A)}
	\EndProof
	\\
	\DeclareFunc{PositiveRaysTopolgy}{\TYPE{Topology}(\Int)}
	\DefineNamedFunc{PositiveRaysTopology}{}{\T_{+\infty}}
	{
		\Big\{ [n,\ldots,+\infty) \Big| n \in \Int \Big\} \cup \{\emptyset,\Int\}
	}
	\\
	\Theorem{PositiveRayTopologyHasContinuousAddition}
	{
		(+_\Int) \in \TOP\Big( (\Int,\T_{+\infty})^2,(\Int,\T_{+\infty})\Big)
	}
	\AssumeIn{U}{\T_{+\infty}}
	\Say{\Big(n,[1]\Big)}{\Elim \T_{+\infty}}{
		\sum n \in \Int \. U = [n, + \infty )
	}
	\Say{[2]}{\Elim (+_\Int)[1]}
	{
		(+_\Int)^{-1}U = \Big\{ (k,l) \in \Int^2 \Big|  k + l \ge n   \Big\}		
	}
	\Say{[3]}{\Lambda (k,l) \in (+_\Int)^{-1}U [2]\THM{SumIneq}[2] }
	{
		\forall (k,l) \in (+_\Int)^{-1}U \. 
		[k,\ldots,+\infty) \times (l,\ldots,+\infty) \subset (+_\Int)^{-1}U
	}
	\Conclude{[U.*]}{\THM{ProductTopologyBase}[3]}
	{
		(+_\Int)^{-1}U \in \T\Big( (\Int,\T_{+\infty})^2 \Big)
	}
	\DeriveConclude{[*]}{\Elim \TOP}
	{
		(+_\Int) \in 
		\TOP\Big( (\Int,\T_{+\infty})^2,(\Int,\T_{+\infty})\Big)
	}
	\EndProof
	\\
	\Theorem{PositiveRaysIsNotGroupTopology}
	{
		(\Int,\T_{+\infty}) \not \in \TGRP
	}
	\Say{[1]}{\Elim \T_{+\infty}(0)}{
		[0,\ldots,+\infty)  \in  \T_{+\infty}
	}
	\Say{[2]}{\Elim \inv_\Int}{
		\inv_\Int [0,\ldots,+\infty)
		=  (-\infty, \ldots,0] \not \in \T_{+\infty}
	}
	\Say{[3]}{[1][2]}{\inv_\Int \not \in \Aut_\TOP(\Int,\T_{+\infty})}
	\Conclude{[*]}{\Elim \TGRP [3]}
	{
		(\Int,\T_{+\infty})	\not \in \TGRP
	}
	\EndProof
}\Page{
	\Conclude{G}{\mathbf{GL}(\Reals,2)}{\TGRP}
	\\
	\Conclude{x}{\left( \begin{array}{cc} \dfrac{1}{n} & \dfrac{1}{n^2} \\ 0 & 1 \end{array}  \right)}
	{
		\Nat \to G
	}
	\\
	\Conclude{x}{\left( \begin{array}{cc} n & 1 \\ 0 & 1 \end{array}  \right)}
	{
		\Nat \to G
	}
	\\
	\Explain{  
		$x_n y_n =  \left( \begin{array}{cc} 
			1 & \dfrac{1}{n} +\dfrac{1}{n^2} \\
			 0 & 1 \end{array}  \right) \to  \left( \begin{array}{cc} 1 & 0 \\ 0 & 1 \end{array}  \right)$}
	\\
	\Explain{
		$ y_n x_n = \left( 
			\begin{array}{cc}
			 1 & \dfrac{1}{n} + 1 \\
			 0 & 1	
			\end{array} \right) \to \left( \begin{array}{cc} 1 & 1 \\ 0 & 1 \end{array}  \right)$}
}
\newpage
\subsection{Further Topological Properties}
Topological group structure simplifies work with some topological and metric concepts
\subsubsection{Continuous Homomorphism}
For topological groups it is especially easy to prove that
morphism is continuous.
\Page{
	\Theorem{PointContinuityImplyContinuity}
	{
		\NewLine :		
		\forall G,H \in \TGRP \.
		\forall g \in G \.
		\forall \varphi \in \GRP \And C_g(G,H) \.
		\varphi \in \TGRP(G,H)
	}
	\AssumeIn{h}{\im \varphi}
	\Say{\Big(p,[1]\Big)}{\Elim \im \varphi h}{
		\sum p \in G \. h = \varphi(p)
	}	
	\AssumeIn{U}{\U(h)}
	\Say{[2]}{\Elim \TGRP(H)}{\varphi(g)h^{-1}U \in \U\Big(\varphi(g)\Big)}
	\Say{[3]}{\Elim  C_h(G,H)[2]}{
				\varphi^{-1}\Big( \varphi(g)h^{-1} U \Big) \in \U(g)
	}
	\Conclude{[h.*]}{\Elim \TGRP(G)\Elim \GRP(G,H,\varphi)[1][3]}
	{
	   \varphi^{-1}(U) = pg^{-1} \varphi^{-1}\Big( \varphi(g)h^{-1} U \Big) \in U(p)
	}
	\DeriveConclude{[*]}{\THM{ContinuityIsLocal}}
	{
		\varphi \in \TGRP(G,H)
	}
	\EndProof
	\\
	\Theorem{IdentityOpenessImplyOpeness}
	{
		\forall G,H \in \TGRP \.
		\forall \varphi \in \GRP(G,H) \. \NewLine \.
		\Big( \forall U \in \U_G(e) \. \varphi(U) \in \U_H(e) \Big)
		\Imply \OM(G,H,\varphi)
	}
	\AssumeIn{g}{G}
	\AssumeIn{U}{\U_G(g)}
	\Say{[1]}{\Elim \TGRP}{g^{-1} U \in \U_G(e)}
	\Say{[2]}{[0][1]}{
		\varphi\Big(  g^{-1} U \Big) \in \U_H(e)   
	}
	\Conclude{[g.*]}{
		\Elim \GRP(G,H,\varphi)
		\Elim \TGRP(H)	
	}
	{
		\varphi(U) =  \varphi(g)\Big( g^{-1} U \Big) \in \U_H\Big( \varphi(g) \Big)
	}
	\DeriveConclude{[*]}
	{
		\THM{OpennessIsLocal} 
	}{
			\OM(G,H,\varphi)
	}
	\EndProof
}\Page{
	\Theorem{LeftUnifomityUCCriterion}
	{
		\forall G,H \in \GRP \.
		\forall \varphi : G \to H \.
		\varphi \in \UNI\Big((G,\L_G),(H,\L_H)\Big)
		\iff \NewLine \iff
		\forall V \in \U_H(e) \.
		\exists U \in \U_G(e) \.
		\forall g \in G \.
		\varphi\Big(g U \Big) \subset \varphi(g) V
	}
	\Assume{[1]}{\varphi \in \UNI\Big((G,\L_G),(H,\L_H)\Big)}
	\Assume{B}{\U_H(e)}
	\Say{[2]}{\Elim \L_H(V)}{V_L \in \L_H}
	\Say{\Big(U,[3]\Big)}{
		\Elim \UNI\Big((G,\L_G),(H,\L_H)\Big)(\varphi)
	}{
		\sum U \in \L_G \. (\varphi \times \varphi) U \subset V_L
	}
	\Say{\Big(W,[4]\Big)}{\Elim \BofU\Big(G,\L_G,\big(\U_{G}(e)\big)_L,U\Big)}
	{
		\sum W \in \U_G(e) \. W_L \subset U
	}
	\AssumeIn{g}{G}
	\Say{[5]}{\Elim W_L(g)}{  g W = W_L(g)  }
	\Conclude{[1.*]}{
		\Elim (=)\bigg([5],\varphi\Big( g W \Big)\bigg)
		\Elim \Connector(G, W_L \And U)
		\THM{MonotonicImage}\Big(G,H,W_L(g),U(g)[4]\Big)
		[3]
		\Elim 	V_L\Big( \varphi(g) \Big)
	}
	{
		\NewLine 		
		\varphi\Big( g W \Big) =
		\varphi\Big(  W_L(g)  \Big) \subset 
		\varphi\Big(   U(g) \Big) \subset
		V_L\Big( \varphi(g) \Big)  = \varphi(g) V
	}
	\Derive{[1]}{\Intro(\Imply)}
	{
		\varphi \in \UNI\Big((G,\L_G),(H,\L_H)\Big)
		\Imply
		\forall V \in \U_H(e) \.
		\exists U \in \U_G(e) \.
		\forall g \in G \.
		\varphi\Big(g U \Big) \subset \varphi(g) V
	}
	\Assume{[2]}
	{
		\forall V \in \U_H(e) \.
		\exists U \in \U_G(e) \.
		\forall g \in G \.
		\varphi\Big(g U \Big) \subset \varphi(g) V
	}
	\AssumeIn{V}{\L_H}
	\Say{\Big(W,[3]\Big)}{\Elim \BofU\Big(H,\L_H,\big(\U_{H}(e)\big)_L,V\Big)}
	{
		\sum W \in \U_H(e) \. W_L \subset V
	}
	\Say{\Big(U,[4]\Big)}{[2](W)}
	{
		\sum  U \in \U_G(e) \.
		\forall g \in G \.
		\varphi\Big(g U \Big) \subset \varphi(g) W
	}
	\Conclude{[V.*]}{[4][3]}
	{
		(\varphi \times \varphi) U_L \subset W_L \subset V
	}
	\DeriveConclude{[2.*]}{\Intro \UNI}
	{
		\varphi \in \UNI\Big((G,\L_G),(H,\L_H)\Big)
	}
	\DeriveConclude{[*]}{\Intro (\iff) [1]}
	{
		\NewLine :		
		\varphi \in \UNI\Big((G,\L_G),(H,\L_H)\Big)
		\iff
		\forall V \in \U_H(e) \.
		\exists U \in \U_G(e) \.
		\forall g \in G \.
		\varphi\Big(g U \Big) \subset \varphi(g) V
	}
	\EndProof
	\\
	\Theorem{TopologicalHomomorphismsAreUniformlyContinuous}
	{
		\NewLine ::		
		\forall G,H \in \TGRP \.
		\forall \varphi \in \TGRP(G,H) \.
		\varphi \in \UNI\Big((G,\L_G),(H,\L_H)\Big)
	}
	\NoProof
}
\newpage
\subsubsection{Metrization}
Topologigal groups with a countable base of neighborhood can be metrized in a way compatible with their algebraic structure.
\Page{
		\Theorem{TGRPTrisection}
		{
			\NewLine ::			
			\forall G \in \TGRP \.
			\forall U \in \U(e) \.
			\exists V \in \U(e) \And \SS(G) \.
			VVV \subset U
		}		
		\NoProof
		\\
		\Theorem{LeftGroupMetrization}
		{			
			\forall G \in \TGRP \.
			\forall  \mathcal{N} : \NbhdBase(G,e) \. \NewLine \.
			|\mathcal{N}| \le \aleph_0 \Imply  
			\exists \rho : \LIM(G) \. (G,\rho) \cong_\TOP G 
		}
		\Say{N}{\FUNC{enumerate}(\mathcal{N})}
		{
			\Nat \ToSurj \mathcal{N}
		}
		\Say{\Big( V,[2], \Big)}
		{ 
			\FUNC{rec2}\Big( 
				G, 
				\Lambda n \in \Nat
				\Lambda U \in \U(e) \.
				\THM{TGRPTrisection}(G,U \cap N_n)  
			\Big)  
		}
		{
			\NewLine :			
			\sum \Int_+ \to \U(e) \And \SS(G) \. 
			\forall n \in \Nat \.   V_n V_n V_n \subset U_{n-1}
		}
		\Say{\alpha}
		{
			\Lambda g \in G \. \inf\{ 2^{-n}  | n \in \Int_+ , g \in V_n \}
		}
		{
			G \to \Reals_+
		}
		\Say{[3]}{\Elim \alpha \Elim \Lambda n \in \Int_+ \.  \Elim\SS(G,V_n)}
		{
			\forall g \in G \. \alpha(g) = \alpha(g^{-1})
		}
		\Say{[4]}{\Elim \alpha \Elim \Lambda n \in \Int_+ \. \Elim V_n \U_G(e)}
		{
			\alpha(e) = 0
		}
		\Say{[5]}{\Elim \alpha [2]}
		{
			\forall a,b,c \in G \.
			\alpha(a,b,c) \le 2\max\alpha\Big(\alpha(a),\alpha(b),\alpha(c)\Big)
		}
		\Say{\beta}
		{
			\Lambda g \in G \. 
			\inf \left\{    
				\sum^n_{i=1} \alpha(h_ih^{-1}_{i-1}) \Bigg|
				n \in \Nat, h : \{0,\ldots,n\} \to G , h_0 = e, h_n = g  			
			\right\}      
		}
		{
			G \to \Reals_{++}
		}
		\Assume{h,g}{G}
		\Assume{\varepsilon}{\Reals_{++}}
		\Say{\Big(n,a,[6]\Big)}
		{
			\Elim \beta\left(g,\frac{\varepsilon}{2}\right)
		}
		{
			\sum^\infty_{n=1} \sum_{ \{0,\ldots,n\} \to G }
			a_0 = e \And
			a_n = g \And
			\frac{\varepsilon}{2} + \sum^n_{i=1} \alpha(a_ia^{-1}_{i-1}) = \beta(g)
		}
		\Say{\Big(m,b,[7]\Big)}
		{
			\Elim \beta\left(h,\frac{\varepsilon}{2}\right)
		}
		{
			\sum^\infty_{m=1} \sum_{ \{0,\ldots,m\} \to G }
			b_0 = e \And
			b_n = h \And
			\frac{\varepsilon}{2} + \sum^m_{i=1} \alpha(b_ib^{-1}_{i-1}) = \beta(h)
		}
		\Say{c}{\FUNC{concat}\Big(b, a_{|\{1,\ldots,n\}}h \Big)}
		{
			\{0,\ldots,n+m\} \to G
		}
		\Conclude{\Big[(h,g).*\Big]}{\Elim \beta(gh) \Elim c [6][7]}
		{
			\beta(gh) \le 
			\sum^{n+m}_{i=1} \alpha(c_ic^{-1}_{i+1}) =
			\sum^{n}_{i=1} \alpha(a_ia^{-1}_{i+1}) + 
			\sum^{m}_{i=1} \alpha(b_ib^{-1}_{i+1}) \le
			\beta(g) + \beta(h) + \varepsilon
		}
		\Derive{[6]}{\Intro \forall }
		{
			\forall h,g \in G \. \beta(gh) \le \beta(g) + \beta(h)
		}
		\Say{[7]}{\Elim \beta \Elim \BofU(G,\L_G,\mathcal{N})}
		{
			\forall \F : \Filter(G) \. 
			e \in \lim \F
			\iff
			0 = \lim \beta(\F)
		}
		\Conclude{[*]}{\THM{TopologyByFilters}}
		{
			(G,d_\beta) \cong_\TOP G
		}
		\EndProof
		\\
		\Theorem{RightGroupMetrization}
		{			
			\forall G \in \TGRP \.
			\forall  \mathcal{N} : \NbhdBase(G,e) \. \NewLine \.
			|\mathcal{N}| \le \aleph_0 \Imply  
			\exists \rho : \RIM(G) \. (G,\rho) \cong_\TOP G 
		}
		\NoProof
}\Page{
		\Theorem{VeeGroupMetrization}
		{			
			\forall G \in \TGRP \.
			\forall  \mathcal{N} : \NbhdBase(G,e) \. \NewLine \.
			|\mathcal{N}| \le \aleph_0 \Imply  
			\exists \rho : \veemetric(G) \. (G,\rho) \cong_\TOP G 
		}
		\NoProof
}
\newpage
\subsubsection{Completeness}
Completeness showcases some symmetry between key uniformities
\Page{
	\Theorem{CauchyFilterInversion}
	{
		\forall G \in \TGRP \.
		\forall \F : \TYPE{FilterBase} \. \NewLine \.
		\forall   \CF(G,\L,\F) \iff
		\CF(G,\R,\F^{-1}) 
	}
	\Assume{[1]}{\CF(G,\L,\F)}
	\AssumeIn{U}{\R}
	\Say{\Big(V,[2]\Big)}{\Elim \R(U)}
	{
		\sum V \in \U(e) \. V_R \subset U
	}
	\Say{\Big(F,[3]\Big)}{\Elim \CF(G,\L,\F,V_L^\top) }
	{
		\sum F \in \F \.  F \times F \subset V_L^\top
	}
	\Assume{(x,y)}{V_L^\top}
	\Say{[4]}{\Elim V_L}{ y^{-1}x \in V  }
	\Conclude{\Big[(x,y).*\Big]}{\Intro V_R [4]}
	{
		(x^{-1},y^{-1}) \in V_R
	}
	\Derive{[4]}{\Intro \subset}{(\inv \times \inv)(V_L^\top) \subset V_R }
	\Conclude{[U.*]}{[3][4][2]}{F^{-1} \times F^{-1} \subset U}
	\DeriveConclude{[*]}{\Intro \CF}{\CF(G,\R,\F^{-1})}
	\EndProof
	\\
	\Theorem{TwoSidedCauchyFilters}
	{
		\forall G \in \TGRP \.
		\forall \F : \TYPE{FilterBase} \. \NewLine \.
		\CF(G,\S^\vee,\F) \iff
		\CF(G,\R \And \R,\F) 
	}
	\Conclude{[*]}{\Elim \S^\vee \THM{SupUniformityCauchyFilterbase}(G,(\L,\R))}
	{
		\NewLine :		
		\CF(G,\S^\vee,\F) \iff
		\CF(G,\R \And \R,\F) 
	}
	\EndProof
	\\
	\Theorem{TwoSidedCauchyFilterInversion}
	{
		\NewLine ::		
		\forall G \in \TGRP \.
		\forall \F \in\CF(G,\S^\vee) \.
		\CF(G,\S^\vee,\F^{-1})
	}
	&\text{Combain two previous results.}\\
	\EndProof
	\\
	\Theorem{LeftRightMutualCompleteness}
	{
		\NewLine ::		
		\forall G \in \TGRP \.
		\CUS(G,\L)
		\iff
		\CUS(G,\R)
	}
	\NoProof
	\\
	\Theorem{LeftOrRightCompletenessImplyTwoSided}
	{
		\NewLine ::		
		\forall G \in \TGRP \.
		\CUS(G,\L | \R)
		\Imply
		\CUS(G,\S^\vee)
	}
	\NoProof
}\Page{
	\Theorem{CompleteByNbhd}
	{
		\NewLine ::		
		\forall G \in \TGRP \. 
		\CUS(G,\L_G)
		\iff
		\exists N \in \mathcal{N}(e_G) \.
		\CUS(N,\L_G \cap N^2)
	}
	\Say{[1]}{\Lambda T : \CUS(G,\L_G) \. T}
	{
		\NewLine :		
		\CUS(G,\L_G)
		\Imply
		\exists N \in \mathcal{N}(e_G) \.
		\CUS(N,\L_G \cap N^2)
	}
	\AssumeIn{N}{\mathcal{N}_G(e_G)}
	\Say{\Big(U,[0]\Big)}{\Elim \mathcal{N}_G(e_G)}
	{
		\sum U \in \U_G(e_G) \. U \subset N
	}
	\Assume{[2]}{\CUS(N,\L_G \cap N^2)}
	\Assume{\F}{\CF(G,\L_G)}
	\Say{\Big(F,[3]\Big)}{\Elim \CF(G,\L_G,\F,U_L)}
	{
		\sum F \in \F \. F \times F \subset U_L
	}
	\SayIn{f}{\Elim \Filterbase(G,\F,F)}{F}
	\Say{[4]}{\Elim U_L[3]}{f^{-1}F \subset U}
	\Say{[5]}{\Elim \Filterbase(G,\F)[4]}
	{
		\forall F' \in \F \. f^{-1}F' \cap N \neq \emptyset
	}
	\Say{[6]}{\Elim \TGRP(G) \Intro \CF}
	{\CF\Big( N, \L_G \cap N^2, f^{-1}\F \cap N  \Big)}
	\Say{\Big(g,[7]\Big)}{\Elim\CUS(N,\L_G \cap N^2, f^{-1} \F \cap N)}
	{
		\sum g \in N \. g \in \lim f^{-1} \F \cap N
	}
	\Conclude{[\F.*]}{\Elim \TGRP(G) [7]}{fg \in \lim \F}
	\DeriveConclude{[N.*]}{\Intro \CUS }
	{
		\CUS(G,\L_G)
	}
	\DeriveConclude{[*]}{\Intro \iff [1]}
	{
		\NewLine :		
		\CUS(G,\L_G)
		\iff
		\exists N \in \mathcal{N}(e_G) \.
		\CUS(N,\L_G \cap N^2)
	}
	\EndProof
	\\
	\Theorem{LocallyCompactGroupIsComplete}
	{
		\forall G \in \TGRP \And \LC \.
		\CUS(G,\L)
	}
	\NoProof
	\\
	\Theorem{LocallyCompactGroupRegularity}
	{
		\forall G \in \TGRP \And \LC \And \TYPE{T0} \.
		\TYPE{T4}(G)
	}
	\Assume{E}{\Closed(G)}
	\AssumeIn{\phi}{\TOP(E,\Reals)}
	\Say{\Big( \hat{\id}_E,[1]\Big)}
	{
		\THM{OnePointCompactificationExtenstion}(E,\id)
	}
	{
		\sum \hat{\id}_E \in \UNI(E^\pt,E) \. \hat{\id}_{E|E} = {\id}_E
	}
	\SayIn{\Phi}{\hat{\id}\phi}{\TOP(E^\pt,\Reals)}
	\Say{\Big(\hat\Phi,[2]\Big)}{\THM{TietzeExtensionTheorem}(E^\pt,\Reals,G^\pt,\Phi)}
	{
		\sum \hat\Phi \in \UNI(G^\pt,\Reals) \. \hat\Phi_{|E^\pt} = \Phi
	}
	\Say{\Big( \hat{\id}_G,[4]\Big)}
	{
		\THM{OnePointCompactificationExtenstion}(E,\id)
	}
	{
		\sum \hat{\id}_G \in \UNI(G^\pt,G) \. \hat{\id}_{G|G} = {\id}_G
	}
	\Say{\varphi}{\hat{\id}_G\hat\Phi}{\UNI(G,\Reals)}
	\Conclude{[\phi.*]}{\Elim \varphi}{\varphi_{|E} = \phi}
	\DeriveConclude{[*]}{\THM{TietzeExtensionTHM}}
	{
		\TYPE{T4}(G)
	}
	\EndProof
}
\newpage
\subsubsection{Completion}
One can use symmety properties of Cauchy filters mentioned above to show that separable completion of a sepatable group in its two-sided uniformity is a topological group with a two-sided uniformities again.
\Page{
	\Theorem{LeftCauchyLemma}
	{
		\forall G : \TGRP \.
		\forall \F : \CF(G,\L) \.
		\forall U \in \U(e) \.
		\exists F \in \F \.
		F^{-1}F \subset U
	}
	\NoProof
	\\	
	\Theorem{LeftCauchyFilterProduct}
	{
		\NewLine ::	
		\forall G \in \TGRP \.
		\forall \F,\F' : \CF(G,\L) \.
		\CF(G,\L,\F\F')
	}
	\AssumeIn{U'}{\L}
	\Say{\Big(U,[1]\Big)}{\Elim \L(U')}
	{
		\sum U \in \U(e) \. U_L \subset U'
	}
	\Say{\Big(V,[2]\Big)}{\THM{TGRPTrisection}(G,U)}
	{
		V : \SS(G) \. V \in \U(e) \And VVV \subset U
	}
	\Say{(F',[3])}{\THM{LeftCauchyLemma}(G,\F',V)}
	{
		\sum F' \in \F' \.  {F'}^{-1}F' \subset V
	}
	\SayIn{f}{\Elim \Filterbase(G,\F',F')\Elim \exists}{F'}
	\Say{\Big(W,[4]\Big)}{\THM{TopologicalGroupAltDef}_4(G,V,f^{-1})}
	{
		\sum W \in \U(e) \. 
		f^{-1}Wf \subset V
	}
	\Say{(F,[5])}{\THM{LeftCauchyLemma}(G,\F,W)}
	{
		\sum F \in \F \.  F^{-1}F \subset W
	}
	\Say{[U'.*.1]}{\Elim \FUNC{setProduct}(F,F')}{FF' \in \F\F'}
	\Conclude{[U'.*.2]}{
		\THM{ProductInverse}(G)
		\THM{InverseMeaning}^2(G,f)
		[5]
		[4]
		[3]
		[2]
		[1]
	}
	{
		\NewLine :		
		(FF')^{-1}FF' = 
		{F}'^{-1}F^{-1}FF'^{-1}  =
		{F}'^{-1}ff^{-1}F^{-1}Fff^{-1}F'^{-1} \subset
		{F}'^{-1}ff^{-1}Wff^{-1}F'^{-1} \subset
		{F}'^{-1}f V  f^{-1}{F}'^{-1} \subset \NewLine \subset
		VVV \subset 
		U \subset 
		U'(e)
	}
	\DeriveConclude{[*]}{\Elim \L \Intro \CF}
	{
		\CF(G,\L,\F\F')
	}
	\EndProof
	\\
	\Theorem{RightCauchyFilterProduct}
	{
		\NewLine ::	
		\forall G \in \TGRP \.
		\forall \F,\F' : \CF(G,\R) \.
		\CF(G,\R,\F\F')
	}
	\NoProof
	\\
	\Theorem{UpperTwoSidedCauchyFilterProduct}
	{
		\NewLine ::	
		\forall G \in \TGRP \.
		\forall \F,\F' : \CF(G,\S^\vee) \.
		\CF(G,\S^\vee,\F\F')
	}
	\NoProof
	\\
	\DeclareType{TopologicalGroupCompletion}
	{
		\prod_{G \in \TGRP}
		\sum_{H \in \TGRP}
		\TGRP(G,H) 
	}
	\DefineType{\iota}{TopologicalGroupCompletion}
	{
		\Completion\Big((G,\S^\vee_G),(H,\S^\vee_H),\iota\Big)
	}
}\Page{
	\Theorem{ContinuityByDenseSetAndPoints}
	{
		\forall X \in \TOP \.
		\forall Y : \TYPE{Regular} \.
		\forall \phi : X \to Y \.
		\forall D : \Dense(X) \. \NewLine \.
		\Big( \forall x \in X \. \phi_{|D\cup \{x\}} \in \TOP(D\cup \{x\},Y)\Big) 
		\Imply \phi \in  \TOP(X,Y)
	}
	\AssumeIn{x}{X}
	\AssumeIn{V}{\U(x\;\phi)}
	\Say{C}{\overline{V}}{\Closed(Y)}
	\Say{U}{\phi^{-1}(V) \cap (D \cup \{x\}) }{\U_{D \cup \{x\}}(x)}
	\Say{\Big(W,[2]\Big)}
	{\THM{SubspaceTopology}(X,U)}{\sum W \in \U(x) \. U = W \cap (D \cup \{x\})}
	\Say{[3]}{\Elim W \Elim C}{\phi(W \cap D ) \subset V \subset C}
	\AssumeIn{w}{W}
	\AssumeIn{O}{\U(w\;\phi)}
	\Say{I}{\phi^{-1}(O) \cap (D \cup \{x\}) }{\U_{D \cup \{w\}}(w)}
	\Say{\Big(E,[4]\Big)}
	{\THM{SubspaceTopology}(X,U)}{\sum E \in \U(x) \. I = E \cap (D \cup \{w\})}
	\Say{[5]}{\Elim E}{\phi(E \cap D ) \subset O}
	\Say{[6]}{\Elim E \Elim w \Elim \Dense(X,D)}
	{
		\exists(E \cap W \cap D) 	
	}
	\Say{[7]}{\THM{ImageIntersection}[3][5]}{\phi(W \cap E \cap D) \subset O \cap V}
	\Conclude{[w.*]}{[4][7]\Intro E}{\exists O \cap V}
	\Derive{[x.*]}{\Intro \FUNC{image}\Elim C \THM{ClosureAltDef}}{
		\phi(W) \subset C
	}
	\DeriveConclude{[*]}{
		\THM{ContinuityIsLocalBase}(X,Y,\phi)
		\THM{RegularHasClosedNbhdBase}(Y)}
	{
		\phi \in  \TOP(X,Y)
	}
	\EndProof
	\\
	\Theorem{SeparetedTopologicalGroupHasComplition}
	{
		\forall G \in \TGRP \And \TYPE{T0} \.
		\exists \TGC(G)
	}
	\Say{\Big(H,\iota\Big)}{\THM{SeparatedHasSeparatedCompletion}(G,\S^\vee_G)}
	{
		\TYPE{SeparatedCompletion}(G,\S^\vee_G)	
	}
	\AssumeIn{f,f'}{H}
	\Say{\Big(\F,F',[1]\Big)}{\Elim \TYPE{SeparatedCompletion}(G,H,\iota) }
	{
		\NewLine :		
		\sum \F,\F' : \CF(G,\S^\vee_G) \.
		f = \lim \F \And f' = \lim \F'
	}
	\Say{[2]}{\THM{UpperTwoSidedCauchyFilterProduct}(G,\F,\F')}
	{
		\CF\Big((G,\S^\vee_G),\F\F'\Big) 
	}
	\ConcludeIn{f \cdot_H f'}{\lim \F\F'}{H}
	\Derive{\cdot_H}{\Elim \TGRP(G) \UNI(G,H,\iota)}{H \times H \to H}
	\AssumeIn{f}{H}
	\Say{\Big(\F,[1]\Big)}{\Elim \TYPE{SeparatedCompletion}(G,H,\iota) }
	{
		\sum \F : \CF((G,\S^\vee_G)) \.
		f = \lim \F 
	}
	\Say{[2]}{\THM{TwoSidedCauchyFilterInversion}(G,\F)}
	{
		\CF\Big((G,\S^\vee_G),\F^{-1}\Big) 
	}
	\ConcludeIn{\inv_H f}{\lim \F^{-1}}{H}
	\Derive{\inv_H}{\Elim \TGRP(G) \UNI(G,H,\iota)}{H \to H}
	\Say{[1]}{\THM{ContinuityByDenseSetAndPoints}(H\times H,H)\Elim \cdot_H}
	{
		\cdot_H \in \TOP(H \times H, H)
	}	
	\Say{[2]}{\THM{ContinuityByDenseSetAndPoints}(H,H)\Elim \inv_H}
	{
		\inv_H \in \TOP(H, H)
	}
	\Say{[*.1]}
	{\Elim \TGRP(G) \THM{ContinuityByDenseSetAndPoints}(\ldots)\Intro\TGRP }
	{
		H \in \TGRP
	}
	\Say{[*.2]}{\Elim \cdot_H \Elim \inv_H}{\iota \in \TGRP(G,H)}
	\Conclude{[*.3]}{\Elim \cdot_H \Elim \inv_H}{\U_H = \S^\vee_H}
	\EndProof	
}
\newpage
\subsubsection{Baire's Category}
Being Baire and topological group structure interplay nicely. 
\Page{
	\Theorem{ClopenSubgroupByNonemptyInterior}
	{
		\forall G : \TGRP \.
		\forall H \Sgrp G \.
		\exists \intx H  
		\Imply
		\Clopen(G,H)
	}
	\SayIn{u}{\Elim \exists [0]}{\intx H}
	\Say{\Big( U, [1]\Big)}{\Elim \intx H(u)}
	{
		\sum U \in \U(u) \. U \subset H
	}
	\Say{[2]}{\Lambda h \in H \. \Elim \Aut_{\TOP}(G,\lambda_{hu^{-1}},U)}
	{
		\forall h \in H \.   hu^{-1}U \in \U(h)
	}
	\Say{[3]}{\Lambda h \in H \. \Elim \TYPE{Subgroup}(G, H,hu^{-1},U)[1]}
	{
		\forall h \in H \.   hu^{-1}U \subset H
	}
	\Say{[4]}{\THM{OpenByCover}[2][3]}{\Open(G,H)}
	\Conclude{[5]}{\THM{OpenSubgroupsAreClopen}[5]}{\Open(G,H)}
	\EndProof
	\\
	\Theorem{ClopenSubgroupGeneration}
	{
		\forall G \in \TGRP \.
		\forall U : \SS \And \Open(G) \.
		\Clopen\Big( G, \langle U \rangle_\GRP \Big) 
	}
	\Say{H}{\langle U \rangle_\GRP}{\TYPE{Subgroup}(G)}
	\Say{[1]}{\THM{GeneratedSubgroupIsSuper}(G,U)\Intro H}
	{
		U \subset H
	}
	\Say{[2]}{\Intro \FUNC{interion} [1]}{\exists \intx H}
	\Conclude{[*]}{\THM{ClopenSubgroupByNonemptyInterior}[2]}
	{
		\Clopen(G,H)
	}
	\EndProof
	\\
	\Theorem{FirstCategoryByClopenSets}
	{
		\forall X \in \TOP \. 
		\forall \C : \TYPE{Disjoint}\Big(\Clopen \And \Meager(X)\Big) \.
		X = \bigcup \C
		\Imply
		\neg \exists^* X
	}
	\Say{\Big(N,[1]\Big)}{\Elim ?\Meager(X)}
	{
		\sum N : \C \to \Nat \to \ND(X) \.
		\forall C \in \C \. C = \bigcup^\infty_{n=1} N_{C,n}
	}
	\Say{M}{\Lambda n \in \Nat \. \bigcup_{C \in \C} N_{C,n}}
	{
		\Nat \to ?X
	}
	\Say{[2]}{
		\Lambda n \in \Nat \.
		\Elim M_n 
		\Elim \cl  \Elim\intx
		\Lambda C \in \C \. \Elim \Clopen(X,C)
		\Elim \ND(X,N_{C,n})  
		\THM{EmptySum}(X)
	}
	{
		\NewLine :		
		\forall n \in \Nat \. 
		\intx \cl  M_n = 
		\intx \cl \bigcap_{C \in \C} N_{C,n}  =
		\bigcap_{C \in \C}  \cl \intx  N_{C,n} =
		\bigcap_{C \in \C} \emptyset =
		\emptyset
	}
	\Say{[3]}{\Intro \ND[2]}
	{
		\forall n \in \Nat \. \ND(X,M_n)
	}
	\Say{[4]}{[0][1]\THM{UnionCommutativity}(X)\Intro M_n}
	{
		X = \bigcup_{C \in \C} C = 
		\bigcup_{C \in \C} \bigcap^\infty_{n=1} N_{C,n} = 
		\bigcap^\infty_{n=1} \bigcup_{C \in \C}  N_{C,n} =
		\bigcap^\infty_{n=1} M_n
	}
	\Conclude{[*]}{\Intro \Meager [3][4]}{\Meager(X,X)}
	\EndProof
}\Page{
	\Theorem{TGRPBaireCondition}
	{
		\forall G \in \TGRP \. \Bair(X) \iff \exists^* X
	}
	\Say{[1]}
	{
		\Lambda T : \Bair(X) \. \Elim \Bair(X,X) \Intro \exists^*
	}
	{
		\Bair(X) \Imply \exists^* X
	}
	\Assume{[2]}{\exists^* X}
	\Assume{[3]}{\neg \Bair(X)}
	\Say{\Big(U,[3]\Big)}{\Elim \Bair(X)}
	{
		\sum U \in \T(X) \.
		\exists U \And \neg\exists^* U	
	}
	\SayIn{u}{\Elim \exists[3.1]}{U}
	\SayIn{V}{u^{-1}U}{\U(e)}
	\Say{W}{V \cap V^{-1}}{\SS(X)}
	\Say{[4]}{\Elim W}
	{
		W \in \U(e) \And \Meager(X,W)
	}
	\Say{H}{\langle W \rangle_\GRP}{\TYPE{Subgroup}(G,H)}
	\Say{[5]}{\THM{ClopenSubgroupGeneration}(G,W)\Intro H}
	{
		\Clopen(G,H)
	}
	\Say{[6]}{
		\Lambda n \in \Nat \. 
		\THM{MeagerOpenImage}\left(G^n,G,\prod, W^{\times n}\right)
	}
	{
		\forall n \in \Nat \. \Meager(G,W^n)
	}
	\Say{[7]}{\Elim H \THM{MeagerCountableUnion}(G)[6]}{\Meager(G,H)}
	\Say{\C}{\{ gH | g \in G  \}}{?\Big( \Clopen \And \Meager(G)\Big)}
	\Say{[8]}{\THM{DisjointCosets}(G)\Elim \C}
	{
		\TYPE{Disjoint}\Big( 
			\Clopen \And \Meager(G), \C
		\Big)	
	}
	\Say{[9]}{\Elim \GRP(G) \Elim \C}
	{
		G =  \bigcap_{C \in \C} C
	}
	\Say{[10]}{\THM{FirstCategoryByClopenSets}(G,\C)}
	{
		\neg \exists^*  G 
	}
	\Conclude{[3.*]}{[2][10]}{\bot}
	\DeriveConclude{[2.*]}{\Elim \bot}{\Bair(G,H)}
	\DeriveConclude{[*]}{\Intro \iff [1]}
	{
		\Bair(G,H)
		\iff
		\exists^* G
	}
	\EndProof
	\\
	\Conclude{\BG}{\TGRP \And \Bair}{\Type}
	\\
	\Theorem{DenseGdeltaSubFGroupsAreUnique}
	{
		\forall G : \BG \.
		\forall H \Sgrp G \.
		\Dense \And G_\delta(G,H) \.
		H = G			
	}
	\Say{[1]}{
		\Elim \Dense \And G_\delta(G,H)
		\Intro \Meager(G)
	}{
		\Meager\Big(G,H^\c\Big)
	}
	\Assume{[2]}{G \neq H}
	\Say{[3]}{\Intro \lambda_G[2]}{|\lambda_G G| > 1}
	\Say{[4]}{\Elim \TGRP \THM{MeagerSubset}(G)[1][3]}
	{
		\Meager(G,H)
	}
	\Say{[5]}{\THM{MeagerUnion}[1][4]}{\Meager(G,G)}
	\Conclude{[2]}{\THM{NPGRBairCondition}\Elim \BG(G)[5]}
	{
		\bot
	}
	\DeriveConclude{[*]}{\Elim \bot}{G = H}
	\EndProof
}\Page{
	\Theorem{GDeltaSubgroupIsClosed}
	{
		\forall G : \BG \And \TYPE{CompleteMetricSpace} \.
		\forall H \Sgrp G \. \NewLine \.
		G_\delta(H) \Imply
		\Closed(G,H)
	}
	\Say{[1]}{\THM{ClosureOfSubgroup}(G,H)}{\overline{H} \subset G}
	\Say{[2]}{\THM{ClosedSubsetsAreComplete}(G,\overline{H})}
	{
		\TYPE{CompleteMetricSpace}(\overline{H})
	}
	\Say{[3]}{\THM{BairCategoryTHM2}[2]}
	{
		\Bair(\overline{H})
	}
	\Say{[4]}{\THM{DenseGdeltaSubgroupsAreUnique}(\overline{H},H)}
	{
		H = \overline{H}
	}
	\Conclude{[5]}{\THM{ClosedByClosure}[4]}{\Closed(G,H)}
	\EndProof	
	\\
	\Theorem{DiscontiniousRealEndomorphismsHaveDenseGraphs}
	{
		\NewLine  ::
		\forall \phi \in \End_\GRP(\Reals,+) \.
		\phi \not \in \End_\TOP(\Reals) \Imply
		\Dense\Big(\Reals^2,G(\phi) \Big)
	}
	\NoProof
	\\
	\Theorem{ContinuityByGraph}
	{
		\forall \phi \in \End_\GRP(\Reals,+) \.
		\phi \in \End_\TOP(\Reals) \iff
		G_\delta\Big(\Reals^2, G(\phi)\Big)
 	}
	\NoProof
}
\newpage
\subsubsection{Connectedness}
Connected groups can be generated by any neighborhood of unity.
There is no proofs in this chapter, results are pretty trivial.
\Page{
	\Theorem{ConnectedGroupGeneration}
	{
      \forall G : \TGRP \And \Connected	\. 
      \forall N \in \mathcal{N}(e) \.
      G = \langle N \rangle_\GRP
	}
	\NoProof
	\\
	\Theorem{ConnectedGroupCardinality}
	{
		\forall G : \TGRP \And \Connected \And \TYPE{T0} \.
		G \neq \star \Imply |G| > \aleph_0
	}
	\NoProof
	\\
	\Theorem{ConnectedSubgroupSeparabilityCondition}
	{
		\forall G \in \TGRP \And \Connected \.
		\forall U \in \U(e) \. \NewLine \.
		\Separable(U)
		\Imply \Separable(G)
	}
	\NoProof
	\\
	\Theorem{ConnectedGroupCountability}
	{
		\forall G : \TGRP \And \Connected \And \TYPE{FirstCountable} \And \LC \.
		\NewLine \.
		\TYPE{SecondCountable}(G)
	}
	\NoProof
}
\newpage
\subsubsection{Group of an Interval's Homeomorphisms}
Group of an Interval's Homeomorphisms is an intersting exmple.
it is complete in thir upper two-sided uniformity. 
But not in the on-sided ones.
Theis shows that they don't have corresponding group completions.
\Page{
	\DeclareFunc{homeoAbsoluteValue}
	{
		\TYPE{AbsoluteValue}\Big( \Aut_\TOP[0,1] \Big)
	}
	\DefineNamedFunc{homeoAbsoluteValue}{f}{\upsilon(f)}
	{
		\sup_{t \in [0,1]} \big| t  - f(t)    \big|
	}
	\Say{[1]}{
		\Elim \upsilon(\id)
		\Lambda t \in [0,1] \. \THM{InverseMeaning}(\Reals,t) 
		\Elim | \bullet |
		\Elim \sup 
	}
	{
		\upsilon(\id) = 
		\sup_{t \in [0,1]} \big| t  - t    \big| =
		\sup_{t \in [0,1]} 0	=
		0
	}
	\Say{[2]}
	{
		\Lambda f \in \Aut_\TOP[0,1] \. 
		\Elim \upsilon(f^{-1})
		\LOGIC{Substitution}([0,1],f, y \mapsto f(x))
		\Elim \TYPE{AbsValue}\big(\Reals,|\bullet|\big)
		\Intro 
		\upsilon(f)
	}
	{
		\NewLine :		
		\upsilon(f^{-1}) = 
		\sup_{y \in [0,1]} \big| y  - f^{-1}(y)    \big| =
		\sup_{x \in [0,1]} \big| f(x)  - x \big|  =
		\sup_{x \in [0,1]} \big| x  - f(x) \big| =
		\upsilon(f)
	}
	\Say{[3]}{
		\Lambda f,g \in \Aut_\TOP[0,1]
		\Elim  \upsilon(fg)
		\LOGIC{Substitution}([0,1],f^{-1}, x \mapsto f^{-1}(y) )
		\THM{TriangleIneq}\Big(\Reals,|\bullet|,f^{-1}(y),y,g(y)\Big)
		\NewLine 
		\THM{SupSumIneq}(\Reals)
		\Intro \upsilon(f^{-1})
		\Intro \upsilon(g) 
		[2]
	}
	{
		\upsilon(fg) =
		\sup_{x \in [0,1]}  \big| x  - fg(x)    \big| =
		\sup_{y \in [0,1]} \big|  f^{-1}(y) - g(y) \big| \le \NewLine \le 
		\sup_{y \in [0,1]} \big|  f^{-1}(y) - y| + \big| y -  g(y) \big| \le 
		\sup_{y \in [0,1]} \big|  y- f^{-1}(y) \big| + 
		\sup_{y \in [0,1]} \big| y -  g(y) \big| \le
		\upsilon(f^{-1}) + \upsilon(g) \le 
		\upsilon(f) + \upsilon(g)
	}
	\Assume{f}{\Nat \to \Aut_\TOP[0,1]}
	\Assume{[4]}
	{
		\lim_{n \to \infty}  \upsilon(f_n) = 0
	}
	\AssumeIn{g}{\Aut_\TOP[0,1]}
	\Conclude{[f.*]}
	{
			\lim_{n \to \infty} \Elim \upsilon(g^{-1}f_ng)
			\LOGIC{Substitution}([0,1],g, x \mapsto g(y) )
			\Elim \Aut_\UNI\Big([0,1],g\Big)[4]
	}
	{
		\NewLine :		
		\lim_{n \to \infty} \upsilon(g^{-1}f_ng) =
		\lim_{n \to \infty} \sup_{x \in [0,1]} \big| x  - g^{-1}f_n g(x)\big| =
		\lim_{n \to \infty} \sup_{x \in [0,1]} \big| g(y)  - f_ng(y) \big| = 0
	}
	\DeriveConclude{[*]}{\Intro \TYPE{AbsoluteValue}}
	{
		\TYPE{AbsoluteValue}\Big( \Aut_\UNI[0,1],\upsilon \Big)
	}
	\EndProof
	\\
	\Theorem{HomeoAbsValueProducesUniformMetric}
	{
		\forall f,g \in \Aut_\TOP[0,1] \.
		d_\upsilon(f,g) = \sup_{t \in [0,1]} \Big| f(t) -g(t) \Big| 
	}
	\Conclude{[*]}{
		\Elim d_\upsilon
		\Elim \upsilon
		\LOGIC{Substitution}([0,1],f, f(x) \mapsto x )	
	}
	{
		\NewLine :		
		d_\upsilon(f,g) =
		\upsilon(g^{-1}f) =
		\sup_{x \in [0,1]} \big| x  - g^{-1}f (x)\big| =
		\sup_{y \in [0,1]}  \big| g(y)  - f(y) \big| 
	}
	\EndProof
	\\
	\Theorem{HomeoAreNotLeftComplete}
	{
		\neg \CUS\Big( \Aut_\TOP[0,1],\upsilon,\L \Big)
	}
	\Assume{[1]}{\CUS\Big( \Aut_\TOP[0,1],\upsilon,\L \Big)}
	\Say{[2]}{
		\THM{HomeoAbsValueProducesUniformMetric}
		\Elim
		\CUS\Big( \Aut_\TOP[0,1],\upsilon,\L \Big)
		\Intro \Closed 
	}
	{
		\NewLine :		
		\Closed\Big( \End_\TOP[0,1],\Aut_\TOP[0,1] \Big)
	}
	\Say{f}
	{
		\Lambda n \in \Nat \.
		\Lambda t \in [0,1] \.
		2\sqrt[n]{\frac{1}{2}} t\left[t < \frac{1}{2} \right] + 
		\sqrt[n]{t} \left[t \ge \frac{1}{2} \right]
	}{
		\Nat \to \Aut_\TOP[0,1]
	}
	\Say{[3]}{\Elim \lim_{n \to \infty} f_n}
	{
		\lim_{n \to \infty} f_n =  
		2t\left[t < \frac{1}{2} \right] + \left[t \ge \frac{1}{2} \right]
		\not \in \Aut_\TOP[0,1]
	}
	\Conclude{[1.*]}{\THM{ClosedHasLimits}\Big( \Aut_\TOP[0,1]\Big)[2][3]}
	{
		\bot
	}
	\Derive{[*]}{\Elim \bot}{\neg \CUS\Big( \Aut_\TOP[0,1],\upsilon,\L \Big)}
	\EndProof
} 
\Page{
	\Theorem{HomeoIsUpperComplete}
	{
		\CUS\Big(\Aut_\TOP[0,1],\upsilon,\S^\vee\Big)
	}
	\Assume{f}{\TYPE{CauchySeq}\Big(\Aut_\TOP[0,1],\upsilon,\S^\vee\Big)}
	\Say{[1]}{\THM{TwoSidedCauchyFilters}(\Aut_\TOP[0,1],\upsilon,f)}
	{
		\NewLine :		
		\TYPE{CauchySeq}\Big( \Aut_\TOP[0,1], \| \bullet \|_\infty, f\Big)
		\And
		\TYPE{CauchySeq}\Big( \Aut_\TOP[0,1], \| \bullet \|_\infty, f^{-1}\Big)
	}
	\Say{\Big(g,[2]\Big)}{
		\Elim \CUS\Big( \End_\TOP[0,1], \| \bullet \|_\infty, f  \Big)
	}
	{
		\sum g \in \End_\TOP[0,1] \. \lim_{n \to \infty} f_n = g
	}
	\Say{\Big(h,[3]\Big)}{
		\Elim \CUS\Big( \End_\TOP[0,1], \| \bullet \|_\infty, f^{-1}  \Big)
	}
	{
		\sum h \in \End_\TOP[0,1] \. \lim_{n \to \infty} f_n^{-1} = h
	}
	\Assume{x}{[0,1]}
	\Assume{\varepsilon}{\Reals_{++}}
	\Say{\Big(N,[4]\Big)}{\Elim \upsilon [3](\varepsilon)}
	{ 
		\sum N \in \Nat \. 
		\forall n \ge N \.		
		\Big| gh(x) - gf_n^{-1}(x)  \Big|  < \frac{\varepsilon}{2}
	}
	\Say{\Big(M,[5]\Big)}{\Elim \upsilon [2](\varepsilon)}
	{ 
		\sum M \in \Nat \. 
		\forall m \ge M	\.
		\Big| gf_m^{-1} - x  \Big|  < \frac{\varepsilon}{2}
	}
	\SayIn{n}{\max(N,M)}{\Nat}
	\Conclude{[x.*]}{\THM{TriangleIneq}[4][5]}
	{
		\Big| gh(x) - x \Big| \le  
		\Big| gh(x) - gf_n^{-1}(x)  \Big|
		+
		\Big| gf_n^{-1} - x  \Big| < \varepsilon
	}
	\Derive{[4]}{\Intro \inv}{ gh = \id }
	\Say{[5]}{\LOGIC{ByAnalogy}[4]}{hg=\id}
	\Conclude{[f.*]}{\Elim \Aut_\TOP[0,1][4][5]}
	{
		g \in \Aut_\TOP[0,1]
	}
	\DeriveConclude{[*]}{\Intro \CUS}
	{
		\CUS\Big(\Aut_\TOP[0,1],\upsilon,\S^\vee\Big)
	}
	\EndProof
	\\
	\Theorem{HomeoHasDistinctUniformities}
	{
		\L_\upsilon \neq \S^\vee_\upsilon
		\And
		\L_\upsilon \neq \R_\upsilon
	}
	\NoProof
	\\
	\Theorem{HomeoHasNoLeftGroupCompletion}
	{
		\neg \exists \TGC\Big( \End_\TOP[0,1],\upsilon, \L   \Big)
	}
	\NoProof
}
\newpage
\subsection{Further Group Properties}
 \subsubsection{Initial and Final  Structure} 
 \Page{ 		
 		\DeclareFunc{initialGroupTopology}
 		{
 			\prod G \in \GRP \. 
 			\prod  I \in \SET \.
 			\left( \prod_{i \in I} \sum_{H _i\in \TGRP} \GRP(G,H_i)  \right) \to \Group\Top(G) 
 		}
 		\DefineNamedFunc{initialGroupTopology}{H,\phi}{\W_G(I,H,\phi)}{\W_G(I,H,\phi)}
 		\Explain{Define function 
 			$q : G\times G \to G$ by $q(a,b) = ab^{-1}$}
 		\Explain{
 			Then $\phi_i \circ q(a,b) = \phi_i(ab^{-1}) = \phi_i(a)\phi^{-1}_i(b)$ 
 			is continuous for all $i \in I$}
 		\Explain{
 			So $q$ also must be continuous by properties of the topological group}
 		\Explain{  
 			Thus, $(G,\I_G(I,H,\phi))$ is a topological group}
 		\EndProof
 		\\
 		\Theorem{LeftUniformityInitialization}
 		{
			\NewLine :: 			
 			\forall G \in \GRP \.
 			\forall I \in \SET \.
 			\forall \prod_{i \in \I} (H_i,\phi_i) \sum_{H_i \in \TGRP} \GRP(G,H_i) \.
 			\L_{(G,\W(I,H,\phi))} = \I\Big(I,(H,\L),\phi\Big)
 		}
 		\NoProof
 		\\
 		\Theorem{LeftUniformityInitialization}
 		{
			\NewLine :: 			
 			\forall G \in \GRP \.
 			\forall I \in \SET \.
 			\forall \prod_{i \in \I} (H_i,\phi_i) \sum_{H_i \in \TGRP} \GRP(G,H_i) \.
 			\R_{(G,\W(I,H,\phi))} = \I\Big(I,(H,\R),\phi\Big)
 		}
 		\NoProof
 		\\
 		\Theorem{UpperUniformityInitialization}
 		{
			\NewLine :: 			
 			\forall G \in \GRP \.
 			\forall I \in \SET \.
 			\forall \prod_{i \in \I} (H_i,\phi_i) \sum_{H_i \in \TGRP} \GRP(G,H_i) \.
 			\S^\vee_{(G,\W(I,H,\phi))} = \S^\vee\Big(I,(H,\L),\phi\Big)
 		}
 		\NoProof
 		\\
 		\Theorem{SinTopologyIsInitializedBySin}
 		{
 			\NewLine ::
 			\forall G \in \GRP \.
 			\forall I \in \SET \. 
 			\forall \prod_{i \in I} (H_i,\phi_i) \sum_{H_i : \SIN} \GRP(G,H_i) \.
 			\SIN(G,W(I,H,\phi))
 		}
 		\Explain{
 			Use the fact that
 			$
 				\L_{(G,\W(I,H,\phi))} = \I\Big(I,(H,\L),\phi\Big) =
 				\I\Big(I,(H,\R),\phi\Big) = \R_{(G,\W(I,H,\phi))}
 			$ 
 		}
 		\EndProof
 		\\
 }\Page{
 	\Theorem{TGRPSubgroupIsTGRP}
 	{
 		\forall G \in \TGRP \.
 		\forall H \subset_GRP  G \.
 	    H \in \TGRP
 	}
 	\NoProof
 	\\
 	\Theorem{TGRPSupIsTGRP}
 	{
 		\forall G \in \GRP \.
 		\forall I \in \SET \.
 		\forall \tau : I \to \Group\Top(G) \.
 	    \left(G, \bigvee_{i \in I} \tau_i \right) \in \TGRP
 	}
 	\NoProof
 	\\
 	\Theorem{TGRPProductIsTGRP}
 	{
 		\forall I \in \SET \.
 		\forall G : I \to \TGRP(G) \.
 	    \prod_{i \in I} G_i \in \TGRP
 	}
 	\NoProof
 	\\
  	\DeclareFunc{finalGroupTopology}
 	{
 			\prod G \in \GRP \. 
 			\prod  I \in \SET \.
 			\left( \prod_{i \in I} \sum_{H _i\in \TGRP} \GRP(H_i, G)  \right) \to \Group\Top(G) 
 	}
 	\DefineNamedFunc{finalGroupTopology}{H,\phi}{\S_G(I,H,\phi)}{
		\bigvee  \Big\{ \tau : \Group\Top(G), \forall i \in I \. \phi_i \in \TOP(H_i,G \Big)\} 	
 	}
 	\\
 	\DeclareFunc{approximateGroupTopology}
 	{
 			\prod G \in \GRP \. 
 			\Top(G) \to \Group\Top(G) 
 	}
 	\DefineNamedFunc{approximateGroupTopology}{\tau}{\tau_0}{
		\bigvee  \Big\{ \sigma : \Group\Top(G), \sigma \subset \tau\} 	
 	}
 	\\
 	\DeclareFunc{sinGroupTopology}
 	{
 			\prod G \in \GRP \. 
 			\Top(G) \to \Group\Top(G) 
 	}
 	\DefineNamedFunc{approximateGroupTopology}{\tau}{\tau_0}{
		\bigvee  \Big\{ \sigma : \Group\Top(G), \SIN(G,\sigma) ,  \sigma \subset \tau\} 	
 	}
 }
 \newpage
\subsubsection{Subgroups}
\Page{
	\Theorem{ClosedCenterTHM}
	{
		\forall G \in \TGRP \And \TYPE{T2} \.
		\Closed\big( G, Z(G) \big)     
	}
	\Explain{
		If $g \in G$ then $Z(g) = z^{-1}_g\{e\}$, where $z_g(h) = hgh^{-1}g^{-1}$}
	\Explain{
		As $z_g$ is continuous, the subgroup $Z(g)$ is closed}
	\Explain{
		So, $Z(G) = \bigcap_{g \in G} Z(g)$ has to be closed}
	\EndProof
	\\
	\Theorem{ClosedSubgroupByOpenSet}
	{
		\NewLine ::		
		\forall G \in \TGRP \.
		\forall H \subset_\GRP G \.
		\forall h \in H \.
		\forall U \in \U(h) \.
		\Closed(U,H \cap G) \Imply \Closed(G,H)
	}
	\Explain{   
		$h^{-1}U \cap H \subset h^{-1}U \in \U(e)$, obviously}
	\Explain{
		So there is a symmetric open subset $V \in \U(e)$ such that
		$V \cap H$ is closed in $V$}
	\Explain{
		Assume $b \in \overline{H}$}
	\Explain{
		$c \in b(V \cap H) \neq \emptyset$, thus 
		$b \in cV^{-1} = cV = c \intx V$}
	\Explain{
		Hence $b \in {\cl}_{cV} cV \cap H = {\cl}_{cV} c(V \cap H)$}
	\Explain{
		But then $b \in cV \cap H$ as $V \cap H$ is closed in $V$, whence $b \in H$}
	\EndProof
	\\
	\Theorem{LocallyCompactSubgroupIsClosed}
	{
		\NewLine ::
		\forall G \in \TGRP \And \TYPE{T2} \.
		\forall H \subset_\GRP G \.
		\LC(H) \Imply \Closed(G,H)
	}
	\Explain{
		If $h \in H$ then there is $U \in \U(h)$ such that 
		$H \cap U$ is a compact subset of $U$}
	\Explain{
			But then $H \cap U$ is closed, 
			so by previous result  $H$ is closed}
	\EndProof
	\\
	\Theorem{SubgroupLeftUniformity}
	{
		\forall G \in \TGRP \.
		\forall H \subset_\GRP G \. 
		\L_H = \L_G | H
	}
	\NoProof	
	\\
	\Theorem{SubgroupRightUniformity}
	{
		\forall G \in \TGRP \.
		\forall H \subset_\GRP G \. 
		\R_H = \R_G | H
	}
	\NoProof	
	\\	
	\Theorem{SubgroupUpperSymmetricUniformity}
	{
		\forall G \in \TGRP \.
		\forall H \subset_\GRP G \. 
		\S_H^\vee = \S_G^\vee | H
	}
	\NoProof	
}\Page{
	\Theorem{SubgroupLowerSymmetricUniformity}
	{
		\forall G \in \TGRP \.
		\forall H \subset_\GRP G \. 
		 \S_G^\wedge | H \subset \S_H^\wedge
	}
	\NoProof	
}
\newpage
\subsubsection{Products}
\Page{
	\DeclareFunc{lexProduct}
	{
		\prod I \in \SET \. (I \to \GRP) \to \GRP
	}
	\DefineNamedFunc{lexProduct}{G}{\bigodot_{i \in I} G_i }
	{
		\left\{ g \in \prod_{i \in I} G_i \bigg|  \Big| \{ i \in I | g_i \neq e  \} \Big| < \infty   \right\}
	}
	\\
	\Theorem{LexProductIsDense}
	{
		\forall I \in \Set \.
		\forall G  : I \to \GRP \.
		\Dense \And \TYPE{Normal}\left( \prod_{i \in I} G_i, \bigodot_{i\in I} G_i \right)
	}
	\Explain{
		Take $g \in \prod_{i \in I} G_i$ }
	\Explain{
		Take $U \in \U(g)$
	}
	\Explain{
		Then there is a finiite $F \subset I$ such that 
		$g \in \prod_{i \in F} V_i \times \prod_{i \in \F^\c} G_i$,
		where each $V_i$ is open  in $G_i$}
	\Explain{
		define $g' = \Lambda i \in I \. \If i \in F \Then g_i \Else e_{G_i} \in \bigodot_{i \in I} G_i$}
	\Explain{
		Then $g' \in U$, and as $U$ and $g$ was arbitrary 
		$\bigodot_{i \in I} G_i$ is dense}
	\EndProof
}
\newpage
\subsubsection{Group Action}
\Page{
	\DeclareType{TopologicalGroupAction}
	{
		\prod G  : \TGRP \. \prod X : \TOP \. ?\TYPE{Action}(G,X)
	}
	\DefineNamedType{\omega}{TopologicalGroupAction}{\omega : G \ActOn_{\TOP} X}
	{
		 \omega \in \TOP( G \times X,  X)	
	}
	\\
	\Theorem{TopologicalActionActsByHomeo}
	{
		\NewLine ::		
		\forall G \in \TGRP \.
		\forall X \in \TOP \.
		\forall \omega : G \ActOn_{\TOP} X \.
		\forall g \in \.
		\Lambda x \in X \. gx \in \Aut_\TOP(X)
	}
	\Explain{ 
		Consider $g^{-1}$ as a continuous inverse for $g$}
	\EndProof
	\\
	\Theorem{QuotientMapIsOpen}
	{
		\forall G \in \TGRP \.
		\forall X \in \TOP \.
		\forall \omega : G \ActOn_{\TOP} X \.
		\Open\left(G, \frac{X}{\omega}, \pi_\omega \right)
	}
	\Explain{
		Assume $U$ is open in $X$
	}
	\Explain{
		When $\pi^{-1}_\omega \pi_\omega(U) = \bigcup_{g \in G} g(U)$ 
		is open in $X$ as $g$ is a homeomorphism}
	\Explain{
		As $\frac{X}{\omega}$ is equiped with final topology
		the set $\pi_\omega(u)$ must be open}
	\EndProof
	\\
	\DeclareType{Invariant}{\prod_{G \in \GRP} \prod_{X \in \SET} \. \TYPE{Action}(G,X) \to ?X } 
	\DefineType{I}{Invariant}{
		\Lambda \alpha : G \ActOn X \. 
		\forall a \in A \. O_\alpha(a) \subset A 
	}
	\\
	\DeclareType{InvariantRelation}
	{\prod_{G \in \GRP} \prod_{X \in \SET} \. \TYPE{Action}(G,X) \to ?X^2 } 
	\DefineType{R}{InvariantRelation}{
		\Lambda \alpha : G \ActOn X \.
		\forall (a,b) \in R \.
		\forall g \in G \. 
		(ga,gb) \in R }
	\\
	\DeclareType{InvariantSet}{\prod_{G \in \GRP} \prod_{X \in \SET} \. \TYPE{Action}(G,X) 
		\to ??X^2 } 
	\DefineType{\I}{InvariantSet}{
		\Lambda \alpha : G \ActOn X \. 
		\forall R \in \I \. \TYPE{InvariantRelation}(X,G,R) 
	}
	\\
	\Theorem{InvariantOpenMap}
	{
		\forall G \in \TGRP \.
		\forall X \in \TOP \.
		\forall \omega : G \ActOn_\TOP X \. 
		\forall A \in \TYPE{Invariant}(G,X,\omega) \. \NewLine \.	
		\Open(A,\pi_\omega(A),\pi_{\omega|A} )
	}
	\Explain{ 
		Same proof as in previous theorem}
	\Explain{
		Clearly, $\pi^{-1}_\omega \pi_\omega(U) = \bigcup_{g \in G} g(U) \subset A$
		as $A$ is invariant
	}
	\\
	\EndProof
	\Theorem{InvariantHomeomorphism}
	{
		\forall G \in \TGRP \.
		\forall X \in \TOP \.
		\forall \omega : G \ActOn_\TOP X \. 
		\forall A \in \TYPE{Invariant}(G,X,\omega) \. \NewLine \.
		\forall g \in G \.	
		g \in \Aut_\TOP(A)
	}
	\Explain{ 
		This is obvious}
	\EndProof
}\Page{
	\DeclareType{CompatibleEq}
	{
		\prod G \in \TGRP \.
		\prod X \in \TOP \.
		\TYPE{TopologicalAction}(G,X) \to ?\Eq(X)
	}
	\DefineType{E}{CompatibleEq}
	{
		\Lambda \omega : G \ActOn_\TOP X \.
		\forall g \in G \. \exists \varphi \in \TOP\left( \frac{X}{E} ,\frac{X}{E}\right) \.
		 \omega(g) \pi_G = \pi_G \varphi	
	}
	\\
	\DeclareFunc{inducedAction}
	{
		\NewLine ::
		\prod G \in \TGRP \.
		\prod X \in \TOP \.
		\prod \omega : G \ActOn_\TOP X \. 
		\prod E : \TYPE{CompatibleEq}(X,G,\omega) \.
		G \ActOn \frac{X}{E} 
	}
	\DefineNamedFunc{inducedAction}{}{\widetilde \omega}
	{ \Lambda g \in G \. \Lambda [x] \in \frac{X}{E} \. 
		\Elim \TYPE{CompatibleEq}(G,X,\omega,E,g)[x]
	}
	\\
	\Theorem{ContinuityOfInducedAction}
	{
		\NewLine ::
		\forall G \in \TGRP \.
		\forall X \in \TOP \.
		\forall \omega : G \ActOn_\TOP X \.
		\forall E : \TYPE{CompatibleEq}(X,G,\omega) \. \NewLine \.
		\Open \And \TOP\left(X,\frac{X}{E},\pi_E\right) 
		\Imply
		\widetilde\omega : G	\ActOn_\TOP \frac{X}{G}
	}
	\Explain{
		Obviously $\omega \pi_E = (\id \times \pi_E) \widetilde \omega_E$}
	\Explain{ 
		Assume $U$ is an open set in $\frac{X}{E}$
	}
	\Explain{
		Then $
				V  = \Big((\id \times \pi_E)\widetilde \omega_E\Big)^{-1}(U) = 
				(\omega \pi_E)^{-1}(U) 
			$
			is open
	}
	\Explain{
		But then $\widetilde\omega^{-1}(U) = (\id \times \pi_E)(V)$ is open 
		as $(\id \times \pi_E)$ is open and surjective.
	}
	\EndProof
	\\
	\DeclareFunc{QuotientUniformity}
	{
		\prod (X,\U) \in \UNI \. 
		\prod E : \Eq(X) \. \Unif\left( \frac{X}{E} \right)
	}
	\DefineNamedFunc{QuotientUniformity}{}{\frac{\U}{E}}
	{
		\F\left( \frac{X}{E} ,\pi_E\right)
	}
	\\
	\DeclareFunc{QuotientUniformSpace}
	{
		\prod X \in \UNI \. \Eq(X) \to \UNI
	}
	\DefineNamedFunc{QuotientUniformity}{E}{\frac{X}{E}} 
	{
		\left(\frac{X}{E}, \frac{\U_X}{E} \right)
	}
	\\
	\Theorem{CechUniformQuotientTheorem}
	{
		\NewLine ::		
		\forall (X,\U) \in \UNI \.
		\forall E : \Eq(X) \. \NewLine \.
		\Big( \forall M \in \U \. \exists N \in \U \. N \circ E \circ N \subset E \circ M \circ E \Big)
		\iff
		\UB\left(\frac{X}{E},\frac{\U}{E}, \pi_E \times \pi_E(\U) \right) 
	}
	\NoProof
	\\
	\DeclareType{UniformlyOpen}
	{
		\prod (X,\U),(Y,\W) \in \UNI \. ?\UNI\Big( (X,\U),(Y,\W) \Big)
	}
	\DefineType{\varphi}{UniformlyOpen}
	{
		\forall U \in \U \. \exists W \in \W \. 
		\forall x \in X \.
		W\left( \varphi(x) \right) \subset   \varphi\left( U(x) \right)
	}
}\Page{
	\Theorem{UniformQuotientTheorem2}
	{
		\NewLine ::		
		\forall (X,\U) \in \UNI \.
		\forall E : \Eq(X) \. \NewLine \.
		\Big( \forall M \in \U \. \exists N \in \U \. E \circ N \subset N \circ E \Big)
		\iff \NewLine \iff
		\left(
			\UB\left(\frac{X}{E},\frac{\U}{E}, \pi_E \times \pi_E(\U) \right)
			\And
			\forall M \in \U \. \exists P \in \frac{\U}{E} \. \forall x \in X \.  
				P\Big(\pi_E(x)\Big) \subset \pi_E\Big( M(x) \Big) 
		\right)
		 \NewLine \iff
		\TYPE{UniformlyOpen}\left(X,\frac{X}{E},\pi_E \right)
	}
	\NoProof
	\\
	\Theorem{UniformGroupAction}
	{
		\NewLine ::		
		\forall G \in \GRP \.
		\forall X \in \UNI \.
		\forall \omega : G \ActOn X \. \NewLine \.
		\exists \UB(X) \And \TYPE{InvariantSet}(G,X,\omega) 
		\iff
		\TYPE{UniformlyEquicontinuous}\Big(  X, X, \omega(G) \Big)
	}
	\NoProof
	\\
	\Theorem{InvariantRelationsCommute}
	{
		\NewLine ::		
		\forall G \in \GRP \.
		\forall X \in \UNI \.
		\forall \omega : G \ActOn X \.
		\forall M : \TYPE{InvariantRelation}(G,X,\omega) \.
		M \circ E_\omega = E_\omega \circ M
	}
	\NoProof
	\\
	\Theorem{InvariantUniformityMonotonicity}
	{
		\NewLine ::		
		\forall G \in \GRP \.
		\forall X \in \SET \.
		\forall \omega : G \ActOn X \. \NewLine \.
		\TYPE{Transitive}(G,X,\omega) \Imply		
		\forall \U,\U' : \Unif(X) \.
		\forall \B,\B' : \UB(X) \And \TYPE{InvariantSet}(G,X,\omega)  \.
		\NewLine \.
		\B \subset \B' \Imply \U \subset \U'
	}
	\NoProof
	\\
	\Theorem{InitialInvariantBasis}
	{
		\NewLine ::		
		\forall I,X \in \SET \.
		\forall G \in \GRP \.
		\forall Y : I \to \UNI \.
		\forall \alpha : G \ActOn X \.
		\forall \omega : \prod_{i \in I} \omega_i :G \ActOn Y_i \.
		\forall  f : \prod_{i \in I} X \to Y_i \. \NewLine \.
		\forall \B : \prod_{i \in I} \UB(Y_i) \And \TYPE{InvariantSet}(G,Y_i,\omega_i) \.
		\NewLine \.
		\forall \aleph : \forall i \in I \. \forall g \in G \. f_i \omega_i(g) = \alpha(g) f_i \.
		\exists \UB(X) \And \TYPE{InvariantSet}(G,X,\omega_i)
	}
	\NoProof
}\Page{
	\Theorem{FinestInvariantUniformityExists}
	{
		\forall (X,\U) \in \U \.
		\forall G \in \GRP \.
		\forall \omega : G \ActOn X \.
		\NewLine \.
		\exists \bigvee 
		\{ \V : \Unif(X), \V \subset \U,  \exists \UB(X,\V) \And \TYPE{InvariantSet}(G,X,\alpha)   \}
	}
	\NoProof
	\\
	\DeclareFunc{invariantKernel}
	{
		\prod_{ G \in \GRP} \prod_{X \in \SET} \.
		\TYPE{Action}(G,X) \to ?X^2 \to ?X^2  
	}
	\DefineNamedFunc{invariantKernel}{\omega,R}{\ker_\omega R}
	{\bigcap_{g \in G} (\omega(g) \times \omega(g))(R)}
	\\
	\Theorem{CoarsestInvariantUniformityExists}
	{
		\forall (X,\U) \in \U \.
		\forall G \in \GRP \.
		\forall \omega : G \ActOn X \.
		\NewLine \.
		\exists \bigwedge 
		\{ \V : \Unif(X), \U \subset \V,  \exists \UB(X,\V) \And \TYPE{InvariantSet}(G,X,\omega_i)   \} =
		\NewLine =
		\Big\langle\{
			\ker_\omega U | U \in \U
		\}\Big\rangle_\UNI
	}
	\NoProof
	\\
	\\
	\Theorem{FinalUniformBasis}
	{
		\NewLine ::		
		\forall I,Y \in \SET \.
		\forall G \in \GRP \.
		\forall X : I \to \UNI \.
		\forall \omega : G \ActOn Y \.
		\forall \alpha : \prod_{i \in I} \omega_i :G \ActOn X_i \.
		\forall  f : \prod_{i \in I} X_i \to Y \. \NewLine \.
		\forall \B : \prod_{i \in I} \UB(X_i) \And \TYPE{InvariantSet}(G,X_i,\alpha_i) \. \NewLine
		\. \aleph : \forall g \in G \. \forall i \in I \.  f_i \alpha(g) = \omega_i(g) f_i \.
		\exists \UB(Y) \And \TYPE{InvariantSet}(G,Y,\omega)
	}
	\NoProof
	\\
	\DeclareFunc{permanent}
	{
		\prod_{X \in \SET} (\Nat \to ?X^2) \to X^2 
	}
	\DefineNamedFunc{permanent}{M}{\perm M}
	{
		\bigcup^\infty_{n=1} \bigcup_{\sigma \in S_n} \bigcirc^n_{k=1} M_{\pi(k)} 
	}
	\\
	\Theorem{BasisPermanent}
	{
		\forall X \in \UNI \.
		\forall \B : \UB(X) \.
		\UB\Big( X, \{ \perm B | B : \Nat \to \B \} \Big)
	}
	\NoProof
	\\
	\DeclareFunc{genenerateUniformity}
	{
		\prod_{X \in \SET} \TYPE{DownwardDirected}\big(??X^2\big) \to \Unif(X)
	}
	\DefineNamedFunc{generateUniformity}{\mathfrak{A}}{\langle \mathfrak{A} \rangle_\UNI}
	{
		\bigvee \Big\{ 
			\V : \Unif(X),  
			\forall {V \in \V}  \. 
			\forall \A \in \mathfrak{A}\.
			 \exists_{A \in \A} 
		 	A \subset V   					
		 \Big\}
	}
}\Page{
	\DeclareType{PreUniformity}
	{
		\prod_{X \in \SET} ?\TYPE{WithDiagonal}(X) \And \TYPE{Filterbase}(X^2)
	}
	\DefineType{\U}{PreUniformity}
	{
		\forall U \in \U \. \exists V \in \U \. U^\top \subset V
	}
	\\
	\Theorem{MonotonicPreUniformityBase}
	{
		\NewLine ::		
		\forall X \in \SET \.
		\forall \U : \Nat \downarrow \TYPE{PreUniformity}(X) \.
		\UB\left( \langle \U \rangle_\UNI,  \left\{ \perm U | U : \prod^\infty_{n=1} \U_n    \right\}  \right)   
	}
	\\
	\Theorem{FinalUniformityProduct}
	{
		\NewLine ::		
		\forall I \in \SET \.
		\forall X : I \to \UNI \.
		\forall Y : I \to \SET \. 
		\forall f : \prod_{i \in I} (X_i \to Y_i) \. \NewLine \.
		\prod_{i \in \I} \Big(Y_i, \F(1,X_i,f_i) \Big) =
		\left( \prod_{i \in \I} Y_i, \F\left(1,\prod_{i \in I} X_i, \prod_{i \in I} f_i \right) \right)
	}
	\NoProof
	\\
	\DeclareFunc{groupPermanent}
	{
		\prod_{G \in \GRP} (\Nat \to ?G) \to ?G
	}
	\DefineNamedFunc{groupPermanent}{A}{\perm A}{
		\bigcup^\infty_{n=1} \bigcup_{\sigma \in S_n} \prod^n_{k=1} A_{\sigma(k)}
	}
	\\
	\Theorem{BasisPermanent}
	{
		\forall G \in \TGRP \.
		\forall \B : \TYPE{Base}(X) \.
		\TYPE{Base}\Big( G, \{ \perm B | B : \Nat \to \B \} \Big)
	}
	\NoProof
	\\
	\DeclareFunc{generateGroupTopologyByFilterBase}
	{
		\prod G \in \GRP \.  ?\TYPE{Filterbase}(G) \to \Group\Top(G)
	}
	\DefineNamedFunc{generateGroupTopologyByFilterBase}{\mathfrak{F}}
	{
		\langle \mathfrak{F} \rangle_\TGRP
	}
	{
		\bigvee\Big\{ \T : \Group\Top(G), \forall \F \in \mathfrak{F} \. \lim \F = e  \Big\}
	}
	\\
	\DeclareType{GroupFilterbase}
	{
		\prod_{G \in \GRP} ?\TYPE{Filterbase}(G)
	}
	\DefineType{\F}{GroupFilterbase}
	{
		\forall A \in \F \. \exists B \in \F \. B^{-1} \subset A
		\And 
		\forall A \in \F \. \forall x \in X \. \exists B \in \F \. xB x^{-1} \subset A
	}
	\\
	\Theorem{MonotonicGroupTopologyBase}
	{
		\NewLine ::		
		\forall G \in \GRP 
		\forall \F : \Nat \downarrow \TYPE{GroupFilterbase}(G) \.
		\TYPE{Base}\left( \U_{(G,\langle \F \rangle_\UNI)}(e),  
		\left\{ \perm F | F : \prod^\infty_{n=1} \F_n    \right\}  \right)   
	}
	\NoProof
}\Page{
	\Theorem{StrongGroupTopologyRight}
	{
		\NewLine ::		
		\forall I \in \SET \.
		\forall G : I \to \TGRP \.
		\forall H \in \GRP \. 
		\forall \phi : \prod_{i \in I} \GRP( G_i, H ) \. \NewLine \.
		\R_{  \Big(H, \S(I,G,\phi) \Big)} =
		\F_{H}\Big(I,(G,\R),\phi)
	}
	\NoProof
	\\
	\Theorem{StrongGroupTopologyLeft}
	{
		\NewLine ::		
		\forall I \in \SET \.
		\forall G : I \to \TGRP \.
		\forall H \in \GRP \. 
		\forall \phi : \prod_{i \in I} \GRP( G_i, H) \. \NewLine \.
		\L_{  \Big(H, \S(I,G,\phi) \Big)} =
		\F_{H}\Big(I,(G,\L),\phi)
	}
	\NoProof
	\\
	\Theorem{StrongGroupTopologySIN}
	{
		\NewLine ::		
		\forall I \in \SET \.
		\forall G : I \to \SIN \.
		\forall H \in \GRP \. 
		\forall \phi : \prod_{i \in I} \GRP( G_i, H) \. \NewLine \.
		\SIN\Big(H, \S(I,G,\phi) \Big)
	}
	\NoProof
}
\newpage
\subsubsection{Quotients}
\subsubsection{Isomorphism Theorems}
\subsubsection{Free Topological Groups}
\subsection{Some Analytic Properties[1]}
Many famous theorems of functional analysis can be proved for topological groups.
This chapter will be written on demand in a first return.
Prerequisite: Further Group Properties
\subsubsection{Closed Graph Theorem}
\subsubsection{Nearly Open Mapping Theorem}
\subsubsection{Baire Category and Morphisms}
\subsubsection{Haar Measures}
\subsubsection{Unimodular Groups}
\newpage
\subsection{Representation[5]}
Results about topological groups can be applied to their representations.
Prerequisite: Further Group Properties, Some Analytic Properties,Haar measure, group representations
\subsubsection{Unitary Representations}
\subsubsection{Continuous Characters}
Characters of topological groups are also Continuous
\Page{
	\DeclareFunc{dualGroup}{\TGRP \to \TGRP}
	\DefineNamedFunc{dualGroup}{G}{G^*}{\TGRP(G,\Sphere^1)}
}
\newpage
\subsubsection{Representation with Haar Measure}
\subsubsection{Bhor Compactification}
\subsubsection{Keller's Theorem[6]}
Keller's theorem states that every compact convex body 
in a Hilbert space is homeomorphic to a Hilbert cube.
In 1993 Agaev published a proof, which uses representation of topological groups as a main tool. Prerequisite: Spectral theory
\subsection{Almost Metrizable Groups[$\infty$]}
Almost metrizable spaces are those where every compact admits a countable system of neighborhood. Almost metrizable groups are of some esoteric interest. 
\newpage
\section{Polish Groups}
\subsection{Basics}
\subsubsection{Definition and Examples}
\Page{
	\Conclude{\PG}{\TGRP \And \Polish}{\Type}
	\\
	\DeclareType{\cli}{?\PG}
	\DefineType{G}{\cli}
	{
		\exists \rho : \LIM(G) \.
		\Complete(G,\rho) \And (G,\rho) \cong_\TOP G
	}
	\\
	\Theorem{DiscreteIsCli}
	{
		\forall G \in \TGRP \.
		|G| \le \aleph_0 \And \TYPE{Discrete}(G)
		\Imply
		\cli(G)
	}
	\NoProof
	\\
	\Theorem{NiceIsCli}
	{
		\forall G \in \TGRP \.
		\TYPE{FirstCountable} 
		\And
		\LC
		\And
		\TYPE{T0} \.
		\cli(G)
	}
	& \text{First countable topological groups are LIM metrizable.} \\
	& \text{First countable topological groups are second countable.}\\
	& \text{Second countable and locally compact metrizable spaces are polish.}\\
	& \text{Locally compact topological groups are complete in their left uniformity,
		and hence LIM-complete.}
	\EndProof
	\\
	\Theorem{RealsAndComplexAreCli}
	{
		\cli\Big(
			(\Reals,+),
			(\Complex,+),
			(\Reals\setminus\{0\},*),
			(\Complex\setminus\{0\},*)
		\Big)	
	}
	& \text{Result for additive groups is well known fact of Reals Analysis} \\
	& \text{For multiplicative groups nice is cli.}\\  
	\EndProof
	\\
	\Theorem{ProductOfPolishGroupsIsPolish}
	{
		\forall n \in \sigma(\omega) \.
		\forall G : n \to \PG \.
		\PG\left( \prod^n_{i=1} G\right)
	}
	&\text{product of topological groups is a topological group.} \\
	&\text{Countable product  of polish spaces is polish.}\\
	\EndProof
}\Page{
	\Theorem{UnitaryOperatorsAreCli}
	{
		\forall V \in \HIL{\Complex} \And \Separable \.
		\cli\Big(\mathbf{U}(V)\Big)	
	}
	\Say{[1]}
	{
		\Lambda A,B,C \in \mathbf{U}(V) \.
		\Lambda x \in V \.
		\Elim \mathsf{RING} \Big( \mathbf{B}(V) \Big)
		\Elim \TYPE{Isometry}(V,C)
	}
	{
		\NewLine :		
		\forall A,B,C \in \mathbf{U}(V) \.
		\forall x \in V \.
		\Big\| (CA - CB)x \Big\|  =
		\big\| C(A - B)x \big\|  =
		\big\|(A - B)x\big\|
	}
	\Say{[2]}{\Intro \FUNC{operatorNorm}(V)[1]}
	{
		\forall A,B,C \in \mathbf{U}(V) \.
		\big\| C(A - B)\big\| = \big\| A - B \big\|
	}
	\Derive{[3]}{\Intro \LIM [2]}
	{
		\LIM\Big( \mathbf{U}(V), \Lambda A,B \in \mathbf{U}(V) \. \|A-B\|  \Big)
	}
	\Say{[4]}{
			\THM{MultiplicationContinuousOnBoundedSets}(V,\mathbf{U}(V))
			\THM{AdjoiningIsContinuous}(V)
			\Intro \TGRP
		}
	{
		\NewLine \.		
		\mathbf{U}(V) \in \TGRP
	}
	\Say{[5]}{
		\Elim \mathbf{U}(V)	
	}	
	{
		\mathbf{U}(V) =
		\Big( 
			\Lambda A \in \Sphere\Big(\B(V)\Big) \. A^*A		
		\Big)^{-1}(\id)
		\cap
		\Big( 
			\Lambda A \in \Sphere\Big(\B(V)\Big) \.
			A^*A
		\Big)^{-1}(\id)
	}
	\Say{[6]}
	{
		\THM{MultiplicationContinuousOnBoundedSets}(V,\mathbf{U}(V))
		\Intro \Closed
	}
	{
		\Closed\Big(\B(V),\mathbf{U}(V)\Big)
	}
	\Say{[7]}{\THM{ClosedSetsAreGDelta}[5]}
	{
		G_\delta\Big(\B(V),\mathbf{U}(V) \Big)
	}
	\Say{[8]}{\THM{CompleteSubsetIsGDelta}}
	{
		\Complete\Big(\mathbf{U}(V)\Big)
	}
	\Conclude{[*]}{\Intro \cli [7][4][3]}{\cli\Big( \mathbf{U}(V) \Big)}
	\EndProof
	\\
	\Theorem{UnitaryOperatorsAreNotLocallyCompact}
	{
		\forall V : \HIL{\Complex} \And \Separable \.
		\neg \LC\Big(\mathbf{U}(V) \Big)
	}
	& \text{Fix any $v \in V, \|v\| =1 $, then the evaluation map $\epsilon_v$ 
	  		is a bounded, surjective operator on $\mathcal{B}(V)$.} \\
	& \text{Then by open mapping theorem, $\epsilon_v$ is an open mapping.} \\
	& \text{As $\epsilon_v$ also surjective it preserves local compactness.} \\
	& \text{$\epsilon_v$ maps $\mathbf{U}(V)$ onto $\Sphere(0,1)$, 
			but Spheres are not locally Compact in $V$}    \\ 
	\EndProof
	\\
	\Theorem{CompactMetrizableHomeoArePolish}
	{
		\forall X : \Compact \And \TYPE{Metrizable}(X) \.
		\PG\Big(\Aut_\TOP(X)\Big) 
	}
	\NoProof
	\\
	\Theorem{HomeoOfIntervalAreNotCli}
	{
		\neg \cli\Big( \Aut_\TOP[0,1] \Big)
	}
	\EndProof
	\\
	\Theorem{IsometryGroupIsCLI}
	{
		\NewLine ::		
		\forall X \in \MS \.
		\Complete \And \Separable(X) 
		\Imply
		\cli\Big( \Aut_{\Iso}(X),\W(X,X,\epsilon) \Big)
	}
	\NoProof
}\Page{
	\Theorem{InfinitePermutationsArePolishGroup}
	{
		\PG(S_\infty)
	}
	\NoProof
	\\
	\Theorem{IsometryGroupIsCompact}
	{				
		\forall X \in \MS \.
		\Compact(X)
		\Imply
		\Compacts\Big( \Aut_\TOP(X), \Aut_\Iso(X)\Big)
	}
	& \text{
		Assume $f$ is a sequence of isometries.}\\
	& \text{
		Let $(E_{n})^\infty_{n=1}$ be a sequence of finite $\frac{1}{n}$-nets, such
		that $E_{n+1}\subset E_n$ .
	} \\
	& \text{
		Then, it is possible to select a collection $(g^n)^\infty_{n=1}$ 
		of subsequences of  $f$}\\ 
	& \text{
		Let $g^n$ be converging on each $x \in X_n$ to
		some $L(x)$  and $\forall m \in \Nat \. d\Big(L(x),g^n_m\Big) \le \frac{1}{m}$.}
	\\
	\Explain{
	This is possible as $X$ is compact and $E_{n}$ is finite}
	\Explain{
	Set $h_n = g^n_n$, Then $h$ is converging $D=\bigcup^\infty_{n=1} E_n$ 
	to an isometry $L$}
	\Explain{
		But $D$ is dense in $X$ and $L$ is uniformly continuous, so there is an extension
		$\widehat{L}$ over whole $X$.
	}
	\Explain{
		$\widehat{L}$ is an isometry as metric is continuous.
	}
	\Explain{
		We show that $h$ converges to $L$ pointwise.
	}
	\Explain{
		Assume $x \in X$, assume $\varepsilon \in \Reals_{++}$.
	}
	\Explain{
		Then there is an $y \in D$ such that $d(x,y) < \varepsilon$}
	\Explain{
		Also there is an $N \in \Nat$, such that $d\Big(L(y),h_n(y)\Big) < \varepsilon$
		for all $n \ge N$. Let $n$ be such 
	}
	\Explain{
		So,  $d\Big(\widehat{L}(x),h_n(x)\big) \le 
			d\Big(\widehat{L}(x),\widehat{L}(y)\Big)
			+
			d\Big(\widehat{L}(y),h_n(y)\Big)
			+
			d\Big(h_n(y),h_n(x) \Big) =
			2 d(y,x) + d\Big( L(y), h_n(y) \Big) <
			3\varepsilon
		$
	}
	\Explain{ 
		Hence, $f$ has a subseqequence which converges to $L$ pointwise.}
	\Explain{
		As $f$ was arbitraty, it means that group of isometries is compact.
	}
	\EndProof
	\\
	\Theorem{NatIdealIsGDelta}
	{
		\forall I : \Ideal(\Nat) \. G_\delta(\C,I) \Imply \Closed(\C,I)
	}
	\Explain{
		Cantor's space seen as $\C = ?\Nat = \prod^\infty_{n=1}\frac{\Int}{2\Int}$ is a
		polish group, hence Baire
	}
	\Explain{
		Note, that ideals are subgroups for this structure
	}
	\Explain{
		So, from the theorem in chapter 2.2.5 of this treatise the result follows 
	}
	\EndProof
	\\
	\Theorem{
		FrechetIdealOfNatIsOnlyFSigma
	}
	{
		F_\sigma\Big( \C, \Finite(\Nat)\Big)
		\And
		\neg G_\delta\Big( \C, \Finite(\Nat)\Big)
	}
	\Explain{
		Set of finite subsets is countable and $\C$ is separated,so it is $F_\sigma$
	}
	\Explain{
		By previous result, if Frechet Ideal was $G_\delta$ it would be Closed
	}
	\Explain{ 
		But Finite subsets are dense in $\C$ so this contradicts the definition of
		Frechet's ideal
	}
	\EndProof
}
\newpage
\subsubsection{Baire Groups Redux}
Polish groups are Baier, so Baire property will be useful.
\Page{
	\Theorem{PettisBPTheorem}
	{
		\forall G \in  \BG \.
		\forall A \in \BP(G) \.
		\exists^* A \Imply
		\exists U \in \U(e) \. U \subset A^{-1} A
	}
	\Say{\Big(U,E,[1]\Big)}
	{
		\Elim \exists^* A
	}
	{
		\sum U \in \T(G) \.
		\sum E : \Meager(G) \.		
		A = U \du E
		\And
		\exists U 
	}
	\Say{\Big([2]\Big)}
	{
		\Elim \exists^* A
	}
	{
		\sum U \in \T(G) \.
		\sum E : \Meager(G) \.		
		A = U \du E
		\And
		\exists^* U 
	}
	\SayIn{g}{\Elim \exists^* U}
	{
		U \cap A
	}
	\Say{\Big( V,[3]\Big)}
	{
		\THM{TopologicalGroupAltDef}(G,g^{-1}U)
	}
	{
		\sum V \in \U(e) \. VV^{-1} \subset g^{-1}U
	}
	\Say{[4]}{\Elim \FUNC{SetProduct}[3]}
	{
		\forall v \in V \.    gV \subset U \cap Uv
	}
	\AssumeIn{v}{V}
	\Say{[6]}{\LOGIC{CheckingBooleanTables}}
	{
		(U \cap Uv) \du (A \cap Av) 
		\subset (U \du A) \cup  (U \du A)v =
		E \cup Ev		
	}
	\Say{[7]}{\THM{MeagerUnion}(G)\THM{MeagerSubset}(G)[6]}
	{
		\Meager\Big( G, (U \cap Uv) \du (A \cap Av) \Big)
	}
	\Conclude{[v.*]}
	{
		\Lambda T : A \cap Av = \emptyset \. 
		T[7]\Elim \Bair(G)\Elim \bot
	}
	{
		A \cap Av \neq \emptyset
	}
	\Derive{[5]}{\Intro \forall}
	{
		\forall v \in V \.  A \cap Av \neq \emptyset
	}
	\Conclude{[*]}{\Intro \FUNC{SetPtoduct}}
	{
		V \subset A^{-1}A
	}
	\EndProof
	\\
	\Theorem{BaireGroupMeasurableIsContinuous}
	{
		\NewLine :		
		\forall G  : \BG.
		\forall H \in \TGRP \And \Separable \.
		\forall \phi \in \GRP\And\BM(G,H) \.
		\phi \in \TOP(G,H)
	}
	\Say{\Big(h,[1]\Big)}{\Elim \Separable(H)}
	{
		\sum \Nat \to h \. \Dense(H,\im h)
	}
	\Assume{U}{\U_H(e)}
	\Say{\Big(V,[2]\Big)}
	{
		\THM{TopologicalGroupAltDef}(H,U)
	}
	{
		\sum V \in \U_H(e) \. V^{-1}V \subset U
	}
	\Say{[3]}{\Elim \Dense [1](V)}{H = \bigcup^\infty_{n=1} h_n V}
	\Say{[4]}{\THM{UniversalPreimage}(G,H,\phi)[3]\THM{PreimageUnion}(G,H,\phi)}
	{
		\NewLine :		
		G = \phi^{-1}(H) =
		\phi^{-1} \left(\bigcup^\infty_{n=1} h_n V\right) =
		\bigcup^\infty_{n=1} \phi^{-1}(h_n V)
	}
	\Say{\Big(n,[5]\Big)}
	{
		\Elim \Bair(G)[4]
	}
	{
		\sum^\infty_{n=1} \exists^* \phi^{-1}(h_n V)
	}
	\Say{\Big(W,[6]\Big) }{\THM{PettisBPTheorem}(G,\phi^{-1}(h_n V))}
	{
		\sum W \in \U_G(e) \.  
		W \subset  \Big(\phi^{-1}(h_n V)\Big)^{-1} \phi^{-1}(h_n V)
	}
	\Conclude{[U.*]}{\phi([6])[2]}
	{
		\phi(W) \subset V^{-1}V \subset [U]
	}
	\Derive{[2]}{\Intro C_e}{\phi \in C_e(G,H)}
	\Conclude{[*]}{\THM{PointContinuityImplyContinuity}}
	{
		\phi \in \TOP(G,H)
	}
	\EndProof
	\\
	\Theorem{
		NonMeagerBPSubgroupIsClopen
	}
	{
		\forall G : \BG \.
		\forall H \subset_\GRP G \.
		\BP(H) \And \exists^* H \Imply \Clopen(H)
	}
	\Explain{
		By Pettis Theorem there is an open $U\subset HH^{-1} = H$
	}
	\Explain{
		But, then $H  = UH$ is open, and hence clopen	
	}
	\EndProof	
}
\Page{
	\Theorem{RealSetWithoutBPExists}{
		 \BP^\c(\Reals) \neq \emptyset
	}
	\Explain{
		Let $h$ be a Hamel basis for $\Reals$ taken as $\Rats$-vector space.
	}
	\Explain{
		Define $A=\{ a \in \Reals | a_1 = 0 \}$
	}
	\Explain{
		Then $\Reals$ is a countable union of translates of $A$,
		so $A$ can't be meager
	}
	\Explain{
		Then, if $A$ has Baire property it must be clopen by the previous remark
	}
	\Explain{
		But $\Reals$ are connected, producing a contradiction
	}
	\EndProof
	\\
	\Theorem{ContinuousActionTheorem}
	{
		\NewLine 		
		\forall G \in \GRP \.
		\forall \T : \TYPE{Topology}(G) \.
		\forall X : \TYPE{Metrizable} \.
		\forall \alpha \in \GRP \And \TOP\Big( (G,\T), \Aut_\TOP(X) \Big) \. 
		\NewLine \.
		\Bair \And \TYPE{Metrizable}(X,\T) \And 
		\Big( 
			\forall g \in G \. 
			\lambda_g \in \TOP\big( (X,\T), (X,\T)\big) 
		\Big)
		\Imply
		\alpha \in \TOP\Big( (G,\T)\times X, X \Big) 
	}
	\Explain{Use jpoint continuity theorem from descreptive set theory.}
	\EndProof
	\\
	\Theorem{TopologicalGroupBySeparateContinuity}
	{
		\NewLine 		
		\forall G \in \GRP \.
		\forall \T : \TYPE{Topology}(G) \.
		\NewLine \.
		\Bair \And \TYPE{Metrizable}(X,\T) \And
		\inv_G \in   \TOP\big( (X,\T), (X,\T)\big) \And \NewLine \And
		\Big( 
			\forall g \in G \. 
			\lambda_g,\rho_g \in \TOP\big( (X,\T), (X,\T)\big) 
		\Big)
		\Imply
		(G,\T) \in \TGRP
	}
	\NoProof
}\Page{
	\Theorem{MillerStabilizerTHM}
	{
		\forall G : \BG \.
		\forall X : \TYPE{T1} \And \TYPE{SecondCountable} \.
		\forall \alpha \in \GRP\Big( G, \Aut_\SET(X) \Big) \. \NewLine \.
		\Big( \forall H \Sgrp G \. \Closed(G,H) \Imply \Bair(H) \Big)
		\And \NewLine \And
		\Big( \forall x \in X \. \forall H \Sgrp G \.
		\Closed(G,H)  \Imply   \BM\big(H,X,\Lambda h \in H \. \alpha(h)(x)\big) \Big)
		\Imply \NewLine \Imply
		\forall x \in X \. \Closed\big(G, \Stab(\alpha,x) \big)
	}
	\Say{H}{\overline{\Stab(\alpha,x)}}{\Closed(G)}
	\Say{[1]}{\THM{SubgroupClosure}(H)}{\TYPE{Subgroup}(G,H)}
	\Say{[2]}{[0.1][1]}{\BG(H)}
	\Say{[4]}{\THM{DenseInAClosure}\Big(G,\Stab(\alpha,x)\Big)}
	{
		\Dense\Big(H,\Stab(\alpha,x)\Big)
	}
	\Assume{[5]}{\exists^* \Stab(\alpha,x)}
	\Say{[6]}{\THM{NonMeagerBPSubgroupIsClopen}[5]}
	{
		\Clopen(H,\Stab(\alpha,x))
	}
	\Conclude{[5.*]}{[4][6]}{H = \Stab(\alpha,x)}
	\Derive{[5]}{\Intro(\to)}
	{		
		\exists^*\Stab(\alpha,x) \to H = \Stab(\alpha,x)	
	}
	\Assume{[6]}{\neg \exists^* \Stab(\alpha,x)}
	\Say{\Big(V,[7]\Big)}{\Elim \TYPE{SecondCountable}(X)}
	{
		\sum V: \Nat \to \T(X) \.  \TYPE{BaseOfTopology}(X,\im V) 
	}
	\Say{[8]}{\THM{BaseSeparation}(X,V)}
	{
		\forall x, y \in X \.  x \neq y \to \exists n,m \in \Nat \.
		x \in V_n \And y \in V_m \And V_m \cap V_n = \emptyset
	}
	\Say{\phi}{\Lambda h \in H \. \alpha(h)(x)}{G \to X}
	\Say{A}{\phi^{-1}(V)}{\Nat \to \BP(H)}
	\Say{[9]}{\Elim \Stab \Elim A }
	{
		\forall h \in H \.   
		\Big( \forall n \in \Nat \. A_n h = A_n  \Big)
		\iff
		h \in \Stab(\alpha,x)
	}
	\Say{[10]}{\Elim A}
	{
		\forall h \in H \. h \Stab(\alpha,x) = \bigcap \{ A_n | n \in \Nat, g \in A_n \}
	}
	\Say{[11]}{\THM{FirstTopological01Law}[9]}
	{
		\forall n \in \Nat \. \forall^* A_n \Big| \neg \exists^* A_n
	}
	\Say{[12]}{\Elim \BG(H)[6][11][10]}
	{
		\forall h \in H \. \exists n \in \Nat \. h \in A_n \And \neg \exists^* A_n
	}
	\Say{[13]}{\THM{MeagerUnion}}{\neg \exists H}
	\Conclude{[6.*]}{\Elim \BG(H) [12] }
	{
		\bot
	}
	\DeriveConclude{[*]}{\Elim \bot \Elim |[5]}{H = \Stab(\alpha,x)}
	\EndProof
}
\newpage
\subsubsection{Universal Polish Group}
The proof of this result is rather convoluted and we may need Keller's theorm for it.
\Page{
	\Theorem{UspenskiTHM}
	{
		\forall G : \PG \.
		\exists H \Sgrp \Aut_\TOP[0,1]^\Nat \.
		H \cong_\TGRP G
	}
	\NoProof
	\\
	\Conclude{G^\star}{\Aut_\TOP[0,1]^\Nat}{\TGRP}
}
\newpage
\subsubsection{Selectors and Transversal}
\Page{
	\DeclareType{Selector}{\prod_{X \in \SET} \Eq(X) \to ?(X \to X)}
	\DefineType{\sigma}{Selector}
	{
		\Lambda E : \Eq(X)	 \.	
		\forall  C : \EqClass{X,E} \.
		\forall x,y \in C \.
		\sigma(x) = \sigma(y)
		\And
		\sigma(x) \in C 	
	}
	\\
	\DeclareType{Transversal}{\prod_{X \in \SET} \Eq(X) \to ??X}
	\DefineType{T}{Transversal}
	{
		\Lambda E : \Eq(X) \. 
		\forall C : \EqClass{X,E} \.
		|C  \cap T| = 1	
	}
	\\
	\DeclareFunc{selectorAsTransversal}{
		\prod_{X \in \SET} 
		\prod E : \Eq(X) \.
		\Selector(X,E) \to \Transversal(X,E)
	}
	\DefineNamedFunc{selectorAsTransversal}{\sigma}{\sigma}{\im \sigma}
	\\
	\DeclareFunc{transversalAsSelector}{
		\prod_{X \in \SET} 
		\prod E : \Eq(X) \.
		\Transversal(X,T) \to \Transversal(X,T)
	}
	\DefineNamedFunc{transversalAsSelector}{T}{T}{
		\Lambda x \in X \. 		
		\Elim \TYPE{Singleton}(X) 
		\Elim \TYPE{Transversal}\big(T,[x]\big)
	}
	\\
	\DeclareFunc{saturation}
	{
		\prod_{X \in \SET}
		\Eq(X) \to ?X \to ?X
	}
	\DefineNamedFunc{saturation}{E,A}{[A]_E}
	{
		\{
			x \in X : \exists a \in A : xEa
		\}	
	}
	\\
	\Theorem{BorelSelectors}
	{
		\forall X : \Polish \.	
		\forall E : \Eq \. \NewLine \.
		\forall [0.1] : \forall C : \EqClass{X,E} \. \Closed(X,C) \.  \NewLine \.
		\forall [0.2] : \forall U \in \T(X) \. [U]_E \in \B(X) \. \NewLine \.
		\exists \sigma : \Selector(X) \.
		\sigma \in \End_\BOR(X)
	}
	\Say{\varphi}
	{
		\Lambda x \in X \.
		[x]_E 	
	}{
		X \to \Effros(X)
	}
	\Say{[1]}
	{
		\Lambda U \in \T(X) \. 
		\Elim \varphi \Intro [U]_E
		[0.2]
	}
	{
		\forall U \in \T(X) \.
		\varphi^{-1}\Big\{ A \in \Effros(X) : \exists A \cap U    \Big\} = [U]_E \in \B(X)
	}
	\Conclude{[2]}{\Intro \BOR [1]}
	{
		\varphi \in \End_\BOR(X)	
	}
	\Say{\Big(\delta,[3]\Big)}
	{
		\THM{SelectionTHM}(X)
	}
	{
		\sum \delta : \Nat \to \BOR\Big(X,\Effros(X)\Big) \.
		\forall A \in \Effros(X) \. \Dense\Big(A, \delta_\Nat(A) \Big)
	}
	\Say{\sigma}{\varphi \delta_1}{\BOR(X,X)}
	\Conclude{[*]}{\Elim \sigma [3]}{\Selector(X,E,\sigma) }
	\EndProof
	\\
	\Theorem{SubgroupSelector}
	{
		\forall G : \PG \.
		\forall H : \Closed \And \TYPE{Subgroup}(G) \.
		\exists \sigma  \in \Aut_\BOR(G) \. \NewLine \.
		\Selector\Big(G,\TYPE{Coset}(G,H),\sigma\Big)
	}
	\Say{[1]}{\Elim \TGRP (G)}{\forall g \in G \. \Closed(G,gH)}
	\Say{[2]}{\Lambda U \in \T(G) \Elim \TGRP (G) \Elim \TOP(G)}
	{
		\forall U \in \T(G) \.
		[U]_H = \bigcup_{g \in G} gU \in \T(G)
	}
	\Conclude{[3]}{\THM{BorelSelector}(G,\sim_H)[1][2]}
	{
		\exists \sigma  \in \Aut_\BOR(G) \.
		\Selector\Big(G,\TYPE{Coset}(G,H),\sigma\Big)
	}
	\EndProof
}
\newpage
\subsubsection{Borel Space of Polish Groups}
\Page{
	\Theorem{ClosedSubgroupsAreBorel}
	{
		\Closed \And \TYPE{Subgroup}(G^\star) \in \B\Big( \Effros(G^\star) \Big)
	}
	\Say{S}{\Closed \And \TYPE{Subgroup}(G^\star)}{??G^\star}
	\Say{\Big(\delta,[1]\Big)}
	{
		\THM{SelectionTHM}(G^\star)
	}
	{
		\sum \delta : \Nat \to \BOR\Big(G^\star,\Effros(G^\star)\Big) \.
		\forall A \in \Effros(G^\star) \. \Dense\Big(A, \delta_\Nat(A) \Big)
	}
	\Conclude{[*]}{}
	{
		S =  \Big\{ A \in \Effros(G^\star) \Big| e \in A \Big\}
		\cap \bigcap^\infty_{n=1} \bigcap^\infty_{m=1}
		(\delta_n \delta_m^{-1})^{-1}
		\{
			(x,A) \in G^\star \times \Effros(G^\star) \Big|
			x \in A  
		\}
 	}
 	\EndProof
}
\newpage
\subsubsection{Standard Borel Groups}
\Page{
	\DeclareType{\Borg}
	{
		?\sum G \in \GRP \. \SA(G)
	}
	\DefineType{(G,\A)}{\Borg}
	{
		\NewLine \iff		
		\circ_G \in \BOR\Big((G,\T) \times (G,\T),(G,\T)\Big)
		\And
		\Lambda g \in G \. g^{-1} \in \BOR\Big((G,\T),(G,\T)\Big)
	}
	\\
	\DeclareFunc{groupOfBorelAsGroup}
	{
		\Borg \to \GRP
	}
	\DefineNamedFunc{groupOfBorelAsGroup}{G,\A}{(G,\A)}
	{
		G
	}
	\\
	\DeclareFunc{groupOfBorelAsMeasurableSpace}
	{
		\Borg \to \BOR
	}
	\DefineNamedFunc{groupOfBorelAsMeasurablelSpace}{G,\A}{(G,\A)}
	{
		(G,\A)
	}
	\\
	\Conclude{\SBG}{\Borg \And \SBS}{\Type}
	\\
	\Theorem{StandardGroupTopologyUniqueness}
	{
		\forall G : \SBG \.
		\forall \A,\B : \TYPE{Topology}(G) \. \NewLine \.
		\forall [0.1] : \PG\Big( (G,\A) \And (G,\B)\Big)  \.
		\forall [0.2] : \alg(G) = \sigma(\A) = \sigma(\B) \.
		\A = \B
	}
	\Say{[1]}{\THM{BorelIsBairMeasurable}[0.2]}
	{
		\NewLine :		
		\BM\Big( (G,\A),(G,\B), {\id}_G \Big)
		\And
		\BM\Big( (G,\B),(G,\A), {\id}_G \Big)
	}
	\Say{[2]}{\THM{BairMeasurableIsContinuous}[1][0.1]}
	{
		\TOP\Big( (G,\A),(G,\B), {\id}_G \Big)
		\And
		\TOP\Big( (G,\B),(G,\A), {\id}_G \Big)
	}
	\Conclude{[3]}{\Elim \id [2]}{\A = \B}
	\EndProof
	\\
	\DeclareType{Polishable}{?\SBG}
	\DefineType{G}{Polishable}
	{
		\exists \T : \TYPE{Topology}(G) \.
		\PG( G,\T)  \And
		\alg(G) = \sigma(\A)
	}
	\\
	\DeclareFunc{eventuallyOneGroup}{\TGRP}
	\DefineNamedFunc{eventuallyOneGroup}{}{\Torus^\Nat_1}
	{
		\Big\{ s \in \Torus^\Nat   :  \big|\{ n \in \Nat \. s_n \neq 1\}\big| < \infty  \Big\}	
	}
	\\
	\Theorem{EventuallyOneGroupIsBorel}{\Torus^\Nat_1 \in \B(\Torus^\Nat)}
	&  \Torus^\Nat_1 = \bigcup^\infty_{n=1} \Torus^n \times \{1\}^\Nat \\
	\EndProof
	\\
	\Theorem{EventuallyOneGroupIsStandard}{\SBG(\Torus^\Nat_1) }
	\Explain{ The space $\bigsqcup_{n=1} \Torus^n$ is standard Borel}
	\Explain{ Factorizing by closed sets should produce $\Torus^\Nat_1$ with polish topology}
	\EndProof
}\Page{
	\Theorem{EventuallyOneGroupIsNotPolishable}
	{
		\neg \Polishable\Big(\Torus^\Nat_1\Big)
	}
	\Assume{[1]}{\Polishable\Big(\Torus^\Nat_1\Big)}
	\Say{\Big(\T,[2]\Big)}{\Elim [1]}
	{
		\sum \T : \TYPE{Topology}(X) \.
		\PG(\Torus^\Nat_1,\T)
		\And
		\alg(\Torus^\Nat_1) = \sigma(\T)	
	}
	\Say{[3]}{\THM{PolishIsGDelta}[2.1]}
	{
		G_\delta( \Torus^\Nat, \Torus^\Nat_1)
	}
	\Say{[4]}{\THM{GDeltaSubgroupIsClosed}[3]}
	{
		\Closed(\Torus^\Nat,\Torus^\Nat_1)
	}
	\Say{x}
	{
		\Lambda n \in \Nat \.
		\Lambda m \in \Nat \.
		\exp\left(\frac{\i}{n}\right)
	}
	{
		\Nat \to \Torus^{\Nat}
	}
	\Say{[6]}{\Elim x \Elim \Torus^\Nat_1}
	{
		\forall n \in \Nat \. x_n \not \in \Torus^\Nat_1	
	}
	\Say{[7]}{\Elim x \Elim \Torus^\Nat \Elim \Torus^\Nat_1}{
		\lim_{n \to \infty} x_n = 1 \in \Torus^\Nat_1
	}
	\Conclude{[1.*]}{\THM{ClosedBySequences}[4][7]}{\bot}
	\DeriveConclude{[*]}{\Elim \bot}{\neg \Polishable(\Torus^\Nat)}
	\EndProof
	\\
	\Theorem{L2SequencesArePolishable}{\Polishable(l_2)}
	\Explain{ $l_2$ is polish as a separable Hilbert space}
	\EndProof
}
\newpage
\subsection{Borel Action}
\subsubsection{E Separation}
\Page{
	\DeclareType{EInvariant}
	{
		\prod X \in \Set \.
		\Eq(X) \to ??X
	}
	\DefineType{A}{EInvariant}
	{
		\forall x \in A \. \forall y \in [x]_E \. y \in A
	}
	\\
	\Theorem{ESeparation}
	{
		\forall X : \SBS \.
		\forall E : \Eq(X) \.
		\forall [0.1] : E \in \Sigma^1_1(X^2) \. \NewLine \.
		\forall A,B \in \Sigma^1_1(X) \And \TYPE{EInvariant}(X,E) \.
		\forall [0.2] : \TYPE{DisjointPair}(X,A,B) \. \NewLine \.
		\exists C \in \B(X) \And \TYPE{EInvariant}(X,E) \.
		A \subset C \And B \cap C = \emptyset
	}
	\Explain{ Assume $X$ is Polish without loss of generality.}
	\Explain{ It is possible to give strong topology to the set $\frac{X}{E}$, 
		so the projection $\pi_E$ is continuous}
	\Explain{ With this structure $\pi_E(A),\pi_E(B)$ are anlytic in $\frac{X}{E}$}
	\Explain{ They also disjoint as they were E-Invariant}
	\Explain{ As $E$ is analytic set itself, we can realize by a pair of continuous maps 
		  $\phi_1,\phi_2 : \B \to X$ }
	\Explain{ Then, $x \sim_E y$ iff there is a $b \in \B$ such that $\phi_1(b) = x$ and $\phi_2(b)=y$}
	\Explain{ Thus, $\frac{X}{E}$ is equivalent to pushout $X \sqcup_{\B,\phi} X$}
	\Explain{ So, $\frac{X}{E}$ must be Polish }
	\Explain{ Now apply Suslin separation theorem in $\frac{X}{E}$ to separate
		$A$ and $B$ by some $C$ }
	\Explain{ Then $\pi_E^{-1}(C)$ is Borel and $E$-invariant, it also separates $A$ and $B$}
	\NoProof
	\\
	\Theorem{PolishTopologicalGroupCondition}
	{
		\forall G \in \GRP \.
		\forall \T : \TYPE{PolishTopology}(G) \. \NewLine \.
		\forall [0] : \forall g \in G \. \lambda_g, \rho_g \in \Aut_\TOP(G,\T) \.
		(G,\T) \in \TGRP
	}
	\NoProof
}\Page{
	\Theorem{BlackwellTHM}
	{
		\NewLine ::		
		\forall X : \SBS \.
		\forall A : \Nat \to \S_X \.
		\forall S \subset X \.
		\TYPE{EInvariant}(X,E,S) \And \B(X)
		\iff
		S \in \sigma(\im A)
		\NewLine
		\where \quad
		E =  \Big\{ (x,y) \in X^2 :  \forall n \in \Nat \. x \in A_n \iff y \in A_n  \Big\}			
	}
	\Explain{ By properties of logical $\iff$ it is obvious that $E$ is equivalence}
	\Explain{ Firstly, we show that each $A_n$ is $E$-invariant}
	\Explain{ consider $x \in A_n$ and $y \in [x]_E$, then by definition of $E$ we also have $y \in A_n$ }
	\Explain{ It is clear that union of $E$-invariant sets is $E$-invarinat}
	\Explain{ Now assume that $B$ is $E$-invariant}
	\Explain{ Assume $x \in B^\c$ and $y \in [x]_E$}
	\Explain{ If $y$ was in $B$ then by symmetry $x$ would also be in $B$, so $y \in B^\c$}
	\Explain{ So, invariant subsets form a $\sigma$-algebra containing all $A_n$}
	\Explain{ Thus, $\sigma(\im A)$ is all $E$-invariant}
	\Explain{ Now assume $B$ is Borel amd $E$-invariant}
	\Explain{ Note, that equivalence class of $E$ are Borel and correspond to elements of $\C$}
	\Explain{ For $c \in \C$ the set 
	$\alpha_c = \bigcap_{c_n = 1} A_n \cap \bigcap_{c_n = 0} A_n^\c$ 
	is equivalence class of $E$}
	\Explain{ And every equivalence class of $E$ can be expressed as som $\alpha_c$   }
	\Explain{ Thus, every equivalence class belongs to $\sigma(\im A)$ }
	\Explain{
		Nevertheless, we always can express $B = \bigcup_{c \in C } \alpha_c$
		for some $C \subset \C$}
	\Explain{
		Consider a mapping $\psi:\frac{X}{E} \to \C$ defined by $\psi(\alpha_c) = c$}
	\Explain{
		This mapping is a measurable (see argument obout prebase next) injection}
	\Explain{
		So $C = \psi[B]_E$ must be measurable in $\C$}
	\Explain{
		but topology on $\C$ can be generated by prebase of sets of form
		$\{ c \in \C | c_n = j \}$, where $j=1,0$ and $n \in \Nat$}
	\Explain{
		And such sets also generate the Borel algebra of $\C$}
	\Explain{
		Now sets $C = \{ c \in \C | c_n = 1 \}$ corresponds directly to $A_n$ }
	\Explain{ So, $B$ must belong to $\sigma(\im A)$  }
	\EndProof
}
\newpage
\subsubsection{Subject}
\Page{
	\Conclude{\TYPE{BorelAction}}
	{
		\Lambda X : \SBS \.
		\Lambda G : \SBG \.
		G \ActOn_\BOR X 
		= \NewLine =
		\Lambda X : \SBS \.
		\Lambda G : \SBG \.
		\GRP\Big(G, \Aut_\BOR(X)\Big)
		\And
		\BOR(G\times X,X)
	}
	{
		\NewLine :		
		\SBS \to \SBG \to \Type
	}
	\\
	\Theorem{BorelOrbitRelationIsAnalytic}
	{
		\NewLine ::		
		\forall X : \SBS \.
		\forall G : \SBG \.
		\forall \alpha : G \ActOn_\BOR X \. 
		E_\alpha \in \Sigma^1_1(X^2)
	}
	\Explain{Enrich topology on $X \times G$ so $\alpha$ is continuous}
	\Explain{Then, mapping $\beta : (x,g) \mapsto \Big(x,\alpha(x,g)\Big)$ is also continuous}
	\Explain{But its image is $E_\alpha$}
	\EndProof
	\\
	\Theorem{FreeBorelOrbitIsBorel}
	{
		\NewLine ::
		\forall X : \SBS \.
		\forall G : \SBG \.
		\forall \alpha : G \ActOn_\BOR X \. 
		\TYPE{Free}(G,X,\alpha) 
		\Imply
		E_\alpha \in \B(X^2)
	}
	\Explain{ In this case $\beta$ will be injective}
	\Explain{ So $E_\alpha$ is Borel by Injective Image Theorem}
	\EndProof
	\\
	\Theorem{LocallyCompactContinuousOrbitIsFSigma}
	{
		\NewLine ::
		\forall X : \Polish \.
		\forall G : \PG \And \LC \.
		\forall \alpha : G \ActOn_\TOP X \. 
		E_\alpha \in F_\sigma(X^2)
	}
	\NoProof
	\\
	\Theorem{MillerBorelOrbitTHM}
	{
		\NewLine ::
		\forall G : \PG \.
		\forall X : \SBS \.
		\forall \alpha : G \ActOn_\BOR X \.
		\forall x \in X \.
\		O_\alpha(x) \in \S_X
	}	
	\Say{[1]}{\THM{MillerStabilizerTHM}(G,\alpha)}
	{
		\forall x \in X \. \Closed\Big(G, \Stab(\alpha,x)\Big)
	}
	\Say{\Big(T,[2]\Big)}{\THM{SubgroupSelector}[1](x)}
	{
		T \in \B(G) \.
		\forall g \in G \. 
		\Big| T \cap  g\Stab(\alpha,x)\Big| = 1
	}
	\Say{[3]}{\Elim \TYPE{GroupAction}(G,X,\alpha)}
	{
		\forall g,h \in G \. 
		 gx = hx 
		 \iff
		 \exists f \in G \.
		 g,h \in f\Stab(\alpha,x)
	}
	\SayIn{\varphi}{\Lambda g \in G \. gx}{\BOR(G,X)}
	\Say{[4]}{\Elim \varphi [2][3]}
	{
		\Inj\Big(T,X,\varphi_{|T}\Big) 
	}
	\Say{[5]}{\Elim \varphi \Intro O_\alpha}{\varphi(T) = O_\alpha(x) }
	\Conclude{[6]}{\THM{InjectiveImageTHM}}
	{
		O_\alpha(x) \in \S_X
	}
	\EndProof
}
\Page{
	\Theorem{BorelHomo}
	{
		\forall G : \PG \.
		\forall H : \SBG \.
		\forall \varphi \in \BOR \cap \GRP(G,H) \.
		\varphi(G) \in \S_H		
	}
	\Explain{ Define Borel action $\alpha : G \ActOn H$ by $\alpha(g,h) = \varphi(g)h$}
	\Explain{  Then $\varphi(G) = O_\alpha(e)$}
	\Explain{ By Miller's Theorem it must be measurable}
	\EndProof
}
\newpage
\subsubsection{Vaught Transfom}
\Page{
	\DeclareFunc{actionSaturation}
	{
		\prod G \in \GRP \. 
		\prod X \in \SET \.
		\prod \alpha : G \ActOn X \.
		?X \to ?X
	}
	\DefineNamedFunc{actionSaturation}{A}{[A]_\alpha}
	{
		\{  x \in X : \exists g \in G \. gx \in A \}
	}
	\\
	\DeclareFunc{actionHull}
	{
		\prod G \in \GRP \. 
		\prod X \in \SET \.
		\prod \alpha : G \ActOn X \.
		?X \to ?X
	}
	\DefineNamedFunc{actionHull}{A}{(A)_\alpha}
	{
		\{  x \in X : \forall g \in G \. gx \in A \}
	}
	\\
	\Theorem{SaturationAndHullRelation}
	{
		\forall G \in \GRP \.
		\forall X \in \SET \.
		\forall \alpha : G \ActOn X \.
		\forall A \subset X \.
		(A)_\alpha \subset A \subset [A]_\alpha
	}
	\Explain{ This is obvious}
	\EndProof
	\\
	\Theorem{AnalyticSaturation}
	{
		\NewLine ::
		\forall G : \SBG \.
		\forall X : \SBS \.
		\forall \alpha : G \ActOn_\BOR X \.
		\forall A \in \S_X \.
		[A]_\alpha \in \Sigma^1_1(X)
	}
	\Explain{View $[A]_\alpha$ as an image of $G \times A$ under $\alpha$}
	\EndProof
	\\
	\Theorem{CoanalyticHull}
	{
		\NewLine ::
		\forall G : \SBG \.
		\forall X : \SBS \.
		\forall \alpha : G \ActOn_\BOR X \.
		\forall A \in \S_X \.
		(A)_\alpha \in \Pi^1_1(X)
	}
	\Explain{View $(A)_\alpha^\c$ as an image of $G \times A^\c$ under $\alpha$}
	\EndProof
	\\
	\DeclareFunc{nonmeagerVaughtTransform}
	{
		\prod G \in \SBG \. 
		\prod X \in \SBS \. \NewLine \.
		\prod \alpha : G \ActOn_\BOR X \.
		?X \to ?X
	}
	\DefineNamedFunc{nonmeagerVaughtTransform}{A}{A^\star_\alpha}
	{
		\{  x \in X : \exists^* g \in G \. gx \in A \}
	}
	\\
	\DeclareFunc{comeagerVaughtTransform}
	{
		\prod G \in \SBG \. 
		\prod X \in \SBS \. \NewLine \.
		\prod \alpha : G \ActOn_\BOR X \.
		?X \to ?X
	}
	\DefineNamedFunc{comeagerVaughtTransform}{A}{A^{\du}_\alpha}
	{
		\{  x \in X : \forall^* g \in G \. gx \in A \}
	}
	\\
	\DeclareFunc{nonmeagerLocalVaughtTransform}
	{
		\prod G \in \SBG \. 
		\prod X \in \SBS \. \NewLine \.
		\prod \alpha : G \ActOn_\BOR X \.
		\T(G) \to ?X \to ?X
	}
	\DefineNamedFunc{nonmeagerLocalVaughtTransform}{A,U}{A^{\star U}_\alpha}
	{
		\{  x \in X : \exists^* g \in U \. gx \in A \}
	}
	\\
	\DeclareFunc{comeagerLocalVaughtTransform}
	{
		\prod G \in \SBG \. 
		\prod X \in \SBS \. \NewLine \.
		\prod \alpha : G \ActOn_\BOR X \.
		\T(G) \to ?X \to ?X
	}
	\DefineNamedFunc{comeagerLocalVaughtTransform}{A,U}{A^{\du U}_\alpha}
	{
		\{  x \in X : \forall^* g \in U \. gx \in A \}
	}
}
\Page{
	\Theorem{VaughtTransformsRelation}
	{
		\NewLine ::
		\forall G \in \SBG \. 
		\forall X \in \SBS \. \NewLine \.
		\prod \alpha : G \ActOn_\BOR X \.
		\forall A \subset X \.
		(A)_\alpha \subset A^{\du}_\alpha \subset A^{\star}_\alpha \subset  A \subset [A]_\alpha
	}
	\Explain{Obvious}
	\EndProof
	\\
	\Theorem{VaughtTransformsInvariant}
	{
		\forall G \in \SBG \. 
		\forall X \in \SBS \. \NewLine \.
		\prod \alpha : G \ActOn_\BOR X \.
		\forall A \subset X \.
		\TYPE{Invariant}( G,X,\alpha, A^{\du}_\alpha \And A^{\star}_\alpha )
 	}
	\NoProof
	\\
	\Theorem{VaughtTransformInvarianceCriterion}
	{
		\forall G \in \SBG \. 
		\forall X \in \SBS \. \NewLine \.
		\prod \alpha : G \ActOn_\BOR X \.
		\forall A \subset X \.
		\TYPE{Invariant}( G,X,\alpha, A)
		\iff
		A^{\du}_\alpha = A^{\star}_\alpha
	}
	\NoProof
	\\
	\Theorem{LocalVaughtTransformIsBorel}
	{
		\forall G \in \SBG \. 
		\forall X \in \SBS \. \NewLine \.
		\prod \alpha : G \ActOn_\BOR X \.
		\forall A \subset \S_X \.
		\forall U \in \T(G)   \.
		A^{\du U}_\alpha, A^{\star U}_\alpha \in \S_X
	}
	\Explain{ Follows from Novikov-Montgomery theorem}
	\EndProof
}
\newpage
\section*{Sources}
\begin{enumerate}
\item Wilansky A. - Topology for Analysis (1970) 
\item Понтрягин. Л. С - Непрерывные группы (1973)
\item Roelcke W. ; Dierolf S. - Uniform Structures on Topological Groups (1981) 
\item Page W. - Topological Uniform Structures (1989) 
\item Агеев С. М. - Топологические доказательства теоремы Келлера 
и ее эквивариантного аналога (1993)
\item Kechris A. - Classical Descriptive Set Theory (1995) 
\item Gau S.  - Invariant Descriptive Set Theory (2008) 
\item Rosendal C. - Coarse Geometry of Topological Groups (2021)
\end{enumerate}
\end{document} 
