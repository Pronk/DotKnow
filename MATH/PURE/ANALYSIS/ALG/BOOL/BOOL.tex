\documentclass[12pt]{scrartcl}
\usepackage{mathtools}
\usepackage{amsmath}
\usepackage{amsfonts}
\usepackage{hyperref}
\usepackage{amssymb}
\usepackage{ wasysym }
\usepackage{accents}
\usepackage{extpfeil}
\usepackage{graphicx}
\usepackage{scalerel}
\usepackage{esvect}
\usepackage{upgreek}
\usepackage[dvipsnames]{xcolor}
\usepackage[a4paper,top=5mm, bottom=5mm, left=10mm, right=2mm]{geometry}
%Markup
\newcommand{\TYPE}[1]{\textcolor{NavyBlue}{\mathtt{#1}}}
\newcommand{\FUNC}[1]{\textcolor{Cerulean}{\mathtt{#1}}}
\newcommand{\LOGIC}[1]{\textcolor{Blue}{\mathtt{#1}}}
\newcommand{\THM}[1]{\textcolor{Maroon}{\mathtt{#1}}}
%META
\renewcommand{\.}{\; . \;}
\newcommand{\de}{: \kern 0.1pc =}
\newcommand{\extract}{\LOGIC{Extract}}
\newcommand{\where}{\LOGIC{where}}
\newcommand{\If}{\LOGIC{if} \;}
\newcommand{\Then}{ \; \LOGIC{then} \;}
\newcommand{\Else}{\; \LOGIC{else} \;}
\newcommand{\IsNot}{\; ! \;}
\newcommand{\Is}{ \; : \;}
\newcommand{\DefAs}{\; :: \;}
\newcommand{\Act}[1]{\left( #1 \right)}
\newcommand{\Example}{\LOGIC{Example} \; }
\newcommand{\Theorem}[2]{& \THM{#1} \, :: \, #2 \\ & \Proof = \\ } 
\newcommand{\DeclareType}[2]{& \TYPE{#1} \, :: \, #2 \\} 
\newcommand{\DefineType}[3]{& #1 : \TYPE{#2} \iff #3 \\} 
\newcommand{\DefineNamedType}[4]{& #1 : \TYPE{#2} \iff #3 \iff #4 \\} 
\newcommand{\DeclareFunc}[2]{& \FUNC{#1} \, :: \, #2 \\}  
\newcommand{\DefineFunc}[3]{&  \FUNC{#1}\Act{#2} \de #3 \\} 
\newcommand{\DefineNamedFunc}[4]{&  \FUNC{#1}\Act{#2} = #3 \de #4 \\} 
\newcommand{\NewLine}{\\ & \kern 1pc}
\newcommand{\Page}[1]{ \begin{align*} #1 \end{align*}   }
\newcommand{ \bd }{ \ByDef }
\newcommand{\NoProof}{ & \ldots \\ \EndProof}
%LOGIC
\renewcommand{\And}{\; \& \;}
\newcommand{\ForEach}[3]{\forall #1 : #2 \. #3 }
\newcommand{\Exist}[2]{\exists #1 : #2}
\newcommand{\Imply}{\Rightarrow} 
\newcommand{\Intro}{\LOGIC{I}}
\newcommand{\Elim}{\LOGIC{E}}
\newcommand{\Type}{\TYPE{Type}}
%%STD
\newcommand{\Int}{\mathbb{Z} }
\newcommand{\NNInt}{\mathbb{Z}_{+} }
\newcommand{\Reals}{\mathbb{R} }
\newcommand{\Complex}{\mathbb{C}}
\newcommand{\Rats}{\mathbb{Q} }
\newcommand{\Sphere}{\mathbb{S}}
\newcommand{\Ball}{\mathbb{B}}
\newcommand{\Nat}{\mathbb{N} }
\newcommand{\EReals}{\stackrel{\mathclap{\infty}}{\mathbb{R}}}
\newcommand{\ERealsn}[1]{\stackrel{\mathclap{\infty}}{\mathbb{R}}^{#1}}
\DeclareMathOperator*{\centr}{center}
\DeclareMathOperator*{\argmin}{arg\,min}
\DeclareMathOperator*{\id}{id}
\DeclareMathOperator*{\im}{Im}
\DeclareMathOperator*{\supp}{supp}
\newcommand{\EqClass}[1]{\TYPE{EqClass}\left( #1 \right)}
\newcommand{\Cat}{\TYPE{Category}}
\newcommand{\Mor}{\mathcal{M}}
\newcommand{\Obj}{\mathcal{O}}
\newcommand{\End}{\mathrm{End}}
\newcommand{\Aut}{\mathrm{Aut}}
\newcommand{\Func}[2]{\TYPE{Functor}\left( #1, #2 \right)}
\mathchardef\hyph="2D
\newcommand{\Surj}{\TYPE{Surjective}}
\newcommand{\Inj}{\TYPE{Injective}}
\newcommand{\Bij}{\TYPE{Bijective}}
\newcommand{\ToInj}{\hookrightarrow}
\newcommand{\ToMono}{\xhookrightarrow}
\newcommand{\ToSurj}{\twoheadrightarrow}
\newcommand{\ToEpi}{\xtwoheadrightarrow}
\newcommand{\ToBij}{\leftrightarrow}
\newcommand{\ToIso}{\xleftrightarrow}
\newcommand{\Arrow}{\xrightarrow}
\newcommand{\Set}{\TYPE{Set}}
\newcommand{\du}{\; \triangle \;}
\renewcommand{\c}{\complement}
\renewcommand{\i}{\mathbf{i}}
\newcommand{\Eqmod}[3]{#1 = #2 \quad \mathrm{mod} \quad #3}
%%ProofWritting
\newcommand{\Say}[3]{& #1 \de #2 : #3, \\}
\newcommand{\SayIn}[3]{& #1 \de #2 \in #3, \\}
\newcommand{\Conclude}[3]{& #1 \de #2 : #3; \\}
\newcommand{\Derive}[3]{& \leadsto #1 \de #2 : #3, \\}
\newcommand{\DeriveConclude}[3]{& \leadsto #1 \de #2 : #3 ; \\} 
\newcommand{\Assume}[2]{& \LOGIC{Assume} \; #1 : #2, \\}
\newcommand{\AssumeIn}[2]{& \LOGIC{Assume} \; #1 \in #2, \\}
\newcommand{\As}{\; \LOGIC{as } \;} 
\newcommand{\QED}{\; \square}
\newcommand{\EndProof}{& \QED \\}
\newcommand{\Proof}{\LOGIC{Proof} \; }
%SetTheory
\newcommand{\NonEmpty}{\TYPE{NonEmpty}}
\newcommand{\Finite}{\TYPE{Finite}}
\newcommand{\Countable}{\TYPE{Countable}}
\newcommand{\SetEq}{\TYPE{SetEq}}
\newcommand{\Ideal}{\TYPE{Ideal}}
\newcommand{\SIdeal}{\TYPE{\sigma\hyph \Ideal}}
\newcommand{\SA}{\TYPE{\sigma \hyph Algebra}}
\newcommand{\genIdeal}[1]{\left\langle #1 \right\rangle_\mathcal{I}}
%CategoryTheory
%Types
\newcommand{\Cov}{\TYPE{Covariant}}
\newcommand{\Contra}{\TYPE{Contravariant}}
\newcommand{\NT}{\TYPE{NaturalTransform}}
\newcommand{\UMP}{\TYPE{UnversalMappingProperty}}
\newcommand{\CMP}{\TYPE{CouniversalMappingProperty}}
\newcommand{\paral}{\rightrightarrows}
%functions
\newcommand{\op}{\mathrm{op}}
\newcommand{\obj}{\mathrm{obj}}
\DeclareMathOperator*{\dom}{dom}
\DeclareMathOperator*{\codom}{codom}
\DeclareMathOperator*{\colim}{colim}
%variable
\newcommand{\C}{\mathcal{C}}
\newcommand{\A}{\mathcal{A}}
\newcommand{\B}{\mathcal{B}}
\newcommand{\D}{\mathcal{D}}
\newcommand{\I}{\mathcal{I}}
\newcommand{\J}{\mathcal{J}}
\newcommand{\R}{\mathcal{R}}
%Cats
\newcommand{\CAT}{\mathsf{CAT}}
\newcommand{\SET}{\mathsf{SET}}
\newcommand{\PARALLEL}{\bullet \paral \bullet}
\newcommand{\WEDGE}{\bullet \to \bullet \leftarrow \bullet}
\newcommand{\VEE}{\bullet \leftarrow \bullet \to \bullet}
%OrderTheory
%Types
\newcommand{\Poset}{\TYPE{Poset}}
\newcommand{\Toset}{\TYPE{Toset}}
\newcommand{\Pres}{\TYPE{PreorderedSet}}
\newcommand{\WF}{\TYPE{WellFounded}}
\newcommand{\WO}{\TYPE{WellOrdered}}
\newcommand{\II}{\TYPE{InitialInterval}}
\newcommand{\UB}{\TYPE{UpperBound}}
\newcommand{\LUB}{\TYPE{LowerUpperBound}}
\newcommand{\LB}{\TYPE{LowerBound}}
\newcommand{\ULB}{\TYPE{UpperLoweBound}}
%Cats
\newcommand{\POSET}{\mathsf{POSET}}
\newcommand{\ORD}{\mathsf{ORD}}
%Symbols
\renewcommand{\P}{\mathsf{P}}
%\newcommand{\F}{\mathsf{F}}
%\newcommand{\U}{\mathsf{U}}
%Algebra
%Groups
%Types
\newcommand{\Group}{\TYPE{Group}}
\newcommand{\Abel}{\TYPE{Abelean}}
\newcommand{\Sgrp}{\subset_{\mathsf{GRP}}}
\newcommand{\Nrml}{\vartriangleleft}
\newcommand{\FG}{\TYPE{FiniteGroup}}
\newcommand{\Stab}{\mathrm{Stab}}
\newcommand{\FGA}{\TYPE{FinitelyGeneratedAbelean}}
\newcommand{\DN}{\TYPE{DirectedNormality}}
\newcommand{\ActsOn}{\curvearrowright}
%Func
\DeclareMathOperator{\tor}{tor}
\DeclareMathOperator{\ord}{ord}
\DeclareMathOperator{\bool}{bool}
\DeclareMathOperator{\rank}{rank}
%Cats
\newcommand{\GRP}{\mathsf{GRP}}
\newcommand{\ABEL}{\mathsf{ABEL}}
%Ops
\newcommand{\SDP}{\rightthreetimes}
%LINEAR
%Types
\newcommand{\Basis}{\TYPE{Basis}}
%Func
\DeclareMathOperator{\Span}{span}
%Cats
\newcommand{\VS}{\mathsf{VS}}
%FIELDS
\newcommand{\Field}{\TYPE{Field}}
%RINGS
%TYPE
\newcommand{\Ring}{\TYPE{Ring}}
\newcommand{\CR}{\TYPE{CommutativeRing}}
\newcommand{\ID}{\TYPE{IntegralDomain}}
\newcommand{\UFD}{\TYPE{UniqueFactorizationDomain}}
\newcommand{\PID}{\TYPE{PrincipleIdealDomain}}
\newcommand{\FGI}{\TYPE{FinitelyGeneratedIdeal}}
\newcommand{\ER}{\TYPE{EuclideanRing}}
\newcommand{\DVR}{\TYPE{DiscreteValuationRing}}
\newcommand{\MoFT}{\TYPE{MonoidOfFiniteType}}
\newcommand{\GA}{\TYPE{GradedAbelean}}
\newcommand{\Principle}{\TYPE{PrincipleIdeal}}
%CATS
\newcommand{\RING}{\mathsf{RING}}
\newcommand{\ANN}{\mathsf{ANN}}
\newcommand{\GRING}{\mathsf{GRING}}
\newcommand{\RNG}{\mathsf{RNG}}
%FUNCS
\DeclareMathOperator{\lcd}{lcd}
\DeclareMathOperator{\lc}{lc}
\DeclareMathOperator{\cont}{cont}
\DeclareMathOperator{\antideg}{antideg}
%Numbers
%Integers
%FUNCS
\DeclareMathOperator{\divi}{div}
\DeclareMathOperator{\remi}{rem}
\DeclareMathOperator{\Frac}{Frac}
%Algebras
\newcommand{\ALG}[1]{#1\hyph\mathsf{ALG}}
\newcommand{\ALGE}[1]{#1\hyph\mathsf{ALGE}}
%Topology
%General Topology
%Types
\newcommand{\TS}{\TYPE{TopologicalSpace}} 
\newcommand{\LF}{\TYPE{LocallyFinite}}
\newcommand{\PN}{\TYPE{PerfectlyNormal}}
\newcommand{\IP}{\TYPE{IsolatedPoint}}
\newcommand{\Compact}{\TYPE{Compact}}
\newcommand{\Compacts}{\TYPE{CompactSubset}}
\newcommand{\SCompact}{\TYPE{\sigma\hyph Compact}}
\newcommand{\LCompact}{\TYPE{LocallyCompact}}
\newcommand{\Perfect}{\TYPE{Perfect}}
\newcommand{\Limit}{\TYPE{Limit}}
\newcommand{\Clopen}{\TYPE{Clopen}}
\newcommand{\Closed}{\TYPE{Closed}}
\newcommand{\Dense}{\TYPE{Dense}}
\newcommand{\ND}{\TYPE{NowhereDense}}
\newcommand{\Meager}{\TYPE{Meager}}
\newcommand{\Comeager}{\TYPE{Comeager}}
\newcommand{\Bair}{\TYPE{Bair}}
\newcommand{\ExtDisc}{\TYPE{ExtremelyDisconnected}}
\newcommand{\SP}{\TYPE{SouslinProperty}}
%FUNC
\DeclareMathOperator*{\intx}{int}
\DeclareMathOperator*{\cl}{cl} 
\DeclareMathOperator*{\boundary}{\partial} 
\DeclareMathOperator{\combo}{\triangledown} 
\DeclareMathOperator{\diag}{\triangle} 
\DeclareMathOperator{\rem}{rem}
%CATS
\newcommand{\TOP}{\mathsf{TOP}}
\newcommand{\HC}{\mathsf{HC}}
\newcommand{\CG}{\mathsf{CG}}
%Symbols
\newcommand{\T}{\mathcal{T}}
\newcommand{\U}{\mathcal{U}}
\newcommand{\V}{\mathcal{V}}
\renewcommand{\O}{\mathcal{O}}
\renewcommand{\d}{\mathrm{d}}
\newcommand{\F}{\mathcal{F}}
\newcommand{\X}{\mathcal{X}}
%\newcommand{\d}{\mathrm{d}}
%Metric Topology
%TYPE
\newcommand{\Lip}{\mathrm{Lip}}
\newcommand{\Complete}{\TYPE{Complete}}
%FUNC
\DeclareMathOperator{\diam}{diam}
\DeclareMathOperator{\osc}{osc}
%CATS
\newcommand{\Semiiso}{\mathsf{SMS}_{\circ \to \cdot}}
\newcommand{\Iso}{\mathsf{MS}_{\circ \to \cdot}}
\newcommand{\SMS}{\mathsf{SMS}}
\newcommand{\MS}{\mathsf{MS}}
\newcommand{\UNI}{\mathsf{UNI}}
\newcommand{\UNIS}{\mathsf{UNIS}}
%Boolean Algebra
%TYPE
\newcommand{\Bool}{\mathbb{B}}
\newcommand{\Alg}{\TYPE{Algebra}}
\newcommand{\BR}{\TYPE{BooleanRing}}
\newcommand{\BA}{\TYPE{BooleanAlgebra}}
\newcommand{\PD}{\TYPE{PairwiseDisjointElements}}
\newcommand{\PoU}{\TYPE{PartitionOfUnity}}
\renewcommand{\SS}{\TYPE{StoneSpace}}
\newcommand{\TK}{\mathcal{TK}}
\newcommand{\BL}{\TYPE{BooleanLattice}}
\newcommand{\Fix}{\mathrm{Fix}}
\newcommand{\OC}{\TYPE{OrderClosed}}
\newcommand{\SOC}{\TYPE{SequentiallyOrderClosed}}
\newcommand{\oC}{\TYPE{OrderContinuous}}
\newcommand{\sC}{\TYPE{\sigma\hyph Continuous}}
\newcommand{\OD}{\TYPE{OrderDense}}
\newcommand{\REing}{\TYPE{RegularEmbedding}}
\newcommand{\REed}{\TYPE{RegularEmbeded}}
\newcommand{\REable}{\TYPE{RegularEmbedable}}
\newcommand{\OComplete}{\TYPE{OrderDedekindComplete}}
\newcommand{\TAlgebra}{\TYPE{\tau\hyph Algebra}}
\newcommand{\OCompletes}{\TYPE{OrderDedekindCompleteSubset}}
\newcommand{\SComplete}{\TYPE{\sigma\hyph DedekindComplete}}
\newcommand{\SCompletes}{\TYPE{\sigma\hyph DedekindCompleteSubset}}
\newcommand{\LS}{\mathcal{LS}}
\newcommand{\POpen}{\TYPE{PseudoOpen}}
\newcommand{\od}{\mathbf{OD}}
\newcommand{\mgr}{\mathbf{MGR}}
\newcommand{\nd}{\mathbf{ND}}
\newcommand{\CCC}{\TYPE{WithCountableChainCondition}}
\newcommand{\CSI}{\TYPE{\omega_1\hyph SaturatedIdeal}}
\newcommand{\WD}{\TYPE{(\sigma,\infty)\hyph WeaklyDistributive}}
\newcommand{\Aless}{\TYPE{Atomless}}
\newcommand{\PA}{\TYPE{PurelyAtomic}}
\newcommand{\Homog}{\TYPE{Homogeneous}}
\newcommand{\FS}{\TYPE{FullSubgroup}}
\newcommand{\CFS}{\TYPE{CountablyFullSubgroup}}
\newcommand{\EI}{\TYPE{ExchangingInvolution}}
\newcommand{\SwS}{\TYPE{SubgroupWithSeparators}}
\newcommand{\SwmI}{\TYPE{SubgroupWithManyInvolutions}}
%FUNC
\DeclareMathOperator{\upr}{upr}
\DeclareMathOperator{\Atom}{Atom}
\DeclareMathOperator{\Supp}{Supp}
\newcommand{\genFS}[1]{\left\langle #1 \right\rangle_\mathrm{F}}
\newcommand{\genCFS}[1]{\left\langle #1 \right\rangle_\mathrm{CF}}
\DeclareMathOperator{\Sep}{Sep}
\DeclareMathOperator{\Tr}{Tr}
%CATS
\newcommand{\BOL}{\mathsf{BOL}}
\newcommand{\BOOL}{\mathsf{BOOL}}
%SYMBOL
\newcommand{\Z}{\mathsf{Z}}
\author{Uncultured Tramp} 
\title{Boolean Algebra}
\begin{document}
\maketitle
\newpage
\tableofcontents
\newpage
\section{Boolean Rings and Algebras}
\subsection{Stone Theory}
\subsubsection{Definition And Examples}
\Page{
	\DeclareType{BooleanRing}{?\RNG}
	\DefineType{B}{BooleanRing}{\forall b \in B \. b^2 = b}
	\\
	\DeclareType{BooleanAlgebra}{?\RING}
	\DefineType{B}{BooleanAlgebra}{\forall b \in B \. b^2 = b}
	\\
	\Theorem{AlgebraOfSubsets}
	{
		\forall X \in \SET \.   
		\BA(?X, \cap  ,\du)
	}
	\Say{[1]}{
		\Intro \RNG \Big( 
			\Lambda A,B,C : ?X \. 
			\LOGIC{CheckingTruthTables}\big(A \cap (B \du c) ), (A \cap B) \du (A \cap C)\big)  
		\Big)
	}
	{
		(?X,\cap,\du) \in \RNG
	}
	\Say{[2]}{\Intro \RING 
		\Big( 
			[1],
			\Lambda A : ?X \. \LOGIC{CheckingTruthTables}\big(  A \cap X, A \Big)
		\Big)
		}
	{
		(?X,\cap,\du) \in \RING
	}
	\Conclude{[*]}{ 
		\Intro \BA \Big( 
			[2], 
			\LOGIC{CheckingTruthTables}\big( A \cap A, A \big)   
		\Big)  
	}
	{
		\BA(X^2,\cap,\du)
	}
	\EndProof
	\\
	\DeclareType{SetTheoreticAlgebra}{\prod_{X \in \SET} ?^3X}
	\DefineNamedType{\A}{SetTheoreticAlgebra}{\TYPE{Algebra}(X)}
	{
		\NewLine \iff
		(\emptyset,X \in \A) 
		\And
		( \forall A,A' \in \A \. A \du A' \in \A)
		\And
		( \forall A,A' \in \A \. A \cap A' \in \A)
	}   
	\\
	\Theorem{SetTheoreticAlgebraIsBoolean}
	{
		\forall X \in \SET \.
		\forall \A : \Alg(X) \. 
		\BA(\A,\cap,\du)
	}
	\NoProof
	\\
	\Conclude{\TYPE{Bool}}{\Bool = \top |\top}{\Type}
	\\
	\DeclareFunc{BooleanAdd}{\Bool \times \Bool \to \Bool}
	\DefineNamedFunc{BooleanAdd}{a,b}{a + b}{a \oplus b}
	\\
	\DeclareFunc{BooleanMult}{\Bool \times \Bool \to \Bool}
	\DefineNamedFunc{BooleanMult}{a,b}{ab}{a \wedge b}
	\\
	\Theorem{BoolIsASet}{\Bool \in \SET}
	\NoProof
	\\
	\Theorem{BoolIsBooleanAlgbra}{\BA(\Bool,\cdot,+)}
	\NoProof
}
\Page{
	\Theorem{BoolIsAField}{\Field(\Bool,\cdot,+)}
	\NoProof
	\\
	\Theorem{BooleanRingHasChar2}
	{
		\forall A : \mathsf{CRNG} \.
		\forall a \in A \. a + a = 0
	}
	\AssumeIn{a}{A}
	\Say{[2]}{\Elim \BR(A)[1](a + a)\THM{BinomialExpansion}(A,a,a,2)\Elim \BR(A)[1](a)}
	{
		 \NewLine :
		 a + a =  
		 ( a + a)^2 =
		 a^2 + 2a^2 + a^2 = 
		 (a + a) + (a + a)
	}
	\Conclude{[*]}{[2] - (a + a)}{a + a) = 0}
	\DeriveConclude{[*]}{\Intro \forall}
	{
		\forall a \in A \. a + a = 0
	}
	\\
	\Theorem{BooleanRingIsCommutative}
	{
		\forall A : \BR \.
		A \in \mathsf{CRNG}
	}
	\Assume{a,b}{A}
	\Say{[1]}
	{
		\Elim \BR(A)(a + b)
		\THM{BinomialExpansion}(A,a,b,2)
		\Elim \BR(A)(a)
		\Elim \BR(A)(a)
	}
	{
	     \NewLine :
	     a + b = 
    	     (a + b)^2
	     a^2 + ab + ba + b^2 = 
	     a + ab + ba + b
	}
	\Say{[2]}{[1] - a - b - ab}{-ab = ba}
	\Conclude{\Big[(a,b).*]}{\THM{BooleanRingHasChar2}[2]}
	{
		ab = ba
	}
	\Derive{[*]}{\Intro \mathsf{CRNG}}{A \in \mathsf{CRNG}}
	\NoProof
	\\
	\Theorem{BooleanSumByParity}
	{
		\forall n \in \Nat \.
		\forall b : [1,\ldots,n] \to \Bool \.
		\bigoplus^n_{i=1} b_i = 1 \iff  \TYPE{Odd}\Big|b^{-1}(1)\Big|
	}
	\Say{\libra}
	{
		\Lambda n \in \Nat \. 
		\forall b : [1,\ldots,n] \to \Bool \.
		\bigoplus^n_{i=1} b_i = 1 \iff \TYPE{Odd}\Big| b^{-1}(1) \Big|
	}
	{
		\Nat \to \Type
	}
	\Assume{b}{[1,\ldots,1] \to \Bool}
	\Assume{[1]}{\bigoplus^1_{i=1} b_i = 1}
	\Say{[2]}{ [1]\Elim \FUNC{iteratedOperator}(\Bool,b,1) }
	{
		1 = \bigoplus^1_{i=1} b_i = b_1
	}
	\Say{[3]}{\THM{SingletonPreimage}[2]}{b^{-1}(1) = \{1\}}
	\Say{[4]}{\Big| [3] \Big|}{\Big| b^{-1}(1)\Big| = 1}
	\Conclude{[1.*]}{\THM{OnnIsOdd}[4]}{ \TYPE{Odd} \Big| b^{-1}(1)\Big| }
	\Derive{[1]}{\Intro \Imply}{ \bigoplus^1_{i=1} b_i \Imply \TYPE{Odd} \Big|b^{-1}(1) \Big|  }
}\Page{
	\Assume{[2]}{\TYPE{Odd} \Big| b^{-1}(1)\Big|}
	\Say{[3]}{
		\THM{SubsetCardinality}\Big([1,\ldots,1],b^{-1}(1)\Big)
		\THM{SingletonCardinality}\Big( [1,\ldots,1] \Big)
	}
	{
		\Big| b^{-1}(1)\Big| \le \Big| [1,\ldots,1] \Big| = 1
	}
	\Say{[4]}{\Elim \TYPE{Odd}[2][3]}
	{
		\Big| b^{-1}(1) \Big| = 1
	}
	\Say{[5]}{\Elim \mathsf{CARD}[4]}{b^{-1}(1)=\{1\}}
	\Say{[6]}{\THM{SingletonPreimage}[5]}{b(1) = 1}
	\Say{[7]}{\Elim \FUNC{IteratedOperatort}[6]}{ \bigoplus^1_{i=1} b_i = 1}
	\Derive{[2]}{\Intro \Imply}
	{
		\TYPE{Odd}\Big| b^{-1}(1) \Big| \Imply \bigoplus^1_{i=1} b_i = 1
	}
	\Conclude{[b.*]}{ \Intro \iff [2][3]    }
	{
		\bigoplus^1_{i=1} b_i = 1 \iff \TYPE{Odd}\Big| b^{-1}(1) \Big| 
	}
	\Derive{[1]}{\Intro \forall \Intro \libra}{\libra(1)}
	\AssumeIn{n}{\Nat}
	\Assume{[2]}{\libra(n)}
	\Assume{b}{[1,\ldots,n+1] \to \Bool}
	\Assume{[3]}{\bigoplus^{n+1}_{i=1} b_i = 1}
	\Say{[4]}{\Elim \FUNC{iteratedOperator}(\oplus)[3]}
	{
		1 = \bigoplus^{n+1}_{i=1} b_i = b_{n+1} \oplus \bigoplus^n_{i=1} b_i
	}
	\Say{[5]}{\Elim \bigoplus [4]}
	{
		b_{n+1} = 0 \And \bigoplus^n_{i=1} b_i = 1
		\Bigg|
		b_{n+1} = 1 \And \bigoplus^n_{i=1} b_i = 0
	}
	\Conclude{[3.*]}{\Elim \libra [2][5]}
	{
		\TYPE{Odd}\Big| b^{-1}(1)\Big|
	}
	\Derive{[3]}{\Imply}
	{
		\bigoplus^{n+1}_{i=1} b_i = 1
		\Imply
		\TYPE{Odd}\Big| b^{-1}(1)\Big|
	}
	\Say{[4]}{\THM{PigeonholePrinciple}}
	{
		\TYPE{Odd}\Big| b^{-1}(1)\Big|	
		\Imply
		\bigoplus^{n+1}_{i=1} b_i = 1
	}
	\Conclude{[n.*]}{\Intro (\iff)[3][4]}
	{
		\bigoplus^{n+1}_{i=1} b_i = 1
		\Imply
		\TYPE{Odd}\Big| b^{-1}(1)\Big|
	}
	\Derive{[n.*]}{\Elim \Nat}
	{
		\forall n \in \Nat \. \libra(n) 
	}
	\EndProof
	\\
	\Theorem{SymmetricSumByParity}
	{
		\forall X \in \SET \.
		\forall n \in \Nat \.
		\forall A : n \to ?X \.
		\sum^n_{i=1} A_n = 
		\Big\{  x \in X : \TYPE{Odd}\big\{ i \in [1,\ldots,n] : x \in A_i  \big\}       \Big\}
	}
	\NoProof
}
\newpage
\Page{
	\Theorem{FiniteBooleanRingCard}
	{
		\forall A : \BR \.
		|A| < \infty \Imply
		\exists k \in \Int_+ \.
		|A| = 2^k 
	}
	\NoProof
	\\
	\Theorem{FiniteBooleanRingIsAlgebra}
	{
		\forall A : \BR \.
		|A| < \infty  \Imply
		\BA(A)
	}
	\NoProof
}
\subsubsection{First form of Stone's Theorem}
\Page{
	\Theorem{FirstStoneLemma}
	{
		\forall A : \BR \.
		\forall I : \Ideal(A) \.
		\forall a \in  A \setminus I \.
		\exists A \Arrow{\phi} \Bool : \RNG \.
		\phi(a) = 1 \And \phi(I) = \{0\}
	}
	\Say{\Big(J,[1]\Big)}{\THM{MaximalIdealExists}(A,I,a)}
	{
		\sum J : \TYPE{MaximaIdeal}(A,a) \. 
		I \subset J 
	}
	\Say{K}{\Lambda b \in A \.  \{ c \in A : cb \in J  \}}{A \to \Ideal(A)}
	\Say{[2]}{\Elim K [1] \Elim \TYPE{MaximalIdeal}(A,a,J)}
	{
		\forall b \in A \.
		a \not \in K_b \Imply K_b = J
	}
	\Say{[3]}{\Elim \BR(A)[2]}
	{
		K_a = J
	}
	\Say{[4]}{[3][2]}{ \forall b \in J^\c \. K_b = J}
	\Say{[5]}{[4][4]}{\forall b,c \in J^\c \. bc \in J^\c}
	\AssumeIn{b,c}{J^\c}
	\Say{[6]}{[5](bc)}{bc \in J^\c}
	\Say{[7]}{[4][6]}{K_{bc} = J}
	\Say{[8]}{ 
		\Elim \RNG(A,bc,b,c)  
		\Elim \BR(A)(b)
		\Elim \BR(A)(c)
		\THM{BooleanRingHasChar2}
	}
	{
		\NewLine :
		bc(b + c) = 
		b^2 c + bc^2 =
		bc + bc =
		0
	}
	\Say{[9]}{\Elim K_{bc}[8]}{b + c \in K_{ab}}
	\Conclude{[*]}{[9][7]}{b + c \in J}
	\Derive{[6]}{\Intro \forall}{\forall b,c \in J^\c \. a + b \in J}
	\Say{\phi}{\Lambda b \in A \. \bigwedge_{j \in J} j\; !\!= b}{A \to \Bool}
	\Say{[7]}{\Elim \phi(a) [1]}{\phi(a) = 1}
	\Say{[8]}{\Elim \phi(I) [1]}{\phi(I) = \{0\}}
	\AssumeIn{c,b}{A}
	\Assume{[9]}{\phi(c + b) = 0 }
	\Say{[10]}{\Elim \phi [9]}
	{
		c + b \in J
	}
	\Say{[11]}{\Elim \TYPE{Ideal}(A,j)[10]}
	{
		c,b \in J \Big| c,b \in J^\c 
	}
	\Conclude{[9.*]}{\Elim \phi [3][6] \Elim \Bool }{\phi(c) + \phi(b) = \phi(c + b)}
	\Derive{[9]}{\Intro \Imply}
	{
		\phi(c+b)=0 \Imply \phi(c + b) = \phi(c) + \phi(b)
	}
	\Assume{[10]}{\phi(c + b) = 1}
	\Say{[10]}{\Elim \phi [9]}
	{
		c + b  \not \in J
	}
	\Say{[11]}{[10][6]}
	{
		(c,b) \in J \times J^\c \Big| (c,b) \in J^\c \times J 
	}
	\Conclude{[10.*]}{ \Elim \phi [11] \Elim \Bool  }{\phi(c) + \phi(b) = \phi(c + b)}
	\Derive{[10]}{\Intro \Imply}
	{
		\phi(c + b) = 1 \Imply\phi(c + b) = \phi(c) + \phi(b)
	}
	\Say{\Big([c,b].*\Big)}{\Elim | \THM{BooleanAlternative}\Big( \phi(c +b) \Big)[4][5]} 
	{
		\phi(c + b) = \phi(c) + \phi(b)
	}
	\Assume{[11]}{\phi(cb) = 1}
	\Say{[12]}{\Elim \phi [11]}
	{
		cb  \not \in J
	}
	\Say{[13]}{\Elim \Ideal[12]}{c,b \not \in J}
	\Conclude{[11.*]}{ [13]\Elim \phi \Elim \Bool}{\phi(c)\phi(b) = \phi(cb)}
	\Derive{[11]}{\Intro \Imply}
	{
		\phi(cb)=1 \Imply \phi(cb) = \phi(c)\phi(b)
	}
	\Assume{[12]}{\phi(cb) = 0}
	\Say{[13]}{\Elim \phi [11]}
	{
		cb  \in J
	}
	\Say{[14]}{[5][14]}{c \in J | b \in J}
	\Conclude{[12.*]}{ [14]\Elim \phi \Elim \Bool}{\phi(c)\phi(b) = \phi(cb)}
}\Page{
	\Derive{[12]}{\Intro \Imply}
	{
		\phi(cb)=0 \Imply \phi(cb) = \phi(c)\phi(b)
	}
	\Conclude{\Big([c,b].*\Big)}{\Elim | \THM{BooleanAlternative}\Big( \phi(cb) \Big)[4][5]} 
	{
		\phi(cb) = \phi(c)\phi(b)
	}
	\DeriveConclude{[*]}{\Intro \RNG}
	{
		\phi \in \RNG(A,\Bool)
	}
	\EndProof
	\\
	\DeclareFunc{StoneSpace}{ \BR \to \SET}
	\DefineNamedFunc{StoneSpace}{A}{Z_A}{\Big\{ A \Arrow{\phi} \Bool : \RNG  : \phi \neq 0 \Big\}}
	\\
	\DeclareFunc{StoneRepresentation}{\prod A : \BR \. A  \to ?Z_A}
	\DefineNamedFunc{StoneRepresentation}{a}{S_A(a)}
	{
		\{ \phi \in Z_A :  \phi(a) = 1 \}
	}
	\\
	\Theorem{StoneTHM1stForm}
	{
		\forall A : \BR \.
		\TYPE{Injective} \And \RNG(A,?Z_A,S_A)
	}
	\Assume{a,b}{A}
	\Assume{[1]}{a \neq b}
	\Assume{[2]}{a \in \langle b \rangle, b \in \langle a \rangle}
	\Say{\Big( c,d,[3]\Big)}{\Elim \Ideal [2]}
	{
		b = ac \And a = bd 
	}
	\Say{[4]}{[3]^3\Elim \BR(A)}
	{
		a = acd = bcd^2 = bcd \And b = bcd 
	}
	\Conclude{[5]}{[4][1]}{\bot}
	\Derive{[2]}{\Elim \bot}
	{
	 	a \not \in \langle b \rangle 
		 \Big|
		b \not \in \langle a \rangle
	}
	\Assume{[3]}{a \not \in \langle b \rangle}
	\Say{\Big(\phi,[4]\Big)}{\THM{FirstStoneLemma}\Big(A,\langle b\rangle, a \Big)}
	{
		\sum A \Arrow{\phi} \Bool : \phi(a) = 1 \And \phi\langle b \rangle = \{0\}
	}
	\Say{[5]}{\Elim \FUNC{image}[4]}{\phi(b) = 0}
	\Conclude{[3.*]}{\Elim S_A [3][4]}
	{
		S_A(a) \neq S_A(b)
	}
	\Derive{[3]}{\Intro \Imply}{a \not \in \langle b \rangle \Imply S_A(a) \neq S_A(b)}
	\Assume{[4]}{b \not \in \langle a \rangle}
	\Say{\Big(\phi,[5]\Big)}{\THM{FirstStoneLemma}\Big(A,\langle a\rangle, b \Big)}
	{
		\sum A \Arrow{\phi} \Bool : \phi(b) = 1 \And \phi\langle a \rangle = \{0\}
	}
	\Say{[6]}{\Elim \FUNC{image}[5]}{\phi(a) = 0}
	\Conclude{[4.*]}{\Elim S_A [3][4]}
	{
		S_A(a) \neq S_A(b)
	}
	\Derive{[4]}{\Intro \Imply}{b \not \in \langle a \rangle \Imply S_A(a) \neq S_A(b)}
	\Conclude{\Big[(a,b).*\Big]}{\Elim(|)[2][3][4]}{S_A(a) \neq S_A(b)}
	\Derive{[*.1]}{\Intro \TYPE{Injective}}{\TYPE{Injective}\Big(A,?Z_A, S_A\Big)}
	\Conclude{[*.2]}{\Elim Z_A \Elim \RNG \Elim \Bool \Intro \RNG}
	{
		S_Z \in \RNG(A,?Z_A)
	}
	\EndProof
}
\newpage
\subsubsection{Translating set theoretic notions}
\Page{
	\DeclareFunc{andOperator}{\prod A : \BR(X) \. A^2 \to A}
	\DefineNamedFunc{andOperator}{a,b}{a \cap b}{ab}
	\\
	\DeclareFunc{andOperator}{\prod A : \BR(X) \. A^2 \to A}
	\DefineNamedFunc{andOperator}{a,b}{a \cup b}{a + b + ab}
	\\
	\DeclareFunc{symmetricDifferenceOperator}{\prod A : \BR(X) \. A^2 \to A}
	\DefineNamedFunc{symmetricDifferenceOperator}{a,b}{a \du b}{a + b}
	\\
	\DeclareFunc{setMinusOpera}{\prod A : \BR(X) \. A^2 \to A}
	\DefineNamedFunc{andOperator}{a,b}{a \setminus b}{ (a + b)a}
	\\
	\DeclareType{Disjoint}{\prod A : \BR(X) \. ?A^2  }
	\DefineNamedType{(a,b)}{Disjoint}{a \bot b}{ab = 0}
	\\
	\DeclareType{\PD}{\prod A : \BR(X) \. ??A  }
	\DefineType{P}{\PD}{ \forall a,b \in A  \.  a \bot b}	
	\\
	\DeclareType{PartitionOfUnity}{\prod A : \BR(X) \. ?\PD(A)  }
	\DefineType{P}{PartitionOfUnity}
	{
		\forall c \in A \. c \neq 0 \Imply \exists a  \in P : ac \neq 0 
	}
	\\
	\Theorem{PartitionOfUnityIsMaximalDisjoint}
	{
		\forall A : \BR \.
		\forall P : \PD(A) \.
		\NewLine \.
		\PoU(A,P) \iff
		P \in \max \PD(A)
	}
	\NoProof
	\\
	\Theorem{DisjointHasPartitionOfUnity}
	{
		\forall A : \BR \.
		\forall B : \PD(A) \. \NewLine \. 
		\exists P : \PoU(A) \.
		P \subset B 
	}
	\NoProof
	\\
	\DeclareType{Refinement}{\prod A : \BR \. \PoU(A) \to ?\PoU(A) }
	\DefineType{Q}{Refinement}{\Lambda P : \PoU(A) \. \forall p \in P \. \exists q \in Q : pq = q }
}
\newpage
\subsubsection{Order of Boolean Ring}
\Page{
	\DeclareType{BooleanOrder}{\prod A : \BR \. ?(A \times A)}
	\DefineNamedType{a,b}{BooleanOrder}{a \le b}{ab = a}
	\\
	\Theorem{BooleanOrderByStoneRepresentation}
	{
		\forall A : \BR \.
		\forall a,b \in A \. 
		a \le b \iff S_A(a) \subset S_A(b)
	}
	\NoProof
	\\
	\Theorem{BooleanOrderIsPartialOrder}
	{
		\forall A : \BR \.
		\TYPE{PartialOrder}\Big(A, \le \Big)
	}
	\NoProof
	\\
	\DeclareFunc{booleanRingAsPoset}{\BR \to \POSET}
	\DefineNamedFunc{booleaRingAsPoset}{A}{A}{(A,\le)}
	\\
	\Theorem{MinimalElementInBooleanRing}
	{
		\forall A : \BR \. \min A = 0 
	}
	\NoProof
	\\
	\Theorem{MaximalElementInBooleanAlgebra}
	{
		\forall A : \BA \. \max A = e_A 
	}
	\NoProof
	\\
	\Theorem{BooleanRingIsLattice}
	{
		\forall A : \BR \. (A,\cap,\cup) \in \mathsf{LAT}
	}
	\NoProof
	\\
	\DeclareFunc{booleanRingAsLattice}{\BR \to \mathsf{LAT}}
	\DefineNamedFunc{booleaRingAsLattice}{A}{A}{(A,\cap,\cup)}
}
\newpage
\subsubsection{Topology of Stone Space}
\Page{
	\DeclareFunc{StoneTopology}{\prod A : \BR \. ??Z_A}
	\DefineNamedFunc{StoneTopology}{}{\T}
	{
		\Big\{ U \subset Z_A : \forall u \in U \. \exists a \in A :  u \in S_A(a) \subset U   \Big\}
	}
	\\
	\Theorem{StoneTopologyIsTopology}
	{
		\forall A : \BR \. \TYPE{Topology}(Z_A,\T_A)
	}
	\Say{[1]}{\Elim \T_A}{\emptyset \in \T_A}
	\AssumeIn{f}{Z_A}
	\Say{\Big(a,[2]\Big)}{\Elim Z_A(f)}{\sum a \in A \. f(a) = 1}
	\Say{[3]}{\Elim S_A [2]}{ f \in S_A(a)}
	\Conclude{[f.*]}{\Elim S_A(a)}
	{
		S_A(a) \subset Z_A
	}
	\Derive{[1]}{\Elim \T_Z}{ Z_A \in \T_Z}
	\Assume{\U}{?\T_Z}
	\AssumeIn{u}{\bigcup \U}
	\Say{\Big(U,[2]\Big)}{\Elim \FUNC{union} \U}{ \sum U \in \U \. u \in U  }
	\Say{[3]}{\Elim \TYPE{Subset}(\U,U)}{U \in \T_A}
	\Say{\Big(a,[4]\Big)}{\Elim \T_Z(U,u)}
	{
		\sum a \in A \.  u \in S_A(a) \subset U
	}
	\Conclude{[u.*]}{\THM{SubsetOfUnion}[4](\U,U)}{ u \in S_A(a) \subset \bigcup \U }
	\DeriveConclude{[\U.*]}{\Elim \T_A}{\bigcup \U \in \T_A}
	\Derive{[2]}{\Intro \forall}{\forall \U \in ?\T_A \. \bigcup \U \in \T_A}
	\AssumeIn{n}{\Nat}
	\Assume{U}{[1,\ldots,n] \to \T_A}
	\AssumeIn{u}{\bigcap^n_{i=1} U_n }
	\Say{[3]}{\Elim \FUNC{intersect}(U,u)}{ \forall n \in \Nat \. u \in U_n}
	\Say{\Big(a,[4]\Big)}{\Elim \T_A [3] }
	{
		\sum a : [1,\ldots,n]  \to A : \forall i \in [1,\ldots,n] \. u \in S_A(a_n) \subset U_n 
	}
	\Say{b}{\prod^n_{i=1} a_i}{A}
	\Say{[5]}{\Elim S_A [4]}{ \forall i \in [1,\ldots,n] \. u(a_n) = 1}
	\Say{[6]}{ \Elim b \Elim \RNG (A,\Bool)[5] \Elim \Bool  }{ 
		u(b) = 
		u\left(\prod^n_{i=1} a_i \right) =  
		\prod^n_{i=1} u(a_i) = 
		\prod^n_{i=1} 1 =
		1
	}
	\Conclude{[u.*]}{\Elim S_A(b)[6] \Elim b \Elim S_A(a) \THM{IntersectOfSubsets}(S_A(a),U)[4] }
	{ 
		u \in S_A(b) \subset \bigcap^n_{i=1} S_A(a_n) \subset \bigcap^n_{i=1} U_n
	}
	\DeriveConclude{[n.*]}{\Elim \T_A}{\bigcap^n_{i=1} U_n \in \T_A}
	\Derive{[3]}{\Intro \forall \Intro \forall}
	{
		\forall n \in \Nat \. 
		\forall U : [1,\ldots,n] \to \T_A \.
		\bigcap^n_{i=1} U_n \in \T_A
	}
	\Conclude{[*]}{\Intro \TYPE{Topology}[1][2][3]}{\TYPE{Topology}(Z_A,\T_A)}
	\EndProof
}
\Page{
	\DeclareFunc{StoneSpace}{\BR \to \TOP}
	\DefineNamedFunc{StoneSpace}{A}{Z_A}{(Z_A,\T_A)}
	\\
	\Theorem{StoneRepresentationIsOpen}
	{
		\forall a \in A \. S_A(a) \in \T(Z_A)
	}
	\AssumeIn{f}{S_A(a)}
	\Conclude{[f.*]}{\Elim f \THM{SelfSubset}(S_A(a))}{ f \in \S_A(a) \subset \S_A(a)}
	\DeriveConclude{[*]}{\Elim Z_A \Elim \T_A}{S_A(a) \in \T(Z_A)}
	\EndProof
	\\
	\Theorem{StoneSpaceIsHausdorff}
	{
		\forall A : \BR \. \TYPE{T2}(Z_A)
	}
	\AssumeIn{u,v}{Z_A}
	\Assume{[1]}{u \neq v}
	\Say{\Big( a, [2] \Big)}{\Elim Z_A[1]}
	{
		\sum a \in A \. u(a) = 1 \And v(a) = 0 | u(a)=0 \And v(a) = 1
	}
	\Assume{[3]}{u(a) = 1 \And v(a) = 0}
	\Say{\Big(b,[4])}{\Elim Z_A(v)}{\sum b \in A \. v(b) = 1}
	\SayIn{b'}{ b + ba}{A}
	\Say{[5]}{\Elim b'}{u(b') = 0 \And v(b') = 1}
	\Say{[6]}{\Intro S_A [3]}{ u \in S_A(a) \And v \not \in S_A(a)}
	\Say{[7]}{\Intro S_A [5]}{v \in S_A(b') \And u \not \in S_A(b')}
	\Conclude{[8]}{\Elim b' \Elim S_A}{S_A(a) \cap S_A(b') = \emptyset}
	\DeriveConclude{[*]}{\THM{StoneRepresentationIsOpen} \Intro \TYPE{T2}}{\TYPE{T2}(Z_A)}
	\EndProof
}\Page{
	\Theorem{StoneRepresentationIsClopen}
	{
		\forall A : \BR(A) \.
		\forall a \in A \.
		\forall \Clopen\big( Z_A, S_A(a) \big)
	}
	\AssumeIn{f}{S_A^\c(a)}
	\Say{[1]}{\Elim S_A(a,f)}{f(a) = 0}
	\Say{\Big(b,[2]\Big)}{\Elim Z_A(f)}{\sum_{b \in A} f(b) = 1}
	\Say{[3]}{[1][2]\Elim \RNG(A,\Bool,f)}{f( b + ba )  = 1}
	\Say{[4]}{\Elim S_A(b + ba) [3]}{ f \in S_A(b + ba)}
	\Conclude{[f.*]}{\THM{UnionMembership}[4] }{f \in \bigcup_{b \in A} S_A(b + ba)}
	\Derive{[1]}{\Intro \subset}{S_A^\c(a) \subset \bigcup_{b \in A} S_A(b + ba)}
	\AssumeIn{f}{\bigcup_{b \in A} S_A}
	\Say{\Big(b,[2]\Big)}{\Elim \cup(f)}{\sum b \in A \. f \in S_A(b + ba)}
	\Say{[3]}{\Elim S_A [2]}{f(b + ba) = 1}
	\Say{[4]}{\Elim \RNG(A,\Bool,f)[3]}{f(a) = 0}
	\Conclude{[f.*]}{\Elim S_A [4]}{f \in S_A^\c(a)}
	\Derive{[2]}{\Intro \subset}{ \bigcup_{b \in A} S_A(b + ba) \subset S_A^\c(a)}
	\Say{[3]}{\Intro \TYPE{SetEq}[1][2]}{S_A^\c(a) = \bigcup_{b \in A} S_A(b + ba)}
	\Say{[4]}{\THM{StoneRepresentationIsOpen}(A)[3]\Elim \TYPE{Topology}(Z_A,\T_A)}
	{
		S_A^\c(a) \in \T(Z_A)
	}
	\Say{[5]}{\Intro \Closed(Z_A)[4]}{\Closed\Big(Z_A,S_A(a)\Big)}
	\Conclude{[*]}{\THM{SoneRepresentationIsOpen}(A,a)[5]\Intro \Clopen}{\Clopen\Big(Z_a,S_A(a)\Big)}
	\EndProof
	\\
	\Theorem{StoneSpaceIsZeroDimensional}
	{
		\forall A : \BR \. \dim_\TOP Z_A = 0
	}
	\NoProof
}\Page{
	\Theorem{StoneRepresentationIsCompact}
	{
		\forall A : \BR \.
		\forall a \in A \. 
		\Compacts(Z_A,S_A(a))
	}
	\Assume{\F}{\TYPE{Ultrafilter}\; \T\Big(S_A(a)\Big)}
	\AssumeIn{a}{A}
	\Say{[1]}{\Elim \T_A \Elim \TYPE{Ultrafilter}(\F)}
	{
		\TYPE{ConvergentFilter}(\Bool,\F(a))
	}
	\Conclude{f(a)}{\lim \F(a)}{\Bool}
	\Derive{f}{\Intro \to}{ A \to \Bool  }
	\Say{[2]}{\Elim f}{\forall b \in A \. f(b) = \lim \F(b)}
	\Say{[3]}{\Elim \lim \F(A)}{\forall b \in A \. \forall U \in \U(f(b)) \. U \in \F(b)}
	\Say{[4]}{\Elim S_A(a) \Elim \F \Elim f}{f(a) = 1}
	\AssumeIn{b,c}{A}
	\Say{\Big(U,[5]\Big)}{\Elim f(b+c)}{\sum U \in \F \. \forall u \in U \. u(b+c) =f(b+c)}
	\Say{\Big(V,[6]\Big)}{\Elim f(b)}{\sum V \in \F \. \forall v \in V \. v(b) =f(b)}
	\SayIn{W}{U \cap V}{W}
	\Say{[7]}{\Elim W [5][6]}{\forall w \in W \. w(c) = f(b + c) + f(b)}
	\Conclude{\Big[ (b,c).* \Big]}{\Elim f [7]}{f(b + c) = f(b) + f(c)}
	\Derive{[5]}{\Intro \forall}{\forall b,c \in A \. f(b + c) = f(b) + f(c)}
	\AssumeIn{b,c}{A}
	\Say{\Big(U,[6]\Big)}{\Elim f(bc)}{\sum U \in \F \. \forall u \in U \. u(bc) =f(bc) \And u(b) = f(b) \And u(c) = f(c)}
	\Conclude{\Big[7.*\Big]}{\Elim f [6]}{f(bc) = f(b)f(c)}
	\Derive{[6]}{\Intro \forall}{\forall b,c \in A \. f(bc) = f(b)f(c)}
	\Say{[7]}{\Elim S_a(A)[4][5][6]}{f \in S_a(A)}
	\AssumeIn{U}{\U_(f) }
	\Say{\Big( b, [8]\Big)}{\Elim \T_A(U,f)}{\sum b \in A \. f \in S_A(b) \subset U}
	\Say{[9]}{\Elim S_A(b)[8]}{\forall s \in S_A(a) \. s \in  S_A(b)  \iff  s(b) = f(b) = 1}
	\Say{\Big( V, [10]\Big)}{\Elim f [9][8]}{\sum V \subset S_A(b) \subset U \. V \in \F}
	\Conclude{[U.*]}{\Elim \TYPE{Filter}\Big( \T\big(S_A(a)\Big), \F \Big)[10]}{U \in \F}
	\DeriveConclude{[\F.*]}{\Elim \TYPE{FilterLimit}}{f = \lim \F }
	\Derive{[1]}{\THM{CompactByUltrafilters}}{\Compact\big( S_A(a)\big)}
	\Conclude{[*]}{\THM{CompactAsSubset}[1]}{\Compacts\big(Z_A,S_A(a)\big)}
	\EndProof
	\\
	\Theorem{StoneSpaceIsLocallyCompact}
	{
		\forall A : \BR \. \LCompact(Z_A)
	}
	\NoProof
}
\Page{
	\Theorem{CompactOpenAreStoneRepresentation}
	{
		\NewLine ::
		\forall A : \BR \.
		\forall U \in \T(Z_A) \.
		\Compacts(Z_A,U)
		\Imply
		\exists a \in A \. U = S_A(a)
	}
	\Say{[1]}{\Elim \T_A(U)}
	{
		\forall f \in U \. 
		\exists a \in A :
		f \in S_A(a) \subset U
	}
	\Say{\Big(n,a,[2]\Big)}{\Elim \Compacts(Z_A,U)[1]}
	{
		\sum^\infty_{n=1} \sum_{a :[1,\ldots,n] \to A} U = \bigcup^n_{i=1} S_A(a_i)
	}
	\SayIn{b}{\bigcup^\infty_{n=1}a_i}{A}
	\AssumeIn{f}{U}
	\Say{\Big(i,[3]\Big)}{\Elim \FUNC{union}[2]}{f \in S_A(a_i)}
	\Say{[4]}{\Elim S_A(a_i)(f)}{f(a_i) = 1}
	\Say{[5]}{\Elim b\Elim \RNG(A,\Bool,f)[4]}{f(b) = 1}
	\Conclude{[f.*]}{\Elim S_A(b)[5]}{ f \in S_A(b)  }
	\Derive{[3]}{\Intro \subset}{U \subset S_A(b)}
	\AssumeIn{f}{S_A(b)}
	\Say{[4]}{\Elim S_A}{f(b) = 1}
	\Say{ \Big(i,[5] \Big) }{\Elim \RNG(A,\Bool,f)\Elim b[4]}{\sum^n_{i=1} f(a_i) = 1}
	\Say{[6]}{\Elim S_A [5]}{f \in S_A(a_i)}
	\Conclude{[f.*]}{\THM{SubsetUnion}(S_A(a_i),S_A(a))[3]\Elim \subset}{f \in U}
	\Derive{[4]}{\Intro \subset}{U \subset S_A(b)}
	\Conclude{[*]}{\Intro \TYPE{SetEq}[3][4]}{U =S_A(b)}
	\EndProof
	\\
	\Theorem{BooleanAlgebraByCompactness}
	{
		\forall A : \BR \.
		\BA(A) \iff \Compact(Z_A)
	}
	\NoProof
	\\
	\Theorem{StoneSpaceAsCantorSubset}
	{
		\forall A : \BR \.
		Z_A \subset_\TOP \Bool^A
	}
	\NoProof
	\\
	\Conclude{\SS}{\TYPE{T2} \And \LCompact \And \TYPE{OneDimensional}}{?\TOP}
	\\
	\Conclude{\TK}{\Lambda X \in \TOP \. \Compacts \And \TYPE{Open}(X) }{\prod_{X \in \TOP} \Alg(X)}
}\Page{
	\Theorem{StoneHomeomorphism}
	{
		\forall X : \SS \. 
		Z_{\TK(X)} \cong_\TOP X
	}
	\AssumeIn{f}{Z_{\TK(X)}}
	\Say{\Big(U,[1]\Big)}{\Elim Z_{\TK(X)}}
	{
		\sum U  \in \TK(X) \.  f(U) = 1
	}
	\Say{\A}{\{ A \in \TK(X) : f(A) = 1  \}}{?\TK(X)}
	\Say{[2]}{\Elim \A [1]}{\A \neq \emptyset}
	\Say{[3]}{\Elim Z_{\TK(X)} \Elim \A\Elim \RNG(\TK(X),\Bool)}
	{
		\forall A,B \in \A \. A \cap B \neq \emptyset
	}
	\Say{[4]}{\THM{CantorIntersectionTHM}[2][3]}{\bigcap \A \neq \emptyset}
	\AssumeIn{a}{\bigcap \A}
	\Say{[5]}{\Elim \A [1]}{a \in U}
	\Assume{V}{\TK(\A)}
	\Assume{[6]}{a \in V }
	\Say{[7]}{\Intro \FUNC{Intersect} [5][6]}{a \in V \cap U}
	\Say{[8]}{\Elim \A(a)[7]}{f(V \setminus U) = 0 = f(U \setminus C)}
	\Say{[9]}{\Elim Z_{\TK(X)}[1]\Elim \RNG\Big(\TK(X),\Bool,f\Big)[8]}
	{
		1 =
		f(V \cup U) =
		f(V \setminus U) + f(U\cap V) + f(U \setminus V) = 
		f(U \cap V)
	}
	\Say{[10]}{\Elim Z_{\TK(X)}[9]}{f(V) = 1}
	\Conclude{[a.*]}{\Elim \A [10]}{V \in \A}
	\Derive{[5]}{\Intro \iff \Elim \A }{
		\forall a \in \bigcap \A \. 
		\forall V \in \TK(X) \.
		a \in V
		\iff
		f(V) = 1
	}
	\Assume{a,b}{\bigcup \A}
	\Say{\Big(V,[6]\Big)}{\Elim \SS(X)[5](a,b))}{
		\sum V \in \TK(X) \. a \in V \And b \not \in V
	}
	\Say{[7]}{[5][6]}{ 1 = f(V) = 0}
	\Conclude{\Big[(a,b).* \Big]}{[7][7]}{\bot}
	\Derive{[6]}{\Intro \mathsf{CARD}}{\left| \bigcap \A \right| \le 1}
	\Say{[7]}{[4][6]}{ \left| \bigcap \A \right| = 1 }
	\Say{\Big(\varphi(f), [8] \Big)}{\Elim \TYPE{Singleton}[7]}
	{
		 \sum \varphi(f) \in X \. \bigcap \A = \Big\{ \varphi(X)  \Big\}
	}
	\Conclude{[f.*]}{ [5]\big( \varphi(f), [8] \big)  }
	{
		\forall W \in \TK(X) \. \varphi(f) \in W \iff f(W) = 1 
	}
	\Derive{\Big( \varphi, [1] \Big)}{\Intro \sum}
	{
		\sum \varphi : Z_{\TK(X)} \to X \.
		\forall f \in Z_{\TK(X)} \.
		\forall W \in \TK(X) \. \varphi(f) \in W \iff f(W) = 1 
	}
	\Say{[2]}{\Elim Z_{\TK(X)} [1] \Intro \varphi }{ \TYPE{Bijection}\Big( Z_{\TK(X)}, X, \varphi \Big) }
	\AssumeIn{U}{\T(X)}
	\Say{[3]}{\Elim \varphi^{-1}(U)}
	{
		\varphi^{-1}(U) =
		\bigcup \Big\{  S_{\TK(X)}(V) \Big|  V \in \TK(X) \And V \subset U   \Big\}
	}
	\Conclude{[U.*]}{
		\Elim \TYPE{Topology}(Z_{\TK(X)},\T_{\TK(X)})[3]
	}
	{
		\varphi^{-1}(U) \in \T_{\TK(X)}
	}
	\Derive{[3]}{\Elim \TOP}{\varphi \in \TOP(Z_{\TK(X)},X)}
	\Say{[4]}{\Elim \varphi}
	{
		\forall V \in \TK(X) \. \varphi\Big( S_{\TK(X)}(V) \Big)  = V
	}
	\Conclude{[*]}{\Elim \TYPE{Base}\Big( S_{\TK(X)}\Big) [4][3]}
	{
		\TYPE{Homeo}\Big( Z_{\TK(X)}, X, \varphi  \Big)
	}
	\EndProof
}
\newpage
\subsubsection{Identifying  Lattices as Boolean Algebras}
\Page{
	\DeclareType{\BL}{? \Big(\TYPE{DistributiveLattice} \And \TYPE{ComplimentaryLattice}\Big)}
	\DefineType{L}{\BL}{\exists 0 \in L :  0 = \min L \And \forall a \in L \. a \wedge \neg a = 0}
	\\
	\Theorem{BooleanAlgebraIsBooleanLattice}{\forall A : \BA \. \BL(A)}
	\AssumeIn{a,b,c}{A}
	\Conclude{\Big[(a,b,c).*]}{\Elim (\cap,\cup) \Elim \RNG(A) \Elim \BA(A) \Intro (\cap,\cup)}
	{
		\NewLine :
		a \cap (b \cup c) = 
		a ( b + c + bc) = 
		ab + ac + abc =
		ab + ac + a^2 bc =
		(a \cap b) \cup (a \cap c)
	}
	\Derive{[1]}{\Intro \TYPE{DistributiveLattice}}
	{
		\TYPE{DistributiviveLattice}(A)
	}
	\Say{n}{\Lambda a \in A \. e + a}{A \to A}
	\AssumeIn{a}{A}
	\Conclude{\Big[a.*]}{\Elim n \THM{BooleanRingHasChar2}(A)}
	{
		n^2(a) = e + e + a = a
	}
	\Derive{[2]}{\Intro \forall}{\forall a \in A \. n^2(a) = a}
	\AssumeIn{a,b}{A}
	\Assume{[3]}{a \le b}
	\Say{[4]}{\Elim \TYPE{BooleanOrder}[3]}{ ab = a  }
	\Say{[5]}{\Elim(n(a)n(b)) [4] \THM{BooleanRingHasChar2}(A) \Intro n(b)   }
	{
		\NewLine :
		n(a)n(b) = 
		(a + e)(b + e) = 
		ab + b + a + e =
		b + e = 
		n(b)
	}
	\Conclude{\Big[(a,b).*\Big]}{\Intro \TYPE{BooleanOrder}}{n(b) \le n(a)}
	\Derive{[3]}{\Intro \Imply\Intro \forall}
	{
		\forall a,b \in A \.
		a \le b \Imply n(b) \le n(a)
	}
	\Say{[4]}{\Intro \TYPE{Complement}[2][3]}
	{
		\TYPE{Complement}(A,n)
	}
	\Say{[5]}{\Elim \TYPE{BooleanOrder}\Intro \min \Intro 0}{0 = \min A}
	\AssumeIn{a}{A}
	\Conclude{[a.*]}{\Elim \cap \Elim n(a) \Elim \RNG(A) \Elim \BR(A) \THM{BooleanRingHasChar2}}
	{
		a \cap n(a) = 
		a(a + e) = 
		a^2 + a  =
		a + a =
		0
	}
	\Derive{[6]}{\Intro \forall}
	{
		\forall a \in A \. 
		a \cap n(a) = 0
	}
	\Conclude{[*]}{\Intro \BL [1][4][5][6]}{\BL(A)}
	\EndProof
}\Page{
	\Theorem{DeMorganaLaw1}
	{
		\forall L : \BL \.
		\forall a,b \in L \.
		\neg(a \vee b) =  \neg a \wedge \neg b
	}
	\Say{[1]}{\Elim(a \vee b,b)}
	{
		a \vee b \ge a
	}
	\Say{[2]}{\Elim(a \wedge b,a)}
	{
		a \vee b \ge b
	}
	\Say{[3]}{\Elim \TYPE{ComplimentaryLattice}(L)[1]}
	{
	  	\neg(a \vee b) \le \neg a
	}
	\Say{[4]}{\Elim \TYPE{ComplimentaryLattice}(L)[2]}
	{
	  	\neg(a \vee b) \le \neg b
	}
	\Say{[5]}{\Intro \wedge [3][4]}{\neg(a \vee b) \le \neg a \wedge \neg b}
	\AssumeIn{c}{L}
	\Assume{[6]}{c \le \neg a \wedge \neg b}
	\Say{[7]}{\Elim \wedge [6]}{ c \le \neg a \And c \le \neg b }
	\Say{[8]}{\neg [6]}{ \neg c \ge a \And \neg c \ge b }
	\Say{[9]}{\Intro \vee}{ \neg c \ge a \vee b}
	\Conclude{[c.*]}{\neg [9]}{ c \le \neg (a \vee b)}
	\Derive{[6]}{\Intro \forall}
	{
		\forall c \in L \.  
		c \le \neg a \wedge \neg b
		\Imply
		c \le \neg (a \vee b)
	}
	\Conclude{[*]}{\Elim \mathsf{LAT}(L)[1][6]}
	{
		\neg a \wedge \neg b = \neg (a \vee b )
	}
	\EndProof
	\\
	\Theorem{DeMoraganaLaw}{
		\forall L : \BL \. 
		\forall a,b \in L \. 
		\neg a \vee \neg b = \neg(a \wedge b)
	}
	\NoProof
	\\
	\Theorem{BooleanAlgebraIdentification}
	{
		\forall L : \BL \.
		\exists \Delta : L \times L \to L \.
		\NewLine \. 
		\BA(L,\wedge,\Delta) \And 
		\FUNC{order}(L) = \FUNC{order}(L,\wedge,\Delta)
	}
	\Say{\oplus}{\Lambda a,b \in L \. (a \wedge \neg b) \vee (\neg a \wedge b) }
	{
		L \times L \to \L
	}
	\AssumeIn{a,b,c}{L}
	\Conclude{\Big[(a,b,c).*\Big]}{
		\Elim (a \oplus b) \Elim \oplus (c)
		\Elim \TYPE{DestributiveLattice}(L)
		\THM{DeMorganaLaw1}(L)
		\THM{DeMorganaLaw2}(L)
		\NewLine
		\Elim \TYPE{DestributiveLattice}(L)
		\THM{DeMorganaLaw1}(L)
		\THM{DeMorganaLaw2}(L)
		\Intro (b \oplus c) \Intro \oplus (a)
	}
	{
		\NewLine :
		(a \oplus b) \oplus c = 
		\Big( (a \wedge \neg b) \vee (\neg a \wedge b) \Big) \oplus c = \NewLine =  
		\bigg( \Big( (a \wedge \neg b) \vee (\neg a \wedge b) \Big) \vee \neg c \bigg)
		\vee
		\bigg( \neg\Big( (a \wedge \neg b) \vee (\neg a \wedge b) \Big) \vee  c \bigg) = \NewLine =
		(a \wedge \neg b \neg c) \vee (\neg a \wedge b \neg c) \vee 
		\neg a \wedge \neg b \wedge c \vee a \wedge b \wedge \neg c = \NewLine = 
		\bigg( a \vee \neg\Big( (b \wedge \neg c) \vee (\neg b \wedge c) \Big)  \bigg)
		\vee
		\bigg( \neg a \vee \Big( (b \wedge \neg c) \vee (\neg b \wedge c) \Big)  \bigg) =
		a \oplus \Big( (b \wedge \neg c) \vee (\neg b \wedge c) \Big)  = 	
		a \oplus (b \oplus c ) 
	}
	\Derive{[1]}{\Intro \TYPE{Associative}}{\TYPE{Associative}(L,\oplus)}
	\AssumeIn{a}{L}
	\Conclude{[a.*]}{
		\Elim (a \oplus 0) 
		\Elim \BL(L) 
		\Elim \mathsf{LAT}(L) 
	}
	{
		a \oplus 0 = 
		(a \wedge \neg 0) \vee (\neg a \wedge 0) =
		a \vee 0 =
		a
	}
	\Derive{[2]}{\Intro \TYPE{Neutral}}{\TYPE{Neutral}(L,\oplus,0)}
	\AssumeIn{a}{L}
	\Say{[3]}{
		\Elim (a \oplus a) 
		\Elim \BL(L) 
		\Elim \mathsf{LAT}(L) 
	}
	{
		a \oplus a = 
		(a \wedge \neg a) \vee (\neg a \wedge a) =
		0 \vee 0 =
		0
	}
	\Conclude{[*]}{\Intro \TYPE{Invertible}[1][2]}
	{
		\TYPE{Invertible}\Big( L, \oplus,a \Big)
	}
	\Derive{[3]}{\Intro \forall}
	{
		\forall a \in L  \. \TYPE{Invertible}(L,\oplus,a)
	}
	\Say{[4]}{\Intro \GRP [1][2][3]}
	{
		(L,\oplus) \in \GRP
	}
}\Page{
	\Assume{a,b,c}{L}
	\Conclude{\Big[(a,b,c).*\Big]}
	{
		\Elim (a \oplus b)
		\Elim \TYPE{DestributiveLattice}(L)
		\Elim \BL(L)
		\Elim \TYPE{DestributiveLattice}(L)
		\Intro (\oplus)
	}
	{
		\NewLine :
		( a \oplus b) \wedge c = 
		\Big( (a \wedge \neg b) \vee (\neg a \wedge b) \Big) \wedge c =
		(a \wedge \neg b \wedge  c) \vee (\neg a \wedge b \wedge c) =
		\NewLine = 
		(a \wedge c) \wedge \neg(b \wedge c) \vee \neg(a \wedge c) \wedge (b \wedge c) =
		(a \wedge c) \oplus (b \wedge c)
	}
	\Derive{[5]}{\Intro \RNG}{ (L,\oplus,\wedge) \in \RNG }
	\Say{[6]}{\Elim \BL(L) [5] \Intro \RING}{(L,\oplus,\wedge) \in \RING}
	\Say{[7]}{\Elim \mathsf{LAT}(L)[6]}{\BA(L,\oplus,\wedge)}
	\Conclude{[*]}{\Elim \mathsf{LAT}(L)\Intro \FUNC{order}}
	{
		\FUNC{order}(L) = \FUNC{order}(L,\wedge,\Delta)	
	}
	\EndProof
}
\newpage
\subsubsection{Extension of boolean rings to algebras}
\Page{
	\DeclareFunc{doubleAddition}{\prod A : \BR \. \Big((A \sqcup A) \times (A \sqcup A)\Big) \to (A \sqcup A)}
	\DefineNamedFunc{doubleAddition}{ (0,a),(0,b) }{(0,a) +' (0,b)}{(0,a + b)}
	\DefineNamedFunc{doubleAddition}{ (0,a),(1,b) }{(0,a) +' (1,b)}{(1,a + b)}
	\DefineNamedFunc{doubleAddition}{ (1,a),(0,b) }{(1,a) +' (0,b)}{(1,a + b)}
	\DefineNamedFunc{doubleAddition}{ (1,a),(1,b) }{(1,a) +' (1,b)}{(0,a + b)}
	\\
	\DeclareFunc{doubleMult}{\prod A : \BR \. \Big((A \sqcup A) \times (A \sqcup A)\Big) \to (A \sqcup A)}
	\DefineNamedFunc{doubleMult}{ (0,a),(0,b) }{(0,a) \cdot' (0,b)}{(0,ab)}
	\DefineNamedFunc{doubleMult}{ (0,a),(1,b) }{(0,a) \cdot' (1,b)}{(0,a + ab)}
	\DefineNamedFunc{doubleMult}{ (1,a),(0,b) }{(1,a) \cdot' (0,b)}{(0,b + ab)}
	\DefineNamedFunc{doubleMult}{ (1,a),(1,b) }{(1,a) \cdot' (1,b)}{(1,a + b + ab)}
	\\
	\DeclareFunc{complementationEmbedding}{\prod A : \BR \. A \to (A \sqcup A)}
	\DefineNamedFunc{complementationEmbedding}{a}{a^\c}{(1,a)}
	\\
	\DeclareFunc{doubleExtenstion}{\BR \to \BA}
	\DefineNamedFunc{doublextension}{A}{A'}{ \Big(A \sqcup A,+',\cot\Big) }
	\Say{[1]}{\Elim A'}{(A',+) \cong_\GRP \Bool \oplus A}
	\AssumeIn{ (x,a),(y,b),(z,c)}{A'}
	\Conclude{\Big[\big((x,a),(y,b),(z,c)\big).*\Big]}
	{
		\Elim^2 (\cdot') \Elim \RNG(\Bool \And A) \Intro (\cdot')
	}
	{
		\NewLine :
		\Big((x,a)(y,b) \Big) (z,c) =
		(xy,xb + ya + ab)(z,c)  =
		( xyz, xyc + xzb  +xbc + yac + yza +  zab + abc  ) = \NewLine
		(x,a) (yz, yc + zb + bc) = (x,a) \Big( (y,b),(z,c)\Big)
	}
	\Derive{[2]}{\Intro \forall}{\forall a,b,c \in A' \. a(bc) =(ab)c}
	\AssumeIn{ (x,a),(y,b),(z,c)}{A'}
	\Conclude{\Big[\big((x,a),(y,b),(z,c)\big).*\Big]}
	{
		\Elim (+')
		\Elim (\cdot')
		\Elim \RNG(\Bool \And A)
		\Intro (+')
		\Intro (\cdot')
	}
	{
		\NewLine : 
		(x,a)\Big( (y,b) + (z,c) \Big) = 
		(x,a)(y+z,b + c) =
		\Big( x(y + z),  a(b + c) +  x(b + c) + (y + z)a \Big) = \NewLine = 
		( xy + xz,   ab + ac + xb + xc + ya + za ) = 
		(xy, ab + xb + ya) + (xz,ac + xc + za ) = \NewLine =  
		(x,a)(y,b) + (x,a)(y,b)
	}
	\Derive{[3]}{\Intro \forall}{\forall a,b,c \in A' \. a(b + c) = ab + ac}
	\Say{[4]}{ \Intro \RNG[2][3] }{A' \in \RNG}
	\Assume{(x,a)}{A'}
	\Conclude{\Big[(x,a).*\Big]}{\Elim (\cdot)\Elim \BR(A \And \Bool)}
	{
		(x,a)^2  = (x^2, ax + ax + a^2) = (x,a)
	}
	\Derive{[5]}{\Intro \BR}{\BR(A')}
	\Assume{(x,a)}{A'}
	\Conclude{\Big[(x,a).*\Big]}{\Elim (\cdot') \THM{ZeroMult}(A)}
	{
		(x,a)(1,0) = (x, a + x0 + 0a  ) = (x,a)
	}
	\DeriveConclude{[*]}{\Intro \BA}{\BA(A)}
	\EndProof
}
\Page{
	\Theorem{IdealInExtension}{\forall A : \BR \. \Ideal(A',A)}
	\NoProof
	\\
	\Theorem{StoneSpaceOfExtensionIsOnePointCompactification}
	{
		\forall A : \BR \. 
		Z_{A'} \cong_\TOP Z_A^*
	}
	\AssumeIn{f}{Z_{A} \sqcup \{0\}}
	\Say{\varphi(f)}{\Lambda (x,a) \in A' \. x + f(a)}{ A' \to \Bool}
	\Assume{(x,a),(y,b)}{A'}
	\Conclude{\Big[\big((x,a),(y,b)\big).*\Big]}
	{
		\Elim (+')
		\Elim \varphi(f) 
		\Elim \GRP(A',\Bool)(f)
		\Intro \varphi(f)
	}
	{
		\NewLine : 
		\varphi(f)\Big( (x,a) + (y,b) \Big) = 
		\varphi(f)( x + y,a + b  )
		x + y + f(a + b) = 
		x + y + f(a) + f(b) = \NewLine = 
		\varphi(f)(x,a) + \varphi(f)(y,b)
	}
	\Derive{[1]}{\Intro \GRP}{\varphi(f) \in \GRP(A',\Bool)}
	\Assume{(x,a),(y,b)}{A'}
	\Conclude{\Big[\big((x,a),(y,b)\big).*\Big]}
	{
		\Elim (\cdot')
		\Elim \varphi(f) 
		\Elim \RNG(A,\Bool,f)
		\Intro (\cdot')
	}
	{
		\NewLine :
		\varphi(f)\Big( (x,a)(y,b) \Big) =
		\varphi(f)( xy,ya + xb + ab )  =
		xy +   f(ya + xb + ab) = 
		xy + yf(a) + xf(b) + f(a)f(b) = \NewLine = 
		\Big(x + f(a)\Big)\Big(y + f(b)\Big) =
		\varphi(f)(x,a)\varphi(f)(y,b)
	}
	\DeriveConclude{[f.*]}{\Intro \RNG}{\varphi(f) \in  \RNG(A',\Bool)}
	\Derive{\varphi}{\Intro(\to)}{Z_A \sqcup \{0\} \to Z_{A'}}
	\AssumeIn{f}{Z_{A'}}
	\SayIn{g}{f_{|A}}{Z_A \sqcup \{0\}}
	\AssumeIn{(x,a)}{A'}
	\Conclude{\Big[(x,a).*\Big]}
	{
		\Elim \varphi
		\Elim g
		\Elim Z_{A'}(f)
		\Elim \RNG(A',\Bool,f)
	}
	{
		\varphi(g)(x,a) =
		x + g(a)  =
		f(x,0) + f(0,a) = 
		f(x,a)
	}
	\DeriveConclude{[f.*]}
	{
		\Intro(\to,=)
	}
	{
		f = \varphi(g)
	}
	\Derive{[1]}{\Intro \TYPE{Surjective}}
	{
		\TYPE{Surjective}\Big( Z_A \sqcup \{0\}, Z_{A'} ,\varphi\Big)
	}
	\Say{[2]}{[1]\THM{InjectiveCardinlaity}[1]}{ \Big| Z_{A'}\setminus \varphi(Z_A) \Big| = 1}
	\AssumeIn{(x,a)}{A'}
	\Say{[3]}{\Elim S \Elim \varphi }
	{
		\varphi^{-1}\Big(S_{A'}(x,a)\Big) = 
		\If x \Then S_A^\C(a) \Else S_A(a)
	}
	\Conclude{\Big[(x,a).*\Big]}{ \THM{StoneRepresentationIsClopen}[3]   }
	{
		\varphi^{-1}\Big( S_{A'}(x,a)\Big) \in \T(Z_A)
	}
	\Derive{[3]}{\Intro \TOP}{ \varphi \in \TOP(Z_A,Z_{A'})}
	\AssumeIn{a}{A}
	\Say{[4]}{\Elim S \Elim \varphi }
	{
		\varphi\Big(S_{A}(a)\Big) = 
		\varphi\Big(S_{A'}(0,a)\Big)
	}
	\Conclude{\Big[(x,a).*\Big]}{ \THM{StoneRepresentationIsClopen}[4]   }
	{
		\varphi\Big( S_{A}(a)\Big) \in \T(Z_{A'})
	}
	\Derive{[4]}{\Intro \TYPE{Homeo}}{ \varphi : \TYPE{Homeo}\Big(Z_A,\varphi(Z_{A})\Big)}
	\Conclude{[*]}{[4]\Elim \T_{A'}}{Z_{A'} \cong Z_A^*}
	\EndProof
}
\newpage
\subsection{Exploiting the ring structure}
\subsubsection{Subalgebras}
\Page{
	\DeclareFunc{categoryOfBooleanRings}{\CAT}
	\DefineNamedFunc{categoryOfBooleanRings}{}{\BOL}{\Big( \BR, \RNG  ,\circ,\id\Big)}
	\\
	\DeclareFunc{categoryOfBooleanRings}{\CAT}
	\DefineNamedFunc{categoryOfBooleanRings}{}{\BOOL}{\Big( \BA, \RING  ,\circ,\id\Big)}
	\\
	\DeclareFunc{complement}{\prod A : \BA \. A \to A}
	\DefineNamedFunc{complement}{a}{a^\c}{a + e}
	\\
	\Theorem{LawOfExcludedMiddle}{\forall A : \BA \. \forall a \in A \. a \cap a^\c = 0}
	\Conclude{[*]}{\Elim(\cap)\Elim \c \Elim \RING(A) \Elim \BA(A) \THM{BooleanRingHasChar2}}
	{
		\NewLine :
		a \cap a^\c = 
		aa^\c =
		a(a + e) =
		a^2 + a =
		a + a = 
		0
	}
	\EndProof
	\\
	\Theorem{BooleanSubalgebraCriterion1}
	{
		\forall A \in \Bool \. 
		\forall B \subset A \.
		B \subset_\Bool A \iff
		\NewLine \iff
		0 \in B \And \forall a,b \in B \. a \cup b \in B \And \forall a \in B \. a^\c \in B
	}
	\Assume{[1.1]}{0 \in B}
	\Assume{[1.2]}{\forall a,b \in B \. a \cup B \in B}
	\Assume{[1.3]}{\forall a \in B \. a^\c \in B}
	\Say{[2]}{[1.1][1.3]}{e \in B}
	\Say{[3]}{[1.3][1.2][1.3]}{\forall a,b \in B \. ab = (a^\c \cup b^\c )^\c \in B}
	\Say{[4]}{[1.3][3][1.2]}{\forall a,b \in B \. a + b = (a \cap b^\c) \cup (a^\c \cap b) \in B }
	\Conclude{[*]}{\Intro \BOOL [2][3][4]}{B \subset_\BOOL A}
	\EndProof
	\\
	\Theorem{BooleanSubalgebraCriterion2}
	{
		\forall A \in \Bool \. 
		\forall B \subset A \.
		B \subset_\Bool A \iff
		\NewLine \iff
		B \neq \emptyset \And \forall a,b \in B \. a \cap b \in B \And \forall a \in B \. a^\c \in B
	}
	\Assume{[1.1]}{B \neq \emptyset}
	\Assume{[1.2]}{\forall a,b \in B \. a \cap B \in B}
	\Assume{[1.3]}{\forall a \in B \. a^\c \in B}
	\SayIn{ a }{\Elim \TYPE{NonEmpty}(B)}{B}
	\Say{[2]}{\THM{LawOfExluededMiddle}(A,a)[1.3][1.2]}{ 0 = a \cap a^\c \in B }
	\Say{[3]}{[1.3][1.2][1.3]}{\forall a,b \in B \. a \cup b = (a^\c \cap b^\c )^\c \in B}
	\Conclude{[*]}{\THM{BooleanSubalgebraCriterion1}[2][3][1.3]}{B \subset_\Bool A}
	\EndProof
}
\newpage
\Page{
	\Theorem{SubalgebraGenratedByAdditionalElement}
	{
		\forall A \in \BOOL \.
		\forall B \subset_\BOOL A \.
		\forall a \in A \.
		\{ (b \cap a) \cup (c \setminus a) | b,c \in B  \} \subset_\BOOL A
	}
	\Say{C}{\{ (b \cap a) \cup (c \setminus a) | b,c \in B  \} }{?A}
	\Say{[1]}{\Elim(0)\Elim(\cup)}
	{  
		(0 \cap a) \cup (0 \setminus a) = 0 \cup 0 = 0
	}
	\Say{[2]}{\Elim \TYPE{Subring}(A,B)\Elim C [1]}{ 0 \in C}
	\AssumeIn{d}{C}
	\Say{\Big(b,c,[3]\Big)}{\Elim C(d)}{\sum b,c \in B \. d = (b \cap a) \cup (c \setminus a)}
	\Conclude{[d.*]}{[3]\LOGIC{CheckingTruthTables}\Elim C}
	{
		\NewLine :
		d^\c  =   \Big( (b \cap a) \cup (c \setminus a) \Big)^\c  =
		(b^\c \cup a^\c) \cap  (c^\c \cup a) =
		(b^\c \cap a) \cup (c^\c \setminus a) \in C
	}
	\Derive{[3]}{\Intro \forall}{\forall d \in C \. d^\C \in C }
	\Assume{d,d'}{C}
	\Say{\Big(b,c,[4]\Big)}{\Elim C(d)}{\sum b,c \in B \. d = (b \cap a) \cup (c \setminus a)}
	\Say{\Big(b',c',[5]\Big)}{\Elim C(d')}{\sum b',c' \in B \. d' = (b' \cap a) \cup (c' \setminus a)}
	\Conclude{\Big[(d,d').*\Big]}{[4][5]\LOGIC{CheckingTruthTables}\Elim C}
	{
		\NewLine :
		d \cup d'  = 
		(b \cap a) \cup (c \setminus a) 
		\cup
		(b' \cap a) \cup (c' \setminus a) =  
		\Big( (b \cup b') \cap a \Big) \cup \Big( (b \cup b') \setminus a \Big)  
		\in C
	}
	\Derive{[4]}{\Intro \forall}{ \forall d,d' \in C \. d \cup d' \in C}
	\Conclude{[*]}{\THM{BooleanSubalgebraCriterion1}[4]}
	{
		C \subset_\BOOL A
	}
	\EndProof
	\\
	\DeclareFunc{oneElementSubalgebraExtension}
	{
		\prod_{A \in \BOOL} \TYPE{Subalgebra}(A) \to A \to \TYPE{Subalgebra}(A)
	}
	\DefineNamedFunc{oneElementSubalgebraExtension}{B,c}{B_c}
	{
		\{ (b \cap a) \cup (x \setminus a) | b,c \in B \} 
	}
	\\
	\Theorem{OneElementExtensionProperty}
	{
		\forall A \in \BOOL \.
		\forall B \subset_\BOOL A \.
		\forall a \in A \. 
		B \subset_\BOOL B_a \And a \in B_a
	}
	\NoProof
}
\subsubsection{Ideals}
\Page{
	\Theorem{IdealCriterion}
	{
		\forall A \in \BOOL \.
		\forall I \subset A \.
		\Ideal(A,I) \iff \NewLine \iff
		0 \in I \And 
		\forall a,b \in I \. a \cup b \in I \And
		\forall a \in I \. \forall b \in A \. b \le a \Imply b \in I
	}
	\Assume{[1]}{\Ideal(A,I)}
	\Say{[*.1]}{ \Elim \Ideal(A,I }
	{
		0 \in I
	}
	\AssumeIn{a,b}{I}
	\Say{[2]}{\Elim \Ideal(A,I)(a,b)}{ ab \in I  }
	\Conclude{\Big[(a,b).*\Big]}{ \Elim (a \cup b) \Elim \TYPE{Subgroup}(A,I)[2]}
	{
			a \cup b = ab +a + b \in I
	}
	\Derive{[*.2]}{\Intro \forall }{\forall a,b \in I \. a \cup b \in I}
	\AssumeIn{a}{I}
	\AssumeIn{b}{A}
	\Assume{[2]}{b \le a}
	\Say{[3]}{\Elim \TYPE{BooleanOrder}}{b = ab}
	\Conclude{[a.*]}{\Elim \Ideal(A,I)[3]}{b \in I}
	\DeriveConclude{[*.3]}{\Intro \Imply \Intro^2 \forall}
	{
		\forall a \in I \. \forall b \in A \. b \le a \Imply b \in I	
	}
	\Derive{[1]}{ \Intro \Imply   }
	{
		\Ideal(A,I) \Imply
		0 \in I \And 
		\forall a,b \in I \. a \cup b \in I \And
		\forall a \in I \. \forall b \in A \. b \le a \Imply b \in I
	}
	\Assume{[2.1]}{0 \in I}
	\Assume{[2.2]}{\forall a,b \in I \. a \cup b \in I}
	\Assume{[2.3]}{\forall a \in I \. \forall b \in A \. b \le a \Imply b \in I  }
	\AssumeIn{a}{A}
	\AssumeIn{i}{I}
	\Say{[3]}{\Elim \BA(A)}{ai^2 = ai}
	\Say{[4]}{\Elim \TYPE{BooleanOrder}(A)[3]}{ai \le i}
	\Conclude{[a.*]}{[2.3][4]}{ai \in I}
	\Derive{[3]}{\Intro^2 \forall}{\forall a \in A \. \forall i \in I \. ai \in I}
	\AssumeIn{a,b}{A}
	\Say{[4]}{[3](b^\c,a)}{ab^\c \in I}
	\Say{[5]}{[3](a^\c,a)}{a^\c b \in I}
	\Conclude{\Big[(a,b).*\Big]}{\Elim \oplus [2.2]}
	{
		a + b = 
		ab^\c \cup a^\c b \in I
	}
	\Derive{[4]}{\Intro \forall}{\forall a,b \in I \. a + b \in I}
	\Conclude{[2.*]}{\Intro \Ideal(A)[2.1][3][4]}{\Ideal(A,I)}
	\Derive{[2]}{\Intro \Imply}
	{
		0 \in I \And 
		\forall a,b \in I \. a \cup b \in I \And
		\forall a \in I \. \forall b \in A \. b \le a \Imply b \in I	
		\Imply
		\Ideal(A,I)
	}
	\Conclude{[*]}{\Intro \iff[1][2]}
	{
		\Ideal(A,I) \iff
		0 \in I \And 
		\forall a,b \in I \. a \cup b \in I \And
		\forall a \in I \. \forall b \in A \. b \le a \Imply b \in I
	}
	\EndProof
}
\Page{
	\Theorem{PrincipleIdealStructure}
	{
		\forall A \in \BOOL \. 
		\forall a \in A \.
		\langle a \rangle = 
		\{
			b \in A : b \le a
		\}
	}
	\AssumeIn{b}{\langle a \rangle}
	\Say{\Big(c,[1]\Big)}{\Elim \langle a \rangle(b)}
	{
		b = ca
	}
	\Say{[2]}{[1]\Elim \BA(A)[1]}{ab = ca^2 = ca = b}
	\Conclude{[3]}{\Elim \TYPE{BooleanOrder}[2]}{ b \le a  }
	\Derive{[1]}{\Intro \subset}{ \langle a \rangle \subset \{ b \in A : b \le a \}  }
	\AssumeIn{b}{A}
	\Assume{[2]}{b \le a}
	\Say{[3]}{\Elim (\le) [2]}{ba = b}
	\Conclude{[b.*]}{\Elim \langle a \rangle [3]}{ b \in \langle a \rangle}
	\Derive{[2]}{\Intro \subset}{\{b \in A : b \le a \} \subset \langle a \rangle }
	\Conclude{[*]}{\Intro \TYPE{SetEq}[1][2]}
	{
		\{b \in A : b \le a \} = \langle a \rangle
	}
	\EndProof
	\\
	\Theorem{PrincipleIdealIsAlgebra}
	{
		\forall A \in \BOOL \.
		\forall a \in A \.
		\langle a \rangle \in \BOOL
	}
	\NoProof
}
\newpage
\subsubsection{Morphisms}
\Page{
	\Theorem{MorphismPreservesOrer}
	{
		\forall A,B \in \BOOL \.
		\forall A \Arrow{f} B : \BOOL \.
		\forall x,y \in A \.
		x \le y \Imply f(x) \le f(y)
	}
	\Say{[1]}{\Elim (\le)[0]}{xy = x}
	\Say{[2]}{\Elim \BOOL(A,B,f)[1]}{f(x)f(y) = f(xy) = f(x)}
	\Conclude{[*]}{\Intro (\le)[2]}{f(x) \le f(y)}
	\EndProof
	\\
	\Theorem{IntersectionIsSurjectiveHomo}
	{
		\forall A \in \BOOL \.
		\forall a \in A \.
		\lambda_a : \BOOL \And \TYPE{Surjective}\Big(A,\langle a \rangle \Big)
	}
	\NoProof
	\\
	\DeclareFunc{fixedPointAlgebra}{\prod_{A \in \BOOL} \End_{\BOOL}(A) \to \BOOL}
	\DefineNamedFunc{fixedPointAlgebra}{f}{\Fix(f)}{\{ a \in A : f(a) = a  \}}
	\\
	\Theorem{BooleanPosetIsomorphismIsBooleanIsomorphism}
	{
		\forall A,B \in \Bool \.
		\forall A \ToIso{f} B : \POSET \.
		A \ToIso{f} B : \BOOL
	}
	\Say{[1]}{\THM{PosetIsomorphismPreservesMin}(A)}{f(0) = 0}
	\Say{[2]}{\THM{PosetIsomorphismPreservesMax}(A)}{f(e) = e}
	\Say{[3]}{\THM{PosetIsomorphismPresevesLatticeStructure}(A)}
	{
		\NewLine :
		\forall a,b \in A \.
		f(a \cap b) = f(a) \cap f(b) 
		\And
		f(a \cup b) = f(a) \cup f(b)
	}
	\Say{[4]}{[3.2] \THM{LawOfRxcludedMiddle}(A)  [2]}
	{
		\forall a \in A \. 
		f(a) \cup f(a^\c) =
		f(a \cup a^\c) =
		f(e) = 
		e
	}
	\Say{[5]}{[3.1] \THM{LawOfRxcludedMiddle}(A)  [1]}
	{
		\forall a \in A \. 
		f(a) \cap f(a^\c) =
		f(a \cap a^\c) =
		f(0) = 
		0
	}
	\Say{[6]}{\THM{UniqueComplementatioTheorem}[4][5]}
	{
		\forall a \in A \.
		f(a^\c) = f^\c(a)
	}
	\Conclude{[*]}{\Intro(\oplus)[6][3.2][3.1]}
	{
		\TYPE{Isomorphism}(\BOOL,A,B,f)
	}
	\EndProof
}\Page{
	\Theorem{HomomorphismExtension}
	{
		\forall A,B \in \BOOL \.
		\forall A' \subset_{\BOOL} A \.
		\forall A' \Arrow{f} B : \BOOL \.
		\forall c \in A \.
		\forall v \in B \.
		\NewLine :
		\forall [0] : \forall a,b \in A' \. a \le c \le b \iff f(a) \le v \le f(b) \.
		\exists A'_c \Arrow{f'} B : \BOOL :
		f'(c) = v \And f'_{|A'}  = f
	}
	\AssumeIn{d}{A'_c}
	\Say{\Big(a,b,[1] \Big)}{\Elim A'_c(d)}
	{
		\sum a,b \in A' : d = (a\cap c) \cup (b\setminus c)
	}
	\AssumeIn{a',b'}{A'}
	\Assume{[2]}{d = (a' \cap c) \cup (b' \setminus c)}
	\Say{[3]}{ [1][2] \cap c }{ a \cap c = d \cap c =  a' \cap c }
	\Say{[4]}{\Intro (\du) [3]}{ (a \du a') \cap c = 0}
	\Say{[5]}{\Intro (\setminus) a}{c \subset (a \du a')^\c }
	\Say{[6]}{[0][5]}{v \subset (f(a) \du f(a'))^\c}
	\Say{[7]}{\Intro (\cap)[6]}{ f(a) \cap v = f(a') \cap v}
	\Say{[8]}{[1][2] \setminus c }{b\setminus c = d\setminus c = b'\setminus c}
	\Say{[9]}{\Intro \du}{(b \du b')\setminus c = 0}
	\Say{[10]}{\Elim (\setminus)[9]}{ b \du b' \subset c}
	\Say{[11]}{[0][10]}{f(b) \du f(b') \subset v}
	\Say{[12]}{\Intro (\setminus)[11]}{\Big(f(b) \du f(b')\Big) \setminus v = 0}
	\Say{[13]}{\Elim \du [12]}{f(b) \setminus v = f(b') \setminus v}
	\Conclude{\Big[(a',b').*\Big]}{ [7]\cup[13]  }
	{
		(f(a) \cap v) \cup (f(b) \setminus v) 
		=
		(f(a') \cap v) \cup (f(b') \setminus v) 
	}
	\Derive{[2]}{\Intro \forall}
	{
		\forall a',b' \in A' \. 
		d = (a' \cap c) \cup (b' \setminus c) \Imply
		(f(a) \cap v) \cup (f(b') \setminus v)
	}
	\Conclude{f'(d)}{(f(a) \cap v) \cup (f(b) \setminus v)}{B}
	\Derive{f'}{\Intro \to }{A'_c \to B}
	\Say{[*.1]}{
		\Elim A'_c
		\Elim f'
		\Elim \Bool(A',B,f)
		\Elim \RING(B)
		\Elim (\cup)
	}
	{
		\NewLine :
		f'(c) = 
		f'\Big( (e \cap c) \cup (0 \setminus c)  \Big) =
		f(e) \cap v \cup (f(0) \setminus v) = 
		e \cap v \cup (0 \setminus v) =
		v \cup 0 = 
		v
	}
	\AssumeIn{a}{A'}
	\Conclude{[a.*]}{  
		\THM{IntersectDifferenceDecomposition}(a,c)
		\Elim f'
		\THM{IntersectDifferenceDecomposition}\Big(f(a),v\Big)
	}
	{
		f'(a) = 
		f'\Big( (a \cap c) \cup (a \setminus c)  \Big) =
		(f(a) \cap v) \cup (f(a) \setminus v)  =
		f(a)
	}
	\Derive{[*.2]}{\Intro \forall}{\forall a \in A' \. f'(a) = f(a) }
	\AssumeIn{d}{A'_c}
	\Say{\Big(a,b,[1] \Big)}{\Elim A'_c(d)}
	{
		\sum a,b \in A' : d = (a\cap c) \cup (b\setminus c)
	}
	\Conclude{[d.*]}{
		[1]
		\LOGIC{CheckingTruthTable}
		\Elim f'
		\Elim \BOOL(A,B,f) 
		\LOGIC{CheckingTruthTable}
		\Intro f'
	}
	{
		\NewLine :
		f'(d^\c) = 
		f'\Big( (a \cap c) \cup (b\setminus c) \Big)^\c = 
		f'\Big( (a^\c \cap c) \cup (b^\c \setminus c) =
		(f(a^\c) \cap v) \cup  (f(b^\c) \setminus v) = \NewLine = 
		(f^\c(a) \cap v) \cup (f^\c(b) \setminus v) = 
		\Big( \big(f(a) \cap v\big) \cup \big( f(b) \setminus v\big) \Big)^\c =
		\Big( f'(d) \Big)^\c
	}
	\Derive{[1]}{\Intro \forall}
	{
		\forall d \in A'_c \. f'(d^\c) = \Big( f'(d) \Big)^\c
	}
}\Page{
	\AssumeIn{d,d'}{A'_c}
	\Say{\Big(a,b,[2] \Big)}{\Elim A'_c(d)}
	{
		\sum a,b \in A' : d = (a\cap c) \cup (b\setminus c)
	}
	\Say{\Big(a',b',[3] \Big)}{\Elim A'_c(d)}
	{
		\sum a',b' \in A' : d' = (a'\cap c) \cup (b'\setminus c)
	}
	\Conclude{[4.*]}{
		[2][3]
		\THM{CheckingTruthTables}
		\Elim f'
		\Elim \BOOL(A',B,f)
		\THM{CheckingTruthTables}
		\Intro f'
	}
	{
		\NewLine : 
		f'(d \cup d') = 
		f'\Big( (a \cap c) \cup (b \setminus c) \cup (a' \cap c) \cup (b' \setminus c) \Big) = 
		f'\Big( \big( (a \cup a') \cap c \big) \cup \big ( (b \cup b') \setminus c \big) \Big) = \NewLine = 
		\Big(f(a \cup a') \cap v\Big) \cup \Big( f(b \cup b') \setminus v \Big) =  
		f(a) \cap v \cup f(b) \setminus v  \cup f(a') \cap v \cup f(b') \setminus v = 
		f'(d) \cup f'(d')
	}
	\Derive{[2]}{\Intro \forall}
	{
		\forall d,d' \in A'_c \. 
		f'(d \cup d') = f'(d) \cup f'(d')
	}
	\Conclude{[*.3]}{\Intro \BOOL [1][2]}{f' \in \BOOL(A'_c,B)}
	\EndProof
}
\newpage
\subsubsection{Quotitent Algebras} 
\Page{
	\Theorem{BooleanQuotientAlgebra}
	{
		\forall A \in \BOOL  \.
		\forall I : \Ideal(A) \.
		\frac{A}{I} \in \BOOL
	}
	\AssumeIn{[a]}{\frac{A}{I}}
	\Conclude{\Big[[a].*\Big]}{\Elim \frac{A}{I} \Elim \BA(A)}
	{
		[a]^2 = [a^2] = [a]
	}
	\Derive{[*]}{\Intro \BA}{\BA\left( \frac{A}{I} \right)}
	\EndProof
	\\
	\Theorem{QuotientOrder}
	{
		\forall A \in \BOOL \.
		\forall I : \Ideal(A) \.
		\forall [a],[b] \in \frac{A}{I} \. 
		[a] \le [b]
		\Imply
		a \setminus b \in I
	}
	\Say{[1]}{\Elim (\le)[0]}
	{
		[a][b] = [a]
	}
	\Say{[2]}{\Elim(\setminus)[1]\THM{BooleanRingHasChar2}\left( \right)}
	{
		[a]\setminus [b] = 
		[a] + [a][b] =
		[a] + [a] = 0
	}
	\Conclude{[*]}{\Elim \BOOL\left(A,\frac{A}{I},[\cdot]\right)}
	{
		a \setminus b \in I
	}
	\EndProof
}
\newpage
\subsubsection{Stone Functor}
\Page{
	\DeclareFunc{setOfIdeals}{\Contra(\RING,\POSET)}
	\DefineNamedFunc{setOfIdeals}{R}{\I(R)}{\Ideal(R)}
	\DefineNamedFunc{setOfIdeals}{R,S,f}{\I_{R,S}(f)}{\Lambda I \in \I(S) \. f^{-1}(S)}
	\\
	\Theorem{StoneTopologyRingIdealsCorrespondance}
	{
		\forall B \in \BOOL \.
		\T(Z_B) \cong_\POSET \I(B)
	}
	\AssumeIn{U}{\T(Z_B)}
	\Say{F(U)}{\{ b \in B : S_B(b) \subset U   \}}{?B}
	\AssumeIn{b}{F(U)}
	\Say{[1]}{\Elim F(U)(b)}{S_B(b) \subset U}
	\AssumeIn{a}{B}
	\Say{[2]}{\Elim \FUNC{BooleanOrder}}{ab \le b}
	\Say{[3]}{\THM{BooleanOrderBy}[2][1] }{ S_B(ab) \subset S_B(b) \subset U }
	\Conclude{[b.*]}{\Elim F(U)[3]}{ab \in F(U)}
	\DeriveConclude{[U.*]}{\Intro \I}{F(U) \in \I(B)}
	\Derive{F}{\Intro(\to)}{\T(Z_B) \to \I(B)}
	\SayIn{G}{\Lambda I \in  \I(B) \. \bigcup_{b \in \I} S_B(b)}{\Poset\Big(\I(B),  \T(Z_B)\Big)} 
	\Say{[1]}{\Elim F \Elim G}{FG = \id \And GF = \id}
	\Conclude{[*]}{\Intro \cong [1]}{\T(Z_B) \cong_\POSET \I(B)}
	\EndProof
	\\
	\Theorem{StoneHomoAndCCorespondance}
	{
		\forall A,B \in \BOOL \.
		\exists \BOOL(A,B) \ToIso{\gamma} \TOP(Z_B,Z_A)  \.
		\NewLine \.
		\forall \varphi \in \BOOL(A,B) \.
		\forall a \in A \. 
		S_B\Big( \varphi(a) \Big) = \Big(\gamma(\varphi)\Big)^{-1}\Big( S_A(a) \Big)
	}
	\AssumeIn{f}{\TOP(Z_B,Z_A)}
	\AssumeIn{a}{A}
	\Say{[1]}{\THM{StoneRepresentationsAreClopen}(A,a)}
	{\Clopen(Z_A,S_A(a))}
	\Say{[2]}{\THM{ClopenCPreimage}[1]}
	{
		\Clopen\bigg(Z_B, f^{-1}\Big(S_A(a)\Big) \bigg)
	}
	\Say{[3]}{\THM{ClosedSubsetOfCompactIsCompact}[2]}
	{
		\Compacts\Big( Z_B, f^{-1}\Big(S_A(a)\Big) \bigg)
	}
	\Say{\Big(b,[4] \Big)}{\THM{CompactOpenIsStoneRepresentation}[2][3]}
	{
		\sum_{b \in B} f^{-1}\Big( S_A(a)\Big) = S_B(b)
	}
	\Conclude{\delta(f)(a)}{b}{B}
	\Derive{\delta(f)(a)}{\Intro(\to)}{A \to B}
	\Conclude{[f.*]}{\Elim S_B \Elim \delta(f)}{\delta(f) \in \BOOL(A,B)}
	\Derive{\delta}{\Intro(\to)}{\TOP(Z_A,Z_B) \to \BOOL(A,B)}
}\Page{
	\AssumeIn{\varphi}{\BOOL(A,B)}
	\AssumeIn{f}{Z_B}
	\Conclude{\gamma(\varphi)(f)}{\varphi f}{Z_A}
	\Derive{\gamma(\varphi)}{\Intro(\to)}{Z_B \to Z_A}
	\AssumeIn{a}{A}
	\AssumeIn{f}{ \Big(\gamma(\varphi)\Big)^{-1} \Big( S_A(a) \Big)  }
	\Say{[1]}{\Elim \FUNC{preimage}}{\gamma(\varphi)(f) \in S_A(a)}
	\Say{[2]}{\Intro \gamma\Big( \varphi(a) \Big)[1] \Elim S_A(a)}
	{
		f\Big( \varphi(a) \Big) =
		\bigg(\Big(\gamma(\varphi)\Big)(f)\bigg)(a) =
		1
	}
	\Conclude{[f.*]}{\Elim S_{B}\Big( \varphi(a) \Big)[2]}
	{
		f \in S_B\Big( \varphi(a) \Big) 
	}
	\Derive{[1]}{\Intro \subset}{
		\Big( \gamma(\varphi)\Big)^{-1}\Big(S_A(a)\Big)
		\subset
		S_B\Big(\varphi(a)\Big)
	}
	\AssumeIn{f}{S_B(\varphi(a))}
	\Say{[2]}{\Elim \gamma\Big( \varphi(a) \Big)[1] \Elim S_B\Big(\varphi(a)\Big)}
	{	
		\bigg(\Big(\gamma(\varphi)\Big)(f)\bigg)(a) =
		f\Big( \varphi(a) \Big) =
		1
	}
	\Conclude{[*]}{\Elim Z_A(a)}
	{
		f \in \Big(\gamma(\varphi) \Big)^{-1}\Big(S_A(a)\Big)
	}
	\Derive{[2]}{\Intro \subset}
	{
		S_B\Big(\varphi(a)\Big) \subset
		\Big( \gamma(\varphi) \Big)^{-1}\Big( S_A(a) \Big)
	}
	\Conclude{[a.*]}{\Intro (=)[1][2]}
	{
		\Big( \gamma(\varphi)\Big)^{-1}\Big(S_A(a)\Big)
		=
		S_B\Big(\varphi(a)\Big)	
	}
	\DeriveConclude{[\varphi.*]}{\Elim \T_A}
	{
		\gamma(\varphi) \in \TOP(Z_B,Z_A)
	}
	\Derive{\gamma}{\Intro (\to)}
	{
		\BOOL(A,B) \to \TOP(Z_B,Z_A)
	}
	\AssumeIn{\varphi}{\TOP(Z_B,Z_A)}
	\AssumeIn{f}{Z_B}
	\Conclude{[f.*]}{\Elim \delta \Elim \gamma \Elim S \Elim \Lambda}
	{
		\NewLine :
		\delta \gamma(\varphi)(f) =
		\delta\Big( \Lambda a \in A \.  \varphi^{-1}S_B^{-1} \big(S_A(a)\big)  \Big)(f) =
		\Lambda a \in A \. f\Big( \varphi^{-1}S_B^{-1} \big( S_A(a) \big) \Big) =
		\Lambda a \in A \. \varphi(f)(a) =
		\varphi(f)
	}
	\Derive{[1]}{\Intro(=,\to)}{\delta \gamma = \id}
	\AssumeIn{\varphi}{\BOOL(A,B)}
	\AssumeIn{a}{A}
	\Conclude{[a.*]}{\Elim \gamma \Elim \delta \Elim S}
	{
		\NewLine :
		\gamma \delta(\varphi)(a) = 
		S_B^{-1} \Big( \delta(\varphi) \Big)^{-1}  \big(S_A(a)\big) =
		\varphi(a)
	}
	\Derive{[2]}{\Intro(=,\to)}{\gamma \delta = \id}
	\Conclude{{*}}{[1][2]}{
		\BOOL(A,B) \ToIso{\gamma} \TOP(Z_B,Z_A) 
	}
	\EndProof
	\\
	\DeclareFunc{functorOfStone}{\Contra(\Bool,\HC)}
	\DefineNamedFunc{functorOfStone}{B}{\Z(B)}{A}
	\DefineNamedFunc{functorOfStone}{f}{\Z_{A,B}(f)}{\THM{StoneHomoAndCCorrespondance}}
}
\Page{
	\Theorem{StoneFunctorMirrorsInjection}
	{
		\forall A,B \in \BOOL \.
		\forall A \Arrow{f} B : \BOOL \.
		\NewLine 
		\TYPE{Injective}(A,B,f) 
		\iff
		\TYPE{Surjective}\Big( \Z(A),\Z(B),\Z_{A,B}(f) \Big)
	}
	\NoProof
	\\
	\Theorem{StoneFunctorMirrorsSurjection}
	{
		\forall A,B \in \BOOL \.
		\forall A \Arrow{f} B : \BOOL \.
		\NewLine
		\TYPE{Surjective}(A,B,f) 
		\iff
		\TYPE{Injective}\Big( \Z(A),\Z(B),\Z_{A,B}(f) \Big)
	}
	\NoProof
	\\
	\DeclareFunc{principalIdealProjection}
	{
		\prod_{A \in \BOOL}
		\prod_{a \in A}
		\BOOL\Big(A,\langle a \rangle \Big)
	}
	\DefineNamedFunc{principalIdealProjection}{b}{\pi_a(b)}{ab}
	\\
	\Theorem{StonePrincipalIdealEmbedding}
	{
		\forall A \in \BOOL \.
		\forall a \in A \.
		\TYPE{TopologicalEmbedding}\Big( \Z\langle a \rangle, \Z(A), \Z_{A,\langle a \rangle}(\pi_a)  \Big)
	}
	\NoProof
	\\
	\Theorem{StonePrincipalIdealEmbeddingIsStoneRepresentation}
	{
		\NewLine
		:: 
		\forall A,B \in \BOOL \.
		\forall a \in A \.
		\Z_{A,\langle a \rangle}(\pi_a)\Big(\Z\langle a \rangle\Big) = S_a(A)
	}
	\NoProof
}
\subsection{Order Continuity}
\subsubsection{Inf and Sup}
\Page{
	\Theorem{SupremumComplementation}
	{
		\forall A \in \BOOL \.
		\forall B \subset A \.
		\forall c \in A \.
		\forall b = \sup B \.
		c \setminus b = \inf(c \setminus B)
	}
	\NoProof
	\\
	\Theorem{InfimumComplementation}
	{
		\forall A \in \BOOL \.
		\forall B \subset A \.
		\forall c \in A \.
		\forall b = \inf B \.
		c \setminus b =  \sup(c \setminus B)
	}
	\NoProof
	\\
	\Theorem{SupremumMult}
	{
		\forall A \in \BOOL \.
		\forall B \subset A \.
		\forall c \in A \.
		\forall b \in \sup B \.
		bc \in  \sup Bc
	}
	\NoProof
	\\
	\Theorem{SupremumMult2}
	{
		\forall A \in \BOOL \.
		\forall B, C \subset A \.
		\forall b \in \sup B \.
		\forall c \in \sup C \.
		bc =  \sup BC
	}
	\NoProof
	\\
	\Theorem{InfinumMult}
	{
		\forall A \in \BOOL \.
		\forall B \subset A \.
		\forall c \in A \.
		\forall b \in \inf B \.
		bc =  \inf Bc
	}
	\NoProof
	\\
	\Theorem{InfimumMult2}
	{
		\forall A \in \BOOL \.
		\forall B, C \subset A \.
		\forall b \in \inf B \.
		\forall c \in \inf C \.
		bc  =  \inf BC
	}
	\NoProof
}
\Page{
	\Theorem{SupremumAsUnion}
	{
		\forall A \in \BOOL \.
		\forall B \subset A \.
		b = \sup B 
		\iff 
		S_A(b) =  \overline{\bigcup_{c \in B} S_A(c) }
	}
	\AssumeIn{b}{A}
	\Assume{[1]}{b = \sup B}
	\Say{[2]}{\Elim_1 \sup B [1]}{\forall c \in B \. b \ge c}
	\Say{[3]}{[2]\THM{BooleanOrderByStoneRepresentation}\Intro \cup}
	{
		\bigcup_{c \in B} S_A(c) \subset  S_A(b) 
	}
	\Say{[4]}{ \Elim_2 \sup B  }{ \forall a \in A \. a \ge B \Imply a \ge c  }
	\Say{[5]}{[4]\THM{BoleanOrderByStoneRepresentation} \Intro \inf }
	{
		S_A(b) = \min\left\{  U \in \TK\;\Z(A) : \bigcup_{c \in B} S_A(c)  \right\}
	}
	\Say{[6]}{\Intro \FUNC{closure}[2]\THM{StoneRepresentationIsClopen}(A,b)}
	{
		\overline{\bigcup_{c \in A} S_A(c)} \subset S_A(b)
	}
	\AssumeIn{f}{ S_A(b) \setminus \overline{\bigcup_{c \in A} S_A(c)}    }
	\Say{\Big(U,[7]\Big)}{\THM{StoneRepresentationIsOpen}(A,b)\Elim \TYPE{ZeroDimensional}\Big(A\Big)}
	{
		\NewLine :
		\sum U : \Clopen\Big( \Z(A) \Big) f \in U \And 
		U \cap \overline{\bigcup_{c \in A} S_A(c)} = \emptyset
	}
	\Say{V}{S_A(b) \setminus U}{\Clopen\Big( \Z(A) \Big)}
	\Say{[8]}{\Elim V [6][7]}{ \overline{\bigcup_{c \in A} S_A(c)} \subset V  }
	\Say{[9]}{\Elim V \Elim (\setminus) [7]}{ V \subsetneq S_A(b)}
	\Conclude{[10]}{[9][5]}{\bot}
	\DeriveConclude{[1.*]}{\Elim \bot [6]}{ S_A(b) = \overline{\bigcup_{c \in B} S_A(c)}}
	\Derive{[1]}{\Intro \Imply}{ b = \inf B \Imply S_A(b) = \overline{\bigcap_{c \in B} S_A}         }
	\Assume{[2]}{ S_A(b) = \overline{\bigcap_{b \in B} S_A(c)}   }
	\Say{[3]}{\Elim \FUNC{closure}[2]}{\bigcap_{b \in B} S_A(c) \subset S_A(b)  }
	\Say{[4]}{\THM{BooleanOrderByStoneRepresentation}[3]}{  B \le b  }
	\Say{[5]}{\Elim \FUNC{closure} \THM{BooleanOrderByStoneRepresentation}^2[3] }
	{
		\forall a \in A \. a \ge B \Imply  a \ge b
	}
	\Conclude{[2.*]}{\Intro \sup [4][5]}{ b = \sup B}
	\Derive{[2]}{\Intro \Imply}
	{
		S_A(b) = \overline{\bigcap_{c \in B} S_A(c)} 
		\Imply
		b = \sup B
	}
	\Conclude{[*]}{\Intro \iff [1][2]}
	{
		b = \sup B \iff S_A(b) = \overline{\bigcap_{c \in B} S_A(c)}
	}
	\EndProof
}\Page{
	\Theorem{InfimumAsIntersect}
	{
		\forall A \in \BOOL \.
		\forall B \subset A \.
		b = \inf B 
		\iff 
		S_A(b) =  \intx \bigcap_{c \in B} S_A(c) 
	}
	\NoProof
	\\
	\Theorem{ZeroInfimumCriterion}
	{
		\forall A \in \BOOL \.
		\forall B \subset A \.
		b = \inf B 
		\iff 
		\ND\left( \Z(A) ,\bigcap_{c \in B} S_A(c) \right)
	}
	\NoProof
	\\
}
\newpage
\subsubsection{Sigma Algebras and Ideals}
\Page{
	\DeclareType{\OC}{?\POSET}
	\DefineType{X}{\OC}{ 
		\Big(\forall D : \TYPE{Directed}(X) \. \exists \sup A \Big)
		\And
		\Big(\forall D : \TYPE{Directed}(X^{\op}) \. \exists \inf A \Big)
	}
	\\
	\DeclareType{\SOC}{?\POSET}
	\DefineType{X}{\SOC}{ 
		\Big(\forall x : \Nat\uparrow X \. \exists \sup_{n=1} x_n \Big)
		\And
		\Big(\forall x : \Nat \downarrow X \. \exists \inf_{n=1} x_n \Big)
	}
	\\
	\Conclude{\TYPE{SigmaAlgebra}=\SA}{ \BOOL \And \SOC }{\Type}
	\\
	\Conclude{\TYPE{SigmaSubalgebra}=\TYPE{\sigma \hyph Subalgebra} = ?_{\BOOL}^\sigma}
	{ 
		\Lambda A \in \BOOL \.  
		\TYPE{Subring}(A) \And \SOC 
	}{ \NewLine :  \BOOL \to \SET}
	\\
	\Conclude{\TYPE{SigmaIdeal}=\SIdeal = \I^\sigma}
	{ 
		\Lambda A \in \BOOL \.
		\TYPE{Ideal}(A) \And \SOC
	}{\BOOL \to \SET}
	\\
	\Theorem{SigmaAlgebraBySuprema}
	{
		\forall A \in \BOOL \.
		\SA(A)
		\iff
		\forall x : \Nat \uparrow A \. \exists \sup_{n=1} x_n \in A
	}
	\NoProof
	\\
	\Theorem{SigmaAlgebraByInfima}
	{
		\forall A \in \BOOL \.
		\SA(A)
		\iff
		\forall x : \Nat \uparrow A \. \exists \inf_{n=1} x_n \in A
	}
	\NoProof
	\\
	\Theorem{SigmaAlgebraHasAllSuprema}
	{
		\forall A \in \BOOL \.
		\SA(A)
		\Imply
		\forall x : \Nat \to A \. \exists \sup_{n=1} x_n \in A
	}
	\Say{a}{\Lambda n \in \Nat \. \bigvee^n_{i=1} x_i}{ \Nat \uparrow A  }
	\SayIn{s}{\sup_{n=1} a_n}{A}
	\Say{[1]}
	{
		\Lambda n \in \Nat \. \Elim \bigvee(x_{|[1,\ldots,n]}) \Intro a \Elim s \Elim_1 \sup
	}
	{
		\forall n \in \Nat \. x_n \le \bigvee^n_{i=1} x_n = a_n \le s
	}
	\AssumeIn{b}{A}
	\Assume{[2]}{\forall n \in \Nat \. x_n \le b }
	\Say{[3]}{\Lambda n \in \Nat \. \Elim a_n \Elim \vee[2] }{\forall n \in \Nat \. a_n = \bigvee^n_{i=1} x_n \le b}
	\Conclude{[4]}{\Elim s \Elim_2 s [3]}{s \le b}
	\Derive{[2]}{\Intro \Imply \Intro \forall}{  \forall b \in A \. \Big(\forall n \in \Nat \. x_n \le b)  \Imply s \le b }
	\Conclude{[*]}{\Intro \sup [1][2]}{ s = \sup_{n=1} x_n  }
	\EndProof
}\Page{
	\Theorem{SigmaAlgebraHasAllInfima}
	{
		\forall A \in \BOOL \.
		\SA(A) 
		\Imply
		\forall x : \Nat \to A \. \exists \sup_{n=1} x_n \in A
	}
	\NoProof
	\\
	\Theorem{SigmaIdealCriterion}
	{
		\forall A \in \BOOL \.
		\forall I \in \I(A) \.
		I \in \I_\sigma(A) \iff
		\forall x : \Nat \to I \. \exists \sup_{n=1} x_n \in I
	}
	\NoProof
	\\
	\DeclareFunc{sigmaClosure}{\prod A : \SA \. ?A  \to \TYPE{\sigma\hyph Subalgebra}(A)}
	\DefineNamedFunc{sigmaClosure}{B}{\sigma(B)}{\bigcap \{ C : \TYPE{\sigma\hyph Subalgebra}(A) : B \subset C\}}
	\\
	\DeclareType{DifferenceClass}{\prod_{A \in \BOOL} ??_\BOOL^\sigma A}
	\DefineType{C}{DifferenceClass}{\forall a,b \in A \. a \setminus b \in A}
}\Page{
	\Theorem{DifferenceClassLemma}{ 
		\forall A : \SA
		\forall I \subset_{\mathsf{MONO}} (A,\wedge) \.
		\forall C : \TYPE{DifferenceClass}(A) \.
		I \subset C 
		\Imply
		\sigma(I) \subset C
	}
	\Say{\I}
	{
		\Big\{
			I \subset J \subset C : 
			\forall a,b \in J \. ab \in J
		\Big\}
	}
	{
		?\TYPE{Subobjectc}\Big( \mathsf{MONO}, A \Big)
	}
	\Say{J}{\THM{ZornLemma}(\I)}
	{
		\max \I
	}
	\Say{B}{\{ a \in A : \forall j \in J \. aj \in C \}}{?A}
	\AssumeIn{a}{B}
	\Say{[2]}{\Elim J \Elim \I \Intro e}{  e \in J   }
	\Conclude{[a.*]}{\Elim B(a,e)[2]}{a \in C}
	\Derive{[*]}{\Intro \subset}{B \subset C}
	\Say{[3]}{\Elim B \Elim J \Elim \I}{J \subset B}
	\AssumeIn{c}{A \setminus J}
	\Say{K}{J \cup  \{ cb | b \in J \}}{?A}
	\Say{[4]}{\Elim K (e) }{ c \in K}
	\Say{[5]}{[4]\Elim K}{ B \subsetneq K}
	\Say{[6]}{\Elim J [3][5]}{K \not \in \J}
	\Say{[7]}{\Elim \Elim \I \Elim K [3]}{\forall a,b \in J \. ab \in K}
	\AssumeIn{a,b}{J}
	\Conclude{\Big[(a,b).*\Big]}{ \Elim \TYPE{Commutative}(\wedge)\Elim K}
	{ 
		(ca)b  =
		(ca)(cb) =
		c(ab) \in K
	}
	\Derive{[8]}{\Intro \forall}{\forall a,b \in J \. (ca)b = (ca)(cb) \in K}
	\Say{[9]}{\Elim K [7][8]}{\forall k,k' \in K \. kk' \in K}
	\Say{[10]}{\Elim \I [6][9]}{K \not \subset C}
	\Say{\Big( b,[11] \Big)}{\Elim K [10]}{\sum b \in J \. cb \not \in C}
	\Conclude{[c.*]}{\Intro B}{c \not \in B}
	\Derive{[4]}{\Intro \TYPE{SetEq}[3]}{J = B}
	\AssumeIn{a,b}{J}
	\AssumeIn{c}{J}
	\Conclude{[c.*]}{\THM{DiffIntersercDistributivity}(a,c,c)\Elim \TYPE{Semigroup}(A,J) \Elim \TYPE{DifferenceClass}(A,C)}
	{ 
		\NewLine :
		(a \setminus b)c  =   
		(ac) \setminus b \in C
	}
	\Derive{[5]}{\Intro \forall}{\forall c \in J \. (a \setminus b)c \in C }
	\Say{[6]}{\Elim B [5]}{a \setminus b \in B}
	\Conclude{\Big[(a,b).*\Big]}{\Elim\Big(=,[3],[6]\Big)}{a \setminus b \in J}
	\Derive{[5]}{\Intro \forall }{ \forall a,b \in J \. a\setminus b \in J  }
	\Say{[6]}{\THM{BooleanSubAlgebraCriterion}[5]}{J \subset_{\BOOL} A}
	\Conclude{[*]}{\Elim \TYPE{DifferenceClass}(C)[6]\Intro \sigma}{\sigma(I) \subset \sigma(J) \subset C}
	\EndProof
}
\newpage
\subsubsection{Sigma-continuity}
\Page{
	\DeclareType{OrderContinuous}
	{
		\prod X,Y \in \mathsf{LAT} \.  ?\mathsf{LAT}(X,Y)
	}
	\DefineType{f}{OrderContinuous}{ 
		\Big( \forall x : \Nat \uparrow X \. \sup_{n=1} f(x_n) = f\big(\sup_{n=1} x_n\big) \Big)
		\And
		\Big( \forall x : \Nat \downarrow X \. \inf_{n=1} f(x_n) = f\big(\inf_{n=1} x_n\big) \Big)
	}
	\\
	\Theorem{OrderContinuousByPreimage}
	{
		\forall X,Y \in \mathsf{LAT} \.
		\forall f  \in \mathsf{LAT}  \.
		\sC(X,Y,f)
		\iff
		\NewLine
		\iff
		\forall A : \SOC(Y) \.
		\SOC\Big(X,f^{-1}(A)\Big)
	}
	\Assume{[1]}{\sC(X,Y,f)}
	\Assume{A}{\SOC(Y)}
	\Assume{x}{\Nat \uparrow f^{-1}(A)}
	\AssumeIn{s}{X}
	\Assume{[2]}{s = \sup_{n=1} x_n}
	\Say{[3]}{\Elim \sC(X,Y,f)}{ f(s) = \sup_{n=1} f(x_n) }
	\Say{[4]}{\Elim \SOC(Y,A)[3]\Elim \FUNC{preimage}(x)}{f(s) \in A}
	\Conclude{[s.*]}{\Intro \FUNC{preimage}[4]}{s \in f^{-1}(A)}
	\DeriveConclude{[A.*]}{\Intro \SOC}{ \SOC\Big( X, f^{-1}(A) \Big) }
	\DeriveConclude{[1.*]}{\Intro \forall \Intro \Imply}
	{
		\sC(X,Y,f)
		\Imply
		\NewLine \Imply
		\forall A : \SOC(Y) \.
		\SOC\Big(X,f^{-1}(A)\Big) 	
	}
	\Assume{[2]}
	{
		\forall A : \SOC(Y) \.
		\SOC\Big(X,f^{-1}(A) \Big)
	}
	\Assume{x}{\Nat \uparrow X}
	\AssumeIn{s}{X}
	\Assume{[3]}{s = \sup x_n}
	\Say{[4]}{\Elim \sup x_n \Elim \POSET(X,Y,f)}{f(x) \le f(s)}
	\AssumeIn{y}{Y}
	\Assume{[5]}{f(x) \le y}
	\Say{A}{\{ z \in Y : z \le y \}}{?Y}
	\Say{[6]}{\Elim A \Intro \SOC}{\SOC(Y,A)}
	\Say{[7]}{[2][6]}{\SOC\Big(X,f^{-1}(A)\Big)}
	\Say{[8]}{\Elim A \Intro f^{-1}}{ \forall n \in \Nat \. x_n \in  f^{-1}(A)}
	\Say{[9]}{\Elim [7][8][3]}{s \in f^{-1}(A)}
	\Say{[10]}{\Elim \FUNC{preimage}[9]}{f(s) \in A}
	\Conclude{[y.*]}{\Elim A [10]}{f(s) \le y}
	\Derive{[5]}{\Intro \forall \Intro \Imply}
	{
		\forall y \in Y \.  y \ge f(x) \Imply y \ge f(s)
	}
	\Conclude{[x.*]}{\Intro \sup}{\sup_{n=1} f(x_n) = f(s)}
	\DeriveConclude{[2.*]}{\Intro \sC}{ \sC(X,Y,f)}
}\Page{
	\Derive{[2]}{\Intro \Imply}
	{
		\forall A : \SOC(Y) \.
		\SOC\Big(X,f^{-1}(A) \Big)
		\Imply
		\NewLine
		\Imply
		\sC(X,Y,f)
	}
	\Conclude{[*]}{\Intro(\iff)[1][2]}
	{
		\sC(X,Y,f)
		\iff
		\NewLine
		\iff
		\forall A : \SOC(Y) \.
		\SOC\Big(X,f^{-1}(A) \Big)
	}
	\EndProof
	\\
	\Theorem{OCByOCAtZero}
	{
		\forall A,B \in \BOOL \.
		\forall f \in \BOOL(A,B) \.
		\Big( 
			\forall x : \Nat \downarrow A \. 
			\inf_{n=1} x_n = 0 
			\Imply
			\inf_{n=1} f(x_n) = 0 
		\Big)
		\Imply \NewLine \Imply 
		\sC(A,B,f)
	}
	\Assume{x}{\Nat \downarrow A}
	\AssumeIn{a}{A}
	\Assume{[1]}{\inf_{n=1} x_n = a }
	\Say{y}{\Lambda n \in \Nat \. (x_n \setminus a)}{\Nat \downarrow A}
	\Say{[2]}{\Lambda n \in \Nat \. \THM{ZeroIsMinimum}(A,y_n)}{ \forall n \in \Nat \. y_n \ge 0}
	\AssumeIn{b}{A}
	\Assume{[3]}{\forall n \in \Nat \. y_n \ge b}
	\Say{[4]}{\Elim y [3]}{ba = 0}
	\Say{[5]}{\Elim y \Elim \inf [3][1]}{ a \le  b \cup a \le a} 
	\Say{[6]}{\THM{DoubleIne}[5]}{b \cup a = a}
	\Conclude{[b.*]}{[6][4]}{b  = 0}
	\Derive{[3]}{\Intro \forall}
	{
		\forall b \in A \. \Big( \forall n \in \Nat \. y_n \ge b) \Imply b = 0
	}
	\Say{[4]}{\Intro \inf [2][3]}{\inf_{n=1} y_n = 0}
	\Say{[5]}{[0][4]}{\inf_{n=1} f(y_n) = 0}
	\Say{[6]}{\Elim_1 \inf [1]}{\forall n \in \Nat \. x_n \ge a}
	\Say{[7]}{\Elim y [y]}{ x = y \cup a}
	\Say{[8]}{ [7] \Elim \BOOL(A,B,f) \Elim \cup   }
	{
			f(x) = 
			f(y \cup a) =
			f(y) \cup f(a)
			\ge f(a)
	}
	\AssumeIn{b}{B}
	\Assume{[9]}{ f(x) \ge b}
	\Say{[10]}{\Elim \BOOL(A,B,f)\Intro y}{f(y) \ge \big(b \setminus f(a)\big)}
	\Say{[11]}{\Elim_2 [5][10] \THM{ZeroIsMin}(B)}{b \setminus f(a) =  0 }
	\Conclude{[b.*]}{\Elim (\setminus)[11] \Intro \TYPE{BooleanOrder}(B)}{f(a) \ge b}
	\Derive{[9]}{\Intro \Imply \Intro \forall }
	{
		\forall b \in B \. f(x) \ge b \Imply f(a) \ge b
	}
	\Conclude{[x.*]}{\Intro \inf [8][9]}{\inf f(x_n) = f(a)}
	\Derive{[1]}{\Intro \forall }{\forall x : \Nat \downarrow A \.  \inf_{n=1} f(x_n) = f\big( \inf_{n=1} x_n \big) }
	\Say{[2]}{[1]^\c}{\forall x : \Nat \uparrow A \. \sup f(x_n) = f\big( \sup_{n=1} x_n \big)}
	\Conclude{[*]}{\Intro \sC}{\sC(A,B,f)}
	\EndProof
}
\newpage
\Page{
	\Theorem{BooleanOrderContinuitySup}
	{
		\forall A,B \in \BOOL \.
		\forall f : \sC(A,B) \.  \NewLine \.
		\forall X : \TYPE{Countable}(A) \.
		f(\sup X) = \sup f(X)
	}
	\AssumeIn{a}{A}
	\Assume{[1]}{a = \sup X}
	\Say{x}{\FUNC{enumerate}(X)}{\Nat \ToBij X}
	\Say{y}{\Lambda n \in \Nat \. \bigvee^n_{i=1} x_i}{\Nat \uparrow A}
	\Say{[2]}{\Elim y \Elim_1 \sup [1]}{\forall n \in \Nat \. y_n \le a}
	\AssumeIn{a'}{A}
	\Assume{[3]}{\forall n \in \Nat \. y_n \le a'}
	\Say{[4]}{\Elim y[3]}{\forall n \in \Nat \. x_n \le a'}
	\Say{[5]}{\Elim x [4]}{\forall z \in X \. z \le a' }
	\Conclude{[a'.*]}{\Elim \sup [1][5]}{ a \le a'  }
	\Derive{[3]}{\Intro \Imply \Intro \forall}
	{
		\forall a' \in A \.
		y \le a' \Imply a \le a'
	}
	\Say{[4]}{\Intro \sup [2][3]}{\sup_{n=1} y_n = a}
	\Say{[5]}{\Elim \sC(A,B,f)[4]}{ \sup_{n=1} f(y_n) = f(a) }
	\Say{[6]}{\Elim_1 \sup [5] \Elim y \THM{BooleanMorphismIsOrderPreserving}(A,B,f)}
	{
		\forall z \in X \. f(z) \le f(a)
	}
	\AssumeIn{b}{B}
	\Assume{[7]}{\forall z \in X \. f(z) \le b}
	\Say{[8]}{[7]\Intro y \THM{BooleanMorphismIsOrderPreserving}}
	{
		\forall n \in \Nat \. f(y_n) \le b
	}
	\Conclude{[b.*]}{\Elim_2 \sup [5][8]}{ f(a) \le b  }
	\Derive{[7]}{\Intro \Imply \Intro \forall}
	{
		\forall b \in B \.
		f(X) \le b \Imply f(a) \le b
	}
	\Conclude{[*]}{\Intro \sup [6][7][1]}{\sup f(X) = f(\sup X)}
	\EndProof
	\\
	\Theorem{BooleanOrderContinuityInf}
	{
		\forall A,B \in \BOOL \.
		\forall f : \sC(A,B) \.  \NewLine \.
		\forall X : \TYPE{Countable}(A) \.
		f(\inf X) = \inf f(X)
	}
	\NoProof
}
\Page{
	\Theorem{OrderContinuousByPoUImages}
	{
		\forall A,B \in \BOOL \.
		\forall f : \BOOL(A,B) \.
		\sC(A,B,f)
		\iff 
		\NewLine
		\iff
		\forall P : \PoU(A) \. 
		|P| \le \aleph_0 \Imply
		\PoU\Big(B, f(P) \Big)
	}
	\Assume{[1] }{\sC(A,B,f)}
	\Assume{P}{\PoU(A)}
	\Assume{[2]}{|P| \le \aleph_0}
	\Say{[3]}{\Elim \PoU(A) \Intro \sup}
	{
		\sup P = e
	}
	\Say{[4]}{\Elim \sC(A,B,f)[3] \Elim \BOOL(A,B,f)}{ \sup f(P) = f(e) = e   }
	\AssumeIn{x,y}{f(P)}
	\Assume{[5]}{x \neq y}
	\Say{\Big(a,b,[6]\Big)}{\Elim \FUNC{image}[4]}
	{
		\sum a,b \in P \.  f(a) = x \And f(b) = y 
	}
	\Say{[7]}{\Elim (=,\to)[5][6]}{a \neq b}
	\Conclude{\Big[(x,y).*\Big]}{
		[6]
		\Elim \RNG(A,B,f)
		\Elim \PoU(A,P)
		\Elim \RNG(A,B,f)
	}
	{  
		xy = 
		f(a)f(b) = 
		f(ab) =
		f(0) =
		0
	}
	\DeriveConclude{[1.*]}{\Intro \PoU [4]}
	{
		\PoU\Big( B, f(P)  \Big)
	}
	\Derive{[1]}{\Intro \Imply \Intro \forall \Intro \Imply}
	{
		\NewLine
		\sC(A,B,f)  \Imply
		\forall P : \PoU(A) \.
		\Big(
			|P| \le \aleph_0 
			\Imply 
			\PoU\big(B, f(P)\big)
		\Big)
	}
	\Assume{[2]}
	{
		\forall P : \PoU(A) \.
		\Big(
			|P| \le \aleph_0 
			\Imply 
			\PoU\big(B, f(P)\big)
		\Big)	
	}
	\Assume{x}{\Nat \uparrow A}
	\AssumeIn{a}{A}
	\Assume{[3]}{\sup_{n=1} x_n = a}
	\Say{P}{ \Im X \cup \{a^\c\}}{\TYPE{Countable}(A)}
	\Say{[4]}{\Elim P \Elim \sup [3] \Intro \PoU}{\PoU(A)}
	\Say{[5]}{[2](P)}{\PoU\big(B, f(P) \big)}
	\Say{[6]}{\Elim \PoU\big(B, f(P) \big) \Intro \sup}
	{
		\sup f(P) = e
	}
	\Conclude{[x.*]}{ \Elim P \Elim \BOOL(A,B,f)\Elim \PoU\big(B,f(P)\big)[6]}
	{
		\NewLine :
		\sup_n f(x_n) = 
		\sup_n f(P) \setminus \{ f(a^\c) \} =
		\sup_n f(P) \setminus \{ f^\c(a) \} = 
		f(a)
	}
	\DeriveConclude{[2.*]}{\THM{OCByOCAtZero}}
	{
		\sC(A,B,f)
	}
	\Derive{[2]}{\Intro \Imply}
	{
		\NewLine :
		\forall P : \PoU(A) \.
		\Big(
			|P| \le \aleph_0 
			\Imply 
			\PoU\big(B, f(P)\big)
		\Big)	
		\Imply
		\sC(A,B,f)
	}
	\Conclude{[*]}{\Intro \iff [1][2]}
	{
		\NewLine : 
		\forall P : \PoU(A) \.
		\Big(
			|P| \le \aleph_0 
			\Imply 
			\PoU\big(B, f(P)\big)
		\Big)	
		\iff
		\NewLine
		\iff
		\sC(A,B,f)
	}
	\EndProof
	\\
	\Theorem{BooleanIsomorphismIsOrderContinuous}
	{
		\forall A,B \in \BOOL \.
		\forall A \ToIso{f} B : \BOOL \.
		\sC(A,B,f)
	}
	\NoProof
}
\Page{
	\Theorem{OrderContinuousPreimagingOrderSubalgebra}
	{
		\forall A,B \in \BOOL \.
		\forall f : \sC(A,B) \.
		\NewLine \. 
		\forall B' \subset_\BOOL^\sigma B \.
		f^{-1}(B') \subset_\BOOL^\sigma A
	}
	\NoProof
	\\
	\Theorem{OrderContinuousOrderSubalgebraImage}
	{
		\forall A,B \in \BOOL \.
		\forall f : \sC(A,B) \.
		\forall C \subset A \.
		\NewLine \. 
		f\;\sigma \; C =  \sigma \; f \; C
	}
	\Say{[1]}{\Elim \sigma \THM{OrderContinuousPreimagingOrderSubalgebra}}
	{
		\sigma \;C \subset f^{-1} \; \sigma f \; C
	}
	\Conclude{[*]}{f[1]}
	{
		f \sigma \; C \subset \sigma f \; C
	}
	\EndProof
	\\
	\Theorem{OrderContinuousOrderSubalgebraImage}
	{
		\forall A,B \in \BOOL \.
		\forall f : \sC \And \TYPE{Surjective}(A,B) \.
		\NewLine \.
		\forall C \subset A \.
		\sigma C = A
		\Imply
		B = f\;\sigma \; C =  
		\sigma \; f \; C 
	}
	\Say{[1]}{\THM{OrderContinuousSubalgebraImage}(A,B)}
	{
		f \; \sigma \; C \subset \sigma \; f \; C 
	}
	\Say{[2]}{[0]\Elim \TYPE{Surjective}(f)}{ f \; \sigma C =  f A = B}
	\Say{[3]}{\THM{UniversumSubset}\Big( B, \sigma \; f \; C \Big)[2]}
	{
		\sigma \; f \; C \subset B =  f \; \sigma \; C
	}
	\Conclude{[*]}{\Intro \TYPE{SetEq}[1][3]}
	{
		\sigma \; f \; C = f \; \sigma \; C
	}
	\EndProof
}
\subsubsection{Order-density}
\Page{
	\DeclareType{OrderDense}
	{
		\prod_{A \in \BOOL} ?A
	}
	\DefineType{D}{OrderDense}{\forall a \in A \. \exists d \in D : d \neq 0 \And d \le a}
	\\
	\Theorem{DensitySupTHM}
	{
		\forall A \in \BOOL \.
		\forall D \in \TYPE{OrderDense}(A) \.
		\forall a \in A \.
		\exists C : \TYPE{PairwiseDisjointElements}(A) \.
		\NewLine \.
		C \subset D \And a = \sup D
	}
	\Say{D'}{\{d \in D : d \le a  \}}{?D}
	\Say{\C}{\{ C : \TYPE{PairwiseDisjointElements}(A) : C \subset D'   \}}{?\TYPE{PairwiseDisjointElements}(A)}
	\Say{C}{\THM{ZornLemma}(\C)}{\max \C }
	\Say{[1]}{\Elim C \Elim \C \Elim D'}{C \le a}
	\AssumeIn{b}{A}
	\Assume{[0]}{C \le b}
	\Assume{[2]}{a \setminus b \neq 0}
	\Say{\Big(d,[3]\Big)}{\Elim \TYPE{OrderDense}(D)(a \setminus b)}{\sum d \in D \. d \neq 0 \And d \le (a \setminus b) }
	\Say{[4]}{\Elim a [3]\THM{SetDifferenceOrder}(a,b)}{d \le (a \setminus b) \le a }
	\Say{[5]}{\Elim D' [4]}{d \in D'}
	\Say{\Big( c,[5]\Big)}{\Elim \C \Elim C }{ \sum c \in C \. cd \neq 0  }
	\Say{[6]}{[0](c) }{  c \le b  }
	\Say{[7]}{\Elim \TYPE{BooleanOrder}[6]}{cb = c}
	\Say{[8]}{[5][3.2]\Elim(\setminus) \Elim \RNG(A)[7]\THM{BooleanRingHasChar2}(A)}
	{
		\NewLine
		0 \neq 
		cd \le 
		c( a \setminus b) = 
		c( a  + ab   ) =
		ca + cab =
		ca + ca = 0
	}
	\Conclude{[2.*]}{ \THM{ZeroIsMinimal}(A)[8]}{\bot}
	\Derive{[2]}{\Elim(\bot)}{a \setminus b = 0}
	\Conclude{[b.*]}{\Elim(\setminus)[2]\Intro \TYPE{BooleanOrder}}{ a \le b}
	\Derive{[2]}{\Intro \Imply \Intro \forall}{ \forall b \in A \. C \le b \Imply a \le b}
	\Conclude{[*]}{\Intro \sup [1][2]}{\sup C = a }
	\EndProof
}\Page{
	\Theorem{OrderDenseContainsPartitionOfUnity}
	{
		\forall A \in \BOOL \.
		\forall D : \OD  \. \NewLine \.
		\exists P : \PoU(A) : P \subset D
	}
	\Say{\Big(P,[1] \Big)}{\THM{DensitySupTHM}(A,D,e)}{\sum  P: \PD(A) \. P \subset D \And e = \sup P}
	\AssumeIn{a}{A}
	\Assume{[2]}{a \neq 0}
	\Assume{[3]}{\forall p \in P \. pa = 0}
	\SayIn{b}{a^\c}{A}
	\AssumeIn{p}{P}
	\Say{[4]}{ \Elim(b) \Elim \RNG(A) [3](p) }
	{
		pb = 
		p(e + a) = 
		p + ap = 
		p
	}
	\Conclude{[p.*]}{ \Intro \TYPE{BooleanOrder}[4] }
	{
		p \le b
	}
	\Derive{[4]}{\Intro \TYPE{SetLe}}{P \le b}
	\Say{[5]}{\Elim b [2]}{b \neq e}
	\Say{[6]}{\THM{UnityIsMax}[5]}{ b < e }
	\Conclude{[a.*]}{\Elim \sup [1.2][4][6]}{\bot}
	\DeriveConclude{[*]}{\Elim \bot \Intro \PoU}{\PoU(A,P)}
	\EndProof
}
\newpage
\subsubsection{Regular embeddings}
\Page{
	\Conclude{\REing}{\prod_{A,B} \in \BOOL \. \BOOL \And \TYPE{Injective} \And \sC(A,B)}
	{
		\BOOL^2 \to \SET
	}
	\\
	\DeclareType{\REed}{\prod_{A \in \BOOL} ?\TYPE{Subring}(A)}
	\DefineType{B}{\REed}{
		\REing(A,B,\iota_B)
	}
	\\
	\DeclareType{\REable}{\BOOL \to ?\BOOL}
	\DefineType{A}{\REable}{
		\Lambda B \in \BOOL \. \exists \REing(A,B) 
	}
	\\
	\Theorem{OrderDenseIsEmbeddable}
	{
		\forall B \in \BOOL \.
		\forall D : \OD \And \TYPE{Subring}(B) \.
		\REed(B,D)
	}
	\Assume{x}{\Nat \downarrow D}
	\Assume{[1]}{\inf_{n=1} x_n =_B 0}
	\Assume{[2]}{\inf_{n=1} x_n {\neq}_A 0}
	\Say{\Big(a,[3]\Big)}{\Elim \inf [2]}
	{
		\sum a \in A \.
		0 < a < x 
	}
	\Say{\Big(d,[4]\Big)}{\Elim \OD(A,D)(a)}
	{
		\sum d \in D \.
		0 \neq d \le a
	}
	\Say{[5]}{[3][4]}{ d < x }
	\Say{[6]}{\Intro \inf [5]}{\inf_{n=1} x_n \neq_B 0}
	\Conclude{[x.*]}{[6][1]}{\bot}
	\Derive{[1]}{\Elim \bot \Intro \Imply \Intro \forall}
	{
		\forall x : \Nat \downarrow D \.
		\inf_{n=1} x_n =_B 0 
		\Imply
		\inf_{n=1} x_n =_A 0
	}
	\Conclude{[*]}{\THM{OCByOCAtZero}[1]\Intro \REed}
	{
		\REed(B,D)
	}
	\EndProof
	\\
	\Theorem{OrderCKernelIsSigmaIdeal}
	{
		\forall A,B \in \BOOL \.
		\forall f  : \sC \And \BOOL(A,B) \.
		\SIdeal\Big(A ,\ker f \Big)
	}
	\Assume{x}{\Nat \uparrow \ker f}
	\AssumeIn{a}{A}
	\Assume{[1]}{\sup_{n=1} x_n = a}
	\Say{[3]}{\Elim \ker f (x)}{f(x) = 0}
	\Say{[4]}{\Elim \sC(A,B,f)[3]}
	{
		0 = 
		\inf_{n=1} 0 =
		\inf_{n=1} f(x_n) = 
		f\Big( \sup_{n=1} x_n \Big) =
		f(a)
	}
	\Conclude{[x.*]}{\Elim \ker f [4]}{a \in \ker f}
	\DeriveConclude{[*]}{\THM{SigmaIdealBySup}}
	{
		\SIdeal\Big( A,\ker f \Big)
	}
	\EndProof
}
\Page{
	\Theorem{OrderCBySigmaIdeal}
	{
		\forall A,B \in \BOOL \.
		\forall f \in \BOOL(A,B) \.
		\NewLine \.
		\SIdeal\Big(A, \ker f \Big) 
		\And
		\REed \Big( B, \im f \Big)
		\Imply
		\sC(A,B,f)
	}
	\Assume{x}{\Nat \downarrow A}
	\AssumeIn{a}{A}
	\Assume{[1]}{\inf_{n=1} x_n = 0}
	\AssumeIn{b}{f(A)}
	\Assume{[2]}{ f(x) \ge  b > 0 }
	\Say{\Big(s,[3]\Big)}{\Elim \TYPE{Image}(b)}
	{
		\sum s \in A \. b = f(s)
	}
	\Say{[4]}{\Elim \BOOL(A,B)[3][2]} 
	{
		f(s \setminus x) = 
		f(s) \setminus f(x) = 
		b \setminus f(x) = 
		0
	}
	\Say{[5]}{\Elim \ker f [4]}
	{
	 	s \setminus x \in   (\ker f)^\Nat
	}
	\Say{[6]}{ \THM{ComplementSup}  [1]\Elim \setminus}
	{
		\sup_{n=1} s \setminus x_n =
		s \setminus \inf_{n=1} x_n =
		s \setminus 0 =
		s 
	}
	\Say{[7]}{\Elim \SIdeal(A,\ker f)[6]}
	{
		s \in \ker f
	}
	\Say{[8]}{\Elim \ker f [7][3]}{b = 0}
	\Conclude{[b.*]}{[2][8]}{\bot}
	\Derive{[2]}{\Intro \inf}{\inf_{n=1} f(x_n) =_{f(A)} 0}
	\Conclude{[x.*]}{\Elim \REed(B,\im f)}{\inf_{n=1} f(x_n) =_B 0}
	\DeriveConclude{[*]}{\THM{OCByOCAtZero}}{\sC(A,B,f)}
	\EndProof
	\\
	\Theorem{SigmaIdealTHM}
	{
		\forall A \in \BOOL \.
		\forall I : \Ideal(A) \.
		\SIdeal(A,I)
		\iff
		\sC\left(A,\frac{A}{I},\pi_I \right)
	}
	\NoProof
}
\newpage
\subsubsection{Order-Continuity and Stone Spaces}
\Page{
	\Theorem{OrderContiniousOpenImage}
	{
		\forall A,B \in \BOOL \.
		\forall f : \oC(A,B) \. \NewLine \. 
		\forall U \in \T \; \Z \; B \.
		U \neq \emptyset
		\Imply
		\intx  (\Z\;f)(U) \neq \emptyset
	}
	\Say{u}{\Elim \TYPE{NonEmpty}}{U}
	\Say{\Big( b,[2]\Big)}{\THM{StoneBase}(B,U,u)}{ u \in S_B(b) \subset U  }
	\Say{[3]}{\THM{InterioIsMonotonic}[2]}
	{
		\intx (\Z\;f)\Big( S_B(b) \Big) \subset \intx (\Z \; f)(U)
	}
	\Say{[4]}{\THM{StoneRepresentationIsCompact}(B,b)\;\THM{CompactImage}\Big( \Z\;B,\Z\;A,\Z\;f\Big)}
	{
		\NewLine :
		\Compacts\bigg( \Z\;A, (\Z\;f)\Big( S_B(b)\Big) \bigg)
	}
	\Say{[5]}{\THM{HausdorffCompactIsDense}[4]}
	{
		\Closed\bigg( \Z;A, (\Z\;f)\Big( S_B(b)\Big) \bigg)	
	}
	\Assume{[6]}{\intx (\Z\;f)(U) = \emptyset}
	\Say{[7]}{ \Intro \ND  [5][6] }
	{
		\ND\bigg( \Z\;A, (\Z\;f)\Big( S_B(b)\Big)
	}
	\Say{D}{
		 \bigg\{ a \in A : S_A(a) \cap (\Z\;f)\Big( S_B(b)\Big) = \emptyset \bigg\}    
	}{?A}
	\Say{[8]}{ \Elim \ND [7]\Elim D \Intro \OD}
	{
		\OD( A, D)
	}
	\Say{[9]}{\THM{OrderDenseSup}[8]}{\sup D = e}
	\Say{[10]}{\Elim \oC(A,B,f) [9] \Elim \BOOL(A,B,f)}{\sup f(D) = e }
	\Say{\Big( d,[11]\Big)}{\Elim_2 \sup [10](u)}
	{
		\sum d \in D \. f(d)b \neq  0
	}
	\Say{\Big(v,[12]\Big)}{\THM{StoneRepresentationTHM}[11]}
	{
		\sum v \in \S_B(b) \.   v\Big( f(d)b \Big) = 1
	}
	\Say{[13]}{\Intro \Z [12]}{ (\Z \; f)(v) \in S_A(d) \cap (\Z\;f)\Big( S_B(b) \big)   }
	\Conclude{[6.*]}{\Elim D(d)[13]}{\bot}
	\DeriveConclude{[*]}{\Elim \bot}
	{
		\intx \;  (\Z\;f)(U) \neq \emptyset
	}
	\EndProof
	\\
	\Theorem{OrderContinuousNDPreimage}
	{
		\forall A,B \in \BOOL \.
		\forall f : \oC(A,B) \. \NewLine \. 
		\forall N : \ND( \Z \; A ) \.
		\ND\Big( \Z \; B , (\Z \; f)^{-1}(N) \Big)
	}
	\Assume{N}{\ND(\Z\;A)}
	\Say{[1]}{\Elim \ND\Big( \Z\;A,N\Big)}
	{
		\intx \overline{N} = \emptyset
	}
	\Say{C}{(\Z\;  f)^{-1}(\overline{N})}{\TYPE{Cloed}\Big( \Z\;B\Big)}
	\Say{U}{\intx C}{ \TYPE{Open}\Big( \Z\; B \Big)  }
	\Say{[2]}{\Elim \intx \Elim U [1]}
	{
		\intx (\Z \; f)( U ) = \emptyset
	}
	\Say{[3]}{\THM{OrderContinuousOpenImage}[2]}
	{
		U = \emptyset
	}
	\Conclude{[*]}{\Elim \FUNC{closure} \Elim U \Elim C[3] \Intro \ND}
	{
		\ND\Big( \Z\;A, (\Z \; f)^{-1}(N)  \Big)
	}
	\EndProof
}
\Page{
	\Theorem{OrderContinuityByNowhereDenseOreimage}
	{
		\forall A,B \in \BOOL \. 
		\forall f \in \BOOL(A,B) \.
		\NewLine \.
		\Big(
			\forall N : \ND\Big( \Z \; A \Big) \.
			\ND\Big(\Z \; B, (\Z \; f)^{-1}(N) \Big)
		\Big)
		\Imply
		\oC\Big(A,B,f\Big)
	}
	\Assume{X}{?A}
	\Assume{[1]}{\inf X = 0}
	\Say{N}{ \bigcap_{x \in X} S_A(x)}{?\Z \; A}
	\Say{[2]}{ [1] \Elim N \THM{ZeroInfinumCriterion} }
	{
		\ND\Big( \Z \; A,  N   \Big)
	}
	\Say{[3]}{[0][2]}
	{
		\ND\Big( \Z\;B, (\Z \; f)^{-1}(N) \Big)
	}
	\Say{[4]}{\Elim N \Elim \THM{IntersectPreimage} \Elim (\Z\; f)}
	{
		(\Z \;f)^{-1}(N) = 
		\bigcap_{x \in X  } (\Z\;f)^{-1}\Big( S_A(x) \Big) =
		=
		\bigcap_{x \in X  S_A( f(x)  }
	}
	\Conclude{[X.*]}
	{
		\THM{ZeroInfimumCriterion}[3][4]	
	}
	{
		\inf f(X) = 0
	}
	\DeriveConclude{[*]}
	{
		\THM{OCByOCAtZero}
	}
	{
		\oC(A,B,f)
	}
	\EndProof
}
\newpage
\subsubsection{Upper Envelopes}
\Page{
	\DeclareType{upperEnvelope}{\prod_{A \in \BOOL} \prod_{B \subset_\BOOL A} A \to ?B}
	\DefineNamedType{b}{upperEnvelope}{\Lambda a \in A \. b =\upr_B(a)}
	{
		b = \inf\{ x \in B : x \ge a   \}
	}
	\\
	\Theorem{UprAndSupCommute}
	{
		\forall A \in \BOOL \. 
		\forall B \subset_\BOOL A \.
		\forall X \subset A \.
		\forall y \in A \.
		\forall b \in B \.
		\NewLine \. 
		\bigg(
		\Big( \forall x \in X \. \exists \upr_B(x) \Big) 
		\And y = \sup X \And
		b = \sup_{x \in X} \upr_B(x) 
		\bigg)
		\Imply
		b = \upr_B(y)
	}
	\NoProof
	\\
	\Theorem{UprAndIntersectCommute}
	{
		\forall A \in \BOOL \. 
		\forall B \subset_\BOOL A \.
		\forall a \in A \.
		\forall b \in B \.
		\NewLine \.
		\exists \upr_B(a) \Imply
		\upr_B(a \cap b)  = \upr_B(a) \cap b
	}
	\NoProof
}
\newpage
\subsection{Order-Completenes}
\subsubsection{Sigma-Completenes}
\Page{
	\DeclareType{\SComplete}{?\mathsf{LATT}}
	\DefineType{L}{\SComplete}{\forall  x : \Nat \to L \. \exists \inf x \And \exists \sup x} 
	\\
	\DeclareType{\SCompletes}{\prod_{L \in \mathsf{L}}?L}
	\DefineType{A}{\SCompletes}{\forall  x : \Nat \to A \. \exists \inf x \And \exists \sup x} 
	\\
	\Theorem{SigmaCompleteQuotient}
	{
		\forall B : \SComplete \.
		\forall I : \SIdeal(B) \.
		\SComplete\left(\frac{B}{I}\right)
	}
	\Assume{[x]}{\Nat \to \frac{B}{I}}
	\Say{\Big(b,[1]\Big)}{\Elim \SComplete (B)}
	{
		\sum b \in B \. b = \sup_{n=1} x_n
	}
	\Conclude{[x.*]}{\THM{SigmaIdealTHM}}
	{
		\sup_{n=1} [x_n] = [b]
	}
	\DeriveConclude{[*]}{\Intro \SComplete}
	{
		\SComplete\left( \frac{B}{I} \right)
	}
	\EndProof
	\\
	\Theorem{SigmaAlgebraFactorization}
	{
		\forall X \in \SET \.
		\forall A : \SA(X) \.
		\forall I : \SIdeal(X) \. \NewLine \.
		\SComplete\left( \frac{A}{I \cap A} \right)
	}
	\NoProof
	\\
	\Theorem{SigmaCompleteSubalgebras}
	{
		\forall A \in \BOOL \And \SComplete \.
		\forall B \subset^\sigma_\BOOL A \. 
		\NewLine \. 
		\SCompletes(A,B)
	}
	\NoProof
	\\
	\Theorem{SigmaCompleteIdeal}
	{
		\forall A \in \BOOL \And \SComplete \.
		\forall I : \SIdeal(A)  \. 
		\NewLine \. 
		\SCompletes(A,I)
	}
	\NoProof
}\Page{
	\Theorem{SigmaCompleteImage}
	{
		\forall A,B \in \BOOL \.
		\forall f : \sC(A,B) \.
		\SComplete(A)
		\Imply
		\NewLine 
		\Imply
		\SOC(B,f(A))
	}
	\Assume{y}{\Nat \uparrow f(A)}
	\AssumeIn{b}{B}
	\Assume{[1]}{\sup_{n=1} y_n = b}
	\Say{\Big(x,[2]\Big)}{\Elim f(A)(y)}
	{
		\sum x : \Nat \uparrow A \. 
		y = f(x)
	}
	\Say{\Big(a,[3]\Big)}
	{ 
		\Elim \SComplete(A)(x)
	}
	{
		\sum_{a \in A} a = \sup_{n=1} x_n
	}
	\Say{[4]}{[1][2]\Elim \sC(A,B,f)[3]}
	{
		b = 
		\sup_{n=1} y_n = 
		\sup_{n=1} f(x_n) =
		f\Big( \sup_{n=1} x_n) =
		f(a)
	}
	\Conclude{[y.*]}{\Elim f(A) [4]}{b \in f(A)}
	\DeriveConclude{[*]}{\Intro \SOC}
	{
		\SOC\Big(B,f(A)\Big)
	}
	\EndProof
	\\
	\Theorem{SigmaContinuousByCompleteImage}
	{
		\forall A \in \BOOL \.
		\forall B : \SComplete \.
		\forall f : \TYPE{Injective} \And \BOOL(A,B) \.
		\NewLine
		\SOC\Big(B,f(A)\Big) \Imply
		\sC(A,B,f)
	}
	\Assume{x}{\Nat \downarrow A}
	\Assume{[1]}{\inf_{n=1} x_n = 0}
	\Say{\Big(b,[2]\Big)}{\Elim \SComplete(B)\Big( f(x) \Big)}
	{
		\sum_{b \in B} \inf_{n=1} f(x_n) = b
	}
	\Say{[3]}{\Elim \SOC\Big(B,f(A)\Big)[2]}
	{
		b \in f(A)
	}
	\Say{\Big( a,[4]\Big)}{\Elim f(A)[3]}{\sum a \in A \. b = f(a)}
	\Assume{[5]}{a \neq 0}
	\Say{[6]}{\Elim \TYPE{Injective} \And \BOOL(A,B,f)[5]}
	{
		b \neq 0
	}
	\Say{\Big(n,[7]\Big)}{\Elim \inf [1][5](a)}
	{
		\sum_{n=1}^\infty  x_n < a
	}
	\Say{[8]}{\Elim \BOOL \And \TYPE{Injective}(A,B,f)(a)[4]}
	{
		f(x_n) < f(a) = b
	}
	\Conclude{[*.5]}{\Elim \inf [2][8]}{\bot}
	\Derive{[5]}{\Elim \bot}{a = 0}
	\Conclude{[x.*]}{[2][4]\Elim \BOOL(A,B,f)[5]}
	{
		\inf_{n=1} f(x_n) = 0
	}
	\Derive{[*]}{\THM{OCByOCAtZero}}
	{
		\sC(A,B,f)
	}
	\EndProof
}
\Page{
	\Theorem{SigmaCompleteSubalgebraCriterion}
	{
		\forall A : \SComplete \And \BOOL \.
		\forall B \subset_\BOOL A \.
		\NewLine
		\SOC(A,B) 
		\iff 
		\SComplete(B)
		\And 
		\sC\Big(B,A, \iota_B \Big)
	}
	\NoProof
	\\
	\Theorem{SigmaCompleteSigmaGenetrationCommutation}
	{
		\NewLine ::
		\forall A : \SComplete \And \BOOL \.
		\forall B \in \BOOL \. 
		\forall X \subset A \.
		\forall f : \sC(A,B,f) \.
		\NewLine \. 
		f\; \sigma \; X = \sigma \; f \; X
	}
	\Say{[1]}{\Elim \sigma \Elim \FUNC{image}(f,X)}
	{
		f\; X \subset f \; \sigma \; X
	}
	\Say{[2]}{\THM{SigmaCompleteImage}\Big( \sigma X, B, f\Big)}
	{
		\SOC\Big(B, f \sigma X\Big)
	}
	\Say{[3]}{\Elim \sigma f X [1][2]}
	{
		\sigma \; f \;X \subset f \sigma X
	}
	\Say{[4]}{\THM{OrderContinuousOrderSubalgebraImage}(A,B,f,X)}
	{
		f \; \sigma \; X \subset f \; \sigma \; X
	}
	\Conclude{[*]}{\Intro \TYPE{SetEq}[3][4]}
	{
		f \; \sigma \; X = \sigma \; f \; X
	}
	\EndProof
	\\
	\Theorem{IsomorphismByOrderDenseInjecttion}
	{
		\forall A  : \TAlgebra \.
		\forall B : \BOOL \.
		\NewLine \. 
		\forall f : \TYPE{Injective} \And \BOOL(A,B) \.
		\OD\Big(f(A),B\Big)
		\Imply
		\TYPE{Isomorphism}\Big(\BOOL,A,B,f \Big)
	}
	\NoProof
}
\newpage
\subsubsection{Morphism Extension}
\Page{
	\Theorem{SigmaCompleteExtensionLemma}
	{
		\forall A : \SComplete \And \BOOL \.
		\forall B \subset_\BOOL^\sigma A \.
		\forall a \in A \.
		B_a \subset_\BOOL^\sigma A
	}
	\Assume{x}{\Nat \to B_a}
	\Say{\Big(y,z,[1]\Big)}
	{
		\THM{SubalgebraGeneratedByOneElement}(A,B,a,x)
	}
	{
		\sum y,z : \Nat \to B \.
		x = (y \setminus a) \cup (z \cap a)
	}
	\Say{\Big(y',[2]\Big)}{ \Elim \SComplete (A,y) \Elim \SOC(A,B)   }
	{
		\sum_{y' \in B}  y' = \sup_{n=1} y_n 
	}
	\Say{\Big(z',[3]\Big)}{ \Elim \SComplete (A,z) \Elim \SOC(A,B)   }
	{
		\sum_{z' \in B}  z' = \sup_{n=1} z_n 
	}
	\SayIn{x'}{(y'\setminus a) \cup (z' \cap a)}{B_a}
	\Say{[4]}{\Elim \sup [2] \Elim \sup [3]\Elim x'}
	{  x \le x'  }
	\AssumeIn{x''}{A}
	\Assume{[5]}{x \le x''}
	\Say{[6]}{[5]\THM{IntersectionSup}[3]}{x'' \ge \sup_{n=1} z_n \cap a   = z' \cap a}
	\Say{[7]}{[5] \THM{ComplementSup}[2] }{x'' \ge \sup_{n=1} y_n \setminus a =  y' \setminus a}
	\Conclude{[x''.*]}{\Intro x' [6][7]}{x'' \ge x'}
	\Derive{[5]}{\Intro \Imply \Intro \forall }
	{
		\forall x'' \in A \. x'' \ge x \Imply x'' \ge x
	}
	\Conclude{[x.*]}{\Intro \sup [4][5]}{x' = \sup_{n=1} x_n}
	\DeriveConclude{[*]}{\Intro \SOC }{B_a \subset_\BOOL^\sigma A}
	\EndProof
}\Page{
	\Theorem{HomomorphismExtensionTHM}
	{
		\forall A  : \BOOL \. 
		\forall B  : \BOOL \And \OComplete \.
		\NewLine
		\forall C \subset_\BOOL A \.
		\forall C \Arrow{f} B : \BOOL \.
		\exists A \Arrow{\hat f} B : \BOOL :
		{\hat f}_{|C} = f
	}
	\Say{P}{ 
		\Big\{  
			D \Arrow{f} B   : \BOOL \Big| 
			C \subset_\BOOL D \subset_\BOOL A 
		\Big\}   
	}
	{
		\SET\bigg(\uparrow\Big( \BOOL \Big) \bigg) 
	}
	\Say{[1]}{\Elim P (f)}{f \in P}
	\Say{\hat f}{\THM{ZornLemma}(P)}
	{
		\max P	
	}
	\Say{D}{\dom \hat f}{\TYPE{Subring}(A)}
	\Assume{[2]}{D \neq A}
	\SayIn{a}{\Intro (\setminus)[2] \Elim (A \setminus D)}{A \setminus D}
	\Say{X}{ \{ d \in D :  d \le a   \}  }{?D}
	\SayIn{y}{ \sup \hat f(X)}{B}
	\AssumeIn{d,d'}{D}
	\Assume{[3]}{d \le a \le d'}
	\Say{[4]}{\Elim X (d) [3] }{ d \in X  }
	\Say{\Big[(d,d').*.2\Big]}{\Elim y \Elim_1 \sup [4]}{  \hat f(d) \le y  }
	\Say{[6]}{\Elim X [3]\Intro d'}{ X \le d'}
	\Say{[7]}{\Elim \BOOL(D,B,\hat f)[6]}{f(X) \le f(d')}
	\Conclude{\Big[(d,d').*.2\Big]}{\Elim_2 \sup [7]}{ y \le \hat f(d')   }
	\Derive{[3]}{\Intro \Imply \Intro \forall}
	{
		\forall d,d' \in D \.
		\forall 
		(d \le a \le d')
		\Imply
		f(d) \le y \le f(d')
	}
	\Say{\Big( g ,[4]\Big)}{\THM{HomonorphismExtension}(A,B,D,\hat f,a,y)[3]}
	{
		\sum D_a \Arrow{g} B : \BOOL \. g_{|D} = \hat f \And g(a) = y
	}
	\Say{[5]}{\Elim P [4.1]}{g \in P}
	\Say{[6]}{\Elim a \Elim g}{\hat f < g}
	\Conclude{[2.*]}{[6]\Elim \max P(\hat f)}{\bot}
	\DeriveConclude{[*]}{\Elim \bot  }{ \dom \hat f = A}
	\EndProof
}
\newpage
\subsubsection{Loomis-Sikorski Representation}
\Page{
	\DeclareFunc{LoomisSikorskiAlgebra}{\prod A \in \BOOL \And \SComplete \TYPE{\sigma\hyph Subalgebra}(?\Z \; A) }
	\DefineNamedFunc{LoomisSikorskiAlgebra}{}{\LS(A)}{\Big\{ U \du M  \Big|  U \in \TK\;\Z\;A \And M \in \mathbf{MGR}\;\Z\;A \Big\}}
	\Say{[1]}{\Elim \LS(A)}{ \emptyset,A \in \LS(A)}
	\AssumeIn{U \du M,U' \du M'}{\LS(A)}
	\Say{[\ldots*.1]}{
		\LOGIC{CheckingTruthTable}
		\Elim \TYPE{Subring}\Big(?\Z\;A ,\TK \;\Z\;A\Big)
		\Elim\SIdeal\Big(?\Z\;A,\mathsf{MGR}(\Z\;A)\Big)\Elim \LS(A)
	}{
		\NewLine :
		(U \du M) \du (U' \du M') =
		(U \du U') \du (M \du M') \in \LS(A)
	}
	\Conclude{[\ldots*.2]}{
		\LOGIC{CheckingTruthTable}
		\Elim \TYPE{Subring}\Big(?\Z\;A ,\TK \;\Z\;A\Big)
		\Elim\SIdeal\Big(?\Z\;A,\mathsf{MGR}(\Z\;A)\Big)\Elim \LS(A)	
	}{  
		\NewLine :
		(U  \du M) \cap (U' \du M') =  
		(U \cap U') \du (M \cap M') \du (U \cap M') \du (U' \cap M) 
		\in \LS(A)
	}
	\Derive{[2]}{\Intro \TYPE{Subring}}{ \LS(A) \subset_\BOOL ?\Z\;A   }
	\Assume{U \du M}{\Nat \to \LS(A)}
	\Say{\Big(a,[3]\Big)}{ \THM{ClopenAreStoneRepresentations}(A,U) }
	{
		\sum a : \Nat \to A \. 
		\forall n \in \Nat \.
		S_A(a_n) = U_n
	}
	\SayIn{a'}{\sup_{n=1} a_n}{A}
	\Say{[4]}{\THM{SupremumStonrRepresentation}\Elim a'}
	{
		S_A(a') = \overline{\bigcup_{n=1} U_n}
	}
	\Say{[5]}{\THM{NowhereDenseClosure}[4]}
	{
		\ND\left( \Z\;A, S_A(a') \setminus \bigcup_{n=1} U_n\right)
	}
	\Say{E}{ S_A(a') \setminus \bigcup_{n=1} U_n \du M_n}{\Nat \to ?\Z\;A}
	\Say{[6]}{\Elim E \Elim \du}
	{
		E \subset  
		\left(S_A(a') \setminus \bigcup_{n=1}\right) 
		\cup
		\bigcup_{n=1} M_n
	}
	\Say{[7]}{\THM{MeagerSubset}[5][6]}
	{
		E \in \mathsf{MGR}(A)
	}
	\Conclude{\Big[(U \du M).*\Big]}{\THM{UnionAsSup}\Intro E \Elim \du}
	{
		S_A(a') 
		\sup_{n=1} U_n \du M_n = 
		\bigcup^\infty_{n=1} U_n \du M_n =
		S_A(a') \du E
	}
	\Derive{[3]}{\Intro \exists\Intro \forall}
	{
		\forall U \du M : \Nat \to \LS(A) \. 
		\exists \sup_{n=1} U_n \du M_n
	}
	\Conclude{[*]}{\Intro (\subset_\BOOL^\sigma)}
	{
		\LS(A) \subset_\BOOL^\sigma ? \Z \; A
	}
	\EndProof
}\Page{
	\Theorem{LoomisSikorskiRepresentation}
	{
		\forall A : \BOOL \And \SComplete \. 
		\frac{\LS(A)}{\mathbf{MGR}\;\Z\;A} \cong_\BOOL A
	}
	\AssumeIn{[U \du M]}{\frac{\LS(A)}{\mathbf{MGR}\;\Z\;A}}
	\Say{\Big(\varphi[U \du M],[1] \Big)}{\THM{OpenCompactsAreStoneRepresentations}(A,U)}
	{
		\sum \varphi[U \du M] \in A \. \NewLine \. U = S_A\Big(\varphi[U \du M]\Big)
	}
	\AssumeIn{[U' \du M']}{\frac{\LS(A)}{\mathbf{MGR}\;\Z\;A}}
	\Assume{[2]}{[U \du M] = [U' \du M']}
	\Say{\Big( M'',[3]\Big)}{\Elim \FUNC{quotientRing}[2]}
	{ \sum M'' \in \mathbf{MGR}\;\Z\;A \. U = U' \du M \du M' \du M''}
	\Say{[4]}{\Elim \TYPE{Bair}[3]\THM{BairCategoryTHM}(\Z\;A)}{ M \du M' \du M = \emptyset    }
	\Conclude{\Big[[U \du M].*\Big]}{[4][3][1]}
	{   
		 \varphi[U \du M]  =
		 \varphi[U' \du M'] 
	}
	\DeriveConclude{\varphi}{\Intro(\to)}
	{
		\TYPE{Isomorphism}\left( \BOOL, \frac{\LS \; A}{\mathbf{MGR}\; \Z \; A}, A \right)
	}
	\EndProof
	\\
	\Theorem{LoomisSikorskiTHM}
	{
		\forall A \in \BOOL \. 
		\SComplete(A) 
		\iff
		\NewLine
		\iff
		\exists X \in \SET :
		\exists \A : \SA(X) : 
		\exists \I \in \I_\sigma(\A) \.
		A \cong_{\BOOL} \frac{\A}{\I}
	}
	\NoProof
}
\newpage
\subsubsection{Algebra of Open Domains}
\Page{
	\DeclareFunc{nowhereDenseIdeal}{\prod_{X \in \TOP} \Ideal(?X)}
	\DefineNamedFunc{nowhereDenseIdeal}{X}{\nd(X)}
	{
		\ND(X)		
	}
	\AssumeIn{A,B}{\nd(X)}
	\Say{[1]}{\THM{NowhereDenseUnion}(X,A,B)}{A \cup B \in \nd(X)}
	\Say{[2]}{ \THM{SymmetricDifferenceSubset}(X,A,B)  }{A \du B \subset A \cup b}
	\Conclude{\Big[(A,B).*\Big]}{\THM{NowhereDenseSubset}[1][2]}{A \du B \in \nd(X)}
	\Derive{[1]}{\Intro \forall }
	{
		\forall A,B \in \nd(X) \. A \du B \in \nd(X)
	}
	\AssumeIn{A}{\nd(X)}
	\Assume{B}{?X}
	\Say{[2]}{\THM{IntersectionIsSubset}(X,A,B)}{A \cap B \subset A }
	\Conclude{[A.*]}{\THM{NowhereDenseSubset}[2]}{A \cap B \subset \nd(X)}
	\Derive{[2]}{\Intro \forall}{\forall A \in \nd(X) \. \forall B \subset X \. AB \in \nd(X)}
	\Conclude{[*]}{\Intro \Ideal [1][2]}{\Ideal\Big(?X,\nd(X)\Big)}
	\EndProof
	\\
	\DeclareFunc{weaklyBoundedAlgebra}{\TOP \to \BOOL}
	\DefineNamedFunc{weaklyBoundedAlgebra}{X}{\Sigma(X)}{ \Big\{ A \subset X : \boundary A \in \nd(X)  \Big\} }
	\Say{[1]}{\Elim \boundary \emptyset \Type \Ideal\Big(?X,\nd(X)\Big) }
	{
		\boundary \emptyset = \emptyset \in \nd(X)
	}
	\Say{[2]}{\Elim \Sigma(X)[1]}{\emptyset \in \sigma(X)}
	\AssumeIn{A,B}{\Sigma(X)}
	\Say{[3]}{\THM{ClosureUnion}(X,A,B)}{\overline{A \cup B} = \overline{A} \cup \overline{B}}
	\Say{[4]}{\THM{SubsetOfUnion}\THM{InteriorMonotonic}(X)}
	{
		\intx(A) \cup \intx(B) \subset \intx(A \cup B)
	}
	\Say{[5]}{
		\Elim \boundary(A \cup B)
		[3][4]
		\LOGIC{CheckingTruthTables}
		\Intro \boundary A 
		\Intro \boundary B
	}
	{
		\NewLine  :
		\boundary(A \cup B) = 
		\overline{A \cup B} \setminus \intx(A \cup B) \subset 
		\Big(\overline{A} \cup \overline{B}\Big) \setminus \Big( \intx A \cup \intx B \Big) \subset \NewLine \subset
		\Big( \overline{A} \setminus \intx(A) \Big) \cup \Big( \overline{B} \setminus \intx{B} \Big) =
		\boundary A \cup \boundary B
	}
	\Say{[6]}{
		\Elim \Sigma(A) 
		\Elim \Sigma(B) 
		\Elim \nd(X) 
		\THM{NowhereDenseUnion}(X)
		\THM{NowhwereDenseSubset}(X)[5]
	}
	{
		\boundary(A \cup B) \in \nd(X)
	}
	\Conclude{\Big[(A,B).*\Big]}{\Elim \Sigma(X)[6]}
	{
		A \cup B \in \Sigma(X)
	}
	\Derive{[3]}{\Intro \forall}
	{
		\forall A,B \in \Sigma(X) \.
		A \cup B \in \Sigma(X)
	}
	\AssumeIn{A}{\Sigma(X)}
	\Say{[4]}{\Elim \Sigma(X)(A) }{\boundary A \in \nd(X)}
	\Say{[5]}{\THM{BoundaryComplement}[4]}
	{
		\boundary A^\c = 
		\boundary A  \in \nd(X)
	}
	\Conclude{[A.*]}{\Elim \Sigma(X)[5]}
	{
		A^\c \in \Sigma(X)
	}
	\Derive{[4]}{\Intro \forall}
	{
		\forall A \in \Sigma(X) \. A^\c \in \Sigma(X)
	}
	\Conclude{[*]}{\THM{SubalgebraCritertion}[2][3][4]}
	{
		\Sigma(X) \in \BOOL
	}
	\EndProof
}\Page{
	\Theorem{NowhereDenseAreWeaklyBounded}
	{
		\forall X \in \TOP \. 
		\nd(X) \subset \Sigma(X)
	}
	\AssumeIn{A}{\nd(X)}
	\Say{[1]}{\THM{NowhereDenseClosureIsNowhereDense}(X,A)}{ \overline{A} \in \nd(X)}
	\Say{[2]}{\THM{BoundaryInClosure}(X,A)}{\boundary A \subset \overline{A}}
	\Say{[3]}{\THM{NowhereDenseSubset}[1][2]}{ \boundary A \in \nd(X)   }
	\Conclude{[A.*]}{\Elim \Sigma(X)[3]}{A \in \Sigma(X)}
	\DeriveConclude{[*]}{\Intro \subset}{ \nd(X) \subset \Sigma(X)}
	\EndProof
	\\
	\DeclareFunc{openDomainAlgebra}{\TOP \to \BOOL}
	\DefineNamedFunc{openDomainAlgebra}{X}{\od(X)}
	{
		\frac{\Sigma(X)}{\nd(X)}
	}
	\\
	\Theorem{OpenDomainRepresentation}
	{
		\forall X \in \TOP \.
		\forall A \in \Sigma(X) \.
		\exists! U : \TYPE{OpenDomain} \.
		U \du A \in \nd(X)
	}
	\SayIn{U}{\intx \overline{A}}{\T(X)}
	\Say{[1]}{\THM{ClosedSetInteriorIsOpenDomain}(X)\Elim U}
	{
		\TYPE{OpenDomain}(X,U)
	}
	\Say{[2]}{ \Elim U \Elim \intx \Intro \intx \Intro \boundary }
	{
		U \setminus A =
		\intx \overline{A} \setminus A \subset
		\overline{A} \setminus \intx A = 
		\boundary A 
	}
	\Say{[3]}{\Elim \Sigma(X)(A) \THM{NowhereDenseSubset}[2] }
	{
		U \setminus A \in \nd(X)
	}
	\Say{[4]}{ \Elim U \Elim \intx \Intro \intx \Intro \boundary }
	{
		A \setminus U =
		A \setminus \intx \overline{A}  \subset
		\overline{A} \setminus \intx A = 
		\boundary A 
	}
	\Say{[5]}{\Elim \Sigma(X)(A) \THM{NowhereDenseSubset}[4] }
	{
		A \setminus U \in \nd(X)
	}
	\Say{[6]}{\THM{SymmetricDifferenceExpression}(X,A,U)\THM{NowhereDenseUnion}(X)[3][5]}
	{
		\NewLine :		
		A \du U = (A \setminus U) \cup (U \setminus A) \in \nd(X)
	}
	\Assume{V}{\TYPE{OpenDomain}(X)}
	\Assume{[7]}{A \du V \in \nd(X)}
	\Say{[8]}{\Elim \Ideal\Big(?X,\nd(X)\Big)[6][7]}
	{
		U \du V \in \nd(X)
	}
	\Say{[9]}{\Elim \overline{U} \THM{AntitoneSetDifference}(X) \THM{DifferenceSubsets}(X)}
	{
		V \setminus \overline{U} \subset V \setminus U \subset U \du V
	}
	\Say{[10]}{\THM{OpenAndNowhereDense}[9][8]}
	{
		V \setminus \overline{U} = \emptyset
	}
	\Say{[11]}{ \THM{OpenInterior}(X,V) \Elim(\setminus)[10] \THM{MonotonicInterior}(X) \Elim \TYPE{OpenDomain}(X,U)  }
	{
		V = \intx V \subset \intx \overline{U} = U
	}
	\Say{[12]}{\Elim \overline{U} \THM{AntitoneSetDifference}(X) \THM{DifferenceSubsets}(X)}
	{
		U \setminus \overline{V} \subset U \setminus V \subset U \du V
	}
	\Say{[13]}{\THM{OpenAndNowhereDense}[12][8]}
	{
		V \setminus \overline{U} = \emptyset
	}
	\Say{[14]}{ \THM{OpenInterior}(X,U) \Elim(\setminus)[13] \THM{MonotonicInterior}(X) \Elim \TYPE{OpenDomain}(X,V)  }
	{
		U = \intx U \subset \intx \overline{V} = V
	}
	\Conclude{[*]}{\Intro \TYPE{SetEq}[11][14]}{U = V}
	\EndProof
	\\
	\Theorem{SumOfOpenDomains}
	{
		\forall X \in \TOP \.
		\forall A,B \in \od \; X \.
		A + B = \intx \Big( A \du B\Big)
	}
	\NoProof
	\\
	\Theorem{ProductOfOpenDomains}
	{
		\forall X \in \TOP \.
		\forall A,B \in \od \; X \.
		AB = A \cap B
	}
	\NoProof
}\Page{
	\Theorem{OpenDomainsInfinum}{
		\forall X \in \TOP \. 
		\forall A \subset \od \; X \.
		\inf A = \intx \bigcap A 
	}
	\Say{F}{\intx \bigcap A}{\TYPE{Open}(X)}
	\Say{[1]}{\Elim F \THM{OpenDomainIntersectionInterior}(X) \THM{InteriorOfClosedSetIsOpenDomain}}
	{
		\NewLine :		
		F = \intx \bigcap A =  \intx \overline{\bigcap A} \in \od \; X
	}
	\Say{[2]}{\Elim F  \THM{IntersectioIsSubset} \THM{InteriorIsSubset} \Intro F}
	{
		\forall U \in A \. F \subset U 
	}
	\Say{[3]}{\Intro \od(X) [1][2]}{ \forall U \in A \. F \le_{\od(X)} U   }
	\AssumeIn{G}{\od(X)}
	\Assume{[4]}{\forall U \in A \. G \le_{\od(X)} U}
	\Say{[5]}{\THM{SubsetOfIntersection}(X)\Elim \od(X)[4]}
	{
		G \subset \bigcap A
	}
	\Say{[6]}{\THM{OpenInteriorSubset}(A,G)[5]\Intro F}{G \subset F}
	\Conclude{[G.*]}{\Intro \od(X)[6]}{G \le_{\od(X)} F}
	\DeriveConclude{[*]}{\Intro \inf \Elim F[3]}{ \inf A = \intx \bigcap A  }
	\EndProof
	\\
	\Theorem{OpenDomainsSupremum}{
		\forall X \in \TOP \. 
		\forall A \subset \od \; X \.
		\sup A = \intx \overline{\bigcup A} 
	}
	\SayIn{F}{\intx \overline{\bigcup A}}{\od(X)}
	\AssumeIn{U}{A}
	\Say{[1]}{\THM{SubsetOfUnion}(X,A,U)}{U \subset \bigcup A}
	\Say{[2]}{[1]\THM{ClosureIsUperset}\left( \bigcup \right)}{U \subset \overline{\bigcup A}}
	\Say{[3]}{[2]\THM{OpenInteriorSubset}\Intro F}{U \subset F}
	\Conclude{[*]}{[3]\Intro \od(X)}{ U \le_{\od(X)} F }
	\Derive{[1]}{\Intro \forall}{\forall U \in A \. U \le_{\od(X)} F}
	\AssumeIn{G}{\od(X) }
	\Assume{[2]}{\forall U \in A \. U \le_{\od(X)} G}
	\Say{[3]}{\THM{SubsetUnion}[2]}
	{
		\bigcup A \subset G
	}
	\Conclude{[G.*]}{
		\THM{InteriorIsMonotonic}(X) 
		\THM{ClosureIsMonotonic}(X)
		\Elim \od(X)(G)
		\Intro F
		\Intro \od(X)
	}
	{
		\NewLine :
		F \le_{\od(X)} G
	}
	\DeriveConclude{[*]}{\Intro \sup}
	{
		\sup A = \intx \overline{\bigcup A}
	}
	\EndProof
	\\
	\Theorem{OpenDomainAlgebraIsDedekindComplete}
	{
		\forall X \in \TOP \.
		\OComplete( \od \; X)
	}
	\NoProof
}\Page{
	\DeclareType{\POpen}
	{
		\prod_{X,Y \in \TOP} ?\TOP(X,Y) 
	}
	\DefineType{f}{\POpen}{\forall N : \ND(Y) \. \ND\Big( X,f^{-1}(N)\Big)}
	\\
	\DeclareFunc{HomoOD}
	{
		\prod_{X,Y \in \TOP} \POpen(X,Y) \to  
		\BOOL \And \oC\Big( \mathbf{OD}(Y), \mathbf{OD}(X) \Big)
	}
	\DefineNamedFunc{HomoOD}{f}{\tilde{f}}
	{
		\Lambda U \in \mathbf{OD}(Y) \. \intx_X \overline{f^{-1}(U)} 
	}
	\Assume{U,V}{\od(Y)}
	\Say{[1]}{\THM{IntersectionIsSubet}(U,V)}{U \cap V \subset U \And U \cap V \subset V}
	\Say{[2]}{\Elim \TYPE{Monotonic}\Big( \tilde f \Big)[1]}
	{
		{\tilde f}(U \cap V) \subset {\tilde f}(U)
		\And
		{\tilde f}(U \cap V) \subset {\tilde f}(V)
	}
	\Say{[3]}{\THM{SubsetIntersection}[2]}
	{
		{\tilde f}(U \cap V) \subset {\tilde f}(U) \cap {\tilde f}(V)
	}
	\Assume{[4]}
	{
		{\tilde f}(U \cap V) \neq {\tilde f}(U) \cap {\tilde f}(V)
	}
	\Say{G}
	{
		{\tilde f}(U \cap V) \setminus 
		\overline{ {\tilde f}(U) \cap {\tilde f}(V)} 
	}
	{
		?X
	}
	\Say{[5]}{[3][4]\Elim \od(X)\Big({\tilde f}(U \cap V) \Big) \Intro G}
	{ 
		G \neq \emptyset 
	}
	\Say{M}{\overline{f(G)}}
	{
		\TYPE{Closed}(Y)
	}
	\Say{[6]}{\Elim M}
	{
		G \subset f^{-1}(M)
	}
	\Say{[7]}{\THM{RegularOpenDifferenceIsNotMeage}[6]}
	{
		\neg \ND\Big( X, f^{-1}(M) \Big)
	}
	\Say{[8]}{ \Elim \POpen(X,Y,f)[7]  }
	{
		\neg \ND\Big( Y, M \Big)
	}
	\SayIn{H}{ \intx M  }{\T(Y)}
	\Say{[9]}{\Elim G \Elim (\tilde f) \THM{InetriorSubset}}
	{
		G \subset {\tilde f}(U) \subset \overline{f^{-1}(U)}
	}
	\Say{[10]}{ \overline{f([9])} \Intro M  }
	{
		M =
		\overline{f(G)} \subset 
		\overline{f\overline{f^{-1}(U)} }  
		\subset
		\overline{U}
	}
	\Say{[11]}{\Intro H [10] \Elim \do(Y,U) }
	{
		H \subset \intx \overline{U} = U
	}
	\Say{[12]}{\Elim G \Elim (\tilde f) \THM{InetriorSubset}}
	{
		G \subset {\tilde f}(V) \subset \overline{f^{-1}(V)}
	}
	\Say{[13]}{ \overline{f([9])} \Intro M  }
	{
		M =
		\overline{f(G)} \subset 
		\overline{f\overline{f^{-1}(V)} }  
		\subset
		\overline{V}
	}
	\Say{[14]}{\Intro H [10] \Elim \do(Y,V) }
	{
		H \subset \intx \overline{V} = V
	}
	\Say{[15]}{f^{-1}\Big([14][11]\Big)\Intro (\tilde f)}
	{
		f^{-1}(H) \subset f^{-1}(V \cap U) \subset
		{\tilde f}(V \cap U)
	}
	\Say{[16]}{\Elim H \Intro  G[15]}
	{
		\emptyset \neq 
		G \cap f^{-1}(H)  \subset
		G \cap {\tilde f}(V \cap W)
	}
	\Say{[17]}{\Elim G [16]}{\bot}
	\DeriveConclude{[1]}{\Elim \bot}
	{
		{\tilde f}(UV) = 
		\Big({\tilde f}(U)\Big)
		\Big( {\tilde f}(V)\Big)
	}
	\AssumeIn{U}{\od(Y)}
	\Say{V}{\overline{U}^\c}{\od(Y)}
	\Say{[2]}{\Elim V \Elim \tilde f \Elim \FUNC{preimage}}
	{
		{\tilde f}(V) \cap {\tilde f}(U) = \emptyset
	}
	\Say{[3]}{\Elim V \Intro \Dense}
	{
		\TYPE{Open} \And \Dense\Big( Y, U \cup V  \Big)
	}
	\Say{[4]}{\Elim \FUNC{preimage} \Intro {\tilde f}}
	{
		f^{-1}(U \cup V) = 
		f^{-1}(U) \cup f^{-1}(V) \subset
		{\tilde f}(U) \cup {\tilde f}(V)
	}
	\Conclude{[U.*]}{[3][4]\Elim \POpen(X,Y,f)\THM{DifferenceOfOpenDomiansIsNotMeager}}
	{
		\tilde f(V) =  \overline{ \tilde{f}(U) }^\c
	}
	\Derive{[2]}{\Intro \forall}{\forall U \in \od(Y) \.  {\tilde f}(U^\c) = (\tilde f)^\c(U)}
	\Say{[3]}{\Intro \BOOL[1][2]}{{\tilde f} \in \BOOL\Big( \od(Y),\od(X) \Big)}
}
\Page{
	\Assume{A}{?\od(Y)}
	\Assume{[4]}{Y = \sup A}
	\Say{[5]}{\THM{OpenDominSupremum}[4]\Intro \TYPE{Dense} }{\TYPE{Open} \And \Dense\left( Y, \bigcup A \right)}
	\Say{[6]}{\THM{PreimageUnion}(X,Y,f,A)\Intro \tilde f}
	{
		f^{-1} \bigcup A = \bigcup f^{-1}(A) \le  \bigcup {\tilde f}(A) 
	}
	\Say{[7]}{\Elim \POpen(X,Y,f)[5][6]}
	{
		\TYPE{Open} \And \Dense\left( X, \bigcup {\tilde f}(A) \right) 
	}
	\Conclude{[A.*]}{\THM{OpenDomainSuperemum}[7]}{  \sup {\tilde f}(A) = X     }
	\DeriveConclude{[*]}{\Intro \oC }{\oC\Big(\od(Y),\od(X), \tilde f \Big)}
	\EndProof
	\\
	\Theorem{OrderCompleteBooleanAlgebraIsExtremlyDisconnected}
	{
		\NewLine ::
		\forall B \in \BOOL \.
		\OComplete(B) \Imply
		\ExtDisc(\Z\;B)
	}
	\AssumeIn{U}{\T\;\Z\;B}
	\Say{A}{ \{ a \in A :  S_B(a) \subset U \}}{?B}
	\SayIn{a}{\sup A}{B}
	\Say{[1]}{\Elim A \THM{StoneTHM}(B)\THM{OpenAsUnionCover}(U)}{U = \bigcup_{a \in A} S_B(a)}
	\Say{[2]}{\Elim a \Elim \sup \Intro \FUNC{closure}}{\overline{U} = S_B(a)}
	\Conclude{[U.*]}{\Elim \Clopen [2]}{ \overline{U} \in \T\;\Z\:B  }
	\DeriveConclude{[*]}{\Intro \ExtDisc}
	{
		\ExtDisc(\Z\;B)
	}
	\EndProof
	\\
	\Theorem{CompactOpenAlgebraOfAlgebraWithExtremelyDisconnectedStoneSpace}
	{
		\NewLine ::
		\forall B \in \BOOL \.
		\ExtDisc(\Z \; B)
		\Imply
		\od(\Z\;B) = \TK(\Z\;B)
	}
	\AssumeIn{U}{\TK(\Z\;B)}
	\Say{[1]}{\THM{ClosedClosure}(\Z\;B)\THM{OpenInterior}(\Z\;B)}
	{
		\intx \overline{U} = \intx U = U
	}
	\Conclude{[2]}{\Elim \od(\Z\;B)[1] }
	{
		U \in \od(Z\;B)
	}
	\Derive{[1]}{\Intro \subset}{ \TK(\Z\;B) \subset \od(\Z\;B)}
	\AssumeIn{U}{\od(\Z\;B)}
	\Say{[2]}{\Elim \od(\Z\;B,U)}
	{
		\intx \overline{U} = U
	}
	\Say{[3]}{\Elim \ExtDisc(\Z\;B)\THM{OpenInterion}}{\overline{U} = U}
	\Conclude{[4]}{\Elim \TK\;\Z\;B[3]}{U \in \TK\;\Z\;B}
	\Derive{[2]}{\Intro \subset}
	{
		 \od(\Z\;B) \subset \TK\;\Z\;B
	}
	\Conclude{[*]}{\Intro \TYPE{SetEq}[1][2]}
	{
		\od(\Z\;B) = \TK \; \Z \; B
	}
	\EndProof
}\Page{
	\Theorem{AlgebraIsDedekindCompleteIfStoneSpaceOpenDomainAlgebraIsCompectOpenAlgebra}
	{
		\NewLine ::
		\forall B \in \BOOL \.
		\od(\Z\;B) = \TK \; \Z \; B
		\Imply
		\OComplete(B)
	}
	\Assume{A}{?B}
	\Say{\A}{ S_B(A)}{?\TK\;\Z\;B}
	\Say{[1]}{\Elim \A [0] \Intro \A}{ \A \subset \OD(\Z\;B,)   }
	\SayIn{U}{\sup \A}{\OD(\Z\;B)}
	\Say{[2]}{\Elim U [0] \Intro U}{U \in \TK \;\Z\;B}
	\Say{\Big(a, [3] \Big)}{\THM{StoneRepresentationTHM}(B,U)}
	{
		\sum a \in B \.  U = S_B(a)
	}
	\Conclude{[A.*]}{\Elim a \Intro \sup \Intro a}{\sup A = a}
	\Derive{[*]}{\Intro \OComplete}{\OComplete(B)}
	\EndProof
	\\
	\Theorem{OpenDomainContravariantFunctor}
	{
		\NewLine ::
		\forall X,Y : \ExtDisc \.
		\forall X \Arrow{f} Y \.
		f_* \in \BOOL\Big(\od(Y),\od(X)\Big)
	}
	\NoProof
	\\
	\Theorem{OpenDomainOrderContinuityImplyClopen}
	{
		\NewLine ::
		\forall X,Y : \ExtDisc \And \HC \.
		\forall X \Arrow{f} Y : \TOP\.
		\NewLine
		\TYPE{Surjective}(X,Y,f)
		\And
		\oC\Big(\od(Y),\od(X),f_*\Big)
		\Imply 
		\forall 
		U \in \TK(X) \.
		f(U) \in \TK(Y)
	}
	\Say{[1]}{\THM{ClosedMappingTHM}(X,Y,f)}{\TYPE{ClosedMapping}(X,Y,f)}
	\AssumeIn{U}{\TK(X)}
	\Say{[2]}{\Elim \TYPE{ClosedMappingTHM}(X,Y,f)}
	{
		\TYPE{Closed}\Big(Y,f(U)\Big)
	}
	\Say{[3]}{\Elim \ExtDisc(Y)[2]}
	{
		\intx f(U) \in \TK(Y)
	}
	\Say{\mathcal{V}}{\{V \in \TK(Y) : V \subset f(U)\}}{?\TK(Y)}	
	\Say{[4]}{
		\Elim \od(Y)
		\Big( \intx f(U) \Big) \THM{InteriorAsUnion} 
		\Intro \mathcal{V}
		\THM{OpenDomainSup}
	}
	{
		\intx f(U) = 
		\intx \overline{\intx f(U) } =
		\intx \overline{\bigcup \mathcal{V}} =
		\sup \mathcal{V}
	}
	\Say{[5]}{ \Elim \V  }{ f_*(\V) \le U  }
	\Say{[]}{\ldots}{}
	\Say{[]}{\Intro f_* [4] \Elim \oC\Big(\od(Y),\od(X),f_*\Big)  }
	{
		\NewLine :		
		f^{-1}\Big( \intx f(U)\Big) = 
		f_*\Big( \intx f(U) \Big) =
		f_*\Big( \sup \V \Big) = 
		\sup f_*( \V) = 
		\intx \overline{\bigcup_{V \in \mathcal{V}} f^{-1}(V)} =
		\overline{\bigcup_{V \in \mathcal{V}} f^{-1}(V)} =
		U
	}
	\NoProof
}\Page{
	\Theorem{ClopenImplyOpen}
	{
		\forall X,Y : \ExtDisc \And \HC \.
		\forall X \Arrow{f} Y : \TOP\.
		\NewLine
		\TYPE{Surjective}(X,Y,f)
		\And
		U \in \TK(X) \.
		f(U) \in \TK(Y)
		\Imply
		\TYPE{OpenMap}\Big(X,Y,f\Big)
	}
	\NoProof
	\\
	\Theorem{OpenImplyOpenDomainOrderContinuity}
	{
		\forall X,Y : \ExtDisc \And \HC \.
		\forall X \Arrow{f} Y : \TOP\.
		\NewLine
		\TYPE{Surjective}(X,Y,f)
		\And
		\TYPE{OpenMap}\Big(X,Y,f\Big)
		\Imply
		\oC\Big(\od(Y),\od(X),f_* \Big)
	}
	\NoProof
	\\
	\Theorem{DenseSubetOpenDomainTransition}
	{
		\forall X \in \TOP \.
		\forall Y  : \Dense(X) \. \NewLine \.
		\TYPE{Isomorphism}\Big(\BOOL,\od(X),\od(Y),\Lambda U \in \od(X) \. U \cap Y\Big)
	}
	\NoProof
}
\newpage
\subsubsection{Dedekind Completion}
\Page{
	\Theorem{StoneRepresentationIsInjectiveOC}
	{
		\NewLine ::		
		\forall B \in \BOOL \.
		\BOOL \And \oC \And \TYPE{Injective}\Big( B, \od \; \Z \; B, S_B \Big)
	}
	\Assume{A}{?B}
	\Assume{[1]}{\inf A  = 0 }
	\Say{[2]}{\THM{ZeroInfinumCriterion}[1]}{\ND\left(\Z \; B \right)}
	\Conclude{[A.*]}{}{\inf S_B(A) = \intx \bigcap_{a \in A} S_B(a) = \emptyset}
	\Derive{[*]}{}{\oC\Big( B, \od \; \Z \; B,S_B\Big)}
	\EndProof
	\\
	\Theorem{BooleanAlgebraCompletionUniversalProperty}
	{
		\NewLine ::		
		\forall B \in \BOOL \.
		\forall C \in \BOOL \And \OComplete \.
		\forall f \in \BOOL \And \oC(B,C) \. \NewLine \.
		\exists! {\hat f} \in \BOOL \And \oC\Big( \od(\Z \; B  ), C \Big) :
		S_B{\hat f} = f
	}
	\AssumeIn{U}{\od(\Z \; B)}
	\Say{A}{\{ a \in B : S_B(a) \subset U \}}{?B}
	\Conclude{{\hat f}(U)}{ \sup f(A)}{C}
	\Derive{\hat f}{\Intro(\to)}
	{
		\BOOL\And \oC\Big(\od(\Z\;B),C\Big)
	}
	\Conclude{[*]}{\Elim {\hat f}}
	{
		S_B {\hat f} = f
	}
	\EndProof
}
\newpage
\subsubsection{Principle Ideals}
\Page{
	\Theorem{PrincipleIdealsAreOrderComplete}
	{
		\forall B \in \BOOL \. 
		\forall I : \Principle(B) \.
		\OC(B,I) 
	}
	\Say{\Big( b,[1]\Big)}{\Elim \Principle(B.I)}
	{
		\sum b \in B \. I = \langle b \rangle
	}
	\Assume{A}{?I}
	\AssumeIn{s}{A}
	\Assume{[2]}{s = \sup A}
	\Say{[3]}{\THM{PrincipleIdealStructure}(B,I,A)[1]}{A \le b}
	\Say{[4]}{\Elim \sup A [2][3]}{ s \le b}
	\Conclude{[s.*]}{\THM{PrincipleIdealStructure}(B,I,s)[1][4]}{s \in I}
	\DeriveConclude{[*]}{\Intro \OC}{\OC(B,I)}
	\EndProof
	\\
	\Theorem{DedekindCompleteByPrincipleIdeals}
	{
		\forall B \in \BOOL \.
		\OComplete(B) 
		\iff
		\NewLine
		\iff
		\forall I : \Ideal \And \OC(B) \.
		\Principle(I)		
	}
	\Assume{[1]}{\OComplete(B)}
	\Assume{I}{\Ideal \And \OC(B)}
	\SayIn{s}{\sup I}{B}
	\Say{[2]}{\Elim s \Elim \OC(B,I)}{s \in I}
	\Say{[3]}{\Elim s \Elim \sup I [2] \Elim \Ideal(I)}{I = \{ b \in B : b \le s\}}
	\Conclude{[1.*]}{\THM{PrincipleIdealStructure}[3]}{\Principle(B,I)}
	\Derive{I}{\Intro \forall \Intro \Imply}
	{
		\forall B \in \BOOL \.
		\OComplete(B) 
		\Imply
		\NewLine
		\Imply
		\forall I \in \Ideal \And \OC(B) \.
		\Principle(I)
	}
	\Assume{[2]}
	{
		\forall I : \Ideal \And \OC(B) \.
		\Principle(I)
	}
	\Assume{A}{?B}
	\Say{I}{\langle A \rangle_\tau}{\Ideal \And \OC(B)}
	\Say{\Big(s,[2]\Big)}{[2](A)}{\sum s \in B \. I = \langle s \rangle}
	\Say{[3]}{\THM{PrincipleIdealStructure}[2]\Elim A}{ \forall a \in A \. a \le s}
	\AssumeIn{z}{B}
	\Assume{[4]}{\forall a \in A . a \le z  }
	\Say{[5]}{[2][2]\Elim I [4]\THM{PrincipleIdealsAreOrderComplete}}{   \langle s \rangle =    I   \subset \langle z \rangle}
	\Say{[6]}{[5]\Elim \Principle\Big(B,\langle s \rangle\Big) }{s \in \langle  z\rangle}
	\Conclude{[z.*]}{\THM{PrincipleIdealStructure}[6]}{s \le z}
	\DeriveConclude{[A.*]}{\Intro \sup [3]}{s = \sup A}
	\DeriveConclude{[*]}{\Intro \Imply \Intro \iff [1]}
	{
		\OComplete(B) 
		\iff
		\NewLine
		\iff
		\forall I : \Ideal \And \OC(B) \.
		\Principle(I)
	}
	\EndProof
}
\Page{
	\Theorem{OrderClosedSubalgerbraOfPrincipleIdeal}
	{
		\NewLine ::		
		\forall A : \TAlgebra \.
		\forall B \subset_\BOOL^\tau A \.
		\forall a \in A \.
		\{ ab | b \in B \} \subset_\BOOL^\tau \langle a \rangle
	}
	\Say{C}{\{ab | b \in B \}}{\TYPE{Subalgebra}(A)}
	\Say{[1]}{
		\Elim C 
		\THM{PrincipleIdealStructure}
	}
	{
		\TYPE{Subalgebra}\Big(\langle a \rangle, C\Big)
	}
	\Assume{X}{?C}
	\Say{Y}{\{ b \in B : ab \in X \}}{?B}
	\SayIn{b}{\sup Y}{B}
	\Say{[2]}{\THM{BooleanRingIsALattice}\Elim C \Elim b}
	{
		\forall x \in X \. x \le ab
	}
	\AssumeIn{z}{\langle a \rangle}
	\Assume{[3]}{\forall x \in X \. x \le z}
	\Say{Z}{ \{ u \in A : ua = z \}}{?A}
	\SayIn{z'}{\sup Z}{A}
	\Say{[4]}{\Elim z'}{ z' = a^\c \vee z}
	\AssumeIn{y}{Y}
	\Say{\Big(x,u,[5]\Big)}{\Elim Y(y)}
	{
		\sum x \in X \. \sum u \in \langle a^\c \rangle \.
		y = x \vee u
	}
	\Conclude{[y.*]}{\THM{BooleanRingIsALattice}[4][5]}
	{
		 y \le z'
	}
	\Derive{[5]}{\Intro \forall}
	{
		\forall	y \in Y \. y \le z'
	}
	\Say{[6]}{\Elim \sup \Elim z' [5]}{b \le z'}
	\Conclude{[z.*]}{\THM{BooleanRingIsALattice}\Elim z'}
	{
		ab \le az' = z
	}
	\DeriveConclude{[X.*]}{\Intro \sup [2]}{ \sup X = ab}
	\DeriveConclude{[*]}{\Intro \TYPE{OrderClosedSubalgebra}}
	{
		C \subset_\BOOL^\tau \langle a \rangle
	}
	\EndProof
	\\
	\Theorem{KernelIsAPrincipleIdeal}
	{
		\forall A : \TAlgebra \.
		\forall B \in \BOOL \. \NewLine \. 
		\forall f : \oC \And \BOOL(A,B) \.
		\Principle(A,\ker f)
	}
	\Assume{X}{? \ker f}
	\SayIn{a}{\sup X}{A}
	\Say{[1]}{\Elim a \THM{BooleanOrderContinuousSup}(A,B,f,X) \Elim \ker f \Elim \sup}
	{
		f(a) = 
		f(\sup X) =
		\sup f(X) =
		\sup \{0\} =
		0 
	}
	\Conclude{[X.*]}{\Elim \ker f [1]}{a \in \ker f}
	\Derive{[1]}{\Intro \OC}{\OC(A,\ker f)}
	\Conclude{[*]}{\THM{DedekindCompleteByPrincipleIdeals}(B)(\ker f)}
	{
			\Principle(A,\ker f)
	}
	\EndProof
	\\
	\DeclareFunc{kernelElement}{
		\prod A : \TAlgebra \. \NewLine \.
		\prod_{B \in \BOOL} 
		\Big(\oC \And \BOOL(A,B)\Big)	\to A
	}
	\DefineNamedFunc{kernelElement}{f}{k_f}
	{\THM{KernelIsAPrincipleIdeal}\Elim \Principle(A)}
}
\Page{
	\Theorem{AlgebraDeterminedByKernelElement}
	{
		\forall A : \TAlgebra \.
		\forall B \in \BOOL \. \NewLine \. 
		\forall f : \Surj \And \oC \And \BOOL(A,B) \.
		B \cong_\BOOL \langle k_f^\c \rangle
	}
	\Say{[1]}{\THM{IsomorphismTHM}}
	{
		B \cong_\BOOL \frac{A}{\ker f}
	}
	\AssumeIn{[a]}{\frac{A}{\ker f}}
	\Say{\Big( u,v, [2] \Big)}{\Elim \c (k_f)(a)}
	{
		\sum u \in \langle k_f \rangle \.
		\sum v \in \langle k_f^\c \rangle \.
		a = u + v
	}
	\Conclude{[a.*]}{\Elim k_f [2]}{ [a] = [v]  }
	\DeriveConclude{[*]}{\Intro \cong_\BOOL [1]}
	{
			B \cong \langle k^\c_f \rangle
	}
	\EndProof
}
\newpage
\subsubsection{Upper Envelopes in Complete Algebras}
\Page{
	\Theorem{DisjointUpperEnvelopesTHM}
	{
		\forall A : \TAlgebra \.
		\forall B \subset_\BOOL^\tau A \. 	
		\forall a \in A \.
		\upr_B(a)\upr_B(a^\c) = 0
		\Imply
		a \in B
	}
	\Say{X}{\{ b \in B : a \le b\}}{?B}
	\Say{[1]}{\Elim X}{ a \le X }
	\Say{[2]}{\Elim \OC(A,B)\Elim \upr_B(a) \Elim \sup [1]}{a \le \upr_B(a) }
	\Say{X}{\{ b \in B : a^\c \le b\}}{?B}
	\Say{[3]}{\Elim Y}{ a^\c \le Y }
	\Say{[4]}{\Elim \OC(A,B)\Elim \upr_B(a^\c) \Elim \sup [3]}{a^\c \le \upr_B(a^\c) }
	\Say{[5]}{\THM{ComplementProduct}[2][4]}
	{
		a = \upr_B(a)
	}
	\Conclude{[*]}{\Elim \upr_B(a)[5]}{a \in B}
	\EndProof
	\\
	\Theorem{UpperEnvelopesIdentityExtension}
	{
		\forall A : \TAlgebra \.
		\forall B \subset_\BOOL^\tau A \. 	
		\forall a \in A \.
		\forall b \in B \.
		\NewLine
		\upr^\c_B(a^\c) \le b \le \upr_B(a) 
		\iff
		\exists A \Arrow{f} B : \BOOL :
		f_{|B} = {\id}_B \And f(a) = b
	}
	\NoProof
	\\
	\Theorem{UpperEnvelopeAndSubalgebraExtension}{
		\NewLine ::		
		\forall A : \TAlgebra \.
		\forall B \subset^\tau_\BOOL A \.
		\forall a \in A \.
		\forall c \in B_a \.
		ac = a \upr_B(ac)
	}
	\NoProof
}
\newpage
\subsubsection{Basically Disconnected Spaces }
\Page{
	\DeclareType{Cozero}{\prod_{X \in \TOP}??X}
	\DefineType{A}{Cozero}{\exists X \Arrow{f} \Reals : \TOP : A = \{ x \in X : f(x) \neq 0 \}}
	\\
	\DeclareType{BasicallyDisconnected}{?\TOP}
	\DefineType{X}{BasicallyDisconnected}{
		\forall A \subset X \. 
		\overline{A} \in \T(X)
		\iff
		\TYPE{Cozero}(X,A)
	}
	\\
	\Theorem{SigmaCompleteByBasicallyDisconnectedStoneSpace}
	{
		\NewLine :: 		
		\forall B \in \BOOL \. 
		\SA(B) \iff  \TYPE{BasicallyDisconnected}(\Z\;B)
	}
	\NoProof
}
\newpage
\subsubsection{Algebra of Ideals}
\Page{
	\DeclareFunc{idealComplement}{\prod_{B \in \BOOL} \I_\tau(B) \to \I_\tau(B)}
	\DefineNamedFunc{idealComplement}{I}{\overline{I}}
	{\{ b \in B : \forall i \in I \. ib = 0 \}}
	\\
	\DeclareFunc{idealJoin}{\prod_{B \in \BOOL}\I_\tau^2(B) \to \I_\tau(B)}
	\DefineNamedFunc{idealJoin}{I,J}{I \vee J}{ \overline{\overline{I}\cap\overline{J}}}
	\\
	\Theorem{TauIdealsAreBooleanLattice}
	{
		\forall B \in \BOOL \. 
		\BL\Big( \I_\tau(B),\cap,\vee \Big)
	}
	\NoProof
	\\
	\Theorem{TauIdealsAreTauAlgebra}
	{
		\forall B \in \BOOL \. 
		\TAlgebra\Big( \I_\tau(B) \Big)
	}
	\NoProof
	\\
	\Theorem{PrincipalIdealGenerationIsInjection}
	{
		\NewLine ::		
		\forall B \in \BOOL \. 
		\Inj \And \oC \And \BOOL\Big( B, \I_\tau(B)  , \Lambda b \in B \. \langle b \rangle \Big)
	}
	\NoProof
	\\
	\Theorem{PrincipleIdealsAreOrderDense}
	{
		\forall B \in \BOOL \.
		\OD\Big(\I_\tau(B), \Principle(B) \Big)
	}
	\NoProof
	\\
	\Theorem{PrinicapalIdealAreOrderCompletion}
	{
		\forall B \in \BOOL \.
		\I_\tau(B) \cong_\BOOL \od(\Z\;B)
	}
	\NoProof
}
\newpage
\subsection{Category Limits}
\subsubsection{Products}
\Page{
	\Theorem{ProductOfBooleanAlgebrasIsBooleanAlgebra}
	{
		\forall I \in \SET \.
		\forall B : I \to \BOOL \.
		\prod_{i \in I} B_i \in \BOOL
	}
	\AssumeIn{b}{\prod_{i \in I} B_i}
	\Conclude{[b.*]}{
		\Elim \prod_{i\in I} B_i, b 
		\Lambda i \in \I \. \Elim \BOOL(B_i)
		\Intro \prod_{i \in I} B_i, b
	}
	{
		b^2 = (b_i^2)_{i=1} = (b_i)_{i=1} = b	
	}
	\DeriveConclude{[*]}{\Intro \BOOL}
	{
		\prod_{i \in I} B_i \in \BOOL
	}
	\EndProof
	\\
	\Theorem{BooleanProductOrderIsAProductOrder}
	{
		\forall I \in \SET \.
		\forall B : I \to \BOOL \.
		\NewLine
		\forall a,b \in \prod_{i \in I} B \.
		a \le b \iff \forall i \in I \. a_i \le b_i
	}
	\NoProof
	\\
	\DeclareFunc{algebraProductEmbedding}
	{
		\prod_{I \in \SET} 
		\prod_{B : I \to \BOOL} 
		\prod_{i \in I} 
		\BOOL\left(B_i,\prod_{i \in I} B_i\right)
	}
	\DefineNamedFunc{algebraProductEmbedding}{i}{\theta_i}
	{
		\Lambda b \in B_i \. \Lambda j \in I \.
		\If j == i \Then b \Else 0 	
	}
	\\
	\Theorem{AlgebraProductPartitionOfUnity}
	{
		\NewLine ::		
		\forall I \in \SET \.
		\forall B : I \to \BOOL \.
		\PoU\left( \prod_{i \in I} B_i, \{\theta_i(e_{B_i}) | i \in I \} \right)
	}
	\NoProof
}\Page{
	\Theorem{ProductStructureByFinitePartitionOfUnity}
	{	
		\forall B \in \BOOL \.
		\forall P : \PoU(B) \And \TYPE{Finite} \.
		\NewLine	\.
		\TYPE{Isomorphism}
		\left( 
				\BOOL,
				B,
				\prod_{p \in P}	\langle p \rangle,
				\Lambda b \in B \. \Lambda p \in P \. bp	
		\right)
	}
	\Say{\varphi}{\Lambda b \in B \. \Lambda p \in P \. bp }
	{\BOOL\left( B, \prod_{p \in P} \langle p \rangle \right)}
	\Say{[1]}{\Elim PoU(P,A)\Elim \varphi \Intro \ker}{\ker \varphi = \{0\}}
	\Say{[2]}{\THM{ZeroKernelTHM}[1]}
	{
		\Inj\left( B, \prod_{p \in P} \langle P \rangle , \varphi \right)	
	}
	\AssumeIn{t}{\prod_{p \in P} \langle p \rangle}
	\SayIn{b}{\sum_{p \in P} t_p}{B}
	\AssumeIn{p}{P}
	\Conclude{[p.*]}{
		\Elim \varphi \Elim b
		\Elim \RNG(B)
		\THM{PrincipleIdealStructure}(B,p)
		\Elim \TYPE{Disjoint}(B,P)
		\Elim \TYPE{BooleanOrder}	
	}
	{
		\NewLine :		
		\varphi_p(b) =   
		p \sum_{q \in P} t_q = 
		\sum_{q \in P} p t_q = 
		p t_p =
		t_p
	}
	\DeriveConclude{[t.*]}{\Intro(=,\to)}{\varphi(b) = t}
	\Derive{[3]}{\Intro \Surj}
	{\Surj\left(B, \prod_{p \in P} \langle p \rangle, \varphi \right)}
	\Conclude{[*]}{\Intro \TYPE{Isomorphism} [1][2]}
	{\TYPE{Isomorphism}\left(\BOOL,B, \prod_{p \in P} \langle p \rangle, \varphi \right)}
	\EndProof
}\Page{
	\Theorem{ProductStructureByCountablePartitionOfUnity}
	{	
		\NewLine ::		
		\forall B  : \SA \.
		\forall P : \PoU(B) \And \TYPE{Countable} \.
		\NewLine	\.
		\TYPE{Isomorphism}
		\left( 
				\BOOL,
				B,
				\prod_{p \in P}	\langle p \rangle,
				\Lambda b \in B \. \Lambda p \in P \. bp	
		\right)
	}
	\Say{\varphi}{\Lambda b \in B \. \Lambda p \in P \. bp }
	{\BOOL\left( B, \prod_{p \in P} \langle p \rangle \right)}
	\Say{[1]}{\Elim PoU(P,A)\Elim \varphi \Intro \ker}{\ker \varphi = \{0\}}
	\Say{[2]}{\THM{ZeroKernelTHM}[1]}
	{
		\Inj\left( B, \prod_{p \in P} \langle P \rangle , \varphi \right)	
	}
	\AssumeIn{t}{\prod_{p \in P} \langle p \rangle}
	\SayIn{b}{\sup_{p \in P} t_p}{B}
	\AssumeIn{p}{P}
	\Conclude{[p.*]}{
		\Elim \varphi \Elim b
		\Elim \SA(B)
		\THM{PrincipleIdealStructure}(B,p)
		\Elim \TYPE{Disjoint}(B,P)
		\Elim \TYPE{BooleanOrder}
		\Elim \sup	
	}
	{
		\NewLine :		
		\varphi_p(b) =   
		p \sup_{q \in P} t_q = 
		\sup_{q \in P} p t_q = 
		\sup \{ 0, t_p  \} =
		t_p
	}
	\DeriveConclude{[t.*]}{\Intro(=,\to)}{\varphi(b) = t}
	\Derive{[3]}{\Intro \Surj}
	{\Surj\left(B, \prod_{p \in P} \langle p \rangle, \varphi \right)}
	\Conclude{[*]}{\Intro \TYPE{Isomorphism} [1][2]}
	{\TYPE{Isomorphism}\left(\BOOL,B, \prod_{p \in P} \langle p \rangle, \varphi \right)}
	\EndProof
}\Page{
	\Theorem{ProductStructureByPartitionOfUnity}
	{		
		\forall B  : \TAlgebra \.
		\forall P : \PoU(B) \.
		\NewLine	\.
		\TYPE{Isomorphism}
		\left( 
				\BOOL,
				B,
				\prod_{p \in P}	\langle p \rangle,
				\Lambda b \in B \. \Lambda p \in P \. bp	
		\right)
	}
	\Say{\varphi}{\Lambda b \in B \. \Lambda p \in P \. bp }
	{\BOOL\left( B, \prod_{p \in P} \langle p \rangle \right)}
	\Say{[1]}{\Elim PoU(P,A)\Elim \varphi \Intro \ker}{\ker \varphi = \{0\}}
	\Say{[2]}{\THM{ZeroKernelTHM}[1]}
	{
		\Inj\left( B, \prod_{p \in P} \langle P \rangle , \varphi \right)	
	}
	\AssumeIn{t}{\prod_{p \in P} \langle p \rangle}
	\SayIn{b}{\sup_{p \in P} t_p}{B}
	\AssumeIn{p}{P}
	\Conclude{[p.*]}{
		\Elim \varphi \Elim b
		\Elim \SA(B)
		\THM{PrincipleIdealStructure}(B,p)
		\Elim \TYPE{Disjoint}(B,P)
		\Elim \TYPE{BooleanOrder}
		\Elim \sup	
	}
	{
		\NewLine :		
		\varphi_p(b) =   
		p \sup_{q \in P} t_q = 
		\sup_{q \in P} p t_q = 
		\sup \{ 0, t_p  \} =
		t_p
	}
	\DeriveConclude{[t.*]}{\Intro(=,\to)}{\varphi(b) = t}
	\Derive{[3]}{\Intro \Surj}
	{\Surj\left(B, \prod_{p \in P} \langle p \rangle, \varphi \right)}
	\Conclude{[*]}{\Intro \TYPE{Isomorphism} [1][2]}
	{\TYPE{Isomorphism}\left(\BOOL,B, \prod_{p \in P} \langle p \rangle, \varphi \right)}
	\EndProof
}
\newpage
\subsubsection{Products of Subset Algebras}
\Page{
	\Theorem{SetAlgebraProductRepresentation}
	{
		\forall I \in \SET \.
		\forall X : I \to \SET \.
		\forall A : \prod_{i \in I} \Alg(X_i) \.
		\NewLine \.
		\prod_{i \in I} A_i \cong_\BOOL 
		\left\{
					S \subset \bigsqcup_{i \in I} X_i : 
					\forall i \in I \.
					\Big\{ x | (i,x) \in S   \Big\} \in A_i
		\right\}		
	}
	\SayIn{B}{
		\left\{
					S \subset \bigsqcup_{i \in I} X_i : 
					\forall i \in I \.
					\Big\{ x | (i,x) \in S   \Big\} \in A_i
		\right\}	
	}{\BOOL}
	\Say{\varphi}{\Lambda S \in \prod_{i \in I} A_i \. \bigsqcup_{i \in I} S_i }
	{\TYPE{Isomorphism}\left(\BOOL,\prod_{i\in I} A_i,B \varphi \right)}
	\EndProof
	\\
	\Theorem{SetAlgebraProductFactrorizationRepresentation}
	{
		\NewLine ::		
		\forall I \in \SET \.
		\forall X : I \to \SET \.
		\forall A : \prod_{i \in I} \Alg(X_i) \.
		\forall J : \prod_{i \in I} \Ideal(A_i) \.
		\NewLine \.
		\prod_{i \in I} \frac{A_i}{J_i} \cong_\BOOL
		\frac{ 
		\left\{
					S \subset \bigsqcup_{i \in I} X_i : 
					\forall i \in I \.
					\Big\{ x | (i,x) \in S   \Big\} \in A_i
		\right\} }
		{
			\left\{
					S \subset \bigsqcup_{i \in I} X_i : 
					\forall i \in I \.
					\Big\{ x | (i,x) \in S   \Big\} \in J_i
			\right\}
		}		
	}
	\NoProof
}
\newpage
\subsubsection{Products of Open Domain Algebras}
\Page{
	\Theorem{OpenDomainAlgebraAsProduct}
	{
		\NewLine ::		
		\forall X \in \TOP \.
		\forall \U : \TYPE{Disjoint}\;\T\;X \.
		\Dense\left(X, \bigcup \U \right) \Imply
		\od(X) \cong \prod_{U \in \U} \od(U)
	}
	\Say{[1]}{\Lambda U \in \U \. \THM{HomoOD}\Big(U,X,\iota_U \Big)  }
	{
		\forall U \in \U \. \BOOL\Big( \od(X),\od(U), V \mapsto V \cap U \Big)
	}
	\Say{\Big(f,[2]\Big)}{\THM{ProductUniversalProperty}[1]}
	{
		\NewLine :		
		\sum! f \in \BOOL\left(\od(X), \prod_{U \in \U} \od(U) \right) \. 
		\forall U \in \U \. f\pi_U = \Lambda V \in \od(X) \.  V \cap U
	}
	\Say{[4]}{\Elim \Dense[0]}
	{
		\forall V \in \od(X) \.
		V \cap \bigcup \U = \emptyset
		\iff
		V = \emptyset
	}
	\Say{[5]}{\Elim \bigcup [3]}
	{
		\forall V \in \od(X) \.
		\Big( \forall U \in \U \. U \cap V = \emptyset \Big)
		\iff
		V = \emptyset
	}
	\Say{[6]}{[3][5]\Intro \ker}{\ker f = \{ \emptyset \}}
	\Say{[7]}{\THM{ZeroKernelTM}[6]}{\Inj\left(\od(X),\prod_{U \in \U}\od(U),f\right)}
	\AssumeIn{W}{\prod_{U \in \U} \od(U)}
	\Say{[8]}{\Elim \TYPE{Disjoint}(\U)\Elim W}
	{
		\TYPE{Disjoint}( \T(X), W )	
	}
	\SayIn{W'}{\bigcup_{U \in \U} W_U}{\T(X)}
	\SayIn{H}{\intx \overline{U}}{\od(X)}
	\AssumeIn{U}{\U}
	\Conclude{[U.*]}{
		\Elim H 
		\THM{InteriorSubset}(X,U)\THM{ClosureSubset}(X,U)
		[8] 
		\Elim \od(U) 
	}
	{
			\NewLine :			
			U \cap H =
			U \cap \intx \overline{W'} =
			{\intx}_U \; \overline{U \cap W'} =
			{\intx}_U \; \overline{W_U} =
			W_U
	}
	\Derive{[9]}{\Intro \forall}{\forall U \in \U \. U \cap H = W_U}
	\Conclude{[W.*]}{\Elim H [2][9]}
	{
			f(H) = W  
	}
	\Derive{[8]}{\Intro \Surj}{\Surj\left( \od(X), \prod_{U \in \U} \od(U),f \right)}
	\Conclude{[*]}{\Intro \TYPE{Isomorphism}[7][8]}
	{
		\TYPE{Isomorphism}\left( \BOOL, \od(X),\prod_{U \in \U} \od(U), f \right)
	}
	\EndProof
}
\newpage
\subsubsection{Coproducts}
\Page{
	\DeclareFunc{booleanCorpoduct}
	{
		\prod_{I \in \SET} ( I \to \BOOL ) \to \BOOL
	}
	\DefineNamedFunc{booleanCoproduct}{B}{\bigotimes_{i \in I} B_i}
	{
		\TK \prod_{i \in I} \Z \; B_i
	}
	\\
	\DeclareFunc{booleanCanonicalEmbedding}
	{
		\prod_{I \in \SET} 
		\prod_{B : I \to \BOOL}  
		\prod_{i \in I}  
		\BOOL\left(B_i , \bigotimes_{j \in I} B_j \right) 
	}
	\DefineNamedFunc{booleanCanonicalEmbedding}{b}{\iota_i(b)}
	{
		\pi_i^{-1}\Big( S_{B_i}(b) \Big)
	}
	\\
	\Theorem{CoproductStoneSpace}
	{
		\forall I \in \SET \.
		\forall B : I \to \BOOL \.
		\Z \; \bigotimes_{i \in I} B_i  \cong_\TOP \prod_{i \in \I} \Z\;B_i
	}
	\Say{[1]}{\THM{TychonoffTHM}(I,\Z\;B)}
	{
		\Compact\left(\prod_{i \in I} \Z\;B_i\right)
	}
	\AssumeIn{p}{\prod_{i \in I} \Z \; B_i}
	\Assume{U}{\U(p)}
	\Say{\Big(J,V,[2]\Big)}{\THM{ProductTopologyBase}(I,\Z\;B,p,U)}
	{
		\NewLine :		
		\sum J : \TYPE{Finite}(I) \.
		\sum V : \prod_{j \in J} \T \; \Z \; B_j \.
		p \in \prod_{j \in J} V_j \times \prod_{j \in J^\C} \Z\;B_j \subset U 
	}
	\Say{\Big(W,[3]\Big)}{\Lambda j \in J \. \Elim \TYPE{ZeroDimensional}(\Z\;B_j,V_j)}
	{
		\NewLine :		
		\sum W \in \prod_{j\in J} \Clopen(\Z\;B_j,p_j) \.
		\forall j \in J \. p_j \in W_j \subset V_j
	}
	\Say{H}{\bigcup_{j \in J} \prod_{i \in I} \If j == i \Then W_j \Else \Z\;B_i}
	{
		\Clopen\left( \prod_{i \in I} \Z \; B_i\right)
	}
	\Conclude{[*]}{\Elim H}{ p \in H \subset U }
	\Derive{[2]}{\Intro \dim_\TOP}{ \dim_\TOP \prod_{i \in i} \Z \; B_i = 0}
	\Say{[3]}{\THM{T2Product}(I,\Z\;B)}
	{
		\TYPE{T2}\left( \prod_{i \in I} \Z\;B_i \right)	
	}
	\Say{[4]}{\Intro \TYPE{StoneSpace}[1][2][3]}
	{
		\TYPE{StoneSpace}\left( \prod_{i \in I} \Z\;B_i \right)
	}	
	\Conclude{[*]}{\THM{StoneHomomorphism}[4]\Intro \FUNC{booleanCoproduct}}
	{
		 \prod_{i \in I} \Z\;B_i \cong_\TOP  \Z \bigotimes_{i \in I} B_i
	}
	\EndProof
}
\Page{
	\Theorem{BooleanCoproduct}
	{
		\TYPE{Coproduct}(\BOOL,\FUNC{booleanCoproduct})		
	}
	\AssumeIn{I}{\SET}
	\Assume{B}{I \to \BOOL}
	\AssumeIn{A}{\BOOL}
	\Assume{f}{\prod_{i \in I}\BOOL(B_i,A)}
	\Say{\Big(H,[1]\Big)}
	{
		\THM{ProductUniversalProperty}
		(
			\TOP,
			\Z\;B,
			\Z\;A,
			\Z\;f
		)
	}
	{
		\NewLine :		
		\sum \Z\;A \Arrow{H} \prod_{i=1}\Z\;B_i : \TOP \.
		\forall i \in I \. \Z\;f_i = H \pi_i
	}
	\Say{\varphi}{\THM{CoproductStoneSpace}(I,B)}
	{
		\TYPE{Isomorphism}\left(\TOP, \prod_{i=1} \Z\;B_i,\Z \bigotimes_{i=1} B_i \right)
	}
	\Say{h}{S_{\bigotimes_{i \in I} B_i}\varphi^{-1}H^{-1}S^{-1}_A}
	{\BOOL\left(\bigotimes_{i\in I} B_i,A\right)}
	\AssumeIn{i}{I}	
	\Conclude{[i.*]}{
		\Elim \iota_i \Elim h
		\Elim S
		[1]
		\Elim 
		\Z \; f
	}
	{
		\NewLine :
		\iota_i h =
		S_{B_i} \pi_i^{-1} S_{\bigotimes_{i\in I} B_i} \varphi^{-1}H^{-1}S^{-1}_A =
		S_{B_i} \pi_i^{-1} H^{-1}S^{-1}_A =
		S_{B_i} (\Z \; f_i)^{-1} S^{-1}_A = 
		f_i
	}
	\Derive{[2]}{\Intro \forall}
	{
		\forall i \in I \. 
		\iota_i h = f_i
	}
	\Assume{g}{\BOOL\left(\bigotimes_{i \in I} B_i,  A\right)}
	\Assume{[3]}{\forall i \in I \. \iota_i g = f_i}
	\Say{[4]}{\Z[3]}
	{
		\forall i \in I \.		
		(\Z \; g)\pi_i = \Z \;f_i
	}
	\Say{[5]}{\Elim \exists! [1][4]}{\Z \; g = H}
	\Conclude{[g.*]}{ 
		\THM{StoneHomoAndCCorespondance}\left( 
			\bigotimes_{i \in I} B_i, A  \right) 
		\Intro h   
	}{g = h}
	\DeriveConclude{[I.*]}{\Intro \Imply \Intro \forall}
	{
		\forall  \bigotimes_{i \in I} B_i \Arrow{g} A : \BOOL \.
		(\forall i \in I \. \iota_i g = f_i) 
		\Imply
		g = h 
	}	
	\DeriveConclude{[*]}{\Intro \TYPE{Coproduct}}
	{
		\TYPE{Coproduct}(\FUNC{booleanCoproduct})
	}
	\EndProof
	\\
	\Theorem{CopoductGeneration}
	{
		\forall I \in \SET \.
		\forall B : I \to \BOOL \.
		\bigotimes_{i \in I} B_i = \left\langle \bigcup_{i=1} \iota_i(B_i) \right\rangle_{\RING}
	}
	\Say{[1]}{\Elim \iota \Elim \FUNC{generateSubring}}
	{
		\left\langle \bigcup_{i=1} \iota_i(B_i) \right\rangle_{\RING} \subset 
		\bigotimes_{i \in I} B_i
	}
	\Say{\Big(h,[2]\Big)}{\THM{CoproductUniversalProperty}}
	{
		\sum! \bigotimes_{i \in I} B_i \Arrow{h} 
		\left\langle \bigcup_{i=1} \iota_i(B_i) \right\rangle_{\RING} : \BOOL \.
		\forall i \in \I \. \iota_i h = \iota_i
	}
	\Say{[3]}{\Elim \exists ! [2]}{ h = \id}
	\Conclude{[*]}{\Elim h [1][3]}
	{
		\left\langle \bigcup_{i=1} \iota_i(B_i) \right\rangle_{\RING} = 
		\bigotimes_{i \in I} B_i
	}
	\EndProof
}
\Page{
		\DeclareFunc{coproductBase}
		{
			\prod_{I \in \SET}
			\prod_{B : I \to \BOOL}
			?\bigotimes_{i \in I} B_i
		}
		\DefineNamedFunc{coproductBase}{}{C(I,B)}
		{
			\left\{ 
				b \in  \bigotimes_{i \in I} B_i : 
				\exists J \subset I :
				a : \prod_{j \in J} B_j : 
				b = \inf_{j \in J} \iota_j(a_j)   
			\right\}
		}
		\\
	\Theorem{CoproductBaseExpression}
		{
			\forall I \in \SET \.
			\forall B : I \to \BOOL \.
			\forall b \in \bigotimes_{i \in I} B_i \.
			\exists  S \subset  C(I,B) : 
			b = \sup S
		}
		\Say{\D}{
			\left\{
				P : \PoU\left(\bigotimes_{i \in I} B_i\right) \And
					\TYPE{Finite} :
					P \subset C(I,B)
			\right\}}
		{
			\NewLine :			
			?\left(
				\PoU\left(\bigotimes_{i \in I} B_i\right) \And
				\TYPE{Finite}
			\right)
		}
		\Say{A}{
			\left\{
				b \in  \bigotimes_{i \in I} B_i :
				\exists D \in \D : \exists D' \subset \D  : b = \sup D
			\right\}		
		  }
		  {
		  	?\bigotimes_{i \in I} B_i
		  }
		\Say{[1]}{\Elim A}{e,0 \in A}
		\Say{[2]}{\Elim C(I,B)\Intro (\cdot_{\bigotimes_{i \in I} B_i})}
		{
			\forall x,y \in C(I,B) \. xy \in C(I,B)
		}
		\AssumeIn{x,y}{A}
		\Say{\Big(D,D',[3]\Big)}{\Elim A (x)}
		{
			\sum_{D \in \D} 
			\sum_{D' \subset D}
			x = \sup D'
		}
		\Say{\Big(E,E',[4]\Big)}{\Elim A (y)}
		{
			\sum_{E \in \D} 
			\sum_{E' \subset E}
			y = \sup E'
		}
		\Say{F}{DE}{?\bigotimes_{i \in i} B_i}
		\Say{[5]}{\Elim F[2]}{DE \in \D}
		\Say{\Big[(x,y).*.1\Big]}
		{
			\Elim \PoU\left(\bigotimes_{i \in I} B_i,D\right)[1]
			\Intro x^\c
			\Intro A
		}	
		{x^\c = \sup D \setminus D' \in A}
		\Say{[6]}{\Intro \subset [3][4]\Intro F}{D'E' \subset F}		
		\Conclude{\Big[(x,y).*.2\Big]}
		{
			[5][6] 
			\THM{BooleanRingIsALattice}
			\left( \bigotimes_{i \in I} B_i \right) \Intro xy \Intro A
		}
		{
			xy = \sup D'E' \in A
		}
		\Derive{[3]}{\THM{BooleanSubalgebraCriterion2}}
		{
		 	A \subset_\BOOL \bigotimes_{i \in I} B_i
		}
		\Say{[4]}{\Elim \D \Elim \c}
		{
				\forall i \in I \. \forall b \in B_i \.
				\Big\{ \iota_i(b),\iota_i^\c(b) \Big\} \in \D	
		}
		\Say{[5]}{[4]\Elim A}{
			\forall i \in I \.
			\forall b \in B_i \.
			\iota_i(b) \in A
		}
		\Say{[6]}{\THM{CoptoductGenerates}(I,B)[3][5]}
		{
			A = \bigotimes_{i \in I} B_i
		}
		\Conclude{[*]}{[6]\Elim A \Elim \D}
		{
			\forall b \in \bigotimes_{i \in I} B_i \.
			\exists S \subset C(I,B) \.
			b = \sup S
		}
		\EndProof
		\\
		\Theorem{CoproductBaseIsOrderDense}
		{
			\forall I \in \SET \.
			\forall B : I \to \BOOL \.
			\OD\left(\bigotimes_{i \in I} B_i, C(I,B)  \right)
		}
		\NoProof				
}
\Page{
	\Theorem{CanonicalEmbeddingIsOrderC}
	{
		\forall I \in \SET \.
		\forall B : I \to \BOOL \.
		\forall i \in I \.
		\oC\left( B_i, \bigotimes_{i \in I} B_i \right)
	}
	\Assume{A}{?B_i}
	\Assume{[1]}{\inf A = 0}
	\AssumeIn{p}{\bigotimes_{i \in I} B_i}
	\Assume{[2]}{p \neq 0}
	\Say{\Big(c,[3]\Big)}{\Elim \OD\left( \bigotimes_{i \in I} B_i, C(I,B) \right)(c)}
	{
		\sum c \in C(I,B) \. 0 < c \le p
	}
	\Say{\Big(J,b,[4] \Big)}{\Elim C(I,B,c)}
	{
		\sum J \subset I \. 
		\sum b : \prod_{j \in J} B_j \.
		c = \inf_{j \in j} \iota_j(b_j)
	}
	\Say{J'}{J \cup \{i\}}{?I}
	\Say{b'}{\Lambda j \in J' \. \If j \in J \Then b_j \Else \iota_i(e) }
	{
		\prod_{j \in J'} B_j
	}
	\Say{[5]}{\Elim b'[4]}{\forall j \in J' \. b'_j \neq 0}
	\Say{\Big( a, [6] \Big)}{\Elim b [4][1]}
	{
		\sum a \in B_i \.  a \not \ge b'_i	
	}
	\Say{[7]}{\Elim \TYPE{BooleanOrder}[6]\Intro (\setminus)}
	{b'_i \setminus a \neq \emptyset }
	\SayIn{t}{\Elim \TYPE{NonEmpty}[7]}{b'_i \setminus a}
	\SayIn{\Big(z, [8]\Big)}{\Elim \TYPE{NonEmpty}(c)\Elim t}{ \sum z \in c \. z_i = t}
	\Say{[9]}{[8] \Intro \iota_i}{z \in c \setminus \iota_i(a) }
	\Conclude{[p.*]}{\Intro \TYPE{BooleanOrder}[3][9]}
	{
		p \not \le \iota_i(a)
	}
	\DeriveConclude{[A.*]}{\Intro \inf}{\inf \iota_i(A) = 0}
	\DeriveConclude{[*]}{\Intro \oC}{\oC\left( B_i, \bigotimes_{i \in I} B_i,\iota_i\right)}
	\EndProof
	\\
	\Theorem{ZeroCorproduct}
	{
		\forall I \in \SET \.
		\forall B : I \to \BOOL \.
		(\exists i \in I : B_i = \star)
		\Imply
		\bigotimes_{i \in I} B_i = \star
	}
	\NoProof
	\\
	\Theorem{CanonicalEmbeddingIsInjection}
	{
		\forall I \in \SET \.
		\forall B : I \to \BOOL \.
		\bigotimes_{i \in I} B_i \neq \star
		\Imply
		\NewLine 
		\Imply		
		\forall i \in I \. 
		\Inj\left( B_i, \bigotimes_{i \in I} B_i, \iota_i \right)
	}
	\NoProof
}
\Page{
	\Theorem{CoproductNonzeroInf}
	{
		\forall I \in \SET \.
		\forall B : I \to \BOOL \.
		\bigotimes_{i \in I} B_i \neq \star
		\Imply
		\NewLine
		\Imply
		\left(		
			\forall J : \TYPE{Finite}(I) \.
			\forall a : \prod_{j \in J} B_j \.
			\forall j \in J \. a_j \neq 0 
			\Imply
			\inf_{j \in J} \iota_j(a_j) \neq 0 
		\right)
	}
	\NoProof
	\\
	\Theorem{CanonicalEmbeddingEquality}
	{
		\forall I \in \SET \.
		\forall B : I \to \BOOL \.
		\bigotimes_{i \in I} B_i \neq \star
		\Imply
		\NewLine
		\Imply
		\bigg(		
			\forall i,j \in I \.
			i \neq j 
			\Imply
			\Big(
				\forall a \in B_i \.
				\forall b \in B_j \. 
				\iota_i(a) = \iota_j(b) 
				\Imply
				\big(
					a = e \And b = e
					\big|
					a = 0 \And b = 0 
				\big)
			\Big)
		\bigg)
	}
	\NoProof
	\\
	\Theorem{CoproductPartition}
	{
		\forall I \in \SET \.
		\forall B : I \to \BOOL \.
		\forall \I : \TYPE{Partition}(I) \.
		\bigotimes_{i \in I} B_i 
		\cong_\BOOL \bigotimes_{J \in \I} \bigotimes_{j \in J} B_j 
	}
	\NoProof
}
\newpage
\subsubsection{Coproducts of Subset Algebras}
\Page{
	\Theorem{SetAlgebraCoproductRepresentation}
	{
		\forall I \in \SET \.
		\forall X : I \to \SET \.
		\forall A : \prod_{i \in I} \Alg(X_i) \.
		\NewLine
		\bigotimes_{i \in I} A_i \cong_\BOOL
		\left\langle
			\left\{
				\left\{
						x \in \prod_{j \in I} X_j : x_i \in S
				\right\} \Bigg| i \in I, S \in A_i
			\right\}
		\right\rangle_{\RING}
	}
	\SayIn{B}{
		\left\langle
			\left\{
				\left\{
						x \in \prod_{j \in I} X_j : x_i \in S
				\right\} \Bigg| i \in I, S \in A_i
			\right\}
		\right\rangle_{\RING}
	}{\BOOL}	
	\Say{f}{
		\Lambda i \in I \. 
		\Lambda S \in A_i \. 
		\left\{
						x \in \prod_{j \in I} X_j : x_i \in S
		\right\}  
	}{ \prod_{i \in I} \BOOL(A_i,B)  }
	\Say{\Big(h, [1] \Big)}
	{
		\THM{CoproductUniversalProperty}(I,A,B,f)
	}
	{
		\sum h \in \BOOL\left( \bigotimes_{i \in I} A_i, B\right)
	}
	\Say{[2]}{ \Elim f [1]\Elim B }{\Surj\Act{\bigotimes_{i \in I} A_i,B,h}}
	\AssumeIn{p}{\bigotimes_{i \in I} A_i}
	\Assume{[3]}{p \neq 0}	
	\Say{\Big(c,[4]\Big)}{\Elim \OD\Act{\bigotimes_{i \in I} A_i, C(I,A)}[3]}
	{
		\sum c \in C(I,A) \. 0 < c \le p
	}
	\Say{\Big(J,S,[5]\Big)}{\Elim C(I,A,c)}
	{
		\sum J  : \TYPE{Finite}(I) \. 
		\sum S : \prod_{j \in J} A_j \. 
		c = \inf_{j \in J} \iota_j(S_j)
	}
	\SayIn{t}{\prod_{j \in J} \iota_j(S_j)}{\bigotimes_{i \in I} B_i}
	\Say{[6]}{\Elim \inf [4][5]}{\forall j \in J \. S_j \neq \emptyset}
	\Say{[7]}{\THM{BooleanRingIsALattice}\Elim t[5]}{
			t \le c
	}
	\Say{[8]}{
		\Elim t
		\Elim \BOOL\Act{\bigotimes_{i \in I}A_i,B}
		[1]
		\Elim f
		[6]	
	}
	{
	    \NewLine :		
		h(t)  =
		h\Act{ \prod_{j \in J} \iota_j(S_j)} =
		\prod_{j \in J} \iota_j h (S_j)  =
		\prod_{j \in J} f_j(S_j) = 
		\left\{ x \in \prod_{i \in I} X_i : \forall j \in J \.  x_j \in S_j \right\}
		\neq \emptyset
	}
	\Conclude{[p.*]}{\THM{BooleanMorphismIsMonotonic}[4][7][8]}
	{
		f(p) \neq 0	
	}
	\Derive{[3]}{\Intro \ker h}{\ker h = \{0\}}	
	\Say{[4]}{\THM{ZeroKernelTHM}[3]}
	{
		\Inj\Act{\bigotimes_{i \in I} A_i,B,h}
	}
	\Conclude{[*]}{\Intro \TYPE{Isomorphism}[2][4]}
	{
		\TYPE{Isomorphism}\Act{\BOOL,\bigotimes_{i \in I} A_i,B,h}
	}
	\EndProof
}\Page{
	\Theorem{SetAlgebraCoproductFactorizationRepresentation}
	{
		\NewLine ::		
		\forall I \in \SET \.
		\forall X : I \to \SET \.
		\forall A : \prod_{i \in I} \Alg(X_i) \.
		\forall J : \prod_{i \in I} \Ideal(A_i)
		\NewLine
		\bigotimes_{i \in I} \frac{A_i}{J_i} \cong_\BOOL
		\frac{
		\left\langle
			\left\{
				\left\{
						x \in \prod_{j \in I} X_j : x_i \in S
				\right\} \Bigg| i \in I, S \in A_i
			\right\}
		\right\rangle_{\RING}
		}
		{
		\left\langle
			\left\{
				\left\{
						x \in \prod_{j \in I} X_j : x_i \in S
				\right\} \Bigg| i \in I, S \in J_i
			\right\}
		\right\rangle_\I
		}
	}
	\NoProof
}
\newpage
\subsubsection{Tensors}
\Page{
	\DeclareFunc{tensor}
	{
		\prod_{I \in \SET} \prod_{B : I \to \BOOL}
		\prod J : \TYPE{Finite}(I) \.
		\prod_{j \in J} B_j \to \bigotimes_{i \in I} B_j
	}
	\DefineNamedFunc{tensor}{b}{\bigotimes_{j \in J} b_j}
	{ \prod_{j \in J} \iota_j(b_j)}
	\\
	\Theorem{TensorDistributivity}
	{
		\forall I \in \SET \.
		\forall B : I \to \BOOL \.
		\forall J : \TYPE{Finite}(B)
		\forall n \in \Nat \.
		\forall b : \prod_{j \in J} B^n_j \. \NewLine \.
		\bigotimes_{j\in J} \prod^n_{k=1} b_{j,k} =
		\prod^n_{k=1} \bigotimes_{j \in J} b_{j,k}
	}
	\NoProof
	\\
	\Theorem{ZeroTensor}
	{
		\forall I \in \SET \.
		\forall B : I \to \BOOL \.
		\forall J : \TYPE{Finite}(B) \.
		\forall b : \prod_{j \in J} B_j \.
		\NewLine \.		
		\bigotimes_{j \in J} b_j = 0
		\iff
		\exists j \in J : b_j = 0
	}
	\NoProof
}\Page{
	\Theorem{TensorApproximation}
	{
		\forall A,B \in \BOOL \.
		\forall p \in A \otimes B \.
		\NewLine \.
		\exists \A : \TYPE{Finite} \And \PoU(A) :
		\exists b : \A \to B :
		p = \sup_{a \in \A} a \otimes b_a
	}
	\Say{C}
	{
		\{
			p \in A \otimes B : 
			\exists \A : \TYPE{Finite} \And \PoU(A) :
			\exists b : \A \to B :
			p = \sup_{a \in \A} a \otimes b_a
		\}
	}
	{
		?(A \otimes B)
	}
	\Say{[1]}{\Elim \sup \Elim C}{e = \sup_{a \in \{e_A\}} a \otimes e_B \in C}
	\AssumeIn{c,c'}{C}
	\Say{\Big(\A,b,[2]\Big)}{\Elim C (c)}
	{
			\sum \A : \Finite \And \PoU(A) \.
			\sum b : \A \to B \.
			c = \sup_{a \in \A} a \otimes b_a	
	}
	\Say{\Big(\A',b',[3]\Big)}{\Elim C (c')}
	{
			\sum \A' : \Finite \And \PoU(A) \.
			\sum b' : \A' \to B \.
			c' = \sup_{a \in \A'} a \otimes b'_a	
	}
	\Say{[4]}{\Elim \PoU(A,\A)\Elim \PoU(A,\A)\Intro \PoU}
	{
		\NewLine :		
		\PoU(A,\A\A')
	}
	\Say{b''}{\Lambda aa' \in \A\A' \. b_ab'_{a'}}{\A\A' \to B}
	\Conclude{\Big[(c,c').*\Big]}
	{
		[2][3]
		\THM{BooleanRingIsALattice}(A \otimes B)
		\THM{TensorDistribuibity}(A,B)
		\Intro b''	
		\Elim C
	}
	{
		\NewLine :		
		cc' =
		(\sup_{a \in \A} a \otimes b_a)(\sup_{a \in \A'} a \otimes b'_a) =
		\sup_{a \in \A} \sup_{a' \in \A'}  (a \otimes b_a)(a' \otimes b'_{a'}) =
		\sup_{a \in \A} \sup_{a' \in \A'}  aa' \otimes b_a b'_{a'} =
		\sup_{a \in \A\A'} a \otimes b''_a 
		\in C
	}
	\Derive{[2]}{\Intro \forall}{\forall c,c' \in C \. cc' \in C}
	\AssumeIn{c}{C}
	\Say{\Big(\A,b,[3]\Big)}{\Elim C (c)}
	{
			\sum \A : \Finite \And \PoU(A) \.
			\sum b : \A \to B \.
			c = \sup_{a \in \A} a \otimes b_a	
	}
	\Say{[4]}
	{
		[3]	
		\THM{BooleanRingIsALattice}(A \otimes B)
		\THM{TensorDistribuibity}(A,B)
		\Elim \PoU(A,\A)
		\NewLine
		\Elim \c \Elim \BOOL(A) 
		\Elim \otimes
		\Elim \sup
	}
	{
		c(\sup_{a \in \A} a \otimes b^\c_a) =
		(\sup_{a \in \A} a \otimes b^\c_a)(\sup_{a \in \A} a \otimes b_a) \le
		\sup_{a \in \A} \sup_{a' \in \A}  (a \otimes b^\c_a)(a' \otimes b^\c_{a'}) 
		= \NewLine =
		\sup_{a \in \A} \sup_{a' \in \A} 	aa' \otimes b^\c_a b_{a'} = 
		\sup_{a \in \A}  a^2 \otimes b^\c_a b_a =
		\sup_{a \in \A}  a \otimes 0 =
		\sup_{a \in \A} 0 =
		0
	}	
	\Say{[5]}{\THM{LatticeMainimalElement}(A \otimes B)[4]}
	{
		c(\sup_{a \in \A} a \otimes b^\c_a)
	}
	\Say{[6]}{
		[3]
		\THM{BooleanRingIsALattice}(A \otimes B)
		\Elim \otimes 
		\Lambda a \in \A \. \THM{ComplementSum}(B,b_a)
		\Elim \PoU(\A)
	}
	{
		\NewLine :		
		c + (\sup_{a \in \A} a \otimes b^\c_a) =
		(\sup_{a \in \A} a \otimes b_a) + (\sup_{a \in \A} a \otimes b^\c_a) \ge 
		\sup_{a \in \A}  a \otimes b_a + a \otimes b^\c_a = 
		\sup_{a \in A} a \otimes ( b_a + b^\c_a) =
		\sup_{a \in A} a \otimes e = 
		e
	}
	\Say{[7]}{\THM{LatticeMaximalElement}(A \otimes B)[6]}
	{
		c + (\sup_{a \in \A} a \otimes b^\c_a) = e
	}
	\Conclude{[c.*]}{\THM{LawOfExcludedMiddle}\Elim C}
	{
		c^\c =  \sup_{a \in \A} a \otimes b^\c_a \in C
	}
	\Derive{[3]}{\Intro \forall}{\forall c \in C \. c^\c \in C}
	\Say{[4]}{\THM{BooleanSubalgebraCriterion2}[1][2][3]}{C \subset_\BOOL A \otimes B}
	\Say{[5]}
	{
		\Lambda a \in A \. 
		\Elim \sup \
		\Elim \otimes 
		\THM{LatticeMinimalElement}(A \otimes B) 
		\Elim C
	}
	{
		\forall a \in A \. a \otimes e = \sup \{ a \otimes e, a^\c \otimes 0 \} \in C
	}
	\Say{[6]}{\Lambda b \in B \. \Elim \sup \Elim C}
	{
		e \otimes b = \sup_{a \in \{e\}} a \otimes b \in C
	}
	\Say{[7]}{\THM{CoproductGeneration}(A,B)[4][5][6]}
	{
		B = C
	}
	\Conclude{[*]}{\Elim C [7]}
	{
		\forall p \in A \otimes B \.
		\exists \A : \TYPE{Finite} \And \PoU(A) :
		\exists b : \A \to B :
		p = \sup_{a \in \A} a \otimes b_a
	}
	\EndProof
	\\
	\Theorem{TensorBound}
	{
		\forall A,B \in \BOOL \.
		\forall p \in A \otimes B \.
		p \neq 0 
		\Imply
		\left(
			\exists a \in A  :
			\exists b \in B :
			a \neq 0 \And b \neq 0 \And a \otimes b \le p
		\right)
	}
	\NoProof
}
\Page{
	\Theorem{CoproductOfPowerSetsIsIncomplete}
	{
		\neg \SA(?\Nat \otimes ?\Nat)
	}
	\Say{A}{\Big\{\{n\}\otimes\{n\} \Big| n \in \Nat   \Big\}}
	{
		?(?\Nat \otimes ?\Nat)
	}
	\AssumeIn{c}{?\Nat \otimes ?\Nat}	
	\Assume{[1]}{c = \inf A}
	\Say{\Big(k,a,b,[2]\Big)}{\THM{CoproductBasrExpression}(?\Nat,?\Nat,c)}
	{
		\sum_{k=1}^\infty  
		\sum_{a,b : [1,\ldots,k] \to ?\Nat} 
		c = \sup_{1\le i \le k} a_i \otimes b_i
	}
	\Say{\Big(m,n,[3]\Big)}{\THM{PigionholePrinciple} [1][2]}
	{
		\sum^\infty_{m,n=1} m \neq n \And
		\Big\{ i \in [1,\ldots,k] : m \in a_i \Big\} =
		\Big\{ i \in [1,\ldots,k] : n \in a_i \Big\}
	}
	\Say{[4]}{\Elim A [1] \Elim \inf}
	{
		\{n\} \otimes \{n\} \le c
	}
	\Say{\Big(j,[5]\Big)}{\THM{TensorDistribuivity}{?\Nat,?\Nat}[4]}
	{
		\NewLine :		
		\sum j \in [1,\ldots,k] \.
		\Big(a_j \cap \{n\}\Big) \otimes \Big(b_j \cap \{n\}\Big) =
		\Big(\{n\} \cap \{n\}\Big) \otimes ( a_j \otimes b_j ) \neq 0
	}
	\Say{[6]}{\THM{ZeroTensor}[4]}{ n \in a_j \And n \in b_j }
	\Say{[7]}{[6][3]}{m \in a_j}
	\Say{[8]}{
		\THM{TensorDistirbutivity}(?\Nat,?\Nat)
		[6][7]
		\THM{ZeroTensor}(?\Nat,?\Nat)	
	}
	{
		\NewLine :		
		(a_j \otimes b_j)\Big(\{m\} \otimes \{n\}\Big) = 
		\Big( a_j \cap \{m\} \Big) \otimes \Big( b_j \cap \{n\} \Big)
		\neq 0
	}
	\Say{[9]}{\Elim \sup [2][8]}
	{
		c\Big(\{m\} \otimes \{n\}\Big) \neq 0
	}
	\Say{[10]}{\Elim \inf [1]\Elim A}
	{
		c\Big(\{m\} \otimes \{n\}\Big) = 0
	}
	\Conclude{[c.*]}{\Intro \bot [9][10]}{\bot}
	\DeriveConclude{[*]}{\Intro \SA}
	{
		\neg \SA(?\Nat \otimes ?\Nat)
	}
	\EndProof
}
\newpage
\subsubsection{General Limits}
\Page{
	\Theorem{BooleanAlgebrasIsBicomplete}
	{
		\TYPE{Bicomplete}(\BOOL)
	}
	\AssumeIn{I}{\SET}
	\Assume{B}{I \to \BOOL}
	\Assume{R}{?I^2}
	\Assume{f}{\prod_{(i,j)\in R} \BOOL(B_i,B_j)}
	\SayIn{L}{
		\left\{
			x \in \prod_{i \in I} B_i :
			\forall (i,j) \in R \.
			x_j = f_{i,j}(x_i) 
		\right\}
	}{\BOOL}
	\Say{[0]}
	{
		\Elim L
	}
	{
		\forall (i,j) \in R \.
		\pi_j = \pi_i f_{i,j}
	}
	\AssumeIn{A}{\BOOL}
	\Assume{g}{\prod_{i \in I} \BOOL(A,B_i)}
	\Assume{[1]}{\forall (i,j) \in R \. g_i f_{i,j} = g_j}
	\Say{\Big(h,[A.*.1]\Big)}{\THM{ProductUniversalProperty}(\BOOL,I,B,A,g)}
	{
		\sum! A \Arrow{h} \prod_{i \in I} B_i : \BOOL \.
		\forall i \in I \. h\pi_i = g_i
	}
	\Say{[2]}{[A.*.1][1][A.*.1]}
	{
		\forall (i,j) \in R \.
		h \pi_j =  g_j = g_i f_{i,j} = h \pi_i f_{i,j}
	}
	\Conclude{[A.*.2]}{\Intro L [2]}{ \im h \subset L }
	\DeriveConclude{[I.*.2]}{\Intro \lim [0]}
	{ (L,\pi) = \lim (\BOOL,B,f )}	
	\Derive{[1]}{\Intro \Complete}{\Complete(\BOOL)}
	\AssumeIn{I}{\SET}
	\Assume{B}{I \to \BOOL}
	\Assume{R}{?I^2}
	\Assume{f}{\prod_{(i,j)\in R} \BOOL(B_i,B_j)}
	\Say{J}{ 
		\bigg\langle 
			\Big\{ 
				\iota_i(b) + f_{i,j}\iota_j(b)
				\Big|
				(i,j) \in R , b \in B_i   
			\Big\} 
		\bigg\rangle_\I 
	}
	{
		\Ideal\Act{\bigotimes_{i \in I} B_i}
	}
	\SayIn{C}{\frac{\bigotimes_{i \in I} B_i}{J}}{\BOOL}
	\Say{\iota'}{\Lambda i \in I \. \iota_i \pi_J}{ \prod_{i \in i} \BOOL(B_i,C)  }
	\Say{[2]}{\Elim C \Elim \iota'}{\forall (i,j) \in R \. f_{i,j}\iota_j' = \iota_i'}
	\AssumeIn{A}{\BOOL}
	\Assume{g}{\prod_{i \in I} \BOOL(B_i,A)}
	\Assume{[3]}{\forall (i,j) \in R \.  f_{i,j} g_j = g_i}
	\Say{\Big(h,[4]\Big)}{\THM{CoproductUniversalProperty}(\BOOL,I,B,A,g)}
	{
		\NewLine :		
		\sum! \bigotimes_{i \in I} B_i \Arrow{h} A : \BOOL \.
		\forall i \in I \. \iota_i h = g_i
	}
	\Say{\Big(\hat h,[5]\Big)}{
		\THM{QuotientMorphis}\Act{
			\bigotimes_{i \in i} B_i,A,J,h}[3]
		}
		{
			\sum!  C \Arrow{\hat h} A : \BOOL \.
			\pi_J \hat h = h
		}
	\Conclude{[A.*.1]}
	{
		\Lambda i \in I \.
		\Elim \iota'_i 
		[5]
		[4.i]
	}
	{
		\forall i \in I \. 
		\iota'_i h' = 
		\iota_i \pi_J h' =
		\iota_i h =
		g_i
	}
}
\Page{
	\DeriveConclude{[I.*]}{\Intro \colim [2]}
	{
		(C,\iota') = \colim (\BOOL,B,f)
	}
	\Derive{[2]}{\Intro \TYPE{Comomplete}}
	{
		\TYPE{Cocomple}(\BOOL)
	}
	\Conclude{[*]}{\Intro \TYPE{Bicomplete}[1][2]}
	{
		\TYPE{Bicomplete}(\BOOL)
	}
	\EndProof
}
\newpage
\subsection{Further Properties}
\subsubsection{Countable Chain Condition}
\Page{
	\DeclareType{\CCC}{?\BOOL}
	\DefineType{A}{\CCC}{\forall D : \PD(A) \. |D| \le \aleph_0}
	\\
	\DeclareType{\SP}{?\TOP}
	\DefineType{X}{\SP}{\forall D : \TYPE{Disjoint}\;\T\;X \. |D| \le \aleph_0}
	\\
	\Theorem{StoneCCCTHM}
	{
		\forall A \in \BOOL \.
		\CCC(A) \iff \SP(\Z\;A)
	}
	\NoProof
	\\
	\DeclareType{CountablySaturatedIdeal}
	{
		\prod_{A : \BOOL} ?\Ideal(A)
	}
	\DefineNamedType{I}{CountablySaturatedIdeal}{\CSI}
	{
		\NewLine		
		\forall D : \PD(A) \. D \subset A\setminus I 
		\Imply |D| \le \aleph_0
	}
	\\
	\Theorem{CCCQuotient}
	{
		\forall A : \SA \.
		\forall I : \SIdeal(A) \.
		\NewLine \.		
		\CCC\Act{\frac{A}{I}}
		\iff
		\CSI(A,I)
	}
	\Assume{[1]}{\CCC\Act{\frac{A}{I}}}
	\Assume{D}{\PD(A)}
	\Assume{[2]}{D \subset A \setminus I}
	\AssumeIn{a,b}{D}
	\Assume{[3]}{a \neq b}
	\Assume{[4]}{\pi_I(a) = \pi_I(b)}
	\Say{\Big(i,[5]\Big)}{\Elim \pi_I[4]}{\sim i \in I \. b = a + i}
	\Say{[6]}{\THM{IdealContainsZero}\Elim \setminus [2] (a\And b)}
	{
		a \neq 0 \And b \neq 0
	}
	\Say{[7]}{
		[5]\Elim \BOOL(A,a) \Elim \Ideal(I,A)[6][2]	
	}{ab = a(a + i) = a^2 + ia = a + ia \neq 0}	
	\Say{[8]}{\Elim \PD(A,D,a,b)[3]}{ab = 0}
	\Conclude{\Big[(a,b).*\Big)}{\Intro \bot[7][8]}{\bot}
	\Derive{[3]}{\Elim \bot \Intro \Imply \Intro \forall}
	{
		\forall a,b \in D \. a \neq b \Imply \pi_I(a) \neq \pi_I(b)	
	}
	\Say{[4]}{\Elim \CCC[1]\Big(\pi_I(D)\Big)}{\Big| \pi_I(D) \Big| \le \aleph_0}
	\Say{[5]}{\Intro \Inj [3]}{\Inj\left(D,\frac{A}{I}, \pi_{I|D} \right)}
	\Conclude{[D.*]}{\THM{InjectivePreservesCard}[4][5]}{
		|D| \le \aleph_0	
	}
	\DeriveConclude{[1.*]}{\Intro \CSI}{\CSI(A,I)}
	\Say{[1]}{\Intro\Imply}{\CCC\Act{\frac{A}{I}} \Imply \CSI(A,I)}
}\Page{
	\Assume{[2]}{\CSI(A,I)}
	\Assume{[3]}{\neg \CCC\left(\frac{A}{I}\right)}
	\Say{\Big(D,[4]\Big)}{\Elim \CCC [3]}
	{
		\sum D : \PD\left( \frac{A}{I} \right) \.
		|D| > \aleph_0
	}
	\Say{(\kappa,d)}{\THM{WellOrderingEnumeration}(D)}{
		\sum \kappa \in \ORD \.
		 \TYPE{Bijection}(\kappa,D)
	}
	\Say{[5]}{
		\THM{BijectionPreservesCardinality}(\kappa,D,d)
		[4]
		\THM{OrdinalityCardianlityBound}
		}
	{
		\omega_1 \le \kappa
	}
	\Assume{\xi}{\omega_1}
	\Say{\Big(a,[5]\Big)}
	{
		\Lambda \xi \in \omega_1 \.		
		\Elim \TYPE{Surjective}\left(A,\frac{A}{I}, \pi_I,d_\xi\right)
	}
	{		
		\sum_{ a : \omega_1 \to A}
		\forall \xi \in \omega_1 \. 
		d_\xi = \pi_I(a_\xi)
	}
	\Say{[6]}{\Elim a \Elim \PD(D)}{\forall \xi \in \omega_1 \. a_\xi \not \in I}
	\AssumeIn{\xi}{\omega_1}
	\AssumeIn{\eta}{\xi}
	\Say{[7]}{\Elim \ORD(\xi,\eta)}{\eta \neq \xi}
	\Say{[8]}{\Elim \PD(D,d_\xi,d_\eta)}{d_\xi d_\eta =0}
	\Conclude{[\eta.*]}{[5][8]}{a_\xi a_\eta \in I}
	\Derive{[7]}{\Intro \forall}
	{
		\forall \eta \in \xi \. a_\xi a_\eta \in I
	}
	\SayIn{v_\xi}{\sup_{\eta \in \xi} a_\xi a_eta}{I}
	\Conclude{c_\xi}{a_\xi \setminus v_\xi}{A \setminus I}
	\Derive{(v,c)}{\Intro \to}{\omega_1 \to I \times (A \setminus I) }
	\AssumeIn{\xi}{\omega_1}
	\AssumeIn{\eta}{\xi}
	\Conclude{[\xi/*]}{\Elim c_\eta \Intro v_\xi}{
		c_\xi c_\eta \le c_\xi a_\eta \le c_\xi v_\xi = 0 	
	}
	\Derive{[8]}{\Intro \PD}{ \PD(A,\im c)}
	\Conclude{[3.*]}{\Elim \CSI(A,I)[8]}{\bot}
	\DeriveConclude{[2.*]}{\Elim \bot}{\CCC\Act{\frac{A}{I}}}
	\DeriveConclude{[*]}{\Intro \Imply \Intro \iff}
	{
		\CCC\Act{\frac{A}{I}} \iff \CSI(A,I)
	}
	\EndProof
}\Page{
	\Theorem{CCCSupInfAnalogy}
	{
		\forall A : \CCC \.
		\forall X \subset A \.
		\exists Y : \TYPE{Countable}(X) \.
		\NewLine \.
		\mathrm{ub} \; Y = \mathrm{ub} \; X  \And \mathrm{lb} \; Y = \mathrm{lb} \; X
	}
	\Say{C}{\bigcup_{x \in X} \{a \in A : a \le x \}}{?A}	
	\Say{\D}{\{ D : \PD(X) : D \subset C  \}}{?\PD(X)}
	\Say{\Big(D,[1]\Big)}{\THM{ZornLemma}(\D,\subset)}
	{
		\sum D \in \D \. D = \max \D
	}
	\Say{[2]}{\Elim \CCC(A,D) }{|D| \le \aleph_0}
	\Say{\Big(x,[3]\Big)}
	{
		\Lambda d \in D \. \Elim C(d)  
	}
	{
		\sum_{x : D \to X} \forall d \in D \. d \le x_d 
	}
	\Say{Y_0}{\im x}{?X}
	\Say{[4]}{\Elim Y_0 \THM{SurjectiveCardinalityBound}[2]}
	{
		|Y_0| \le \aleph_0
	}
	\AssumeIn{a}{A}
	\Assume{[5]}{Y_0 \le a}
	\Assume{[6]}{X \not \le a}
	\Say{\Big( x', [7]\Big)}{\Elim (\not \le) [6]}
	{
		\sum x' \in X \. x' \not \le a
	}
	\Say{a'}{x' \setminus a}{A}
	\Say{[8]}{\Elim a' \Elim \TYPE{BooleanOrder}(A)[7]\Elim(\setminus)}{a' \neq 0}
	\Say{[9]}{\Elim a' \THM{SetminusOrder}(A)}{a' \le x'}
	\Say{[10]}{\Elim C [9]}{a' \in C}
	\Assume{d}{D}
	\Say{[11]}{[5][10]\Elim(\setminus)}{0 = x_d a' }
	\Conclude{[d.*]}{\Elim x [3][11]}{da' = 0}
	\Derive{[12]}{\Intro \forall}{\forall d \in D \. da' = 0}
	\Say{D'}{D \cup \{a'\}}{?C}
	\Say{[13]}{\Elim D' [2] \Intro \PD(A) \Intro D'}
	{
		\PD(A,D')
	}
	\Conclude{[a.*]}{\Elim D' [1][13] \Intro \bot}{\bot}
	\Derive{[5]}{\Elim \bot \Intro \Imply \Intro \forall}
	{
		\forall a \in A \. Y_0 \le a \Imply X \le a	
	}
	\Say{[6]}{\Intro \FUNC{upperBounds} \Intro \TYPE{SetEq}[5]}
	{
		\mathrm{ub}\; X = \mathrm{ub} \; Y_0
	}
	\Say{\Big(Y_1, [7]\Big)}{\LOGIC{ByAnalogyAndDuality}(Y_0)}
	{
		\sum Y_1 \subset X \.  \mathrm{lb}\; X = \mathrm{lb} \; Y_0
	}
	\Say{Y}{Y_1 \cup Y_0}{?X}
	\Conclude{[*]}{\Elim Y_1 [6][7]}
	{
		\mathrm{ub}\; X = \mathrm{ub} \; Y
		\And
		\mathrm{lb}\; X = \mathrm{lb} \; Y
	}
	\EndProof
	\\
	\Theorem{CCCAlgebraUpgrade}
	{
		\forall A : \SA \And \CCC \.
		\TAlgebra(A) 
	}
	\NoProof
	\\
	\Theorem{CCCSOCIsOC}
	{
		\forall A : \CCC \.
		\forall X : \SOC(A) \.
		\NewLine \.
		\OC(A,X)
	}
	\NoProof
}
\Page{
	\Theorem{CCCMonotonicContinuityUpgrade}
	{
		\NewLine \.		
		\forall A : \CCC \.
		\forall P \in \POSET \. \NewLine \.
		\forall A \Arrow{f} P : \POSET \.
		\sC(A,P,f) \Imply \oC(A,P,f)
	}
	\NoProof
	\\
	\Theorem{CCCIffOrdinalCondition}
	{
		\forall A : \SA \.
		\CCC(A)
		\iff
		\NewLine
		\iff
		\{ 
			a : \omega_1 \to A : 
			\forall \eta,\xi \in  \omega_1 \. 
			\eta < \xi \Imply
			a_\eta < a_\xi    
		\} = \emptyset
	}
	\Assume{[1]}{\CCC(A)}
	\Assume{a}{\omega_1 \to A}
	\Assume{[2]}
	{
		\forall \eta, \xi \in \omega_1 \.
		\eta < \xi 
		\Imply
		a_\eta < a_\xi
	}
	\Say{b}{\Lambda \eta \in \omega_1 \. a_{\sigma(\eta)} \setminus a_{\eta}}
	{
		b : \omega_1 \to A
	}
	\Say{B}{\im b}{?A}
	\Say{[3]}{\Elim B [2] \Intro \PD \Intro B }
	{
		\PD(A,B)
	}
	\Say{[4]}{\Elim B [2] \THM{ordinalCardinality}(\omega_1) \Intro B}
	{
		|B| > \aleph_0
	}
	\Conclude{[a.*]}{\Elim \CCC(A)[3][4]}{\bot}
	\DeriveConclude{[1.*]}
	{
		\Elim \bot \Intro \emptyset
	}
	{
		\{ 
			a : \omega_1 \to A : 
			\forall \eta,\xi \in  \omega_1 \. 
			\eta < \xi \Imply
			a_\eta < a_\xi    
		\} = \emptyset
	}
	\Derive{[1]}
	{
		\Intro \Imply
	}
	{
		\CCC(A) \Imply 
		\{ 
			a : \omega_1 \to A : 
			\forall \eta,\xi \in  \omega_1 \. 
			\eta < \xi \Imply
			a_\eta < a_\xi    
		\} = \emptyset
	}
	\Assume{[2]}{
		\{ 
			a : \omega_1 \to A : 
			\forall \eta,\xi \in  \omega_1 \. 
			\eta < \xi \Imply
			a_\eta < a_\xi    
		\} = \emptyset
	}
	\Assume{D}{\PD(A)}
	\Assume{[3]}{|D| > \aleph_0}
	\Say{\Big(D',[4]\Big)}{\THM{FirstUncountableOrdinal}[3]}
	{
		\sum D' \subset D \. |D'| = |\omega_1| \And \forall d \in D' \. d \neq 0
	}
	\Say{d}{\Elim \TYPE{EqCardinality} [4]}
	{
		\Bij(\omega_1,D'',d) 
	}
	\Say{\Big(P,[01]\Big)}{\THM{DijointElementsHavePartitionOfUnity}(D')}
	{
		\sum P : \PoU \. D' \subset P
	}
	\SayIn{a_0}{d_0}{A}
	\Say{[i.0]}{\LOGIC{EmptyTruth}}
	{
		\forall \eta \in 0 \. a_\eta < a_0
	}
	\Say{[j.0]}{\Elim p \Elim \PD(A,P)}
	{
		\forall \eta \in \kappa \. \eta > 0 \Imply a_0 p_\eta = 0 
	}
	\Say{\Big( \kappa, p, [02])}{\THM{ExtendEnumeration}(D',d,P)[01]}
	{
		\sum \kappa \in \ORD \.
		\sum P : \Bij(\kappa,P)
		\omega_1 \le \kappa \.
		p_{|\omega_1} = d
	}	
	\AssumeIn{\xi}{\omega_1}
	\Assume{[5]}{\neg \Limit(\xi)}
	\Say{\Big(\eta,[6]\Big)}{\Elim \Limit [5] }
	{
		\sum \eta \in \omega_1 \. \xi = \sigma(\eta)
	}
	\SayIn{a_\xi}{d_\xi \vee a_\eta}{A}
	\AssumeIn{\zeta}{\xi}
	\Say{[7]}{\Elim \ORD(\xi,\zeta)[6]}
	{
		\zeta \le \eta < \xi
	}
	\Conclude{[\zeta.*]}{
		\Elim a_\xi
		\Elim \PD(A,D) \Elim a_\eta [j.\eta]
		[i.\eta](\zeta)	
	}
	{
		a_\xi  = d_\xi \vee a_\eta > 
		a_\eta \ge 
		a_\zeta
	}
	\Derive{[i.\xi]}{\Intro \forall}
	{
		\forall \zeta \in \xi \. a_\zeta < a_\xi
	}
	\AssumeIn{\zeta}{\kappa}
	\Assume{[7]}{\xi < \zeta}
	\Conclude{[\zeta.*]}{
		\Elim a_\xi
		\Elim \PoU(A,P)[7]
		[j.\eta]	
	}
	{
		p_\zeta a_\xi = 
		p_\zeta (d_\xi \vee a_\eta) =
		p_\zeta d_\xi \vee p_\zeta a_\eta = 
		0 \vee 0 =
		0	 
	}
	\DeriveConclude{[j.\xi]}{\Intro \forall}
	{
		\forall \zeta \in \kappa \.
		\xi < \zeta
		\Imply
		p_\zeta a_\xi = 0
	}
	\Derive{[5]}{\Intro \exists \Intro \Imply}
	{
		\neg \Limit(\xi) \Imply 
		\exists a_\xi \in A :
		\Big(  \forall \zeta \in \xi \. a_\zeta < a_\xi \Big)
		\And
		\Big(  
			\forall \zeta \in \kappa \.
			\xi < \zeta
			\Imply 
			a_\xi p_\zeta = 0 
		\Big)
	}
}
\Page{
	\Assume{[6]}{\Limit(\xi)}
	\SayIn{a_\xi}{ \sup_{\eta \in \xi} d_\eta }{A}
	\AssumeIn{\eta}{\xi}
	\Assume{[7]}{a_\eta \not \le a_\xi}
	\SayIn{b}{a_\eta \setminus a_\xi}{A}
	\Say{[8]}{\Elim b [7] \Intro b}{b \neq 0}
	\Say{\Big( \zeta, [9]\Big)}{}
	{
		\sum \zeta \in \kappa \.
		\zeta > \xi \Imply d_\zeta a_\eta \neq 0 
	}
	\Say{[10]}{[9.1]\Elim(\eta)}{ \zeta > \eta }
	\Say{[11]}{[j.\eta](\zeta)[10]}{ d_\zeta a_\eta = 0 }
	\Conclude{[7.*]}{[10][11]}{\bot}
	\Derive{[7]}{\Elim \bot}{a_\eta \le a_\xi}
	\Say{\Big(\zeta,[8]\Big)}{\Elim \Limit(\xi,\eta)}
	{
		\sum \zeta \in \xi \. \eta < \zeta 
	}
	\Say{[9]}{[j.\eta](\zeta)[8]}{ d_\zeta a_\eta = 0}
	\Conclude{[*.\xi]}{\Elim a_\xi \Elim \sup [7][9]}{a_\eta < a_\xi}
	\DeriveConclude{[i.\xi]}{\Elim \bot \Intro \forall}
	{
		\forall \eta \in \xi \. a_\eta < a_\xi
	}
	\AssumeIn{\eta}{\kappa}
	\AssumeIn{[7]}{\eta > \xi }
	\Say{[8]}{\Elim D' \Elim P}
	{
		\forall \zeta \in \xi \.
		d_\zeta p_\eta = 0
	}
	\Conclude{\eta.*]}{\Elim a_\xi \THM{BooleanAlgebraIsALattice} }
	{
		a_\xi p_\eta = 0
	}
	\DeriveConclude{[j.\xi]}{\Intro \Imply \Intro \forall}
	{
		\forall \eta \in \kappa \.
		\eta > \xi
		\Imply
		a_\xi p_\eta = 0
	}
	\DeriveConclude{[6]}{\Intro \exists \Intro \Imply}
	{
		\neg \Limit(\xi) \Imply 
		\exists a_\xi \in A :
		\Big(  \forall \zeta \in \xi \. a_\zeta < a_\xi \Big)
		\And
		\Big(  
			\forall \zeta \in \kappa \.
			\xi < \zeta
			\Imply 
			a_\xi p_\zeta = 0 
		\Big)
	}
	\Conclude{(\xi,i,j)}{\Elim (|) \LOGIC{LEM}\Big(\Limit(\xi)\Big)[5][6]}
	{
		\sum a_\xi \in A :
		\Big(  \forall \zeta \in \xi \. a_\zeta < a_\xi \Big)
		\And
		\Big(  
			\forall \zeta \in \kappa \.
			\xi < \zeta
			\Imply 
			a_\xi p_\zeta = 0 
		\Big)
	}
	\Derive{\Big(a,[5]\Big)}{\THM{TransfiniteInduction}[i.1][j.1]}
	{
		\NewLine :		
		\prod_{\xi \in \omega_1}
		\sum a_\xi \in A :
		\Big(  \forall \zeta \in \xi \. a_\zeta < a_\xi \Big)
		\And
		\Big(  
			\forall \zeta \in \kappa \.
			\xi < \zeta
			\Imply 
			a_\xi p_\zeta = 0 
		\Big)
	}
	\Conclude{[D.*]}{[5.1][2]}{\bot}
	\DeriveConclude{[2.*]}{\Intro \CCC}{\CCC(A)}
	\DeriveConclude{[*]}{[1]\Intro \iff }
	{
		\CCC(A)
		\iff
		\NewLine :
		\iff
		\{ 
			a : \omega_1 \to A : 
			\forall \eta,\xi \in  \omega_1 \. 
			\eta < \xi \Imply
			a_\eta < a_\xi    
		\} = \emptyset
	}
	\EndProof
	\\
	\Theorem{CCCIdealUpgrade}
	{
		\forall A : \CCC \.
		\forall I : \SIdeal(A) \.
		\TYPE{\tau\hyph Ideal}(A,I)
	}
	\NoProof
	\\
	\Theorem{CCCQuuotineTau}
	{
		\forall A \in \BOOL
		\forall I : \TYPE{\tau\hyph Ideal} \And \CSI(a) \.
		\CCC\left( \frac{A}{I} \right) 
	}
	\NoProof
}\Page{
	\Theorem{CCCQuotient2}
	{
		\forall	A : \CCC \.
		\forall I : \SIdeal(A) \. \NewLine \.
		\CCC\left( \frac{A}{I} \right)
	}
	\Say{[1]}{\Elim \CCC(A) \Intro \CSI}{\CSI(A,I)}
	\Say{[2]}{\THM{CCCIdealUpgrade}(A,I)}{\TYPE{\tau\hyph \Ideal} }
	\Conclude{[*]}{\THM{CCCQuotinentTau}(A,I)[1][2]}{\CCC\left( \frac{A}{I} \right)}
	\EndProof
	\\
	\Theorem{CCCSubalgebra}
	{
		\forall A : \CCC \.
		\forall B \subset_\BOOL A \.
		\CCC(B)
	}
	\NoProof
	\\
	\Theorem{CCCByIdealUpgrade}
	{
		\forall A : \BOOL \.
		\Big( 
			\forall I : \SIdeal(A) \.
			\TYPE{\tau\hyph}\Ideal(A) 
		\Big)
		\Imply
		\CCC(A)
	}
	\Assume{D}{\PD(A)}
	\Assume{[1]}{|D| > \aleph_0}
	\Say{\Big(P,[2]\Big)}{\THM{DisjointElementsHavePartitionOfUnity}}
	{
		\sum P : \PoU(A) \. D \subset P 
	}
	\Say{[3]}{\THM{SupersetStrictCardinality}[1][2]}
	{
		|P| > \aleph_0
	}
	\Say{I}{\langle P \rangle_{\I,\sigma}}{\SIdeal(A)}
	\Say{[4]}{\Elim I \Elim \PoU(A,P)[3]}{e \not \in I}
	\Say{[5]}{[0](I)}{\TYPE{\tau\hyph Ideal}(A,I)}
	\Say{[6]}{\Elim I \Elim \PoU(A,P) \Elim \TYPE{\tau\hyph Ideal}(A,I)[5]}
	{ e \in I  }
	\Conclude{[D.*]}{[4][6]}{\bot}
	\DeriveConclude{{*}}{\Elim \bot \Intro \CCC}{\CCC(A)}
	\EndProof
}\Page{
	\Theorem{CCCBySubalgebraUpgrade}
	{
		\forall A : \BOOL \.
		\Big( 
			\forall B \subset^\sigma_\BOOL A \.
			B \subset^\tau_\BOOL A 
		\Big)
		\NewLine
		\Imply
		\CCC(A)
	}
		\Assume{D}{\PD(A)}
	\Assume{[1]}{|D| > \aleph_0}
	\Say{\Big(P,[2]\Big)}{\THM{DisjointElementsHavePartitionOfUnity}}
	{
		\sum P : \PoU(A) \. D \subset P 
	}
	\Say{[3]}{\THM{SupersetStrictCardinality}[1][2]}
	{
		|P| > \aleph_0
	}
	\Say{\Big(p,[4]\Big)}{\Elim \TYPE{Numerous}(P,0)[3]}{\sum p \in P \. p \neq 0 }
	\Say{P'}{P\setminus \{p\}}{\PD(A)}
	\SayIn{p'}{p^\c}{A}
	\Say{[5]}{\THM{InfiniteSubsetFiniteDifference}\big(P,\{p\}\big)[4]}
	{
		|P'| > \aleph_0
	}
	\AssumeIn{a}{\langle p' \rangle_\I}
	\Say{[6]}{
		\Elim p'\Elim \c 
		\THM{PrincipleIdealExpression}(A,p',a)
		\Elim \TYPE{BooleanOrder}(A)
	}{ap = 0}
	\Say{\Big(q,[7]\Big)}{\Elim \PoU(A,P)(a)}{\sum q \in P \. qp \neq 0}
	\Say{[8]}{\Elim(\#,\to)[6][7]}{p \neq q}
	\Conclude{[a.*]}{\Elim P' [8]}{q \in P'}
	\Derive{[6]}{\Intro \PoU}{\PoU\Big(\langle p' \rangle, P' \Big)}
	\Say{[7]}{\THM{PartitionOfUnitySupremum}[6]}{\sup P' = p'}
	\Say{B}{\sigma(P')}{\TYPE{\sigma \hyph Subalgebra}(A)}
	\Say{[8]}{\Elim B \Elim \PoU(A,P)[3]}{p' \not \in B}
	\Say{[9]}{[0](B)}{\TYPE{\tau\hyph Subalgebra}(A,B)}
	\Say{[10]}{\Elim B \Elim  [7] \Elim \TYPE{\tau\hyph Subalgebra}(A,B)[9]}
	{ p' \in B  }
	\Conclude{[D.*]}{[8][10]}{\bot}
	\DeriveConclude{{*}}{\Elim \bot \Intro \CCC}{\CCC(A)}
	\EndProof
	\\
	\Theorem{CCCByMorphismUpgrade}
	{
		\forall A : \BOOL \.
		\NewLine \.
		\Big( 
			\forall B \in \BOOL
			\forall f : \sC(A,B) \.
			\oC(A,B,f) 
		\Big)
		\Imply
		\NewLine
		\Imply
		\CCC(A)
	}
	\NoProof
	\\
	\Theorem{CCCByBeingSurjectiveImage}
	{
		\forall A : \CCC \.
		\forall B \in \BOOL \. \NewLine \. 
		\forall f : \BOOL \And \Surj \And \oC(A,B) \.
		\CCC(B)
	}
	\Say{[1]}{\THM{IsomorphismTHM}(A,B,f)}{B \cong_\BOOL \frac{A}{\ker f} }
	\Say{[2]}{\THM{TauIdealTHM}(A,B,f)}{\TYPE{\tau\hyph \Ideal}(A,B,f)}
	\Say{[3]}{\Elim \CCC(A) \Intro \CSI}{\CSI(A,\ker f)}
	\Say{[4]}{\THM{CCCQuotientTau}[2][3]}{\CCC\Act{\frac{A}{\ker f}}}
	\Conclude{[*]}{[1][4]}{\CCC(B)}
	\EndProof
}\Page{
	\Theorem{CCCByDenseSubalgebra}
	{
		\forall A \in \BOOL \.
		\forall B \subset_\BOOL A \. \NewLine \.
		\CCC(B) \And \OD(A,B) 
		\Imply 
		\CCC(A)
	}
	\Assume{D}{\PD(A)}
	\Say{D'}{D\setminus \{0\}}{\PD(A)}
	\Say{\Big(b,[1]\Big)}{\Elim \OD(A,B)(D')}
	{
		\sum b : D' \to B \. \forall d \in D' \. 0 < b_d \le d 
	}
	\Say{[2]}{\Elim \PD(A)[1]}{\Inj(D',B,b) \And \PD(B,\im b)}
	\Say{[3]}{\Elim \CCC(B)[2.2]}{|\im b| \le \aleph_0}
	\Say{[4]}{\THM{InjectionReflectsCardinalityBounds}}{|D'| \le \aleph_0}
	\Conclude{[D.*]}{\Elim D' \Elim \aleph_0 [4]}{|D| \le \aleph_0}
	\DeriveConclude{[*]}{\Intro \CCC}{\CCC(A)}
	\EndProof
	\\
	\Theorem{CCCByCoproductStruct}
	{
		\forall A,B \in \BOOL \.
		A \not \cong_\BOOL \star \not \cong_\BOOL B \And \CCC(A\otimes B) 
		\Imply
		\CCC(A) \And \CCC(B)
	}
}
\newpage
\subsubsection{Weakly Distributive Algebras }
\Page{
	\DeclareType{\WD}{?\BOOL}
	\DefineType{A}{\WD}
	{
		\forall X : \Nat \to \TYPE{DownwardsDirected}(A) \.
		\NewLine \.
		\Big( \forall n \in \Nat \. \inf X_n = 0 \Big)
		\Imply
		\inf  \Big\{ b \in A : \forall n \in \Nat \. \exists a \in X_n : a \le b\Big\} = 0	
	}
	\\
	\Theorem{WDPoUProperty}
	{
		\forall A : \WD \.
		\forall P : \Nat \to \PoU(X) \. \NewLine \.
		\exists Q : \PoU(X) :
		\forall n \in \Nat \.
		\forall q \in Q \.
		\Big|\{ p \in P_n  : pq \neq 0 \}\Big| < \infty
	}
	\Say{C}{\Lambda n \in \Nat  \. \Big\{ (\sup F)^\c \Big| F : \TYPE{Finite}(P_n) \Big\}}
	{\Nat \to ?A}
	\Say{[1]}{\Elim C \Lambda n \in \Nat \. \Elim \PoU(A,P_n) }
	{
			\forall n \in \Nat \. \inf C_n = 0
	}
	\Say{B}
	{
		\Big\{ b \in A : \forall n \in \Nat \. \exists a \in C_n : a \le b\Big\}
	}{?A}
	\Say{[2]}{\Elim \WD(A)\Elim B [1] }{\inf B = 0}
	\Say{Q'}{\Big\{a \in A : \exists b \in B : ab = 0 \Big\}}{?A}
	\Say{[3]}{\Elim Q' [2]\Intro \OD}{\OD\Big(A, Q'\Big)}
	\Say{\Big( Q,[4]\Big)}{\THM{OrderDenseContainsPoU}(Q)}
	{
		\sum Q :\PoU( A ) \. Q \subset Q'
	}
	\AssumeIn{n}{\Nat}
	\AssumeIn{q}{Q}
	\Say{\Big(b,[5]\Big)}{\Elim Q(q)[4]}
	{\sum b \in B \. bq =0}
	\Say{\Big(c,[6]\Big)}{\Elim B (n,b)}{\sum c \in C_n \. c \le b}
	\Say{\Big(F,[7]\Big)}{\Elim C_n (c)}{\sum F : \TYPE{Finite}(P_n) \. c = (\sup F)^\C}
	\Say{[8]}{ [5] \Intro \c [6][7]}
	{
		q \le  b^\c \le c^\c \le \sup F 
	}
	\Conclude{[n.*]}
	{\Elim \sup [8] \Elim \PoU(A,P_n)}
	{ \Big|\{ p \in P_n : pq \neq 0  \}\Big| < \infty }
	\DeriveConclude{[*]}{\Intro \forall \Intro \forall}
	{
		\forall n \in \Nat \. 
		\forall q \in Q \. 
		\Big|\{ p \in P_n : pq \neq 0  \}\Big| < \infty
	}
	\EndProof
}\Page{
	\Theorem{WDSupProperty1}
	{
		\forall A : \WD \.
		\forall X : \Nat \to ?\TYPE{UpwardDirected}(A) \. 
		\NewLine \.
		\forall x : \Nat \to A \.
		( \forall n \in \Nat \. x_n = \sup X_n ) 
		\Imply
		\inf \Big\{ 
			x_n \setminus b \Big| 
			n \in \Nat ,
			b \in A :
				\forall m \in \Nat \. 
				\exists a \in X_m : b \le a  
		\Big\} = 0
	}
	\Say{B}{\{ b \in A :\forall m \in \Nat \. \exists a \in X_m : b \le a \} }{?A}
	\Say{D}{
		\Lambda n \in \Nat \. 
		\langle x_n^\c \rangle \cup 
		\bigcup_{a \in X_n} \langle a  \rangle
	}
	{
		\Nat \to ?A
	}
	\Say{[1]}{\Elim D [0] \Intro D \Intro \OD}
	{
		\forall n \in \Nat \. \OD(A,P_n)
	}
	\Say{\Big(P,[2]\Big)}{\Lambda n \in \Nat \. \THM{OrderDenseConatinsPoU}(A,D_n)}
	{
		\NewLine :		
		\sum P : \Nat \to \PoU(A) \. \forall n \in \Nat \. P_n \subset D_n
	}
	\Say{\Big(Q,[3]\Big)}{\THM{WDPoUProperty}(A,P)}
	{	
		\NewLine :		
		\sum Q : \PoU(A) \. \forall n \in \Nat \. \forall q \in Q \.
		\Big|\{ p \in P_n : pq \neq 0  \}\Big| < \infty		
	}
	\Say{C}{\{ x_n \setminus b | n \in \Nat, b \in B \}}{?A}
	\AssumeIn{c}{A}	
	\Assume{[4]}{c \le C}
	\Assume{[5]}{c \neq 0}
	\Say{\Big(q,[6]\Big)}{\Elim \PoU(A,Q)[5]}
	{
		\sum q \in Q \. qc \neq 0	
	}
	\Say{P'}{\Lambda n \in \Nat \. \{p \in P_n : pqc \neq 0 \}}
	{
		\Nat \to ?A
	}
	\Say{[7]}{\Elim P' [3]}{\forall n \in \Nat \. |P'| < \infty}
	\Say{p}{\Lambda n \in \Nat \. \sup P'_n}{\Nat \to A}
	\Say{[8]}{\Elim p \Elim \PoU(P)}{\forall n \in \Nat \. qc \le p_n}
	\Say{[9]}{[0]\Elim P'}
	{
		\forall n \in \Nat \. 
		\forall t \in P' \. 
		\exists a \in X_n : t \le a 
	}
	\Say{[10]}{\Elim \TYPE{UpwardsDirected}(X)[9]\Elim p}{
		\TYPE{UpwardsDirectred}(p)
	}
	\Say{[11]}{\Elim B [8]}{qc \in B}
	\Say{[12]}{\Elim B \Elim C(c)}{ \forall b \in B \. bc = 0}
	\Conclude{[c.*]}{[11][12][6]}{\bot}
	\Derive{[*]}{\Intro \inf}{\inf C =0}
	\EndProof
	\\
	\Theorem{WDSupProperty2}
	{
		\forall A : \WD \.
		\forall X : \Nat \to ?\TYPE{UpwardDirected}(A) \. 
		\NewLine \.
		\forall x : \Nat \to A \.
		\forall y \in A \.
		( \forall n \in \Nat \. x_n = \sup X _n \And y = \inf_{n=1} x_n ) 
		\Imply
		\NewLine
		\Imply
		 \sup \Big\{ 
			b \in A :
				\forall m \in \Nat \. 
				\exists a \in X_m : b \le a  
		\Big\} = y
	}
	\Say{[1]}{\THM{WDSupProperty1}(A,X)}
	{
		\inf \Big\{ 
			x_n \setminus b \Big| 
			n \in \Nat ,
			b \in A :
				\forall m \in \Nat \. 
				\exists a \in X_m : b \le a  
		\Big\} = 0
	}
	\Say{[2]}{\Elim \inf [1][0] \Intro \inf}
	{
		\inf \Big\{ 
			y \setminus b \Big| 
			n \in \Nat ,
			b \in A :
				\forall m \in \Nat \. 
				\exists a \in X_m : b \le a  
		\Big\} = 0
	}
	\Conclude{[*]}{\THM{InfComplementation}[1][2]}
	{
		\sup \Big\{ 
			b \in A :
				\forall m \in \Nat \. 
				\exists a \in X_m : b \le a  
		\Big\} = y
	}
	\EndProof
}\Page{
	\Theorem{WDBySupProperty}{
		\forall A \in \BOOL \. \bigg(
		\forall X : \Nat \to ?\TYPE{UpwardDirected}(A) \. 
		\forall x : \Nat \to A \.
		\forall y \in A \.
		\NewLine \.
		( \forall n \in \Nat \. x_n = \sup X _n \And y = \inf_{n=1} x_n ) 
		\Imply
		 \sup \Big\{ 
			b \in A :
				\forall m \in \Nat \. 
				\exists a \in X_m : b \le a  
		\Big\} = y \bigg)
		\Imply
		\NewLine
		\Imply
		\WD(A)
	}
	\Assume{X}{\Nat \to \TYPE{DownwardsDirected}(A)}
	\Assume{[1]}{\forall n \in \Nat \. \inf X_n = 0}
	\Say{B}{ \{b \in A : \forall n \in \Nat \. \exists a \in X_n : a \le b \} }{?A}
	\Say{Y}{\{x^\c :  x \in X \}}{\Nat \to \TYPE{UpwardDirected}(A)}
	\Say{[2]}{\Elim Y \THM{ComplementInf}[1] \Intro Y }
	{
		\forall n \in \Nat \. \sup Y_n = e	
	}
	\Say{[3]}{\THM{ConstantInf}(A,e)}{\inf e = e}
	\Say{[4]}{[0](Y,e,e)[2][3]\Intro B }
	{
		e = 
		\sup \Big\{ 
			b \in A :
				\forall m \in \Nat \. 
				\exists a \in Y_m : b \le a  
		\Big\}
		=
		\sup \{ b^\c |
			b \in B
		\}
	}
	\Conclude{[X.*]}{\THM{ComplementSup}[4]}
	{
		\inf B = 0
	}
	\DeriveConclude{[*]}{\Intro \WD}
	{
		\WD(A)
	}
	\EndProof
	\\
	\Theorem{NowhereDenseStoneSpaceCondition}
	{
		\forall A \in \BOOL \.
		\forall X \subset \Z\;A \.
		\NewLine \.
		\Big( \exists P : \PoU(A) \. \forall p \in P \. S_A(p) \cap X = \emptyset \Big)
		\iff
		\ND(\Z\;A,X)
	}
	\Assume{[1]}{\ND(\Z\;A,X)}
	\Say{D}{\Big\{a \in A : S_A(a) \cap X = \emptyset \Big\}}{?A}
	\Say{[2]}{\Elim \ND(\Z\;A,X) \Intro D\Intro \OD}
	{
		\OD(A,D)
	}
	\Say{\Big(P, [3]\Big)}{\THM{OrderDenseConatinsPoU}(A,D_n)}
	{
		\sum P : \PoU(A) \.	 P \subset D
	}
	\Conclude{[1.*]}{\Elim D [3]}{\forall p \in P \. S_A(p) \cap X = \emptyset}
	\Derive{[1]}{\Intro \exists \Intro \Imply}
	{
		\ND(\Z\;A,X)
		\Imply
		\Big( \exists P : \PoU(A) \. \forall p \in P \. S_A(p) \cap X = \emptyset \Big)
	}
	\Assume{P}{\PoU(A)}
	\Assume{[2]}{\forall p \in P \. S_A(p) \cap X = \emptyset}
	\Say{[3]}{\Elim \PoU(A,P) \Intro \sup }{\sup P = 1 }
	\SayIn{Y}{\bigcap_{p \in P} S_A(p)}{\T(\Z\;A)}
	\Say{[4]}{\Elim Y \Intro \Dense [3] \Intro Y}{\Dense(\Z\;A,Y)}
	\Say{[5]}{\Elim Y [2]}{ X \subset Y^\c}
	\Conclude{[2.*]}{ \Intro \ND [4][5] }{\ND(\Z\;A,X)}
	\DeriveConclude{[*]}{\Intro \Imply \Intro \iff [1]}
	{
		\Big( \exists P : \PoU(A) \. \forall p \in P \. S_A(p) \cap X = \emptyset \Big)
		\iff
		\NewLine
		\iff
		\ND(\Z\;A,X)
	}
	\EndProof
}\Page{
	\Theorem{WDStoneSpaceCondition}
	{
		\forall A \in \BOOL \. \NewLine \.
		\WD(A)
		\iff
		\forall X \in \Meager(\Z \; A) \. \ND(\Z \; A, X)
	}
	\Assume{[1]}{\WD(A)}
	\Assume{X}{\Meager(\Z\;A)}
	\Say{\Big(N,[2]\Big)}{\Elim \Meager(\Z\;A,X)}
	{
		\sum N : \Nat \to \ND(\Z\;A) \.
		X = \bigcup_{n=1} N_n
	}
	\Say{\Big(P,[3]\Big)}
	{
		\Lambda n \in \Nat \. 
		\THM{NowhereDenseStoneSpaceCondition}(A,N_n)
	}
	{
		\NewLine :		
		\sum P : \Nat \to \PoU(n) \.
		\forall n \in \Nat \.
		\forall p \in P_n \.
		N_n \cap S_A(p) = \emptyset
	}
	\Say{\Big(Q,[4]\Big)}
	{
		\THM{WDPoUProperty}(A,P)
	}
	{
		\sum Q : \PoU(A) \.
		\NewLine \.		
		\forall n \in \Nat \.
		\forall q \in Q \.
		\Big|\{ p \in P_n  : pq \neq 0 \}\Big| < \infty
	}
	\AssumeIn{q}{Q}
	\Assume{[5]}{S_A(q) \cap X \neq \emptyset }
	\SayIn{f}{\Elim \TYPE{NonEmpty}[0]}{S_A(q) \cap X}
	\Say{\Big(n,[6]\Big)}{\Elim[2](f)[5]}
	{
		\sum n \in \Nat \.  f \in N_n
	}
	\Say{[7]}{[3](n)}{\forall p \in P_n \. f \not \in S_A(p)}
	\Say{C}{\{ p \in P_n : pq \neq 0  \} }{?P_n}
	\Say{[8]}{\Elim C [4]}{|C| < \infty}
	\Say{F}{\bigcup_{a \in P_n} S_A(p)}
	{
		\Clopen(\Z\;A)
	}
	\Say{U}{S_A(q)\setminus F}{\Clopen(\Z\;A)}
	\Say{\Big( u,[9]\Big)}{\THM{ClopenSetHasStoneRepresentation}(U)}
	{
		\sum u \in A : U = S_A(u)
	}
	\Say{[10]}{\Elim \PoU(P_n)[9]\Elim S_A \Elim F }
	{
		u = 0
	}
	\Say{[11]}{\Elim U [10][9]}{F = S_A(q) }
	\Say{[12]}{[7][11]}{f \not \in S_A(q)}
	\Conclude{[q.*]}{ \Intro \bot [12]}{\bot}
	\Derive{[5]}{\Elim \bot \Intro \forall}{\forall q \in Q \. S_A(q) \cap X = \emptyset }
	\Conclude{[1.*]}{\THM{NowhereDenseStoneSpaceCondition}[5]}
	{
		\ND\Big(\Z\;A, X \Big)
	}
	\Derive{[1]}{\Intro \Imply}
	{
		\WD(A)
		\Imply
		\forall X : \TYPE{Meager}(X) \.
		\ND\Big(\Z\;A, X \Big)
	}
	\Assume{[2]}
	{
		\forall X : \TYPE{Meager}(X) \.
		\ND\Big(\Z\;A, X \Big)
	}
	\Assume{P}{\Nat \to \PoU(A)}
	\Say{X}{\bigcup^\infty_{n=1} \Act{\bigcup_{p \in P_n} S_A(p)}^\c}
	{
		\ND(\Z\;A)
	}
	\Say{\Big(Q,[3]\Big)}{\THM{NowhereDenseStoneSpaceCondition}(A,X)}
	{
		\NewLine :		
		\sum Q : \PoU(A) \. \forall q \in Q \. S_A(q) \cap X = \emptyset
	}
}\Page{	
	\AssumeIn{n}{\Nat}
	\AssumeIn{q}{Q}
	\Say{[4]}{[3](q) }
	{
		S_A(q) \subset X^\c \subset \bigcap^\infty_{m=1} \bigcup_{p \in P_m} S_A(p) 
		\subset  \bigcup_{p \in P_n} S_A(p)
	}
	\Say{\Big(m,p,[5]\Big)}
	{
		\Elim \Compacts\Big(\Z\;A,S_A(q)\Big) [4]
	}
	{
		\sum^\infty_{m=1} 
		\sum p : [1,\ldots,m] \to P_n \. 
		S_A(q) \subset \bigcup^m_{i=1} 	S_A(p_i)
	}
	\Conclude{[P.*]}{\Elim \PoU(A,P_n)\Elim S_A [5]}
	{
		\Big|\{ p \in P_n ; pq \neq 0  \} \Big| \le m \le \infty
	}
	\DeriveConclude{[2.*]}{\THM{WDbySupProperty}}
	{
		\WD(A)
	}
	\DeriveConclude{[*]}{\Intro \Imply \Intro \iff}
	{
		\WD(A)
		\iff
		\forall X \in \Meager(\Z \; A) \. \ND(\Z \; A, X)
	}
	\EndProof
	\\
	\Theorem{RealOpenDomainsAreNotWD}
	{
		\neg\WD\Big( \od(\Reals) \Big)
	}
	\Say{q}{\FUNC{enumerate}(\Rats)}{\TYPE{Bijection}(\Nat,\Rats)}
	\Say{X}{
		\Lambda n \in \Nat \. 
		\Big\{ U \in \od(\Reals) : \forall i \in [1,\ldots,n] \. q_i \in U\Big\}
	}
	{
		\Nat \downarrow (?A)
	}
	\Say{[1]}{
		\Lambda n \in \Nat \.
		\THM{OpenDomainsIndinum}(\Reals, X_n)\Elim \bigcap
		\Elim \intx
	}{
		\NewLine :		
		\forall n \in \Nat \. 
		\inf X_n = \intx \bigcap X_n =
		\intx  \Big\{ q_i | i \in [1,\ldots,n]  \Big\} = \emptyset
	}
	\Say{Y}{ 
		\Big\{ 
			U \in \od(\Reals) : 
			\forall n \in \Nat \. 
			\exists V \in X_n :
			V \le U 
		\Big\}  
	}{?\od(\Reals)}
	\Say{[2]}{\Elim Y \Elim X}{\forall U \in Y \. \Rats \subset U}
	\Say{[3]}{\THM{RationalsAreDense}[2]}{Y = \big\{ [\Reals] \big\}}
	\Say{[4]}{\inf [3] \THM{RealsExist}}{\inf Y = [\Reals] \neq [\emptyset]}
	\Conclude{[*]}{\Elim \WD [1][4]}{\neg\WD\Big( \od(\Reals) \Big)}
	\EndProof
}\Page{
	\Theorem{WeaklyDistributiveByRegularEmbedding}
	{
		\forall A : \WD \.
		\forall B : \REed(A) \.
		\NewLine \.
		\WD(B)
	}
	\Assume{X}{\Nat \to \TYPE{DownwardsDirected}(B)}
	\Say{Y'}{\Big\{
			a \in B : \forall n \in \Nat \. \exists b \in X_n : b \le a    
		\Big\}}{?A}
	\Say{Y}{ \Big\{
			a \in A : \forall n \in \Nat \. \exists b \in X_n : b \le a    
		\Big\}  }{?B}
	\Assume{[1]}{\forall n \in \Nat \. \inf_B X_n = 0}
	\Say{[2]}{\Elim \REed(A,B) [1]}
	{
		\forall n \in \Nat \. \inf_A X_n = 0
	}
	\Say{[3]}{\Elim \WD(A)[2]}{
		\inf_A Y
		= 0
	}
	\AssumeIn{b}{B}
	\Assume{[4]}
	{
		  0 < b \le Y	
	}
	\Say{\Big(a,\beta,[6]\Big)}{[4][3]}
	{
		\sum a \in A \. \sum \beta : \prod^\infty_{n=1} X_n \. 
		\Big( \forall n \in \Nat \.\beta_n \le a \Big) \And a \le b
	}
	\Say{[7]}{[6.1][6.2]}{\Big( \forall n \in \Nat \.\beta_n < b \Big)}
	\Say{[8]}{\Elim Y [7]}{b \in Y}
	\Conclude{[b.*]}{\Intro \inf [4]}{\inf Y = b}
	\Derive{[4]}{\Intro \Imply \Intro \forall}
	{
		\forall b \in B : 0 < b \le Y \Imply b = \inf Y \And b \in Y
	}
	\Assume{[5]}{\inf_B Y \neq 0}
	\Say{\Big(b,[6]\Big)}{\Elim \inf Y}{ \sum b \in B \. 0 < b \le Y}
	\Say{[7]}{[4]\Big(b,[6]\Big)}{b = \inf Y \And b \in Y} 
	\Say{\Big(a,\beta,[8]\Big)}{[6][3]}
	{
		\sum a \in A \. \sum \beta : \prod^\infty_{n=1} X_n \. 
		\Big( \forall n \in \Nat \.\beta_n \le a \Big) \And a < b
	}
	\Say{[9]}{[4][6]\Intro \Atom}{b \in \Atom(B)}
	\Say{[10]}{[9.1][9.2]}{\Big( \forall n \in \Nat \.\beta_n < b \Big)}
	\Say{[11]}{\Elim \Atom(B)[9][10]}{\beta = 0}
	\Say{[12]}{\Elim Y [11] \Intro \inf \Intro Y}{\inf Y = 0}
	\Conclude{[5.*]}{[5][12]}{\bot}
	\DeriveConclude{[X.*]}{\Elim \bot}{\inf Y = 0}
	\DeriveConclude{[*]}{\Intro \WD}{\WD(B)}
	\EndProof
}
\Page{
	\Theorem{WDByDenseSubalgebra}
	{
		\forall A \in \BOOL \.
		\forall B \subset_\BOOL A \. \NewLine \.
		\WD(B) \And \OD(A,B) 
		\Imply 
		\WD(A)
	}
	\Assume{P}{\Nat \to \PoU(A)}
	\Say{\Big(P',[1]\Big)}{\Elim \PoU(A,P)\Elim \OD(A,B)}{
		\NewLine :	
		\sum P' : \Nat \to \PoU(B) \.
		\forall n \in \Nat \. \forall p' \in P' \. \exists p \in P_n : p' \le p  
	}
	\Say{\Big(Q,[2]\Big)}{\THM{WDPoUProperty}(B,P')}
	{
		\NewLine :
		\sum Q : \PoU(B) \. \forall q \in Q \. \forall n \in \Nat \.
		\Big|\{ p \in P'_n :pq \neq 0 \}\Big| < \infty
	}
	\AssumeIn{a}{A}
	\Assume{[3]}{a \neq 0}
	\Say{\Big(b,[4]\Big)}{\Elim \OD(A,B)(a)}
	{
		\sum b \in B \. 0 < b \le a	
	}
	\Say{\Big(q,[5]\Big)}{\Elim \PoU(B,Q)(b)}
	{
		\sum q \in Q \. qb \neq 0 
	}
	\Conclude{[a.*]}{[4][5]}{qa \neq 0}
	\Derive{[3]}{\Intro \PoU}{\PoU(A,Q)}
	\Conclude{[P.*]}{\Elim \PoU(P,A)[2][1]}
	{
		\forall q \in Q \. \forall n \in \Nat \.
		\Big|\{ p \in P_n :pq \neq 0 \}\Big| < \infty
	}
	\DeriveConclude{[*]}{\Intro \WD}{\WD(A)}
	\EndProof
}\Page{
	\Theorem{WDByBeingSurjectiveImage}
	{
		\forall A : \WD \.
		\forall B \in \BOOL \. \NewLine \.
		\forall f : \Surj \And \oC \And \BOOL(A,B) \.
		\WD(B)
	}
	\Say{[1]}{\THM{ClosedMapLemma}\Big(\Z\;B,\Z\;A.\Z\;f \Big)}
	{
		\TYPE{ClosedMap}\Big(\Z\;B,\Z\;A.\Z_{A,B}\;f \Big)
	}
	\Assume{N}{\ND(\Z\;B)}
	\Say{[2]}{
		\Elim 
		\TYPE{ClosedMap}\Big(\Z\;B,\Z\;A.\Z_{A,B}\;f \Big)
		(\overline(N)) 
	}
	{
		\Closed\Big( \Z\;A,(\Z\;f)(\overline{N})\Big)
	}
	\Say{[3]}{\Elim \FUNC{closure}}
	{
		\overline{(\Z\;f)(N)} \subset (\Z\;f)(\overline{N})
	}
	\Assume{[4]}{\intx \overline{(\Z\;f)(N)} \neq \emptyset }
	\Say{[5]}{\THM{InteriorIsSubset}[3]}
	{
		\intx \overline{(\Z\;f)(N)} \subset (\Z\;f)(\overline{N})
	}
	\Say{[6]}{ \THM{InjectivePreimage}(\Z\;f)[5] }
	{
		(\Z\;f)^{-1} \Big( \intx \overline{(\Z\;f)(N)}  \Big)
		\subset 
		\overline{N}
	}
	\Say{[7]}{\Elim \TOP(\Z\;B,\Z\;A,\Z\;f)\Intro \intx [5][4]}
	{
		\intx \overline{N} \neq \emptyset
	}
	\Conclude{[8]}{\Elim \ND(\Z\;B,N)[7]}{\bot}
	\Derive{[4]}{\Elim \bot}{ \intx \overline{(\Z\;f)(N)} = \emptyset }
	\Conclude{[N.*]}{\Intro \ND [4]}{\ND\big(\Z\;A,(\Z\;f)(N)\big)}
	\Derive{[1]}{\Intro \forall}{\forall N \in \ND(\Z\;A) \.\ND\big(\Z\;A,(\Z\;f)(N)\big) }
	\Assume{M}{\Meager(\Z\;B)}
	\Say{\Big(N,[2]\Big)}{\Elim \Meager(\Z\;B,M)}
	{
		\sum N : \Nat \to \ND(\Z\;B) \.
		M = \bigcup_{n=1}^\infty N_n
	}
	\Say{[3]}{[2]\THM{UnionMap}{\Z\;B,\Z\;A,N,(\Z\;f)}}
	{
		(\Z\;f)(M)		
		(\Z\;f)\bigcup_{n=1}^\infty N_n =
		\bigcup_{n=1}^\infty   (\Z\;f)(N_n)
	}
	\Say{[4]}{\Intro \Meager [1][3]}
	{
		\Meager\Big(\Z\;A,(\Z\;f)(M)\Big)
	}
	\Say{[5]}{\THM{WDStoneSpacrCondition}(A)[4]}
	{
			\ND\Big(\Z\;A,(\Z\;f)(M)\Big)
	}
	\Conclude{[M.*]}{\THM{OrderContinuousNDPreimage}[5]}
	{
		\ND(\Z\;B,M)
	}
	\DeriveConclude{[*]}{\THM{WDStoneSpaceCondition}}{\WD(B)}
	\EndProof
}
\newpage
\subsubsection{Atoms}
\Page{
	\DeclareType{Atom}{\prod_{A \in \BOOL} ?A}
	\DefineNamedType{a}{Atom}{a \in \Atom(A)}{ \Big| \langle a \rangle_\I \Big| = 2  }
	\\
	\DeclareType{\Aless}{?\BOOL}
	\DefineType{A}{\Aless}{\Atom(A) = \emptyset}
	\\
	\DeclareType{\PA}{?\BOOL}
	\DefineType{A}{\PA}{ \OD\Big(A,\Atom(A)\Big) }
	\\
	\DeclareFunc{booleanDelta}{\prod_{A} \Atom(A) \to \Z\;A}
	\DefineNamedFunc{booleanDelta}{a}{\delta_a}
	{
		\Lambda b \in A \. [a \le b] 
	}
	\\
	\Theorem{DeltaIsIsolatedPoint}
	{
		\forall A \in \BOOL \. 
		\forall a \in A \.
		\IP\Big(\Z\;A,\delta_a\Big)
	}
	\Assume{[2]}{\Big|S_A(a)\Big| > 1}
	\Say{\Big(f,[3]\Big)}{\Elim S_A(a) [2]}{
		\sum f \in S_A(a) \. f \neq \delta_a  
	}
	\Say{\Big(b,[4]\Big)}{\Elim \delta_a [3] }
	{
		\sum b \in A \. f(b) = 1 \And a \not \le b	
	}
	\Say{[5]}{\Elim \BOOL(A,\Bool,f)\Elim f [4.1] \Elim \Bool}
	{
		f(ab) = f(a)f(b) = 1 \wedge 1 = 1
	}
	\Say{[6]}{\THM{ZeroInKer}(A,\Bool,f)[5]}{ab \neq 0}
	\Say{[7]}{\THM{BooleanOrderProduct}[4.2] }{ab < a}
	\Say{[8]}{[6][7]}{0 < ab < a}
	\Conclude{[2.*]}{\Elim \Atom(A,a)[8]\Intro \bot}{\bot}
	\Derive{[2]}{\Elim \bot }{ \Big| S_A(a) \Big| \le 1 }
	\Say{[3]}{\Elim \Atom(A,a)[2]\THM{ZeroStoneRepresentation}}
	{
			\Big| S_A(a) \Big| = 1
	}
	\Conclude{[*]}{
		\THM{IsolatedPoinProperty} [3] 
		\Intro \delta_a 
		\THM{StoneTopologyBasis}(A)
	}
	{
			\IP\Big(\Z\;A,\delta_a\Big)
	}
	\EndProof
}\Page{
	\Theorem{AtomsByStoneIsolatedPoint}
	{
		\forall A \in \BOOL \.
		\forall f : \IP(\Z\;A) \.
		\exists a \in \Atom(A) \.
		f = \delta_a
	}
	\Say{\Big(U,[1]\Big)}{\THM{IsolatedPointProperty}(\Z\;A,\Z\;A,f)}
	{
		\sum U \in \U(f) \. U = \{f\}
	}
	\Say{[2]}{\Elim \TYPE{T2}(\Z\;A)[1]}{\Clopen(\Z\;A,U)}
	\Say{[3]}{\THM{ClosedIsCompact}[2]\Intro \TK(A)}{U \in \TK(A)}
	\Say{\Big(a,[4]\Big)}{\THM{CompactOpenAreStoneRepresentations}(A,U)}
	{
		\sum_{a \in A} S_A(a) = U	
	}
	\AssumeIn{b}{A}
	\Assume{[5]}{ b < a }
	\Say{[6]}{\THM{StoneRepresentationBooleanOrder}[5]}{S_A(b) \subsetneq S_A(a)}
	\Say{[7]}{[1][4][6]}{S_A(b) = \emptyset}
	\Conclude{[b.*]}{\THM{StoneRepresentationTHM}[7]}
	{
			b = 0	
	}
	\Derive{[5]}{\Intro \Atom}{a \in \Atom(A)}
	\Say{[6]}{\Elim S_A [1][4]}{f(a) = 1}
	\Say{[7]}{
		\Lambda b \in A \.
		\Lambda T : a \le b \.
		\Elim \POSET(A,\Bool,f)[6] T
	}{\forall b \in A \. a \le b \Imply f(b) = 1}
	\AssumeIn{b}{A}
	\Assume{[8]}{a \not \le b}
	\Say{[9]}{\THM{ProductBooleanOrder}(A,a,a,b)}{ab \le a}
	\Say{[10]}{\Elim \Atom(A,a)[8][9]}{ab = 0}
	\Conclude{[b.*]}{
		\THM{UnityMult}\Big(A,f(b)\Big)
		[6]
		\Elim \BOOL(A,\Bool,f)
		[10]
		\THM{ZeroRingImage}(A,\Bool,f)
	}
	{
		\NewLine :
		f(b) = 
		1 \cdot f(b) = 
		f(a)f(b) = 
		f(ab) = 
		f(0) =0
	}
	\Derive{[8]}{\Intro \Imply \Intro \forall}
	{
		\forall b \in A \. a \not \le b \Imply f(b) = 0
	}
	\Conclude{[*]}{\Intro \delta_a [7][8]}{f = \delta_a}
	\EndProof
	\\
	\Theorem{AtomsIsolatedPointsCorrespondance}
	{
		\forall A \in \BOOL \.
		\TYPE{Bijection}\Big(\Atom(A), \IP(\Z\;A), \delta \Big)		
	}
	\Say{[1]}{\THM{DeltaIsIsolatedPoin}(A)\THM{AtomsByStoneIsolatedPoints}}
	{
		\NewLine :		
		\TYPE{Sujection}\Big(\Atom(A),\IP(\Z\;A), \delta \Big)
	}
	\AssumeIn{a,b}{\Atom(A)}
	\Assume{[2]}{\delta_a = \delta_b}
	\Say{[3]}{\Elim \delta_a(a)\Elim(=)[2]}{1 = \delta_a(a) = \delta_b(a)}
	\Say{[4]}{\Elim \delta_b [3]}{b \le a}
	\Say{[5]}{\Elim \delta_b(b)\Elim(=)[2]}{1 = \delta_b(b) = \delta_a(b)}
	\Say{[6]}{\Elim \delta_b [5]}{a \le b}
	\Conclude{\Big[(a,b).*\Big]}{\Elim \TYPE{Symmetric}(A,\le)[4][6]}
	{
		a = b
	}
	\Derive{[2]}{\Intro \Inj}{\Inj\Big( \Atom(A),\IP(\Z\;A),\delta\Big)}
	\Conclude{[*]}{\Intro \Bij }
	{
		\Bij\Big( \Atom(A),\IP(\Z\;A),\delta\Big)	
	}
	\EndProof
}\Page{
	\Theorem{AtomlessStoneExpression}
	{
		\forall A \in \BOOL \.
		\Aless(A) 
		\iff
		\IP(\Z\;A) = \emptyset
	}
	\NoProof
	\\
	\Theorem{PurelyAtomicStoneExpression}
	{
		\forall A \in \BOOL \.
		\PA(A) 
		\iff
		\Dense\Big(\Z\;A,\IP(\Z\;A)\Big) 
	}
	\NoProof
	\\
	\Theorem{CantorAlgebraTHM}
	{
		\forall A \in \BOOL \.
		A \neq \star \And \Aless(A) \And |A|\le \aleph_0 
		\iff
		A \cong_{\BOOL} \TK(\C)
	}
	\Assume{[1]}{ A \neq \star \And \Aless(A) \And |A|\le \aleph_0 }
	\Say{[2]}{\THM{AtomlessStoneExpression}(A)[1.2]\Intro \Perfect}
	{
		\Perfect(\Z\;A)
	}
	\Say{[3]}{\THM{UrysohnMetrizationTheorem}[1.2]}{\TYPE{Metrizable}(\Z\;A)}
	\Say{[4]}{\Elim \Z [1.1]}{\Z\;A \neq \emptyset}
	\Say{[5]}{\THM{BrouwersTopologicalCharOfCantorSet}[2][3][4]}
	{
		\C \cong_\TOP \Z\;A
	}
	\Say{[6]}{\TK(5])}{\TK \; \C \cong_\BOOL \TK \; \Z \; A}
	\Conclude{[*]}{\Elim \TK [4]}{\TK \; \C \cong_\BOOL A}
	\EndProof
	\\
	\Theorem{AtomsInSubalgebra}
	{
		\forall A \in \BOOL \.
		\forall B : \REed(A) \.
		\forall a \in \Atom(A) \.
		\exists b \in \Atom(B) :
		a \le b
	}
	\Say{X}{\{ b \in B : b \ge a \}}{?B}
	\Assume{[1]}{\inf X = 0}
	\Say{[2]}{\Elim X \Intro (\le)}{ a \le X}
	\Say{[3]}{\Elim \REed(A,B)[1][2]}{a = 0}
	\Conclude{[1.*]}{\Elim \Atom(A,a)[3]}{\bot}
	\Derive{\Big(b, [1] \Big)}{\Elim \bot}
	{
		\sum b \in B \. 0 < b \le X
	}
	\AssumeIn{b'}{B}
	\Assume{[2]}{0 < b' < b}
	\Say{[3]}{\Intro(\c)[2]}{b \not \le (b')^\c}
	\Say{[4]}{[1][3]\Intro X}{(b')^\c \not \in X }
	\Say{[5]}{\Elim X [4]}{a \not \le (b')^\c}
	\Say{[6]}{\Elim \c [5]}{ab' \neq 0}
	\Say{[7]}{\Elim \Atom(A,a)[6]}{b' \ge a}
	\Say{[8]}{\Intro X [7]}{b' \in X}
	\Say{[9]}{[8][1]}{b \le b'}
	\Conclude{[b'].*}{\THM{TrichtomyPrinciple}[9][2]}{\bot}
	\DeriveConclude{[*]}{\Elim \bot \Intro \forall \Intro \Atom}
	{
		b \in  \Atom(B)
	}
	\EndProof
	\\
	\Theorem{AtomlessBySubalgebra}
	{
		\forall A \in \BOOL \.
		\forall B : \REed(A) \.
		\Aless(B) \Imply \Aless(A)
	}
	\NoProof
}
\Page{
	\Theorem{PurelyAtomicIsWeaklyDistributive}
	{
		\forall A : \PA \.
		\WD(A)
	}
	\SayIn{U}{\bigcup_{a \in \Atom(A)} \{\delta_a\} }
	{   
		\T\;\Z\;A	
	}
	\Say{[1]}{\Elim U \Elim \PA(A) \THM{PurelyAtomicStoneExpression} \Intro U}{\Dense(\Z\;A)}
	\SayIn{N}{U^\c}{\Closed \And \ND(\Z\;A)}
	\Assume{M}{\Meager(\Z\;A)}
	\Say{[2]}{
		\THM{IntersectionIsSubset}(\Z\;A,N,M)\THM{NowhereDenseSubset}{\Z\;A,N,N \cap M}
	}
	{
		\ND(\Z\;A,N \cap M)
	}
	\Say{[3]}{
	\THM{IntersectionIsSubset}(\Z\;A,M,U)
	\THM{MeagerSubset}{\Z\;A,M,M \cap U}}
	{
		\Meager\Big(\Z\;A ,M \cap U\Big)
	}
	\Say{\Big(S,[4]\Big)}{\Elim \Meager [4]}
	{
		\sum S \Nat \to \ND(\Z\;A) \. M \cap U = \bigcup^\infty_{n=1} S_n
	}
	\Say{[5]}{\THM{IntersectionIsSubset}(\Z\;A,U,M)}{M \cap U \subset U }
	\Say{[6]}{\THM{UnionSubset}[4][5]}{\forall n \in \Nat \. S_n \subset U }
	\Say{[7]}
	{
		\Elim U 
		\THM{AtomsIsolatedPointsCorrespondance}(A) 
		\Intro \TYPE{Discrete} 
		\Intro U
	}
	{
		\TYPE{Discrete}(U)
	}
	\Say{[8]}{\Lambda n \in \Nat \. [6]\Elim \ND(S_n)\Elim \TYPE{Discrete}(U)[7]}
	{
		\forall n \in \Nat \. S_n = \emptyset
	}
	\Say{[9]}{\THM{EmptysetUnion}(\Z\;A)[8][4]}{M \cap U = \emptyset}
	\Say{[10]}{\THM{ComplementDisjointRepresentation}(\Z\;A,M,U)\Elim N}
	{
		M = M \cap N
	}
	\Conclude{[M.*]}{\Elim(=)[10][2]}{\ND(\Z\;A,M)}
	\DeriveConclude{[*]}{\Intro \forall \THM{NowhereDenseStoneSpaceCondition}(A)}
	{\WD(A)}
	\EndProof
	\\
	\Theorem{PurelyAtomicByRegularEmbedding}
	{
		\forall A : \PA \.
		\forall B : \REed(A) \.
		\PA(B)
	}
	\Assume{[1]}{\neg \PA(B)}
	\Say{\Big(b,[2]\Big)}{\Elim \PA(B)}
	{
		\sum_{b \in B} \inf_B \{ c \in B \.  0 < c \le b \} = 0
	}
	\Say{[3]}{\Elim \REed(A)[2]}
	{
		\inf_A \{ c \in B \.  0 < c \le b \} = 0
	}
	\Say{\Big(a,[4]\Big)}{\Elim \PA(A,b)}{\sum a \in \Atom(A) \. a < b}
	\Say{[5]}{ \Elim \Atom(A,a)[4] \Intro \inf_A }
	{
		\inf_A \{ c \in B \.  0 < c \le b \} \ge a > 0
	}
	\Conclude{[1.*]}{\THM{TrichotomyPrinciple}[3][5]}{\bot}
	\DeriveConclude{[*]}{\Elim \bot}{\PA(B)}
	\EndProof
}\Page{
	\Theorem{AtomsOfDenseSubalgebra}
	{
		\forall A \in \BOOL \.
		\forall B \subset_\BOOL A \.
		\OD(A,B) 
		\Imply 
		\Atom(A) = \Atom(B)
	}
	\AssumeIn{a}{\Atom(A)}
	\Say{[1]}{\Elim \Atom(A,a)}{a \neq 0}
	\Say{\Big(b,[2]\Big)}{\Elim \OD(A,B)\Big(a,[1]\Big)}
	{
		\sum b \in B \. 0 < b \le a
	}
	\Say{[3]}{\Elim \Atom(A,a)\Big(b,[2]\Big)}
	{
		a = b
	}
	\Conclude{[a.*]}{\Elim \TYPE{Subring}(A,B)[3]\Intro \Atom}{a \in \Atom(B)}
	\Derive{[1]}{\Intro \subset}
	{
		\Atom(A) \subset \Atom(B)
	}
	\AssumeIn{b}{\Atom(B)}
	\AssumeIn{a}{A}
	\Assume{[2]}{a < b}
	\Assume{[3]}{a \neq 0}
	\Say{\Big(b',[4]\Big)}{\Elim \OD(A,B)\Big(a,[3]\Big)}
	{
		\sum b' \in B \. 0 < b \le a
	}
	\Say{[5]}{[2][4]}{b' < b}
	\Conclude{[a.*]}{\Elim \Atom(B,b)\Big(b',[5]\Big)}
	{
		\bot
	}
	\DeriveConclude{[b.*]}{\Elim \bot \Intro \Imply \Intro \forall \Intro \Atom}
	{
		b \in \Atom(A)
	}
	\DeriveConclude{[*]}{\Intro \subset \Intro \TYPE{SubsetEq} }
	{
		\Atom(A) = \Atom(B)
	}
	\EndProof
	\\
	\Theorem{AtomlessByBeingSurjectiveImage}
	{
		\forall A : \Aless \.
		\forall B \in \BOOL \. \NewLine \.
		\forall f : \Surj \And \oC \And \BOOL(A,B) \.
		\Aless(B)
	}
	\Assume{\delta}{\IP(\Z\;B)}
	\Say{[1]}{
		\THM{T2PointsAreClosed}\Big(\Z\;A,f(\delta)\Big)
		\THM{AtomlessStoneExpression}(A)
		\Elim \intx	
	}
	{
		\NewLine :		
		\intx \overline{\Big\{(\Z\;f)(\delta)\Big\}} =
		\intx \Big\{ (\Z\;f)(\delta) \Big\} =
		\emptyset
 	}
 	\Say{[2]}{\Intro \ND[1]}{\ND\Big(\Z\;A,  \big\{ (\Z\;f)(\delta) \big\} \Big)}
 	\Say{[3]}{\THM{OrderContinuousNDPreimage}(A,B,f)\THM{InjectivePreimage}(A,B,f)}
 	{
 		\ND\Big(\Z\;B, \{\delta\} \Big)
 	}
 	\Say{[4]}{
 		\THM{T2PointsAreClosed}(\Z\;B,\delta)
		\Elim \IP(\Z\;B,\delta)
		\Elim \intx
 	}
 	{
 		\NewLine :		
		\intx \overline{\Big\{ \delta \Big\}} =
		\intx \{\delta\} =
		\{\delta\}
 	}
 	\Say{[5]}{\Elim \ND\Big( \Z \; B, \{\delta\}\Big)}
 	{
 		\intx \overline{\{ \delta \}} = \emptyset
 	}
 	\Conclude{[*.\delta]}{\Elim \TYPE{Singleton}(\Z \; B, \delta)[4][5]}{\bot}
 	\Derive{[1]}{\Elim \bot \Intro \TYPE{Perfect}}{\TYPE{Perfect}(\Z\;B)}
 	\Conclude{[*]}{\THM{AtomlessStoneExpression}[1]}{\Aless(B)}
	\EndProof
}
\Page{
	\Theorem{PurelyAtomicByBeingSurjectiveImage}
	{
		\forall A : \PA \.
		\forall B \in \BOOL \. \NewLine \.
		\forall f : \Surj \And \oC \And \BOOL(A,B) \.
		\PA(B)
	}
	\AssumeIn{b}{B}
	\Assume{[1]}{b \neq 0}
	\Assume{[2]}{ \forall y \in \Atom(B) \. y \not \le b }
	\Say{\Big(a,[3]\Big)}{\Elim \Surj(A,B,f)(a)}
	{
		\sum_{a \in A} f(a) = b
	}
	\Say{[4]}{\THM{HomoZeroImage}[1][3]}{a \neq 0}
	\Say{\A}{\Big\{x \in \Atom(A) : x \le a\}}{?\Atom(A)}
	\Say{[5]}{\Elim \A \Elim \PA(A)}{\sup \A = a}
	\AssumeIn{x}{\A}
	\Say{[6]}{\Elim \A(x)}{x \le a}
	\Say{[7]}{\THM{BooleanHomoIsMonotonic}[6][1]}{f(x) \le b}
	\Assume{[8]}{f(x) \neq 0}	
	\Say{\Big(y,[9]\Big)}{[2][8]}
	{
		\sum_{y \in B}   0 < y < f(x)
	}
	\Say{[10]}{\Elim \Surj(A,B,f)(y)}
	{
		\sum_{x' \in A} f(x') = y
	}
	\Say{[11]}{\THM{HomoZeroImage}[10][9]}{x' \neq 0}
	\Say{[12]}{
		[10]\Elim 
		\TYPE{BooleanOrder}(B)[9]
		[10][4]
		\Elim \BOOL(A,B,f)
	}
	{
			f(x') = 
			y = 
			yf(x)  =
			f(x')f(x) =
			f(x'x)			
	}
	\Say{[13]}{\THM{BooleanHomoIsMonotonic}[12]}
	{
				x'x \not < x'
	}	
	\Say{[14]}{\THM{BooleanProductOrder}(A,x,x')}
	{
		x'x \le  x'
	}
	\Say{[15]}{\THM{TrichtomyPrinciple}[13][14]}{x'x = x'}
	\Say{[16]}{\Elim \TYPE{BooleanOrder}(A)[15]}{ x' \le x }
	\Say{[17]}{\THM{BooleanHomoIsMonotonic}[12][16]}{ x' < x }
	\Conclude{[x.*]}{\Elim \Atom(A,x)[17][11]}{\bot}
	\Derive{[8]}{\Elim \bot \Intro \forall}{\forall x \in \A \. f(x) = 0}
	\Say{[9]}{[8]\Elim \sup}
	{
		\sup f(\A) = \sup \{0\} = 0	
	}
	\Say{[10]}{\Elim \oC(A,B,f)[5][3]}{\sup f(\A) = f(\sup\; \A) = f(a) = b}	
	\Conclude{[b.*]}{[1][5][9]}{\bot}
	\DeriveConclude{[*]}{\Elim \bot \Intro \Imply \Intro \forall \Intro \PA }
	{
		\PA(B)
	}
	\EndProof
}
\newpage
\subsubsection{Homogeneous Algebras}
\Page{
	\DeclareType{\Homog}{?\BOOL}
	\DefineType{A}{\Homog}
	{
		\forall a \in A \. 
		a \neq 0 \Imply A \cong_\BOOL \langle a \rangle_\I 
	}
	\\
	\Theorem{HomogeneousAlternatives}
	{
		\forall A : \Homog \. 
		 A \cong_\BOOL \Bool  |\Aless(A)
	}
	\AssumeIn{a}{\Atom(A)}
	\Say{[1]}{\Elim \Atom(A,a)}{a \neq 0}
	\Say{\Big(b,[2] \Big)}{\Elim \Homog(A)[1]}
	{
		A \cong \langle a \rangle_\I
	}
	\Conclude{[3]}{\Elim \Atom(A,a)[2]}{A \cong \Bool}
	\DeriveConclude{[*]}{\Intro \Aless}
	{
		A \cong_\BOOL \Bool  |\Aless(A)
	}
	\EndProof
}\Page{
	\Theorem{HomegeneousByDenseSubset}
	{
		\forall A : \TAlgebra \. \NewLine \.
		\OD\bigg(A,\Big\{d \in A : \langle d \rangle_\I \cong_\BOOL A \Big\}\bigg)
		\Imply
		\Homog(A)
	}
	\Say{D}{\Big\{d \in A : \langle d \rangle_\I \cong_\BOOL A \Big\}}{?A}
	\Assume{[00]}{A \not \cong_\BOOL \star}
	\Assume{[000]}{\Aless(A)}
	\AssumeIn{a}{A}
	\Assume{[1]}{a \neq 0}
	\Say{\Big(x,[2]\Big)}{\Elim \Aless [000][00]}
	{
		\sum x : \Nat \downarrow \downarrow  A \. x_1 = a 	
	}
	\Say{D'}{
		\{ 
			d \in D \. 
			\forall n \in \Nat \.  
			\forall b \in A \. 
			b \le d  \Imply a_{n+1} \not \le b \not \le a_n
		\}
	}
	{?D}
	\Say{[3]}{\Elim D'\Elim \TAlgebra(A)\Elim\OD(A,D)[0]\Intro D'}
	{
		\OD(A,D')
	}
	\Say{\Big(P,[4]\Big)}{\THM{OrderDenseContainsPartitionOfUnity}(A.D')}
	{
		\sum P : \PoU(A) \. P \subset D'
	}
	\Say{[5]}{\Elim P \Elim \TYPE{StrictlyDecreasing}(\Nat,A,x)\Intro \TYPE{Infinite}}
	{
		|P| = \infty
	}
	\Say{P'}{\{ p \in P : p \le a \}}{?P}
	\Say{[6]}{\Elim P' \Intro \PoU}{ \PoU\Big(\langle a \rangle_\I, P'\Big)  }
	\Conclude{[a.*]}{
		\THM{ProductStructureByPartitionOfUnity}\Big(\langle a \rangle_\I,P'\Big)
		\Elim P' \Elim D
		\NewLine
		\THM{ProductStructureByPartitionOfUnity}\Big(A,P\Big)
		\Elim P \Elim D
		\THM{ProductDecomposition}(\BOOL)
		\NewLine		
		\THM{InfiniteProductCardinality}(P,P')[5]
		\Elim P \Elim D
		\THM{ProductStructureByPartitionOfUnity}\Big(A,P\Big)
	}
	{
		\NewLine		
		\langle a \rangle_\I \cong_\BOOL  
		\prod_{p \in P'} \langle p \rangle_\I\cong_\BOOL 
		A^{P'}  \cong_\BOOL 
		\left(\prod_{p \in P} \langle p \rangle_\I\right)^{P'} \cong_\BOOL
		(A^P)^{P'} \cong_{\BOOL}
		A^{P \times P'} \cong_{\BOOL}
		A^P \cong_{\BOOL}  \NewLine \cong_\BOOL
		\prod_{p \in P} \langle p \rangle_\I
		\cong_{\BOOL} A
	}
	\DeriveConclude{[*]}{\Intro \Homog}{\Homog(A)}
	\EndProof
	\\
	\Theorem{OrderClosureIsHomog}
	{
		\forall A : \Homog \.
		\Homog\Big( \od(\Z\;A) \Big)
	}
	\NoProof
}
\Page{
	\Theorem{HomogeneousCoproduct}
	{
		\forall I \in \SET \.
		\forall A : I \to \Homog \.
		\Homog\left(\bigotimes_{i \in I} A_i \right)
	}
	\Assume{[0]}{\forall i \in I . A_i \neq \star}
	\AssumeIn{i}{I}
	\Assume{[00]}{\Aless(A_i)}
	\Assume{a}{\prod_{j \in I} A_j}
	\Say{J}{\{ j \in I : a_j \neq e \}}{?I}
	\Assume{[1]}{ \Big|J\Big| < \infty }
	\SayIn{t}{\bigotimes_{j \in I} a_j}{\bigotimes_{j \in I} A_i}
	\Say{f}{
		\Lambda j \in I \. 
		\Lambda b  \in \langle a_j\rangle_\I \.
		t\iota_j(b)
	}
	{
		\prod_{j \in I} \BOOL\Big( \langle a_j \rangle_\I, \langle t \rangle_\I \Big)
	}
	\Say{\Big(g,[2]\Big)}{
		\THM{CoproductUniversalProperty}\Big(
			\BOOL,
			I,
			\langle a \rangle_\I,
			\langle t \rangle_\I,
			f			 
		\Big)
	}
	{
		\sum \bigotimes_{j \in I} \langle a_j \rangle_\I
		\Arrow{g} \langle t \rangle_\I \. \forall j \in I \. \iota_j g = f_j 
	}
	\Assume{K}{\Finite(I)}
	\Assume{[3]}{ J \subset K}
	\Assume{b}{\prod_{k \in K} \langle a_k \rangle}
	\SayIn{s}{\bigotimes_{j \in J} b_j}{\bigotimes_{j \in I} A_i }
	\Conclude{[K.*]}{
		\Elim s
		\Elim \otimes
		[2]
		\Elim f
		\Elim \BOOL\left( \bigotimes_{j \in K} A_j \right)
		\Intro \otimes
		\Intro s
		\THM{TensorDistributivity}(I,A,a,b)
		\NewLine :
		\Lambda j \in K \.
		\THM{PrincipleIdealStructure}(A_j,a_j)
		\Elim \TYPE{BooleanOrder}	
	}
	{		
		g(s) =
		g\left(\bigotimes_{j \in K} b_j \right) =
		g\left(\prod_{j \in K} \iota_j(b_j) \right) = \NewLine =
		\prod_{j \in K} \iota_j g (b_j) = 
		\prod_{j \in K} f_j(b_j) = 
		\prod_{j \in K} t \iota_j(b_j) =
		t \prod_{j \in K} \iota_j(b_j) =
		t 	\bigotimes_{j \in K} b_j = 
		t s =
		s 
	}	
	\Derive{[3]}{\Intro \forall}
	{
		\forall K : \Finite(I) \.
		J \subset K \Imply
		\forall b : \prod_{k \in K} \langle a_k \rangle_\I \.
		g\left(\otimes_{k \in k} b_k \right) =  \otimes_{k \in k} b_k
	}
	\AssumeIn{b}{\bigotimes_{j \in I} \langle a_j \rangle_I}
	\AssumeIn{[5]}{b \neq 0}
	\Say{\Big(K,c,[6] \Big)}
	{
		\THM{TensorApproximation}(I,A,b)[5]
	}
	{
		\NewLine :		
		\sum K : \Finite(A) \.
		\sum c : \prod_{k \in K} \langle a_k \rangle_\I \.
		(\forall k \in K \. c_k \neq 0) \And
		\bigotimes_{k \in K} c_k \le b		
	}
	\Say{K'}{J\cap K}{?K}
	\Say{c'}{\Lambda j \in I \. \If j \in K' \Then c_j \Else a_j }
	{
		\prod_{j \in I} \langle a_j \rangle_\I
	}
}\Page{
	\Say{[7]}{
			\Elim \POSET\left( 
				\bigotimes_{j \in I} \langle a_j \rangle_\I, 
				\langle t \rangle_\I,
				g 
			\right)[6.2]\Elim c'
			[6.1]
	}
	{
		\NewLine:
		g(b) \ge	
		g\left( \bigotimes_{k \in K} c_k \right)  = 
		g\left( \bigotimes_{j \in I} c_j' \right) = 
		\bigotimes_{j \in I} c_j' > 0				
	}
	\Conclude{[b.*]}{\Elim (>) [7]}{g(b) \neq 0}
	\Derive{[4]}{\Intro \Imply \Intro \forall \Intro \ker }{\ker g = \{0\}}
	\Say{[5]}{\THM{ZeroKernelTHM}[4]}
	{
		\Inj\left( 
				\bigotimes_{j \in I} \langle a_j \rangle_\I, 
				\langle t \rangle_\I,
				g 
		\right)
	}
	\Assume{K}{\Finite(I)}
	\Assume{c}{\prod_{k \in K} A_k}
	\Say{K'}{K \cap J}{?I}
	\Say{c'}{\Lambda j \in I \. \If j \in K' \Then c_j \Else e }
	{
		\prod_{j \in I} A_j
	}
	\SayIn{d}{\bigotimes_{j \in I} c_j'}{\bigotimes_{j \in I} A_j}\
	\Assume{[6]}{d \le t}
	\Conclude{[K.*]}{
		\Elim \TYPE{BooleanOrder}[6]
		\Elim d \Elim t
		\THM{TrnsorDestributivity}(I,A)
		[3](K',c'a)
	}
	{
		\NewLine :		
		d = 
		dt =
			\left(\bigotimes_{j \in I} c_j'\right)
		\cdot
			\left(\bigotimes_{j \in I} a_j\right)
		= \bigotimes_{j \in I} c_j' a_j =
		g\left( \bigotimes_{j \in I} c_j' a_j \right)
	}
	\Derive{[6]}{
		\Intro \Imply 
		\Intro^2 \forall 
		\THM{TensorApproximation}(I,A) 
		\Intro \Surj
	}
	{
		\Surj\left( 
				\bigotimes_{j \in I} \langle a_j \rangle_\I, 
				\langle t \rangle_\I,
				g 
		\right)
	}
	\Say{[7]}{\Intro \TYPE{Isomorphic}[5][6]}
	{\langle t \rangle_{\I} \cong_\BOOL \bigotimes_{j \in I} \langle a_j \rangle_\I}
	\Conclude{[a.*]}{\Big(\Lambda i \in I \. \Elim \Homog(A_i)\Big)[7]}
	{
		\langle t \rangle_\I \cong \bigotimes_{j \in I} A_j
	}
	\Derive{[1]}{\Intro \Imply \Intro \forall }
	{
		\forall a : \prod_{j \in I} A_j \.
		\Big|\{ j \in I : a_j \neq e \}\Big| < \infty
		\Imply
		\left\langle \bigotimes_{j \in I} a_j  \right\rangle_\I \cong_{\BOOL}
		\bigotimes_{j \in I} A_j
	}
	\Say{T}{
		\left\{ 
			t \in \bigotimes_{j \in I} A_j :
			\exists a :  \prod_{j \in I} A_j :
			\Big|\{ j \in I : a_j \neq e \}\Big| < \infty
			\And
			t = \bigotimes_{j \in I} a_j 
		\right\}
	}{?\bigotimes_{j \in I} A_j}
}\Page{
	\AssumeIn{n}{\Nat}
	\Say{\Big(D,[2]\Big)}{\Intro \PoU [00]\Elim \Aless}
	{
			\sum D : \PoU(A_i) \. |D| = n
	}
	\Say{[3]}{\THM{CoproductPartitionOfUnity}(I,A,i,D)}
	{
		\PoU\left( \bigotimes_{j \in I} A_j, \iota_i(D) \right)
	}
	\Conclude{[n.*]}
	{
		\THM{ProductStructureByPartitionOfUnity}\left(\bigotimes_{j \in I} A_j,
		\iota_i(D)\right) [0][00] 
		[1](\Lambda d \in D \. \star \mapsto d )
		\NewLine
		\THM{InjectiveCannonicalEmbedding}(I,A)
	}
	{
			\bigotimes_{j \in I} A_j \cong_\BOOL
			\prod_{d \in D} \langle d \rangle_\I \cong_\BOOL
			\prod_{d \in D} \bigotimes_{j \in I} A_j \cong_\BOOL
			\left( \bigotimes_{i \in I} A_i \right)^n
	}
	\Derive{[2]}{\Intro \forall}{
		\forall n \in \Nat \. 
		\left( \bigotimes_{i \in I} A_i \right)^n \cong_\BOOL
		\bigotimes_{i \in I} A_i
	}	
	\AssumeIn{a}{\bigotimes_{j \in I} A_j}
	\Assume{[3]}{a \neq 0}
	\Say{\Big(n,t,[4]\Big)}{\THM{TensorApproximation}(I,A,a)[2][0][00]\Intro T}
	{
		\NewLine :		
		\sum n \in \Nat \. 
		\sum t : [1,\ldots,n] \to T \.
		a = \bigvee^n_{k=1} t_k
		\And
		\forall k,l \in [1,\ldots,n] \.
		t_i t_j = 0	
	}
	\Say{[5]}{\Intro \PoU [5]}
	{
		\PoU\Big( \langle a \rangle_\I, \im t  \Big)
	}
	\Conclude{[a.*]}
	{
		\THM{ProductStructureByPartitionOfUnity}
		\Big(\langle a \rangle_\I,  \im t  \Big)
		\Elim T(t)
		[2](n)
	}
	{
		\langle a  \rangle_\I \cong_\BOOL
		\prod^n_{k=1} \langle t_k \rangle_\I \cong_\BOOL
		 \left( \bigotimes_{i \in I} A_i \right)^n \cong_\BOOL
		 \bigotimes_{i \in I} A_i
	}
	\DeriveConclude{[*]}{\Intro \forall \Intro \Homog}
	{
		\Homog\left( \bigotimes_{i \in I} A_i \right)
	}
	\EndProof
}
\newpage
\subsection{Automorphisms Group of a Boolean Algebra[!!]}
\subsubsection{Gluing Lemmas}
\Page{
	\Theorem{FiniteGluingLemma}
	{
		\forall A : \BOOL \.
		\forall I : \Finite \.
		\forall a : I \to A \.
		\forall b : I \to A \.
		\NewLine \.
		\forall [0] : \PoU(A,\im a \And \im b) \.
		\forall f : \prod_{i \in I} \TYPE{Isomorphism}\Big( \BOOL, 
			\langle a_i \rangle, 
			\langle b_i \rangle 
		\Big) \.
		\NewLine \.
		\exists g \in \Aut_\BOOL(A) : \forall i \in I \. g_{|\langle a_i \rangle} = f_i	
	}
	\Say{[1]}{\THM{ProductStructureByFinitePartitionOfUnity}(A,\im a)}
	{
		A \cong_\BOOL \prod_{i \in I} \langle a_i \rangle
	}
	\Say{[2]}{\THM{ProductStructureByFinitePartitionOfUnity}(A,\im b)}
	{
		A \cong_\BOOL \prod_{i \in I} \langle b_i \rangle
	}
	\Say{h}
	{
		\Lambda t \in \prod_{i \in I} A_i \.  
		\Lambda_{i \in I} f_i(t_i)	
	}
	{
		\BOOL\left( 
			\prod_{i \in I} \langle a_i \rangle,
			\prod_{i \in I} \langle b_i \rangle
		\right)
	}
	\SayIn{g}{[1]h[2]^{-1}}{\End_\BOOL(A)}
	\Say{[3]}{
		\Elim g		
		\Elim \TYPE{Isomorphism}\Big( 
			\BOOL, 
			\langle a \rangle,
			\langle b \rangle 
		\Big)}
	{
		g \in \Aut_\BOOL(A)
	}
	\Conclude{[*]}{\Elim \PoU(A,\im a \And \im b) \Elim g }
	{
		\forall i \in I \. g_{|\langle a_i \rangle} = f_i
	}
	\EndProof
	\\
	\Theorem{CountableGluingLemma}
	{
		\forall A : \SA \.
		\forall I : \TYPE{Countable} \.
		\forall a : I \to A \.
		\forall b : I \to A \.
		\NewLine \.
		\forall [0] : \PoU(A,\im a \And \im b) \.
		\forall f : \prod_{i \in I} \TYPE{Isomorphism}\Big( \BOOL, 
			\langle a_i \rangle, 
			\langle b_i \rangle 
		\Big) \.
		\NewLine \.
		\exists g \in \Aut_\BOOL(A) : \forall i \in I \. g_{|\langle a_i \rangle} = f_i	
	}
	\NoProof
	\\
	\Theorem{GluingLemma}
	{
		\forall A : \TAlgebra \.
		\forall I \in \SET \.
		\forall a : I \to A \.
		\forall b : I \to A \.
		\NewLine \.
		\forall [0] : \PoU(A,\im a \And \im b) \.
		\forall f : \prod_{i \in I} \TYPE{Isomorphism}\Big( \BOOL, 
			\langle a_i \rangle, 
			\langle b_i \rangle 
		\Big) \.
		\NewLine \.
		\exists g \in \Aut_\BOOL(A) : \forall i \in I \. g_{|\langle a_i \rangle} = f_i	
	}
	\NoProof
}
\Page{
	\Theorem{FinitePoUPermutationExtension}
	{
		\forall A : \Homog \.
		\forall P,Q : \PoU \And \Finite(A)  \. \NewLine \.
		\forall \theta : \Bij(P,Q) \.
		\exists f \in \Aut_\BOOL(A) \.
		f_{|P} = \theta
	}
	\NoProof
	\\
	\Theorem{CountablePoUPermutationExtension}
	{
		\forall A : \Homog \And \SA \. \NewLine \.
		\forall P,Q : \PoU \And \TYPE{Countable}(A) \. 
		\forall \theta : \Bij(P,Q) \.
		\exists f \in \Aut_\BOOL(A) \.
		f_{|P} = \theta
	}
	\NoProof
	\\
	\Theorem{FinitePoUPermutationExtension}
	{
		\forall A : \Homog \And \TAlgebra \.
		\forall P,Q : \PoU(A) \. \NewLine \.
		\forall \theta : \Bij(P,Q) \.
		\exists f \in \Aut_\BOOL(A) \.
		f_{|P} = \theta
	}
	\NoProof
}
\newpage
\subsubsection{Support of Endomorphisms}
\Page{
	\DeclareType{Supports}{\prod_{A \in \BOOL} \End_{\BOOL}(A) \to ?A}
	\DefineNamedType{a}{Supports}{a \in \Supp(A,f)}{
		\Lambda f \in \End_{\BOOL}(A) \. 
		\forall b \in \langle a^\c \rangle \.
		f(b) = b
	}
	\\
	\DeclareType{WithSupport}{\prod_{A \in \BOOL} ?\End_{\BOOL}(A)}
	\DefineType{f}{WithSupport}{\exists a \in A : \. a = \min \TYPE{Supports}(A,f) }
	\\
	\DeclareFunc{support}{\prod_{A \in \BOOL}\TYPE{WithSupport}(A) \to A}
	\DefineNamedFunc{support}{f}{\supp f}{ \min \Supp(A,f)}
	\\
	\Theorem{SupportIsPreserved}
	{
		\forall A \in \BOOL \.
		\forall f \in \End_\BOOL(A) \.
		\forall a : \TYPE{Supports}(A,f)
		\Imply
		f(a) = a
	}
	\Say{[1]}{\Elim \End_\BOOL(A,f) \Elim \c \Elim \TYPE{Supports}(A,f,a)(a^\c)}
	{
		f^\c(a) = f(a^\c) = a^\c
	}
	\Conclude{[*]}{[1]^\c}{f(a) = a}
	\EndProof
	\\
	\Theorem{UnderSupportIsPreserved}
	{
		\forall A \in \BOOL \.
		\forall f \in \End_\BOOL(A) \.
		\forall a : \TYPE{Supports}(A,f) \.
		\forall b \in \langle a \rangle \.
		f(b) \in  \langle a \rangle
	}
	\Say{[1]}{\Elim \End_\BOOL(A,f) \Elim \c \Elim \TYPE{Supports}(A,f,a)(a^\c)}
	{
		 f^\c(b) = f(b^\c) \ge  f(a^\c) = a^\c
	}
	\Conclude{[*]}{[1]^\c}{f(b) \le a}
	\EndProof
	\\
	\Theorem{SupportComposition}
	{
		\forall A \in \BOOL \.
		\forall f,g \in \End_\BOOL(A) \.
		\forall a : \TYPE{Supports}(A,  f \And g, a) \. \NewLine \.
		\TYPE{Supports}(A,fg)
	}
	\AssumeIn{b}{\langle a^\c \rangle_\I}
	\Say{[1]}{\Elim \TYPE{Supports}(A,f \And g,a)(b)}
	{
		f(b) = b \And g(b) = b
	}
	\Conclude{[b.*]}{[1.1][1.2]}
	{
		fg(b) = g(b) = b
	}
	\DeriveConclude{[*]}{\Intro \forall \Intro \TYPE{Supports}}
	{
		\TYPE{Supports}(A,fg,a)
	}
	\EndProof
	\\
	\Theorem{SupportIsNonEmpty}
	{
		\forall A \in \BOOL \. 
		\forall f \in \End_\BOOL(A) \.
		\Supp(A,f) \neq \emptyset
	}
	\Assume{b}{\langle e^\c \rangle_\I}
	\Say{[1]}{\Elim \langle e^\c \rangle_\I (b)}{b = 0}
	\Conclude{[b.*]}
	{
		[1] \Elim \End_\BOOL(A,f)\THM{ZeroImage}[1]
	}
	{
		f(b) = f(0) = 0	= b
	}
	\Derive{[1]}{\Intro \Supp}{e \in \Supp(A,f)}
	\Conclude{[*]}{\Intro \TYPE{NonEmpty}[1]}{\Supp(A,f) \neq \emptyset}
	\EndProof
}
\Page{
	\Theorem{SupportIsClosedUnderIntersection}
	{
		\forall A \in \BOOL \.
		\forall f \in \End_\BOOL(A) \.
		\forall a,b \in \Supp(A,f) \. \NewLine \.
		ab \in \Supp(A,f)
	}
	\AssumeIn{c}{\langle(ab)^\c\rangle}
	\Say{[1]}{\Elim \c \Elim \langle (ab)^\c\rangle}{cab = 0}
	\Conclude{[1.*]}{
		\THM{DisjointPairUnionDecomposition}(A,a,b,c)[1]
		\Elim \BOOL(A,a,b)
		\NewLine
		\Elim \Supp(A,f,a)(c \setminus a)
		\Elim \Supp(A,f,b)(c \setminus b)
		\THM{DisjointPairUnionDecomposition}(A,a,b,c)[1]	
	}
	{
		\NewLine :		
		f(c) = 
		f\Big((c\setminus a)   \vee (c \setminus b)\Big) =
		f(c \setminus a)  \vee  f( c \setminus b) = 
		c \setminus a \vee c \setminus b  = 
		c
	}
	\DeriveConclude{[*]}{\Intro \Supp}{ab \in \Supp}
	\EndProof
	\\
	\Theorem{SupportContainsGreater}
	{
		\forall A \in \BOOL \.
		\forall f \in \End_\BOOL(A) \.
		\forall a \in \Supp(A,f) \.
		\NewLine \.		
		\forall b \in A \.
		a \le b \Imply
		b \in \Supp(A,f) 
	}
	\NoProof
	\\
	\Theorem{SupportIsInfClosed}
	{
		\forall A \in \BOOL \.
		\forall f  : \oC(A,A) \.
		\forall B \subset \Supp(A,f) \.
		\forall a \in A \.
		\NewLine \.
		a = \inf B \.
		\Imply
		a \in \Supp(A,f)
	}
	\AssumeIn{c}{\langle a^\c \rangle}
	\Say{[1]}{\THM{PrincipleIdealRepresentation}(A,a,c)\Elim \c}{a \le c^\c}
	\Say{C}{B \vee c^\c}{?A}
	\Say{[2]}{\Elim C \THM{JoinOrder}(A)}{B \le C}
	\Say{[3]}{\THM{SupportContainsGreater}[2]}{C \subset \Supp(A,f)}
	\Say{[4]}{
		\Elim C 
		\THM{LatticeSup}(A)
		[0]
		\THM{GreaterJoin}[1] 	
	}
	{
		\inf C = 
		\inf ( B \vee c^\c) = 
		(\inf B) \vee c^\c =
		a \vee c^\c = 
		c^\c
	}
	\Say{[5]}
	{
		[4] 
		\Elim \oC(A,A,f)
		\THM{SupportIsPreserved}(A,f,C)[2]	
		[4]
	}
	{
		\NewLine :		
		f\Big(c^\c\Big) = 
		f(\inf C) =
		\inf f(C) = 
		\inf C =
		c^\c
	}
	\Conclude{[c.*]}{\Big(\Elim \End_\BOOL(A,f)[5]\Big)^\c}
	{
		f(c) = c
	}
	\DeriveConclude{[*]}{\Intro \Supp}{a \in \Supp(A,f)}
	\EndProof
}\Page{
	\Theorem{InjectiveSupportSwitch}
	{
		\forall A \in \BOOL \.
		\forall f \in \End_\BOOL(A) \.
		\forall \iota \in \End_\BOOL(A) \And \Inj(A,A) \. \NewLine \.
		\forall a \in \Supp(A,f\iota) \.
		f(a) \in \Supp(A,\iota f)
	}
	\AssumeIn{b}{\langle f^\c(a) \rangle}
	\Say{[1]}{\THM{PrincipleIdealRepresentation}\Big(A,f(a),b\Big)}
	{b f(a) = 0}
	\Say{[2]}{ 
		\THM{SupportIsPreserve}(A,f\iota,a) 
		\Elim \End_\BOOL(A,\iota) 
		[1]
		\THM{ZeroHomo}(A,A,\iota)  
	}
	{
		\NewLine :		
		\iota(b) a =  
		\iota(b) f\iota(a) =
		\iota\Big( b f(a) \Big)  =
		\iota(0) =
		0
	}
	\Say{[3]}{\THM{PrincipleIdealRepresentation}[2]}
	{
		\iota(b) \in \langle a^\c \rangle
	}
	\Say{[4]}{\Elim \Supp(A,f,a)[3]}{\iota f \iota(b) = \iota(b)}
	\Conclude{[b.*]}{\Elim \Inj(A,A,\iota)[4]}{\iota f(b) = b}
	\DeriveConclude{[*]}{\Intro \Supp(A,\iota f)}{f(a) \in \Supp(A,\iota f)}
	\EndProof
	\\
	\Theorem{InjectiveSupportReductionByCommutation}
	{
		\forall A \in \BOOL \.
		\forall f \in \End_\BOOL(A) \. \NewLine \.
		\forall \iota \in \End_\BOOL(A) \And \Inj(A,A) \. 
		\iota f = f \iota
		\Imply
		\Big(
		\forall a \in A \.
		\iota(a) \in \Supp(A,f) 
		\Imply
		a \in \Supp(A, f)	
		\Big)
	}
	\AssumeIn{a}{A}	
	\Assume{[1]}{\iota(a) \in \Supp(A,f)}
	\Assume{b}{\langle a^\c \rangle_\I}
	\Say{[2]}{\THM{PrincipleIdealRepresentation}(A,a^\c,b)}{ab = 0}
	\Say{[3]}{
		\THM{ZeroHomo}(A,A,f)
		[2] 
		\Elim \End_\BOOL{A,\iota}
	}
	{ 
			0 = \iota(ab) = \iota(a)\iota(b)
	}
	\Say{[4]}{\THM{PrincipleIdealRepresentation}(A,\iota(a),\iota(b)[3]}
	{
		\iota(b) \in \langle \iota^\c(a) \rangle_\I
	}
	\Say{[5]}{[0]\Elim \Supp\Big(A,f,\iota(a)\Big)[4]}
	{
		f \iota(b) = \iota f (b) = \iota(b)
	}
	\Conclude{[b.*]}{\Elim\Inj(A,A,\iota)[5]}
	{
		f(b) = b
	}
	\DeriveConclude{[*]}{\Intro \Supp(A,f)}
	{
		a \in \Supp(A,f)
	}
	\EndProof
	\\
	\Theorem{CommutationByDisjointSupport}
	{
		\forall A \in \BOOL \.
		\forall f,g \in \End_\BOOL(A) \. 
		\NewLine \.
		\forall a \in \Supp(A,f) \.
		\forall b \in \Supp(A,g) \.
		ab = 0 
		\Imply
		fg = gf
	}
	\SayIn{c}{(a \vee b)^\c}{A}
	\AssumeIn{t}{A}
	\Say{[1]}{[0]\Elim c}
	{
		t = at + bt + ct
	}
	\Say{[2]}{\THM{UnderSupportIsPreserved}(A,a,at)}
	{
		f(at) \le a
	}
	\Say{[3]}{\THM{UnderSupportIsPreserved}(A,b,bt)}
	{
		f(bt) \le b
	}
	\Conclude{[*.1]}
	{
		[1]
		\Elim \End_A(fg)
		\Elim \Supp(A,f) \Elim \Supp(A,g) [2][3] 
		\Elim \End_A(gf)
		[1]	
	}
	{
		\NewLine :		
		fg(t) = 
		fg(at + bt + ct) = 
		fg(at) + fg(bt) + fg(ct) = 
		f(at)  +  g(bt) + ct =
		gf(at) +gf(bt) + gf(ct) = \NewLine =
		gf((at + bt + ct) =
		gf(t)  
	}
	\DeriveConclude{[*]}{\Intro(=,\to)}
	{
		fg = gf
	}
	\EndProof
}
\Page{
	\Theorem{IterativeInjectiveSupport}
	{
		\forall A \in \BOOL \.
		\forall \iota \in \End_\BOOL(A) \And \Inj(A,A) \.
		\forall n \in \Nat \.	
		\NewLine \.
		\forall a \in A \.
		a \in \Supp(A,\iota^n)
		\iff
		\iota(a) \in \Supp(A,\iota^n)
	}
	\Say{[1]}
	{
		\THM{InjectiveSupportReductionByCommutation}(A,\iota,\iota^n,a)
	}
	{
		\iota(a) \in \Supp(A,\iota^n)
		\Imply
		a \in \Supp(A,\iota^n)
	}
	\Say{[2]}
	{
		\THM{InjectiveSupportSwitch}(A,\iota,\iota^{n-1},a)
	}
	{
		a \in \Supp(A,\iota^n)
		\Imply
		\iota(a) \in \Supp(A,\iota^n)
	}
	\Conclude{[*]}{\Intro \iff}
	{
		a \in \Supp(A,\iota^n)
		\iff
		\iota(a) \in \Supp(A,\iota^n)
	}
	\EndProof
	\\
	\Theorem{InverseSupport}
	{
		\forall A \in \BOOL \.
		\forall f \in \Aut_\BOOL(A) \.
		\forall a \in \Supp(A,f) \.
		a \in \Supp(A,f^{-1})
	}
	\AssumeIn{b}{\langle a \rangle_\I}
	\Conclude{[b.*]}
	{
		\Elim \TYPE{Inverse}\Big( \Aut_\BOOL(A), f \Big)
		\Elim \Supp(A,f,a,b)	
	}
	{
		b = 
		ff^{-1}(b) = 
		f^{-1}(b) 
	}
	\Derive{[*]}{\Intro \Supp(A,f)}
	{
		a \in \Supp(A,f^{-1})
	}
	\EndProof
	\\
	\Theorem{EndomorphismSupportImpliesSumBound}
	{
		\forall A \in \BOOL \.
		\forall f \in \End_\BOOL(A) \.
		\forall a \in \Supp(A,f) \. \NewLine \.
		\forall x \in A  \.
		f(x) + x \le a
 	}
 	\SayIn{b}{a^\c}{A}
 	\Say{[1]}{x\THM{LawOfExcludedMiddle}(a)\Elim b}
 	{
 		x = x(a + b) = xa + xb
 	}
 	\Say{[2]}{\THM{UnderSupportIsPreserved}(A,f,a,xa)}
 	{
 		f(xa) \le a
 	}
 	\Say{[*]}
 	{
 		[1]
 		\Elim \Aut_\BOOL(A,f) 
 		\Elim \Supp(A,f,a)\Big(xa^\c\Big)
 		\THM{BooleanRingHasChar2}(A)
 		\THM{BooleanRingIsALattice}(A) 
 		\NewLine
 		\THM{BooleanSumBound}(A)[2]
 	}
 	{
 		f(x) + x  = 
 		f(xa +xa^\c) +  xa +xa^\c = 
 		f(xa) + f(xa^\c) + xa^\c + xa  = \NewLine =
 		f(xa) + xa^\c + xa^\c + xa  = 
 		f(xa)  + xa \le  a
 	}
 	\EndProof
 	\\
 	\Theorem{DecompositionBoundPropertyImplication}
 	{
 		\forall A \in \BOOL \.
 		\forall f \in \End_\BOOL(A) \.
 		\forall a \in A \.
 		\NewLine \.
 		\Big(  \forall x \in A \.  f(x) + x \le a     \Big) 
 		\Imply
 		\Big( \forall x \in A \. f(x)x = 0 \Imply x \le a \Big)
 	}
 	\AssumeIn{x}{A}
 	\Assume{[1]}{f(x)x = 0}
 	\Conclude{[x.*]}{\THM{DisjointSumBound}[1]\Big(A,x,f(x)\Big)[0](x)}
 	{
 		x \le f(x) + x \le a
 	}
 	\Derive{[*]}{\Intro \Imply \Intro \forall}
 	{
 		\forall x \in A \. f(x)x = 0 \Imply x \le a
 	}
 	\EndProof
}\Page{
 	\Theorem{YetAnotherImplication}
 	{
 		\forall A \in \BOOL \.
 		\forall f \in \End_\BOOL(A) \.
 		\forall a \in A \.
 		\NewLine \.
 		\Big( \forall x \in A \. f(x)x = 0 \Imply x \le a \Big)
 		\Imply
 		\Big( \forall x \in A \. x \neq 0 \And x \le a^\c \Imply f(x)x \neq 0 \Big)
 	}
 	\AssumeIn{x}{A}
 	\Assume{[1]}{x \neq 0}
 	\Assume{[2]}{x \le a^\c}
 	\Assume{[3]}{f(x)x = 0}
 	\Say{[4]}{[0](x)[3]}{x \le a}
 	\Say{[5]}{  
 		\THM{BooleanRingIsIsALattice}[2][4]
 		\THM{LawOfExcludedMiddle}(A,a)    
 	}
 	{  x \le a^\c a = 0   }
 	\Say{[6]}
 	{
 		\THM{BooleanRingMinmalElement}[5]
 	}
 	{
 		x = 0
 	}
 	\Conclude{[x.*]}{[6][1]}{\bot}
 	\DeriveConclude{[*]}{\Elim \bot \Intro \Imply \Intro \forall}
 	{
 		\forall x \in A \. x \neq 0 \And x \le a^\c \Imply f(x)x \neq 0
 	}
 	\EndProof
 	\\
 	\Theorem{AutomotphismSupportCondition}
 	{
 		\forall A \in \BOOL \.
 		\forall f \in \Aut_\BOOL(A) \.
 		\forall a \in A \.
 		\NewLine \.
 		\Big( \forall x \in A \. x \neq 0 \And x \le a^\c \Imply f(x)x \neq 0 \Big)
 		\Imply
 		a \in \Supp(A,f)
 	}
 	\AssumeIn{[1]}{a \not \in \Supp(A,f)}
 	\Say{\Big(b,[2]\Big)}{\Elim \Supp(A,f)}
 	{
 		\sum b \in \Big\langle a^\c \Big\rangle_\I \.  f(b) \neq b
 	}
 	\Say{[3]}{\THM{ZeroHomo}[2]}{b \neq 0}
 	\Say{[4]}{\Intro[2][3]}{
			 	b \setminus f(b) \neq 0
			 	\Big|
			 	f(b) \setminus b \neq 0 
 	}
 	\Assume{[5]}{b \setminus f(b) \neq 0}
 	\SayIn{c}{b \setminus f(b)}{A}
 	\Say{[6]}{\Elim c \Elim \setminus \Elim b}
 	{
 		c = b \setminus f(b) \le b \le a^\c
 	}
 	\Say{[7]}{[0](c)[5][6]\Elim c \Elim \setminus}
 	{  
 		0 \neq c f(c) = 
 		\Big(b \setminus f(b)\Big) \Big(f(b) \setminus f^2(b) \Big)  = 0
 	}
 	\Conclude{[5.*]}{
 		\Intro \bot [7]
 	}
 	{
		 	\bot
 	}
 	\Derive{[5]}{\Elim \bot }
 	{
			 b \setminus f(b) = 0 	
 	}
 	\Say{[6]}{\Elim | [5]}{f(b) \setminus b \neq 0}
 	\Say{[7]}{\Elim \Aut_\Bool(A,f^{-1})[6]}{b \setminus f^{-1}(b) \neq 0 }
 	\Say{[8]}{\Elim \setminus \Elim b }
 	{
 		 b \setminus f^{-1}(b) \le  b \le a^\c
 	}
 	\Say{[9]}{[0][7][8]}
 	{
 		  0 \neq \Big( b \setminus f^{-1}(b) \Big)
 		 \Big( f(b) \setminus b \Big) = 0
 	}
 	\Conclude{[1.*]}{\Intro \bot [9]}{\bot}
 	\DeriveConclude{[*]}{\Elim \bot}{a \in \Supp(A,f)}
 	\EndProof
 }\Page{
 	\Theorem{SuppConjugation}
 	{
		\forall A \in \BOOL \.
		\forall g \in \End_\BOOL(A) \.
		\forall f \in \Aut_\BOOL(A) \. 
		\NewLine \.
		\forall a \in \Supp(A,g) \.
		f(a) \in \Supp\Big(A,f^{-1}gf \Big)	
 	}
 	\AssumeIn{b}{\Big\langle f^\c(a) \Big\rangle_\I}
	\Say{[1]}{
		\THM{PrincipleIdealStrucure}\Big(A,f(a),b\Big)
		\Elim \End_\BOOL(A,f)
	}
	{
			b \le f^\c(a) = f\Big( a^\c \Big) 
	}
	\Say{[2]}{ \THM{MonotonicBooleanMorphism}\Big(A,f^{-1}\Big)[1]}
	{
		f^{-1}(b) \le a^\c	
	}
	\Say{[3]}{\Elim \Supp(A,g,a)\Big( f^{-1}(b) \Big)}
	{
		f^{-1}g(b) = f^{-1}(b)
	}
	\Conclude{[b.*]}{\Aut_\BOOL(A,f)[4]}
	{
		f^{-1}gf(b) = b
	}
	\DeriveConclude{[*]}{\Intro \Supp}
	{
		f(a) \in \Supp\Big( A, f^{-1}gf \Big)
	}
	\EndProof
	\\
	\Theorem{SuppConjugationEq}
 	{
		\forall A \in \BOOL \.
		\forall g,f \in \Aut_\BOOL(A) \. 
		\forall h \in \End_\BOOL(A) \.
		\NewLine \.
		\forall a \in \Supp(A,g) \.
		g_{|\langle a \rangle_\I} = f_{|\langle a \rangle_\I}
		\Imply
		f^{-1}hf = g^{-1}hg
 	}
 	\NoProof
 	\\
 	\Theorem{ContinuousHaveSupport}
 	{
		\forall A  : \TAlgebra \.
		\forall f : \End_\BOOL(A) \And \oC(A,A) \. \NewLine \.
		\TYPE{WithSupport}(A,f)
 	}
 	\NoProof
 	\\
 	\Theorem{SupportIsPreserved2}
 	{
 		\forall A : \TAlgebra \.
 		\forall f : \Aut_\BOOL(A) \. 
 		f(\supp f) = \supp f
 	}
 	\NoProof
 	\\
 	\Theorem{SupportBySums}
 	{
 		\forall A : \TAlgebra \.
 		\forall f : \Aut_\BOOL(A) \. 
 		\supp f = \sup \Big\{ a + f(a)  \Big|   a \in A   \Big\}
 	}
 	\NoProof
 	\\
 	\Theorem{SupportByZeroImages}
 	{
 		\forall A : \TAlgebra \.
 		\forall f : \Aut_\BOOL(A) \. 
 		\supp f = \sup \Big\{  a \in A : a + f(a) = 0   \Big\}
 	}
 	\NoProof
}
\Page{
	\Theorem{SupportOfInverse}
 	{
 		\forall A : \TAlgebra \.
 		\forall f : \Aut_\BOOL(A) \. 
 		\supp f^{-1} = \supp f
 	}
 	\NoProof
 	\\
 	\Theorem{SupportOfTheConjugate}
 	{
 		\forall A : \TAlgebra \.
 		\forall f,g : \Aut_\BOOL(A) \. 
 		\supp f^{-1}gf =  f(\supp g)
 	}
 	\NoProof
}
\newpage
\subsubsection{Periodic and Aperiodic Parts Theorem}
\Page{
	\DeclareType{Periodic}{\prod_{A \in \BOOL} ?\Aut_\BOOL(A)}
	\DefineType{f}{Periodic}
	{
		A \neq \emptyset \And (
		\exists n \in \Nat :  f^n = \id 
		\And 
		\forall i \in [1,\ldots,n-1] \.		
		\supp f^i = e 	)
	}
	\\
	\DeclareFunc{period}{\prod_{A \in \BOOL} ?\Aut_\BOOL(A)}
	\DefineNamedFunc{period}{f}{\pi(f)}
	{
		\min (\exists n \in \Nat :  f^n = \id) 	
	}
	\\
	\DeclareType{Aperiodic}{\prod_{A \in \BOOL} ?\Aut_\BOOL(A)}
	\DefineType{f}{Aperiodic}
	{
		\forall n \in \Nat \. \supp \; f^n = e 	
	}
	\\
	\DeclareType{WithAllSupports}
	{
		\prod_{A \in \BOOL} ?\Big( \End_\BOOL(A) \And \TYPE{Injective}(A,A) \Big)
	}
	\DefineType{f}{WithAllSupports}
	{
		\forall n \in \Nat \.  \TYPE{WithSupport}(A,f^n)
	}
}\Page{
	\Theorem{PeriodicAperiodicPartsTHM}
	{
		\forall A : \SA \.
		\forall f : \TYPE{WithAllSupports} \.
		\exists p : \Inj\Big( \omega_0 + 1, A  \Big) :
		\NewLine :
		\PoU(A,\im p) 
		\And
		\Big(
			\forall k \in \omega_0 + 1 \.
			f(p_k) \le p_k
		\Big)
		\And
		\NewLine
		\And
		\bigg(
			\forall n \in \Int_+ \. 
			p_n \neq 0 \Imply
			\TYPE{Periodic}\Big(
				\langle p_n \rangle_\I, 
				f_{|\langle p_n \rangle_\I} 
			\Big)
			\And
			\pi(f_{|\langle p_n \rangle_\I}) = n
		\bigg)
		\And
		\TYPE{Aperiodic}
		\Big(
			\langle p_{\omega_0} \rangle_\I, 
			f_{|\langle p_{\omega_0} \rangle_\I} 
		\Big)
	}
	\Say{p}
	{
		\THM{BoundedTransfiniteInduction}
		\Bigg(
			\omega_0 + 1,
			\neg \supp f, \NewLine ,
			\Lambda n \in \omega_0 \. \Lambda a :[1,\ldots,n-1] \to A \.
				\bigwedge^n_{i=1} (\neg a_i) \setminus \supp f^n,
			\Lambda k  : \TYPE{Limit} \. \Lambda k \to A \.
			\inf_{n \in k} \supp f^n
		\Bigg)
	}
	{
		(\omega_0 + 1) \to A
	}
	\Say{P}{\im p}{?A}
	\Say{[*.1]}{ \Elim P \Elim p \Intro \PoU}{\PoU(A,P)}
	\Say{[2]}
	{
		\Lambda n \in \Nat \.		
		\Elim p \THM{IterativeInjectiveSupport}(A,f,p_{n-1})
		\THM{SupportIsPreserved}(A,f^n)
	}
	{
		\NewLine		
		\forall n \in \Nat \.  f(p_n) = p_n
	}
	\Say{[3]}{ \Elim p_{\omega_1} \Elim \inf}
	{
		\forall n \in \Nat \. p_{\omega_0} \le \supp f^n 
	}
	\Say{[4]}{\THM{UnderSupportIsPreserved}[3]}
	{
		\forall n \in \Nat \. f(p_{\omega_0}) \le \supp f^n
	}
	\Say{[5]}{\Elim \inf \Elim p_{\omega_0} [4] }
	{
			f(p_{\omega_0}) \le p_{\omega_0}
	}
	\Say{[*.2]}{[2][5]}
	{
		\forall n \in \omega_0 + 1 \. f(p_n) \le p_n 
	}
	\AssumeIn{n}{\Nat}
	\Assume{[00]}{p_n \neq 0}
	\AssumeIn{a}{\langle p_n \rangle_\I}
	\Assume{[000]}{a \neq 0}
	\Say{[a.*.2]}{\Elim p_n \Elim a}{\forall k \in [1,\ldots,n-1] \. f^{k}(a) \neq a}
	\Say{[7]}{\Elim p_n \Elim a \Elim \Intro \supp f^n}{a \supp f^n = 0}
	\Conclude{[a.*.1]}{\Elim \Supp [7]}{ f^n(a) = a}
	\Derive{[6]}{\Intro \forall }{
		\forall a \in \langle p_n \rangle_\I \.
		a \neq 0 \Imply
		\forall k \in [1,\ldots,n-1] \. f^{k}(a) \neq a
		\And 
		f^n(a) = a
	}
	\Say{[n.*.1]}{\Intro \TYPE{Periodic}[6][00]}
	{
		\TYPE{Periodic}\Big(
			\langle p_n \rangle_\I,
			f^n_{|\langle p_n \rangle_\I}
			\Big)
	}
	\Conclude{[n.*.2]}{[n.*.1][6]\Intro \pi}{\pi(f_{|\langle p_n \rangle_\I}) = n}
	\Derive{[*.3]}{\Intro \Imply \Intro \forall}
	{
		\forall n \in \Nat \. 
			p_n \neq 0 \Imply
			\TYPE{Periodic}\Big(
				\langle p_n \rangle_\I, 
				f_{|\langle p_n \rangle_\I} 
			\Big)
			\And
			\pi(f_{|\langle p_n \rangle_\I}) = n
	}
	\Conclude{[*.4]}{\Elim \supp \Elim p_{\omega_0} \Intro \TYPE{Aperiodic}}
	{
		\TYPE{Aperiodic}
		\Big(
			\langle p_{\omega_0} \rangle_\I, 
			f_{|\langle p_{\omega_0} \rangle_\I} 
		\Big)
	}
	\EndProof	
}
\newpage
\subsubsection{Full Subgroups}
\Page{
	\DeclareType{\FS}{\prod_{A \in \BOOL} \TYPE{Subgroup}\Big(\Aut_\BOOL(A)\Big)}
	\DefineType{G}{\FS}
	{
		\forall I \in \SET \.
		\forall a : \Inj(I,A) \.		
		\forall f : I \to G \.
		\NewLine \.		
		\forall [0] : \PoU(A,\im a)  \.
		\forall g \in \Aut_\BOOL(A) \. \NewLine \.
		\forall [00] : \forall b \in A \. 
		\Big(			
			(\exists i \in I : b \le a_i)
			\Imply
			g(b) = f_i(b)  
		\Big)
		\Imply
		g \in G
	}
	\\
	\DeclareType{\CFS}{\prod_{A \in \BOOL} \TYPE{Subgroup}\Big(\Aut_\BOOL(A)\Big)}
	\DefineType{G}{\CFS}
	{
		\forall I : \TYPE{Countable} \.
		\forall a : \Inj(I,A) \.		
		\forall f : I \to G \.
		\NewLine \.		
		\forall [0] : \PoU(A,\im a)  \.
		\forall g \in \Aut_\BOOL(A) \. \NewLine \.
		\forall [00] : \forall b \in A \. 
		\Big(			
			(\exists i \in I : b \le a_i)
			\Imply
			g(b) = f_i(b)  
		\Big)
		\Imply
		g \in G
	}
	\\
	\DeclareFunc{generateFullSubgroup}
	{
		\prod_{A \in \BOOL} 
		?\Aut_\BOOL(A) \to \FS(A)
	}
	\DefineNamedFunc{generateFullSubgroup}{X}
	{
		\genFS{X}
	}
	{
		\bigcap \Big\{ G : \FS(A) : X \subset G \Big\}
	}
	\\
	\DeclareFunc{generateCoubtablyFullSubgroup}
	{
		\prod_{A \in \BOOL} 
		?\Aut_\BOOL(A) \to \CFS(A)
	}
	\DefineNamedFunc{generateCountablyFullSubgroup}{X}
	{
		\genCFS{X}
	}
	{
		\bigcap \Big\{ G : \CFS(A) : X \subset G \Big\}
	}
	\\
	\Theorem{FullSubgroupGeneratedByGroupExpression}
	{
		\forall A \in \BOOL \.
		\forall G \subset_\GRP \Aut_\BOOL(A) \.
		\NewLine \.
		\genFS{G} =
		\Big\{
			f \in  \Aut_\BOOL(A) :
			\forall a \in A \setminus \{0\} .
			\exists b \in \langle a \rangle_\I \setminus \{0\} :
			\exists g \in G :
			\forall c \in \langle b \rangle_\I  \.
			f(c) = g(c)  
		\Big\}	
	}
	\Say{H}
	{
		\Big\{
			f \in  \Aut_\BOOL(A) :
			\forall a \in A \setminus \{0\} .
			\exists b \in \langle a \rangle_\I \setminus \{0\} :
			\exists g \in G :
			\forall c \in \langle b \rangle_\I  \.
			f(c) = g(c)  
		\Big\}
	}
	{
		? \Aut_\BOOL(A)
	}
	\AssumeIn{h,h'}{H}
	\AssumeIn{a}{A}
	\Assume{[1]}{a \neq 0}
	\Say{\Big(b, g, [2]\Big)}
	{
		\Elim H(h,a)[1]
	}
	{
		\sum b \in \langle a \rangle_\I \. 
		\sum g \in G \.		
		b \neq 0 \And  
		\forall c \in \langle b \rangle_\I \.
		h(c) = g(c)
	}
	\Say{[3]}{\Elim \End_\BOOL(A,g)[2.1]}{g(b) \neq 0}
	\Say{\Big(b', g', [4]\Big)}	
	{
		\Elim H\Big(h',g(b)\Big)[3]
	}
	{
		\sum b' \in \Big\langle g(b) \Big\rangle_\I \. 
		\sum g' \in G \.		
		b' \neq 0 \And  
		\forall c \in \langle b' \rangle_\I \.
		h'(c) = g'(c)
	}
	\AssumeIn{c}{\langle b \rangle_\I}
	\Say{[5]}{
		\THM{BooleanMorphismIsMonotonic}(A,A,g)
		\THM{PrincipleIsealExpression(A,b,c)}
	}
	{
		g(c) \in \Big\langle g(b) \Big\rangle_\I 
	}
	\Conclude{\Big[h,h'\Big].*}{[4.2][5][2.2](c)}{gg'(c) = gh'(c) = hh'(c)}
	\Derive{[1]}{\Intro \forall \Intro^2 \exists \Intro \forall\Intro H \Intro \forall}
	{
		\forall h,h' \in H \. hh' \in H
	}
}\Page{
	\AssumeIn{h}{H}
	\AssumeIn{a}{A}
	\Assume{[2]}{a \neq 0}
	\Say{[3]}{\Elim \End_\BOOL(A,h^{-1})[2]}{h^{-1}(a) \neq 0}
	\Say{\Big(b, g, [4]\Big)}
	{
		\Elim H\Big(h,h^{-1}(a)\Big)[2]
	}
	{
		\sum b \in \Big\langle h^{-1}(a) \Big\rangle_\I \. 
		\sum g \in G \.		
		b \neq 0 \And  
		\forall c \in \langle b \rangle_\I \.
		h(c) = g(c)
	}
	\Say{[h.*.1]}{\THM{BoolenMorphismIsMonotonic}(A,h)}{h(b) \le a}
	\Say{[h.*.2]}{\Elim \End_\BOOL(A,h)[4.1]}
	{
		h(b) \neq 0
	}
	\AssumeIn{c}{\langle h(b) \rangle_\I}
	\Say{[5]}{
		\THM{PrincipleIdealStructur}\Big(A,h(b),c\Big)
		\THM{BooleanMorphismIsMonotonic}\Big(A,A,h^{-1}\Big)	
	}{h^{-1}(c)  \le b}
	\Say{[6]}{[4.2][5]\Elim \TYPE{Inverse}}
	{
		h^{-1}g(c) = h^{-1}h(c) = c
	}
	\Conclude{[h.*]}{g^{-1}[6]}
	{
		h^{-1}(c) = g^{-1}(c)
	}
	\Derive{[2]}{\Intro \forall \Intro^2 \exists \Intro \forall\Intro H \Intro \forall}
	{
		\forall h \in H \. h^{-1} \in H
	}
	\Say{[3]}{\Intro \GRP[1][2]}{ H \in \GRP}
	\AssumeIn{I}{\SET}
	\Assume{a}{\Inj(I,A)}
	\Assume{h}{I \to H}
	\Assume{[4]}{\PoU(A,\im a)}
	\AssumeIn{f}{\Aut_\BOOL(A)}
	\Assume{[5]}
	{
		\forall b \in A \. 
		(\exists i \in I : b < a_i) \Imply f(b) = h_i(b)
	}
	\AssumeIn{b}{A}
	\Assume{[6]}{b \neq 0}
	\Say{\Big(i,[7]\Big)}{\Elim \PoU[4](b)[6]}
	{
		\sum_{i \in I} a_ib \neq 0
	}
	\Say{[8]}{\THM{LatticeMeetsIneq}(A,a_i,b)}{a_ib \le a_i }
	\Say{\Big(b', g, [9]\Big)}
	{
		\Elim H\Big(h_i,a_ib\Big)[7]
	}
	{
		\sum b' \in \langle a_ib \rangle_\I \. 
		\sum g \in G \.		
		b' \neq 0 \And  
		\forall c \in \langle b' \rangle_\I \.
		h_i(c) = g(c)
	}
	\Say{[b.*.1]}{ 
		\THM{PrincipleIdealStructure}(A, a_ib,b' ) 
		\THM{LatticeJoinIneq}(A,b',a_i)   
	}
	{
		b' \le a_ib \le b
	}
	\AssumeIn{c}{\langle b' \rangle_\I }
	\Say{[10]}{
		\THM{PrincipleIdealStructure}(A, b',c )
		\THM{PrincipleIdealStructure}(A, a'b,b' ) 
		\THM{LatticeJoinIneq}(A,a_i,b)
	}{
		\NewLine :		
		c \le b' \le a'b \le a' 
	}
	\Conclude{[b.*.1]}{[5](c)[10][8](c)}{
			f(c) = h_i(c) = g(c)
	}
	\DeriveConclude{[I.*]}{\Intro H}{f \in H}
	\Derive{[4]}{\Intro \FS}{\FS(A,H)}
	\Say{[5]}{\Elim \genFS{G}[4]}{\genFS{G} \subset H}
	\Assume{F}{\FS(A)}
	\Assume{[6]}{G \subset F}
	\AssumeIn{h}{H}
	\Say{D}{  
		\Big\{
			 b \in A :
			\exists g \in G :
			\forall c \in \langle b \rangle_\I  \.
			f(c) = g(c)  
		\Big\}
	}
	{
		? A
	}
	\Say{[7]}{\Elim H(h) \Elim D \Intro \OD}
	{
		\OD(A,D)
	}
	\Say{\Big(P,[8]\Big)}
	{
		\THM{OrderDenseContainsPartitionOfUnity}[7]
	}
	{
		\sum P : \PoU(A) \. P \subset A 
	}
	\Say{\Big(g,[9]\Big)}{\Elim P \Elim H}
	{
		\sum g : P \to G \.
		\forall p \in P \.
		\forall c \in \langle p \rangle_\I \. g_p(c) = h(c)			
	}
	\Say{[10]}{\Elim g [6]}{\forall p \in P \. g_p \in F}
	\Conclude{[h.*]}{\Elim \FS(A,F)(P,P,g,h)[10][9]}{h \in F}	
}
\Page{
	\DeriveConclude{[F.*]}{\Intro \subset}{H \subset F}
	\Derive{[6]}{\Intro \Imply \Intro \forall }
	{
		\forall F : \FS(A) \. G \subset F \Imply H \subset F
	}
	\Say{[7]}{\Elim \genFS{G}}
	{
		H \subset \genFS{G}
	}
	\Conclude{[*]}{\Intro \TYPE{SetEq}[5][7]}{\genFS{G} = H}
	\EndProof
	\\
	\Theorem{CountablyFullSubgroupGeneratedByGroupElement}
	{
		\forall A : \SA \.
		\forall g \in \Aut_\BOOL(A) \.
		\NewLine \.
		\genCFS{g } =
		\Big\{
			f \in  \Aut_\BOOL(A) :
			\exists p : \Int \to A : \PoU(A,\im p) 
			\And \NewLine \And
			\forall n \in \Nat \. 
			\forall b \in \langle p_n \rangle_\I  \.
			f(b) =  g^n(b)  
		\Big\}	
	}
	\Say{H}
	{
		\Big\{
			f \in  \Aut_\BOOL(A) :
			\exists p : \Int \to A : \PoU(A,\im p) 
			\And \NewLine \And
			\forall n \in \Nat \. 
			\forall b \in \langle p_n \rangle_\I  \.
			f(b) =  g^n(b)  
		\Big\}	
	}{ ? \Aut_\BOOL(A)}
	\Assume{h,h'}{H}
	\Say{\Big(p,[1]\Big)}{\Elim H(h)}
	{
		\sum p : \Int \to A \.
		\PoU(A,\im p) 
		\And
		\forall n \in \Nat \.
		\forall b \in \langle p_n \rangle_\I \.
		h(b) = g^n(b)
	}
	\Say{\Big(p',[2]\Big)}{\Elim H(h')}
	{
		\sum p' : \Int \to A \.
		\PoU(A,\im p') 
		\And
		\forall n \in \Nat \.
		\forall b \in \langle p_n' \rangle_\I \.
		h'(b) = g^n(b)
	}
	\Say{p''}{\Lambda n \in \Int \. \bigvee_{n = l + k} p_l \wedge h^{-1}(p_k')}
	{
		\Int \to A
	}
	\Assume{n,m}{\Int}
	\Assume{[3]}{n \neq m}
	\Assume{k,l,t,s}{\Int}
	\Assume{[4]}{n = k + l}
	\Assume{[5]}{m = t + s}
	\Say{[6]}{[3][4][5]}{k \neq t \Big| l \neq s }
	\Conclude{\Big[(k,l,t,s).*\Big]}
	{
		\Elim \PD(A,\im p \And \im p')[6]	
	}{
		p_l h^{-1}(p'_k) p_s h^{-1}(p'_t) = 0
	}
	\Derive{[4]}{\Intro \forall \Intro^2 \Imply}
	{
		\forall k,l,t,s \in \Int \. 
		(n = k + l \And m=t + s)
		\Imply 
		p_l  h^{-1}(p'_k) p_s h^{-1}(p'_t) = 0 	
	}
	\Conclude{\Big[(n,m).*]}
	{
		\Elim p''
		\Elim \TYPE{DistributiveLattice}(A)
		[4]
		\Elim \sup
	}
	{
		\NewLine :		
		p''_n p''_m  =
		\bigvee_{n = l + k} p_l h^{-1}(p_k') \bigvee_{m = t + s} p_t h^{-1}(p_s')= 
		\bigvee_{n=l+k,m=t + s} p_l h^{-1}(p'_k)  p_s h^{-1}(p'_t) =
		\bigvee_{n=l+k,m=t + s} 0 =
		0
	}
	\Derive{[3]}{\Intro \PD}{\PD(A,\im p'')}
	\AssumeIn{a}{A}
	\Assume{[4]}{a \neq 0}
	\Say{\Big(n,[5]\Big)}{\Elim \OD(A,\im p)\Big(a,[4]\Big)}
	{
		\sum n \in \Int \. p_n a \neq 0
	}
	\Say{\Big(m,[6]\Big)}{\Elim \OD(A,h^{-1}\im p')\Big(p_n a,[5]\Big)}
	{
		\sum m \in \Int \. h^{-1}(p_m') p_n a \neq 0
	}
	\Conclude{[7.*]}{
		\Elim p''
		\THM{OrderContinuousMult}(A)
		\Elim \sup
		[6]	
	}
	{
		\NewLine		
		p_{m + n}''a =
		\left(\bigvee_{n + m = l + k} p_l h^{-1}(p_k')\right)a =
		\bigvee_{n + m = l + k} p_l h^{-1}(p_k') a \ge 
		p_n g^{-1}(p_m') a > 0
	}
	\Derive{[4]}{\Intro \OD}{\OD(A,\im p'')}
	\Say{[5]}{\THM{PoUIffODAndDisjoint}[3][4]}{\PoU(A,\im p'')}
}\Page{
	\AssumeIn{k,l}{\Int}
	\AssumeIn{c}{\langle h^{-1}(p_k)p_l \rangle_\I}
	\Say{[7]}{
		\THM{PrincipleIdealExpression}( h^{-1}(p_k)p_l )
		\THM{LatticeJointIneq}(A,p_l,cp'_k)}
	{
		c \le  h^{-1}(p'_k) p_l \le p_l
	}
	\Say{[8]}{
		\THM{PrincipleIdealExpression}( h^{-1}(p_k)p_l )
		\THM{BoooleanMorphismIsMonotonic}(A,A,h)
		\NewLine
		\THM{LatticeJointIneq}(A,p_l,cp'_k)	
	}
	{
			h(c) \le  p'_k h(p_l) \le p'_k
	}
	\Conclude{\Big[(l,k).*\Big]}{
		[2](k)[8]
		[1](l)[9]
		\THM{ExpMult}\Big(\End_\BOOL(A)\Big)
	}
	{
		hh'(c) 
		= hg^k(c) 
		= g^lg^l(c) 
		= g^{l +k}(c)
	}
	\Derive{[6]}{\Intro \forall \Intro \forall}
	{
		\forall k,l \in \Int \.
		\forall c \in  \langle h^{-1}(p_k)p_l \rangle_\I \.
		hh'(c) = g^{l +k}(c)
	}
	\Assume{n}{\Nat}
	\Assume{c}{\langle p''_n \rangle_\I}
	\Say{[7]}{\THM{PrincipleIdealExpression}(A,p''_n,c)}{c \le p''_n}	
	\Conclude{[n.*]}
	{
		\Elim \TYPE{BooleanOrder}(A)[7]
		\Elim p''_n
		\Elim \oC(A,A,hh')\THM{MultiplicationIsOrderContinuous}(A)
		\NewLine
		[6] \THM{LatticeJoinIneq}
		\Elim \oC(A,A,g^{k+l})\THM{MultiplicationIsOrderContinuous}(A)	
		\Intro p''_n \NewLine
		\Elim \TYPE{BooleanOrder}(A)[7]
	}
	{
		\NewLine :		
		hh'(c)  =
		hh'( cp''_n)=
		hh'\left(
			c \left( 
				\bigvee_{k+l =n} h^{-1}(p'_k)p_l
			\right) 
		\right) =
		\bigvee_{k+l =n} hh'\Big(ch^{-1}(p'_k)p_l\Big) = 
		\bigvee_{k+l =n} g^{k+l}\Big(ch^{-1}(p'_k)p_l\Big) = \NewLine =
	    g^{k+l}\left(  c\bigvee_{k+l =n} h^{-1}(p'_k)p_l  \right) =
		g^{k+l}(cp''_n) =
		g^{k+1}(c)
	}
	\DeriveConclude{\Big[ (h,h').*\Big]}{\Intro H}{hh' \in H}
	\Derive{[1]}{\Intro \forall}{\forall h,h' \in H \. hh' \in H}
	\AssumeIn{h}{H}
	\Say{\Big(p,[2]\Big)}{\Elim H(h)}
	{
		\sum p : \Int \to A \.
		\PoU(A,\im p) 
		\And
		\forall n \in \Nat \.
		\forall b \in \langle p_n \rangle_\I \.
		h(b) = g^n(b)
	}
	\Assume{n}{\Nat}
	\Assume{c}{\Big\langle h(p_n) \Big\rangle }
	\Say{[3]}{
		\THM{PrincipleIdealStructru}(A,p_n,c)
		\THM{BooleanMorphismIsMonotonic}(A,A,h^{-1})
	}
	{
		h^{-1}(c) \le p_n
	}
	\Say{[4]}{
		\Elim \TYPE{Inverse} 		
		[2](n)[3]
	}
	{
		  c = hh^{-1}(c)   = g^n h^{-1}(c)	
	}
	\Conclude{[n.*]}
	{
		\Elim \Aut_\BOOL(A)(g^n)[4]
	}
	{
		h^{-1}(c) = g^{-n}(c)
	}
	\DeriveConclude{[h.*]}{\Intro h}{h^{-1} \in H}
	\Derive{[2]}{\Intro \forall}{
		\forall h \in H \.
		h^{-1} \in H 
	}
	\Say{[3]}{\Intro \GRP [2][3]}{H \in \GRP}
	\Assume{I}{\TYPE{Countable}}
	\Assume{a}{\Inj(I,A)}
	\Assume{h}{I \to H}
	\Assume{[4]}{\PoU(A,\im a)}
	\AssumeIn{f}{\Aut_\BOOL(A)}
	\Assume{[5]}
	{
		\forall b \in A \. 
		(\exists i \in I : b < a_i) \Imply f(b) = h_i(b)
	}
	\AssumeIn{i}{I}
	\Say{\Big(p^i,[6]\Big)}{\Elim H(h_i)}
	{
		\sum p^i : \Int \to A \.
		\PoU(A,\im p^i) 
		\And
		\forall n \in \Nat \.
		\forall b \in \langle p_n^i \rangle_\I \.
		h_i(b) = g^n(b)
	}
	\Conclude{q_i}{\Lambda n \in \Int \. a_ip_n^i}{\Int \to A}
	\Derive{q}{\Intro(\to)}{I \to \Int \to A}
	\Say{p}{\Lambda n \in \Int \. \bigvee_{i \in I} q_{i,n}}{\Int \to A}
	\Say{[6]}{\Elim p \Elim \PoU(A,\im a) }
	{
		\PoU(A,\im p)
	}
}\Page{	
	\AssumeIn{n}{\Int}
	\AssumeIn{I}{I}
	\AssumeIn{c}{\langle p_n^i a_i \rangle_\I}
	\Say{[7]}
	{
		\THM{PrincipleIdealStructue}(A, p_n^i a_i, c)
		\THM{JoinIneq}(A,a_i,p_n^i)	
	}{c \le p_n^i a_i \le a_i}	
	\Say{[8]}
	{
		\THM{PrincipleIdealStructue}(A, p_n^i a_i, c)
		\THM{JoinIneq}(A,p_n^i, a_i)	
	}{c \le p_n^i a_i \le p_n^i}	
	\Conclude{[n.*]}
	{
		[5](c)[7]\ldots
	}
	{	
		f(c) =  h_i(c) =  g^n(c)
	}
	\Derive{[7]}{\Intro^3 \forall}
	{
		\forall n \in \Int \. 
		\forall i \in I \.
		\forall c \in  \langle p_n^i a_i \rangle_\I \.
		g(c) = g^n(c)
	}
	\AssumeIn{n}{\Int}
	\AssumeIn{c}{\langle p_n \rangle_\I}
	\Conclude{[n.*]}
	{
		\Elim \TYPE{BooleanOrder}(A)\Elim c
		\Elim p_n
		\Elim \oC(A,A,f')\THM{MultiplicationIsOrderContinuous}(A)
		\NewLine
		[7] \THM{LatticeJoinIneq}
		\Elim \oC(A,A,g^{n})\THM{MultiplicationIsOrderContinuous}(A)	
		\Intro p_n \NewLine :
		\Elim \TYPE{BooleanOrder}(A)\Elim c
	}
	{
		f(c) =
		f(cp_n) =
		f\left( c \bigvee_{i \in I} p_n^i a_i \right) = 
		\bigvee_{i \in I}  f\left( c  p_n^i a_i  \right) = 
		\bigvee_{i \in I}  f\left( c  p_n^i a_i  \right) = 
		\bigvee_{i \in I} g^n \left( c  p_n^i a_i  \right) = \NewLine =
		g^n\left(  c \bigvee_{i \in I} p_n^i a_i   \right) =
		g^n(cp_n) =
		g^n(c)
	}
	\DeriveConclude{[I.*]}{\Intro H}{f \in H}
	\Derive{[4]}{\Intro \CFS}{\CFS(A,H)}
	\Say{[5]}{\Intro \genCFS{g}[4]}{\genCFS{g} \subset H}
	\Assume{G}{\CFS(A)}
	\Assume{[6]}{g \in G}
	\AssumeIn{h}{H}
	\Say{\Big(p^i,[7]\Big)}{\Elim H(h)}
	{
		\sum p : \Int \to A \.
		\PoU(A,\im p) 
		\And
		\forall n \in \Nat \.
		\forall b \in \langle p_n \rangle_\I \.
		h(b) = g^n(b)
	}
	\Conclude{[G.*]}{\Elim \CFS(A,G) [6][7]}{h \in G}
	\Derive{[6]}{\Intro \genCFS{g}}{H \subset \genCFS{g} }
	\Conclude{[*]}{\Intro \TYPE{SetEq}[5][6]}{H = \genCFS{g} }
	\EndProof
	\\
	\Theorem{FullSubgroupGeneratedByGroupElement}
	{
		\forall A \in \BOOL \.
		\forall g \in \Aut_\BOOL(A) \.
		\NewLine \.
		\genFS{g } =
		\Big\{
			f \in  \Aut_\BOOL(A) :
			\forall a \in A \setminus \{0\} \.
			\exists b : \langle a \rangle_\I \setminus \{0\} :
			\exists n \in \Int :
			\forall c \in \langle a \rangle_\I \.
			f(c) =  g^n(c)  
		\Big\}	
	}
	\NoProof
}\Page{
	\Theorem{CompleteAlgebraFullGroupIndifference}
	{
		\forall A : \TAlgebra \.
		\forall g \in \Aut_\BOOL(A) \.
		\genFS{g} = \genCFS{g}
	}
	\Say{[1]}{\Elim \genFS{g} \Elim \genCFS{g} \Intro \subset}{\genCFS{g} \subset \genFS{g}}
	\AssumeIn{f}{\genFS{g}}
	\Say{D}{ 
		\Big\{ 
			a \in A : 
			\exists n \in \Int : 
			\forall b \in \langle a \rangle_\I \.
			f(b) = g^n(b)   
		\Big\}     
	}{?A}
	\Say{[2]}{\THM{FullSubgroupGeneratedByGroupElement}(A,g,f) \Elim D \Intro \OD}
	{\OD(A,D)}
	\Say{\Big(P,[3]\Big)}{\THM{OrderDenseContainsPartitionOfUnity}}
	{
		\sum P : \PoU(A) \. P \subset D
	}
	\Say{p}
	{		\Lambda n \in \Int \.  
			\bigvee \Big\{ q \in P : \forall b \in \langle q \rangle_\I \.
			f(b) = g^n(b)    \Big\} 
	}{\Int \to A}
	\Say{[4]}{\Elim \PoU(A,P) \Elim p \THM{MultIsOrderC}(A) \Intro \PoU}
	{
		\PoU(A,\im p)
	}
	\Say{[5]}{\Lambda n \in \Int \. \Elim p \Elim \oC(A,A,f \And g^n)}
	{
		\forall n \in \Nat \. \forall b \in \langle p_n \rangle_\I \.
		 f(b) = g^n(b)
	}
	\Conclude{[f.*]}{\THM{CountablyFullSubgroupGeneratedByGroupElement}(A)[4][5]}
	{
		f \in \genCFS{g}
	}
	\Derive{[2]}{\Intro \subset }{\genFS{g} \subset \genCFS{g}}
	\Conclude{[*]}{\Intro \TYPE{SetEq}[1][2]}{\genFS{g} = \genCFS{g}}
	\EndProof
	\\
	\Theorem{CompleteAlgebraElementsFullGroupSuppExpression}
	{	
		\NewLine
		::
		\forall A : \TAlgebra \.
		\forall g \in \Aut_\BOOL(A) \.
		\genFS{g} = \left\{ 
			f \in \Aut_\BOOL(A) :     \bigwedge_{n \in \Int} \supp fg^n = 0
			\right\}
	}
	\Say{H}{   
		\left\{
			f \in \Aut_\BOOL(A)
			:     \bigwedge_{n \in \Int} \supp fg^n = 0
		\right\}	
	}{?\Aut_\BOOL(A)}	
	\AssumeIn{f}{\genFS{g}}
	\Say{[1]}
	{
		\Elim \supp
		\Intro \FUNC{inverse}(\Int)
		\Elim \Int
		\THM{SupInverse}(A)
		\Elim \c
		\THM{FullSubgroupGeneratedByElement}(A,g,f)
		\NewLine
		\THM{DensitySupTHM}(A,e)
		\Elim \c
	}
	{
		\NewLine :		
		\bigwedge_{n \in \Int} \supp fg^n  =
		\bigwedge 
		\{  
			a \in A : \exists n \in \Nat \.  
			\forall b \in \langle a^\c \rangle_\I \.
			fg^n(b) = b                   
		\} = \NewLine
		\bigwedge 
		\{  
			a \in A : \exists n \in \Nat \.  
			\forall b \in \langle a^\c \rangle_\I \.
			f(b) = g^{-n}(b)                   
		\} = 
		\bigwedge 
		\{  
			a \in A : \exists n \in \Nat \.  
			\forall b \in \langle a^\c \rangle_\I \.
			f(b) = g^{n}(b)                   
		\} =  \NewLine
		\neg \bigvee
		\Big\{  
			 a^\c |a \in A : \exists n \in \Nat \.  
			\forall b \in \langle a^\c \rangle_\I \.
			f(b) = g^{n}(b)                   
		\Big\} = 
		\neg \bigvee
		\Big\{  
			 a \in A : \exists n \in \Nat \.  
			\forall b \in \langle a \rangle_\I \.
			f(b) = g^{n}(b)                   
		\Big\} = \NewLine
		=
		e^\c = 
		0
	}
	\Conclude{[2]}{\Elim H [1]}
	{
		f \subset H
	}
	\Derive{[1]}{\Intro \subset}{\genFS{g} \subset H}
	\Assume{f}{H}
	\Say{X}{
		\Big\{  
			 a \in A : \exists n \in \Nat \.  
			\forall b \in \langle a \rangle_\I \.
			f(b) = g^{n}(b)                   
		\Big\}}{?A} 
	\Say{[2]}{\Elim H(f)\Intro X}{
		e = 
		\bigvee
		\Big\{  
			 a \in A : \exists n \in \Nat \.  
			\forall b \in \langle a \rangle_\I \.
			f(b) = g^{n}(b)                   
		\Big\} = \sup X
	}
	\AssumeIn{a}{A}
	\Assume{[3]}{a \neq 0}
	\Say{\Big(x,n,[4]\Big)}{\Elim \sup [2]}{
		\sum x \in X \. \sum n \in \Int \.
		ax \neq 0
		\And
		\forall b \in \langle x \rangle \.
		f(b) = g^n(b)
	}
	\Conclude{[a.*]}{\THM{MeetIneq}(A)[4]\Elim x}
	{
		\forall b \in \langle ax \rangle_\I \.
		f(b) =  g^n(b)
	}
	\DeriveConclude{[f.*]}{\THM{FullSubgroupGeneratedByElement}(A,g,f)}
	{
		f \in \genFS{g}
	}
	\DeriveConclude{[*]}{\Intro \subset \Intro \SetEq [1]}
	{
		H = \genFS{g}
	}
	\EndProof
}
\Page{
	\Theorem{ElementsFullSubgroupFixedPoint}{
		\forall A \in \BOOL \.
		\forall f \in \Aut_\BOOL(A) \.
		\forall \phi \in \genFS{f} \. 
		\forall a \in A \.
		\NewLine \.
		f(a) = a \Imply \phi(a) = a	
	}
	\Say{G}{\Big\{ g \in \Aut_\BOOL(A) : g(a) =a \Big\}}
	{\TYPE{Subgroup}\Big(\Aut_\BOOL(A)\Big)}
	\AssumeIn{I}{\Set}
	\Assume{p}{\Inj(I,A)}
	\Assume{[1]}{\PoU(A,\im p)}	
	\Assume{g}{\Inj(I,G)}
	\AssumeIn{f}{\Aut_\BOOL(A)}
	\Assume{[2]}
	{
		\forall b \in A \. \forall i \in I \. 
		b \le a_i \Imply  f(b) = g_i(b)
	}
	\Say{[3]}{
		\Elim \PoU(A,\im p)
		\Elim \oC(A,A,f)
		\THM{OrderContinuousMult}(A,I,p,a)
		[2] \NewLine
		\Lambda i \in I \. \Elim \BOOL(A,A,g_i) \Elim G(g_i)
		\THM{OrderContinuousMult}(A,I,p,a)
		[2]
		\Elim \oC(A,A,f)
		\NewLine
		\Elim \PoU(A,\im p)
		\Elim \BOOL(A,A,f)
		\Elim \RING(A)
	}
	{
		\NewLine :		
		f(a) = 
		f\left(a \bigvee_{i \in I} p_i \right) =
		\bigvee_{i \in I} f(ap_i) = 
		\bigvee_{i \in I} g_i(ap_i) =
		a \bigvee_{i \in I} g_i(p_i) =
		a \bigvee_{i \in I} f(p_i)  =
		a f\left( \bigvee_{i \in I} p_i \right) =
		a f(e) =
		ae =
		a 
	}
	\Conclude{[I.*]}{\Elim G[3]}{f \in G}
	\Derive{[1]}{\Intro \FS}{\FS{A,G}}
	\Say{[2]}{\Elim \genFS{g}[0][1]}{\genFS{g} \subset G}
	\Conclude{[*]}{\Elim G [2]}{\forall \phi \in \genFS{G} \. \phi(a) = a}
	\EndProof
	\\
	\Theorem{ElementsFullSubgroupSupport}
	{
		\forall A \in \BOOL \.
		\forall f \in \Aut_\BOOL(A) \.
		\forall g \in \genFS{f} \.
		\Supp(f) \subset \Supp(g)
	}
	\AssumeIn{a}{\Supp(f)}
	\AssumeIn{b}{\langle a^\c \rangle_\I}
	\Say{[1]}{\Elim \Supp(f,a,b)}{f(b) = b}
	\Conclude{[b.*]}{\THM{ElementsFullSubgroupFixedPoint}[1]}{ g(b) = b}
	\DeriveConclude{[a.*]}{\Intro \Supp}{a \in \Supp(g)}
	\DeriveConclude{[*]}{\Intro \subset}{\Supp(f) \subset \Supp(g)}
	\EndProof
}
\newpage
\subsubsection{Recurrence}
\Page{
	\DeclareFunc{limsup}{\prod A : \TAlgebra . (\Nat \to A) \to A}
	\DefineNamedFunc{limsup}{a}{\limsup_{n=1} a_n}
	{
		\inf_{n=1} \sup_{k \ge n} a_k
	}
	\\
	\Theorem{LimsupIsAFixedPoint}
	{
		\forall A : \TAlgebra \.
		\forall f  : \oC(A,A) \And \End_\BOOL(A) \.
		\forall a \in A \.  \NewLine \.
		f \Big( \limsup_{n=1} f^n(a)  \Big) = 
		\limsup_{n=1} f^n(a)
	}
	\Conclude{[*]}{
		\Elim \limsup
		\Elim \oC(A,A,f)
		\Intro 2
		\THM{SupIsMonotonic}(A)\Elim \inf	
	}
	{
			\NewLine :			
			f \Big( \limsup_{n=1} f^n(a)  \Big) = 
			f \bigg(  \inf_{n=1} \sup_{k \ge n} f^k(a)  \bigg) = 
			\inf_{n=1} \sup_{k \ge n} f^{k+1}(a) =
			\inf_{n=2} \sup_{k \ge n} f^{k}(a) =
			\inf_{n=1} \sup_{k \ge n} f^{k}(a) =
			\limsup_{n=1} f^n(a)
	}
	\EndProof
	\\
	\Theorem{IteratedSupLemma}
	{
		\forall A : \TAlgebra \. 
		\forall f  : \oC(A,A) \And \End_\BOOL(A) \.
		\forall a \in A \.
		\NewLine
		a \le \sup_{n=1} f^n(a) \Imply
		\forall k \in \Nat \.
		\sup_{n = k} f^n(a) =  \sup_{n = 1} f^n(a) 
	}
	\Say{\sagittarius}{
		\Lambda m \in \Nat \.
		\Big( \forall k \in [1,\ldots,m] \.  
			\sup_{n = k} f^n(a) =  \sup_{n = 1} f^n(a)
		\Big)
	}{ \Nat \to \Type  }
	\Say{[1]}{\Intro(=,A,\sup_{n = 1} f^n(a))}
	{
		\sup_{n = 1} f^n(a) = \sup_{n = 1} f^n(a)
	}
	\Say{[2]}{\Intro \sagittarius [1] }
	{
		\sagittarius(1) 
	}
	\Say{[00]}{f[0] \Elim \oC(A,A,f)}
	{
		f(a) \le \sup_{n = 2}  f^n(a)
	}
	\AssumeIn{m}{\Nat}
	\Assume{[2]}{\sagittarius(m)}
	\Conclude{[m.*]}{
		\Lambda n \in [ m + 1, \ldots, \infty) \.
		\Intro n + 1
		\Elim \oC(A,A,f)
		\Elim \sagittarius [2]
		\Elim \oC(A,A,f)
		[00]
	}
	{
		\NewLine :		
		\sup_{n = m + 1} f^n(a) =
		\sup_{n = m}  f^{n+1}(a)    =
		f\Big( \sup_{n = m}  f^n(a) \Big) = 
		f\Big( \sup_{n = 1}  f^n(a) \Big) =
		\sup_{n = 2}  f^n(a) =
		\sup_{n = 1}  f^n(a)
 	}
 	\Derive{[2]}{\Intro \Imply \Intro \forall}
 	{
 		\forall m \in \Nat \.
 		\sagittarius(m) \Imply \sagittarius(m+1) 
 	}
 	\Conclude{[*]}{\Elim \Nat [1][2]}{
 		\forall m \in \Nat \. 
 		\sagittarius(m) 
 	}
 	\EndProof	
	\\
	\DeclareType{RecurrentOn}{\prod_{A \in \BOOL} A \to ?\End_\BOOL(A)}
	\DefineType{f}{RecurrentOn}
	{
		\Lambda	a \in A \. 
		\forall b \in \langle a \rangle \. 
		b \neq 0 
		\Imply
		\exists k \in \Nat \.   af^k(b) \neq 0
	}
	\\
	\DeclareType{DoublyRecurrentOn}{\prod_{A \in \BOOL} A \to ?\Aut_\BOOL(A)}
	\DefineType{f}{DoublyRecurrentOn}
	{
		\Lambda	a \in A \. 
		\TYPE{RecurrentOn}(A,a, f \And f^{-1} )
	}
}
\Page{
	\Theorem{RecurrenceOnImpliesBound}
	{
		\forall A : \TAlgebra \.
		\forall f \in \Aut_\BOOL(A) \.
		\forall a \in A \.
		\NewLine \.
		\TYPE{RecurrentOn}(A,a,f)
		\Imply
		\forall m \in \Nat \.
		a \le \sup_{n=m} f^{-n}(a)
	}
	\Say{b}
	{
		a \setminus \sup_{n=m} f^{-n}(a)
	}{?A}
	\AssumeIn{[1]}{b \neq 0}
	\Say{[2]}{\Elim b \THM{SetminusIneq}}{ b \le a }
	\Say{\Big( k, [3]\Big)}{\Elim \TYPE{RecurrentOn}(A,a,f)[1][2]}
	{
		\sum k \in \Nat \.   f^k(b)a \neq 0 
	}
	\Say{[4]}{\Elim b \Elim \TYPE{Inverse} \Elim \setminus}
	{
		af^k(b) = 
		af^k(a)  \setminus_{n =1} f^{n-k}(a) =
		0 
	}
	\Conclude{[1.*]}{[3][4]}{\bot}
	\Derive{[1]}{\Elim \bot}{b  = 0 }
	\Say{[2]}{\Elim \setminus \Elim b [1]}
	{
		a \le \sup_{n=1} f^{-n}(a)	
	}
	\Conclude{[*]}{[2]\THM{IteratedSupLemma}(A,a,f^{-1})}
	{
		\forall m \in \Nat \. 
		a \le \sup_{m=1} f^{-n}(a)
	}
	\EndProof
	\\
	\Theorem{RecurrenceOnImpliesBound}
	{
		\forall A : \TAlgebra \.
		\forall f \in \Aut_\BOOL(A) \.
		\forall a \in A \.
		\NewLine \.
		a \le \sup_{n=1} f^{-n}(a)
		\Imply
		\TYPE{RecurrentOn}(A,a,f)
	}
	\AssumeIn{b}{A}
	\Assume{[1]}{b \le a}
	\Assume{[2]}{b \neq 0}
	\Say{[3]}{[1][0]}
	{
		b \le  \sup_{n=1} f^{-n}(a)
	}
	\Say{[4]}{\Elim \TYPE{BooleanOrder}(A)\THM{oCMult}(A)}
	{
		b = b \sup_{n=1} f^{-n}(a) =
		\sup_{n=1} f^{-n}(a)
	}
	\Say{\Big( n,[5]\Big)}{\Elim \sup [4]}
	{
		\sum_{n=1}^\infty f^{-n}(a)b \neq 0
	}
	\Conclude{[b.*]}{f^{n}[5]}
	{
				a f^n(b) \neq 0	
	}
	\DeriveConclude{[*]}{\Intro \TYPE{RecurrentOn}}
	{
		\TYPE{RecurrentOn}(A,a,f)
	}
	\EndProof	
	\\
	\Theorem{
		DoubleRecurrenceSymmetry	
	}
	{
		\forall A \in \BOOL \.
		\forall a \in A \.
		\forall f : \TYPE{DoublyRecurrentOn}(A,a) \. \NewLine \.
		\TYPE{DoublyRecurrentOn}\Big(A,a,f^{-1}\Big)	
	}
	\NoProof
}\Page{
	\DeclareFunc{inducedAutomorphism}
	{
		\prod A : \TAlgebra \.
		\forall a \in A  \.
		\TYPE{DounblyRecurrentOn}(A,a) \to
		\Aut_\BOOL\langle a \rangle_\I
	}
	\DefineNamedFunc{inducedAutomorphism}{f}
	{f_a}{
		\Lambda b \in \langle a \rangle_\I \.
		f^n(b) \;
 		\where \;		
 		b  \le  a f^{-n}(a) \setminus \sup_{1 \le k < n} f^{-k}(a)		
	}
	\Say{p}{\Lambda n \in \Nat \. af^{-n}(a) \setminus \sup_{1 \le k < n} f^{-k}(a)}
	{
		\Nat \to A
	}
	\Say{q}{\Lambda n \in \Nat \. af^{n}(a) \setminus \sup_{1 \le k < n} f^{k}(a)}
	{
		\Nat \to A
	}
	\Say{[1]}{\Elim p \Intro \PD}{\PD\Big(\langle a \rangle_\I,\im p\Big)}
	\AssumeIn{b}{\langle a \rangle_\I}
	\Say{N}{\{  n \in \Nat \.  af^{n}(b) \neq 0  \}}{?\Nat}
	\Say{[2]}{\Elim \TYPE{DoublyRecurrentOn}(A,a,f)\Elim N }{N \neq \emptyset}
	\SayIn{n}{\min N}{\Nat}
	\Say{[3]}{\Elim n}{af^{n}(b) \neq 0}
	\Say{[4]}
	{
		\Lambda k \in [1,\ldots, n-1] \. 
		\Elim \Aut_\BOOL(A,f^{-k}) 
		\Elim n \Elim \min \Elim N
		\Elim \Aut_\BOOL(A,f^{-k})
	}	
	{
		\NewLine :		
		\forall k \in [1,\ldots,n - 1] \.
		bf^{-k}(a) = 
		f^{-k}(f^k(b)a)	= 
		f^{-k}(0) = 0
	}
	\Conclude{[b.*]}{
		\Elim p_n
		\Elim \TYPE{BooleanOrder}(A,a,b) \THM{MeetDifference}(A)
		[4]
		\Elim \Aut_\BOOL(A,f^{-k}) 
		[3] \Elim \Aut_\BOOL(A,f^{-k})
	}{  
				\NewLine : 				
				b p_n = 
				b a f^{-n}(a) \setminus \sup_{1 \le k < n} f^{-k}(a)  =
				b f^{-n}(a) \setminus \sup_{1 \le k < n} b f^{-k}(a)  =
				b f^{-n}(a)  =
				f^{-n}\Big(b f^n(a)\Big) \neq 0
	}
	\Derive{[2]}{\Intro \PoU[1]}
	{
		\PoU\Big(\langle a \rangle_\I,\im p\Big)	
	}
	\Say{[3]}{\Elim q \Intro \PoU}
	{
		\PoU\Big(\langle a \rangle_\I,\im q\Big)
	}
	\AssumeIn{n}{\Nat}
	\Conclude{[n.*]}{}
	{
		f^n(p_n) =
		f^n\left( af^{-n}(a) \setminus \sup_{1 \le k < n} f^{-k}(a)  \right) =  
		f^n(a)a \setminus_{1 \le k < n} f^{k}(a) =
		q_n
	}
	\Derive{[4]}{\Intro \forall}
	{
		\forall n \in \Nat \.
		f^n(p_n) = q_n 
	}
	\Say{f_a}
	{
		\Lambda  \bigvee_{n=1}^\infty b_ip_i \in \langle a \rangle_\I \.
		\bigvee_{n=1}^\infty f^n(b_i) q_i
	}
	{
		\langle a \rangle_\I \to \langle a \rangle_\I
	}
	\Say{[5]}{
		\Elim \PoU\Big(\langle a \rangle_\I, \im p\Big) 
		\Elim \PoU\Big(\langle a \rangle_\I, \im q\Big) 
		}
		{
			f_a \in \Aut_\BOOL \langle a \rangle_\I
		}
	\EndProof
	\\
	\Theorem{InducedHomomorphismInverse}
	{
		\forall A : \TAlgebra \.
		\forall a \in A \. 
		\forall f : \TYPE{DoublyRecurrentOn}(A,a) \.
		(f_a)^{-1} = (f^{-1})_a
	}
	\NoProof
}\Page{
	\Theorem{InducedHomomorphismPoU}
	{
		\forall A : \TAlgebra \.
		\forall a \in A \. 
		\forall f : \TYPE{DoublyRecurrentOn}(A,a) \.
		\forall n \in \Int_+ \. \NewLine \.
		\exists p : \Nat \to \langle a \rangle_\I :
		\PoU\Big(\langle a \rangle_\I,\im p\Big) \.
		\forall k \in \Int_+ \.
		\forall c \in \langle p_k \rangle_\I \.
		f^n_a(c) = f^{n+k}(c)
	}
	\Say{\Big(p,q,[1] \Big)}{\Elim f_a}
	{
		\sum p,q : \Nat \to \langle a \rangle_\I \.
		\PoU\Big(\langle a \rangle_\I,\im p\Big)
		\And
		\PoU\Big(\langle a \rangle_\I,\im q\Big)
		\And \NewLine \And
		\Big(\forall n \in \Nat \. f^n(p_n) = q_n \Big)
		\And 
		\forall \bigvee^\infty_{n=1} b_n p_n  \in \langle a \rangle_\I\.
		f_a\left( \bigvee^\infty_{n=1} b_n p_n \right) = \bigvee^\infty_{n=1} f^n(b_n) q_n
	}
	\Say{w_0}{\Lambda k \in \Int_0 \. \If k == 0 \Then a \Else 0}{
		\Int_+ \to \langle a \rangle_\I }
	\Say{[2]}{\Elim w_0}{\PoU\Big(\langle a \rangle_\I,\im w_0\Big)}
	\Say{[3]}{
		\Lambda k \in \Int_+ \. 
		\Lambda c \in \langle w_{0,k} \rangle_\I
		\THM{ZeroExponentTHM}\Big( \Aut_\BOOL(A), f \Big)
		\Elim \id
		\Elim w_0(x)
	}
	{
		\NewLine :		
		\forall k \in \Int_0 \.
		\forall  c \in \langle w_{0,k} \rangle_\I \.
		f^0_a(c) = \id(c) =  c = f^k(c)
	}
	\Assume{n}{\Int_+}
	\Assume{w_n}{\Int_+ \to \langle a \rangle_\I }
	\Assume{[4]}{\PoU\Big(\langle a \rangle_\I,\im w_n\Big)}
	\Assume{[5]}
	{
		\forall k \in \Int_+		
		\forall c \in \langle w_{n,k} \rangle_\I \. 
		f^n_a(c) = f^{n+k}(c)}
	\Say{b}{ \Lambda k,t \in \Int_+ \. f_a^{-1}(w_{n,k})p_t }
	{
		\Int_+^2 \to  \langle a \rangle_\I
	}
	\Say{w_{n+1}}{\Lambda k \in \Int_+ \. \bigvee_{k + 1 = t + s} b_{t,s} }
	{
		\Int_+ \to \langle a \rangle_\I
	}
	\Say{[n.*.1]}{
		\Elim \PoU\Big(\langle a \rangle_\I,\im w_n\Big) 
		\Elim \PoU\Big(\langle a \rangle_\I,\im p\Big) 
		\NewLine :
		\Elim w_{n+1}
		\Intro \PoU
		\Intro w_{n+1}
	}
	{	
		\PoU\Big(\langle a \rangle_\I,\im w_{n+1}\Big) 
	}
	\AssumeIn{k}{\Int_+}
	\AssumeIn{c}{\langle w_{n+1,k} \rangle_\I}
	\Conclude{[n.*.2]}{
			\Elim \TYPE{BooleanOrder}(w_{n+1,k},c)
			\Elim w_{n+1,k}
			\Elim b
			\Elim \oC\Big( \langle a \rangle_\I, f_a  \Big)
			\Elim \TYPE{Inverse}\Big(\Aut_\BOOL\langle a \rangle_\I, f_a \Big)
			\NewLine
			[1][3]
			\Elim \oC\Big( A, f  \Big)
			[1]^3
			\Intro b
			\Intro w_{n+1,k}
			\Elim \TYPE{BooleanOrder}(w_{n+1,k},c)
	}
	{
		\NewLine :		
		f^{n+1}_a(c) = 
		f^{n+1}_a(cw_{n+1,k}) =
		f^{n+1}_a\left( c \bigvee_{k  + 1 = t + s}  b_{t,s} \right) =
		 f_a f^n_a   \left( c \bigvee_{k + 1 = t + s}  f^{-1}_a(w_{n,s}) p_t \right) 
		  = \NewLine =
		\bigvee_{k + 1 = t + s }   f_a f^n_a \Big( c f^{-1}_a(w_{n,s}) p_t   \Big) = 
		\bigvee_{k + 1 = t + s} f^n_a (  w_{n,s} f_a(p_t c)) =
		\bigvee_{k + 1 = t + s}   f^{n+s}\Big(  w_{n,s}f^{t}( c p_t) \Big) =  \NewLine =
		f^{n + k + 1}\left( 
				c  \bigvee_{k + 1 = t + s}  f^{-t}(w_{n,s}) p_t
		\right) =
		f^{n + k + 1}\left( 
				c  \bigvee_{k + 1 = t + s}  f^{-t}(w_{n,s} q_t )
		\right) =
		f^{n + k + 1}\left( 
				c  \bigvee_{k + 1 = t + s}  f^{-1}_a(w_{n,s} q_t )
		\right) = \NewLine =
		f^{n + k + 1}\left( 
				c  \bigvee_{k + 1 = t + s}  f^{-1}_a(w_{n,s} ) p_t )
		\right) = 
		f^{n + k + 1}\left( 
				c  \bigvee_{k + 1 = t + s}  b_{t,s} 
		\right) =  
		f^{n + k + 1}(c w_{n+1,k}) =
		f^{n + k + 1}(c) 
 	}
 	\DeriveConclude{[*]}{\Elim \Nat}
 	{
 		\forall n \in \Int_+ \.
		\exists p : \Nat \to \langle a \rangle_\I :
		\PoU\Big(\langle a \rangle_\I,\im p\Big) \.
		\forall k \in \Int_+ \.
		\forall c \in \langle p_k \rangle_\I \. \NewLine \.
		f^n_a(c) = f^{n+k}(c)
 	}
 	\EndProof
}\Page{
	\Theorem{InducedHomomorphismRecurrence}
	{
		\forall A : \TAlgebra \.
		\forall a \in A \. 
		\forall f : \TYPE{DoublyRecurrentOn}(A,a) \.
		\forall n \in \Nat \. \NewLine \.
		\forall b \in \Big\langle a f^{-n}(a) \Big\rangle_\I \.
		\exists b' \in \langle b \rangle_\I :
		\exists k \in [0,\ldots,n] : 
		\forall c \in \langle b' \rangle_\I \.
		f^n(c) = f_a^k(c)
	}
	\Say{\sagittarius}
	{
		\Lambda n \in \Nat \.
		\forall m \in [1,\ldots,n] \. 
		\forall b \in \Big\langle a f^{-n}(a) \Big\rangle_\I \.
		\exists b' \in \langle b \rangle_\I :
		\exists k \in [0,\ldots,n] : 
		\forall c \in \langle b' \rangle_\I \.
		f^n(c) = f_a^k(c)	
	}{ \NewLine : \Nat \to \Type}
	\Say{\Big(p,q,[1] \Big)}{\Elim f_a}
	{
		\sum p,q : \Nat \to \langle a \rangle_\I \.
		\PoU\Big(\langle a \rangle_\I,\im p\Big)
		\And
		\PoU\Big(\langle a \rangle_\I,\im q\Big)
		\And \NewLine \And
		\Big(\forall n \in \Nat \. f^n(p_n) = q_n \Big)
		\And 
		\forall \bigvee^\infty_{n=1} b_n p_n  \in \langle a \rangle_\I\.
		f_a\left( \bigvee^\infty_{n=1} b_n p_n \right) = \bigvee^\infty_{n=1} f^n(b_n) q_n
	}	
	\Say{[2]}
	{
		\Lambda c \in \Big\langle a f^{-1}(a) \Big\rangle_\I \. [1](1,c)
	}
	{
		\forall c \in \langle b \rangle_\I \.
		f(c) = f_a(c)
	}
	\Say{[3]}{\Intro \sagittarius [2]}{\sagittarius(1)}
	\Assume{n}{\Nat}
	\Assume{[4]}{\sagittarius(n)}
	\AssumeIn{b}{\genIdeal{af^{-n-1}(a)}}
	\Say{\Big( v, [5] \Big)}{\LOGIC{ByConstruction}\Big( p, [1]\Big)}
	{
		\sum v : \{1,\ldots,n + 1\} \to \genIdeal{af^{-n}(a)} \. 
		b = \bigvee^n_{i=1} v_i p_i
	}
	\Assume{6}{b \neq 0}
	\Say{[7]}{[5][6]}{\Big\{ k \in \{1,\ldots,n + 1\} : v_k \neq 0  \Big\} \neq \emptyset}
	\SayIn{k}{\min \Big\{ k \in \{1,\ldots,n + 1\} : v_k \neq 0  \Big\} }{\{1,\ldots,n +1\} } 
	\Say{b'}{ v_kp_k}{\genIdeal{bp_k}}
	\Say{[8]}{\Elim b' \Elim f_a}{\forall c \in \genIdeal{b'} \. f_a(c) = f^{k}(c)}
	\Say{[9]}{
		\THM{ExponentMult}\Big( \Aut_\BOOL(A), f, k, n - k \Big)
		\Elim b' \THM{MonotonicBooleanMorphism}(A,A,f)
		\Elim b	
	}{
		\NewLine :		
		f^k f^{n-k+1}(b') = f^{n+1}(b') 
		\le f^n(b) 
		\le a 
	}
	\Say{\Big(c, l, [10]\Big)}{\Elim \sagittarius [4][9](n + 1 - k)}
	{
		\NewLine :		
		\sum c \in \genIdeal{f^{k}(b')} \.
		\sum l \in \{1, \ldots, n + k - k \} \.
		\forall d \in \genIdeal{c} \.
		f^{n-1-k}(d) = f_a^k(d)		
	}
	\SayIn{b''}{f^{k}(c)}{\genIdeal{b}}
	\Conclude{[n.*]}{
		\Lambda d \in \genIdeal{b''}
		\THM{ExponentMult}\Big( \Aut_\BOOL(A), f, k, n - k \Big)
		[10][8]
	}
	{
		 f^{n+1}(d) = 
		 f^k f^{n+1 - k}(d)
		 f_a f_a^{l} = f_a^{l+1}(d)
	}
	\DeriveConclude{[*]}{\Elim \Nat [3] \Elim \sagittarius}
	{
		\forall n \in \Nat \. 
		\forall b \in \Big\langle a f^{-n}(a) \Big\rangle_\I \.
		\exists b' \in \langle b \rangle_\I :
		\exists k \in [0,\ldots,n] : 
		\forall c \in \langle b' \rangle_\I \.
		f^n(c) = f_a^k(c)
	}
	\EndProof
}
\Page{
		\Theorem{LateRecurrenceDivision}
	{
		\forall A : \TAlgebra \.
		\forall a \in A \. 
		\forall f : \TYPE{DoublyRecurrentOn}(A,a) \.
		\forall m,n \in \Int_+ \.
		\NewLine \.
		\Big(
			\forall k \in \{1,\ldots,m-1\} \.
			af^{k}(a) = 0
		\Big)
		\Imply
		\exists d : \left\{ 1, 
			\ldots,  
			\left\lfloor \frac{n}{m} \right\rfloor 
		\right\} \to  A : \NewLine :
		\PD(A,\im d) 
		\And
		\sup \im d = a f^{-n}(a)
		\And \NewLine \And 
		\forall k \in \left\{ 1, \ldots,  \left\lfloor \frac{n}{m} \right\rfloor \right\} \.
		\forall c \in \genIdeal{d_k} \.
		f^n(c) = f^k_a(c)
	}
	\Say{\sagittarius}
	{
		\Lambda n \in \Nat \.
		\forall t \in [1,\ldots,t] \.
		\exists d : \left\{ 1, 
			\ldots,  
			\left\lfloor \frac{t}{m} \right\rfloor 
		\right\} \to  A : 
		\PD(A,\im d) 
		\And \NewLine \And
		\sup \im d = a f^{-t}(a)
		\And 
		\forall k \in \left\{ 1, \ldots,  \left\lfloor \frac{t}{m} \right\rfloor \right\} \.
		\forall c \in \genIdeal{d_k} \.
		f^t(c) = f^k_a(c)
	}{   \Nat \to \Type}
	\Say{[1]}{[0]\Elim \sagittarius}
	{
		\forall t \in [1,\ldots,m-1] \. \sagittarius(t) 
	}
	\Assume{[2]}{m = n}
	\SayIn{d}{af^{-n}a}{A}
	\AssumeIn{c}{\genIdeal{d}}
	\Say{[3]}{\Lambda k \in \{1,\ldots,n-1\} \. f\big([0]\big)[2]}
	{
		\forall k \in \{1,\ldots,n-1\} \.   af^{-k}(a) = 0
	}
	\Conclude{[2.*]}{\Elim f_a [3] \Elim d}{f^n(c) = f_a(c)}	
	\Derive{[2]}{\Intro \sagittarius}{\sagittarius(m)}
	\Say{\Big(p,q,[3] \Big)}{\Elim f_a}
	{
		\sum p,q : \Nat \to \langle a \rangle_\I \.
		\PoU\Big(\langle a \rangle_\I,\im p\Big)
		\And
		\PoU\Big(\langle a \rangle_\I,\im q\Big)
		\And \NewLine \And
		\Big(\forall n \in \Nat \. f^n(p_n) = q_n \Big)
		\And 
		\forall \bigvee^\infty_{n=1} b_n p_n  \in \langle a \rangle_\I\.
		f_a\left( \bigvee^\infty_{n=1} b_n p_n \right) = \bigvee^\infty_{n=1} f^n(b_n) q_n
	}
	\Say{[4]}{\Intro \sagittarius [2]}{\sagittarius(1)}
	\AssumeIn{n,s}{\Nat}
	\Assume{[5]}{\sagittarius(n)}
	\Assume{[6]}{n + 1 = ms}
	\Say{t}{\Lambda i \in \{m,\ldots,n\} \. \left\lfloor \frac{i}{m} \right\rfloor}
	{ \{m,\ldots,n\} \to \{1,\ldots,s-1\}  }
	\Say{\Big(d,[7]\Big)}{\Elim \sagittarius[4]}
	{
		\sum d : \{m,\ldots,n\} \times \{1,\ldots, s-1\} \to A \.
		\forall i \in \{m,\ldots,n\} \. \NewLine \.
		\PoU\Big( \genIdeal{af^{-1}(a)}, \im d_i \Big)
		\And 
		\forall j \in \{1,\ldots,t_i\} \.
		\forall c \in \genIdeal{d_{i,j}} \.
		f^i(c) = f^j_a(c)
	}
	\Say{[8]}{ 
		\Intro p_n \Elim \BOOL\Big(A, f^{-i}(a) \Big)
		\Lambda i \in \{m,\ldots, n\}
		\Elim \PoU \Big( \genIdeal{af^{-i}(a)}, \im d_i \Big) \NewLine
		\Elim \TYPE{DistiributiveLattice}(A)
	} 
	{
		\NewLine :		
		a f^{-n-1}(a) = 
		p_{n+1} \vee \bigvee^{n}_{i=m} p_i f^{-n-1}(a) =
		p_{n+1} \vee \bigvee^{n}_{i=m} p_i f^{-i}\Big( a f^{i-n-1}(a) \Big) =
		p_{n+1}  \vee  \bigvee^{n}_{i=m} p_i f^{-i}
		\left( \bigvee^{t_i}_{j=1} d_{n+1-i,j} \right) =
		\NewLine =
		p_{n+1} \vee \bigvee^{n}_{i=m} \bigvee^{t_i}_{j=1} p_i f^{-i}(d_{n +1-i,j})
	}
	\Say{d_{n+1}}{
		\Lambda i \in \{1,\ldots,s\} \.
		\If i == 1 \Then p_{n+1} \Else
		\bigvee_{i=m}^{n + 1 - m} p_i f^{-i}(d_{n+1-i,j-1}) 
	}
	{
		\{1,\ldots,s\} \to A
	}
	\Say{[9]}{\Elim d_{n+1} [8][7]}{\PoU\Big( \genIdeal{af^{-n-1}(a)}, \im d_{n+1} \Big)}
}
\Page{
	\AssumeIn{k}{\{1,\ldots,s\}}
	\AssumeIn{i}{\{m,\ldots,n\}}
	\Assume{b}{\genIdeal{p_i f^{-i}(d_{n+1-i,k-1})}}
	\Say{[10]}{
			\Elim f_a 
			\THM{PrincipleIdealStructrue}\Big(A,p_i f^{-i}(d_{n-i,k-1}) ,b\Big)
			\THM{BooleanMorphismIsMonotonic}(A,A,f^i)
			\NewLine
			\THM{BooleanMeet}(A)	
	}
	{
		f_a(b) = f^i(b) \le  f^i(p_i) d_{n+1-i,k-l} \le d_{n-i,k-i}
	}
	\Conclude{[k.*]}{
		\THM{ExponenentMult}\Big(\Aut_\BOOL(A),f,i,n-i\Big)
		\Intro f_a
		[10]
		\THM{ExponenentMult}\Big(\Aut_\BOOL(A)\Big)
	}
	{
		\NewLine :		
		f^{n+1}(b) = 
		f^{i}f^{n+1-i}(b) =
		f_a f^{n+1-i}(b) =
		f^k_a(b)
	}
	\DeriveConclude{[n.*]}{\Intro \forall}
	{
		\forall k \in \{1,\ldots,s\} \.
		f^{n+1}(d_{n+1,k}) = f_a^k(d_{n+1,k})
	}
	\DeriveConclude{[*]}{\Elim \Nat}
	{
		\forall m,n \in \Int_+ \.
		\NewLine \.
		\Big(
			\forall k \in \{1,\ldots,m-1\} \.
			af^{k}(a) = 0
		\Big)
		\Imply
		\exists d : \left\{ 1, 
			\ldots,  
			\left\lfloor \frac{n}{m} \right\rfloor 
		\right\} \to  A : \NewLine :
		\PD(A,\im d) 
		\And
		\sup \im d = a f^{-n}(a)
		\And \NewLine \And 
		\forall k \in \left\{ 1, \ldots,  \left\lfloor \frac{n}{m} \right\rfloor \right\} \.
		\forall c \in \genIdeal{d_k} \.
		f^n(c) = f^k_a(c)
	}
	\EndProof
	\\
	\Theorem{InnerRecurrenceCriterion}
	{
		\forall A : \TAlgebra \.
		\forall a \in A \. 
		\forall f : \TYPE{DoublyRecurrentOn}(A,a) \.
		\forall b \in \genIdeal{a} \. \NewLine \.
		\TYPE{DoublyReccurent}(A,b,f)
		\iff
		\TYPE{DoublyReccurent}\Big(\genIdeal{a},b,f_a\Big)
	}
	\Assume{[1]}{\TYPE{DoublyReccurent}(A,b,f)}
	\AssumeIn{c}{\genIdeal{b} \setminus \{0\}}
	\Say{\Big(p,q,[2] \Big)}{\Elim f_b}
	{
		\sum p,q : \Nat \to \langle b \rangle_\I \.
		\PoU\Big(\langle b \rangle_\I,\im p\Big)
		\And
		\PoU\Big(\langle b \rangle_\I,\im q\Big)
		\And \NewLine \And
		\Big(\forall n \in \Nat \. f^n(p_n) = q_n \Big)
		\And 
		\forall \bigvee^\infty_{n=1} d_n p_n  \in \langle a \rangle_\I\.
		f_a\left( \bigvee^\infty_{n=1} d_n p_n \right) = \bigvee^\infty_{n=1} f^n(d_n) q_n
	}
	\SayIn{n}{\min \{ n \in \Nat : f^{-n}(c)b \neq 0  \}}{\Nat}
	\Say{\Big( [3]\Big)}{\Elim q  \Elim n }{cq_n \neq 0}
	\Say{\Big(d,[4]\Big)}{\THM{LateRecurrenceDivision}(A,b,f,n,1)\Elim n}
	{
		\sum d : \{1,\ldots, n \} \to \genIdeal{af^{-n}(a)}  \.  \NewLine \.
		\sup \im d = af^{-n}(a)
		\And
		\forall k \in \{1,\ldots,n\} \. 
		\forall x \in \genIdeal{d_k} \. f^n(x) = f^k_a(c)
	}
	\Say{\Big(k,[5]\Big)}{[4][3]}{ \sum k \in \{1,\ldots,n\} \. d_k c  f^{-n}(c) \neq 0}
	\Say{x}{d_k c}{ \genIdeal{c}  }	
	\Say{[6]}{\Elim \Aut_\BOOL(A,f) \Elim x \Elim c \Elim^2 \TYPE{BooleanOrder}(A)[5]}
	{        
				 f^{-n}( b f^n(x) ) = f^{-n}( b ) x =  
				 f^{-n}(b) bx = x \neq 0	
	}
	\Say{[7]}{\Elim \Aut_\BOOL(A,f)[1][4](x)}{  0  \neq bf^n(x) =  bf_a^k(x) }	
	\Say{[1.*.1]}{\Elim \TYPE{BooleanOrder}(A)(x,c)[7]}
	{
		bf_a^{k}(c) \neq 0
	}
	\SayIn{m}{\min \{ m \in \Nat : f^{m}(c)b \neq 0  \}}{\Nat}
	\Say{\Big( [8]\Big)}{\Elim p \Elim m }{cp_n \neq 0}
	\Say{\Big(d,[9]\Big)}{\THM{LateRecurrenceDivision}(A,b,f^{-1},m,1)\Elim n}
	{
		\sum d' : \{1,\ldots, m \} \to \genIdeal{af^{m}(a)}  \. \NewLine \.
		\sup \im d' = af^{m}(a)
		\And
		\forall k \in \{1,\ldots,n\} \. 
		\forall x \in \genIdeal{d_k'} \. f^n(x) = f^k_a(c)
	}
	\Say{\Big(l,[10]\Big)}{[8][9]}{ \sum k \in \{1,\ldots,n\} \. d_l' c f^{m}(b)   \neq 0}
	\Say{y}{d_l' c}{ \genIdeal{c}  }	
	\Say{[10]}{\Elim \Aut_\BOOL(A,f) \Elim y \Elim c \Elim^2 \TYPE{BooleanOrder}(A)[10]}
	{        
				 f^{m}( b f^{-m}(y) ) = f^{m}( b ) y = 
				 f^{m}(b) b y \neq 0	
	}
	\Say{[11]}{\Elim \Aut_\BOOL(A,f)[10][9](y)}{  0  \neq bf^{-m}(y) =  bf_a^{-l}(y) }	
	\Conclude{[1.*.2]}{\Elim \TYPE{BooleanOrder}(A)(y,c)[7]}
	{
		bf_a^{-l}(c) \neq 0
	}
}\Page{
	\Derive{[1]}{\Intro \Imply}{ 
		\TYPE{DoublyReccurent}(A,b,f)
		\Imply
		\TYPE{DoublyReccurent}\Big(\genIdeal{a},b,f_a\Big)
	}
	\Assume{[2]}{\TYPE{DoublyReccurent}\Big(\genIdeal{a},b,f_a\Big)}
	\AssumeIn{c}{\genIdeal{b} \setminus \{0\}}
	\Say{\Big( n, [3] \Big)}{\Elim \TYPE{Reccurent}\Big(\genIdeal{a},b, f_a,c\Big)}
	{
		\sum n \in \Nat \. f^n_a(c)b \neq 0
	}
	\Say{\Big(p,[4]\Big)}
	{
		\THM{InducedHomomorphismPoU}(A,a,f,n)
	}
	{
		\sum p : \Nat \to \genIdeal{a} \.
		\PoU\Big(\genIdeal{a},\im p \Big) \.
		\forall k \in \Nat \. \forall d \in \genIdeal{p_k} \.
		 f^n_a(d) = f^{n + k}(d)
	}
	\Say{[5]}{f^{-n}_a[3]}{c f^{-n}_a(b) \neq 0}
	\Say{\Big(k,[6]\Big)}{\Elim \PoU\Big(\genIdeal{a},\im p, c f^{-n}_a(b), [5] \Big)}
	{
		\sum k \in \Nat \.  p_k c f^{-n}_a(b) \neq 0
	}
	\SayIn{d}{  p_k c  }{ \genIdeal{c} }
	\Say{ [2.*.1] }{[4][6]}
	{
		f^{n+k}(d)b = 
		f^n_a(d)(b) \neq 0
	}
	\Say{\Big( m, [7] \Big)}{\Elim \TYPE{Reccurent}\Big(\genIdeal{a},b, f_a^{-1},c\Big)}
	{
		\sum m \in \Nat \. f^{-m}_a(c)b \neq 0
	}
	\Say{\Big(q,[8]\Big)}
	{
		\THM{InducedHomomorphismPoU}(A,a,f^{-1},m)
	}
	{
		\sum q : \Nat \to \genIdeal{a} \.
		\PoU\Big(\genIdeal{a},\im q \Big) \.
		\forall l \in \Nat \. \forall d' \in \genIdeal{q_k} \.
		 f^{-m}_a(d') = f^{-m - l}(d')
	}
	\Say{[9]}{f^{m}_a[3]}{c f^{m}_a(b) \neq 0}
	\Say{\Big(l,[10]\Big)}{\Elim \PoU\Big(\genIdeal{a},\im q, c f^{m}_a(b), [9] \Big)}
	{
		\sum l \in \Nat \.  q_l c f^{m}_a(b) \neq 0
	}
	\SayIn{d'}{  q_l c  }{ \genIdeal{c} }
	\Conclude{ [2.*.2] }{[4][6]}
	{
		f^{-m-l}(d')b = 
		f^{-m}_a(d')(b) \neq 0
	}
	\DeriveConclude{[*]}{\Intro \iff [1]}
	{
		\TYPE{DoublyReccurent}(A,b,f)
		\iff
		\TYPE{DoublyReccurent}\Big(\genIdeal{a},b,f_a\Big)
	}
	\EndProof
	\\
	\Theorem{InnerRecurrenceExpression}
	{
		\forall A : \TAlgebra \.
		\forall a \in A \. 
		\forall f : \TYPE{DoublyRecurrentOn}(A,a) \.
		\forall b \in \genIdeal{a} \. \NewLine \.
		\TYPE{DoublyReccurent}(A,b,f)
		\Imply
		f_b = (f_a)_b
	}
	\NoProof
	\\
	\Theorem{FixedPointRecurrence}
	{
		\forall A : \TAlgebra \.
		\forall a \in A \. 
		\forall f : \TYPE{DoublyRecurrentOn}(A,a) \.
		\forall b \in \Fix(f) \. \NewLine \.
		\TYPE{DoublyReccurent}(A,ab,f)
	}
	\AssumeIn{c}{\genIdeal{ab} \setminus \{0\} }
	\Say{[2]}{\THM{MeetIneq}(A)\Elim(a,b,c)}{c \le a}
	\Say{\Big( n, [3] \Big)}{\Elim \TYPE{RecurrentOn}\Big(A,a,f,c\Big)}
	{
		\sum n \in \Nat \. f^n(c)a \neq 0
	}
	\Say{[c.*.1]}{\Elim \Fix(f,b)\Elim \BOOL(A,A,f^n) \Elim \TYPE{BooleanOrder}(A)[3]}
	{
	   f^n(c)ab = 
	   f^n(c)af^n(b) = 
	   f^n(cb)a = 
	   f^n(b)a \neq 0
	}
	\Say{\Big( m, [4] \Big)}{\Elim \TYPE{RecurrentOn}\Big(A,a, f^{-1},c\Big)}
	{
		\sum m \in \Nat \. f^{-m}(c)a \neq 0
	}
	\Conclude{[c.*.2]}{\Elim \Fix(f,b)\Elim \BOOL(A,A,f^n) \Elim \TYPE{BooleanOrder}(A)[4]}
	{
	   \NewLine :	   
	   f^{-m}(c)ab = 
	   f^{-m}(c)af^{-m}(b) = 
	   f^{-m}(cb)a = 
	   f^{-m}(b)a \neq 0
	}
	\DeriveConclude{[*]}{\Intro \TYPE{DoublyRecurrentOn}}
	{
		\TYPE{DoublyReccurent}(A,f,ab)
	}
	\EndProof
}

\Page{
	\Theorem{FixedPointInducedMorphism}
	{
		\forall A : \TAlgebra \.
		\forall a \in A \. 
		\forall f : \TYPE{DoublyRecurrentOn}(A,a) \.
		\forall b \in \Fix(f) \. \NewLine \.
		f_{ab} = f_{a|\genIdeal{ab}}
	}
	\Say{[1]}{\Lambda n \in \Nat \. \Elim \Fix(f^{-n},b) \Elim \BOOL(A,a)}
	{
		\forall  n \in \Nat  \. f^{-n}(a) a^2 b = f^{-n}(ab) ab
	}
	\Conclude{[*]}
	{
		\Elim f_{ab} [1]
	}
	{
		f_{ab} = f_{a|\genIdeal{ab}}
	}
	\EndProof
	\\
	\Theorem{InucedMorphismFixedPoint}
	{
		\forall A : \TAlgebra \.
		\forall a \in A \. 
		\forall f : \TYPE{DoublyRecurrentOn}(A,a) \.
		\forall b \in \Fix(f) \. \NewLine \.
		ab \in \Fix(f_a)
	}
	\NoProof
	\\
	\Theorem{AperiodicInducedMorphism}
	{
		\forall A : \TAlgebra \.
		\forall a \in A \. 
		\forall f : \TYPE{DoublyRecurrentOn}(A,a) \.
		\forall b \in \Fix(f) \. \NewLine \.
		\TYPE{Aperiodic}(A,f) \Imply \TYPE{Aperiodic}\Big(\genIdeal{a},f_a\Big)
	}
	\Say{[1]}{\Elim \TYPE{Aperiodic}(A,f) \Elim \supp }
	{
		\forall b \in A \setminus{0} \. \forall n \in \Nat \. \exists c \in \genIdeal{b} \.
		f^n(c) \neq c
	}
	\Assume{b}{\genIdeal{a} \setminus \{0\}}
	\AssumeIn{n}{\Nat}
	\Say{\Big(p,[2]\Big)}
	{
		\THM{InducedHomomorphismPoU}(A,a,f,n)
	}
	{
		\sum p : \Nat \to \genIdeal{a} \.
		\PoU\Big(\genIdeal{a},\im p \Big) \. \NewLine \.
		\forall k \in \Nat \. \forall d \in \genIdeal{p_k} \.
		 f^n_a(d) = f^{n + k}(d)}	
	\Say{\Big(k,[3]\Big)}{\Elim \PoU\Big(\langle a \rangle_\I,\im p,b\Big) }
	{
		\sum k \in \Nat \.  bp_m \neq 0
	}
	\Say{\Big(d,[4] \Big)}
	{
		[1](bp_m, n + k )
	}
	{
		\sum d \in \genIdeal{bp_m} \. 
		f^{n+k}(d) \neq dc
	}
	\Conclude{[b.*]}{[4][2]}{f_a^n(d) \neq d}
	\DeriveConclude{[*]}{\Intro \TYPE{Apperiodic}}
	{\TYPE{Aperiodic}\Big(\genIdeal{a},f_a\Big)}
	\EndProof
	\\
	\Theorem{TrinitaryLemma}
	{
		\forall A : \TAlgebra \.
		\forall a \in A \. 
		\forall f : \TYPE{DoublyRecurrentOn}(A,a) \.
		\forall b \in \genIdeal{a} \.
		\NewLine 		
		f(a)a = 0 \And b f_a(b) = 0 
		\Imply  \PD\Big( A, \{ b, f(b), f^2(b) \} \Big)  
	}
	\Say{[1]}{f[0.1]}{f^2(b)f(b) = 0}
	\Assume{c}{\genIdeal{ f^2(b) b }}
	\Say{[2]}{[0.1]\Elim f_a \Elim c}{ f^{-2}(c) = f_a^{-1}(c)}
	\Say{[3]}{\THM{BooleanMorphismIsMonotonic}(A,A,f^{-2})\Elim c [2]}
	{
						f^{-2}(c) \le f^{-2}(b) b = f_a^{-1}(b) b
	}
	\Say{[4]}{\THM{BooleanMorphismIsMonotonic}(\genIdeal{a},\genIdeal{a},f_a)[3][0.2]}
	{
						f^{-2}f_a(c) \le  b f_a(b) = 0
	}
	\Conclude{[5]}{\Elim \Aut_\BOOL(A,f^{-2}f_a)[4]}{c = 0 }
	\Derive{[2]}{\Elim \genIdeal{ f^2(b) b } }{  f^2(b) b = 0}
	\Conclude{[*]}{\Intro \PD [1][2]}{\PD\Big( A, \{ b, f(b), f^2(b) \} \Big)}
	\EndProof
}
\Page{
	\DeclareFunc{extendedInducedIsomorphism}
	{
		\sum A : \TAlgebra \.
		\sum a \in A \.
		\TYPE{DoublyRecurrentOn}(A,a)\.
		\to
		\Aut_\BOOL(A)
	}
	\DefineNamedFunc{extendedInducedIsomorphism}{f}{\tilde{f_a}}
	{
		\Lambda ba + ca^\c \in A \.  f_a(ba)  + ca^\c
	}
	\\
	\Theorem{ExtendedInducedIsomorphismInFullSubgroup}
	{
		\NewLine :: 		
		\forall A : \TAlgebra \.
		\forall a \in A \. 
		\forall f : \TYPE{DoublyRecurrentOn}(A,a) \. 
		\tilde{f_a} \in \genFS{f}
	}
	\NoProof
	\\
	\DeclareType{Recurrent}{\prod_{A : \BOOL} ?\Aut_\BOOL(A)}
	\DefineType{f}{Recurrent}{\forall a \in A \. \TYPE{RecurrentOn}(A,a,f)}
	\\
	\DeclareType{DoublyRecurrent}{\prod_{A : \BOOL} ?\Aut_\BOOL(A)}
	\DefineType{f}{DoublyRecurrent}{\forall a \in A \. \TYPE{DoublyRecurrentOn}(A,a,f)}
	\\
	\Theorem{RecurrentCondition}
	{
		\forall A \in \BOOL \.
		\forall f \in \Aut_\BOOL(A) \. \NewLine \.
		\TYPE{Recurrent}(A,f) \iff
		\forall 
		a \in A  \. a = \sup_n af^{n}(a)
	}
	\Assume{[1]}{\TYPE{Recurrent}(A,f)}
	\AssumeIn{a}{A}
	\AssumeIn{b}{\genIdeal{a}}
	\Say{[2]}{\Elim \TYPE{Recurrent}(A,f,b)}{\TYPE{RecurrentOn}(A,b,f)}
	\Assume{[3]}{\forall n \in \Nat \. f^{n}(a) b  = 0}
	\Assume{[4]}{b \neq 0}	
	\Say{\Big( n, [5] \Big)}{\Elim \TYPE{RecurrentOn}(A,a,f)[2][4]}
	{
		\sum n \in \Nat \. bf^{n}(b) \neq  0 	
	}
	\Say{[6]}{\THM{ZeroIsMinimal}(A)[5]\THM{MeetIneq}(A)}{  0 < bf^{n}(b) \le  af^{n}(b)   }
	\Say{[7]}{\THM{TrichtomyPrinciple}[6]}{af^{n}(b) \neq 0}
	\Conclude{[1.*]}{[7][3](n)}{\bot}
	\Derive{[1]}{\Intro \Imply}
	{
		\TYPE{Recurrent}(A,f) \Imply
		\forall 
		a \in A  \. a = \sup_n af^{n}(a)
	}
	\Assume{[2]}
	{
		\forall 
		a \in A  \. a = \sup_n af^{n}(a)
	}
	\NoProof
	\\
	\DeclareType{RelativeAtom}{\prod A : \BOOL \. \prod B : \TYPE{Subalgebra}(A) \. ?A}
	\DefineNamedType{a}{RelativeAtom}{a \in \Atom_A(B)}
	{
		\forall c \in \genIdeal{a} \. \exists b  \in B : c = ab
	}
	\\
	\\
	\DeclareType{RelativelyAtomless}
	{\prod A : \BOOL \.?\TYPE{Subalgebra}(A) \. ?A}
	\DefineType{B}{RelativelyAtomless}
	{
		\Atom_A(B) = \emptyset
	}
}
\Page{
	\Theorem{AperidocConditionForRecurrent}
	{
		\forall A \in \BOOL \. 
		\forall f \in  \TYPE{Recurrent}(A) \.
		\TYPE{Aperiodic}(A,f) 
		\iff
		\NewLine
		\iff
		\TYPE{RelativelyAtomless}\Big( A, \Fix(f) \Big)
	}
	\Assume{[1]}{\TYPE{Aperiodic}(A,f)}
	\Assume{a}{\Atom_A\Big(\Fix(f)\Big)}
	\Say{[2]}{\Elim \TYPE{Recurrent}(A,f,a)}{\TYPE{RecurrentOn}(A,a,f)}
	\SayIn{n}{\min \{n\in \Nat : f^n(a)a \neq 0 \}}{\Nat}
	\AssumeIn{b}{\genIdeal{af^n(a)}}
	\Say{\Big( b, [4]\Big)}{\Elim \Atom_A\Big(\Fix(f)\Big)}
	{
		\sum c \in \Fix(f) \.
		b = ca
	}
	\Say{[5]}{[4]\Elim \BOOL(A,A,f^n) \Elim \Fix(f,c) \THM{MeetIneq}(A)\Elim b [4]}
	{
		f^n(b) = 
		f^n(ca) = 
		cf^n(a) \ge 
		caf^n(a) \ge b
	}
	\SayIn{d}{\bigvee^{n-1}_{k=0} f^k(b)}{A}
	\Say{[6]}{\Elim \sup \Elim d [5]}{ f(d) \ge d  }
	\Say{[7]}{\THM{SupremumLemma}[6]}{f(d) = d}
	\Say{[8]}{\Lambda k \in [1,\ldots,n-1] \. \Elim \BOOL(A,A,f^k)\Elim b \Elim n }
	{
		\NewLine 
		\forall k \in [1,\ldots,n-1] \.		
		f^n(b) f^k(b)  =
		f^k\Big( f^{n-k}(b) b \Big)  \le
		f^k\Big( f^{n-k}(a) a \Big) = 0
	}
	\Say{[9]}{\Elim d [7]}{f^n(b) \le \sup_{0 < k < n} f^k(b)}	
	\Conclude{[b.*]}{[5][8][9]}{f^n(b) = b}
	\Derive{[4]}{\Intro \forall}{\forall b \in \genIdeal{af^{n}(a)} \. f^n(b) = b}
	\Conclude{[a.*]}{\Elim \TYPE{Aperiodic} }{\bot}
	\DeriveConclude{[1.*]}{\Elim \bot}{ \Atom_A\Big( \Fix(f) \Big) = \emptyset}
	\Derive{[1]}{\Intro \Imply}
	{
		\TYPE{Aperiod}(A,f) 
		\Imply
		\Atom_A\Big( \Fix(f) \Big) = \emptyset
	}
	\Assume{[2]}{  \Atom_A\Big( \Fix(f) \Big) = \emptyset  }
	\Assume{[3]}{\TYPE{Aperiodic}(A,f)}
	\SayIn{n}{\min \{ n \in \Nat : \supp f^n \neq e \}}{\Nat}
	\Say{\Big( a, [4]  \Big)}{\Elim n \Elim \supp f^n}
	{
		\sum a \in A \. a \neq 0 \And \forall b \in \genIdeal{a} \. f^n(b) = b	
	}
	\Assume{b}{\genIdeal{a}}
	\Assume{[5]}{b \neq 0}
	\AssumeIn{k}{[1,\ldots,n-1]}
	\Say{\Big(c,[6]\Big)}{\Elim n \Elim \supp (k)}
	{
		\sum c \in \genIdeal{b} \.
		f^k(c) \neq c
	}
	\Say{d}{
		\If  
		c \setminus  f^k(c) \neq 0
		\Then
		c \setminus  f^k(c)	 
		\Else	
		c \setminus f^{n-k}(c)
	}
	{
		\genIdeal{c}
	}
	\NoProof	
}\Page{
	\DeclareType{Ergodic}{\prod_{A \in \BOOL} ?\Aut_\BOOL(A)}
	\DefineType{f}{Ergodic}{\forall a,b \in A \setminus \{0\}  \. 
		\exists n \in \Nat : f^n(a)b \neq 0	
	}
	\\
	\Theorem{AperidocConditionForErgodic}
	{
		\forall A \in \BOOL \. 
		\forall f \in  \TYPE{Ergodic}(A) \.
		\TYPE{Aperiodic}(A,f) 
		\iff
		\Aless(A)
	}
	\NoProof
	\\	
}
\newpage
\subsubsection{Interaction with Stone Spaces}
\Page{
	\Theorem{StoneAutomorphismAgreement}
	{
		\forall A \in \BOOL \.
		\forall a,b \in A \.
		\forall \genIdeal{a} \Arrow{f} \genIdeal{b} : \BOOL \. \NewLine \.
		\forall \varphi \in \Aut_\BOOL(A) \.
		\varphi_{|\genIdeal{a}} = f \iff
		\Z(\varphi)_{|\Z\;\genIdeal{b}} = \Z(f)
	}
	\Assume{[1]}{\varphi_{|\genIdeal{a}} = f}	
	\AssumeIn{u}{\Z \; \genIdeal{b}}
	\Conclude{[u.*]}{
		\Elim \Z(f)
		[0]
		\Intro \Z(\varphi)	
	}
	{
		\Z(f)(b) = 
		u \circ f  =
		u \circ \varphi  =
		\Z(\varphi)(b) 
	}
	\DeriveConclude{[1.*]}{\Intro(=,\to)}{\Z(\varphi)_{|\Z \; \genIdeal{b}} = \Z(f) }
	\Derive{[2]}{\Intro \Imply}
	{
		\varphi_{|\genIdeal{a}} = f \Imply
		\Z(\varphi)_{|\Z\;\genIdeal{b}} = \Z(f)
	}
	\Assume{[2]}{\Z(\varphi)_{|\Z\;\genIdeal{b}} = \Z(f)}
	\Conclude{[2.*]}{\TK[2]}{f = \varphi_{|\genIdeal{a}}}
	\DeriveConclude{[*]}{\Intro \iff [1]}
	{
		\varphi_{|\genIdeal{a}} = f \iff
		\Z(\varphi)_{|\Z\;\genIdeal{b}} = \Z(f)
	}
	\EndProof
	\\
	\Theorem{StoneSupports}
	{
		\forall A \in \BOOL \.
		\forall f \in \End_\BOOL(A) \.
		\forall a \in A \. 
		a \in \Supp f \
		\iff \NewLine \iff
		S_A(a) \subset \overline{\Big\{v \in  \Z(A): \Z(f)(v) \neq v \Big\}}
	}
	\Assume{[1]}{a \in \Supp(f)}
	\Say{[2]}{\Elim \Supp(f,a)}{\forall b \in \genIdeal{a^\c} \. f(b) = b}
	\Say{[3]}{\Intro S_A(a) \Intro \Z(f)[2]}
	{
		\forall v \in S_A^\c(a) \. \Z(f)(v) = v
	}
	\Say{[4]}{[3]^\c}
	{
		\Big\{v \in  \Z(A): \Z(f)(v) \neq v \Big\} \subset S_A(a)
	}
	\Conclude{[1.*]}{\Intro {\cl}_{\Z(A)}[4]}
	{
		\overline{ \Big\{ v \in  \Z(A): \Z(f)(v) \neq v \Big\}} \subset S_A(a)
	}
	\DeriveConclude{[*]}{}
	{
		a \in \Supp g 
		\iff
		S_A(a) \subset \overline{\Big\{v \in  \Z(A): \Z(f)(v) \neq v \Big\}}	
	}
	\EndProof
	\\
	\Theorem{StoneSupports}
	{
		\forall A \in \BOOL \.
		\forall f \in \End_\BOOL(A) \.
		\forall a \in A \. 
		a  = \supp f \
		\iff \NewLine \iff
		S_A(a) = \overline{\Big\{v \in  \Z(A): \Z(f)(v) \neq v \Big\}}
	}
	\NoProof
}\Page{
	\Theorem{FullSubgroupDenseProperty}
	{
		\forall A : \TAlgebra \.
		\forall f,g \in \Aut_\BOOL(A) \. \NewLine \.
		g \in \genFS{f} \iff 
		\Dense\left( 
			\Z(A) , \bigcup_{n = -\infty}^\infty \intx  
			\Big\{ 
				v \in \Z(A) :
			    \Z(g)(v) = \Z(f^n)(v)   
			\Big\}  
		\right)
	}
	\Say{\Big(p,[1]\Big)}
	{\THM{CountablyFullSubgroupGeneratedByGrouopElement}(A,f,g)}
	{
		\NewLine :		
		\sum p : \Int \to A \. 
		\PoU(A,\im p) \And 
		\forall n \in \Int \. 
		\forall b \in \genIdeal{p_n} \.
		g(b)  = f^n(b)	
	}
	\Say{[2]}{
		\Z [1.2]	
	}
	{
		\forall n \in \Int	\.	
		S_A(a) = 
		\Big\{
			v \in \Z(A) :
			(\Z \; g)(v) = (\Z \; f^n)(v)
		\Big\}	
	}	
	\Say{[3]}{\THM{SupremumStoneExpression}(A) \Elim \PoU(A,\im p) [2] }
	{
		\NewLine :		
		\Z(A) =  
		\cl \bigcup_{n=-\infty}^\infty \intx S_A(p_n) =
		\bigcup_{n = -\infty}^\infty \intx  
			\Big\{ v \in \Z(A) :\Z(g)(v) = \Z(f^n)(v) \Big\}    
	}
	\Conclude{[*]}{\Intro \Dense [3]}
	{
		\Dense\left( 
			\Z(A) , \bigcup_{n = -\infty}^\infty \intx  
			\Big\{ 
				v \in \Z(A) :
			    \Z(g)(v) = \Z(f^n)(v)   
			\Big\}  
		\right)
	}
	\EndProof
	\\
	\Theorem{FullSugroupComeager}
	{
		\forall A : \TAlgebra \.
		\forall f,g \in \Aut_\BOOL(A) \. \NewLine \.
		g \in \genFS{f} \iff 
		\Comeager\bigg( 
			\Z(A)  ,
			\Big\{ 
				v \in \Z(A) :
			    \Z(g)(v) \in \{  \Z(f^n)(v)   | n \in \Nat \} 
			\Big\}  
		\bigg)
	}
	\Assume{[1]}
	{
		\Comeager\bigg( 
			\Z(A)  ,
			\Big\{ 
				v \in \Z(A) :
			    \Z(g)(v) \in \{  \Z(f^n)(v)   | n \in \Nat \} 
			\Big\}  
		\bigg)
	}
	\Say{F}{\Lambda n \in \Int \. \Big\{ v  \in \Z(A) \. \Z(g)(v) = \Z(f^n)(v)  \Big\}}
	{
		\Int \to ?\Z(A)
	}
	\Say{[2]}{\THM{NowhereDenseConstruction}(\Z\;A,F)}
	{
		\forall n \in \Int \. \ND\Big( \Z\;A, F_n \setminus \intx F_n \Big)
	}
	\Say{[3]}{\Elim \Comeager[2][1]\Intro \Comeager}
	{
		\Comeager\left(\Z\;A, \bigcup^\infty_{n=-\infty} \intx F_n \right)
	}
	\Say{[4]}{\THM{BairTHM}[3]}
	{
		\Dense\left(\Z\;A, \bigcup^\infty_{n=-\infty} \intx F_n \right)
	}
	\AssumeIn{a}{A}
	\Say{\Big( n, [6] \Big) }{\Elim \Dense [4](a)}
	{
		\sum a \in A \.  S_A\Big( g(a) \Big) \cap \intx F_n \neq \emptyset
	}
	\Say{\Big(b,[7]\Big)}{\THM{StoneTHM}[6]}
	{
		\sum b \in A \. S_A(b) \subset   S_A\Big( g(a) \Big) \cap \intx F_n
	}
	\Say{[8]}{\Elim \Z(g)[7]}{  g^{-1}(b) \le a  }
	\Conclude{[a.*]}{\Elim F_n [8]}
	{
		\forall c \in \genIdeal{b} \.
		g(c) = f^n(c)	
	}
	\DeriveConclude{[*]}{\THM{FullSubgroupGeneratedByGroupElement}}
	{
		g \in \genFS{f}
	}
	\EndProof
}
\Page{
	\Theorem{RecurrentStoneCriterion}
	{
		\forall A \in \BOOL \.
		\forall f \in \End_\BOOL(A) \.
		\forall a \in A \.
		\TYPE{RecurrentOn}(A,f,a) 
		\iff
		\NewLine 
		\iff
		S_A(a) \subset \overline{\bigcup_{n=1}^\infty \Z(f^n)\Big(S_A(a)\Big)}
	}
	\Assume{[1]}{\TYPE{RecurrentOn}(A,f,a)}
	\Say{[2]}{\Elim \TYPE{RecurrentOn}(A,f,a)}
	{
		\forall b \in \genIdeal{a} \setminus \{0\} \.
		\exists k \in \Nat :
		a f^k(b) \neq 0
	}
	\Say{[3]}{\Intro S_A \Intro \Z [2]}
	{
		\forall b \in \genIdeal{a} \setminus \{0\} \.
		\exists k \in \Nat :
		S_A(a) \cap \Big( \Z(f^k) \Big)^{-1}S_A(b) \neq \emptyset
	}
	\Say{[4]}{\Intro \FUNC{image} [3]}
	{
		\forall b \in \genIdeal{a} \setminus \{0\} \.
		\exists k \in \Nat :
		\Big( \Z(f^k) \Big)\Big(S_A(a)\Big) \cap S_A(b) \neq \emptyset
	}
	\Say{[5]}
	{
		\Intro \Dense [4]
	}
	{
		\Dense
		\left( S_A(a),   
			S_A(a) \cap \bigcup^\infty_{k=1} \Z(f^k)\Big(S_A(a)\Big)   
		\right)
	}
	\Conclude{[*]}{\Intro \cl [5]}
	{
		S_A(a) \subset \overline{\bigcup^\infty_{n=1} \Z(f^k)\Big(S_A(a)\Big) }
	}
	\EndProof
	\\
	\Theorem{InducedHomeomorphismSetting}{
		\forall A : \SA \.
		\forall a \in A \.
		\forall f : \TYPE{RecurrentOn}(A,a) \. \NewLine \.
		\bigcup_{n=1}^\infty G_n =
		S_A(a) \cap \bigcup^\infty_{n=1} \Z(f^{-n})\Big( S_A(a) \Big)	\NewLine
		\where \NewLine
		G = \Lambda k \in \Nat \.
		\Big\{
			v \in \genIdeal{a} :
			f^k(v) \in S_A(a)   \And
			\forall i \in \{1,\ldots,k-1\}
			f^i(v) \not \in S_A(a)  
		\Big\}	
	}
	\NoProof
	\\
	\Theorem{InducedHomeomorphismSetting}{
		\forall A : \SA \.
		\forall a \in A \.
		\forall f : \TYPE{RecurrentOn}(A,a) \. \NewLine \.
		\bigcup_{n=1}^\infty G_n =
		S_A(a) \cap \bigcup^\infty_{n=1} \Z(f^{-n})\Big( S_A(a) \Big)	\NewLine
		\where \NewLine
		G = \Lambda k \in \Nat \.
		\Big\{
			v \in \genIdeal{a} :
			f^k(v) \in S_A(a)   \And
			\forall i \in \{1,\ldots,k-1\}
			f^i(v) \not \in S_A(a)  
		\Big\}	
	}
	\NoProof
	\\
	\Theorem{InducedHomeomorphismProperty}{
		\forall A : \SA \.
		\forall a \in A \.
		\forall f : \TYPE{RecurrentOn}(A,a) \. \NewLine \.
		\forall k \in \Nat \.
		\forall v \in G_k \.
		\Z(f_a)(v) = \Z^k(f)(v)
		\where \NewLine
		G = \Lambda k \in \Nat \.
		\Big\{
			v \in \genIdeal{a} :
			f^k(v) \in S_A(a)   \And
			\forall i \in \{1,\ldots,k-1\}
			f^i(v) \not \in S_A(a)  
		\Big\}	
	}
	\NoProof
}
\newpage
\subsubsection{Exchanging Automorphisms}
\Page{
	\DeclareType{AutomorphismChain}
	{
		\BOOL \to  ? \TYPE{MorphismChain}\Big(\BOOL^*\Big)
	}
	\DefineNamedType{(n,B_\bullet,a_\bullet,f_\bullet)}{AutomorphismChain}
	{
		a_1 \Arrow{f_1} a_2 \Arrow{f_2} \ldots \Arrow{f_{n}} a_{n+1} : 
		\TYPE{AutomorphismChain}(A)
	}
	{
		\NewLine :		
		\Lambda A \in \BOOL \.		
		n < \infty  \And
		\forall i \in \sigma(n) \.  A_i = B \And
		\forall i \in n \.  f_i \in \Aut_\BOOL(A)    	
	}
	\\
	\DeclareType{ExchangeChain}
	{
		\prod_{A \in \BOOL} ?\TYPE{AutomorphismChain}\Big(\BOOL^*\Big)
	}
	\DefineNamedType{(n,a_\bullet,f_\bullet)}{ExchangeChain}
	{
		a_1 \Arrow{f_1} a_2 \Arrow{f_2} \ldots \Arrow{f_{n}} a_{n+1} : 
		\TYPE{ExchangeChain}(A)
	}
	{
		\NewLine \iff
		\PD(A,\im a)  	
	}
	\\
	\DeclareFunc{exchangingAutomorphism}
	{
		\prod_{A \in \BOOL} 
		\TYPE{ExchangeChain} 
		\to
		\Aut_\BOOL(A) 
	}
	\DefineNamedFunc{exchangingAutomorphism}
	{
		n, a_\bullet, f_\bullet
	}
	{
		\overleftarrow{a_{1 f_1} a_{2f_2} \ldots_{f_n}a_{n+1}}
	}
	{
		\Lambda b \in A \.
		\If \exists i \in n : b \le a_i \Then f_i(b) \Else \NewLine
		\If b \subset a_{n+1} \Then  \prod^n_{i=1} f^{-1}_{n+1-i}(b) 
		\Else b
	}
	\\
	\Theorem{ExchangingAutomorphismOrder}
	{
		\forall A \in \BOOL \.
		\forall a_1 \Arrow{f_1} a_2 \Arrow{f_2} \ldots \Arrow{f_{n}} a_{n+1} : 
		\TYPE{ExchangeChain}(A) \. \NewLine \.
		\ord \overleftarrow{a_{1 f_1} a_{2f_2} \ldots_{f_n}a_{n+1}} = n + 1
	}
	\NoProof
	\\
	\Theorem{PreservationUnderCycling}
	{
		\forall A \in \BOOL \.
		\forall a_1 \Arrow{f_1} a_2 \Arrow{f_2} \ldots \Arrow{f_{n}} a_{n+1} : 
		\TYPE{ExchangeChain}(A) \. \NewLine \.
		\forall \gamma : \TYPE{Cycle}(n) \.
		\overleftarrow{a_{1 f_1} a_{2f_2} \ldots_{f_n}a_{n+1}} =
		\overleftarrow{a_{\gamma(1) f_{\gamma(1)}} a_{\gamma(2)f_{\gamma(2)}}
		 \ldots_{f_{\gamma(n)}}a_{\gamma(n) + 1}}
	}
	\NoProof
	\\
	\Theorem{ExchangingAutomorphismConjugation}
	{
		\forall A \in \BOOL \.
		\forall a_1 \Arrow{f_1} a_2 \Arrow{f_2} \ldots \Arrow{f_{n}} a_{n+1} : 
		\TYPE{ExchangeChain}(A) \. \NewLine \.
		\forall \phi \in \Aut_\BOOL(A) \.
		\phi^{-1}\overleftarrow{a_{1 f_1} a_{2f_2} \ldots_{f_n}a_{n+1}} \phi=
		\overleftarrow{\phi(a_{1})_{ \phi^{-1} f_1 \phi} 
		\phi(a_2)_{\phi^{-1}f_2\phi}
		 \ldots_{\phi^{-1}f_{n}\phi}\phi(a_{n + 1})}
	}
	\NoProof
	\\
	\Theorem{ExchangingAutomorphismsComposition}
	{
		\forall A \in \BOOL \. \NewLine \.
		\forall a_1 \Arrow{f_1} a_2 \Arrow{f_2} \ldots \Arrow{f_{n}} a_{n+1}, 
		\forall b_1 \Arrow{g_1} b_2 \Arrow{b_2} \ldots \Arrow{g_{n}} b_{n+1} :
		\TYPE{ExchangeChain}(A) \. \NewLine \.
		\PD\Big( A, \im a \cup \im b\Big) \Imply \NewLine \Imply
		\overleftarrow{a_{1 f_1} a_{2f_2} \ldots_{f_n}a_{n+1}}
		\overleftarrow{b_{1 g_1} b_{2f_2} \ldots_{g_n}b_{n+1}} =
		\overleftarrow{(a_1 + b_2)_{f_1g_1} (a_2 + b_2)_{f_2g_2} 
		\ldots_{f_ng_n}(a_{n+1} + b_{n+1})}
	}
	\NoProof
}
\Page{
	\Theorem{ExchangingAutomotphismInAFullSubgroup}
	{		
		\forall A \in \BOOL \.
		\forall a_1 \Arrow{f_1} a_2 \Arrow{f_2} \ldots \Arrow{f_{n}} a_{n+1} 
		:\TYPE{ExchangeChain}(A) \.
		\NewLine
		\overleftarrow{a_{1 f_1} a_{2f_2} \ldots_{f_n}a_{n+1}} \in 
		\genCFS{f_1,\ldots,f_n}
	}
	\NoProof
	\\
	\Theorem{ExchangingInvolutionAsjoining}
	{
		\forall A \in \BOOL \.
		\forall a \Arrow{f} b,b \Arrow{g} c : \TYPE{ExchangeChain}(A) \. \NewLine 
		\PD\Big(A, \{a,b,c\} \Big)		
		\overleftarrow{a_f b}
		\overleftarrow{b_g c}  =
		\overleftarrow{a_f b_g c} 
	}
	\NoProof
	\\
	\DeclareType{ExchangingInvolution}
	{
		\prod_{A \in \BOOL} \Aut_\BOOL(A)
	}
	\DefineType{f}{ExchangingInvolution}
	{
		\exists a \Arrow{g} b : \TYPE{ExchangeChain}(A) \.
		f = \overleftarrow{a_g b}
	}
}
\newpage
\subsection{Factorization Theorems in an Automorphisms Group[!!]}
\subsubsection{Separators and Transversals}
\Page{
		\DeclareType{Separator}{
			\prod_{A \in \BOOL}
			\Aut_\BOOL(A) \to ?A
		}
		\DefineNamedType{a}{Separator}
		{
			\Lambda f \in 	\Aut_\BOOL(A) \. a \in \Sep(f) \.	
		}
		{
			\NewLine \iff 			
			\Lambda f \in \Aut_\BOOL(A) \.
			af(a) = 0
			\And
			\forall b \in A \. 
			\forall n \in \Int_+ \.
			bf^n(a) = 0
			\Imply
			f(b) = b
		}
		\\
		\DeclareType{Transversal}
		{
			\prod_{A \in \BOOL}
			\Aut_\BOOL(A) \to ?A
		}
		\DefineNamedType{a}{Transversal}
		{
			\Lambda f \in \Aut_\BOOL(A) \. a \in \Tr(f) 	
		}
		{
			\NewLine \iff			
			\sup_{n \in \Int} f^n(a) = e\And
			\forall n \in \Int \.
			\forall b \in \genIdeal{af^n(a)} \.
			f^n(b) = b
		}
		\\
		\Theorem{TransversalConstructionLemma}
		{
			\forall A \in \BOOL \.
			\forall f \in \Aut_\BOOL(A) \.
			\forall n \in \Nat \.
			\forall a : \prod^n_{k=1} \Sep(f^k) \. \NewLine \.
			f^{n+1} = \id \Imply \exists \Tr(f)
		}
		\Say{X}{\Big\{ f^i(a_j) \Big| i,j \in [1,\ldots,n] \Big\}}{\Finite(A)}
		\Say{B}{\langle X \rangle_\RING}{\TYPE{Subring}(A)}
		\Say{[1]}{\Elim \Finite(A,X) \Elim B}{|B|< \infty}
		\Say{[2]}{[1]\Intro \PA}{\PA(B) }
		\SayIn{G}{\langle f  \rangle_\GRP }{\GRP}
		\Say{\alpha}{\Lambda g \in G \. \Lambda b \in B \. g(b)}{ G \ActsOn \Atom(B)}
		\Say{\O}{\Big\{ O_\alpha(b) | b \in \Atom(B)  \Big\}}
		{\TYPE{Partition}\Big(\Atom(B)\Big)}		
		\AssumeIn{C}{\O}
		\SayIn{m}{|C|}{\Nat}
		\Say{[3]}{[0]\Elim m}{m \le n + 1}
		\AssumeIn{c}{C}
		\AssumeIn{d}{{\genIdeal{c}}_{,A}}
		\Say{[4]}{\Elim m \Elim \FUNC{orbit}}
		{
			\forall k \in \Int \. f^{m+k}(c) = c
		}
		\Say{[5]}{\Elim \Sep(f^m,a_m)[4]}{\forall k \in \Int \. a_m f^k(c) = 0}
		\Say{[6]}{\Elim d [5]}{ \forall k \in \Int \.  a_m f^k(d) = 0}
		\Conclude{[C.*]}{\Elim \Sep(f^m,a_m)[6]}{f^m(d) = d}
		\Derive{[3]}{\Intro \forall}
		{
			\forall C \in \O \. 
			\forall c \in C \.
			\forall d \le_A c \.  f^{|C|}(d) = d
		}
		\SayIn{b}{\THM{FiniteChoice}(\O)}{\prod_{C \in \O} C}
		\SayIn{t}{\bigvee_{C \in \O} b_C}{A}
		\SayIn{[4]}{
			\Elim t \Elim \Aut_\BOOL(A,f)
			\Elim \O
			\Elim \TYPE{Partition}\Big(\O, \Atom(B) \Big) \Elim \PA(B)
			\Elim \TYPE{Subring}(A,B)
		}
		{
			\NewLine :			
			\bigvee_{k = 1}^n f^k(t) =  
			\bigvee_{k = 1}^n \bigvee_{C \in \O} f^k(b_C)  =
			\bigvee_{C \in \O} \bigvee  C = 
			\bigvee B = 
			e_A
		}
}
\Page{
	\AssumeIn{m}{\Int}
	\Assume{c}{\genIdeal{tf^m(t)}}
	\Say{k}{m \mod n + 1}{[0,\ldots, n]}
	\Say{[5]}{\Elim c [0]}{c \le tf^k(t)}
	\Say{\Big(\O',d,[6]\Big)}{\Elim c \Elim t \Elim \Atom(B,b)}
	{
		\sum \O' \subset \O \.  
		\sum d : \prod_{C \in \O'} {\genIdeal{b_C}}_A
		c = \bigvee_{C \in \O'} d_C
	}
	\Say{[7]}{\Lambda C \in \O' \. \Elim d_C \Elim b_C  \Elim \Atom(B,b_C)
		\Elim \TYPE{Partition}\Big(\O,\Atom b\Big) [5] }
	{
		\forall C \in \O' \. f^k(d_C) \le f^k(b_C) = b_C
	}
	\Say{[8]}{\Elim \FUNC{orbit} [7]}
	{
		\forall C \in \O' \.  k \; \vdots \; |C|
	}
	\Say{[9]}{[3][8]}{\forall C \in \O \. f^k(d_C) = d_C}
	\Conclude{[m.*]}{\Elim m [0][9]\Elim \Aut_\BOOL(A,f)[6]}{f^m(c) = f^k(c) = c }
	\DeriveConclude{[*]}{[4]\Intro \Tr(f)}{t \in \Tr(f)}
	\EndProof
	\\
	\Theorem{ExchangingInvolutionBySeparator}
	{
		\forall A \in \BOOL \.
		\forall f : \TYPE{Involution}(A) \. \NewLine \.
		\EI(A,f) \iff \exists \Sep(f)
	}
	\Assume{[1]}{\EI(A,f)}
	\Say{\Big( a,b,g,[2]\Big)}{\Elim [1] }
	{
		\sum a \Arrow{g} b : \TYPE{ExchangeChain} \.
		f = \overleftarrow{a_g b}
	}
	\Say{[3]}{\Elim \TYPE{ExchangeChain}(a,b,g)}{ab = 0}
	\Say{[4]}{ \Elim [2] \big(f(a)\big)}{f(a) = b}
	\Say{[5]}{[4][3]}{af(a) = 0}
	\Say{[6]}{[2] \Elim \FUNC{exchangingAutomorphism}}
	{
		\forall c \in A \. \Big(\forall n \in \Int \. cf^n(a) = 0  \Big) \Imply f(c) = c
	}
	\Conclude{[1.*]}{\Intro \Sep(f)}{a \in \Sep(f)}
	\Derive{[1]}{\Intro \Imply}{ 
		\EI(A,f) \Imply \exists \Sep(f)
	}
	\AssumeIn{a}{\Sep(f)}
	\Say{[2]}{\Elim_1 \Sep(f,a)}{af(a) = 0}
	\Say{[3]}{\Elim \TYPE{Involution}(A,f)(a)}{f^2(a) = a}
	\Say{[5]}{
		\Lambda c \in \genIdeal{(a + f(a))^\c} \. 
		\THM{DeMorganaLaw}(A)[2] \Elim \c
	}
	{
		\forall c \in \genIdeal{(a + f(a))^\c} \. ac \And f(a)c = 0
	}
	\Say{[6]}{[5][3]}{
		\forall c \in \genIdeal{(a + f(a))^\c} \. 
		\forall n \in \Int \.  f^n(a)c = 0
	}
	\Say{[7]}{\Elim_2 \Sep(f,a)[6]}{
		\forall c \in \genIdeal{(a + f(a))^\c} \. 
		f(c) = c
	}
	\Conclude{[a.*]}{\Elim \FUNC{exchangingAutomorphism} [2][3][7]}
	{
		f = 
		\overleftarrow{a_f f(a)}
	}
	\Conclude{[*]}{\Intro \iff [1]}
	{
		\forall A \in \BOOL \.
		\forall f : \TYPE{Involution}(A) \. 
		\EI(A,f) \iff \exists \Sep(f)
	}
	\EndProof
}\Page{
	\Theorem{ExchangingInvolutionByTransversal}
	{
		\forall A \in \BOOL \.
		\forall f : \TYPE{Involution}(A) \. \NewLine \.
		\EI(A,f) \iff \exists \Tr(f)
	}
	\Assume{[1]}{\EI(A,f)}
	\SayIn{a}{\THM{ExchangingInvolutionBySeparator}(A,f)[1]}
	{
		\Sep(f)
	}
	\Say{[2]}{\Elim \TYPE{Involution}(A,f)}{f^2 = \id}
	\Conclude{t}{\THM{TransversalConstructionLemma}(A,f,a)[1]}
	{
		\Tr(f)
	}
	\Derive{[1]}{\Intro \exists \Intro \Imply}
	{
		\EI(A,f) \Imply \exists\Tr(f)
	}
	\AssumeIn{t}{\Tr(f)}
	\Say{[2]}{\Elim \TYPE{Involution}(A,f)(t)}{f^2(t) = t}
	\Say{[3]}{\Elim \Tr(f,t)[2]}{f(t) \vee t = e}
	\SayIn{a}{t^\c}{A}
	\Say{[4]}
	{
		\Elim a
		\Elim \Aut_\BOOL(A,f) 
		[3]
		\Elim \c 	
	}
	{
		f(a)a = 
		f(t^\c)t^\c =
		\Big( f(t) t \Big)^\c =
		e^\c = 
		0 
	}
	\Say{[5]}{\THM{DeMorganaLaw}(A)[4]\Elim t \Elim \Aut_\BOOL(A)\THM{DoubleNegationLaw}(A)}
	{
		\forall  b \le (a + f(a) )^\c \. b \le t f(t)
	}
	\Say{[6]}{\Elim \Tr(f,t)[5]}
	{
		\forall  b \le (a + f(a) )^\c \. f(b) = b
	}
	\Say{[7]}{\Elim \TYPE{Involution}(A,f)(a)}{f^2(a) = a}
	\Conclude{[a.*]}{\Elim \FUNC{exchangingAutomorphism} [4][6][7]}
	{
		f = 
		\overleftarrow{a_f f(a)}
	}
	\DeriveConclude{[*]}{\Intro \iff [1]}
	{
		\EI(A,f) \iff \exists \Tr(f)
	}
	\EndProof
}
\newpage
\subsubsection{Frolik's Theorem}
\Page{
	\Theorem{FroliksLemma1}
	{
		\forall A : \SA \.
		\forall f \in \Aut_\BOOL(A) \.
		\forall s \in \Sep(f) \.
		\NewLine \.
		\exists y \in A :
		y \vee f(y) \vee f^2(y) \in \Supp(f)
		\And
		yf(y) = 0
	}
	\SayIn{a}{\bigvee_{n=1}^\infty f^n(s)}{A}
	\SayIn{b}{\bigvee_{n=1}^\infty f^{-n}(s)}{A}
	\Say{[1]}{\Elim_1 \Sep(f,s) }{f(s)s = 0}
	\Say{[2]}{\Elim_2 \Sep(f,s) \Intro \Supp }{ s \cup a \cup b \in \Supp f}
	\Say{x}{\Lambda n \in \Int_+ \. f^n(s) \setminus \bigvee^{n-1}_{k=1} f^k(s)}
	{ \Nat \to A}
	\Say{[3]}{\Elim x \Elim a}{a \vee s = \bigvee_{n=1}^\infty x_n} 
	\Say{[4]}{\Elim x \Intro \PD}{\PD\Big( A ,\im x \Big)}
	\SayIn{y_1}{ \bigvee_{n=1}^\infty x_{2n} \setminus f^{-1}(s)}{A}
	\Say{[5]}{f^{-1}[1]}{f^{-1}(s)s=0} 
	\Say{[6]}{\Elim y_1 [5]}{ s \le y_1 \le s \vee a }
	\Say{[7]}{
		\Lambda n \in \Nat \.	
		\Elim x
		\Elim \Aut_\BOOL(A,f)
		\Intro x
	}
	{
		\NewLine :		
		\forall n \in \Nat \.
		f\Big( x_{2n} \setminus f^{-1}(s) \Big) =
		f\left(  f^{2n}(s) \setminus \bigvee^{2n-1}_{k=-1} f^k(s) \right) = 
		f^{2n+1}(s) \setminus \bigvee^{2n}_{k=0} f^k(s) = x_{2n + 1}
	}
	\Say{[8]}{\Elim y_1 [7] \Elim \PD(A,\im x)[4]}
	{
		f(y_1)y_1 = 0
	}
	\Say{[9]}{\Elim y_1 \Elim a}{f(y_1) \le a}
	\Say{[10]}{\Elim y_1 \Elim a}{a \setminus f^{-1}(s) \le y_1 \vee f(y_1) }
	\SayIn{c}{s \setminus a}{A}
	\Say{[11]}{
		\Lambda i,j \in \Int \.
		\Lambda T : i < j \.	
		\Elim  \Aut_\BOOL(A,f) 
		\Elim c
		\THM{DifferenceProductBound}(A)
		\Elim a
		\THM{ZeroImage}(A,A,f) 		 	 	
	}
	{
		\NewLine :		
		\forall i,j \in \Int \.   i < j \Imply	
		f^i(c) f^j(c) = 
		f^j\Big(  c f^{i-j}(c) \Big) = 
		f^i\Big( \big( s \setminus a) \big( f^{j-i}(s) \setminus f^{j-i}(a)\big)\Big) \le
		f^i\Big( s \setminus f^{j-i}(a) \Big) = 0
	}
	\Say{[12]}{ \Elim c 
		\THM{CommonDiferenceUnion}(A)
		\THM{TelescopingUnion}(A) 
		\setminus \Intro b
	}
	{
		\NewLine :	
		\bigvee_{n=1}^\infty f^{-n}(c) = 
		\bigvee_{n=1}^\infty 
		\left( f^{-n}(s) \setminus \bigvee_{k={-n+1}}^\infty f^k(s) \right)
		=
		\bigvee_{n=1}^\infty 
		\left( f^{-n}(c) \setminus \bigvee_{k={-n+1}}^{-1} f^k(c) \right)
		\setminus (s \vee a)=  \NewLine =
		\bigvee_{n=1}^\infty  f^{-n}(a) \setminus (s \vee a) = 
		b \setminus (s \vee a)
 	}
 	\Say{[13]}{
 		\Lambda k \in \Nat \.
 		\Lambda i \in \Int_+ \.
 		\Elim \Aut_\BOOL(A,f) \.
 		\Elim c 
 		\THM{ZeroImage}(A,A,f) 		
 	}
 	{
		\NewLine : 		
 		\forall k \in \Nat \. 
 		\forall i \in \Int_+ \.
 		f^{-k}(c) f^i(s) = 
 		f^{-k}\Big( c f^{i + k}(s) \Big) = 
 		f^{-k}(0) =
 		0 
 	}
 	\Say{[14]}{[13]\Intro y_1}{\forall k \in \Nat \. f^{-k}(cy_1) = 0}
	\SayIn{y}{y_1 \vee \bigvee^\infty_{n=1} f^{-2n}(c)}{A} 	
 	\Say{[15]}{\Elim y \Elim \TYPE{AssociativeLattice}(A) [14][11]}{  yf(y)  = 0}
}
\Page{
	\Say{[16]}{\Elim y [12]}
	{
		y \vee f(y) \vee f^{-1}(y) \ge
		y_1 \vee f(y_1) \vee f^{-1}(s) \vee \bigvee_{n =1 1}^\infty f^{-1}(c) \ge 
		s \vee a \vee \Big(b \setminus (s \vee a)\Big) = 
		s \vee a \vee b 	
	}
	\Say{[17]}{\THM{SupportContainsGreater}[2][16]}
	{
		y \vee f(y) \vee f^{-1}(y) \in \Supp(f)
	}
	\Conclude{[*]}{\Intro  f^{-1} [17]}
	{
		f^{-1}(y) \vee 
		f\Big( f^{-1}(y) \Big)
		\vee
		f^2\Big( f^{-1}(y) \Big) \in \Supp(f)
	}
	\EndProof
	\\
	\Theorem{TripleSupportImpliesSequenceSupport}
	{
		\NewLine ::
		\forall A : \SA \.
		\forall f \in \Aut_\BOOL(A) \.
		\forall y \in A \.
		y \vee f(y) \vee f^2(y) \in \Supp(f) 
		\And y f(y) = 0		
		\Imply \NewLine \Imply
		\exists a : \Nat \to A :
		\bigvee^\infty_{n=1} f(a_n) \setminus a_n \in \Supp(f)
	}
	\Say{a}{\Lambda n \in \Nat \. f^{n-2}(y)}{\Nat \to A}
	\Say{[1]}{\Elim a [0.2] \THM{LatticeJoinIsGreater}}
	{
		\bigvee^\infty_{n=1}  f(a_n) \setminus a_n = 
		\bigvee^\infty_{n=0} f^n(y)  \ge
		 y \vee f(y) \vee f^2(y)
	}
	\Conclude{[*]}{\THM{SupportContainsGreater}[1][0.1]}
	{
		\bigvee^\infty_{n=1} f(a_n) \setminus a_n \in \Supp(f) 
	}
	\EndProof
}\Page{
	\Theorem{FroliksLemma2}
	{
		\forall A : \SA \.
		\forall f \in \Aut_\BOOL(A) \.
		\forall a : \Nat \to A \.
		\bigvee^\infty_{n=1} f(a_n) \setminus a_n \in \Supp(f) 
		\Imply \NewLine \Imply 
		\exists \Sep(f)
	}
	\SayIn{b}{ 
		\Lambda n \in \Nat \. 
		\bigvee^\infty_{k=-\infty}  f^k\Big(f(a_n) \setminus a_n \Big)
		}
		{\Nat \to A}
	\Say{[1]}{\Elim b}{\forall n \in \Nat \. f(b_n) = b_n}
	\Say{c}{\Lambda n \in \Nat \. b_n \setminus \bigvee^n_{k=1} b_n}
	{
		\Nat \to A
	}
	\Say{[2]}{\Elim c [1]}{\forall n \in \Nat \. f(c_n) = c_n}
	\Say{[3]}{\Elim c \Elim \setminus }{\PD(A,\im c)}
	\SayIn{s}{ \bigvee_{n=1}^\infty c_n \Big(a_n \setminus f^{-1}(a_n) \Big)  }
	{
		A
	}
	\Say{[4]}{\Elim s \Elim \sC(A,A,f)
		[2]
		\THM{MultiplicationIsOrderContinuous}(A)
		\NewLine 
		\Elim \PD(A,\im c)[3]
		\Elim \setminus}
	{
		sf(s) =
		\left( \bigvee_{n=1}^\infty c_n \Big(a_n \setminus f^{-1}(a_n)\Big)\right)
		\left( \bigvee_{n=1}^\infty f(c_n) \Big(f(a_n) \setminus a_n\Big)\right) =
		\NewLine =
		\left( \bigvee_{n=1}^\infty c_n \Big(a_n \setminus f^{-1}(a_n)\Big)\right)
		\left( \bigvee_{n=1}^\infty c_n \Big(f(a_n) \setminus a_n\Big)\right) =
		\bigvee_{n,m=1}^\infty  c_n c_m  
		\Big(a_n \setminus f^{-1}(a_n)\Big) 
		\Big(f(a_m) \setminus a_m\Big) =  \NewLine =
		\bigvee_{n=1}^\infty c_n 
		\Big(a_n \setminus f^{-1}(a_n)\Big) 
		\Big(f(a_n) \setminus a_n\Big) =
		0
	}
	\AssumeIn{x}{A}
	\Assume{[5]}{\forall n \in \Int \. f^n(s)x = 0}
	\Say{[6]}{
		\THM{OrderContinuousMult}[5]
		\Elim s \Elim \sC(A,A,f)
		\Intro b 
		\Elim \TYPE{BooleanOrder}(A,b,c)
		\Elim c
		\Elim b
	}
	{
		 \NewLine :		  
		  0   = \left(\bigvee_{n=-\infty}^\infty f^n(s) \right) x  =
		  \left(\bigvee_{n=-\infty}^\infty \bigvee_{m=1}^\infty f^n(c_m )
		  \Big( f^n(a_m) \setminus f^{n-1}(a_m)\Big) \right) x = \NewLine =
		  \left(\bigvee_{m=1}^\infty  c_m \bigvee_{n=-\infty}^\infty
		  \Big( f^n(a_m) \setminus f^{n-1}(a_m)\Big) \right) x =
		  \left(\bigvee^\infty_{m=1} c_m b_m\right) x = 
		  \left(\bigvee^\infty_{m=1} c_m \right) x = 
		  \left(\bigvee^\infty_{m=1} b_m \right) x \ge \NewLine \ge
		  \left(\bigvee^\infty_{m=1} f(a_m) \setminus a_m \right) x
	}
	\Conclude{[x.*]}{\Elim \Supp [1]}{f(x) = x}
	\DeriveConclude{[*]}{\Intro \Sep(f)[3]}{s \in \Sep(f) }
	\EndProof
}
\Page{
	\Theorem{SixfoldLemma}
	{
		\forall A : \SA \.
		\forall f \in \Aut_\BOOL(A) \.
		\exists s : \Sep(f)
		\iff \NewLine \iff
		\exists a,a',b,b',c,d \in A :
		f(a) = b
		\And
		f(a') = b' 
		\And
		f(b') = c
		\And
		f(b \vee c) = a \vee a' \And
		\forall u \le d \. f(u) = u	
	}
	\AssumeIn{s}{\Sep(f)}
	\Say{\Big(u,[1]\Big)}{\THM{FroliksLemma1}(A,f,s)}
	{
		\sum u  \in A \. 
		uf(u) = 0 \And  		
		u \vee f(u) \vee f^2(u) \in \Supp(f)
	}
	\SayIn{c}{f^2(u) \setminus (f(u) \vee u )}{A}
	\SayIn{b'}{f^{-1}(c)}{A}
	\SayIn{a'}{f^{-1}(b')}{A}
	\SayIn{b}{f(u) \setminus b'}{A}
	\SayIn{a}{u \setminus a'}{A}
	\SayIn{d}{\Big(u \vee f(u) \vee f^2(u)\Big)^\c}{A}
	\Say{[4]}{\Elim d \Elim c  }
	{
		\PoU\Big(A, \{ c,a,a',b,b', d \} \Big)
	}
	\Conclude{[s.*]}{\ldots}
	{
		f(b \vee c) = 
		f\Big( (u \vee b' \vee d)^\c \Big) =
		\Big( f(u) \vee f(b') \vee f(d) \Big)^\c = 
		\Big( f(u) \vee c \vee d \Big)^\c  = 
		u = 
		a \vee a'
	}
	\NoProof
	\\
	\Theorem{SupportBySeparator}
	{
		\forall A \in \SA \.
		\forall f \in \Aut_\BOOL(f) \.
		\forall s \in \Sep(f) \.
		\exists s' \in A : s' = \supp f
	}
	\Say{\Big(u,[1]\Big)}{\THM{FroliksLemma1}(A,f,s)}
	{
		\sum u  \in A \. 
		uf(u) = 0 \And  		
		u \vee f(u) \vee f^2(u) \in \Supp(f)
	}
	\SayIn{s'}{u \vee f(u) \vee f^2(u)}{\Supp(f)}
	\AssumeIn{a}{\Supp(f)}
	\Assume{[2]}{a < s'}
	\Say{[3]}{f[1.1]}{f(u)f^2(u) = 0}
	\Say{[4]}{f[2]}{f^2(u)f^3(u) = 0 }
	\Say{[5]}{\Elim \Supp(f,a)\Elim s' [1.1][3][4]}
	{
		s' \setminus a = 0
	}
	\Conclude{[a.*]}{\THM{TrichtomyPrinciple}(A)\THM{ReminderRule}(A)[5]}{\bot}
	\DeriveConclude{[*]}{
		\Elim \bot 
		\Elim < 
		\THM{SupportIsClosedUnderIntersections}(A,f) \Intro\supp
	}{
		s' = \min \Supp f =  \supp f 
	}
	\EndProof
}
\Page{
	\Theorem{FroliksTHM}
	{
		\forall A : \TAlgebra \.
		\forall f \in \Aut_\BOOL(A) \.
		\exists \Sep(f)
	}
	\Say{P}{\Big\{ a \in A : f(a)a  = 0\Big\}}{?A}
	\Say{[1]}{\Elim P \THM{ZeroImage}(A,f)}{0 \in P}
	\Say{[2]}{\Intro \exists [1]}{\exists P}
	\Assume{C}{\TYPE{Chain}(P)}
	\SayIn{c}{\bigvee C}{A}
	\AssumeIn{a,b}{C}
	\Say{\Big(d, [3] \Big)}{\Elim \TYPE{Chain}(C,a,b)}
	{
		\sum d \in C \.  a \le d \And b \le d
	}
	\Conclude{\Big[(a,b).*\Big]}
	{
		\THM{BooleanMorphismIsMonotonic}(A,A,f)[3]
		\THM{MonotonicMeet}(A)[3]		
		\Elim P(d)	
	}
	{
		af(b) \le df(d) = 0 
	}
	\Derive{[3]}{\Intro \forall}{\forall a,b \in C \. af(b) = 0}
	\Say{[4]}{\Elim C \Elim \oC(A,A,f) [3] \Elim \vee}
	{
		f(c)c = 
		f\left(\bigvee C \right) \bigvee C  = 
		\bigvee_{a \in C} \bigvee_{b \in C} a f(b) = 
		\bigvee_{a \in C} \bigvee_{b \in C} 0 =
		0
	}
	\Conclude{[C.*]}{\Elim P [4]}{c \in P}
	\Derive{\Big(s,[3]\Big)}{\THM{ZornsLemma}[2]}
	{
		\sum s \in P \. s = \max P
	}
	\AssumeIn{a}{A}
	\Assume{[4]}{\forall n \in \Int \. f^n(s)a = 0 }
	\Assume{[5]}{a \neq f(a)}
	\Say{[6]}{\Intro \setminus [5]}{a \setminus f(a) \neq 0 \Big| f(a) \setminus a \neq 0}
	\Assume{[7]}{ a \setminus f(a) \neq 0 }
	\SayIn{b}{a \setminus f(a)}{A}
	\SayIn{c}{s \vee b}{A}
	\Say{[8]}{\Elim c [4]}{c > s}
	\Say{[9]}{ 
		\Elim c 
		\Elim \Aut_\BOOL(A,f) \Elim \TYPE{AssociativeLattice}(A) \Elim   
		\Elim P(s) \Elim b [4] 	
	}
	{
			\NewLine :			
			f(c)c = 
			f(s \vee b)s \vee b  =
			f(s)s \vee f(b)s \vee f(s)b \vee f(b)b = 0
	}
	\Say{[10]}{\Elim P [9]}{c \in P}
	\Conclude{[7.*]}{[10][8][3]}{\bot}
	\Derive{[7]}{\Intro \Imply}{  a \setminus f(a) \neq 0 \Imply \bot    }
	\Assume{[8]}{ f(a) \setminus a \neq 0 }
	\SayIn{b}{f(a) \setminus a}{A}
	\SayIn{c}{s \vee b}{A}
	\Say{[9]}{\Elim c [4]}{c > s}
	\Say{[10]}{ 
		\Elim c 
		\Elim \Aut_\BOOL(A,f) \Elim \TYPE{AssociativeLattice}(A) \Elim   
		\Elim P(s) \Elim b [4] 	
	}
	{
			\NewLine :			
			f(c)c = 
			f(s \vee b)s \vee b  =
			f(s)s \vee f(b)s \vee f(s)b \vee f(b)b = 0
	}
	\Say{[11]}{\Elim P [10]}{c \in P}
	\Conclude{[8.*]}{[11][9][3]}{\bot}
	\Derive{[8]}{\Intro \Imply}{  f(a) \setminus a \neq 0 \Imply \bot }
	\Conclude{[a.*]}{\Elim | [6][7][8]}{\bot}
	\DeriveConclude{[*]}{\Elim \bot \Intro \forall \Intro \Sep \Elim P(s)}
	{s \in \Sep(f)}
	\EndProof	
}
\newpage
\subsubsection{Towards  Factorization by Exchanging Involutions}
\Page{
	\Theorem{ExchangingInvolutionsInTheCompleteAlgebra}
	{
		\NewLine ::		
		\forall A : \TAlgebra \.
		\forall f : \TYPE{Involution}\Big( \Aut_\BOOL(A) \Big) \. 
		\EI(A,f) 
	}
	\Say{[1]}{\THM{FrolicsTHM}(A,f)}{\exists \Sep(f)}
	\Conclude{[*]}{\THM{ExchangingInvolutionBySeparator}(A,f)[1]}
	{
		\EI(A,f)	
	}
	\EndProof
	\\
	\Theorem{ExchangingAutomorphisimInTheCompleteAlgebra}
	{
		\NewLine ::		
		\forall A : \TAlgebra \.
		\forall f : \TYPE{Periodic}(A) \.
		\pi(f) \ge 2 \Imply
		\exists a \in A : \NewLine :
		\PoU\bigg( A, \Big\{ f^k(a) \Big| k \in \big[0,\ldots, \pi(f) - 1 \big]\Big\}\bigg)
		\And f =  \overleftarrow{a_{1 f} f(a)_{f} \ldots_{f}f^{\pi(f)-1}(a)}
	}
	\Say{[1]}{
		\Lambda k \in \big[0,\ldots, \pi(f) - 1 \big]
		\THM{FrolicsTHM}(A,f^k)
	}
	{
		\forall k \in  \big[0,\ldots, \pi(f) - 1 \big] \.
		\exists \Sep(f^k)
	}
	\SayIn{a}{\THM{TransversalConstructionLemma}(A,f)[1]\Elim \TYPE{Periodic}(A,f)}
	{ \Tr(f) }
	\Say{[2]}{
		\Lambda k \in \big[0,\ldots, \pi(f) - 1 \big]
		\Elim \TYPE{Periodic}(A,f,k)
	}
	{
		\forall k \in  \big[0,\ldots, \pi(f) - 1 \big] \.
		\supp f^k = e
	}
	\Say{[3]}{\Elim \supp [2] \Elim_2 \Tr(f,a)}
	{ 
		\forall k,l \in 	\big[0,\ldots, \pi(f) - 1 \big] \.
		k \neq l \Imply f^k(a) f^l(a) = 0
	}
	\Say{[*.1]}{\Elim_1 \Tr(f,a) [3]\Intro \PoU}
	{
		\PoU\bigg( A, \Big\{ f^k(a) \Big| k \in \big[0,\ldots, \pi(f) - 1 \big]\Big\}\bigg)
	}
	\Conclude{[*]}{\Intro \FUNC{exchangingInvolution}[*.1]}
	{
		f =  \overleftarrow{a_{1 f} f(a)_{f} \ldots_{f}f^{\pi(f)-1}(a)}
	}
	\EndProof
}\Page{
	\Theorem{TransversalAggregation}
	{
		\forall A : \SA \.
		\forall f \in \Aut_\BOOL(A) \.
		\forall a : \Nat \to A \. 
		\NewLine \.
		\Big( \forall n \in \Nat \. f(a_n) = a_n \And \exists \Tr(f_{|\genIdeal{a_n}})\Big)
		\Imply		
		\exists \Tr(f_{|\genIdeal{b}}) 
		\quad  \where \quad
		b = \bigvee^\infty_{n=1} a_n
	}
	\Say{t}{\Elim \exists [0]}{\prod^\infty_{n=1} \Tr(\genIdeal{a_n})}
	\Say{[1]}{\Elim b \Elim \sC(A,A,f) [0.1] \Intro  b}
	{
		f(b) =
		f\left( \bigvee^\infty_{n=1} a_n \right) =
		\bigvee_{n=1}^\infty f(a_n) =
		\bigvee_{n=1}^\infty a_n =
		b
	}
	\SayIn{u}{\bigvee^\infty_{n=1} \left( t_n \setminus \bigvee^{n-1}_{k=1} a_k \right)}{A}
	\Say{[2]}{\Elim u \THM{MonotonicSup}(A) \Intro u}{u \le b}
	\Say{[3]}{
		 \Elim u
		 \Lambda n \in \Int \. \Elim \oC(A,A,f^n)  [0.1]
		 \Lambda m \in \Nat \. \Elim_1 \Tr(f_{|\genIdeal{a_m}},t_m)
		 \Intro b
	}{
		   \NewLine :
		   \bigvee^\infty_{n= - \infty} f^n(u) =
		   \bigvee^\infty_{n= - \infty} 
		   f^n\left( \bigvee^\infty_{m=1} \left(t_n \setminus \bigvee^{m-1}_{k=1} a_k 
		   \right) \right) =
		   \bigvee^\infty_{n= - \infty} \bigvee^\infty_{m=0} 
		   \left( f^n(t_m) \setminus  \bigvee^{m-1}_{k=1} a_k \right)  =
		   \bigvee^\infty_{m=0} a_m = b
		}
	\Say{[4]}{
		\Lambda n \in \Int \.
		\Lambda c \le uf^n(u) \.
		\Elim \TYPE{BooleanOrder}\Big(A, c, f^n(u)u\Big)
		\Elim 	u
		\Elim \oC(A,A,f^n)
		\NewLine
		\THM{OrderContinuousMult}(A)
		\Elim \setminus 
		\Elim \oC(A,A,f^n)
		\Elim \Tr(f)[0.2]
		\Elim \setminus
		\NewLine
		\Elim \oC(A,A,f^n)
		\THM{OrderContinuousMult}(A)
		\Intro u 
		\Elim \TYPE{BooleanOrder}\Big(A, c, f^n(u)u\Big)
	}
	{
		\NewLine :
		\forall n \in \Int \. 
		\forall c \le uf^n(u) \.		
		f^n(c) = 
		f^n\Big(cf^n(u)u\Big) = 
		f^n\left(cf^n\left( \bigvee^\infty_{m=1}  
		\left( t_m \setminus \bigvee^{m-1}_{l=1} a_l \right)\right) 
		\bigvee^\infty_{k=1}  \left( t_k \setminus \bigvee^{k-1}_{h=1} a_h \right)\right) =
		\NewLine =		
		f^n\bigvee^\infty_{m=1} \bigvee^\infty_{k=1} cf^n\left( t_m \setminus 
		\bigvee^{m-1}_{l=1} a_l \right)
		\left( t_k \setminus \bigvee^{k-1}_{h=1} a_h \right) =
		f^n\bigvee^\infty_{m=1}  cf^n\left( t_m \setminus 
		\bigvee^{m-1}_{l=1} a_l \right)
		\left( t_m \setminus \bigvee^{m-1}_{l=1} a_l \right) =
		\NewLine
		=
		\bigvee^\infty_{m=1} f^n\left( cf^n\left( t_m \setminus 
		\bigvee^{m-1}_{l=1} a_l \right)
		\left( t_m \setminus \bigvee^{m-1}_{l=1} a_l \right) \right) =
		\bigvee^\infty_{m=1}  cf^n\left( t_m \setminus 
		\bigvee^{m-1}_{l=1} a_l \right)
		\left( t_m \setminus \bigvee^{m-1}_{l=1} a_l \right) = \NewLine =
		\bigvee^\infty_{m=1} \bigvee^\infty_{k=1} cf^n\left( t_m \setminus 
		\bigvee^{m-1}_{l=1} a_l \right)\
		\left( t_k \setminus \bigvee^{k-1}_{h=1} a_h \right) =
		cf^n\left( \bigvee^\infty_{m=1}  
		\left( t_m \setminus \bigvee^{m-1}_{l=1} a_l \right)\right) 
		\bigvee^\infty_{k=1}  \left( t_k \setminus \bigvee^{k-1}_{h=1} a_h \right) =
		cf^n(u)u =
		c
	}
	\Conclude{[*]}{\Intro \Tr(f)[3][4]}
	{
		u \in \Tr(f_{\genIdeal{b}})
	}
	\EndProof
	\\
	\Theorem{InverseTransversality}
	{
		\forall A \in \BOOL \.
		\forall f \in \Aut_\BOOL(A) \.
		\forall t \in \Tr(f) \.
		t \in \Tr(f^{-1} )
	}
	\NoProof
	\\
	\DeclareFunc{downstreamElement}
	{
		\prod A : \SA \.
		\Big(A \times \Aut_\BOOL(A)\Big) \to A
	}
	\DefineNamedFunc{downstreamElement}{a,f}{a^*_f}
	{
		\bigvee^\infty_{n=-\infty} 
		\left( f^n(a) \setminus \bigvee^\infty_{k=n+1} f^k(a) \right)
 	}
 	\\
 	\DeclareFunc{upstreamElement}
	{
		\prod A : \SA \.
		\Big(A \times \Aut_\BOOL(A)\Big) \to A
	}
	\DefineNamedFunc{upstreamElement}{a,f}{a^f_*}
	{
		\bigvee^\infty_{n=-\infty} 
		\left( f^n(a) \setminus \bigvee^{n-1}_{k=-\infty} f^k(a) \right)
 	}
}
\Page{
	\Theorem{DownstreamElementIsFixed}
	{
		\forall A : \SA \.
		\forall f \in \Aut_\BOOL(A) \.
		\forall a \in A \.
		a^*_f \in \Fix(f)
	}
	\Conclude{[*]}{
		\Elim a^*_f 
		\Elim \oC(A,A,f)
		\Intro a^*_f	
	}
	{
		\NewLine :		
		f(a^*_f) = 
		f\bigvee^\infty_{n=-\infty} 
		\left( f^n(a) \setminus \bigvee^\infty_{k=n+1} f^k(a) \right) =
		\bigvee^\infty_{n=-\infty} 
		\left( f^n(a) \setminus \bigvee^\infty_{k=n+1} f^k(a) \right) =
		a^*_f
	}
	\EndProof
	\\
	\Theorem{DownstreamElementAdmitsTransversals}
	{
		\forall A : \SA \.
		\forall f \in \Aut_\BOOL(A) \.
		\forall a \in A \.
		\exists \Tr(f_{|\genIdeal{a^*_f}})
	}
	\SayIn{t}{a \setminus \bigvee^{\infty}_{n=1} f^n(a)}{A}
	\Say{[1]}{\Elim t \Elim a^*_f \Elim \vee}
	{
		t \le a^*_f
	}
	\Say{[2]}{ \Elim a^*_f \Elim \oC(A,A,f) \Elim \Aut_\BOOL(A,f) \Intro t} 
	{
		\NewLine :		
		a^*_f =  \bigvee^\infty_{n=-\infty} 
		\left( f^n(a) \setminus \bigvee^\infty_{k=n+1} f^k(a) \right) =
		\bigvee^\infty_{n=-\infty} f^n
		\left( a \setminus \bigvee^{\infty}_{n=1} f^n(a)\right) = 
		\bigvee^\infty_{n=-\infty} f^n(t)
	}
	\Say{[3]}{\Elim t \Elim \setminus}
	{
		\forall n \in \Int \. n \neq 0 \Imply tf^n(t) = 0
	}
	\Conclude{[*]}{\Intro \Tr(f_{|\genIdeal{a^*_f}})[2][3]}
	{
		t \in f_{|\genIdeal{a^*_f}}
	}
	\EndProof
	\\
	\Theorem{UpstreamElementIsFixed}
	{
		\forall A : \SA \.
		\forall f \in \Aut_\BOOL(A) \.
		\forall a \in A \.
		a^f_* \in \Fix(f)
	}
	\NoProof
	\\
	\Theorem{UpstreamElementAdmitsTransversals}
	{
		\forall A : \SA \.
		\forall f \in \Aut_\BOOL(A) \.
		\forall a \in A \.
		\exists \Tr(f_{|\genIdeal{a^f_*}})
	}
	\NoProof
	\\
	\Theorem{TransversalFactorizationTHM}
	{
		\forall A : \SA \.
		\forall f \in \Aut_\BOOL(A) \.
		\forall t \in \Tr(A) \. \NewLine \.
		\exists \alpha,\beta : \EI(A) \.
		f = \alpha \beta
		\And
		\alpha,\beta \in \genCFS{f}
	}
	\NoProof
	\\
	\DeclareType{SubgroupWithSeparators}
	{
		\prod A \in \BOOL \.  ??_\GRP \Aut_\BOOL(A)
	}
	\DefineType{G}{\SwS}{ \forall g \in G \. \exists \Sep(g)}
	\\
	\Theorem{SwSHasSupports}
	{
		\forall A : \SA \.
		\forall G : \SwS(A) \.
		\forall g \in G \.
		\exists a \in A : a = \supp g	
	}
	\NoProof
}
\Page{
	\Theorem{ExistanceOfTransversalInSwSCondition}
	{
		\NewLine ::		
		\forall A : \SA \.
		\forall G : \SwS  \.
		\forall g \in G \.
		\forall n  \in \Nat \.
		\Big(g^n = \id 
		\Imply
		\exists \Tr(g) \Big)
	}
	\Say{[1]}{\Lambda k \in [1,\ldots,n-1] \. \Elim \SwS(G,g^k)}
	{
		\forall k \in [1,\ldots,n-1] \. \exists \Sep(g^k)
	}
	\Conclude{[*]}{\THM{TransversalConstructionLemma}[0][1]}
	{
		\exists \Tr(g)
	}
	\EndProof
	\\
	\Theorem{SWSLocalization}
	{
		\forall A : \SA \.
		\forall G : \SwS  \.
		\forall g \in G \.
		\exists \Tr\Big( g_{|\genIdeal{a}}  \Big)
		\NewLine \quad \where \quad
		a = \left( \bigwedge_{n=1}^\infty \supp g^n \right)^\c
	}
	\NoProof
	\\
	\Theorem{CountablyFullSwSLemma}
	{
		\NewLine ::		
		\forall : \SA \.
		\forall G : \SwS \And \CFS(A) \. \NewLine \. 
		\forall a \in  \Fix(G) \.
		\SwS \And \CFS\Big(\genIdeal{a},G_{|\genIdeal{a}}\Big)
	}
	\NoProof
}
\newpage
\subsubsection{The Great Exchange}
\Page{
	\DeclareType{TriplingSequence}
	{
		\prod_{A \in \BOOL}\Aut_\BOOL(A)\to ?(\Int_+ \downarrow A)
	}
	\DefineType{a}{TriplingSequence}
	{
		a_0 = e
		\And
		\Bigg(\forall n \in \Int \. \TYPE{DoublyRecurrentOn}(A,a_n,g) 
			\And \NewLine \And 
			\bigvee^\infty_{m=1} g^m( a_n ) = 		
			\bigvee^\infty_{m=1} g^{-m}( a_n )	= e
			\And 
			\PD\Big(A,\big\{a_{n+1},g_{a_n}(a_{n+1}),g_{a_n}^2(a_{n+1})\big\}\Big)
		\Bigg)
	}
	\\	
	\Theorem{TriplingSequenceConstruction}
	{
		\NewLine :
		\forall A : \SA \.
		\forall  G : \SwS \And \CFS(A) \.
		\forall g \in G \.	\NewLine \.
		\TYPE{Aperiodic}(A,g) 
		\Imply
		\exists \TYPE{TriplingSequence}(A,g)
 	}
 	\NoProof
 	\\
 	\Theorem{TheGreatExchangeLemma}
 	{
 		\forall A : \SA \.
 		\forall f : \TYPE{Aperiodic}(A) \.
 		\forall a : \TYPE{TriplingSequence}(A,f) \. \NewLine \.
 		\exists \phi \in \genCFS{f} :
 		\EI(A,\phi) \And \bigwedge^\infty_{n=1} \supp( \phi f )^n =  0
 	}
 	\NoProof
 	\\
 	\Theorem{
		 	TransversalCompletionLemma
 	}
 	{
		\NewLine  		
 		\forall A : \SA \. 
 		\forall G : \SwS \And \CFS(A) \.
 		\forall g \in G \. \NewLine \.
 		\exists \phi \in G :
 		\EI(A,\phi) \And \exists \Tr(\phi g) 
 	}
 	\NoProof
 	\\
 	\Theorem{
 		MainFactorizationTHM
 	}
 	{
 		\NewLine  		
 		\forall A : \SA \. 
 		\forall G : \SwS \And \CFS(A) \.
		\forall g \in G \. \NewLine \.
		\exists \alpha,\beta,\gamma \in G :
		\EI(A,\alpha \And \beta \And \gamma)
		\And g = \alpha \beta \gamma
 	}
 	\NoProof
 	\\
 	\Theorem{
 		CompleteFactorizationTHM
 	}
 	{
 		\NewLine  		
 		\forall A : \TAlgebra \. 
 		\forall G :  \FS(A) \.
		\forall g \in G \. \NewLine \.
		\exists \alpha,\beta,\gamma \in G :
		\EI(A,\alpha \And \beta \And \gamma)
		\And g = \alpha \beta \gamma \And
		\supp g \in \Supp \alpha \cap \Supp \beta \cap \Supp \gamma
 	}
 	\NoProof
}
\newpage
\subsubsection{Subgroups with many involutions and simplicity of it all}
\Page{
	\DeclareType{\SwmI}{\prod_{A \in \BOOL} ??\GRP \Aut_\BOOL(A)}
	\DefineType{G}{\SwmI}{
		\forall a \in A \. a \neq 0
		\Imply
		\exists g \in G :
		\TYPE{Involution}\Big(\Aut_\BOOL(A), g \Big) 
		\And \NewLine		
		\And a \in \Supp(g)
	}
	\\
	\Theorem{AtomlessHomogeneousHasManyInvolutions}{
		\NewLine ::		
		\forall A : \Aless \And \Homog \.
		\SwmI\Big(A,\Aut_\BOOL(A)	\Big)
	}
	\NoProof
	\\
	\Theorem{SubgroupWithManyExchangingInvolutions}
	{
		:: \NewLine 		
		\forall A : \TAlgebra \.
		\forall G : \FS\And\SwmI(A) \.
		\NewLine \.
		\forall a \in A \. a \neq 0 
		\Imply 
		\exists g \in G :
		\EI(A, g ) 
		\And \NewLine		
		\And a \in \Supp(g)
	}
	\NoProof
	\\
	\Theorem{
		NormalSubgroupsAreInvariantIdeals	
	}
	{
		\NewLine ::		
		\forall A : \TAlgebra \.
		\forall G : \FS\And\SwmI(A) \. \forall H \subset G \. \NewLine \.
		H \Nrml G
		\iff
		\exists I : \Ideal(A) \And \TYPE{Invariant}(G,A) :
		H = \{ g \in G : \supp g \in I  \}
	}
	\NoProof
	\\
	\Theorem{SimplicityTHM}
	{
		\forall A : \TAlgebra \And \Homog \.
		\TYPE{Simple}\Big( \Aut_\BOOL(A) \Big)
	}
	\NoProof
}
\newpage
\subsection{Simple Functions[!]}
This chapter represents knowledge from Fremlin's measure theory  361,362; prereq: OTVS
\section{Applications towards Analysis[!]}
Possible sources for this chapter is the book by Vladimirov; prereq: Spectral Analysis
\section{Applications towards Logic and Set Theory [!]}
Possible sources for this chapter is handbook of Boolean Algebras by Monk et al. and lecture notes by Podzorov; prereq: Forcing and M-Logic
\newpage 
\section*{Sources:}
\begin{enumerate}
\item MEASURE THEORY by D.H.Fremlin, chapters 31 and 38
\end{enumerate}
\end{document}
