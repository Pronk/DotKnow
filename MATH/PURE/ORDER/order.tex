  \documentclass[12pt]{article}
\usepackage{mathtools}
\usepackage{amsmath}
\usepackage{amsfonts}
\usepackage{amssymb}
\usepackage{ wasysym }
\usepackage{accents}
\usepackage[dvipsnames]{xcolor}
\usepackage[top=20mm, bottom=20mm, left=30mm, right=10mm]{geometry}
%Markup
\newcommand{\TYPE}[1]{\textcolor{NavyBlue}{\mathtt{#1}}}
\newcommand{\FUNC}[1]{\textcolor{Cerulean}{\mathtt{#1}}}
\newcommand{\LOGIC}[1]{\textcolor{Blue}{\mathtt{#1}}}
\newcommand{\THM}[1]{\textcolor{Maroon}{\mathtt{#1}}}
%META
\renewcommand{\.}{\; . \;}
\newcommand{\de}{: \kern 0.1pc =}
\newcommand{\extract}{\LOGIC{Extract}}
\newcommand{\where}{\LOGIC{where}}
\newcommand{\If}{\LOGIC{if} \;}
\newcommand{\Then}{ \; \LOGIC{then} \;}
\newcommand{\Else}{\; \LOGIC{else} \;}
\newcommand{\IsNot}{\; ! \;}
\newcommand{\Is}{ \; : \;}
\newcommand{\DefAs}{\; :: \;}
\newcommand{\Act}[1]{\left( #1 \right)}
\newcommand{\Example}{\LOGIC{Example} \; }
%%STD
\newcommand{\Int}{\mathbb{Z} }
\newcommand{\NNInt}{\mathbb{Z}_{+} }
\newcommand{\Reals}{\mathbb{R} }
\newcommand{\Rats}{\mathbb{Q} }
\newcommand{\Nat}{\mathbb{N} }
\newcommand{\EReals}{\stackrel{\mathclap{\infty}}{\mathbb{R}}}
\newcommand{\ERealsn}[1]{\stackrel{\mathclap{\infty}}{\mathbb{R}}^{#1}}
\DeclareMathOperator*{\centr}{center}
\DeclareMathOperator*{\argmin}{arg\,min}
\DeclareMathOperator*{\argmax}{arg\,max}
\DeclareMathOperator*{\id}{id}
\DeclareMathOperator*{\im}{Im}
\newcommand{\EqClass}[1]{\TYPE{EqClass}\left( #1 \right)}
\newcommand{\Cate}{\TYPE{Category}}
\newcommand{\Func}[2]{\TYPE{Functor}\left( #1, #2 \right)}
\mathchardef\hyph="2D
\newcommand{\Surj}[2]{\TYPE{Surjective}\left( #1, #2 \right)}
\newcommand{\ToInj}{\hookrightarrow}
\newcommand{\ToBij}{\leftrightarrow}
\newcommand{\Set}{\TYPE{Set}}
\newcommand{\du}{\; \triangle \;}
\renewcommand{\c}{\complement}
\renewcommand{\And}{\; \& \;}
%%ProofWritting
\newcommand{\A}{\LOGIC{Assume} \;} 
\newcommand{\As}{\; \LOGIC{as } \;} 
\newcommand{\E}{ \; \LOGIC{Extract} } 
\newcommand{\QED}{\; \square}
\newcommand{\ByDef}{\eth} 
\newcommand{\ByConstr}{\jmath}  
\newcommand{\Alt}{\LOGIC{Alternative} \;}
\newcommand{\CL}{\LOGIC{Close} \;}
\newcommand{\More}{\LOGIC{Another} \;}
\newcommand{\Proof}{\LOGIC{Proof} \; }
%MetricGeometry
\newcommand{\Ball}[3]{ \mathbb{B}^{#1}\left(#2,#3\right) }
\newcommand{\ClBall}[3]{ \overline{ \mathbb{B}}^{#1}\left(#2,#3\right) }
\newcommand{\ToP}{\overset{p}{\to}}
\newcommand{\ToU}{\rightrightarrows}
%LinearAlgebra
%TYPES
\newcommand{\VS}[1]{\TYPE{VectorSpace}\left( #1 \right)}
\newcommand{\Lin}[1]{\mathcal{L}\left( #1 \right)}
\newcommand{\vs}[1]{\mathsf{VS}\left( #1 \right)}
\DeclareMathOperator*{\rank}{rank}
%FUNK
\DeclareMathOperator{\rk}{rank}
\author{Uncultured Tramp} 
\title{Order Theory}
%Simbpls
\renewcommand{\L}{\mathcal{L}}
%Topology
%TYPES
\newcommand{\TS}{\TYPE{TopologicalSpace}}
%MeasureTheory
%TYPES
\newcommand{\SA}[1]{\TYPE{\sigma \hyph  Algebra}\left( #1 \right) }
\newcommand{\SF}[1]{\TYPE{\sigma \hyph  Finite}\left( #1 \right) }
\newcommand{\CA}[1]{\TYPE{CountablyAdditive}\left( #1 \right) }
\newcommand{\FA}[1]{\TYPE{Charge}\left( #1 \right) }
\newcommand{\LS}{\TYPE{Lebesgue \hyph Stieltjes}}
\newcommand{\DF}{\TYPE{DistributionFunction}}
\renewcommand{\AE}[1]{\quad \mathrm{a\. e\.} \left[#1\right] \,}
\newcommand{\SI}[1]{\TYPE{\sigma \hyph  Ideal}\left( #1 \right) }
%Simbols
\newcommand{\F}{\mathcal{F}}
\renewcommand{\O}{\Omega}
\newcommand{\B}{\mathcal{B}}
\renewcommand{\l}{\lambda}
\renewcommand{\P}{\mathbb{P}}
%Probability
\newcommand{\RA}{\TYPE{RamdomVariable}}
\renewcommand*{\E}{\mathbb{E} \,}
\begin{document}
\maketitle
\begin{center}
\end{center}
\tableofcontents
\newpage
\section{Orders}
$$\TYPE{PartialOrder} :: \prod A : \Set \. ?\TYPE{Relation}(A)$$
$$ R : \TYPE{PartialOrder} \de  \TYPE{Reflexive} \& \TYPE{Antisymmetric} \& \TYPE{Transitive}  $$
\\ 
$$\TYPE{Poset} :: \sum A : \Set \. \TYPE{PartialOrder}   $$
\\ 
$$\FUNC{implicit} :: \TYPE{Poset} \to \Set   $$
$$ \FUNC{implicit} (A, \le) =  (A, \le) \de A $$
\\
$$
\FUNC{order} :: \prod (A, \le) : \TYPE{Poset} \. \TYPE{PartialOrder}(A)
$$
$$
 \FUNC{order} = \le \de \le
$$
\\
$$
\TYPE{TotalOrder} :: ?\TYPE{PartialOrder}(A) $$
$$
 R : \TYPE{TotalOrder} \iff \forall a,b \in A \. (a,b) \in R | (b,a) \in R
$$
\\
$$
\TYPE{Chain} :: \prod A : \TYPE{Poset} \. ??A
$$
$$
C : \TYPE{Chain} \iff \le_{A|C} : \TYPE{TotalOrder}(C)
$$
\\
$$
\TYPE{Chain} :: \prod A : \TYPE{Poset} \. ??A
$$
$$
C : \TYPE{Chain} \iff \le_{A|C} : \TYPE{TotalOrder}(C)
$$
\\
$$
\TYPE{UpperBound} :: \prod A : \TYPE{Poset} \. ?A \to ?A
$$
$$
a : \TYPE{UpperBound}(X) \iff \forall x \in X \. x \le_A a 
$$
\\
$$
\TYPE{LeastUpperBound} :: \prod A : \TYPE{Poset} \. \prod X \subset A \.  ?\TYPE{UpperBound}(X)
$$
$$
a : \TYPE{LeastUpperBound} \iff \forall b :   \TYPE{LeastUpperBound}(X) \. a \le x 
$$
\\
$$
\TYPE{Maximal} :: \prod P : \TYPE{Poset} \. ?P 
$$
$$
m : \TYPE{Maximal} \iff \forall a \in P : m \le a \. m = a
$$
\\
$$
\TYPE{Minimal} :: \prod P : \TYPE{Poset} \. ?P 
$$
$$
m : \TYPE{Minimal} \iff \forall a \in P : a \le m \. m = a
$$
\\
$$
\TYPE{LowerBound} :: \prod A : \TYPE{Poset} \. ?A \to ?A
$$
$$
a : \TYPE{LowerBound}(X) \iff \forall x \in X \. a \le_A x 
$$
\\
$$
\TYPE{GreatestLowerBound} :: \prod A : \TYPE{Poset} \. \prod X \subset A \.  ?\TYPE{LowerBound}(X)
$$
$$
a : \TYPE{LeastUpperBound} \iff \forall b :   \TYPE{LeastUpperBound}(X) \. x \le a 
$$
\\
$$
\TYPE{Top} :: \prod A : \TYPE{Poset} \. ?A\\
$$
$$
1 : \TYPE{Top} \iff \forall a \in A \. a \le 1 
$$
\\
$$
\TYPE{bottom} :: \prod A : \TYPE{Poset} \. ?A\\
$$
$$
0 : \TYPE{bottom} \iff \forall a \in A \. 0 \le 1 
$$
\newpage
\section{Lattices}
$$\TYPE{Lattice} :: ?\TYPE{Poset} $$
$$ L : \TYPE{Lattice} \iff \forall  a,b \in L \. \exists \TYPE{LeastUpperBound}\{a,b\} \&
 \exists \TYPE{GreaterUpperBound}\{a,b\}    $$
 \\ 
 $$\FUNC{join} :: \prod L : \TYPE{Lattice} \. L \to L \to L $$
 $$\FUNC{join}(a,b) = a\vee b \de \ByDef\TYPE{Lattice}(L)(a,b)_1 \LOGIC{Extract}$$
 \\
 $$\FUNC{meet} :: \prod L : \TYPE{Lattice} \. L \to L \to L $$
 $$\FUNC{meet}(a,b) = a\wedge b \de \ByDef\TYPE{Lattice}(L)(a,b)_2 \LOGIC{Extract}$$
 \\
 $$
 \TYPE{CompleteLattice} :: ?\TYPE{Lattice}
 $$
 $$
  L : \TYPE{CompleteLattice} \iff  \forall A \subset L \. \exists \TYPE{LeastUpperBound}\, A \; \& \;
 \exists \TYPE{GreaterUpperBound} \, A
 $$
 \\
 $$
 \TYPE{Sublattice} :: \prod L : \TYPE{Lattice} \. ??L
 $$
 $$
   X : \TYPE{Sublattice} \iff \forall a,b \in X \. a \vee b \in X \And a \wedge b \in X   
 $$
 \\
 \begin{multline*}
  \THM{CompleteSubsetComplete} :: \forall L : \TYPE{CompleteLattice} \. \forall S \subset   L  : 1_L \in S : \forall T  \subset S : T \neq \emptyset 
  \. \\ \bigwedge T \in S \. S : \TYPE{CompleteLattice}
 \end{multline*}
 Proof $\approx$ \\
 Assume hypothesis.
 By assumption $S$ is closed under joins.
 Now let $\mathrm{UB}(T)$ be set of all upper bounds of $T$ inside $S$. As $1 \in S$  $\mathrm{UB}(T) \neq \emptyset $. But this means that $\bigvee T = \bigwedge \mathrm{UB}(T) \in S$ and this is exactly what we need. \\
 $\square$ 
\end{document}
