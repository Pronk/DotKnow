\documentclass[12pt]{scrartcl}
\usepackage{mathtools}
\usepackage{amsmath}
\usepackage{amsfonts}
\usepackage{hyperref}
\usepackage{amssymb}
\usepackage{ wasysym }
\usepackage{accents}
\usepackage{extpfeil}
\usepackage{graphicx}
\usepackage{scalerel}
\usepackage{esvect}
\usepackage{upgreek}
\usepackage[dvipsnames]{xcolor}
\usepackage[a4paper,top=5mm, bottom=5mm, left=10mm, right=2mm]{geometry}
%Markup
\newcommand{\TYPE}[1]{\textcolor{NavyBlue}{\mathtt{#1}}}
\newcommand{\FUNC}[1]{\textcolor{Cerulean}{\mathtt{#1}}}
\newcommand{\LOGIC}[1]{\textcolor{Blue}{\mathtt{#1}}}
\newcommand{\THM}[1]{\textcolor{Maroon}{\mathtt{#1}}}
%META
\renewcommand{\.}{\; . \;}
\newcommand{\de}{: \kern 0.1pc =}
\newcommand{\extract}{\LOGIC{Extract}}
\newcommand{\where}{\LOGIC{where}}
\newcommand{\If}{\LOGIC{if} \;}
\newcommand{\Then}{ \; \LOGIC{then} \;}
\newcommand{\Else}{\; \LOGIC{else} \;}
\newcommand{\IsNot}{\; ! \;}
\newcommand{\Is}{ \; : \;}
\newcommand{\DefAs}{\; :: \;}
\newcommand{\Act}[1]{\left( #1 \right)}
\newcommand{\Example}{\LOGIC{Example} \; }
\newcommand{\Theorem}[2]{& \THM{#1} \, :: \, #2 \\ & \Proof = \\ } 
\newcommand{\DeclareType}[2]{& \TYPE{#1} \, :: \, #2 \\} 
\newcommand{\DefineType}[3]{& #1 : \TYPE{#2} \iff #3 \\} 
\newcommand{\DefineNamedType}[4]{& #1 : \TYPE{#2} \iff #3 \iff #4 \\} 
\newcommand{\DeclareFunc}[2]{& \FUNC{#1} \, :: \, #2 \\}  
\newcommand{\DefineFunc}[3]{&  \FUNC{#1}\Act{#2} \de #3 \\} 
\newcommand{\DefineNamedFunc}[4]{&  \FUNC{#1}\Act{#2} = #3 \de #4 \\} 
\newcommand{\NewLine}{\\ & \kern 1pc}
\newcommand{\Page}[1]{ \begin{align*} #1 \end{align*}   }
\newcommand{ \bd }{ \ByDef }
\newcommand{\NoProof}{ & \ldots \\ \EndProof}
%LOGIC
\renewcommand{\And}{\; \& \;}
\newcommand{\ForEach}[3]{\forall #1 : #2 \. #3 }
\newcommand{\Exist}[2]{\exists #1 : #2}
\newcommand{\Imply}{\Rightarrow} 
\newcommand{\Intro}{\LOGIC{I}}
\newcommand{\Elim}{\LOGIC{E}}
%TYPE THEORY
\newcommand{\Type}{\TYPE{Type}}
%%STD
\newcommand{\Int}{\mathbb{Z} }
\newcommand{\NNInt}{\mathbb{Z}_{+} }
\newcommand{\Reals}{\mathbb{R} }
\newcommand{\Complex}{\mathbb{C}}
\newcommand{\Rats}{\mathbb{Q} }
\newcommand{\Sphere}{\mathbb{S}}
\newcommand{\Ball}{\mathbb{B}}
\newcommand{\Nat}{\mathbb{N} }
\newcommand{\EReals}{\stackrel{\mathclap{\infty}}{\mathbb{R}}}
\newcommand{\ERealsn}[1]{\stackrel{\mathclap{\infty}}{\mathbb{R}}^{#1}}
\DeclareMathOperator*{\centr}{center}
\DeclareMathOperator*{\argmin}{arg\,min}
\DeclareMathOperator*{\id}{id}
\DeclareMathOperator*{\im}{Im}
\DeclareMathOperator*{\supp}{supp}
\newcommand{\EqClass}[1]{\TYPE{EqClass}\left( #1 \right)}
\newcommand{\End}{\mathrm{End}}
\newcommand{\Aut}{\mathrm{Aut}}
\mathchardef\hyph="2D
\newcommand{\ToInj}{\hookrightarrow}
\newcommand{\ToMono}{\xhookrightarrow}
\newcommand{\ToSurj}{\twoheadrightarrow}
\newcommand{\ToEpi}{\xtwoheadrightarrow}
\newcommand{\ToBij}{\leftrightarrow}
\newcommand{\ToIso}{\xleftrightarrow}
\newcommand{\Arrow}{\xrightarrow}
\newcommand{\Set}{\TYPE{Set}}
\newcommand{\du}{\; \triangle \;}
\renewcommand{\c}{\complement}
\renewcommand{\i}{\mathbf{i}}
\newcommand{\Eqmod}[3]{#1 = #2 \quad \mathrm{mod} \quad #3}
%%ProofWritting
\newcommand{\Say}[3]{& #1 \de #2 : #3, \\}
\newcommand{\SayIn}[3]{& #1 \de #2 \in #3, \\}
\newcommand{\Conclude}[3]{& #1 \de #2 : #3; \\}
\newcommand{\Derive}[3]{& \leadsto #1 \de #2 : #3, \\}
\newcommand{\DeriveConclude}[3]{& \leadsto #1 \de #2 : #3 ; \\} 
\newcommand{\Assume}[2]{& \LOGIC{Assume} \; #1 : #2, \\}
\newcommand{\AssumeIn}[2]{& \LOGIC{Assume} \; #1 \in #2, \\}
\newcommand{\As}{\; \LOGIC{as } \;} 
\newcommand{\QED}{\; \square}
\newcommand{\EndProof}{& \QED \\}
\newcommand{\Proof}{\LOGIC{Proof} \; }
\newcommand{\Explain}[1]{& \text{#1.} \\}
\newcommand{\Exclaim}[1]{& \text{#1!} \\}
%SetTheory
\newcommand{\NonEmpty}{\TYPE{NonEmpty}}
\newcommand{\Finite}{\TYPE{Finite}}
\newcommand{\Countable}{\TYPE{Countable}}
\newcommand{\Ideal}{\TYPE{Ideal}}
\newcommand{\Inj}{\TYPE{Injective}}
\newcommand{\Surj}{\TYPE{Surjective}}
\newcommand{\Bij}{\TYPE{Bijective}}
\newcommand{\SIdeal}{\TYPE{\sigma\hyph \Ideal}}
\newcommand{\SA}{\TYPE{\sigma \hyph Algebra}}
%CategoryTheory
%Types
\newcommand{\Cov}{\TYPE{Covariant}}
\newcommand{\Contra}{\TYPE{Contravariant}}
\newcommand{\NT}{\TYPE{NaturalTransform}}
\newcommand{\UMP}{\TYPE{UnversalMappingProperty}}
\newcommand{\CMP}{\TYPE{CouniversalMappingProperty}}
\newcommand{\paral}{\rightrightarrows}
%functions
\newcommand{\op}{\mathrm{op}}
\newcommand{\obj}{\mathrm{obj}}
\DeclareMathOperator*{\dom}{dom}
\DeclareMathOperator*{\codom}{codom}
\DeclareMathOperator*{\colim}{colim}
%variable
\newcommand{\C}{\mathcal{C}}
\newcommand{\A}{\mathcal{A}}
\newcommand{\B}{\mathcal{B}}
\newcommand{\D}{\mathcal{D}}
\newcommand{\I}{\mathcal{I}}
\newcommand{\J}{\mathcal{J}}
\newcommand{\R}{\mathcal{R}}
%Cats
\newcommand{\CAT}{\mathsf{CAT}}
\newcommand{\SET}{\mathsf{SET}}
\newcommand{\PARALLEL}{\bullet \paral \bullet}
\newcommand{\WEDGE}{\bullet \to \bullet \leftarrow \bullet}
\newcommand{\VEE}{\bullet \leftarrow \bullet \to \bullet}
%OrderTheory
%Types
\newcommand{\Poset}{\TYPE{Poset}}
\newcommand{\Toset}{\TYPE{Toset}}
\newcommand{\Pres}{\TYPE{PreorderedSet}}
\newcommand{\WF}{\TYPE{WellFounded}}
\newcommand{\WO}{\TYPE{WellOrdered}}
\newcommand{\II}{\TYPE{InitialInterval}}
\newcommand{\UB}{\TYPE{UpperBound}}
\newcommand{\LUB}{\TYPE{LowerUpperBound}}
\newcommand{\LB}{\TYPE{LowerBound}}
\newcommand{\ULB}{\TYPE{UpperLoweBound}}
%Cats
\newcommand{\POSET}{\mathsf{POSET}}
\newcommand{\ORD}{\mathsf{ORD}}
%Symbols
\renewcommand{\P}{\mathsf{P}}
%\newcommand{\F}{\mathsf{F}}
%\newcommand{\U}{\mathsf{U}}
%Algebra
%Groups
%Types
\newcommand{\Group}{\TYPE{Group}}
\newcommand{\Abel}{\TYPE{Abelean}}
\newcommand{\Sgrp}{\subset_{\mathsf{GRP}}}
\newcommand{\Nrml}{\vartriangleleft}
\newcommand{\FG}{\TYPE{FiniteGroup}}
\newcommand{\Stab}{\mathrm{Stab}}
\newcommand{\FGA}{\TYPE{FinitelyGeneratedAbelean}}
\newcommand{\DN}{\TYPE{DirectedNormality}}
\newcommand{\ActsOn}{\curvearrowright}
%Func
\DeclareMathOperator{\tor}{tor}
\DeclareMathOperator{\ord}{ord}
\DeclareMathOperator{\bool}{bool}
\DeclareMathOperator{\rank}{rank}
%Cats
\newcommand{\GRP}{\mathsf{GRP}}
\newcommand{\ABEL}{\mathsf{ABEL}}
%Topology
%General Topology
%Types
\newcommand{\TS}{\TYPE{TopologicalSpace}} 
\newcommand{\LF}{\TYPE{LocallyFinite}}
\newcommand{\LC}{\TYPE{LocallyCompact}}
\newcommand{\PN}{\TYPE{PerfectlyNormal}}
\newcommand{\Open}{\TYPE{Open}}
\newcommand{\Compact}{\TYPE{Compact}}
\newcommand{\Compacts}{\TYPE{CompactSubset}}
\newcommand{\SCompact}{\TYPE{\sigma\hyph Compact}}
\newcommand{\LCompact}{\TYPE{LocallyCompact}}
\newcommand{\Perfect}{\TYPE{Perfect}}
\newcommand{\Limit}{\TYPE{Limit}}
\newcommand{\Clopen}{\TYPE{Clopen}}
\newcommand{\Closed}{\TYPE{Closed}}
\newcommand{\Separable}{\TYPE{Separable}}
\newcommand{\Dense}{\TYPE{Dense}}
\newcommand{\ND}{\TYPE{NowhereDense}}
\newcommand{\Meager}{\TYPE{Meager}}
\newcommand{\Comeager}{\TYPE{Comeager}}
\newcommand{\Bair}{\TYPE{Baire}}
%FUNC
\DeclareMathOperator*{\intx}{int}
\DeclareMathOperator*{\cl}{cl} 
\DeclareMathOperator*{\boundary}{\partial} 
\DeclareMathOperator{\combo}{\triangledown} 
\DeclareMathOperator{\diag}{\triangle} 
\DeclareMathOperator{\rem}{rem}
%CATS
\newcommand{\TOP}{\mathsf{TOP}}
\newcommand{\HC}{\mathsf{HC}}
\newcommand{\CG}{\mathsf{CG}}
%Symbols
\newcommand{\T}{\mathcal{T}}
\newcommand{\U}{\mathcal{U}}
\newcommand{\V}{\mathcal{V}}
\renewcommand{\O}{\mathcal{O}}
\renewcommand{\d}{\mathrm{d}}
\newcommand{\F}{\mathcal{F}}
\newcommand{\X}{\mathcal{X}}
%\newcommand{\d}{\mathrm{d}}
%Metric Topology
%TYPE
\newcommand{\Lip}{\mathrm{Lip}}
\newcommand{\Complete}{\TYPE{Complete}}
\newcommand{\Metrizable}{\TYPE{Metrizable}}
%FUNC
\DeclareMathOperator{\diam}{diam}
\DeclareMathOperator{\osc}{osc}
\newcommand{\Cell}{\mathbb{B}}
%CATS
\newcommand{\Semiiso}{\mathsf{SMS}_{\circ \to \cdot}}
\newcommand{\Iso}{\mathsf{MS}_{\circ \to \cdot}}
\newcommand{\SMS}{\mathsf{SMS}}
\newcommand{\MS}{\mathsf{MS}}
\newcommand{\UNI}{\mathsf{UNI}}
\newcommand{\UNIS}{\mathsf{UNIS}}
%Boolean Algebra
%TYPE
\newcommand{\Bool}{\mathbb{B}}
\newcommand{\Alg}{\TYPE{Algebra}}
\newcommand{\BR}{\TYPE{BooleanRing}}
\newcommand{\BA}{\TYPE{BooleanAlgebra}}
\newcommand{\PD}{\TYPE{PairwiseDisjointElements}}
\newcommand{\PoU}{\TYPE{PartitionOfUnity}}
\renewcommand{\SS}{\TYPE{StoneSpace}}
\newcommand{\TK}{\mathcal{TK}}
\newcommand{\BL}{\TYPE{BooleanLattice}}
\newcommand{\Fix}{\mathrm{Fix}}
\newcommand{\OC}{\TYPE{OrderClosed}}
\newcommand{\SOC}{\TYPE{SequentiallyOrderClosed}}
\newcommand{\oC}{\TYPE{OrderContinuous}}
\newcommand{\sC}{\TYPE{\sigma\hyph Continuous}}
\newcommand{\OD}{\TYPE{OrderDense}}
\newcommand{\REing}{\TYPE{RegularEmbedding}}
\newcommand{\REed}{\TYPE{RegularEmbeded}}
\newcommand{\REable}{\TYPE{RegularEmbedable}}
\newcommand{\OComplete}{\TYPE{OrderDedekindComplete}}
\newcommand{\TAlgebra}{\TYPE{\tau\hyph Algebra}}
\newcommand{\OCompletes}{\TYPE{OrderDedekindCompleteSubset}}
\newcommand{\SComplete}{\TYPE{\sigma\hyph DedekindComplete}}
\newcommand{\SCompletes}{\TYPE{\sigma\hyph DedekindCompleteSubset}}
\newcommand{\LS}{\mathcal{LS}}
\newcommand{\POpen}{\TYPE{PseudoOpen}}
\newcommand{\od}{\mathbf{OD}}
\newcommand{\mgr}{\mathbf{MGR}}
\newcommand{\nd}{\mathbf{ND}}
\newcommand{\CCC}{\TYPE{WithCountableChainCondition}}
\newcommand{\CSI}{\TYPE{\omega_1\hyph SaturatedIdeal}}
\newcommand{\WD}{\TYPE{(\sigma,\infty)\hyph WeaklyDistributive}}
\newcommand{\Aless}{\TYPE{Atomless}}
\newcommand{\PA}{\TYPE{PurelyAtomic}}
\newcommand{\Homog}{\TYPE{Homogeneous}}
%\newcommand{\FS}{\TYPE{FullSubgroup}}
%\newcommand{\CFS}{\TYPE{CountablyFullSubgroup}}
%\newcommand{\EI}{\TYPE{ExchangingInvolution}}
%\newcommand{\SwS}{\TYPE{SubgroupWithSeparators}}
%\newcommand{\SwmI}{\TYPE{SubgroupWithManyInvolutions}}
%FUNC
\DeclareMathOperator{\upr}{upr}
\DeclareMathOperator{\Atom}{Atom}
%\DeclareMathOperator{\Supp}{Supp}
%\newcommand{\genFS}[1]{\left\langle #1 \right\rangle_\mathrm{F}}
%\newcommand{\genCFS}[1]{\left\langle #1 \right\rangle_\mathrm{CF}}
%\DeclareMathOperator{\Sep}{Sep}
%\DeclareMathOperator{\Tr}{Tr}
%CATS
\newcommand{\BOL}{\mathsf{BOL}}
\newcommand{\BOOL}{\mathsf{BOOL}}
%SYMBOL
\newcommand{\Z}{\mathsf{Z}}
%Descriptive Set Theory
%TYPE
%\newcommand{\Bool}{\mathbb{B}}
\newcommand{\IS}{\TYPE{InitialSegement}}
\newcommand{\FS}[1]{{#1}{}^*}
\newcommand{\Ext}{\TYPE{Extension}}
\newcommand{\Tree}{\TYPE{Tree}}
\newcommand{\Pruned}{\TYPE{Pruned}}
\newcommand{\PTM}{\TYPE{ProperTreeMorphism}}
\newcommand{\LTM}{\TYPE{LipschitzTreeMorphism}}
\newcommand{\Polish}{\TYPE{Polish}}
\newcommand{\IIPG}{\TYPE{InfiniteIterativeTwoPlayersGame}}
\newcommand{\FPS}{\TYPE{FirstPlayerStrategy}}
\newcommand{\SPS}{\TYPE{SecondPlayerStrategy}}
\newcommand{\FPWS}{\TYPE{FirstPlayerWinningStrategy}}
\newcommand{\SPWS}{\TYPE{SecondPlayerWinningStrategy}}
\newcommand{\CS}{\TYPE{ChoquetSpace}}
\newcommand{\SCS}{\TYPE{StrongChoquetSpace}}
\newcommand{\BP}{\mathbf{BP}}
\newcommand{\MGR}{\mathbf{MGR}}
\newcommand{\cat}{\mathbf{CAT}}
\newcommand{\BM}{\TYPE{BairMeasurable}}
\newcommand{\CGSA}{\TYPE{CountablyGeneratedSigmaAlgebra}}
\newcommand{\MC}{\TYPE{MonotonicClass}}
\newcommand{\PSA}{\TYPE{PointSeparatingAlgebra}}
\newcommand{\SBS}{\TYPE{StandardBorelSpace}}
\newcommand{\IH}{\TYPE{InducedHomomorphism}}
%FUNC
\DeclareMathOperator{\len}{len}
\newcommand{\inits}[2]{{#1}_{|\left[1,\ldots,#2\right]}}
\DeclareMathOperator{\lb}{lb}
\DeclareMathOperator{\WFpart}{WF}
\DeclareMathOperator{\Tr}{Tr}
\DeclareMathOperator{\PTr}{PTr}
\DeclareMathOperator*{\Tll}{{T\;\underline{lim}}}
\DeclareMathOperator*{\Tul}{{T\;\overline{lim}}}
\DeclareMathOperator*{\Tl}{{T\;lim}}
\DeclareMathOperator{\rankcb}{rank_{CB}}
\DeclareMathOperator{\lp}{lp}
\newcommand{\alg}{\mathsf{A}}
%CATS
\newcommand{\TREE}{\mathsf{TREE}}
\newcommand{\FSF}{\mathsf{FS}}
\newcommand{\CRONE}{\mathsf{CRONE}}
\newcommand{\BODY}{\mathsf{BODY}}
\newcommand{\BOR}{\mathsf{BOR}}
\newcommand{\bor}{\mathsf{B}}
\newcommand{\Effros}{\mathsf{EFF}}
%symbols
\newcommand{\K}{\mathsf{K}}
\renewcommand{\H}{\mathrm{H}}
\renewcommand{\L}{\mathcal{L}}
\renewcommand{\P}{\mathcal{P}}
\renewcommand{\S}{\mathcal{S}}
\author{Uncultured Tramp} 
\title{Descriptive Set Theory}
\begin{document}
\maketitle
\newpage
\tableofcontents
\newpage
\section{Polish Topology}
\subsection{Trees}
\subsubsection{Finite lists}
\Page{
	\Conclude{\TYPE{List}}{ 
		\Lambda A \in \SET \. 
		\FS{A} = 
		\Lambda A \in \SET \. 
		\bigsqcup^\infty_{n=0}  A^n 
	}{        
		\SET \to \SET
	}
	\\
	\DeclareFunc{length}{ \prod_{A \in \SET} \FS{A} \to \Int_+  }
	\DefineNamedFunc{length}{(n,a)}{\len (n,a)}{n}
	\\
	\DeclareFunc{values}{ \prod_{A \in \SET} \prod_{(n,a) \in   \FS{A}} A^n }
	\DefineNamedFunc{values}{}{(n,a)}{a}
	\\
	\DeclareType{InitialSegment}
	{  
		\prod_{A \in \SET} \FS{A} \to ?\FS{A}
	}
	\DefineNamedType{s}{\IS}{\Lambda x \in \FS{A} \. s \subset x}
	{\Lambda x \in \FS{A} \. \len(x) \ge \len(s) \And x_{|[1,\ldots,\len(s)]} = s }
	\\
	\DeclareType{\Ext}
	{  
		\prod_{A \in \SET} \FS{A} \to ?\FS{A}
	}
	\DefineType{x}{\Ext}{\Lambda s \in \FS{A} \. \IS(A,x,s)}
	\\
	\DeclareType{Compatible}{\prod_{A \in \SET} ?(\FS{A}\times\FS{A})}
	\DefineType{x,y}{Compatible}{x \subset y | y \subset x}
	\\
	\DeclareType{Incompatible}{\prod_{A \in \SET} ?(\FS{A}\times\FS{A})}
	\DefineNamedType{x,y}{Incompatible}{x \bot y}{ \neg \TYPE{Compatible}(A,x,y)   }
	\\
	\DeclareFunc{concatination}{ \prod_{A \in \SET} \FS{A} \times \FS{A} \to \FS{A}}
	\DefineNamedFunc{concatination}{x,y}{xy}{ 
		\bigg( \len(x) + \len(y), \NewLine
			\Lambda i \in \Big[1,\ldots, \len(x) + \len(y) \Big] \.
			\If i \le \len(x) \Then x_i \Else   y_{i-\len(x)}
		\bigg)
	}
	\\
	\DeclareType{InitialSegment}
	{  
		\prod_{A \in \SET} A^\Nat \to ?\FS{A}
	}
	\DefineNamedType{s}{\IS}{\Lambda x \in A^\Nat \. s \subset x}
	{\Lambda x \in A^\Nat \.  \inits{x}{\len(s)} = s }
}
\Page{
	\DeclareType{\Ext}
	{  
		\prod_{A \in \SET} \FS{A} \to ?A^\Nat
	}
	\DefineType{x}{\Ext}{\Lambda s \in \FS{A} \. \IS(A,x,s)}
	\\
	\DeclareType{Incompatible}{\prod_{A \in \SET} ?(\FS{A}\times A^\Nat)}
	\DefineNamedType{x,y}{Incompatible}{x \bot y}{ \neg \IS(A,x,y)   }
	\\
	\DeclareFunc{infConcatination}{ \prod_{A \in \SET} \Big(\Nat \to  \FS{A}\Big) \to \Big(\FS{A} \sqcup A^\Nat\Big)}
	\DefineNamedFunc{infConcatination}{x}{\prod^\infty_{n=1} x_n}{ 
		\Bigg( \sum^\infty_{n=1} \len(x_m), \NewLine
			\Lambda i \in \left[1,\ldots, \sum^n_{i=1} \len(x_i) \right] \.
			\If i \le \len(x_1) \Then x_{1,i} \Else  \left( \prod^\infty_{n=2} x_n \right)_{i-\len(x)}
		\Bigg)
	}
	\\
	\DeclareType{\Tree}{\prod_{A \in \SET} ??\FS{A} }
	\DefineType{T}{\Tree}{\forall t \in T \. \forall n \in \Big[1,\ldots,\len(t)\Big] \. \inits{t}{n} \in T}
	\\
	\DeclareType{Body}{\prod_{A \in \SET} \Tree(A) \to A^\Nat}
	\DefineNamedType{x}{Body}{x \in [A]}{\forall n \in \Nat \. \inits{x}{n} \in T}
	\\
	\DeclareType{ExstensionComplete}{\prod_{A \in \SET} ??\FS{A}}
	\DefineType{X}{ExtensionComplete}{\forall x \in X \. \forall s : \Ext(A,x) \. s \in X}
	\\
	\DeclareType{Pruned}{\prod_{A \in \SET} ?\Tree(A)}
	\DefineType{T}{Pruned}{\forall t \in T \. \exists s \in T : \IS(A,s,t) \And \len(s) > \len(t)}
}
\newpage
\subsubsection{Discrete Topology}
\Page{
	\DeclareFunc{discreteProductMetric}{\prod_{A \in \SET} \TYPE{Metric}(A^\Nat)}
	\DefineNamedFunc{discreteProducMetric}{}{d}
	{
		\Lambda x,y \in A^\Nat \. 
		\If x == y \Then 0 \Else  
		2^{ -1  - \min\{ n \in \Nat : x_n \neq y_n  \}  } 
	}
	\AssumeIn{x,y,z}{A^\Nat}
	\Assume{[1]}{x = z}
	\Conclude{[1.*]}{\Elim d \Elim \FUNC{ifThenElse}[1] \THM{NonNegSumIsNotLess}\Big(d(x,y)\And d(y,z)\Big)}
	{
		d(x,z) = 0 \le d(x,y) + d(y,z)
	}
	\DeriveConclude{[1]}{\Intro(\Imply)}
	{
		x=z \Imply d(x,z)  \le d(x,y) + d(y,z)
	}
	\Assume{[2]}{x \neq z}
	\SayIn{n}{ \min\{ n \in \Nat : x_n \neq z_n \}}
	{
		\Nat
	}
	\Say{[3]}{\Elim (n)}{x_n \neq z_n}
	\Say{[4]}{\Elim \TYPE{Transitive}(A,=)[3]\Intro y_n}
	{
		(x_n \neq y_n) | (y_n \neq z_n)
	}
	\Say{[5]}{\Intro d [4]}{ d(x,z) \le d(x,y) | d(x,z) \le d(y,z) }
	\Conclude{[2.*]}{ \THM{OrMaxIneq}[5] \THM{NonegMaxNonGreatetThenSum}  }
	{
		d(x,y) \le \max\Big(d(x,y),d(y,z)\Big) \le d(x,y) + d(y,z)
	}
	\DeriveConclude{[2]}{\Intro(\Imply)}
	{
		x \neq z \Imply d(x,z)  \le d(x,y) + d(y,z)
	}
	\Conclude{[*]}{\Elim(|)\LOGIC{LEM}(x=z)[1][2]}{ d(x,z) \le d(x,y) + d(y,z)  }
	\EndProof
	\\
	\Theorem{DiscreteProductMetricMetrizeDiscreteProduct}
	{
		\Big(A^\Nat, d\Big)
		\cong_\TOP  \prod^\infty_{n=1} A 
	}
	\NoProof
	\\
	\Theorem{DiscreteProductMetricIsUltrametric}
	{
		\TYPE{Ultrametric}\Big(A^\Nat,d\Big)
	}
	\NoProof
	\\
	\DeclareFunc{standardBase}{\prod_{A \in \SET} \FS{A} \to \T\Big(A^\Nat\Big)}
	\DefineNamedFunc{standardBase}{s}{N_s}{ \Big\{ a \in A^\Nat : \inits{a}{\len(s)} = s \Big\} }
	\\
}\Page{
	\Theorem{StandardBaseIsBase}
	{
		\Type{Base}\Big( \im N, \T\big( A^\Nat\big)  \Big)
	}
	\AssumeIn{U}{\T\big( A^\Nat \big)}
	\Say{\Big(m,S,[1]\Big)}
	{
		\Elim \FUNC{discreteTopology} \Elim \FUNC{productTopology}
	}
	{
		\sum^\infty_{m=0} \sum_{S \subset A^n}  U = S \times A^\Nat
	}
	\Conclude{[U.*]}{\Intro(N)[1]}{ U = \bigcup_{s \in S} N_s}
	\DeriveConclude{[*]}{\Intro \TYPE{Base}}{\TYPE{Base}\Big( \im N, \T\big( A^\Nat \big) \Big)}
	\EndProof
	\\
	\Theorem{MinimalSpanningStandardBase}
	{
		\forall U \in \T(A^\Nat) \.
		\exists S \subset \FS{A} : 
		U = \bigcup_{s \in S} N_s \And
		\forall t,s \in S \. t \neq s \Imply t \bot s
	}
	\NoProof
	\\
	\Theorem{StandardBaseDensityCondition}
	{
		\forall A \in \SET \.
		\forall U \in \T(A^\Nat) \.
		\forall S : \TYPE{ExtensionComplete}(A) : \NewLine : 
		U = \bigcup_{s \in S} N_s 
		\Imply
		\TYPE{Dense}(A^\Nat,U)
		\iff 
		\TYPE{Dense}(\FS{A},S)
	}
	\Assume{[1]}{\TYPE{Dense}(\FS{A},S)}
	\Say{[2]}{\Elim \TYPE{Dense}[1]}
	{
		\forall s \in \FS{A} \. \exists t \in S : s \subset t
	}
	\AssumeIn{x}{A^\Nat}
	\AssumeIn{n}{\Nat}
	\SayIn{s}{\inits{x}{n}}{\FS{A}}
	\Say{\Big(t,[3]\Big)}{[2](s)}
	{
		\sum t \in S \. s \subset t
	}
	\Say{[4]}{\Elim N [3]}{N_t \subset N_s}
	\Conclude{[x.*]}{[0][4]\THM{UnionIntersect}}
	{
		U \cap N_s \neq \emptyset
	}
	\DeriveConclude{[1.*]}{\THM{DenseByNeighborhoodBase}}
	{
		\TYPE{Dense}(A^\Nat,U)
	}
	\Derive{[1]}{\Intro(\Imply)}
	{
		\TYPE{Dense}(\FS{A},S) 
		\Imply
		\TYPE{Dense}(A^\Nat,U)
	}
	\Assume{[2]}{\TYPE{Dense}(A^\Nat,U)}
	\AssumeIn{s}{\FS{A}}
	\Say{[3]}{\Elim \TYPE{Dense}(A^\Nat,U)}{N_s \cap U \neq \emptyset}
	\Say{[4]}{\Elim \FUNC{union} [0][4]}{ \exists t \in S : N_s \cap N_t \neq \emptyset   }
	\Say{[5]}{\Elim N [4]}{\exists t \in S : t \subset s | s \subset t}
	\Conclude{[s.*]}{\Elim \TYPE{ExtensionComplete}(A,S)[5]}
	{
		\exists t \in S : s \subset t
	}
	\DeriveConclude{[2.*]}{\Intro \TYPE{Dense}}
	{
		\TYPE{Dense}(\FS{A},S)
	}
	\Derive{[2]}{\Intro(\Imply)}
	{
		\TYPE{Dense}(A^\Nat,U)
		\Imply
		\TYPE{Dense}(\FS{A},S) 
	}
	\Conclude{[*]}{\Intro(\iff)[1][2]}	
	{
		\TYPE{Dense}(A^\Nat,U)
		\iff
		\TYPE{Dense}(\FS{A},S)
	}
	\EndProof
}\Page{
	\Theorem{DiscreteProductConvergence}
	{
		\forall A \in \SET \.
		\forall x : \Nat \to A^{\Nat} \. \NewLine \. 
		\forall L \in A^\Nat \.
		L = \lim_{n \to \infty} x_n \iff
		\forall m \in \Nat \. 
		\exists N \in \Nat :
		\forall n \in \Nat :
		n \ge N \Imply x_{n,m} = L_m
	}
	\NoProof
	\\
	\DeclareFunc{tree}{\prod_{A \in \SET} ?A^\Nat \to \TYPE{Pruned}(A)}
	\DefineNamedFunc{tree}{X}{T(X)}
	{
		\{ \inits{x}{n} | x \in X, n \in \Int_+ \}
	}
	\\
	\Theorem{BodyBijection}{ 
		\TYPE{Bijection}\Big( \Lambda T : \TYPE{Pruned}(A) \. [T] ,
			\TYPE{Pruned}(A),
			\TYPE{Closed}(A^\Nat)
		\Big)
	}
	\Say{F}{\Lambda T : \TYPE{Pruned}(A) \. [T]}{\TYPE{Pruned}(A) \to ?A^\Nat}
	\Assume{T}{\TYPE{Pruned}(A)}
	\AssumeIn{x}{F^\c(T)}
	\Say{\Big(n,[1]\Big)}{\Elim x \Elim F}
	{
		\sum n \in \Nat \.  \inits{x}{n} \in T^\c
	}
	\Say{[2]}{\Elim \TYPE{Tree}(T)[1] }
	{
		\forall t \in \FS{A} \. s \subset t \Imply t \not \in T^\c
	}
	\Conclude{[x.*]}{\Intro N_s \Intro F [2]}{N_s \cap F(T) = \emptyset}
	\DeriveConclude{[T.*]}{\THM{CloseByNeigborhoodBase}}
	{
		\TYPE{Closed}\Big( A^\Nat, F(T)\Big))
	}
	\Derive{[1]}{\Intro \im}{\im F \subset \TYPE{Closed}(A^\Nat)}
	\Assume{X}{\TYPE{Closed}(A^\Nat)}
	\Say{Y}{TF(X)}{\TYPE{Closed}(A^\Nat)}
	\AssumeIn{x}{X}
	\Say{[2]}{\Elim T(X)(x)}{\forall n \in \Nat \. \inits{x}{n} \in T(X)}
	\Conclude{[x.*]}{ \Elim F [2] \Intro Y }{x \in Y}
	\Derive{[2]}{\Intro \TYPE{Subset}}{X \subset Y }
	\AssumeIn{y}{Y}
	\Say{[2]}{\Elim Y \Elim F y }{\forall n \in \Nat \. \inits{y}{n} \in T(X)}
	\Say{[3]}{\Elim T(X) \Intro N [2]}{ \forall n \in \Nat \. N_{\inits{y}{n}} \cap X \neq \emptyset }
	\Say{[4]}{\THM{ClosureByNeighborhoodBase}[3]}{y \in \overline{X}}
	\Conclude{[y.*]}{\Elim \FUNC{closure}[4]\Elim \TYPE{Closed}(A^\Nat,X)}{y \in X}
	\Derive{[3]}{\Intro \TYPE{Subset}}{Y \subset X}
	\Conclude{[X.*]}{\Intro \TYPE{SetEq}[2][3]}{Y = X}
	\DeriveConclude{[2]}{\Intro \TYPE{Surjective}}{\TYPE{Surjective}(F,\TYPE{Pruned}(A),\TYPE{Closed}(A^\Nat))}
	\Assume{\alpha,\beta}{\TYPE{Pruned}(A)}
	\Assume{[3]}{F(\alpha) = F(\beta)}
	\Conclude{\Big[(\alpha,\beta)\Big]}{\Elim F \Elim \TYPE{Pruned}(A,\alpha \And \beta)}
	{ \alpha = \beta  }
	\Derive{[3]}{\Intro \TYPE{Injective}}
	{  
		\TYPE{Injective}(F,\TYPE{Pruned}(A),\TYPE{Closed}(A^\Nat))
	}
	\Conclude{[*]}{\Intro \TYPE{Bijective}[2][3]}
	{
		\TYPE{Bijective}(F,\TYPE{Pruned}(A),\TYPE{Closed}(A^\Nat))
	}
	\EndProof
}
\Page{
	\DeclareFunc{rootedTree}{\prod_{A \in \SET} \Tree(A) \to \FS{A} \to \Tree(A) }
	\DefineNamedFunc{rootedTree}{T,s}{T_s}{\{ t \in \FS{A} : st \in T \}}
	\\
	\DeclareFunc{filteredTree}{\prod_{A \in \SET} \Tree(A) \to \FS{A} \to \Tree(A)}
	\DefineNamedFunc{filteredTree}{T,s}{T_{[s]}}{\{ t \in \FS{A} : \neg(t \bot s) \}}
	\\
	\DeclareFunc{finiteSequencesAsPoset}{\prod_{A \in \SET} \POSET}
	\DefineNamedFunc{finiteSequencesAsPoset}{}{\FS{A}}{\Big( \FS{A},\subset\Big)}
	\\
	\DeclareFunc{residualBody}{
		\prod_{A,B \in \SET} 
		\prod T : \Tree(A) \. 
		\prod S : \Tree(B) \. 
		\POSET(T,S) \to ?[T]
	}
	\DefineNamedFunc{residualBody}{f}{D(f)}
	{
		\Big\{ x \in [T] : \lim_{n \to \infty} \len f(\inits{x}{n}) = \infty  \Big\}
	}
	\\
	\DeclareFunc{bodyPushforward}
	{
		\prod_{A,B \in \SET} 
		\prod T : \Tree(A) \.
		\prod S : \Tree(B) \.
		\prod f \in \POSET(T,S) \.
		D(f) \to [S]
	}
	\DefineNamedFunc{bodyPushforward}{x}{f^*(x)}{\bigcup_{n=1}^\infty f(\inits{x}{n})}
	\\
	\DeclareType{ProperTreeMorphism}{\prod_{A,B \in \SET} \prod T : \Tree(A) \prod S : \Tree(B) ?\POSET(A,B) }
	\DefineType{f}{\PTM}{D(f) = [T]}
	\\
	\Theorem{ResidualBodyIsGDelta}
	{
		\forall A,B \in \SET \.
		\forall T : \Tree(A) \.
		\forall S : \Tree(B) \.
		\forall f \in \POSET(T,S) \.
		D(f) \in G_\delta[T]
	}
	\Say{U}{\Lambda m \in \Nat \. \Big\{ x \in [T] : \exists n \in \Nat \. \len f(\inits{x}{n}) \ge  m \Big\}}
	{
		\Nat \to ?[T]
	}
	\AssumeIn{m}{\Nat}
	\AssumeIn{u}{U_m}
	\Say{\big(n,[1]\big)}{\Elim U_m(u)}
	{
		\sum n \in \Nat \. \len f(\inits{x}{n}) \ge m
	}
	\Say{s}{\inits{x}{n}}{A^n}
	\Conclude{[m.*]}{\Elim \POSET(T,S)(f)[1]\Intro N}
	{
		N_s \cap [T] \subset U_m
	}
	\Derive{[1]}{\THM{OpenByOpenCover}\Intro(\forall)}{\forall m \in \Nat \. U_m \in \T[T]}
	\Say{[2]}{\Elim D(f) \Elim \lim  \THM{SetBuilderUniversal} \Intro U_m }
	{ 
		D(f) = 
		\Big\{ x \in [T] : \lim_{n \to \infty} \len f(\inits{x}{n}) = \infty \Big\} = \NewLine = 
		\Big\{ x \in [T] : \forall m \in \Nat \. \exists n \in \Nat : f(\inits{x}{n}) \ge m  \Big\}  =
		\bigcap^\infty_{m=1} \{ x \in [T] : \exists n \in \Nat : f(\inits{x}{n}) \ge m \Big\} = 
		\bigcap^\infty_{m=1} U_m
	}
	\Conclude{[*]}{\Intro G_\delta [2]}{D(f) \in G_\delta[T]}
	\EndProof
}
\Page{
	\Theorem{BodyPushforwardIsContinuous}
	{
		\forall A,B \in \SET \.
		\forall T : \Tree(A) \.
		\forall S : \Tree(B) \.
		\forall f \in \POSET(T,S) \. \NewLine \.
		f^* \in \TOP\Big( D(f), [S] \Big)
	}
	\NoProof
	\\
	\Theorem{InversePushforwardTheorem}
	{
		\forall A,B \in \SET \.
		\forall T : \Tree(A) \.
		\forall S : \Tree(B) \.
		\forall D \in G_\delta [T] \. \NewLine \. 
		\forall \varphi \in \TOP(D,[S]) \.
		\exists! f \in \POSET(T,S) :
		D = D(f) \And \varphi = f^*
	}
	\Say{\Big(U,[1]\Big)}{\Elim G_\delta[T](D) }
	{
		\sum U : \Int_+ \downarrow \T[T] \. U_0 = [T] \And \bigcap_{n=1}^\infty U_n = D
	}
	\Say{k}{\Lambda t \in T \. \min\Big( \len(s), \max\{ k \in \Int_+ : N_t \cap [T]\subset U_k\}\Big)}
	{
		T \to \Int_+
	}
	\Say{f}
	{
		\Lambda t \in T \. 
		\If N_t \cap D \neq \emptyset 
		\Then \max\{ u \in S \. \len(u) \le k(s) \And \varphi(N_t\cap D) \subset N_u \} \NewLine 
		\Else f(\inits{t}{m}) \quad \where \quad m = \max \{ m \in [0,\ldots,\len t] : N_{\inits{t}{m}} \cap D \neq \emptyset\} 
	}
	{
		T \to S
	}
	\AssumeIn{t,t'}{T}
	\Assume{[2]}{t \subset t'}
	\Assume{[3]}{\varphi(N_{t'} \cap D) \neq \emptyset}
	\Say{[4]}{ \THM{NonemptyImage}[3] }{N_{t'} \cap D \neq \emptyset}
	\Say{[5]}{\Elim N [3]}{N_{t'} \subset N_t}
	\Say{[6]}{ \THM{MonotonicIntersect}[3][5] }{N_t \cap D \emptyset}
	\Say{[7]}{\Elim U [4](\len(t'))}{N_{t'} \cap U_{\len t'} \neq \emptyset}
	\Say{[8]}{\Elim U[6](\len(t))}{N_{t} \cap U_{\len t} \neq \emptyset}
	\Say{[9]}{\Intro k(t')[7]}{k(t') = \len(t')}
	\Say{[10]}{\Intro k(t')[8]}{k(t) = \len(t)}
	\Say{[11]}{[9][2][11]}{k(t) \le k(t')}
	\Conclude{[3.*]}{\Elim f(t) \Elim f(t')[5][11] \Intro f(t) \Intro f(t') }
	{
		f(t) \subset f(t')
	}
	\Derive{[3]}{\Intro(\Imply)}
	{
		\varphi(N_{t'} \cap D) \neq \emptyset \Imply f(t) \subset f(t') 
	}
	\Assume{[4]}{\varphi(N_{t'} \cap D)=\emptyset}
	\Conclude{[4.*]}{\Elim f \Elim \FUNC{ifElseThen} [4][3]}
	{
		f(t) \subset f(t')
	}
	\Derive{[4]}{\Intro(\Imply)}{\varphi(N_{t'} \cap D) = \emptyset \Imply f(t) \subset f(t') }
	\Conclude{\Big[(t,t').*\Big]}{ \Elim(|)\LOGIC{LEM}\Big(\varphi(N_{t'} \cap D) = \emptyset\Big)[4][3]  }
	{
		f(t)	\subset f(t')
	}
	\Derive{[2]}{\Intro \POSET}{f \in \POSET(T,S)}
	\AssumeIn{x}{D(f)}
	\Say{[3]}{\Elim D(f)(x)}{\lim_{n \to \infty} f\Big( \inits{x}{n}\Big) = \infty}
	\Say{[4]}{\Elim f [3]}{ 
		\Big(\forall n \in \Nat \.  \varphi( N_{\inits{x}{n}} \cap D) \neq \emptyset\Big)
		\And \NewLine \And
		\Big( \forall m \in \Nat \. \exists b \in B^m : \exists n \in \Nat :
			N_{\inits{x}{n}} \cap [T] \subset U_m \And 
			\varphi( N_{\inits{x}{n}} \cap D  ) \subset N_b		
		\Big)
	}
	\Say{[5]}{\THM{EmptyImage}[4.1]}{\forall n \in \Nat \. N_{\inits{x}{n}} \cap D \neq \emptyset}
	\Say{\Big(y,[6]\Big)}{\Elim N [5] [4.2] \THM{LimByNeighbourhoodBase}}
	{
		\sum y : \prod^n_{i=1} U_n \. x = \lim_{n \to \infty} y_n
	}
	\Conclude{[x.*]}{\Elim U [1] \Elim y[6]}{x = \lim_{n \to \infty} y_n \in D }
	\Derive{[3]}{\Intro \TYPE{Subset}}{D(f) \subset D}
}\Page{
	\AssumeIn{x}{D}
	\Say{[4]}{\Elim \TYPE{NeighborhoodBase}(N)\Elim x}
	{     
		\forall n \in \Nat \. N_{\inits{x}{n}} \cap D \neq \emptyset
	}
	\Say{[5]}{\varphi[4]}
	{
		\forall n \in \Nat \. 
		\varphi(N_{\inits{x}{n}} \cap D) \neq \emptyset
	}
	\Say{[6]}{[4][1]}
	{
		\forall n \in \Nat \.
		N_{\inits{x}{n}} \cap U_n \neq \emptyset
	}
	\Say{[7]}{\Intro k [6]}
	{
		\forall n \in \Nat \. k(\inits{x}{n}) = n
	}
	\Say{[8]}{\Elim \TOP\Big([T],[S]\Big)(\varphi)\Elim \TYPE{T1}[S]}
	{
		\bigcap^n_{i=1} \varphi(N_{\inits{x}{n}} \cap D) = \{ \varphi(x) \}
	}
	\Say{[9]}{\THM{MonotonicIntersectSubset}[8]}{
		\forall m \in \Nat \. 
		\exists n \in \Nat : 
		\varphi(N_{\inits{x}{n}} \cap D) \subset N_{\inits{\varphi(x)}{m}}
	}
	\Say{[10]}{\Intro f [7][9]}{\lim_{n \to \infty} \len f(\inits{x}{n}) = \infty}
	\Conclude{[x.*]}{[10]\Intro D(f)}{x \in D(f)}
	\Derive{[4]}{\Intro \TYPE{Subset}}{D \subset D(f)}
	\Say{[5]}{\Intro \TYPE{SetEq}[3][4]}{D = D(f)}
	\AssumeIn{x}{D}
	\Say{[6]}{\Elim x \ldots}{\forall n \in \Nat \. \varphi(N_{\inits{x}{n}} \cap D) \neq \emptyset}
	\AssumeIn{m}{\Nat}
	\Say{\Big(n,[7]\Big)}{\Elim \TOP\Big([T],[S],\varphi\Big)(N_{\inits{\varphi(x)}{m}})\Elim \TYPE{Base}(N)(x)}
	{
		\sum n \in \Nat \. \varphi(N_{\inits{x}{n}} \cap D) \subset  N_{\inits{\varphi(x)}{m}}
	}
	\Say{[m.*]}{\Intro f [7] }{ \inits{\varphi(x)}{m} \subset f(\inits{x}{n}) }
	\Derive{[7]}{\Intro \forall}{\forall m \in \Nat \. \inits{\varphi(x)}{m} \subset f(\inits{x}{n})}
	\Conclude{[x.*]}{\Intro \TYPE{Union} [7]\THM{SetFunctionUnion}(\varphi(x)) \Intro f^*}
	{
		\varphi(x) = \bigcup^\infty_{n=1} f(\inits{x}{n}) =f^* 
	}
	\DeriveConclude{[*]}{\Intro(\to,=)}{\varphi = f^*}
	\EndProof
	\\
	\DeclareType{LipschitzTreeMorphism}
	{
		\prod_{A,B \in \SET} \prod T : \Tree(A) \. \prod S : \Tree(B) \. ?\POSET(T,S)
	}
	\DefineType{f}{\LTM}{\forall t \in T \. \len f(t) = \len t}
	\\
	\Theorem{LipschitzTreeMorphismPushforwardIsLipschitzMap}
	{
		\NewLine :: 
		\forall A,B \in \SET \.
		\forall T : \Tree(A) \.
		\forall S : \Tree(B) \. 
		\forall f  : \LTM(T,S) \. \NewLine \. 
		f^* \in \Lip\Big(\big([T],d\big),\big([S],d\big)\Big)
	}
	\AssumeIn{x,y}{[T]}
	\Assume{[1]}{x \neq y}
	\Say{\Big(n,[2]\Big)}{\Elim(\to,\#)[1]\Elim \WF(\Nat)}
	{
		\sum n \in \Nat \. x_n \neq y_n \And \inits{x}{n-1} = \inits{y}{n-1}
	}
	\Say{[3]}{\Intro d [2]}{d(x,y) = 2^{-n}}
	\Say{[4]}{\Elim(\to,=)[2.1] \Elim \LTM(f)\Intro f^*}
	{
		f(\inits{x}{n-1}) = \inits{f^*(x)}{n-1} = \inits{f^*(x)}{n-1} = f(\inits{x}{n-1})
	}
	\Conclude{\Big[(x,y).*\Big]}{\Intro d [4][2]}
	{
		d\Big( f^*(x), f^*(y) \Big) \le 2^{-n} = d(x,y)
	}
	\DeriveConclude{{*}}{\Intro \Lip}{f^* \in \Lip\big([T],[S]\big)}
	\EndProof
}
\Page{
	\Theorem{InBairlikeSpaceAllClosedAreRetracts}
	{
		\NewLine ::
		\forall A \in \SET \. 
		\forall H,K : \TYPE{Closed} \And \TYPE{NonEmpty}(A^\Nat) \.
		H \subset K \. 
		\Imply \TYPE{Retract}(K,H)
	}
	\Say{(\le)}{\THM{ZermelosTHM}(H)}{ \TYPE{WellOrdering}(H)   }
	\Say{f}{
		\Lambda k \in K \. 
		\If k \in H 
		\Then k 
		\Else \NewLine 
		\Elim \min\Big\{ 
			h \in H : 
			\inits{h}{n} = \inits{k}{n} \Big|
			n = \max\{ 
				n \in \Nat : 
				\exists h \in H : 
				\inits{h}{n} = \inits{k}{n} 
				\}      
		\Big\}
	}{K \to H}
	\AssumeIn{h}{H}
	\AssumeIn{n}{H}
	\AssumeIn{k}{f^{-1}(N_{\inits{h}{n}}\cap H)}
	\Say{[1]}{\Elim \FUNC{preimage}{k}}{f(k) \in N_{\inits{h}{n}} \cap H}
	\Say{[2]}{\Elim N_{\inits{h}{n}}[1]}{\inits{f(k)}{n} = \inits{h}{n}}
	\Assume{[3]}{\inits{k}{n} = \inits{h}{n}}
	\Conclude{[3.*]}{\Elim f [3]}{N_{\inits{k}{n}} \cap K \subset f^{-1}(N_{\inits{h}{n}})}
	\Derive{[3]}{\Intro(\Imply)}   
	{
		\inits{k}{n} = \inits{h}{n} \Imply N_{\inits{k}{n}} \cap K \subset f^{-1}(N_{\inits{h}{n} \cap H)}
	}
	\Assume{[4]}{\inits{k}{n} \neq \inits{h}{n}}
	\Say{[5]}{\Elim f [4][2]}{f (N_{\inits{k}{n}} \cap K ) = \{h\}}
	\Conclude{[4.*]}{\Intro \TYPE{Preimage}[5]}
	{
		N_{\inits{k}{n}} \cap K  \subset f^{-1}(N_{\inits{h}{n}}\cap H)
	}
	\Derive{[4]}{\Intro(\Imply)}   
	{
		\inits{k}{n} \neq \inits{h}{n} \Imply N_{\inits{k}{n}} \cap K \subset f^{-1}(N_{\inits{h}{n}} \cap H)
	}
	\Conclude{[k.*]}{\Elim(|)\LOGIC{LEM}[3][4]}
	{
		N_{\inits{k}{n}} \cap K \subset f^{-1}(N_{\inits{h}{n}} \cap H)
	}
	\DeriveConclude{[h.*]}{\THM{OpenByOpenCover}}{f^{-1}(N_{\inits{h}{n}}\cap H) \in \T(K)}
	\Derive{[1]}{\Intro \TOP}{f \in \TOP(K,H)}
	\Say{[2]}{\Elim f \Intro \TYPE{Retraction}}{\TYPE{Retraction}(K,H,f)}
	\Conclude{[*]}{\Intro \TYPE{Retract}}{\TYPE{Retract}(K,H)}
	\EndProof
}
\newpage
\subsubsection{Category}
\Page{
	\DeclareFunc{treeCategory}{\CAT}
	\DefineNamedFunc{treeCategory}{}{\TREE}
	{
		\left(
			\sum_{X \in \SET} \Tree(X),
			\PTM,
			\circ,
			\id
		\right)
	}
	\\
	\DeclareFunc{FullTreeFunctor}{\Cov(\SET,\TREE)}
	\DefineNamedFunc{FullTreeFunctor}{X}{\FSF(X)}{\Big(X,\FS{X}\Big)}
	\DefineNamedFunc{FullTreeFunctor}{X,Y,f}{\FSF_{X,Y}(f)}
	{ 
		\Lambda \omega \in \FSF(X) \. 
		\If \omega = \emptyset 
		\Then \emptyset \NewLine
		\Else \FSF_{X,Y}(f)(\inits{\omega}{\len(\omega)-1})f(\omega_{\len(\omega)}) 
	}	
	\\
	\DeclareFunc{CroneFunctor}{\Cov(\SET,\TREE)}
	\DefineNamedFunc{CroneFunctor}{X}{\CRONE(X)}{\Big(X, X \cup \{\emptyset\}  \Big)}
	\DefineNamedFunc{CroneFunctor}{X,Y,f}{\CRONE_{X,Y}(f)}
	{ 
		\Lambda \omega \in \CRONE(X) \. 
		\If \omega = \emptyset 
		\Then \emptyset
		\Else f(\omega)
	}	
	\\
	\DeclareFunc{TreeEmbedding}{\TYPE{ReflexiveEmbedding}(\TREE, \POSET)}
	\DefineNamedFunc{TreeEmbedding}{A,T}{(A,T)}{T}
	\DefineNamedFunc{TreeEmbedding}{X,Y,f}{(X,Y,f)}{f}
	\\
	\DeclareFunc{BodyFunctor}{\Cov \And \TYPE{Full}(\TREE, \MS)}
	\DefineNamedFunc{BodyFunctor}{A,T}{\BODY(A,T)}{[T]}
	\DefineNamedFunc{BodyFunctor}{X,Y,f}{\BODY_{X,Y}(f)}{f^*}
	\\
	\DeclareFunc{treeProduct}{\prod_{\I \in \SET} (\I \to \TREE) \to \TREE}
	\DefineNamedFunc{treeProduct}{(X,T)}{\prod_{i \in \I} (X_i,T_i)}
	{
		\left( \prod_{i \in \I} X_i,    
			\left\{
				x \in \FS{\left(\prod_{i \in \I} X_i\right)}
				\Bigg|
				\forall i \in \. \pi_i^*(x) \in \FS{X_i} 
			\right\}
		\right)
	}
	\\
	\Theorem{PrunedTreeProduct}
	{
		\forall \I \in \SET \.
		\forall X : \I \to \SET \.
		\forall T : \prod_{i \in \I} \TYPE{Pruned}(X_i) \. 
		\TYPE{Pruned}\left( \prod_{i \in \I} X_i, \prod_{i \in \I} T_i \right)
	}
	\NoProof
	\\
	\Theorem{ProductBody}
	{
		\forall \I : \TYPE{Finite} \.
		\forall X : \I \to \SET \. 
		\forall T : \prod_{i \in \I}  \Tree(X_i) \.
		\left[ \prod_{i \in \I} T_i \right] \cong_\TOP \prod_{i \in \I} [T_i]
	}
	\NoProof
	\\
	\DeclareFunc{treeSection}
	{
		\prod A,B \in \SET \.
		\Tree(A \times B) \to A^\Nat \to \Tree(B)
	}
	\DefineNamedFunc{treeSection}{T,a}{T(a)}
	{
		\Big\{ b \in \FS{b} \Big|  \Big(\inits{a}{\len b}, b\Big) \in T  \}
	}
}
\newpage
\subsubsection{Well Foundness}
\Page{
	\DeclareType{IllFounded}{\prod_{A \in \SET} ?\Tree(A)}
	\DefineType{T}{IllFounded}{[T]\neq \emptyset}
	\\
	\DeclareType{WellFounded}{\prod_{A \in \SET} ?\Tree(A)}
	\DefineType{T}{WellFounded}{[T] = \emptyset}
	\\
	\DeclareFunc{leftmostBranch}
	{
		\prod A \in \ORD \. \TYPE{IllFounded}(A) \to A^\Nat
	}
	\DefineNamedFunc{leftmostBranch}{T,n}{(\lb T)_n}
	{
		\min \{ a \in A :  [T_{\inits{\lb T}{n-1}a}] \neq \emptyset \}
	}
	\\
	\DeclareFunc{branchRank}{\prod_{A \in \SET} \prod T :  \TYPE{WellFounded} \. T \to \Int_+ \cup \{\infty\}}
	\DefineNamedFunc{branchRank}{t}{\rank t}{
		\If \{ a \in A : ta \in T  \} = \emptyset
		\Then 0
		\Else \sup \{ 1 + \rank(ta) | a \in A : ta \in T   \}
	}
	\\
	\DeclareFunc{treeRank}{\prod_{A \in \SET}  \TYPE{WellFounded} \to \Int_+ \cup \{\infty\}}
	\DefineNamedFunc{treeRank}{\emptyset}{\rank \emptyset}{0}
	\DefineNamedFunc{treeRank}{T}{\rank T}
	{
		1 + \rank_T \emptyset
	}
}\Page{
	\Theorem{TreeRankAndWellFoundness}
	{
		\forall A,B \in \SET 
		\forall T : \WF(A) \.
		\forall S \in \Tree(B) \. 
		\NewLine \. 
		\Big( \WF(B,S) \And \rank S \le \rank T  \Big)
		\iff \exists \TYPE{StrictlyMonotonic}(S,T)
	}
	\Assume{[1]}{\WF(B,S)}
	\Assume{[2]}{\rank S \le \rank T}
	\Assume{[0]}{S \neq \emptyset}
	\Say{[00]}{\Elim [2]\rank S [0]}{T \neq \emptyset}
	\SayIn{f(\emptyset)}{\emptyset_{\Nat \times A}}{T}
	\Say{[4.0]}{[2]\Elim f(\emptyset)}{ 
		\rank_T f(\emptyset) = 
		(\rank T) - 1    \ge 
		(\rank S) - 1 = 
		\rank_S \emptyset
	}
	\AssumeIn{n}{\Nat}
	\AssumeIn{s}{S}
	\Assume{[3]}{\len s = n}
	\Say{[5]}{\Elim \Big[4.(n-1)\Big](\inits{s}{n-1})}
	{
		\rank f(\inits{s}{n-1}) \ge \rank \inits{s}{n-1}
	}
	\Say{\Big(a,[6]\Big)}{\Elim \rank [5]}
	{
		\sum a \in A \. \rank f(\inits{s}{n-1}) = 1 + \rank f(\inits{s}{n-1})a
	}
	\SayIn{f(s)}{f(\inits{s}{n-1})}{T}
	\Say{[7]}{\Elim \rank s}{\rank \inits{s}{n+1} \ge 1 + \rank s}
	\Say{4.n}{[6][5][7]}
	{
		\rank f(s) =
		(\rank f(\inits{s}{n-1}) )  - 1  \ge 
		(\rank \inits{s}{n-1}) -1  \ge 
		\rank s
	}
	\Conclude{[n.*]}{\Elim f(s) }{ f(s) > f(\inits{s}{n-1})  }
	\DeriveConclude{\Big(f,[4]\Big)}{\Intro \TYPE{StrictlyMonotonic}}
	{
		\sum f : \TYPE{StrictlyMonotonic}(S,T) \. 
		\forall n \in \Nat \.
		\forall s \in S \. \NewLine \. 
		\len s = n \Imply 
		\rank f(s) \ge \rank s
	}
	\Derive{[1]}{\Intro(\Imply)}
	{
		\WF(S) \And \rank S \le \rank T 
		\Imply
		\exists \TYPE{StrictlyMonotonic}(S,T)
	}
	\Assume{f}{\TYPE{StrictlyMonotonic}(S,T)}
	\Assume{[2]}{\TYPE{IllFounded}(S)}
	\SayIn{b}{\Elim \TYPE{IllFounded}(S)}{[S]}
	\Say{[3]}{\Elim \TYPE{StrictlyMonotonic}(f)}
	{
		\lim_{n \to \infty} f(\inits{b}{n}) \in [T]
	}
	\Conclude{[2.*]}{\Elim \WF(T)[3]}{\bot}
	\Derive{[2]}{\Intro \WF(S)[4]}{\WF(S)}
	\Say{[3]}{\Lambda s \in S \. \Elim \TYPE{StrictlyMonotonic}(S,T,f)\Intro \rank s}
	{
		\forall s \in S \. \rank f(s) \ge \rank s
	}
	\Conclude{[2.*]}{\Elim \rank S [3] \Elim \sup  \Intro \rank T }{
		\NewLine :
		\rank S = 
		\sup\{ 1 + \rank s | s \in S\} \le  
		\sup\{ 1 + \rank f(s) | s \in S \} \le
		\sup \{ 1 + \rank t | t \in T \} =
		\rank T
	}
	\Derive{[2]}{\Intro \Imply}
	{
		\exists \TYPE{StrictlyMonotonic}(S,T) 
		\Imply
		\WF(S) \And \rank S \le \rank T
	}
	\Conclude{{*}}{\Intro (\iff)[1][2]}
	{
		\WF(S) \And \rank S \le \rank T 
		\iff
		\exists \TYPE{StrictlyMonotonic}(S,T)
	}
	\EndProof
	\\
	\DeclareFunc{orderTree}{\prod A \in \SET \. \TYPE{Order}(A) \to \Tree(A)}
	\DefineFunc{orderTree}{\le}{
		\Big\{
			a : [1,\ldots,n] \to A 
			\Big|
			n \in \Int_+, \forall i \in [1,\ldots,n-1] \. a_{i+1} < a_i
		\Big\}
	}
	\\
	\Theorem{OrderTreeWellFoundness}
	{
		\forall A \in \SET \.
		\forall (\le) : \TYPE{Order}(A) \. \NewLine \.
		\WF(A,\le) \iff \WF\Big( \FUNC{orderTree}(A,\le) \Big)
	}
	\NoProof
}
\newpage
\Page{
	\DeclareFunc{wellFoundedPart}
	{
		\prod_{A \in \SET} \Tree(A) \to ?A
	}
	\DefineNamedFunc{wellFoundedPart}{T}{\WFpart_T}
	{
		\{ t \in T : \WF(T_t)  \}
	}
	\\
	\DeclareFunc{rankOfWellFoundedPart}
	{
		\prod_{A \in \SET} \prod T : \Tree(A) \. \WFpart_T \to \Int_+ \cup \{\omega\} 
	}
	\DefineNamedFunc{rankOfWellFoundedPart}{t}{\rank t}{
		\If \{ a \in A : ta \in T  \} = \emptyset
		\Then 0 
		\Else  \NewLine \sup \{ 1 + \rank(ta) | a \in A : ta \in T   \}
	}
	\\
	\DeclareFunc{rankOfIllFoundedBranch}
	{
		\prod_{A \in \SET} \prod T : \Tree(A) \. T \to \Int_+ \cup \{\omega\} \cup 
		\{\alpha\}
	}
	\DefineNamedFunc{rankOfIllFoundedBranch}{t}{\rank t}{
		\If t \in \WF_T
		\Then \rank t
		\Else \infty
		\NewLine 
		\where \quad
		\infty = \min \{ a \in \ORD : |a| > \infty \And |a| > |A|  \}
	}
	\\
	\DeclareType{KleeneBrouwerOrder}{\prod A : \Toset \. \TYPE{TotalOrder}(\FS{A})}
	\DefineNamedType{(s,t)}{KleeneBrouwerOrder}{s \le_{\mathrm{KB}} t}
	{
		t \subset s \Big| \NewLine 
		\exists i \in \Big[1,\ldots,\min\big(\len(s),\len(t)\big)\Big] :
		s_i < t_i \And \forall j \in [1,\ldots,i-1] \. s_j = t_j
	}
	\\
	\Theorem{KleeneBrouweTHM}
	{
		\forall A : \WO \.
		\forall T : \Tree(A) \.
		\WF(T) \iff \WO(T,\le_{\mathrm{KB}} )
	}
	\Assume{[1]}{\WF(T)}
	\Assume{X}{?T}
	\Assume{[2]}{\min_{\mathrm{KB}} X = \emptyset}
	\Say{\Big(x,[3]\Big)}{\Elim \min [2]}
	{
		\sum x : \Nat \to X : \forall n,m \in \Nat \. n > m \Imply x_n <_{\mathrm{KB}} x_m
	}
	\Say{\Big(m,i,[4]\Big)}{\Elim \WF(T)[3] }
	{
		\sum m : \Nat \to \Nat \. 
		\prod^\infty_{n=1} i_n \in 
		\Big[1,\ldots \min(\len x_{m_n},\len x_{m_n + 1})\Big] \.
		\NewLine
		\.
		\forall n \in \Nat \.
		x_{m_n,i_n} > x_{m_n + 1,i_n}
		\And
		\forall j \in [1,\ldots,i_n - 1] \.
		x_{m_n,j} = x_{m_n +1, j}
	}
	\Say{[5]}{\Elim \WO(A)[4]}{\lim_{n \to \infty} i_n = \infty}
	\Say{[6]}{ [5][3]  }{ [T] \neq \emptyset }
	\Conclude{[*]}{\Elim \WF(T)}{\bot}
	\Derive{[1]}{\Intro \Imply}
	{
		\WF(T)
		\Imply
		\WO(T,\le_{\mathrm{KB}})
	}
	\Assume{[2]}{\WO(T,\le_{\mathrm{KB}})}
	\AssumeIn{x}{[T]}
	\Say{[3]}{\Elim [T](x) \Intro \min_{\mathrm{KL}}}{\min_{\mathrm{KL}}\{ \inits{x}{n} | n \in \Nat  \} = \emptyset}
	\Conclude{[x.*]}{[2][3]}{\bot}
	\Derive{[3]}{\Intro \TYPE{Emptyset}}{[T] = \emptyset}
	\Conclude{[2.*]}{\Intro \WF(T)[3]}{\WF(T)}
	\Derive{[2]}{\Intro \Imply}
	{   
		\WO(T,\le_{\mathrm{KB}})
		\Imply
		\WF(T)
	}
	\Conclude{[*]}{\Intro(\iff)[1][2]}
	{
		\WF(T)
		\iff
		\WO(T,\le_{\mathrm{KB}})
	}
	\EndProof
}
\newpage
\subsection{Polish Topology}
\subsubsection{Definition and examples}
\Page{
	\DeclareType{CompletelyMetrizable}{?\TOP}
	\DefineType{X}{CompletelyMetrizable}
	{
		\exists d : \TYPE{Metric}(X) :
		\TYPE{Complete}(X,d)
	}
	\\
	\Conclude{\TYPE{Polish}}{\TYPE{Separable} \And \TYPE{CompletelyMetrizable}}{\Type}
	\\
	\Theorem{RealSpacesArePolish}
	{
		\forall n \in [1,\ldots,\omega]_\ORD \. 
		\Polish(\Reals^n )
	}
	\NoProof
	\\
	\Theorem{ComplexSpacesArePolish}
	{
		\forall n \in [1,\ldots,\omega]_\ORD \. 
		\Polish(\Complex^n )
	}
	\NoProof
	\\
	\Theorem{IntervalsArePolish}
	{
		\forall n \in [1,\ldots,\omega]_\ORD \. 
		\Polish(I^n )
	}
	\NoProof
	\\
	\Theorem{TorusIsPolish}
	{
		\forall n \in [1,\ldots,\omega]_\ORD \. 
		\Polish(\mathbb{T}^n )
	}
	\NoProof	
	\\
	\Theorem{DiscreteCountableIsPolish}
	{
		\forall A : \TYPE{Countable} \.
		\Polish(A)
	}
	\NoProof
	\\
	\Theorem{DiscreteInfiniteProductIsPolish}
	{
		\forall A : \TYPE{Countable} \.
		\Polish(A^\Nat)
	}
	\NoProof
	\\
	\DeclareFunc{spaceOfCantor}{\Polish}
	\DefineNamedFunc{spaceOfCantor}{}{\C}{2^\Nat}
	\\
	\DeclareFunc{spaceOfBair}{\Polish}
	\DefineNamedFunc{spaceOfBair}{}{\B}{\Nat^\Nat}
}
\newpage
\subsubsection{Extension of continuous functions }
\Page{
	\DeclareFunc{oscilationAt}
	{
		\prod_{X \in \TOP}
		\prod_{Y \in \MS}
		\prod_{A \subset X}
		(A \to Y) \to X \to \hat \Reals
	}
	\DefineNamedFunc{oscilationAt}{f,x}{\osc_f(x)}
	{
		\inf \Big\{ \diam f(U \cap A)  | U \in \U(x) \Big\}
	}
	\\
	\DeclareType{ContinuityPoint}
	{
		\prod X \in \TOP \.
		\prod Y \in \MS \. 
		(X \to Y) \to ?X
	}
	\DefineNamedType{x}{ContinuityPoint}{\Lambda f : X \to Y \. x \in \C_f}
	{
		\Lambda f : X \to Y \.  \osc_f(x) = 0
	}
	\\
	\Theorem{ContinuityPointsAreGDelta}
	{
		\forall X \in \TOP \.
		\forall Y \in \MS \.
		\C_f \in G_\delta(X)
	}
	\NoProof
	\\
	\Theorem{ClosedSetsAreGDelta}
	{
		\forall X : \TYPE{Metrizable} \.
		\forall F : \TYPE{Closed}(X) \.
		F \in G_\delta(X)
	}
	\NoProof
	\\
	\Theorem{KuratowskiExtensionTHM}
	{
		\forall X : \TYPE{Metrizable} \.
		\forall Y : \TYPE{CompletelyMetrizable} \.
		\forall A \subset X \. \NewLine \. 
		\forall  A \Arrow{f} Y : \TOP \. 
		\exists G \in G_\delta(X) \.
		A \subset G \subset \overline{A}
		\And
		\exists G \Arrow{F} Y : \TOP  \.
		F_{|A} = f
	}
	\Say{G}{\C_f \cap \overline{A}}{G_\delta(X)}
	\AssumeIn{x}{G}
	\Say{[1]}{\Elim \C_f(G)(x)}
	{
		\forall a : \Nat \to A \.
		\lim_{n \to \infty} a_n = x
		\Imply
		\TYPE{Cauchy}(Y,f(a))
	}
	\Say{\Big(a,[2]\Big)}
	{
	  \Elim \overline{A}(G)(x)
	}
	{
		\sum a : \Nat \to A \. x = \lim_{n \to \infty} a_n
	}
	\Conclude{F(x)}{\lim_{n \to \infty} f(a_n)}{Y}
	\Derive{F}{\Intro(\to)}{G \to Y}
	\Say{[1]}{\Elim F \Elim \C_f}{F \in \TOP(G,Y)}
	\Conclude{[*]}{\Elim F}{F_{|A} = f}
	\EndProof
}
\Page{
	\Theorem{LavrentievTHM}
	{
		\forall X,Y : \TYPE{CompletelyMetrizable} \.
		\forall A \subset X \.
		\forall B \subset Y \.
		\forall A \ToIso{f} B : \TOP \. \NewLine \.
		\exists G \in G_\delta(X) :
		\exists H \in G_\delta(Y) :
		\exists  G \ToIso{F} H : \TOP \.
		A \subset G \And B \subset H \And F_{|A} = f
	}
	\Say{\Big(A',F,[1]\Big)}
	{
		\THM{KuratowskiExtenstionTHM}(X,Y,A,f)
	}
	{
		\sum A' \in G_\delta(X) \. 
		\sum A' \Arrow{F'} Y : \TOP \.
		\NewLine \.
		F_{|A} = f \And A \subset A' \subset \overline{A}
	}
	\Say{\Big(B',F',[1]\Big)}
	{
		\THM{KuratowskiExtenstionTHM}(Y,X,B,f^{-1})
	}
	{
		\sum B' \in G_\delta(Y) \. 
		\sum B' \Arrow{F'} X : \TOP \.
		\NewLine \.
		F'_{|B} = f^{-1} \And B \subset B' \subset \overline{B}
	}
	\Say{Z}{F \cap \FUNC{swap}\; F'  }{?(X \times Y)}
	\Say{[3]}{\Elim Z [1][2]}{ f \subset Z \subset A' \times B'}
	\Say{G}{\pi_X(Z)}{?X}
	\Say{H}{\pi_Y(Z)}{?Y}
	\Say{[4]}{\Elim G [3]}{ A \subset G \subset A'}
	\Say{[5]}{\Elim H [3]}{ B \subset H \subset B'}
	\Say{[6]}{\Elim \TOP(A',Y,F) \Elim \overline{A} \Elim \TOP(X,B',F')}
	{
		\forall x \in G \. F'(F(x)) = x
	}
	\Say{[7]}{\Elim \TOP(X,B',F)\Elim \overline{B}\Elim(X,B',F') }
	{
		\forall y \in H \. F(F'(y)) = y
	}
	\Say{h}{F_{|G}}{G \to H}
	\Say{[8]}{\Elim h [7][6] \Elim F \Elim F'}{G \ToIso{h} H : \TOP}
	\Say{[9]}{\THM{GDeltaPreimage}(F \times \id,B')}{G =  (F \times \id)^{-1}(B') \in G_\delta(X)}
	\Conclude{[*]}{\THM{GDeltaPreimage}(\id \times F',A')}{H = (F \times \id)^{-1}(A') \in G_\delta(Y)}
	\EndProof
	\\
	\Theorem{OneSetLavrentievTHM}
	{
		\forall X : \TYPE{CompletelyMetrizable} \.
		\forall A \subset X \.
		\forall A \ToIso{f} A : \TOP \. \NewLine \.
		\exists G \in G_\delta(X) :
		\exists  G \ToIso{F} G : \TOP \.
		A \subset G \And \And F_{|A} = f
	}
	\Say{(G,H,F,[1])}
	{
		\THM{LavrentevTHM}\Big(X,X,A,A,f\Big)
	}
	{
		\sum G,H \in G_\delta(X) \.
		\sum F : G \ToIso{\TOP} H \. 
		A \subset G \And A \subset H
	}
	\Say{G_0'}{G}{G_\delta(X)}
	\Say{[2.0]}{[1]\Elim G_1}{A \subset G_1}
	\Assume{n}{\Nat}
	\SayIn{G_n'}
	{
		G_{n-1}' \cap F(G_{n-1}') \cap F^{-1}(G_{n-1}')
	}
	{
		G_\delta(X)
	}
	\Conclude{[2.n]}{[2.n-1][1]}
	{
		A \subset G_n'
	}
	\Derive{\Big(G',[3]\Big)}{\Intro\Act{\sum}}{ \sum G' : \Int_+ \to G_\delta(X) \. \forall n \in \Nat \. A \subset G_n' }
	\Conclude{H}{\bigcup^\infty_{n=1} G_n'}
	{   
		G_\delta(X)
	}
	\Say{[4]}{\THM{IntersectionSubset}[2]}{A \subset H}
	\AssumeIn{x}{H}
	\Say{\Big( [5] \Big)}{\Elim \TYPE{Intersection} \Elim H \Elim x}
	{
		\forall  n \in \Nat \. x \in G_n'
	}
	\Say{[6]}{\Elim G_n' [5]}{\forall n \in \Nat \. F(x) \in G_n' \And \exists y \in G_{n-1}' : x = F(y) }
	\Conclude{[x.*]}{\Elim H \Elim \TYPE{Intersection}  [6]}{ F(x) \in H \And \exists y \in H : x = F(y) }
	\Derive{[5]}{\Intro(\forall)}{\forall x \in H \. F(x) \in H \And \exists y \in H : x = F(y)}
	\Say{\Big([6]\Big)}{\Intro \FUNC{image}[5]}{ F(H) = H}
	\Say{F'}{F_{|H}}{\Aut_\TOP(H)}
	\EndProof
}
\newpage
\subsubsection{Subsets of Polish spaces}
\Page{
	\Theorem{CompleteSubsetIsGDelta}
	{
		\forall X : \TYPE{Metrizable} \.
		\forall Y \subset X \.
		\TYPE{CompletelyMetizable}(Y) \Imply
		Y \in G_\delta(X)
	}
	\Say{[1]}{\Elim \CAT( \TOP,Y)}{{\id}_Y \in \Aut_{\TOP}(Y)}
	\Say{\Big(G,F,[2]\Big)}{\THM{KuraroveskyExtensionTHM}(X,Y,Y,\id)}
	{
		\NewLine : 
		\sum G \in G_\delta(X) \. 
		\sum F \in \Aut_\TOP(G) \.
		F_{|Y} = \id \And  Y \subset G \subset \overline{Y}
	}
	\Say{[3]}{\THM{CompleteDenseExtension}(G,Y,F)[2]}{F = {\id}_G}
	\Conclude{[*]}{\Elim \id [3]}{G=Y}
	\EndProof
	\\
	\Theorem{GDeltaSubsetIsComplete}
	{
		\forall X  : \TYPE{CompletelyMetrizable} \.
		\forall Y \in G_\delta(X) \.
		\TYPE{CompletelyMetrizable}(Y)
	}
	\Say{(U,[1])}
	{
		\Elim G_\delta(X)
	}
	{
		\sum U : \Nat \to \T(X) \. 
		Y = \bigcap^\infty_{n=1} U_n
	}
	\Say{F}{U^\c}{\Nat \to \TYPE{Closed}(X)}
	\Say{\Big(d,[2]\Big)}{\Elim \TYPE{Metrizable}(X)}
	{
		\sum d : \TYPE{Metric}(X) \.
		(X,d) \cong_\TOP X
	}
	\Say{d'}{ 
		\Lambda x,y \in Y \. 
		d(x,y) + \sum^\infty_{n=0} \min \left\{ 2^{-n} ,  \left| \frac{1}{d(x,F_n)} - \frac{1}{d(y,F_n)} \right|\right\} 
	}
	{
		\TYPE{Metric}(Y)
	}
	\Say{[3]}{\Elim d'}{(Y,d') \cong_\TOP Y }
	\Assume{y}{\TYPE{Cauchy}(Y,d')}
	\Say{[4]}{[3]\Elim d' (y)}{\TYPE{Cauchy}\Big((X,d),y\Big)}
	\Say{\Big(L,[5]\Big)}{\Elim \TYPE{Complete}(X,d)}
	{
		\sum L \in X \. L = \lim_{n \to \infty} y_n
	}
	\Say{[6]}{\Elim d' [5]}
	{
		\forall n \in \Nat \.
		\lim_{i,j \to \infty} \left| \frac{1}{d(y_i,F_n)} - \frac{1}{d(y_j,F_n)}  \right| = 0
	}
	\Say{\Big(r,[7]\Big)}{\Elim \TYPE{Complete}[6]}
	{
		\sum r : \Nat \to \Reals \.
		\forall n \in \Nat 
		\lim_{i \to \infty} \frac{1}{d(y_i,F_n)} = r_n
	}
	\Say{[8]}{\THM{ReciprocalLimit}[7]\Elim \TYPE{Metric}(X,d)}
	{
		\forall n \in \Nat \. \lim_{i \to \infty} d(y_i,F_n) \in \Reals_{++}
	}
	\Conclude{[y.*]}{\Elim F_n [1][8][5]}{ L \in Y }
	\Derive{[*]}{\Intro \TYPE{Complete}}{\TYPE{Complete}(Y,d')}
	\EndProof
	\\
	\Theorem{PolishSubset}
	{
		\forall X : \Polish \.
		\forall Y \subset X \.
		Y \in G_\delta(X) \iff \Polish(Y)
	}
	\NoProof
}
\newpage
\subsubsection{Compacts and trees}
\Page{
	\Theorem{PolishCompact}
	{
		\forall X : \TYPE{Compact} \And \TYPE{Metrizable} \.
		\Polish(X)
	}
	\NoProof
	\\
	\DeclareType{FiniteSplitting}{\prod_{A \in \SET}? \Tree(A)}
	\DefineType{T}{FiniteSplitting}
	{
		\forall t \in T \. 
		\Big|\{a \in A : ta \in T \}\Big| < \infty
	}
	\\
	\Theorem{FiniteSplittingIffCompact}
	{
		\forall A \in \Set \.
		\forall T : \TYPE{Pruned}(A) \.
		\TYPE{Compact}[T] \iff \TYPE{FiniteSplitting}(T)
	}
	\Assume{[1]}{\TYPE{Compact}[T]}
	\AssumeIn{t}{T}
	\Say{\mathcal{O}}{\{ N_s | s \in T : \len s = 1 + \len t \}}{\TYPE{OpenCover}[T]}
	\Say{\Big(\mathcal{O}',[2]\Big)}
	{
		\Elim \TYPE{Compact}[T](\mathcal{O})
	}
	{
		\sum \mathcal{O}' : \TYPE{Subcover}\Big([T],\mathcal{O}\Big) \. |\mathcal{O}'| < \infty
	}
	\Conclude{[t.*]}{\THM{InjectiveCodomainCardinalityBound}\Elim \TYPE{Subcover}\big([T],\O,\O') \Elim \TYPE{Pruned}(T) \Elim \O \Elim }
	{
		\NewLine : 
		| \{ a \in A : ta \in T \} | \le
		|\{ s \in T :  \len s = 1 + \len t  \}| = 
		|\O'| 
		< \infty 
	}
	\DeriveConclude{[1.*]}{ \Intro \TYPE{FiniteSplitting}  }
	{
		\TYPE{FiniteSplitting}(T)
	}
	\Derive{[1]}{\Intro \Imply}
	{
		\Compact[T]
		\Imply
		\TYPE{FiniteSplitting}(T)
	}
	\Assume{[2]}{\TYPE{FiniteSplitting}}
	\Say{[3]}{\Elim [T]\Elim \FUNC{productTopology}\Intro \TYPE{Closed}}{\TYPE{Closed}\Big(A^\Nat,[T]\Big)}
	\Say{[4]}{\THM{CountableDiscreteProductIsComplete}(A)}
	{
		\TYPE{Complete}(A^\Nat,d)
	}
	\Say{[5]}{\Elim \TYPE{FinitieSplitting} \Elim d \Intro \TYPE{TotallyBounded}}
	{
		\TYPE{TotallyBounded}\Big(  [T], d \Big)
	}
	\Say{[6]}{\THM{ClosedCompleteSubset}[3][4]}{\TYPE{Complete}\Big( [T],d\Big)}
	\Conclude{[2.*]}{\THM{TotallyBoundedCompleteIsCompact}[5][6]}
	{
		\TYPE{Compact}[T]
	}
	\Derive{[2]}{\Intro \Imply}
	{
		\TYPE{FiniteSplitting}(T)
		\Imply
		\TYPE{Compact}[T]
	}
	\Conclude{[*]}
	{
		\Intro \iff [1][2]
	}
	{
		\TYPE{Compact}[T]
		\iff
		\TYPE{FiniteSplitting}(T)
	}
	\EndProof
}\Page{
	\Theorem{BairSpaceIsNotSigmaCompact}{\neg \TYPE{\sigma\hyph}\Compact(\B)}
	\Assume{[1]}{\SCompact(\B)}
	\Say{\Big(K,[2]\Big)}{\Elim \SCompact(\B)}
	{
		\sum K : \Nat \to \Compacts(\B) \. \B = \bigcup^\infty_{n=1} K_n
	}
	\Say{[3]}{\THM{CompactIsClosed}(\B,K)}
	{
		\forall n \in \Nat \. \TYPE{Closed}(\B,K_n)
	}
	\Say{\Big(T,[4]\Big)}
	{
		\THM{BodyBijection}[2]
	}
	{
		\sum T : \Nat \to \TYPE{Pruned}(\Nat) \.
		\forall n \in \Nat \. K_n = [T_n]
	}
	\Say{[5]}{\THM{FiniteSplittingIffCompact}[4]\Elim K}
	{
		\forall n \in \Nat \. 
		\TYPE{FiniteSplitting}(\Nat,T_n)
	}
	\Say{k}{\Lambda n,m \in \Nat \. 1 + \max\{ t_m | t \in T_n \} }{\Nat \to \Nat \to \Nat}
	\Say{[6]}{\Elim k \Elim \TYPE{FiniteSplitting}(T)[4]}{\forall n \in \Nat \. \forall x \in K_n \. k_n > x}
	\SayIn{\Delta}{\Lambda n \in \Nat \. k_{n,n}}{\B}
	\Say{[7]}{\THM{TrichtomyPrinciple}[6](\Delta)}
	{
		\forall n \in \Nat \. \Delta \not \in K_n
	}
	\Say{[8]}{\Elim \FUNC{union}[2][7]}{\Delta \not \in \B }
	\Conclude{[9]}{\Elim \B(\Delta)[8]}{\bot}
	\DeriveConclude{[10]}{\Elim(\bot)}{\neg \SCompact(\B)}
	\EndProof
	\\
	\Theorem{K\ddot{o}nigsLemma}
	{
		\forall T : \TYPE{FiniteSplitting}(A) \.
		[T] \neq \emptyset 
		\iff
		|T| = \infty
	}
	\NoProof
}
\newpage
\subsubsection{Universality of the Hilbert's Cube}
\Page{
	\Theorem{UniversalityOfTheHilbertsCube}
	{
		\forall X : \TYPE{Separable} \And \TYPE{Metrizable} \.
		\exists  \TYPE{TopologicalEmbedding}(X,I^\Nat) 
	}
	\Say{\big(d,[1]\big)}
	{
		\THM{BoundedRemtrization}\Elim \TYPE{CompletelyMetrizable}(X)
	}
	{
		\sum d : \TYPE{Metric}(X) \.
		d < 1 \And  \TYPE{Complete}(X,d)
	}
	\Say{D,[2]}{\Elim \TYPE{Separable}(X)}
	{
		\sum D : \TYPE{Dense}(X) \.
		|D| \le \aleph_0
	}
	\Say{\delta}{\FUNC{enumerate}(D)}{\Nat \ToBij D}
	\SayIn{f}{\Lambda x \in X \. \Lambda n \in \Nat \. d(x,\delta_n)}{\TOP(X,I^\Nat)}
	\AssumeIn{x,y}{X}
	\Assume{[3]}{f(x) = f(y)}
	\Say{\Big(a,[4]\Big)}{\Elim \TYPE{Dense}(X,D)(x)}
	{
		\sum a : \Nat \to D \. x = \lim_{n \to \infty} a_n
	}
	\Say{[5]}{\Elim f [3][4]\THM{ConvergenceInMetricSpace}}{y = \lim_{n \to \infty} a_n }
	\Conclude{\Big[(x,y).*\Big]}{\THM{T1HasUniqueLimit}[4][5]}{x = y}
	\Derive{[3]}{\Intro \TYPE{Injective}}{\TYPE{Injective}(X,I^\Nat,f)}
	\Assume{K}{\TYPE{Closed}(X)}
	\AssumeIn{L}{\boundary f(K) \cap f(X)}
	\Say{\Big(y,[4]\Big)}{\Elim \boundary f(K)}
	{
		\sum y : \Nat \to f(K) \. L = \lim_{n \to \infty} y_n
	}
	\Say{\Big(A ,[5]\Big)}{\Elim \FUNC{image}(f)(L)}
	{
		\sum A \in X \. L = f(X)
	}
	\Say{\Big( x,[6]\Big)}{\Elim \FUNC{image}(f)(y)}
	{
		\sum x : \Nat \to K \. 
		\forall n \in \Nat \. 
		y_n = f(n)
	}
	\Say{\Big(a,[7]\Big)}{\Elim \TYPE{Dense}(X,D)(A)}
	{
		\sum a : \Nat \to D \. A = \lim_{n \to \infty} a_n
	}
	\AssumeIn{\varepsilon}{\Reals_{++}}
	\Say{\Big(n,[8])}{\Elim \TYPE{Limit}[4]\left( \frac{\varepsilon}{3}\right)}
	{
		\sum n \in \Nat \.
		d(a_n,A) = L_n  \le \frac{\varepsilon}{3}
	}
	\Say{\Big(m,[9]\Big)}{\Elim \delta [5]}
	{
		\sum m \in \Nat \. \delta_m = a_n 
	}
	\Say{\Big(N,[10]\Big)}
	{
		\Elim \TYPE{Limit} [4]
	}
	{
		\forall k \in \Nat \. k \ge N \Imply \Big| L_n - y_{k,m} \Big| \le \frac{\varepsilon}{3}
	}
	\Say{[11]}{[10][8][5]}{ \forall k \in \Nat \. k \ge N \Imply y_{k,m} \le \frac{2\varepsilon}{3}}
	\Conclude{[\varepsilon.*]}{
		\Lambda k \in \Nat \. 
		\Lambda [0] : k \ge N \. 
		\THM{TriangleIneq}(X,d)(x_k,A,a_n)
		[9][5][6]
		\Elim f
		[8][11][0]
	}
	{
		\forall k \in \Nat \. k \ge N \Imply
		\NewLine \Imply
		d(x_k,A) \le d(x_k,a_n) + d(a_n,A) = 
		d(x_k,\delta_m) + d(\delta_m,A) =
		L_m  + y_{k,m} \le \varepsilon
	}
	\Derive{[8]}{\THM{MetricConvergenceCriterion}}{ A = \lim_{n \to \infty} x_n  }
	\Say{[9]}{\THM{ClosedSetHasLimits}[8]}{A \in K}
	\Conclude{[L.*]}{\Intro \FUNC{Image}(A)}{L \in f(K)}
	\DeriveConclude{[11]}{\THM{ClosedByBoundary}}{\TYPE{Closed}(I^\Nat,f(K))}
	\DeriveConclude{[*]}{\Intro \TYPE{TopologicalEmbedding}}
	{
		\TYPE{TopologicalEmbedding}(X,I^\Nat,f)
	}
	\EndProof
	\\
	\Theorem{PolishSpacesAreHilbertCubesSubspaces}
	{
		\forall X : \Polish \.
		\exists A : G_\delta(I^\Nat) :
		X \cong_\TOP A
	}
	\NoProof
}
\Page{
	\Theorem{CantorSetIsACompactificationOfBairSet}
	{
		\TYPE{Compactification}(\B,\C)
	}
	\Say{(a,b)}{\FUNC{enumerate}(\Nat \times \Nat)}{\Nat \ToBij \Nat \times \Nat}
	\AssumeIn{n}{\B}
	\Say{\beta}{\Lambda k \in \Nat \. \FUNC{binaryExpansion}(n_k)}
	{
		\Nat \to \Nat \to \{0,1\}
	}
	\Conclude{f(n)}{\Lambda k \in \Nat \. \beta_{a_k,b_k}}
	{
		\C
	}
	\Derive{\B \Arrow{f} \C}{\Intro \TOP}{\TOP}
	\Say{[1]}{\Elim f}{f(\B) =  \bigg\{ b \in \C : \Big| b^{-1}(0) \Big| = \infty  \bigg\}}
	\Conclude{[*]}{\Intro \TYPE{Dense}[1]}{\TYPE{Dense}\Big(\C,f(\B)\Big)}
	\EndProof
	\\
	\Theorem{UnitIntervalIsACompactificationOfBairSet}
	{
		\TYPE{Compactification}(\B,I)
	}
	\Say{(a,b)}{\FUNC{enumerate}(\Nat \times \Nat)}{\Nat \ToBij \Nat \times \Nat}
	\AssumeIn{n}{\B}
	\Say{\beta}{\Lambda k \in \Nat \. \FUNC{binaryExpansion}(n_k)}
	{
		\Nat \to \Nat \to \{0,1\}
	}
	\Conclude{f(n)}{\sum^\infty_{k=1} 2^{-k}\beta_{a_k,b_k}}
	{
		I
	}
	\Derive{\B \Arrow{f} I}{\Intro \TOP}{\TOP}
	\Say{[1]}{\Elim f}{\Rats_2 \cap I \subset f(\B) }
	\Conclude{[*]}{\Intro \TYPE{Dense}[1]}{\TYPE{Dense}\Big(I,f(\B)\Big)}
	\EndProof
	\\
	\Theorem{PolishSpacesAsClosedSubset}
	{
		\forall X : \Polish \.
		\exists A : \TYPE{Closed}(X) \.
		A \cong_\TOP X
	}
	\Say{(G,[1])}
	{
		\THM{PolishSpacesAreSubsetsOfHilbertCube}
	}
	{
		\sum G \in G_\delta(I^\Nat) \. G \cong_{\TOP} X
	}
	\Say{(U,[2])}
	{
		\Elim G_\delta(G)
	}
	{
		\sum U : \Nat \to \T(I^\Nat) \.
		G = \bigcap^\infty_{n=1} U_n
	}
	\Say{F}{U^\c}{\Nat \to \TYPE{Closed}(I^\Nat) }
	\Say{f}
	{
		\Lambda x \in X \.
		\Lambda n \in \Nat \.
		\If n \; \LOGIC{is} \; \TYPE{Even} \Then
		x_{n/2} \Else 
		\frac{1}{d(x,F_{(n+1)/2})}
	}
	{
		X \Arrow{\TOP} \Reals^\Nat
	}
	\Say{[3]}{\Elim f \Intro \TYPE{Injective}}
	{
		\TYPE{Injective}(X,\Reals^\Nat,f)
	}
	\Assume{L}
	{
		\TYPE{Limit}\Big( f(X)\Big)
	}
	\Say{\Big(y,[4]\Big)}{\Elim \TYPE{Limit}(f(X),L)}
	{
		\sum y : \Nat \to  f(X) \.
		L = \lim_{n \to \infty} y_n
	}
	\Say{\Big( x, [5]\Big)}
	{
		\Elim \FUNC{image} (y)
	}
	{
		\sum x : \Nat \to X \. y = f(x)
	}
	\Say{[6]}{\Elim f [5][4]}{\TYPE{Convergent}(I^\Nat,x)}
	\SayIn{A}{\lim_{n \to \infty} x_n}{I^\Nat}
	\Say{[7]}{\Elim A \Elim f [5][4]}
	{
		\forall n \in \Nat \. 
		d(A,F_n) \neq 0
	}
	\Say{[8]}{[2][7]}{A \in G}
	\Conclude{[L.*]}{\Intro \FUNC{image}[8][4]\THM{ContinuousImage}}
	{
		L = f(A)  \in f(G)
	}
	\DeriveConclude{[*]}{\THM{ClosedByLimits}}{\TYPE{Closed}(\Reals^\Nat,f(X))}
	\EndProof
}
\newpage
\subsubsection{Universality of Cantor's Set}
\Page{
	\Theorem{CantorsSetUniversality}
	{
		\forall X : \Polish \And \Compact \.
		\exists f \in \TOP(\C,X) \.
		X = f(\C)
	}
	\Say{\Big(A,[1]\Big)}
	{
		\THM{PolishSpacesAreHilbertSpaceSubsets}(X)
	}
	{
		\sum A : G_\delta(I^\Nat) \.
		A \cong_\TOP X
	}
	\Say{A \ToIso{\varphi} X}
	{
		\Elim \TYPE{Isomorphic}(A,X)[1] 
	}
	{
		\TOP
	}
	\SayIn{g}{\Lambda b \in \C \. \sum^\infty_{n=1} b_n 2^{-n}}
	{
		\TOP(\C,I)
	}
	\SayIn{[2]}{\THM{RealsBinaryExpansion}\Elim \psi}{\TYPE{Surjective}(\C,I,g)}
	\Say{\C^\Nat \ToIso{\psi} \C}{\THM{CantorSetPowerHomeo}(\Nat)}
	{
		\TOP
	}
	\Say{h}{\psi g^\Nat}{\TOP(\C,I^\Nat)}
	\Say{[3]}{\THM{SurjectiveComposition}\Elim h[2]\TYPE{Homeomorphism}(\psi)}{\TYPE{Surjective}(\C,I^\Nat,h)}
	\Say{[4]}{\THM{CompactImage}[1]}{\Compacts(I^\Nat,A)}
	\Say{B}{h^{-1}(A)}{\TYPE{Compact}(\C)}
	\Say{[5]}{\Elim B [1][3]}{B \neq \emptyset}
	\Say{r}{\THM{InBairlikespaceAllClosedAreRetracts}(\C,\C,B)[5]}{\TYPE{Retraction}(\C,B)}
	\SayIn{f}{rh\varphi}{\TOP(X,\C)}
	\Conclude{[*]}{
		\THM{SurjectiveCompositon}
		\Elim f 
		\Elim \TYPE{Retraction}(r)[4]
		\Elim \TYPE{Homeomorphism}(\varphi)
		\Elim \TYPE{Surjection}
	}
	{
		f(\C) = X
	}
	\EndProof
}
\newpage
\subsubsection{More Examples}
\Page{
	\Theorem{ContinuousFunctionsArePolish}
	{
		\forall X : \Compact \And \Polish \.
		\forall Y : \Polish \.
		\Polish\Big( C(X,Y) \Big)
	}
	\Assume{f}{\TYPE{Cauchy}\;C(X,Y)}
	\Say{[1]}{\Elim \TYPE{Cauchy}\Big(C(X,Y),f\Big)}
	{
		\forall x \in X \. \lim_{n,m \to \infty}d(f_n(x),f_m(x)) = 0
	}
	\Say{[2]}{\Elim \TYPE{Complete}(Y,d)[1]}
	{
		\forall x \in X \. \TYPE{Convergent}\Big(Y,f(x)\Big)
	}
	\Say{\varphi}{\Lambda x \in X \. \lim_{n \to \infty} f_n(x)}{X \to Y}
	\Say{[3]}{\Elim \TYPE{Cauchy}\Big( C(X,Y),f\Big)\Elim d_u}
	{
		\lim_{n,m \to \infty} \sup_{x \in X} d(f_n(x),f_m(x)) = 0
	}
	\Say{}{}
	{
		d( \varphi(x_m), \varphi(L) ) \le 
		d( \varphi(x_m), f_n(x_m) ) + d(f_n(x_m),f_n(L) )  + d(f_n(L),\varphi(L))
	}
	\AssumeIn{\varepsilon}{\Reals_{++}}
	\Say{\Big(N,[4]\Big)}{\Elim \TYPE{Limit}[3]\left( \frac{\varepsilon}{2} \right) }
	{
		\sum N \in \Nat \. \forall n,m \in \Nat \. n,m \ge N \Imply \sup_{x \in X} d(f_n(x), f_m(x)) < \frac{\varepsilon}{2}
	}
	\AssumeIn{n}{\Nat}
	\Assume{[5]}{n \ge N}
	\AssumeIn{x}{X}
	\Say{[6]}{\Elim \varphi(x)}{\lim_{n \to \infty} f_n(x) = \varphi(x)}
	\Say{\Big(M,[7]\Big)}{\Elim \TYPE{Limit} [6]\left( \frac{\varepsilon}{2} \right)}
	{
		\sum M \in \Nat \. \forall m \in \Nat \. m \ge M \Imply d(f_m(x),\varphi(x)) < \frac{\varepsilon}{2} 
	}
	\SayIn{m}{\max(M,N)}{\Nat}
	\AssumeIn{x}{X}
	\Conclude{[x.*]}{\THM{TriablgeIneq}(Y)\Big(f_n(x),\varphi(x),f_m\Big)[7][4]}
	{
	      d\Big( f_n(x), \varphi(x) \Big) \le
	      d\Big( f_n(x), f_m(x) \Big) + d\Big( f_m(x), \varphi(x) \Big) <
	      \varepsilon
	}
	\DeriveConclude{[\varepsilon.*]}{\Intro d_u}{\sup_{x \in X} d(f_n,\varphi) < \varepsilon}
	\Derive{[4]}{\Intro \forall}{\forall \varepsilon \in \Reals_{++} \. \sup_{x \in X} d(f_n,\varphi)<\varepsilon}
	\AssumeIn{L}{X}
	\Assume{x}{\Nat \to X}
	\Assume{[5]}{\lim_{n \to \infty} x_n = L}
	\Say{[6]}{\Lambda n \in \Nat \. \TYPE{ContinuousByLimits}(f_n,x,L)}
	{
		\forall n \in \Nat \. \lim_{m \to \infty} f_n(x_m) = f_n(L)
	}
	\AssumeIn{\varepsilon}{\Reals_{++}}
	\Say{\Big(n,[7]\Big)}{\Elim \TYPE{Limit}\left(\frac{\varepsilon}{3}\right)}
	{
		\sum^\infty_{n=1} \sup_{x \in X} d( f_n(x), \varphi(x) \Big) < \frac{\varepsilon}{3}
	}
	\Say{\Big(N,[8]\Big)}{\Elim \TYPE{Limit}[6](n)}
	{
		\sum^\infty_{N=1} \forall m \in \Nat \. 
		m \ge N
		\Imply
		d\Big( f_n(x_m), f_n(L) \Big) < \frac{\varepsilon}{3}
	}
	\AssumeIn{m}{\Nat}
	\Assume{[9]}{m \ge N}
	\Conclude{[\varepsilon.*]}{\THM{TriangleIneq}(Y,)[7]^2[8]}
	{
		d( \varphi(x_n),\varphi(L) ) \le
		d( \varphi(x_n), f(x_n) ) + d( f(x_n), f(L) ) + d( f(L),\varphi(L)) <
		\varepsilon
	}
	\DeriveConclude{[L.*]}{\Intro \TYPE{Limit}}{\lim_{n \to \infty} \varphi(x_n) = \varphi(L)}
	\Derive{[7]}{\THM{ContinuousByLimits}}{\varphi \in C(X,Y)}
	\Conclude{[f.*]}{\Intro \TYPE{Limit}[6]}{\lim_{n \to \infty} f_n = \varphi}
	\Derive{[1]}{\Intro \TYPE{Complete}}{\TYPE{Complete}\Big(C(X,Y),d_u\Big)}	
}
\Page{
	\Say{C}{
		\Lambda n,m \in \Nat \. 
		\Big\{f \in C(X,Y) : \forall x,y \in X \. d(x,y) < \frac{1}{m} \Imply d(f(x),f(y)) < \frac{1}{n} \Big\} 
	}{ \Nat \to \Nat \to ?C(X,Y)  }
	\Say{\Big(A,[1]\Big)}
	{
		\THM{CompactHasEpsilonNets}(A)
	}
	{
		\sum A : \Nat \to \TYPE{Finite}(X) \.
		\forall n \in \Nat \.
		\forall x \in X \.
		\exists a \in A :
		d(a,x) < \frac{1}{n}
	}
	\Say{D}{\FUNC{countableRefinementAt}(C,X)}
	{
		\prod^\infty_{n,m=1} \TYPE{Countable}(C_{n,m})
	}
	\Say{[2]}{\Elim D}
	{
		\forall n,m \in \Nat \.
		\forall f \in C_{n,m} \.
		\forall \varepsilon \in \Reals_{++} \.
		\exists g \in D_{n,m} :
		\forall a \in A_n \.
		d( f(a),g(a) ) < \varepsilon
	}
	\Say{E}{\bigcup^\infty_{n,m=1} D_{n,m}}{\TYPE{Countable}\Big(C(X,Y)\Big)}
	\AssumeIn{f}{C(X,Y)}
	\Say{\Big(m,[3]\Big)}{\THM{ArchemedeanProperty}\left(\frac{3}{\varepsilon}\right)}
	{
		\sum m \in \Nat \. m > \frac{3}{\varepsilon}
	}
	\Say{\Big(n,[4]\Big)}{\Elim \TYPE{UniformlyContinuous}(f)\Elim C}
	{
		\sum n \in \Nat \. f \in C_{n,m}
	}
	\Conclude{[f.*]}{\Elim C_{n,m} \Elim A_n [4][2]}
	{
		\exists g \in D_{n,m} \. d(f,g) < \varepsilon
	}
	\Derive{[2]}{\Intro \TYPE{Dense}}{\TYPE{Dense}\Big( C(X,Y), E\Big)}
	\Conclude{[*]}{\Intro \TYPE{Polish}[1][2]}{\TYPE{Polish}\Big( C(X,Y)\Big)}
	\EndProof
	\\
	\DeclareFunc{compactsWithVietorisTopology}
	{
		\TOP \to \TOP
	}
	\DefineNamedFunc{compactsWithVietorisTopology}
	{X}{\K(X)}
	{
		\bigg(
				\Compacts(X), \NewLine 
				\FUNC{fromBase}\Big\{
					\big\{ 
						K : \Compacts :
						K \subset U_0 \And 
						\forall i \in [1,\ldots,n] \.
						K \cap U_i \neq \emptyset
					\big\}
					\Big|
					n \in \Int_+,
					U : [0,\ldots,n] \to \T(X)
				\Big\}
		\bigg)
	}
	\\
	\Theorem{EmptySetIsIsolatedInVietorisTopology}
	{
		\forall X \in \TOP \.
		\TYPE{Isolated}\Big( \K(X), \emptyset_X\Big)
	}
	\Say{[1]}{\Elim \emptyset}{\forall A : \Compacts(X) \.  A \subset \emptyset \Imply A = \emptyset}
	\Say{[2]}{\Elim \K(X)[1]}{ \{\emptyset\} \in \T\Big(\K(X)\Big)  }
	\Conclude{[*]}{\Intro \TYPE{Isolated}[2]}{\TYPE{Isolated}\Big( \K(X),\emptyset_X\Big)}
	\EndProof
	\\
	\DeclareFunc{distanceOfHausdorff}
	{
		\prod X : \TYPE{BoundedMetricSpace} \.
		\TYPE{Metric}\Big( \K(X) \Big)
	}
	\DefineNamedFunc{distanceOfHausdorff}{\emptyset,\emptyset}{d_{\mathrm{H}}(\emptyset,\emptyset)}{0}
	\DefineNamedFunc{distanceOfHausdorff}{A,\emptyset}{d_{\mathrm{H}}(A,\emptyset)}{1}
	\DefineNamedFunc{distanceOfHausdorff}{\emptyset,B}{d_{\mathrm{H}}(\emptyset,B)}{1}
	\DefineNamedFunc{distanceOfHausdorff}{A,B}{d_{\mathrm{H}}(A,B)}
	{
		\max\Big( \max_{a \in A} \min_{b \in B} d(a,b), \max_{b \in B} \min_{a \in A} d(a,b)\Big)
	}
	\\
	\Theorem{VietorisTopologyIsMetrizedByHausdorffMetric}
	{
		\forall X \in \MS \. \K(X) \cong_\TOP \Big( \K(X),d_{\mathrm{H}} \Big)
	}
	\NoProof
}
\Page{
	\Theorem{PolishCompacts}
	{
		\forall X : \TYPE{Separable}\And\TYPE{Metrizable} \. 
		\TYPE{Separable} \And \TYPE{Metrizable}\Big( \K(X) \Big)
	}
	\Say{[1]}{\THM{BoundedRemetrization}(X)\THM{VietorisTopologyIsMetrizedByHausdorffMetric}}
	{
		\TYPE{Metrizable}\Big( \K(X) \Big)
	}
	\Say{\Big(D,[2]\Big)}{\Elim \TYPE{Separable}(X)}
	{
		\sum D : \TYPE{Dense}(X) \. 
		|D| \le \aleph_0
	}
	\Say{A}{\TYPE{Finite}(D)}
	{
		?\K(X)
	}
	\Say{[3]}{\THM{FiniteSetsCardinality}\Elim A}{|A| \le \aleph_0}
	\Say{[4]}{\Elim d_{\mathrm{H}}\Elim A}{\TYPE{Dense}\Big( \K(X) ,A \big)}
	\Conclude{[*]}{\Intro \TYPE{Separable}[4]}{\TYPE{Separeble}\Big( \K(X) \Big)}
	\EndProof
	\\
	\DeclareFunc{topologicalLowerLimit}{\prod_{X \in \TOP} (\Nat \to ?X) \to ?X}
	\DefineNamedFunc{topologicalLowerLimit}{A}{\Tll_{n \to \infty} A_n}
	{
		\bigg\{ x \in X : \Big|\big\{ n \in \Nat : \forall U \in \U(x) \. U \cap A_n \neq \emptyset \big\}\Big|=\infty   \bigg\}
	}
	\\
	\DeclareFunc{topologicalUpperLimit}{\prod_{X \in \TOP} (\Nat \to ?X) \to ?X}
	\DefineNamedFunc{topologicalUpperLimit}{A}{\Tul_{n \to \infty} A_n}
	{
		\bigg\{ x \in X : \Big|\big\{ n \in \Nat : \exists U \in \U(x) \. U \cap A_n = \emptyset \big\}\Big|<\infty   \bigg\}
	}
	\\
	\Theorem{TopologicalLimitsRelation}
	{
		\forall X \in \TOP \.
		\forall A : \Nat \to ?X \.
		\Tll_{n\to\infty} A_n \subset  \Tul_{n \to \infty} A_n
	}
	\NoProof
	\\
	\Theorem{TopologicalLimitsAreClosed}
	{
		\forall X \in \TOP \.
		\forall A : \Nat \to ?X \.
		\TYPE{Closed}\Big(X,\Tll_{n\to\infty} A_n \And \Tul_{n\to\infty} A_n \Big)
	}
	\NoProof
	\\
	\DeclareType{TopologicalLimit}
	{
		\prod_{X \in \TOP} \.
		(\Nat \to ?X) \to ?\TYPE{Closed}(X)
	}
	\DefineNamedType{L}{TopologicalLimit}{\Lambda A : \Nat \to ?X \. \Tl_{n\to\infty} A_n = L  }
	{
		\Lambda A : \Nat \to ?X \.  L = \Tll_{n \to \infty} A_n \And L = \Tul_{n \to \infty} A_n
	}	
	\\
	\Theorem{HausdorffConvergenceAsTopologicalLimit}
	{
		\forall X : \MS  \. 
		\forall K : \Nat \to \K(X) \.
		\forall L \in \K(X) \. \NewLine \.
		L = \lim_{n \to \infty} K_n
		\Imply
		L = \Tl_{n \to \infty} K_n
	}
	\NoProof
}\Page{
	\Theorem{CompactHausdorffConvergence}
	{
		\forall X : \Compact \And \TYPE{BoundedMetricSpace} \.
		\forall K : \Nat \to \K(X) \.
		\forall L \in \K(X) \. \NewLine \.
		L = \Tl_{n \to \infty} K_n
		\iff
		L = \lim_{n \to \infty} K_n
	}
	\NoProof
	\\
	\Theorem{PolishHausdforffisPolish}
	{
		\forall X : \Polish \.
		\Polish( \K(X) )
	}
	\Say{\Big(d,[1]\Big)}
	{
		\Elim \TYPE{CompletelyMetrizable}(X) \THM{BoundedRemetrization}
	}
	{
		\sum d : \TYPE{Metric}(X) \. 
		\NewLine \.
		\TYPE{Complete}(X,d) \And (X,d) \cong_\TOP X \And d < 1
	}
	\Assume{K}{\TYPE{Cauchy}\;\K(X)}
	\Say{L}{\Tul_{n\to \infty} K_n}{\TYPE{Closed}(X)}
	\Say{\Big(F,[2]\Big)}{\Lambda n \in \Nat \. \Elim \TYPE{TotallyBounded}}
	{
		\prod^\infty_{n=1} \prod^\infty_{m=1} 
		\sum F_{n,m} : \TYPE{Finite}(K_n) \.
		\forall x \in K_n \exists f \in F_{n,m} : d(x,f) < 2^{-m-2}
	}
	\Say{\Big(p,[3]\Big)}{\Elim \TYPE{Cauchy}\Big(\K(X),K)}
	{
		\prod^\infty_{n=1} \sum^\infty_{p_n=1} \. 
		\forall i,j \in \Nat \.
		i,j \ge p_n \Imply  d_{\mathrm{H}}(K_i,K_j) < 2^{-m-2}
	}
	\Say{J}{\Lambda n \in \Nat \. \bigcup^{p_n}_{k=n} F_{k,n}}{\Nat \to \TYPE{Finite}(X)}
	\AssumeIn{n}{\Nat}
	\AssumeIn{x}{L}
	\Say{(m,[4])}{\Elim L (x,n) }
	{
		\sum^\infty_{m=1} m \ge p_n \And \forall U \in \U(x) \. U \cap K_m \neq \emptyset
	}
	\Say{[5]}{[3](n,p_n,m)}{d_{\mathrm{H}}(K_{p_n},K_m) < 2^{-n-2}}
	\Say{\Big(u,[6]\Big)}{[4]\left(( \mathbb{B}_d\left(x,  2^{-n-2}  \right)  \right)}
	{
		\sum u \in K_m \.  d(u,x) < 2^{-n-2}
	}
	\Say{\Big(v,[7]\Big)}{\Elim d_{\mathrm{H}}[5](u)}
	{
		\sum v \in K_{p_n} \. d(u,v) < 2^{-n-2}
	}
	\Say{\Big(f,[8]\Big)}{[2](p_n,n)(v)}
	{
		\sum f \in F_{p_n,n} \.  d(v,f) < 2^{-n-2}
	}
	\Say{[9]}{\Elim J_n \Elim \FUNC{union} (f)}
	{
		f \in J_n
	}
	\Conclude{[n.*]}{\THM{TriangleIneq}(X,d)(x,u,v,f)[6,7,8]}
	{
		d(x,f) \le
		d(x,u) + d(u,v) + d(v,f) < 
		2^{-n}
	}
	\Derive{[4]}{\Intro \TYPE{TotallyBounded}}
	{
		\TYPE{TotallyBounded}(X,L)
	}
	\Say{[5]}{\THM{ClosedIsComplete}(X,L)}{\TYPE{Complete}(X)}
	\Say{[6]}{\THM{CompactIffCompleteAndTotallyBounded}[4][5]\Intro \K(X)}
	{
		L \in \K(X)
	}
}
\Page{
	\Assume{\varepsilon}{\Reals_{++}}
	\Say{\Big(N,[7]\Big)}{\Elim \TYPE{Cauchy}(K)}
	{
		\sum N \in \Nat \. 
		\forall i,j \in \Nat \.
		i,j > N 
		\Imply
		d(K_i,K_j) < \frac{\varepsilon}{2}
	}
	\Assume{n}{\Nat}
	\Assume{[8]}{n \ge N }
	\AssumeIn{x}{L}
	\Say{\Big(k,y,[9]\Big)}{\Elim L}{\sum k : \Nat \uparrow\Nat \.  \sum y : \prod^\infty_{n=1} K_{k_n} \. x = \lim_{n \to \infty} y_n}
	\Say{\Big(M,[10]\Big)}{\Elim \TYPE{Limit}(x,y)[9]\left( \frac{\varepsilon}{2} \right)}
	{
		\sum M \in \Nat \. \forall m \in \Nat \. m \ge M \Imply  k_m \ge N \And d(y_m,x) < \frac{\varepsilon}{2}
	}
	\Say{\Big(z,[11]\Big)}{[7](y_{k_M})}{\sum z \in K_n \.  d(z,y_{k_M}) < \frac{\varepsilon}{2}}
	\Conclude{[x.*]}{\THM{TrinagleIneq}(x,y_{k_M},z)[10][11]}
	{
		d(x,z) \le  d(x,y_{k_M}) + d(y_{k_M},z) < \varepsilon
	}
	\Derive{[9]}{\Intro \sup }{ \sup_{x \in L} \inf_{y \in K_n} d(x,y) < \varepsilon  }
	\AssumeIn{y}{K_n}
	\Say{\Big(k,[10]\Big)}{\Elim \TYPE{Cauchy}(K)}
	{
		\sum k : \TYPE{Increasing}(\Nat,\Nat) \. k_1 = n 
		\And \forall i,m \in \Nat \. 
		m \ge k_i \Imply  
		d( K_{k_i}, K_m ) < 2^{-i-1} \varepsilon
	}
	\SayIn{x_1}{y}{K_n}
	\AssumeIn{i}{\Nat}
	\Conclude{\Big(x_{i+1},[11]\Big)}{[10](i,k_{i+1})}
	{
		\sum x_{i+1} \in K_{k_{i+1}} \. d( x_{i+1},x_i) <  2^{-i-1}\varepsilon
	}
	\Derive{\Big(x,[11]\Big)}{\Intro\Act{\prod}}
	{
		\sum x : \prod^\infty_{i=1} K_{k_i} \. 
		\forall i \in \Nat \. d(x_{i+1},x_i) < 2^{-i-1}\varepsilon
	}
	\Say{[12]}{\Intro \TYPE{Cauchy}[11]}{\TYPE{Cauchy}(X,x)}
	\Say{[13]}{\Elim \TYPE{Complete}(X)[12]}{\TYPE{Convergent}(X,x)}
	\SayIn{z}{\lim_{n \to \infty} x_n}{X}
	\Say{[14]}{\Elim L \Elim \Tul \Elim z}{z \in L}
	\Conclude{[y.*]}{\Elim z [11]}{d(y,z) < \varepsilon}
	\Derive{[10]}{\Intro \sup }{ \sup_{x \in K_n} \inf_{y \in L} d(x,y) < \varepsilon  }
	\Conclude{[\varepsilon.*]}{\Intro d_H[9][10]}{ d(K_n,L) < \varepsilon}
	\DeriveConclude{[K.*]}{\Intro \TYPE{Limit}}
	{
		\lim_{n \to \infty} K_n = L
	}
	\DeriveConclude{[*]}{\Elim \TYPE{Complete}}{\TYPE{Complete}(\K(X))}
	\EndProof
}\Page{
	\Theorem{CompactHaudorffMetricIsCompact}{
		\forall X : \Compact \And \Polish \.
		\Compact \And \Polish\Big( \K(X) \Big)
	}
	\Say{\Big(d,[1]\Big)}
	{
		\Elim \TYPE{CompletelyMetrizable}(X) \THM{BoundedRemetrization}
	}
	{
		\sum d : \TYPE{Metric}(X) \. 
		\NewLine \.
		\TYPE{Complete}(X,d) \And (X,d) \cong_\TOP X \And d < 1
	}
	\Say{\Big(F,[2]\Big)}{\Elim \TYPE{TotallyBounded}(X)}
	{
		\sum F : \Nat \to \TYPE{Finite}(X) \. 
		\forall n \in \Nat \. 
		\forall x \in X \.
		\exists f \in F_n :
		d(x,f) < \frac{1}{n}
	}
	\Say{[3]}{\THM{PolishHausdorffPolish}(X)}{\Polish\Big(\K(X)\Big)}
	\Say{[4]}{\Elim \Polish\Big(\K(X)\big)}{ \TYPE{Complete}(\K(X),d_{H})}
	\Say{F'}{\Lambda n \in \Nat \. 2^{F'}}{\Nat \to \TYPE{Finite}\Big( \TYPE{Finite}(X) \Big)}
	\Say{[5]}{\THM{FiniteIsCompact}(F)\Intro F'}{\forall n \in \Nat \. \TYPE{Finite}\Big(\K(X), F'_n \Big)}
	\AssumeIn{n}{\Nat}
	\AssumeIn{K}{\K(X)}
	\SayIn{G}{\left\{ f \in F_n : d(x,K) < \frac{1}{n}  \right\}}{F'_n}
	\Conclude{[n.*]}{\Elim G [2]}{d(K,G)<\frac{1}{n}}
	\Derive{[6]}{\Intro \TYPE{TotallyBounded}}{\TYPE{TotallyBounded}\Big(X ,d_\H\Big)}
	\Conclude{[*]}{\THM{CompactIffCompleteAndTotallyBounded}[4][6]}{\Compact(\K(X))}
	\EndProof
}\Page{
	\Theorem{SingletonIsTopologicalEmbedding}
	{
		\forall X \in \MS \. 
		\TYPE{IsometricEmbedding}\Big(X,\K(X),\Lambda x \in X \. \{x\} \Big)
	}
	\NoProof
	\\
	\Theorem{HausdorffConvergenceByIntersection}
	{
		\forall X \in \MS \.
		\forall K : \TYPE{Decreasing}\Big( \K(X) \Big) \.
		\lim_{n \to \infty} K_n = \bigcap^\infty_{n=1} K_n
	}
	\Say{[1]}{\THM{CompactHausdorffMetricIsCompact}(K_1)}
	{
		\TYPE{Compact}\Big(\K(K_1)\Big)
	}
	\Say{\Big(k,[2]\Big)}{\THM{CompactIsSequinceCompcat}}
	{
		\sum k : \TYPE{Increasing}(\Nat,\Nat) \. \TYPE{Converging}(\Big( \K(K_1),d_\H \Big),K_k)
	}
	\Say{[3]}{\THM{ConvergingIsCauchy}[2]}
	{    
		\TYPE{Cauchy}\Big(\big(\K(K_1),d_\H \big),K_k\Big)
	}
	\Say{[4]}{\Elim \TYPE{Decreasing}\Big(\K(X),K\Big)\Elim \TYPE{Cauchy}[2]}
	{
		\TYPE{Cauchy}\Big( \big(\K(K_1),d_\H\big),K \Big)
	}
	\Say{[5]}{\Elim \TYPE{Complete}\Big(\K(K_1)\Big)[4]}
	{
		\TYPE{Convergent}\Big(\big(\K(K_1),d_\H\big), K \Big)
	}
	\SayIn{L}{\lim_{n\to\infty}K_n}{ \K(X)  }
	\AssumeIn{l}{L}
	\Say{\Big(x,[6]\Big)}{\THM{CompactHausdorffConvergence}(K)}
	{
		\sum x : \prod^\infty_{n=1} K_n \. l = \lim_{n \to \infty} x_n
	}
	\Say{[7]}{\Elim \TYPE{Decreasing}(K)\THM{CompactIsClosed}(K)\THM{ClosedHasLimits}(K)[6](l)}
	{
		\forall n \in \Nat \. l \in K_n
	}
	\Conclude{[l.*]}{\Intro \bigcup [7]}{l \in \bigcup^\infty_{n=1} K_n}
	\Derive{[6]}{\Intro \subset}{L \subset \bigcup^\infty_{n=1} K_n}
	\Say{[7]}{\THM{CompactHausdorffConvergence}(K)\Elim \bigcup^\infty_{n=1} K_n \Intro \subset}
	{
		\bigcup^\infty_{n=1} K_n \subset L
	}
	\Conclude{[*]}{\Intro\TYPE{SetEq}[7][6]}
	{
		\bigcup^\infty_{n=1} K_n = L
	}
	\EndProof
	\\
	\Theorem{ContainmentIsClosedRelation}
	{
		\forall X : \TYPE{Metrizable} \.
		\TYPE{Closed}\Big( X\times\K(X), \big\{ (x,K)\in X \times \K(X) : x \in K \big\} \Big) 
	}
	\NoProof
	\\
	\\
	\Theorem{SubsetIsClosedRelation}
	{
		\forall X : \TYPE{Metrizable} \.
		\TYPE{Closed}\Big( \K(X)\times\K(X), \big\{ (K,L)\in \K(X) \times \K(X) : K \subset L \big\} \Big) 
	}
	\NoProof
	\\
}\Page{
	\Theorem{NonEmptyIntersectionIClosedRelation}
	{
		\forall X : \TYPE{Metrizable} \. \NewLine \.
		\TYPE{Closed}\Big( \K(X)\times\K(X),\big\{ (K,L) \in \K(X) \times \K(L) : K  \cap L \neq \emptyset\big\} \Big)
	}
	\NoProof
	\\
	\Theorem{UnionIsContinuous}
	{
		\forall X : \TYPE{Metrizable} \.
		\Big( \Lambda (A,B) \in \K(X) \times \K(X) \. A \cap B \Big) \in \TOP\Big( \K(X) \times \K(X),\K(X) \Big)
	}
	\NoProof
	\\
	\Theorem{CompactUnionIsContinuous}
	{
		\forall X : \TYPE{Metrizable} \.
		\left( \Lambda \A \in \K^2(X) \. \bigcup \A \right) \in \TOP\Big( \K^2(X),\K(X) \Big)
	}
	\Say{\Big(d,[1]\Big)}
	{
		\Elim \TYPE{Metrizable}(X) \THM{BoundedRemetrization}
	}
	{
		\sum d : \TYPE{Metric}(X) \. 
		\NewLine \.
		(X,d) \cong_\TOP X \And d < 1
	}
	\AssumeIn{\A}{\K^2(X)}
	\Assume{x}{\TYPE{Cauchy}\left(\bigcup \A,d \right)}
	\Say{\Big(K,[1]\Big)}{\Elim \bigcup \A (x)}
	{
		\sum K : \Nat \to \A \. \forall n \in \Nat \. x_n \in K_n
	}
	\Say{\Big(k,[2]\Big)}{\THM{CompactIffSequenceCompact}\left(\bigcup \A,K\right)}
	{
		\sum k : \TYPE{Increasing}(\Nat,\Nat) \.
		\TYPE{Converging}\Big(\K(X),K_k\Big)
	}
	\SayIn{L}{\lim_{n \to \infty} K_{k_n}}{\K(X)}
	\Say{\Big(y,[3]\Big)}{\Elim L \Elim \TYPE{Limit}\Big(\K(X),d_\H\Big)(y)}
	{
		\sum y \in L \. \lim_{n \to \infty} d(y_n,x_{k_n}) = 0
	}
	\Say{\Big(l,[3]\Big)}{\THM{CompactIffSequenceCompact}\Big(L,y\Big)}
	{
		\sum l : \TYPE{Increasing}(\Nat,\Nat) \.
		\TYPE{Converging}(L,y_l)
	}
	\SayIn{z}{\lim_{n \to \infty} y_{l_n}}{L}
	\Say{[4]}{\Elim z \THM{MetricLimitAgrees}[4]}{\lim_{n \to \infty} x_{k_{l_n}} = z}
	\Conclude{[*]}{\THM{CauchyHasSubsequenceLimit}\Big((X,d),x)[4]}{\lim_{n \to \infty} x_{n} = z}
	\Derive{[1]}{\Intro \TYPE{Complete}}{\TYPE{Complete}\left(\bigcap^\infty_{n=1}\A\right)}
}\Page{
	\Assume{n}{\Nat}
	\Say{\Big(\B,[2]\Big)}{
		\THM{CompactIsTotallyBounded}\Big( \big(\K(X),d_\H\big),A\Big)
		\Elim \TYPE{TotallyBounded}\Big(\big(\K(A), d_\H\big),A \Big)
	}
	{
		\NewLine :
		\sum \B : \TYPE{Finite}(A)  \. 
		\forall K \in \A \.
		\exists L \in \B :
		d_\H(K,L) < \frac{1}{2n}
	}
	\Say{\Big(F,[3]\Big)}{
		\THM{CompactIsTotallyBounded}\Big( \big(X,d\big),B\Big)
		\Elim \TYPE{TotallyBounded}\Big(\big(X, d\big),B \Big)
	}
	{
		\NewLine :
		\sum F : \prod_{K \in \B} \TYPE{Finite}(X) \. 
		\forall K \in \B \. \forall x \in K \. \exists y \in F_K \. 
		d(x,y) < \frac{1}{2n}
	}
	\Say{G}{\bigcup_{K \in B} F_B}{\TYPE{Finite}(X)}
	\Assume{x}{\bigcup \A}
	\Say{\Big(K,[4]\Big)}
	{
		\Elim \FUNC{union}(\A)(x)
	}
	{
		\sum_{K \in \A} \. x \in K
	}
	\Say{\Big(L,[5] \Big)}{[2](K)}{\sum L \in \B \. d_H(K,L) < \frac{1}{2n}}
	\Say{\Big(y,[6]\Big)}{\Elim d_H [5](x)}{\sum y \in L \. d(x,y) < \frac{1}{2n}}
	\Say{\Big(z,[7]\Big)}{[3](L,y)}{\sum z \in F_L \. d(y,z) < \frac{1}{2n}}
	\Say{[8]}{\Elim G (z)}{z \in G}
	\Conclude{[n.*]}{\THM{TriangleIneq}(X,x,y,z)[6][7]}
	{
		d(x,z) \le d(x,y) + d(y,z) < \frac{1}{2n}
	}
	\Derive{[2]}{\Intro \TYPE{TotallyBounded}}{\TYPE{TotallyBounded}\left( \bigcup \A \right)}
	\Conclude{[\A.*]}{\THM{CompactIffCompleteAndTotallyBounded}[2]}
	{
		\Compacts\left( X, \bigcup \A \right)
	}
	\Derive{[1]}{\Intro(\forall)}{\forall \A \in \K^2(X) \. \bigcup \A \in \K(X)}
	\Assume{\A}{\TYPE{Converging}\Big( \K^2(X) \Big)}
	\SayIn{\B}{\lim_{n \to \infty} \A_n}{\K^2(X)}
	\AssumeIn{\varepsilon}{\Reals_{++}}
	\Say{\Big(N,[2]\Big)}{\Elim \B \Elim \TYPE{Limit}\Big( \K^2(X) \Big)}
	{
		\sum N \in \Nat \. 
		\forall n \in \Nat \. n \ge N \Imply d_\H(\A_n,\B) < \varepsilon
	}
	\AssumeIn{n}{\Nat}
	\Assume{[3]}{n \ge N}
	\Say{[4]}{[2]\Big(n,[3]\Big)}{d_\H(\A_n,\B) < \varepsilon}
	\Say{[5]}{\Lambda K \in \A_n \. \Elim d_\H[4](K)}{\forall K \in \A_n \. \exists L \in \B \. d_\H(K,L) < \varepsilon}
	\Say{[6]}{\Lambda L \in \B \. \Elim d_\H[4](L)}{\forall L \in \B \. \exists K \in \A_n \. d_\H(K,L) < \varepsilon}
	\Conclude{[\A.*]}{\Elim d_\H [5][6]}{d_\H\left(\bigcup \A_n,\B\right) < \varepsilon}
	\Derive{[*]}{\THM{ContinuousByLimits}}
	{
		\Big(\Lambda \A \in \K^2(X) \. \bigcup \A\Big) \in \TOP( \K^2(X) ,\K(X))
	}
	\EndProof
}\Page{
	\Theorem{HausdorffMetricImageContinuity}
	{
		\forall X,Y : \TYPE{Metrizable} \.
		\forall f \in \TOP(X,Y) \.
		f^* \in \TOP\Big(\K(X),\K(Y)\Big) 
	}
	\Say{\Big(\alpha,[1]\Big)}
	{
		\Elim \TYPE{Metrizable}(X) \THM{BoundedRemetrization}
	}
	{
		\sum \alpha : \TYPE{Metric}(X) \. 
		\NewLine \.
		(X,\alpha) \cong_\TOP X \And \alpha < 1
	}
	\Say{\Big(\beta,[2]\Big)}
	{
		\Elim \TYPE{Metrizable}(X) \THM{BoundedRemetrization}
	}
	{
		\sum \alpha : \TYPE{Metric}(Y) \. 
		\NewLine \.
		(Y,\beta) \cong_\TOP Y \And \beta < 1
	}
	\AssumeIn{K}{\K(X)}
	\AssumeIn{\varepsilon}{\Reals_{++}}
	\Say{[1]}{\THM{CompactImage}(f,K)}{\Compacts\Big(Y,f(K)\Big)}
	\Say{\Big( F, [2] \Big)}{\Elim \TYPE{TotallyBounded}\Big(Y,f(K)\Big)\left(\frac{\varepsilon}{2}\right)}
	{
		\sum F : \TYPE{Finite}\Big(f(K)\Big) \. \forall y \in f(K) \. \exists z \in F : \beta(y,z) < \frac{\varepsilon}{2}
	}
	\Say{\U}{\left\{f^{-1}\left( \mathbb{B}_\beta\left(y,\frac{\varepsilon}{2}\right) \right) \Bigg| y \in \F \right\}}{?\T(X)}
	\SayIn{V}{\bigcup \U}{\T(X)}
	\Say{W}{\Big\{ L \in \K(X) : L \subset V \And \forall U \in \U \. U \cap V \neq \emptyset \Big\}}
	{
		\T\Big( \K(X) \Big)
	}
	\Say{[3]}{ \Elim W [1]  }{ X \in W  }
	\Say{\Big( \delta,[4]\Big)}{\THM{MetricOpenCriterion}[2]}
	{
		\sum \delta \in \Reals_{++} \. \mathbb{B}_{\alpha_\H}(K,\delta) \subset W
	}
	\AssumeIn{L}{\mathbb{B}_{\alpha_\H}(K,\delta)}
	\Say{[5]}{[4](L)}{L \in W}
	\AssumeIn{y}{f(L)}
	\Say{\Big(x,[6]\Big)}{\Elim \FUNC{image}(y)}
	{
		\sum x \in L \. y = f(x) 
	}
	\Say{[7]}{[5](x)}{x \in V}
	\Conclude{[8]}{\Elim V [7][6][2]}{\exists z \in F \. \beta(z,y) = \beta(z,f(x)) < \frac{\varepsilon}{2} < \varepsilon }
	\Derive{[6]}{\Intro \forall}{ \forall y \in f(L) \. \exists z \in f(K) : \beta(y,z) < \varepsilon} 
	\AssumeIn{y}{f(K)}
	\Say{\Big(z,[7]\Big)}{[2](y)}{\sum z \in F \. d(z,y) < \frac{\varepsilon}{2}}
	\Say{\Big(U,[8]\Big)}{\Elim U  (z) }
	{
		\sum U \in \U \.  z \in f(U)
	}
	\Say{[9]}{\Elim W [5](U)}{U \cap L \neq \emptyset}
	\Say{\Big( x, [10]\Big)}{\Elim \TYPE{NonEmpty}[9]\Elim U[8] }
	{
		\sum x \in L \. \beta\Big( f(x),z\Big) < \frac{\varepsilon}{2}
	}
	\Conclude{[y.*]}{\THM{TriangleInrq}\Big((Y,\beta),y,z,f(x)\Big)[7][10]}
	{
		d\Big( y, f(x) \Big) \le d(y,z) + d\Big(z,f(x) \Big) < \varepsilon
	}
	\Derive{[7]}{\Intro \forall}{\forall y \in f(K) \. \exists z \in f(L) \. \beta(y,z) < \varepsilon}
	\Conclude{[\varepsilon.*]}{\Intro \beta_\H[6][7]}{\beta_\H\Big( f(K),f(L) \Big) < \varepsilon}
	\DeriveConclude{[*]}{\THM{EpsilonDeltaContinuity}}{f^* \in \TOP\Big( \K(X),\K(Y)\Big)}
	\EndProof
}
\Page{
	\Theorem{HausdorffMetricProduct}
	{
		\forall X,Y : \TYPE{Metrizable} \.
		(\times) \in \TOP\Big( \K(X) \times \K(Y), \K(X \times Y) \Big)
	}
	\NoProof
	\\
	\Theorem{FiniteSetsSetIsFSigma}
	{
		\forall X : \TYPE{Metrizable} \.
		\TYPE{Finite}(X) \in F_\sigma\Big( \K(X) \Big)
	}
	\NoProof
	\\
	\Theorem{PerfectComapctsSetIsGDelta}
	{
		\forall X : \TYPE{Metrizable} \.
		\TYPE{Perfect} \And \Compacts(X) \in G_\delta\Big( \K(X) \Big)
	}
	\NoProof
	\\
	\DeclareFunc{treeSet}{\prod_{A \in \SET} ?\Big(\FS{A}\to\mathbb{B}\Big)}
	\DefineNamedFunc{treeSet}{}{\Tr(A)}{\FUNC{set}\Big( \Tree(A) \Big)}
	\\
	\DeclareFunc{prunedTreeSet}{\prod_{A \in \SET} ?\Big(\FS{A}\to\mathbb{B}\Big)}
	\DefineNamedFunc{prunedTreeSet}{}{\PTr(A)}{\FUNC{set}\Big( \TYPE{Pruned}(A) \Big)}
	\\
	\Theorem{NatTreeSetIsClosed}{\TYPE{Closed}\Big( \Tr(\Nat), \mathbb{B}^{\FS{\Nat}}\Big)}
	\Assume{T}{\Nat \to \Tr(\Nat)}
	\Assume{[1]}{\TYPE{Converging}\Big( \Bool^{\FS{\Nat}} ,T \Big)}
	\SayIn{S}{\lim_{n \to \infty} T_n}{\Bool^{\FS{\Nat}}}
	\AssumeIn{x}{\Bool^{\FS{\Nat}}}
	\Assume{[2]}{S(x) = 1}
	\Say{\Big(N,[3]\Big)}
	{ \Elim \FUNC{productTopology}[2] \Elim S  }
	{
		\sum N \in \Nat \. \forall n \in \Nat \. n \ge N \Imply T(x) = 1
	}
	\Say{[4]}{[3]\Elim \Tree(T)}{ 
		\forall y \in \FS{\Nat} \. 
		y \subset x \Imply
		\forall n \in \Nat \. 
		n \ge N \Imply T(y) = 1
	}
	\Say{[5]}{\Elim \FUNC{productTopology}[4]\Intro S}
	{
		\forall y \in \FS{\Nat} \. y \subset x \Imply  S(y) = 1
	}
	\Conclude{[6]}{\Elim \Tr(\Nat)[5]}{X \in \Tr(\Nat)}
	\DeriveConclude{[*]}{\THM{ClosedByConvergence}}{\TYPE{Closed}\Big(\Bool^{\FS(\Nat)}, \Tr(\Nat) \Big)}
	\EndProof
}
\Page{
	\Theorem{NatPrunedTreeSetIsGDelta}
	{
		\PTr(\Nat) \in G_\delta\Big( \Bool^{\FS{\Nat}} \Big) 
	}
	\Say{[1]}{\THM{NatTreeSetIsClosed}\;\THM{ClosedIsGDeltaInPolish}}
	{
		\Tr(\Nat) \in G_\delta\Big( \Bool^{\FS{\Nat}} \Big)
	}
	\Say{[2]}{\Elim \PTr(\Nat)}
	{
		\PTr(\Nat) = \Tr(\Nat) \cap \bigcap_{\emptyset \neq w \in \FS{\Nat}} N_{w=0} \cup \bigcap_{w \subset u} N_{u=1}
	}
	\Conclude{[3]}{[1][2]\THM{GdeltaIntersectionIsGdelta}\;\THM{FiniteGdeltaUnionIsGdelta}}
	{
		\PTr(\Nat) \in G_\delta\Big( \Bool^{\FS{\Nat}} \Big)
	}
	\EndProof
	\\
	\Theorem{BoolTreeSetIsClosed}{\TYPE{Closed}\Big( \Tr(\Bool), \Bool^{\FS{\Bool}}\Big)}
	\NoProof
	\\
	\Theorem{BoolPrunedTreeSetIsClosed}{\TYPE{Closed}\Big( \PTr(\Bool), \Bool^{\FS{\Bool}}\Big)}
	\NoProof
	\\
	\Theorem{BodyBijectionIsHomeo}
	{
		\K(\C) \cong_\TOP \PTr(\Bool)
	}
	\NoProof
}
\newpage
\subsubsection{Locally Compact Spaces}
\Page{
	\Theorem{LocallyCompactIsPolishIffSecondCountable}
	{
		\forall X : \LCompact \. \NewLine \. 
		\Polish(X) \iff \TYPE{Metrizable} \And \TYPE{SecondCountable}(X)
	}
	\NoProof
	\\
	\Theorem{LocallyCompactIsPolishIffMetAndSigma}
	{
		\forall X : \LCompact \. \NewLine \.
		\Polish(X) \iff \TYPE{Metrizable}(X) \And \SCompact(X)
	}
	\NoProof
	\\
	\Theorem{LocallyCompactPolishIffCompactlyMetrizable}
	{
		\forall X : \LCompact \. \NewLine \.
		\Polish(X) \iff \TYPE{CompactlyMetrizable}(X)
	}
	\NoProof
	\\
	\Theorem{LocallyCompactPolishIffOpenSubset}
	{
		\forall X : \LCompact \. \NewLine \.
		\Polish(X) \iff \TYPE{Metrizable}(X) \And 
		\exists Y : \TYPE{CompactlyMetrizable} :
		\exists U \in \T(X) \.
		U \cong_\TOP Y
	}
	\NoProof
}
\newpage
\subsubsection{Cantor's schemes}
\Page{
	\DeclareType{CantorSchema}{\prod_{X \in \SET} \FS{\Bool} \to ?X}
	\DefineType{A}{CantorSchema}{ \Big(\forall s \in \FS{\Bool} \. 
		A_{s0} \cap A_{s1} = \emptyset\Big) 
		\And   
		\Big(\forall t,s \in \FS{\Bool} \. t \subset s \Imply A_{s} \subset A_{t}\Big)
	}
	\\
	\DeclareType{TopologicalSchema}{\prod (X,d) \in \MS \. ?\TYPE{CantorSchema}(X)}
	\DefineType{U}{TopologicalSchema}
	{
		\Big(
			\forall s \in \FS{\Bool} \.
			U_s \in \T(X) \And U_s \neq \emptyset
			\And  \diam(U_s) \le 2^{-\len(s)} 
		\Big)
		\And
		\NewLine
		\And
		\forall s,t \in \FS{\Bool} \.
		s \subset t \Imply 
		\overline{U}_t \subset U_s
	}
	\\
	\Theorem{EmbeddingOfCantorSetBySchema}
	{
		\forall X : \Polish \.
		\forall U : \TYPE{MetricSchema}(X) \.
		\exists \TYPE{TopologicalEmbedding}(\C,X)
	}
	\AssumeIn{x}{\C}
	\Say{[1]}{
		\Big(
			\Lambda n \in \Int_+ \. 
			\Elim \TYPE{MetricSchema}(X,U)(\inits{x}{n})[2] 
		\Big)
		\THM{ReductioInfima}
	}
	{
		\lim_{n \to \infty} \diam \overline{U}_{\inits{x}{n}} = 0
	}
	\Conclude{\Big(f(x),[x.*]\Big)}{\THM{CantrorIntersectionTHM}[1]}  
	{
		\sum f(x) \in X \. f(x) = \bigcup^\infty_{n=0} \overline{U}_{\inits{x}{n}}
	}
	\Derive{\Big(f,[1]\Big)}{\Intro\to\Intro \sum}
	{
		\sum f : \C \to X \. 
		\forall x \in \C \.  
		f(x) = \bigcup^\infty_{n=0} \overline{U}_{\inits{x}{n}}
	}
	\Say{[2]}{[1] \Elim \TYPE{MetricSchema}(U)\THM{ContinuousByConvergence}}
	{
		f \in \TOP(\C,X)
	}
	\Say{[3]}{[1] \Elim \TYPE{CantorSchema}(U)\Intro \TYPE{Injective}}
	{
		\TYPE{Injective}\Big(\C,X, f\Big)
	}
	\Say{[4]}{\THM{ProperByCompactDomain}(\C,X,f)}{\TYPE{ProperMap}(\C,X,f)}
	\Say{[5]}{\THM{FirstCountableIsCG}(X)}{X \in \CG}
	\Conclude{[*]}{\THM{InjectiveProperIsEmbedding}[2][3][4][5]}
	{
		\TYPE{TopologicalEmbedding}(\C,X,f)
	}
	\EndProof
}\Page{
	\Theorem{MetricSchemaExists}
	{
		\forall X : \Polish \And \Perfect \.
		\forall d : \TYPE{CompleteMetric}(X) \.
		X \neq \emptyset
		\Imply
		\exists \TYPE{MetricSchema}(X,d)
	}
	\SayIn{U_\emptyset}{X}{\T(X) \And \TYPE{NonEmpty}}
	\AssumeIn{n}{\Int_+}
	\AssumeIn{s}{\FS{\B}}
	\Assume{[1]}{\len(s) = n}
	\SayIn{x}{\Elim \TYPE{NonEmpty}(s)}{U_s}
	\Say{\Big(y,[2]\Big)}{\Elim \Perfect(X)(x)\Elim \TYPE{IsolatedPoint}(x,U_s)}
	{
			\sum y \in U_s \. x \neq y
	}
	\Say{r}{\min\left(2^{-1-n},\frac{d(x,y}{3},d(x,\boundary U_s),d(y,\boundary U_t)\right)}{\Reals_{++}}
	\SayIn{U_{s0}}{\mathbb{B}_d(x,r)}{\T(X) \And \TYPE{NonEmpty}}
	\SayIn{U_{s1}}{\mathbb{B}_d(y,r)}{\T(X) \And \TYPE{NonEmpty}}
	\SayIn{[s.1]}{\Elim r \Elim U_{s0} \Elim U_{s1}}{ U_{s0} \cap U_{s1} = \emptyset}
	\Say{[s.2]}{\Elim r \Elim U_{s0}}{\overline{U}_{s0} \subset U_s}
	\Say{[s.3]}{\Elim r \Elim U_{s1}}{\overline{U}_{s1} \subset U_s}
	\Conclude{[s.4]}{\Elim r \Elim U_{s0}}{\diam U_{s0} < 2^{-n}}
	\Conclude{[n.s.*]}{\Elim r \Elim U_{s1}}{\diam U_{s1} < 2^{-n}}
	\DeriveConclude{[*]}{\Intro \TYPE{MetricSchema}}{\TYPE{MetricSchema}(U)}
	\EndProof
	\\
	\Theorem{CantorSetEmbedding}
	{
		\forall X  : \Polish \And \Perfect \.
		X \neq \emptyset \Imply
		\exists \TYPE{TopologicalEmbedding}(X,\C)
	}
	\NoProof
	\\
	\Theorem{ParfectPolishCardinality}
	{
		\forall X : \Polish \.
		\forall A : \TYPE{PerfectSubset}(X) \.
		|A| = 2^{\aleph_0}
	}
	\NoProof
	\\
	\Theorem{PolishContinuumTHM}
	{
		\forall X : \Polish \.
		|X| > \aleph_0 \Imply |X| = 2^{\aleph_0}
	}
	\NoProof
}
\newpage
\subsubsection{Cantor-Bendixson's ranks}
\Page{
	\Theorem{MonotonicicOrdinalTopologicalBound}
	{
		\NewLine ::
		\forall X : \TYPE{SecondCountable} \.
		\forall a \in \ORD \. 
		\forall F : \TYPE{StrictlyDecreasing}\Big(a, \TYPE{Closed}(X)\Big) \.
		|a| \le \aleph_0
	}
	\Say{\mathcal{V}}{\Elim \TYPE{SecondCountable}(X)}{\sum \mathcal{V} : \TYPE{Base}(X) \. |\mathcal{V}| \le \aleph_0}
	\Say{V}{\FUNC{enumerate}(\mathcal{V})}{\Big[0,\ldots |\mathcal{V}|\Big) \ToBij \mathcal{V} } 
	\Say{N}{\Lambda i \in a \. \{ n \in \Nat | F \cap V_n \neq \emptyset  \}}{a \to ?\Nat}
	\Say{[1]}{\Elim N \Elim F}{\TYPE{StrictlyDecreasing}(a,?\Nat,N)}
	\Say{M}{\Lambda i \in a \. \If i = \max(a) \Then N_i \Else N_i \setminus N_{\sigma(i)}}{a \to ?\Nat}
	\Say{[2]}{\Elim \TYPE{StrictlyDereasing}(a,?\Nat,N)\Elim M}{\forall i \in a \. i \neq \max a \Imply M_a \neq \emptyset}
	\Say{[3]}{\Elim M \Elim \FUNC{setminus}}{\forall i,j \in a \. i \neq j \Imply M_i \cap M_j = \emptyset}
	\SayIn{m}{\LOGIC{Choice}[2]}{\prod_{i < a} M_i}
	\Say{[4]}{\Elim m [3]\Intro \TYPE{Injective}}{\TYPE{Injective}(a,\Nat,m)}
	\Conclude{[*]}{\THM{InjectionCardinalityBound}[4]\THM{NaturalNumbersAreCountable}}{|a| \le \aleph_1}
	\EndProof
	\\
	\DeclareFunc{derivativeOfCantorBendixon}{\ORD \to \TOP \to \TOP}
	\DefineNamedFunc{derivativeOfCantorBendixon}{0,X}{\d^0 X}{X}
	\DefineNamedFunc{derivativeOfCantorBendixon}{\sigma(a),X}{\d^{\sigma(a)} X}{\lim \d^a X}
	\DefineNamedFunc{derivatveOfCantorBendixon}{a,X}{\d^a X}{ \lim \bigcap_{n < a} \d^n X  }
}\Page{
	\Theorem{CantorBendixonRankExists}
	{
		\forall X : \Polish \.
		\exists a \in [0,\epsilon_0) \. 
		\forall b \ge a \.
		\d^b X = \d^a X
	}
	\Say{[1]}{\Lambda a \in [0,\epsilon_0] \. \d^a X}{\TYPE{Decreasing}\Big( [0,\epsilon_0], \TYPE{Closed}(X)\Big)}
	\Say{[2]}{\THM{MonotonicOrdinalTopologicalBound}[1]}
	{
		\neg \TYPE{StrictlyDecreasing}\Big( 
			[0,\epsilon_0],
			\TYPE{Closed}(X), 
			\Lambda a \in [0,\epsilon_0] \.
				\d^a X
		\Big)
	}
	\Say{\Big(a,[3] \Big)}{\Elim \TYPE{StrictlyDecreasing}}{\sum a \in \ORD \. \d^{\sigma(a)} X = \d^a X}
	\AssumeIn{b}{\ORD}
	\Assume{[0]}{b \ge a}
	\Assume{[4]}{\forall c \in [a,b] \. \d^{c} X = \d^a X}
	\Conclude{[b.*]}{\Elim \d [4](b) \Elim \d [3]}
	{
		\d^{\sigma^(b)} X = \d \d^{b} X = \d \d^a X = \d^{\sigma(a)} X = \d^a X
	}
	\Derive{[4]}{\Intro \forall \Intro \Imply}
	{
		\forall b \in \ORD \. 
		(b \ge a) \Imply 
		\Big( \forall c \in [a,b]) \d^{c} X = \d^c X\Big) \Imply 
		\d^{\sigma(b)} X = \d^{a} X 
	}
	\Assume{b}{\TYPE{Limit}}
	\Assume{[0]}{b \ge a}
	\Assume{[5]}{\forall c \in [a,b) \. \d^{\sigma(c)} X = \d^a X }
	\Conclude{b.*}{\Elim \d \Elim \TYPE{Decreasing}[5] \Elim \d}
	{
		\d^{b} X = 
		\d \bigcap_{c < b} \d^c X = 
		\d \d^a X = 
		\d^{\sigma(a)} X =
		\d^a X
	}
	\Derive{[5]}{\Intro \forall \Intro \Imply}
	{
		\forall b \in \TYPE{Limit} \. 
		(b \ge a) \Imply 
		\Big( \forall c \in [a,b) \. \d^{c} X = \d^c X \Big) \Imply 
		\d^{b} X = \d^{a} X 
	}
	\Say{[6]}{\Intro(=)(\d^a X) }
	{
		\d^a X = \d^a X
	}
	\Conclude{[*]}{\THM{TransfinitieInduction}[6][4][5]}
	{
		\forall b \ge a \. \d^b = \d^a X
	}
	\EndProof
	\\
	\DeclareFunc{rankOfCantorBendixson}
	{
		\Polish \to \epsilon_0
	}
	\DefineNamedFunc{rankOfCantorBendixon}{X}{\rankcb X}{\min \{ a \in \ORD : \d^{\sigma(a)} X = \d^a X \} }
	\\
	\DeclareType{PerfectTree}{\prod_{A \in \SET} \Tree(A)}
	\DefineType{T}{PerfectTree}{\forall t \in T \. \exists a,b \in T : t \subset a \And t \subset b \And a \bot b }
	\\
	\Theorem{PerfectIsPruned}{
			\forall A \in \SET \. 
			\forall T : \TYPE{PerfectTree}(A)  \.
			\TYPE{Prunded}(A,T)
		}
	\NoProof
	\\
	\Theorem{PerfectBodyTHM}
	{
		\forall A \in \SET \.
		\forall T : \TYPE{Pruned}(A) \.
		\TYPE{PerfectTree}(A,T) 
		\iff
		\Perfect\big([A]\big)
	}
	\NoProof
}
\newpage
\subsection{Zero Dimensional Spaces and Schemas}
\subsubsection{Dimension Zero}
\Page{
	\DeclareType{ZeroDimensional}{?\TOP}
	\DefineNamedType{X}{ZeroDimensional}{\dim_\TOP X = 0}
	{
		\exists \V : \TYPE{Base}\big(\T(X)\big) : \forall V \in \V \.
		\TYPE{Clopen}(X,V)
	}
	\\
	\Theorem{UltrametricTrianglesAreEquiliterals}
	{
		\NewLine
		:: 
		\forall X : \TYPE{UltrametricSpace} \.
		\forall x,y,z \in X \. 
		d(x,z) \neq d(y,z) 
		\Imply
		d(x,y) = \max\Big( d(x,z),d(y,z)\Big)
	}
	\Say{[1]}{\Elim \TYPE{Ultrametric}(X,d,x,y,z)\Elim \TYPE{Symmetric}(x,d,y,z)}
	{
		d(x,y) \le \max\Big( d(x,z), d(y,z) \Big)
	}
	\Assume{[2]}{d(x,y) < \max\Big( d(x,z),d(y,z) \Big)}
	\Say{[3]}{\Elim \TYPE{Ultrametric}(X,d,x,z,y)}{ d(x,z) \le \max\Big( d(x,y),d(y,z) \Big)  }
	\Say{[4]}{\Elim \TYPE{Ultrametric}(X,d,y,z,x)}{ d(y,z) \le \max\Big( d(y,x),d(x,z) \Big)  }
	\Say{[5]}{[0][3][4]}{d(x,z) \le d(x,y) | d(y,z) \le d(x,y) }
	\Say{[6]}{[5][2]}{d(x,z) \le d(x,y) < d(y,z) | d(y,z) \le d(x,y) < d(x,z) }
	\Conclude{[2.*]}{[6][3][4]}{ \bot }
	\Derive{[*]}{\THM{TrichtomyPrinciple}}{d(x,y) = \max\Big(d(x,z), d(y,z) \Big)}
	\EndProof
}\Page{
	\Theorem{UltrametricAreZeroDim}
	{
		\forall X : \TYPE{UltrametricSpace} \. 
		\dim_\TOP X = 0
	}
	\AssumeIn{x}{X}
	\AssumeIn{r}{\Reals_{++}}
	\Assume{b}{\Nat \to \mathbb{B}_X(x,r)}
	\Assume{[1]}{\TYPE{Converging}(X,b)}
	\SayIn{L}{\lim_{n \to \infty} b_n}{X}
	\Say{t}{d(L,x)}{\Reals_{++}}
	\Assume{[2]}{t = 0}
	\Say{[3]}{\Elim \TYPE{Metric}[2]}{x = L}
	\Conclude{[*]}{[3]\Elim  \FUNC{cell}(X)(x,r)}{ L \in \mathbb{B} }
	\Derive{[2]}{\Intro(\Imply) }
	{
		t = 0 \Imply L \in \mathbb{B}_X(x,r)
	}
	\Assume{[3]}{t > 0}
	\Say{\Big(n,[4]\Big)}{\Elim \TYPE{Limit}(L)}
	{
		\sum n \in \Nat \. d(u_n, L) < t
	}
	\Say{[5]}{\Elim \FUNC{cell}(X)(x,r)(u_n)}
	{
		d(u_n,x) < r
	}
	\Say{[6]}{\THM{UltrametricTriangleAreEquilitertal}(X,u_n,x,L)}
	{
		d(L,x) = d(u_n,x) \Big| d(L,x) = d(u_n,L)
	}
	\Say{[7]}{\THM{LimitMetric}\Elim t}{t \le r}
	\Say{[8]}{[6][4][5][7]}{d(L,x) < r}
	\Conclude{[*]}{\Elim \FUNC{cell}(X)(X,d)[6]}
	{
		L \in \mathbb{B}_X(x,r)
	}
	\Derive{[3]}{\Intro(\Imply) }
	{
		t \neq 0 \Imply L \in \mathbb{B}_X(x,r) 
	}
	\Conclude{[u.*]}{\Elim \LOGIC{LEM}(t=0)[2][3]}
	{
		L \in \mathbb{B}_X(x,r)
	}
	\Derive{[1]}{\THM{ClosedByLimits}}{\TYPE{Closed}\Big( X,\mathbb{B}_X(x,r) \Big)}
	\Conclude{[x.*]}{\Intro \Clopen}{\Clopen\Big(X,\mathbb{B}_X(x,r)\Big)}
	\DeriveConclude{[*]}{\Intro \TYPE{ZeroDimensional}}{\dim_\TOP X = 0}
	\EndProof
}\Page{
	\Theorem{InUltrametricAllBallPointsAreCenters}
	{
		\NewLine ::
		\forall X : \TYPE{UltrametricSpace} \.
		\forall x \in X \.
		\forall r \in \Reals_{++} \.
		\forall y \in \mathbb{B}_d(x,r) \.
		\mathbb{B}_X(x,r) = \mathbb{B}_X(y,r)
	}
	\Say{t}{d(x,y)}{\Reals_+}
	\Say{[1]}{\Elim \Ball_d(x,r)(y)\Elim t }{t < r} 
	\AssumeIn{u}{\Ball_X(x,r)}
	\Say{s}{d(x,u)}{\Reals_+}
	\Say{[2]}{\Elim \Ball_d(x,r)(u)\Elim s}{s < r}
	\Say{[3]}{\THM{UltrametricTriangleAreAllEquiliteral}(X,x,y,u)\Intro s \Intro t}
	{
		d(y,u) = s \Big| d(y,u) = t
	}
	\Say{[4]}{[1][2][3]}{d(y,u) < r}
	\Conclude{[u.*]}{\Elim \Ball_X(y,r)[4]}{u \in \Ball_X(y,r)}
	\Derive{[2]}{\Intro \subset}{\Ball_X(x,r) \subset \Ball_X(y,r)}
	\AssumeIn{u}{\Ball_X(y,r)}
	\Say{s}{d(y,u)}{\Reals_+}
	\Say{[3]}{\Elim \Ball_d(y,r)(u)\Elim s}{s < r}
	\Say{[4]}{\THM{UltrametricTriangleAreAllEquiliteral}(X,x,y,u)\Intro s \Intro t}
	{
		d(x,u) = s \Big| d(x,u) = t
	}
	\Say{[5]}{[1][3][4]}{d(y,u) < r}
	\Conclude{[u.*]}{\Elim \Ball_X(x,r)[4]}{u \in \Ball_X(x,r)}
	\Derive{[2]}{\Intro \subset}{\Ball_X(y,r) \subset \Ball_X(x,r)}
	\Conclude{[*]}{\Intro \TYPE{SetEq}[1][2]}{\Ball_X(y,r) = \Ball_X(x,r)}
	\EndProof
	\\
	\Theorem{UltrametricIntersectionImplyContainment}
	{
		\NewLine ::
		\forall X : \TYPE{UltrametricSpace} \.
		\forall x,y \in X \.
		\forall r,s \in \Reals_{++} \.
		\Ball_X(x,r) \cap \Ball_X(y,s) \.
		\NewLine
		\Ball_X(x,r) \subset \Ball_X(y,s)
		\Big|
		\Ball_X(y,s) \subset \Ball_X(x,r)
	}
	\NoProof
}\Page{
	\Theorem{UltrametricCauchySequences}
	{
		\NewLine ::
		\forall X : \TYPE{UltrametricSpace} \.
		\forall x : \Nat \to X \.
		\TYPE{Cauchy}(X,x) 
		\iff
		\lim_{n \to \infty} d(x_n,x_{n+1}) = 0
	}
	\Assume{[1]}{\lim_{n \to \infty} d(x_n,x_{n+1}) = 0}
	\AssumeIn{\varepsilon}{\Reals_{++}}
	\Say{\Big(N,[2]\Big)}{[1]\Elim \TYPE{Limit}}
	{
		\sum N \in \Nat \.
		\forall n \in \Nat \. 
		n \ge N
		\Imply
		d(x_n,x_{n+1}) < \varepsilon
	}
	\AssumeIn{m}{\Nat}
	\Assume{[3]}{m > N}
	\Assume{[4]}
	{
		\forall k \in [n+1,\ldots,m-1] \. 
		d(x_N,x_k) < \varepsilon
	}
	\Say{[5]}{[4](m-1)}
	{
		d(x_N,x_{m-1}) < \varepsilon
	}
	\Say{[6]}{[2](m-1)}
	{
		d(x_{m-1},x_m) < \varepsilon
	}
	\Conclude{[m.*]}{\Elim \TYPE{Ultrametric}[5][6]}
	{
		d(x_N,x_m) < \varepsilon
	}
	\DeriveConclude{[3]}{\Elim \Nat}
	{
		\forall n \in \Nat \. n \ge N \Imply d(x_N,x_n) < \varepsilon
	}
	\AssumeIn{n,m}{\Nat}
	\Assume{[4]}{n,m \ge N}
	\Say{[5]}{[3](n)}{d(x_N,x_n) < \varepsilon}
	\Say{[6]}{[3](m)}{d(x_N,x_m) < \varepsilon}
	\Conclude{\Big[(n,m).* \big]}{\Elim \TYPE{Ultrametric}[5][6]}
	{
		d(x_n,x_m) < \varepsilon
	}
	\DeriveConclude{[\varepsilon.*]}{\Elim \forall}
	{
		\forall n,m \in \Nat \.
		n,n \ge N 
		\Imply
		d(x_n,x_m) < \varepsilon
	}
	\DeriveConclude{[x.*]}{\Intro \TYPE{Cauchy}}{\TYPE{Cauchy}(X,x)}
	\DeriveConclude{[*]}{\Intro \Imply}
	{
		\lim_{n \to \infty} d(x_n,x_{n+1}) \Imply \TYPE{Cauchy}(X,x)
	}
	\EndProof
	\\
	\Theorem{ClopenSeparation}
	{
		\forall X : \TYPE{SecondCountable}
		\dim_\TOP X = 0
		\Imply \NewLine \Imply
		\forall A,B : \TYPE{ClosedSet}(X) \.
		A \cap B = \emptyset \Imply
		\exists C : \TYPE{Clopen}(C) \.
		A \subset C
		\And
		C \cap B = \emptyset
	}
	\NoProof
	\\
	\Theorem{KuratowskiZeroDimensionalChar}
	{
		\NewLine  ::
		\forall X \in \TOP \.
		\dim_\TOP X = 0
		\iff
		\forall A : \TYPE{Closed}(X) \.
		\TYPE{Retract}(X,A)
	}
	\NoProof
}
\newpage
\subsubsection{Cantor space}
\Page{
	\DeclareType{BrouwerSchema}
	{
		\prod_{X \in \MS} ?\TYPE{MetricSchema}
	}
	\DefineType{U}{BrouwerSchema}
	{
		U_\emptyset
		\forall s \in \FS{\Bool} \. 
		\Clopen(X,U_s) \And U_s = U_{s1} \cup U_{s2}
	}
	\\
	\Theorem{BrouwerSchemaInducesHomeomorphism}
	{
		\forall X : \Polish \And \TYPE{MetricSpace} \.
		\forall U : \TYPE{BrouwerSchema}(X) \.
		X \cong_{\TOP} \C
	}
	\\
	\Theorem{BrouwerSchemaExists}
	{
		\forall X : \Perfect \And \TYPE{CompactMetrizable} \And \TYPE{NonEmpty} \.
		\forall d : \TYPE{Metric}(X) \. \NewLine \.
		(X,d) \cong_\TOP X \And \dim_\TOP X =0
		\Imply
		\exists \TYPE{BrouwerSchema}(X)
	}
	\Say{U_\emptyset}{X}{\Clopen(X)}
	\Say{S_1}{\{\emptyset\}}{?\FS{\Bool}}
	\Say{S_0}{\emptyset}{?\FS{\Bool}}
	\Assume{n}{\Nat}
	\SayIn{s}{\min S_n}{\FS{\Bool}}
	\Say{\Big(\V,[1]\Big)}{\Elim \Compact(X) \Elim \dim_\TOP X = 0}
	{
		\sum \V : \TYPE{Finite}\Big( \Clopen(X) \Big) \. 
		\forall V \in \V \. \diam V < \frac{1}{n}
	}
	\SayIn{m}{|\V|}{\Nat}
	\Say{V}{\FUNC{enumerate}(\V)}{[1,\ldots,m]\ToBij\V}
	\AssumeIn{i}{[1,\ldots,m]}
	\Say{U_{s0^i}}{\bigcup^n_{j=i} V_i}{\Clopen(X)}
	\Conclude{U_{s0^{i-1}1}}{V_i}{\Clopen(X)}
	\Derive{U}{\Intro \to}{ 
		\{ s0^i | i \in [1,\ldots,m]  \} \cup 
		\{s0^{i-1}1 | i \in [1,\ldots,m]\} 
		\to
		\Clopen(X)
	}
	\Conclude{S_{n+1}}{\Big(S_n \setminus \{s\}\Big) \cup \{s0^m\} \cup \Big\{ s0^{i-1}i | i \in [1,\ldots,m]\Big\}}
	{
		?\FS
	}
	\DeriveConclude{U}{\Intro \TYPE{BrouwerSchema}}{\TYPE{BrouwerSchema}(X)}
	\EndProof
	\\
	\Theorem{BrouwerTopologicalCharOFCantorSpace}
	{
		\NewLine ::
		\forall X : \Perfect \And \TYPE{CompactMetrizable} \And \TYPE{NonEmpty} \.
		\dim_\TOP X = 0 \Imply X \cong_\TOP \C  
	}
	\NoProof
}
\newpage
\subsubsection{Lusin's schema}
\Page{
	\DeclareType{LusinSchema}
	{
		\prod_{X \in \TOP}  \FS{\Nat} \to ?X
	}
	\DefineNamedType{L}{LusinSchema}{
		\Big(
			\forall s \in \FS{\Nat} \. \forall n,m \in \Nat \. 
			n \neq m \Imply L_{sn} \cap L_{sm} = \emptyset
		\Big)
		\And
		\Big(
			\forall s,t \in \FS{\Nat} \.
			s \subset t 
			\Imply
			L_t \subset L_s
		\Big)
	}
	\\
	\DeclareType{VanishingDiameter}{\prod_{X \in \MS} \TYPE{LusinSchema}(X) }
	\DefineType{L}{VanishingDiameter}{\forall b \in \B \. \lim_{n \to \infty} \diam L_{\inits{b}{n}} = 0}
	\\
	\DeclareFunc{domainOfLusin}{\prod_{X \in \MS} \TYPE{VanishingDiameter}(X) \to ?\B }
	\DefineNamedFunc{LusinDomain}{L}{D(L)}{\left\{ b \in \B : \bigcap_{n=1}^\infty L_{\inits{b}{n}} \neq \emptyset  \right\}}
	\\
	\DeclareFunc{associatedMap}{\prod_{X \in \MS} \prod L : \TYPE{VanishingDiameter}(X)  \. D(L) \to X }
	\DefineNamedFunc{associatedMap}{b}{f_L(b)}{\Elim \TYPE{Singleton}(X)\bigcap^\infty_{n=1} L_{\inits{b}{n}} }
	\\
	\Theorem{associatedMapIsContinuousInjection}
	{
		\forall X \in \MS \. 
		\forall L : \TYPE{VanishingDiameter}(X) \. \NewLine \.
		f_L \in \TOP \And \TYPE{Injective}\Big( D(L),X \Big)
	}
	\NoProof
	\\
	\Theorem{AssociatedMapIsContinuousInjection}
	{
		\forall X \in \MS \. 
		\forall L : \TYPE{VanishingDiameter}(X) \. \NewLine \.
		\TOP \And \TYPE{Injective}\Big( D(L),X, f_L \Big)
	}
	\NoProof
	\\
}\Page{
	\Theorem{ClosedLusinDomain}
	{
		\forall X : \TYPE{Complete} \. 
		\forall L : \TYPE{VanishingDiameter}(X) \.
		\Big( \forall t \in \FS{L} \. \TYPE{Closed}(X,L_t) \Big)
		\Imply \NewLine \Imply
		\TYPE{Closed}\Big(\B,D(L)\Big)
	}
	\AssumeIn{b}{D^\c(L)}
	\Say{[1]}{\Elim D(L)(K)}{\bigcap^\infty_{n=1} L_{\inits{b}{n}} = \emptyset}
	\Say{\Big(n,[2]\Big)}{\THM{CantorIntersectionTHM}(L_{\inits{b}{\bullet}})[2]}
	{
		L_{\inits{b}{n}} = \emptyset
	}
	\Say{t}{\inits{b}{n}}{\FS{\Nat}}
	\Conclude{[b.*]}{[2]\Elim t}{b \in N_t \subset D^\c(L)}
	\Derive{[1]}{\THM{OpenByOpenCover}}{D^\c(L) \in \T(\B)}
	\Conclude{[*]}{\Intro \TYPE{Closed}[1]}{\TYPE{Closed}\Big( \B, D(L) \Big)}
	\EndProof
	\\
	\Theorem{AssociatedMapEmbeddingCondition}
	{
		\forall X \in \MS \. 
		\forall L : \TYPE{VanishingDiameter}(X) \.\NewLine \.
		\Big( \forall t \in \FS{L} \. \TYPE{Open}(X,L_t) \Big)
		\Imply 
		\TYPE{TopologicalEmbedding}\Big( D(L),X, f \Big)
	}
	\NoProof
	\\
	\DeclareType{AlexandrovUryshonSchema}{\prod X : \TYPE{Complete} \. ?\TYPE{VanishingDiameter}}
	\DefineType{U}{AlexandrovUryshonSchema}
	{
		U_\emptyset = X \NewLine 
		\forall t \in \FS{\Nat} \. U_t \neq \emptyset \NewLine
		\forall t \in \FS{\Nat} \. \Clopen(U_t) \NewLine 
		\forall t \in \FS{\Nat} \. U_t = \bigcup_{n=1}^\infty U_{tn} \NewLine
		\forall t \in \FS{\Nat} \. \diam U_t \le 2^{-\len(t)}
	}
	\\
	\DeclareType{Wiry}{?\TOP}
	\DefineType{X}{Wiry}{\forall K : \Compact(X) \. \intx K = \emptyset}
}
\Page{
	\Theorem{AlexandrovUryshonTHM}
	{
		\forall X : \Polish \And \NonEmpty \And \TYPE{Wiry} \. 
		\dim_\TOP X = 0
		\Imply
		\NewLine
		\Imply
		\exists \TYPE{AlexandrovUryshonSchema}(X)
	}
	\Say{\Big(d,[1]\Big)}
	{
		\Elim \Polish(X) 
	}
	{
		\sum d : \TYPE{Metric}(X) \. 
		\NewLine \.
		(X,\alpha) \cong_\TOP X \And \TYPE{Complete}(X,d) 
	}
	\Say{U_\emptyset}{X}{\Clopen(X)}
	\AssumeIn{b}{\FS{\Nat}}
	\Say{[2]}{\Elim \FUNC{closure}\Elim \FUNC{interiot}}
	{
			\emptyset \neq U_b \subset \intx\overline{U}_b
	}
	\Say{[3]}{\Elim \TYPE{Wiry}[2]}
	{
		\neg \Compact\Big( X,\overline{U}_b \Big)
	}
	\Say{[4]}{\THM{MetricCompact}(X,d)\THM{ClosedIsComplete}(X,d)[3]}
	{
		\neg \TYPE{TotallyBounded}\Big( X, \overline{U}_b \Big)
	}
	\Say{\Big( \V,[5] \Big)}{\Elim \TYPE{TotallyBounded}[5]\Elim \TYPE{ZeroDimensional}(X)}
	{
		\NewLine : 
		\sum \V : \TYPE{ClopenCover} \And \TYPE{IrreducibleCover} \And \TYPE{Disjoint}(X,\overline{U}_b) \. 
		\Big(\forall V \in \V \. \diam V < 2^{-1-\len(b)}\Big)	
	}
	\Say{[6]}{\Elim \TYPE{SecondCountable}(X)\Elim \V}{|\V| = \aleph_0}
	\Say{V}{\FUNC{enumerate}(\V)}{\Nat \ToInj \V}
	\AssumeIn{n}{\Nat}
	\Conclude{U_{bn}}{V_n}{\Clopen(X) \And \NonEmpty}
	\DeriveConclude{[b.*]}{\LOGIC{Define}}{\forall n \in \Nat \. \LOGIC{Defined}(U_{bn})}
	\DeriveConclude{[*]}{\Intro \TYPE{AlexandrovUryshonSchema}}
	{
		\TYPE{AlexandrovUryshonSchema}(X,U)
	}
	\EndProof
	\\
	\Theorem{AlexandrovUryshonSchemaDomain}
	{
		\forall X \in \MS \. 
		\forall U : \TYPE{AlexandrovUryshonSchema}(X) \.
		D(U) = X
	}
	\NoProof
	\\
	\Theorem{AlexandrovUryshonSchemaAssociatesHomeomrphism}
	{
		\NewLine ::
		\forall X \in \MS \. 
		\forall U : \TYPE{AlexandrovUryshonSchema}(X) \.
		\TYPE{Homeomorphism}\Big(\B,X, f_U\Big)
	}
	\NoProof
	\\
	\Theorem{BairSpaceTopChar}
	{
		\forall X \in \TOP \,
		\Polish \And \NonEmpty \And \TYPE{Wiry}(X) \And \dim_\TOP X = 0
		\iff
		X \cong_\TOP \B
	}
	\NoProof
}
\newpage
\subsubsection{Universality of Bair space}
\Page{
	\Theorem{EmbeddingInABairSpace}
	{
		\forall X : \TYPE{Polish} \.
		\dim_\TOP X = 0
		\Imply
		\NewLine
		\Imply
		\exists f : \TYPE{TopologicalEmbedding}(X,\B) \.
		\TYPE{Closed}\Big( f(X), \B \Big)
	}
	\Say{\Big(d,[1]\Big)}
	{
		\Elim \TYPE{Polish}(X) 
	}
	{
		\sum d : \TYPE{Metric}(X) \. 
		\NewLine \.
		(X,\alpha) \cong_\TOP X \And \Complete(X,d)  
	}
	\Say{\Big(U,[2]\Big)}{\Elim \TYPE{ZeroDimensional}(X)}
	{
		\sum U : \TYPE{VanishingDiameter}(X,d) \. 
		\forall t \in \FS{\Nat} \.
		\Clopen(X,U_t)
	}
	\Say{[3]}{\THM{ClosedLusinDomain}[2]}{\Closed\Big(\B,D(U)\Big)}
	\Say{[4]}{\THM{AssociatedMapEmbeddingCondition}[2]}{\TYPE{TopologicalEmbedding}\Big(D(U),X,f_U\Big)}
	\Say{[5]}{\Elim U}{f(\B) = X}
	\Conclude{[*]}{[4][5]}{\TYPE{ToplogicalEmbedding}(X,\B,f_U^{-1})}
	\EndProof
	\\
	\Theorem{EmbeddingInABairSpace}
	{
		\forall X : \Polish \.
		\dim_\TOP X = 0
		\Imply
		\NewLine
		\Imply
		\exists f : \TYPE{TopologicalEmbedding}(X,\C) \.
		f(X) \in G_\delta\Big( \C \Big)
	}
	\NoProof
}\Page{
	\Theorem{BairImageTHM}
	{
		\forall X : \Polish \.
		\exists A : \Closed(\B) :
		\exists \TYPE{Continuous} \And \TYPE{Bijective}(A,X)
	}
	\Say{\Big(d,[1]\Big)}
	{
		\Elim \TYPE{Polish}(X) 
	}
	{
		\sum d : \TYPE{Metric}(X) \. 
		\NewLine \.
		(X,\alpha) \cong_\TOP X \And \Complete(X,d)  
	}
	\SayIn{B_\emptyset}{X}{F_\sigma(X)}
	\AssumeIn{s}{\FS{\Nat}}
	\AssumeIn{B_s}{F_\sigma(X)}
	\Say{\Big(C,[3]\Big)}{\Elim F_\sigma(X)}
	{
		\sum \Nat \Arrow{C} \Closed(X) : \POSET \.
		B_s = \bigcup^\infty_{n=1} C_n
	}
	\Say{\Big(E,[4]\Big)}{\THM{FSigmaClosedDifferenceDecomp}(C,2^{-1-\len(s)})}
	{
			\sum E : \Nat  \to \Nat \to F_\sigma(X) \.
			\NewLine
			\Big(
			\forall n \in \Nat \. C_{n+1} \setminus C_n = \bigcup^\infty_{i=1} E_{n,i}
			\Big)
			\And \NewLine \And
			\Big(
				\forall n,m \in \Nat \.
				\forall i,j \in \Nat \.
				(n,i) \neq (m,j) 
				\Imply
				E_{n,i} \cap E_{m,j} = \emptyset
			\Big)
			\And \NewLine \And
			\forall n,m \in \Nat \.
			\diam E_{n,m} < 2^{-1 - \len(s)}
	}
	\Say{(n,m)}{\FUNC{enumerate}(\Nat \times \Nat)}{\Nat \ToBij \Nat \times \Nat}
	\AssumeIn{k}{\Nat}
	\Conclude{B_{sk}}{E_{n_k,m_k}}{F_\sigma(X)}
	\DeriveConclude{[s.*]}{\Intro \forall}{\forall k \in \Nat \. B_{sk} \in F_\sigma(X)}
	\Derive{\Big(B,[3]\Big)}{\Intro \TYPE{LusinSchema}}
	{
		\sum B : \TYPE{LusinSchema}(X) \.
		\Big(
			\forall s \in \FS{\Nat} \.
			B_s \in F_\sigma(X) 
		\Big)
		\And 
		\NewLine
		\And
		\Big(
			\forall s \in \FS{\Nat} \.
			\diam B_s \le 2^{-1-\len(s)}
		\Big)
		\And
		\NewLine
		\And
		\Big(
			\forall s,t \in \FS{\Nat} \.
			s \subset t \Imply
			\overline{B}_t \subset B_t
		\Big)
		\And
		\NewLine
		\And
		\Big(
			\forall s \in \FS{\Nat}
			B_t = \bigcup^\infty_{k=1} \overline{B}_{tk}
		\Big)
	}
	\Say{[4]}{\Intro D(B) [3.4]}{ f_B\big(D(B)\big) = X  }
	\Say{[5]}{\THM{AssociationMapisContinuousInjecttion}[4]}
	{
		\TYPE{Continuous} \And \TYPE{Bijection}(D(B),X,f)
	}
	\Assume{x}{\Nat \to D(B)}
	\AssumeIn{L}{\B}
	\Assume{[6]}{L = \lim_{n \to \infty} x_n}
	\Say{[7]}{\THM{ConvergenIsCauchy}\Big(D(B),x,[6]\Big)}{\TYPE{Cauchy}\Big( D(B),x \Big)}
	\Say{[8]}{\THM{ContinuousPreservesCauchy}}{\TYPE{Cauchy}\Big((X,d) ,f(x) \Big) }
	\Say{[9]}{\Elim \Complete(X,d)[8][1]}
	{
		\TYPE{Converging}\Big( X,f(x) \Big)
	}
	\Say{y}{\lim_{n \to \infty} f(x_n)}{X}
	\Say{[10]}{\Elim y [3.3]}
	{
		y \in \bigcap \overline{B_{\inits{L}{n}}}
	}
	\Conclude{[x.*]}{\Elim D(B) [10]}
	{
	 	L \in D(B)
	}
	\DeriveConclude{[*]}{\THM{ClosedByLimits}}
	{
		\Closed\Big( \B, D(B) \Big)
	}
	\EndProof
	\\
	\Theorem{BairExtensionTHM}
	{
		\forall X : \Polish \.
		\exists \TYPE{Continuous} \And \TYPE{Surjective}(A,X)
	}
	\NoProof
}
\newpage
\subsubsection{Bair space as subset}
\Page{
	\Theorem{HurwitzCriterion}
	{
		\forall X : \Polish \.
		\Big(
			\exists A : \Closed(X) : 
			A \cong_\TOP \B
		\Big)
		\iff
		\neg \SCompact(X)
	}
	\Assume{A}{\Closed(X)}
	\Assume{[1]}{A \cong_\TOP \B}
	\Say{[2]}{\THM{BairSpaceIsNotSigmaCompact}[1]}
	{
		\neg \SCompact\Big( A \Big)
	}
	\Conclude{[*]}{\THM{CompacClosedSubset}(X,A)[2]}{\neg \SCompact\Big( X \Big)}
	\Derive{[1]}{\Imply(\Arrow)}
	{
		\Big(
			\exists A \subset X : 
			A \cong_\TOP \B
		\Big)
		\Imply
		\neg \SCompact(X)	
	}
	\Assume{[2]}{\neg\SCompact(X)}
	\Say{C_\emptyset}{X}{\TYPE{Closed}(X)}
	\AssumeIn{s}{\FS{\Nat}}
	\Assume{C_s}{\Closed \And \NonEmpty(X) \And \neg \SCompact}
	\Say{H}{ \bigg\{ x \in C_s : \forall U \in \U(x) \. \neg \SCompact\Big(\overline{U \cap C_s} \Big)   \bigg\}   }
	{
		?C_s
	}
	\Assume{h}{\TYPE{Converging}(H)}
	\SayIn{L}{\lim_{n \to \infty} h_n}{C_s}
	\Assume{U}{\U(L)}
	\Say{\Big(n,[4]\Big)}{\THM{NbhdConvergence}(h,L,U)}{\sum^\infty_{n=1} h_n \in U}
	\Conclude{[U.*]}{\Elim H (h_n)\Big(U,[4]\Big)}{\neg \SCompact\Big( \overline{U \cap C_s} \Big)}
	\DeriveConclude{[h.*]}{\Elim H}{L \in H}
	\Derive{[4]}{\THM{ClosedByLimits}(X)}{\Closed(X,H)}
	\Say{[5]}{\Elim \TYPE{NonEmpty}(C_s) \Elim \TYPE{SecondCountable}(X) \Elim H \Elim \SCompact}
	{
		H \neq \emptyset
	}
	\Say{[6]}{\Elim H \Intro \SCompact \Intro \FUNC{setminus}}
	{
		\SCompact\Big( C_s \setminus H \Big)
	}
	\Say{[7]}{\Elim \neg\SCompact(C_s)[6]}{\neg \Compacts\Big(X,H\Big)}
	\Say{\Big( h ,[8] \Big)}{\THM{CompactIffSequencCompact}(H)}
	{
		\sum h : \Nat \to H \. \TYPE{ConvergingSubsequence}\Big(X,h\Big) = \emptyset
	}
	\Say{\Big(U,[9]\Big)}
	{
		\Elim \TYPE{ConvergingSubsequence}(X,h)[8] 
		\THM{NbhdConvergence}
	}
	{
		\NewLine \. 
		\sum U : \prod^\infty_{n=1} \U(h_n) \.
		\Big(
			\forall n \in \Nat \.
			\diam U < 2^{-1-\len(s)} 
		\Big)
		\And
		\Big(
			\forall n,m \in \Nat \.
			n = m 
			\Imply
			\overline{U}_n \cap \overline{U}_m = \emptyset
		\Big)
	}
	\AssumeIn{n}{\Nat}
	\Conclude{C_{sn}}{\overline{C_s \cap U_n}}{\Closed \And \NonEmpty(X) \And \neg \SCompact}
	\DeriveConclude{[n.*]}{\Intro \forall }{ \forall n \in \Nat \. \Closed \And \NonEmpty(X) \And \neg \SCompact(C_{sn})}
	\Derive{\Big(C,[3]\Big)}{\Intro \TYPE{VanishingDiameter} \Intro \sum}
	{
		\NewLine : 
		\sum C : \TYPE{VanishingDiameter}(X) \.
		\forall s \in \FS{\Nat} \. 
		\Closed \And \NonEmpty(X) \And \neg \SCompact(C_{sn})
	}
	\Say{[4]}{\Elim C \Intro D(C) [3.1][3.2] }{D(C) = \B}
	\Say{A}{f_C(\B)}{?X}
}\Page{
	\AssumeIn{L}{\overline{A} }
	\Say{\Big(a,[4]\Big)}{\Elim A [4]}
	{
		\sum a \in \Nat \to A \. 
		L = \lim_{n \to \infty} a_n
	}
	\Say{ \Big( b, [5]  \Big) }
	{
		\Elim A (a)
	}
	{
		\sum b : \Nat \to \B \. 
		a = f_C(b)
	}
	\Say{[6]}{\THM{ConvergingIsCauchy}(X,a)}{\TYPE{Cauchy}(X,a)}
	\AssumeIn{\varepsilon}{\Reals_{++}}
	\Say{\Big( n,[7] \Big)}{\Elim\Type{Archemedian}(\Nat)(\varepsilon)}
	{
		\sum n \in \Nat \. 2^{-n} < \varepsilon
	}
	\Say{\Big(\delta,[8]\Big)}
	{
		\Elim C (n)
	}
	{
		\sum \delta \in \Reals_{++} \.
		\forall s \in \FS{\Nat} \.
		\len s = n 
		\Imply
		\forall x \in C_s \.
		\forall y \in A \.
		d(x,y) < \delta \Imply y \in C_s
	}
	\Say{\Big( N,[9]\Big)}{\Elim \TYPE{Cauchy}(X,a)(\delta}
	{
		\sum N \in \Nat \. 
		\forall n ,m \in \Nat \. 
		n,m \ge N 
		\Imply
		d(a_n,a_m) < \varepsilon
	}
	\Conclude{\Big(s,[\varepsilon.*]\Big)}{[9][8][5]\Elim C}
	{
		\sum s \in \FS(\Nat)\.
		\len(s) = n 
		\And
		\forall k \ge N \.
		\inits{b}{k} = s
	}
	\Derive{[7]}{\Intro \TYPE{Cauchy}}{\TYPE{Cauchy}(\B,b)}
	\SayIn{b'}{\lim_{n=1} b_n}{\B}
	\Say{[8]}{\THM{ContinuousImage}(f_C)\Elim b'[5][4]}{ L = f_C(b')}
	\Conclude{[L.*]}{\Elim A [8]}{L \in A}
	\Derive{[4]}{\THM{ClosedByLimits}}{\Closed(X,A)}
	\AssumeIn{s}{\FS{\Nat}}
	\Say{[5]}{\Elim C[3.1](s)} 
	{
		\forall b \in  N_s \. 
		\exists U \in \U(f_C(b)) \.
		U \subset C_s
	}
	\Conclude{[s.*]}{\Elim C [5]}{f_C(N_s) \in \T(A)}
	\DeriveConclude{[*]}{\Intro \TYPE{Homeo}[3]}{\B \ToIso{f_C} A : \TOP}
	\EndProof
}
\newpage
\subsection{Baire Category and Topological Games }
\subsubsection{Recap}
\Page{
	\DeclareType{\ND}{\prod_{X \in \TOP} ??X}
	\DefineType{A}{\ND}{\intx \overline{A} = \emptyset} 
	\\
	\DeclareType{\Meager}{\prod_{X \in \TOP} ??X}
	\DefineType{B}{\Meager}{\exists A : \Nat \to \ND(X) \. B = \bigcup^\infty_{n=1} A_n }
	\\
	\DeclareType{\Comeager}{\prod_{X \in \TOP} ??X}
	\DefineType{A}{\Comeager}{\Meager(X,A^\c) }
	\\
	\DeclareType{SetIdeal}{\prod_{X \in \SET} ???X}
	\DefineNamedType{I}{SetIdeal}{I:\Ideal(X)}
	{
		\Big(\emptyset \in I\Big) \And
		\Big( \forall A \in I \. \forall B \subset A \. B \in I  \Big) \And
		\Big( \forall A,B \in I \. A \cup B \in I \Big)
	}
	\\
	\DeclareType{SetSigmaIdeal}{\prod_{X \in \SET} ?\Ideal(X)}
	\DefineNamedType{I}{SetSigmaIdeal}{I:\Ideal(X)}
	{
		\forall A : \Nat \to I \. \bigcup^\infty_{n=1} A_n \in I 
	}
	\\
	\DeclareType{Bair}{?\TOP}
	\DefineType{X}{Bair}{\forall A : \Comeager(X) \. \TYPE{Dense}(X,A)}
	\\
	\Theorem{BairCategoryTHM}{ \forall X : \LCompact \And \TYPE{T2} \. \Bair(X)}
	\NoProof
	\\
	\Theorem{MetricBairCategoryTHM}{ \forall X : \Complete \. \Bair(X)}
	\NoProof
	\\
}
\newpage
\subsubsection{Choquet game}
\Page{
	\Conclude{\IIPG}{\Lambda X \in \SET \. \sum T : \TYPE{Pruned}(X) \. [T] \to \Bool }{\SET \to \Type}
	\\
	\DeclareType{\FPS}{\prod (T,w) : \IIPG \. \NewLine \. ?\Big(\TYPE{Subtree} \And \TYPE{NonEmpty}(T)\Big)}
	\DefineType{S}{\FPS}{
		\forall s \in S \.
		\TYPE{Even}(\len(s)) \Imply
		\Big|\{ x \in X :  sx \in S \}\Big| = 1
		\And \NewLine \And
		\TYPE{Odd}(\len(s)) \Imply
		\{ x \in X :  sx \in S \} = \{ x \in X : sx \in T \}
	}
	\\
	\DeclareType{\FPWS}{\prod (T,w) : \IIPG \. \NewLine \.  ?\FPS(T,w)}
	\DefineType{S}{\FPWS}{ \forall x \in [S] \. w(x) = 1}
	\\
	\DeclareType{\SPS}{\prod (T,w) : \IIPG \. \NewLine ?\Big(\TYPE{Subtree} \And \TYPE{NonEmpty}(T)\Big)}
	\DefineType{S}{\SPS}{
		\forall s \in S \.
		\TYPE{Odd}(\len(s)) \Imply
		\Big|\{ x \in X :  sx \in S \}\Big| = 1
		\And \NewLine \And
		\TYPE{Even}(\len(s)) \Imply
		\{ x \in X :  sx \in S \} = \{ x \in X : sx \in T \}
	}
	\\
	\DeclareType{\SPWS}{\prod (T,w) : \IIPG \. \NewLine \. ?\SPS(T,w)}
	\DefineType{S}{\SPWS}{ \forall x \in [S] \. w(x) = 0}
	\\
	\DeclareFunc{legalPosition}{\prod_{X \in \SET} \IIPG(X) \to \TYPE{Pruned}(X)}
	\DefineNamedFunc{legalPositions}{(T,w)}{\lp(T,w)}{T}
	\\
	\DeclareFunc{winningCriterion}{\prod_{X \in \SET} \prod (T,w) : \IIPG(X) \. [T] \to \Bool}
	\DefineNamedFunc{winningCriterion}{}{w_{(T,w)}}{w}
	\\
	\DeclareFunc{gameOfChoquet}{\prod X \in \TOP \And \TYPE{NonEmpty} \.  \IIPG\Big(\T(X)\Big)}
	\DefineNamedFunc{gameOfChoquet}{}{\Game_{Ch}(X)}
	{
		\Bigg(
			\bigcup_{n=0}^\infty \TYPE{Decreasing}\Big([1,\ldots,n],\T \And \TYPE{NonEmpty}(X) \Big),
		\NewLine
		\Lambda U : \Nat \to \T(X) \. \bigcap^\infty_{n=1} U_n != \emptyset
		\Bigg)
	}
}
\Page{
	\Theorem{OxtobyChoquetTHM}
	{
		\forall X \in \TOP \.
		X \neq \emptyset
		\Imply \Big(
		\neg\exists \FPWS\big( \Game_{Ch}(X) \big) 
		\iff
		\Bair(X)
		\Big)
	}
	\Assume{[1]}{\neg\exists \FPWS\big( \Game_{Ch}(X) \big)}
	\Assume{[2]}{\neg \Bair(X)}
	\Say{\Big(U,[3]\Big)}{\THM{EqBairProperty}[2]}
	{
		\sum  U : \Nat \to \T \And \TYPE{Dense}(X) \.
		\bigcap^\infty_{n=1} U = \emptyset
	}
	\Say{T_0}{\{\emptyset\}}{?\FS{\T(X)}}
	\AssumeIn{n}{\Nat}
	\Assume{[4]}{\TYPE{Odd}}
	\Say{\Big(k,[5]\big)}{\THM{OddnesCriterion}[4]}
	{
		\sum k \in \Int_+ \. n = 2k + 1
	}
	\Conclude{T_n}{ 
		\If k == 0
		\Then  \{ 1 \mapsto U_1 \}
		\Else
		\Big\{ s(s_{2k} \cap U_k  ) \Big| s \in T_{2k}    \Big\}
	}{ ?\T^n(X) }
	\Derive{[4]}{\Intro \Imply}
	{
		\TYPE{Odd}(n) \Imply T_n \in ?\T^n(X)
	}
	\Assume{[5]}{\TYPE{Even}(n)}
	\Conclude{T_n}{ 
		\Big\{ sV \Big| s \in T_{n-1}, V \in \T(X) \And V \subset s_{n-1}  \Big\}
	}{ ?\T^n(X) }
	\Derive{[5]}{\Intro \Imply}
	{
		\TYPE{Even}(n) \Imply T_n \in ?\T^n(X)
	}
	\Conclude{[*]}{\Elim(|)\THM{OddOrEven}[4][5]}
	{
		T_n \in ?\T^n(X)		
	}
	\Derive{T}{\Intro \prod}{\prod^\infty_{n=0} ?\T^n(X)}
	\Say{S}{\bigcup^\infty_{n=0} T_n}{?\FS{\T(X)}}
	\Say{[4]}{\Elim S}{\FPS\Big(\Game_{Ch}(X),S\Big)}
	\AssumeIn{V}{[S]}
	\Conclude{[V.*]}{\Elim S [3]}
	{
		\bigcap^\infty_{n=1} V_n \subset \bigcap^\infty_{n=1} U_n = \emptyset
	}
	\DeriveConclude{[5]}{\Elim \Game_{Ch}(X)\Intro \FPWS}
	{
		\FPWS(\Game_{Ch}(X),S)
	}
	\Conclude{[2.*]}{[1](S)}{\bot}
	\DeriveConclude{[1.*]}{\LOGIC{LEM}}{\Bair(X)}
	\Derive{[1]}{\Intro \Imply}
	{
		\Big(\neg \exists  \FPWS\big(\Game_{Ch}(X)\big) \Big)
		\Imply
		\Bair(X)
	}
	\Assume{[2]}{\Bair(X)}
	\Assume{S}{\FPWS\big(\Game_{Ch}(X)\big)}
	\Say{\Big(U,[3]\Big)}{\Elim \FPS\big(\Game_{Ch}(X),S\big)\Elim \TYPE{Singleton}}
	{
		\sum U \in \T(X) \.  \{ s \in S : \len(s) = 1 \} = \{1 \mapsto U\}
	}
	\Say{[4]}{\Elim \Game_{Ch}(X) \Elim U}{U \neq \emptyset}
	\Say{T_0}{\{\emptyset\}}{?\FS{\T(X)}}
	\AssumeIn{n}{\Nat}
	\Assume{[5]}{\TYPE{Odd}}
	\Conclude{T_n}{ 
		\If n  == 0
		\Then  \{ 1 \mapsto U \}
		\Else
		\Big\{ S(s) \Big| s \in T_{2k}    \Big\}
	}{ ?\T^n(X) }
	\Derive{[5]}{\Intro \Imply}
	{
		\TYPE{Odd}(n) \Imply T_n \in ?\T^n(X)
	}
}\Page{
	\Assume{[6]}{\TYPE{Even}(n)}
	\AssumeIn{p}{S}
	\Assume{[7]}{\len(p) = n-1}
	\Say{V}{p_{n-1}}{\T(X) \And \TYPE{NonEmpty}}
	\Say{\V}{
		\max \Big\{ 
			\U : ?\Big(\T(X) \And ?V \And \TYPE{NonEmpty}\Big) : 
			\forall W,W' \in \U \. S(sW) \cap S(sW') = \emptyset   
		\Big\}
	}
	{
		?\T(X)
	}
	\Conclude{t_s}{\{ sW | W \in \V \}}{?\T^n(X)}
	\Derive{t}{\Intro \to}{S_{n-1} \to ?\T^n(X)}
	\Conclude{T_n}{\bigcup_{s \in S_{n-1}} t_s}{ ?\T^n(X) }
	\Derive{[5]}{\Intro \Imply}
	{
		\TYPE{Even}(n) \Imply T_n \in ?\T^n(X)
	}
	\Conclude{[*]}{\Elim(|)\THM{OddOrEven}[4][5]}
	{
		T_n \in ?\T^n(X)		
	}
	\Derive{T}{\Intro \prod}{\prod^\infty_{n=0} ?\T^n(X)}
	\Say{S'}{\bigcup^\infty_{n=0} T_n}{?S}
	\Say{[5]}{\Elim S'}
	{
		\forall n : \TYPE{Even} \.
		\forall s,t \in S' \.
		\len(s) = n = \len(t) \And s \neq t
		\Imply
		s_n \cap t_n = \emptyset
	}
	\Say{[6]}{\Elim S'\Elim \max}
	{
		\forall n : \TYPE{Even} \.
		\forall U \in S' \.
		\len(U) = n \Imply 
		\TYPE{Dense}\Big(U_n,\bigcup\{ V_{n+2} | V \in S' \And \len(V) = n + 2 \} \Big)
	}
	\Say{[7]}{\Elim S' \Intro \TYPE{Pruned}}{\TYPE{Pruned}\big(\T(X),S'\big)}
	\SayIn{W}{
		\Lambda n \in \Nat \. 
		\bigcup \{ s_n |  s \in S', \len(s) = n \}
	}
	{
		\T(X)
	}
	\Say{[8]}{\Elim W \Elim S' [6]}{\forall n \in \Nat \. \TYPE{Dense}\Big(U,W_n\Big)}
	\Say{[9]}{\Elim W [7][5] \Elim \FPWS\big(\Game_{Ch}(X),S\big) \Elim \FUNC{union}}
	{
		\NewLine :
		\bigcap_{n=1}^\infty W_n = 
		\bigcap_{n=0}^\infty \bigcup_{s \in S'_{2n+1}} s_{2n+1}  =
		\bigcup_{x \in |S'|} \bigcap^\infty_{n=1} s_{2n +1} = 
		\bigcup_{x \in |S'|} \emptyset =
		\emptyset
	}
	\Conclude{[2.*]}{\Elim \Bair(X)\THM{BairOpenSubsets}[9]}{\bot}
	\Derive{[2]}{\Intro \Imply}
	{
		\Bair(X)
		\Imply
		\neg \exists \FPWS\Big( \Game_{Ch}(X),S\Big)
	}
	\Conclude{[*]}{\Intro \iff [1][2]}
	{
		\neg\Big( \exists\Big( \Game_{Ch}(X),S\Big) 
		\iff
		\Bair(X)
	}
	\EndProof
	\\
	\DeclareType{ChoquetSpace}{?(\TOP \And \NonEmpty)}
	\DefineType{X}{ChoquetSpace}{\exists \SPWS\Big( \Game_{Ch}(X) \Big)}
	\\
	\Theorem{ChoquetIsBair}
	{
		\forall X : \CS \. \Bair(X)
	}
	\NoProof
}
\Page{
	\Theorem{ChoquetSpaceProduct}
	{
		\forall X,Y : \CS \.
		\CS(X \times Y)
	}
	\NoProof
	\\
	\Theorem{ChoquetSpaceOpenSubsets}
	{
		\forall X : \CS \.
		\forall U \in \T(X) \.
		U \neq \emptyset \Imply
		\CS(U)
	}
	\NoProof
	\\
	\DeclareFunc{strongChoquetGame}
	{
		\prod X \in \TOP \And \TYPE{NonEmpty} \. \IIPG\left( \sum_{U \in \T(X)} U \right)
	}
	\DefineNamedFunc{strongChoquetGame}{}{\Game_{sCh}(X)}
	{
		\Bigg(
			\bigcup_{n=0}^\infty
			\bigg\{
				(U,x) : [1,\ldots, n] \to \sum_{U \in \T(X)} U : 
				\TYPE{Decreasing}\Big( [1,\ldots,n],\T(X) \Big) 
				\And \NewLine \And
				\forall k \in [1,\ldots,n] \. \TYPE{Even}(k) \Imply x_{k} = x_{k-1}
			\bigg\},
			\Lambda (U,x) : \Nat \to \sum_{U \in \T(X)} U \. \bigcap_{n=1} U_n = \emptyset
		\Bigg)
	}
	\\
	\DeclareType{StrongChoquetSpace}{?(\TOP \And \TYPE{NonEmpty}) }
	\DefineType{X}{StrongChoquetSpace}{\exists \SPWS\Big( \Game_{sCh}(X) \Big)}
	\\
	\Theorem{StrongChoquetIsChoquet}
	{
		\forall X : \SCS \. \CS(X)
	}
	\NoProof
	\\
	\Theorem{ChoquetCategoryTHM}{ \forall X : \LCompact \And \TYPE{T2} \. \SCS(X)}
	\NoProof
	\\
	\Theorem{MetricChoquetCategoryTHM}{ \forall X : \Complete \. \SCS(X)}
	\NoProof
	\\
	\Theorem{StrongChoquetGDeltaSubsets}
	{
		\forall X : \SCS(X) \.
		\forall A \in G_\delta(X) \.
		A \neq \emptyset \Imply \NewLine \Imply 
		\SCS(A)
	}
	\NoProof
	\\
}\Page{
	\Theorem{StrongChoquetMapping}
	{
		\forall X : \SCS(X) \.
		\forall Y \in \TOP \.
		\NewLine : 
		\forall f : \TYPE{Surjective} \And \TYPE{Open}(X,Y) \.
		\SCS(Y)
	}
	\NoProof
}
\newpage
\subsubsection{Characterization of polish spaces}
\Page{
	\Theorem{OxtobyPolishCharTHM}
	{
		\forall X : \Polish \And \NonEmpty \. 
		\forall D : \Dense(X) \. \NewLine \. 
		\CS(D) \iff \Comeager(X,D)
	}
	\Say{\Big(d,[1]\Big)}
	{
		\Elim \TYPE{Polish}(X) 
	}
	{
		\sum d : \TYPE{Metric}(X) \. 
		\NewLine \.
		(X,\alpha) \cong_\TOP X \And \Complete(X,d)  
	}
	\Assume{[2]}{\CS(D)}
	\Say{S}{\Elim \CS(D)}{\SPWS\Big( \Game_{Ch}(D) \Big)}
	\Say{\Big(S',[3]\Big)}{\Elim \Dense(D,X) \Elim \SPS\Big( \Game_{Ch}(D),S\Big)}
	{
		\sum S' : \TYPE{Pruned}(\T(X)) \. 
		S \neq \emptyset
		\And \NewLine \And
		\left(
		\forall s \in S' \.
		\prod^{\len(s)}_{i=1} (s_i \cap D) \in S 
		\right)
		\And \NewLine \And
		\left(
			\forall s \in S' \.
			\TYPE{Even}(\len s) \Imply
			\TYPE{Disjoint}\{ V \in \T(X) : \exists U \in \T(X) : sUV \in S'\}
		\right)
		\And \NewLine \And
		\left(
			\forall s \in S' \.
			\TYPE{Even}(\len s) \Imply
			\Dense\left(s_{\len s},\bigcup\{ V \in \T(X) : \exists U \in \T(X) : sUV \in S'\}\right)
		\right)
		\And \NewLine \And
		\left(
			\forall s \in S' \.
			\TYPE{Even}(\len s) \Imply
				\diam s_{\len s} <   2^{-\len s}
		\right)
	}
	\Say{W}{\Lambda n \in \Nat \. \bigcup_{s \in S'_{2n}} s_{2n}}
	{
		\Nat \to \T(X)
	}
	\Say{[4]}{\Elim W [3.1][3.4]}{\forall n \in \Nat \. \Dense\Big( X,W_n)}
	\AssumeIn{x}{\bigcap^\infty_{n=1} W_n  }
	\Say{\Big( U, [5] \Big)}{\Elim W(x) \Elim S'}
	{
		\sum U \in [S'] \. 
		\forall n \in \Nat \. x \in U_n
	}
	\Say{[6]}{\Intro \FUNC{intersect}[5]}{x \in \bigcap^\infty_{n=1} U_n}
	\Say{[7]}{[3.5][6]}{\{x\} = \bigcap^\infty_{n=1} U_n}
	\Conclude{[x.*]}{[7][3.2]\Elim \SPWS\Big(\Game_{Ch}(X),S\Big)}
	{
		x \in D
	}
	\Derive{[5]}{\Intro \subset}{\bigcap^\infty_{n=1} W_n \subset D}
	\Conclude{[2.*]}{\THM{ComeagerByDenseOpenIntersect}\Big(X,D,W,[5]\Big)}{\Comeager(X,D)}
	\Derive{[2]}{\Intro \Imply}
	{
		\CS(D) 
		\Imply
		\Comeager(X,D)
	}
	\Say{[3]}{\Intro \Imply \Intro \CS(D) \Intro \SPWS\Big(\Game_{Ch}(D)\Big)\Elim \Comeager(X,D) }
	{
		\Comeager(X,D)
		\Imply
		\CS(D)
	}
	\Conclude{[*]}{\Intro(\iff)[2][3]}
	{
		\Comeager(X,D)
		\iff
		\CS(D)
	}
	\EndProof
	\\
	\DeclareType{PointFiniteRefinement}{\prod_{X \in \TOP} \prod_{\U : ?\T(X)} \TYPE{Refinement}(X,\U)}
	\DefineType{\V}{PointFiniteRefinement}
	{
		\forall x \in X \.  
		\Big| \{ V \in \V : x \in V \} \Big| < \infty 
	}
}\Page{
	\Theorem{SeparableMetricSpaceHasSmallPointFiniteRefinements}
	{
		\NewLine ::
		\forall X \in \MS \And \TYPE{Separable} \.
		\forall \U : ?\T(X) \. 
		\forall \varepsilon \in \Reals_{++} \.
		\exists \V : \TYPE{PointFreeRefinement}(X,\U) \.
		\NewLine \. 
		\forall V \in \V \.
		\diam V < \varepsilon
	}
	\NoProof
	\\
	\Theorem{ChoquetPolishCharTHM}
	{
		\forall X : \Polish \And \NonEmpty \. 
		\forall D : \Dense(X) \. \NewLine \. 
		\SCS(D) \Imply \Polish(D)
	}
	\Say{\Big(d,[1]\Big)}
	{
		\Elim \TYPE{Polish}(X) 
	}
	{
		\sum d : \TYPE{Metric}(X) \. 
		\NewLine \.
		(X,\alpha) \cong_\TOP X \And \Complete(X,d)  
	}
	\Say{S}{\Elim \SCS(D)}{\SPWS\Big( \Game_{sCh}(D) \Big)}
	\Say{\Big(S',[2]\Big)}{
		\THM{SeparableMetricSpaceHasSmallPointFiniteRefinements}(X,S)
	}
	{
		\sum S' : \TYPE{\Pruned}\left( \prod_{U \in \T(X)} U \right) :	
		\NewLine :
		\left(
			\forall (U',x) \in S' \.
			\exists (U,x) \in S :
			U' \cap D = U
		\right)
		\And \NewLine \And
		\left(
			\forall (U',x) \in S' \.
			\forall i \in \Big[1,\ldots,\len(U',x)\Big] \.
			U'_{i+1} \subset U'_i 
		\right)
		\And \NewLine \And
		\Bigg(
			\forall (W,x) \in S' \.
			\TYPE{Even}\Big(\len (W,x)\Big) \Imply
			W_{\len s} \cap D \subset \bigcup 
			\NewLine
			\pi_1\left\{ 
				(V,y) \in \prod_{V \in \T(X)} V : 
				\exists (U,y) \in \prod_{U \in \T(X)} U : 
				(W,x)(U,y)(V,y) \in S'
			\right\}
		\Bigg)
		\And \NewLine \And
		\Bigg(
			\forall (W,x) \in S' \.
			\TYPE{Even}\Big(\len (W,x)\Big) \Imply
			\forall x' \in X \.
			\NewLine
			\left|\left\{ 
				(V,y) \in \prod_{V \in \T(X)} V : 
				\exists (U,y) \in \prod_{U \in \T(X)} U : 
				(W,x)(U,y)(V,y) \in S', x' \in V
			\right\}\right| < \infty
		\Bigg)
		\And \NewLine \And
		\left(
			\forall (U,x) \in S' \.
			\TYPE{Even}(\len U) \Imply
				\diam U_{\len U} <   2^{-\len U}
		\right)
	}
	\Say{W}{\Lambda n \in \Nat \. \bigcup_{(U,x) \in S'_{2n}} U_{2n}}
	{
		\Nat \to \T(X)
	}
	\Say{[3]}{\Elim \Nat \Elim W [2.1][2.3]}{X \subset \bigcap^\infty_{n=1} W_n}
	\AssumeIn{x}{\bigcap^\infty_{n=1} W_n}
	\Say{[4]}{\Elim W \Intro S'}{ | S'_x | = \infty  }
	\Say{[5]}{\Intro \TYPE{FiniteSplitting}[2.4]}{\TYPE{FiniteSplitting}(S'_x)}
	\Say{[6]}{\THM{K/"onigsLema}[5][4]}{[S_x'] \neq \emptyset}
	\SayIn{(U,y)}{\Elim \TYPE{NonEmpty}}{[S_x']}
	\Say{\Big[ (V,y), [7] \Big]}{[2.1](U,y)}
	{
		\sum (U,y) \in [S] \.
		V = U \cap D
	}
}\Page{
	\Say{[8]}{\Elim \SPWS\Big( \Game_{sCh},S\Big)}
	{
		\bigcap^\infty_{n=1} V_n \neq \emptyset
	}
	\Say{[9]}{
		\THM{IntersectionDistributivity}\left(X,\bigcap^\infty_{n=1} V_n\right)
		\THM{VanishingDiameterIntersection}[8][2.5][7]\Elim X
	}
	{
	       \NewLine : 
	       D \cap \bigcap^\infty_{n=1} W_n   =  
	       \bigcap^\infty_{n=1} W_n \cap D   =
	       \bigcap^\infty_{n=1} V_n = 
	       = \{x\}
	}
	\Conclude{[x.*]}
	{
		\Elim \FUNC{intersection}[9]
	}
	{
		x \in D
	}
	\Derive{[4]}{\Intro \subset [3]\Intro \TYPE{SetEq}}
	{ 
		D = \bigcap^\infty_{n=1} W_i
	}
	\Conclude{[*]}{\Intro G_\delta(X)\THM{PolishSubset}(X)}{\Polish(X)}
	\EndProof
}
\newpage
\subsubsection{Bair property}
\Page{
	\DeclareType{Bimeager}{\prod_{X \in \TOP} ?X \times ?X}
	\DefineNamedType{(A,B)}{Bimeager}{A =^* B}{\Eqmod{A}{B}{\Meager(X)}}
	\\
	\DeclareType{BairProperty}{\prod_{X \in \TOP} ?X}
	\DefineNamedType{B}{BairProperty}{B \in \BP(X)}{\exists U \in \T(X) \. B =^* U}
	\\
	\Theorem{BairPropertyAsSmallestSigmaAlgebra}
	{
		\forall X \in \TOP \. 
		\BP(X) = \sigma\Big( \T(X) \cup \TYPE{Meager}(X) \Big)
	}
	\Say{[1]}{\Elim \BP(X) \Elim \T(X) \Intro \emptyset}{\emptyset \in \BP(X)}
	\Assume{A}{\BP(X)}
	\Say{\Big(U,[2]\Big)}{\Elim \BP(X,A)}{\sum U \in \T(X) \. A =^* U}
	\Say{[3]}{\Elim(A =^* U)[2]}{ \Meager\Big(X,A \du U\Big) }
	\SayIn{V}{\intx U^\c}{\T(X)}
	\Say{[4]}
	{
		\Elim V
		\Elim \du
		\THM{InteriorSubset}
		\THM{InteriorClosureDecomposition}
		\Intro \du
	}
	{
		A^\c \du V = 
		A^\c \du \intx{U^\c} = \NewLine =
		\Big(A^\c \cap \intx^\c {U^\c}\Big)
		\cup
		\Big(A \cap \intx{U^\c}  \Big)
		\subset 
		\Big(A^\c \cap \intx^\c{U^\c} \Big)
		\cup
		\Big( A \cap U^\c  \Big)  = \NewLine = 
		\Big( A^\c \cap  (\overline{U}\setminus U) \Big) \cup
		\Big( A^\c \cap \intx^\c {U^\c} \Big) \cup
		\Big( A \cap U^\c \Big) =
		\Big( A^\c \cap (\overline{U}\setminus U) \Big) \cup
		A \du U
	}
	\Say{[5]}{[3][4]\THM{NowhereDenseResidual}(U)\THM{MeagerSubset}}
	{
		\Meager(X, A^\c \du V)
	}
	\Say{[6]}{\Intro \TYPE{Bimeager}[5]}
	{
		A^\c =^* V
	}
	\Conclude{[A.*]}{\Intro \BP(X)[6]}{A^\c \in \BP(X)}
	\Derive{[2]}{\Intro(\forall)}{\forall A \in \BP(X) \. A^\c \in \BP(X)}
	\Assume{A}{\Nat \to \BP(X)}
	\Say{\Big(U,[3]\Big)}{\Elim \BP(X,A)}{\sum U  : \T(X) \. \forall n \in \Nat \. A =^* U}
	\Say{[4]}{\Elim(A =^* U)[3]}{ \forall n \in \Nat \. \Meager\Big(X,A_n \du U_n\Big) }
	\Say{[5]}{\THM{DifferenceUnionSubset}(A,U)}
	{
		\bigcup^\infty_{n=1} A_n \du \bigcup^\infty_{n=1} U_n \subset
		\bigcup^\infty_{n=1} A_n \du U_n
	}
	\Say{[6]}{[5][4]\THM{MeagerCountableUnion}\THM{MeagerSubset}}
	{
		\Meager\left(X, \bigcup^\infty_{n=1} A_n \du \bigcup^\infty_{n=1} U_n \right)
	}
	\Say{[7]}{
		\Intro \TYPE{Bimeager}[6]
	}
	{
			\bigcup^\infty_{n=1} A_n =^* \bigcup^\infty_{n=1} U_n
	}
	\Conclude{[8]}{\Intro \BP(X) [7]}
	{
		\bigcup^\infty_{n=1} A_n \in \BP(X)
	}
	\Derive{[3]}{\Intro \forall}
	{
		\forall A : \Nat \to \BP(X) \.
		\bigcup^\infty_{n=1} A_n \in \BP(X)
	}
	\Say{[4]}{\Intro \SA[1][2][3]}{\SA\Big( \BP(X) \Big)}
	\Say{[5]}{\Intro \sigma \Elim \sigma \Elim \BP(X) [4]}
	{
		\sigma\Big( \T(X) \cup \Meager(X) \Big) \subset \BP(X)
	}
}\Page{
	\AssumeIn{B}{\BP(X)}
	\Say{\Big(U,[6]\Big)}{\Elim \BP(X)}{ \sum U \in \T(X) \. B =^* U }
	\Say{[7]}{\Elim (B =^* U)}{ \Meager(X, B \du U )  }
	\Say{[8]}{\THM{OneSideSubsetSymmetric}(B,U)}{B \setminus U \subset B \du U}
	\Say{[9]}{\THM{MeagerSubset}[8]}{\Meager(X,B \setminus U )}
	\Say{[10]}{\THM{MeagerSubset}[8]}{\Meager(X,U \setminus B )}
	\Say{[11]}{\THM{DifferenceDecomposition1}(B,U)\THM{IntersectDifferenceDecomposion}(U,B)}{
		\NewLine : 
		B = (U \cap B) \cup (B \setminus U)  =
		\Big( U \setminus (U \setminus B) \Big) \cup (B \setminus U) 
	}
	\Conclude{[B.*]}{\Elim \SA\Big( \sigma\big( \T(X) \cup \Meager(X))[9][10][11] \Big)}
	{
		B \in \sigma\Big( \T(X) \cup \Meager(X) \Big)
	}
	\Derive{[6]}{\Intro \subset}{ \BP(X) \subset \sigma\Big( \T(X) \cup \Meager(X) \Big)}
	\Conclude{[*]}{\Intro \TYPE{SetEq}[5][6]}
	{
		\BP(X) = \sigma\Big( \T(X) \cup \Meager(X) \Big)
	}
	\EndProof
	\\
	\Theorem{BPAsGDelta}
	{
		\forall X \in \TOP \. 
		\forall A \subset X \.
		A \in \BP(X) 
		\iff
		\exists E  \in G_\delta(X) :
		\exists M : \Meager(X) \.
		A = E \cup M
	}
	\NoProof
	\\
	\Theorem{BPAsFSigma}
	{
		\forall X \in \TOP \. 
		\forall A \subset X \.
		A \in \BP(X) 
		\iff
		\exists E  \in F_\sigma(X) :
		\exists M : \Meager(X) \.
		A = E \setminus M
	}
	\NoProof
	\\
	\Theorem{RealBPIsNotTrivial}
	{
		\BP(\Reals) \neq ?\Reals
	}
	\NoProof
}
\newpage
\subsubsection{Localization}
\Page{
	\DeclareType{Forces}
	{
		\prod_{X \in \TOP} \T(X) \to ?X
	}
	\DefineNamedType{A}{Forces}{\Lambda U \in \T(X) \. U \Vdash A}{\Meager\Big( U,U\setminus A\Big)}
	\\
	\Theorem{BairPropertyByForcing}
	{
		\forall X \in \TOP \.
		\forall A \in \BP(X) \.
		X \Vdash (X\setminus A) \Big| 
		\exists U \in \T(U) \.
		U \Vdash A
	}
	\Say{\Big(U,[1]\Big)}{\Elim \BP(X,A)}
	{
		\sum U \in \T(X) \.  \Meager\Big(X,A \du U\Big)
	}
	\Assume{[2]}{U \neq \emptyset}
	\Say{[3]}{\THM{OneSidedSymmetricDifferenceSubset}(U,A)\THM{MeagerSubset}}
	{
		\Meager(X,U \setminus A)
	}
	\Say{[4]}{\THM{SubsetMeager}[3]}{\Meager(U,U\setminus A)}
	\Conclude{[2.*]}{\Intro \Vdash [4]}{U \Vdash A}
	\Derive{[2]}{\Intro(\Imply)}
	{   
		U \neq \emptyset \Imply U \Vdash A
	}
	\Assume{[3]}{U = \emptyset}
	\Say{[4]}{[2][3]}{ \Meager(X,A)  }
	\Conclude{[3.*]}{\Intro \Vdash [4]}
	{
		X \Vdash  X \setminus A
	}
	\Derive{[3]}{\Intro(\Imply)}
	{   
		U = \emptyset \Imply X \Vdash (X \setminus A)
	}
	\Conclude{[*]}{\LOGIC{LEM}(U = \emptyset)[2][3]}
	{
		X \Vdash (X\setminus A) \Big| 
		\exists U \in \T(U) \.
		U \Vdash A	
	}
	\EndProof
	\\
	\Theorem{BairForcing}
	{
		\forall X : \Bair \.
		\forall A \in \BP(X) \.
		X \Vdash (X\setminus A) \oplus 
		\exists U \in \T(U) \.
		U \Vdash A
	}
	\NoProof
	\\
	\DeclareType{WeakBasis}{\prod_{X \in \TOP} ??\Big( \T(X) \And \NonEmpty(X) \Big)}
	\DefineType{\U}{WeakBasis}{\forall V \in \T(X) \And \NonEmpty(X) \. \exists U \in \U : U \subset V}
	\\
	\Theorem{WeakBasisBairPropertyForcing}
	{
		\forall X \in \TOP \.
		\forall \U : \TYPE{WeakBasis}(X) \.
		\forall A \in \BP(X) \. \NewLine \. 
		X \Vdash (X\setminus A) \Big| 
		\exists U \in \U \.
		U \Vdash A
	}
	\NoProof
	\\
	\Theorem{ForcingIntersection}
	{
		\forall X \in \TOP \.
		\forall A : \Nat \to X \.
		\forall U \in \T(X) \.
		U \Vdash \bigcap^\infty_{n=1} A_n
		\iff
		\forall n \in \Nat \. U \Vdash A_n
	}
	\NoProof
}\Page{
	\Theorem{ForcingComplement}
	{
		\forall X : \Bair \.
		\forall B \in \BP(X) \.
		\forall V \in \T(X) \.
		\forall \U  : \TYPE{WeakBasis} \.
		V \Vdash B^\c 
		\iff \NewLine \iff
		\forall U \in \U \And \TYPE{Subset}(V) \.
		U \not \Vdash B
	}
	\NoProof
	\\
	\Theorem{ForcingUnion}
	{
		\forall X : \Bair \.
		\forall B : \Nat \to \BP(X) \.
		\forall V \in \T(X) \.
		V \vdash \bigcup^\infty_{n=1} B_n 
		\iff
		\NewLine \iff
		\forall U \in \U \And \TYPE{Subset}(V)  \. 
		\exists n \in \Nat \.  
		\exists W \in \U \And \TYPE{Subset}(U) \.
		W \Vdash  B_n
	}
	\\
	\DeclareFunc{openApproximation}{\prod_{X \in \TOP} ?X \to \T(X)}
	\DefineNamedFunc{openApproximation}{A}{U_\Vdash(A)}
	{ \bigcup \{ U \in \T(X) : U \Vdash A  \}}
	\\
	\Theorem{MeagerInOpenApproximation}
	{
		\forall X \in \TOP \.
		\forall A \subset X \.
		\Meager\Big( X  ,U_\Vdash(A)\setminus A \Big)
	}
	\NoProof
	\Theorem{OpenApproximationWithBairProperty}
	{
		\forall X \in \TOP 
		\forall A \in \BP(X) \.
		\Meager\Big(X,A \setminus U_\Vdash(A) \Big)
	}
	\NoProof
	\\
	\Theorem{OpenApproximationBimeager}
	{
		\forall X \in \TOP \.
		\forall A \in \BP(X) \.
		U_\Vdash(A) =^* A
	}
	\NoProof
}\Page{
	\Theorem{OpenApproximationIsOpenDomain}
	{
		\forall X \in \TOP \. 
		\forall A \subset X \.
		\TYPE{OpenDomain}(X,U_\Vdash(A))
	}
	\Say{[1]}{\Elim \intx \Elim \FUNC{closure}}{U_\Vdash(A) \subset \intx \overline{U}_\Vdash(A)}
	\Say{[2]}{\THM{MeagerInOpenApproximation}(X,A)}{ \Meager\Big( X, U_\Vdash(A) \setminus A \Big)}
	\Say{[3]}{ [1] \THM{SubsetDifferenceDecomposition}  }{
		\intx \overline{U_\Vdash(A)} \setminus A = 
		\Big(U_\Vdash(A) \setminus A \Big) 
		\cup  
		\Big(
			\intx \overline{U_\Vdash(A)} \setminus \Big( A \cup U_\Vdash(A)  \Big)
		\Big)
	}
	\Say{[4]}
	{
		\THM{InteriorIsSubset}\Big( X,\overline{U_\Vdash(A)}  \Big)
	}
	{
		\intx \overline{U_\Vdash(A)} \subset \overline{U_\Vdash(A)}
	}
	\Say{[5]}
	{
		\THM{SubsetOfUnion}\Big(X,A,U_\Vdash(A)\Big)
	}
	{
		U_\Vdash(A) \subset X \setminus A
	}
	\Say{[6]}
	{
		\THM{DifferenceMonotonicity}[4][5]
	}
	{
		\intx \overline{U_\Vdash(A)} \setminus \Big( A \cup U_\Vdash(A) \Big)
		\subset
		\overline{U_\Vdash(A)} \setminus U_\Vdash(A)
	}
	\Say{[7]}
	{
		\THM{OpenHasMeagerBoundary}\Big(X,U_\Vdash(A)\Big)
	}
	{
		\Meager\Big(X, \overline{U_\Vdash(A)} \setminus U_\Vdash(A) \Big)
	}
	\Say{[8]}
	{
		\THM{MeagerSubset}[7][8]
	}
	{
		\Meager\Big( X,\intx \overline{U_\Vdash(A)} \setminus \big( A \cup U_\Vdash(A) \big) \Big)
	}
	\Say{[9]}{\THM{MeagerUnion}[3][2][8]}
	{
		\Meager(X,\intx \overline{U_\Vdash(A)} \setminus A)
	}
	\Say{[10]}{\Elim U_\Vdash(A) [9][1]}
	{
		U_\Vdash(A) = \intx \overline{U_\Vdash(A)}
	}
	\Conclude{[*]}{\Intro \TYPE{OpenDomain}[10]}{\TYPE{OpenDpmain}\Big( X , U_\Vdash(A) \Big) }
	\EndProof
	\\
	\Theorem{OpenApproximationIsUniqueForcingOpenDomain}
	{
		\forall X : \Bair \. 
		\forall A \in \BP(X) \.
		\forall U : \TYPE{OpenDomain}(X) \. \NewLine \. 
		U =^* A \Imply U = U_\Vdash(A)
	}
	\Say{[1]}{\Elim U_\Vdash(A)[0]}{U \subset U_\Vdash(A)}
	\Say{[2]}{\THM{OpenApproximationBimeager}(X,A) }{U_\Vdash(A) =^* A}
	\Say{[3]}{[0][2]}{U_\Vdash(A) =^* U}
	\Conclude{[*]}{\Elim \TYPE{Bimeager}[3]\Elim \TYPE{OpenDomain}(U)}
	{
		U_\Vdash(A) = U
	}
	\EndProof
	\\
	\DeclareFunc{meagerIdeal}{\prod_{X \in \TOP} \SIdeal(X)  }
	\DefineNamedFunc{meagerIdeal}{}{\MGR(X)}{\Meager(X)}
	\\
	\DeclareFunc{categoryAlgebra}{\TOP \to  \SA  }
	\DefineNamedFunc{categoryAlgebra}{}{\CAT(X)}{\frac{\BP(X)}{\MGR(X)}}
	\\
	\Theorem{OpenDomainAlgebraTHM}
	{
		\prod X : \Bair \.  \forall A \in \cat(X) \. 
		\exists! U : \TYPE{OpenDomain}(X) \.
		A = [U]
	}
	\NoProof
}
\Page{
	\Theorem{CatAlgebraIsCCC}
	{
		\forall X : \Bair \And \TYPE{SecondCountable} \.
		\CCC \Big(\cat(X)\Big)
	}
	\Assume{\U}{\PD\Big(\cat(X)\Big)}
	\Say{\Big( \U', [1]\Big)}{\THM{OpenDomainAlgebra}(\U)}
	{
		\sum \U' \in ?\T(X) \.  
		\forall u \in \U \. \exists! U \in \U' \. u = [U]
		\And \forall U \in \U' \. \exists! U \in \U' : u = [U]
	}
	\Say{[2]}{\Elim \PD\Big(\cat(X),\U\Big)[1]}
	{
		\PD\Big(?X, \U' \Big)
	}
	\Say{[3]}{\Elim \TYPE{SeconCountable})(X)[2]}
	{
		 |\U'| \le \aleph_0)	
	}
	\Conclude{[\U.*]}{[1][3]}{|\U| \le \aleph_0}
	\DeriveConclude{[*]}{\Intro \CCC}
	{
		\CCC\Big( \cat(X) \Big)	
	}
	\EndProof
	\\
	\Theorem{CatAlgebraIsComplete}
	{
		\forall X : \Bair \.
		\TAlgebra\Big( \cat(X) \Big)
	}
	\AssumeIn{\I}{\Set}
	\Assume{u}{\I \to \cat(X)}
	\Say{\Big(U,[1]\Big)}{\THM{OpenDomainTHM}}
	{
		\sum \I \to \TYPE{OpenDomain}(X) \.
		\forall i \in \I \.  u_i = [U_i]
	}
	\Say{V}{\intx \overline{\bigcup_{i \in \I} U_i}}{\TYPE{OpenDomain}(X)}
	\Say{[2]}{
			\Lambda i \in I \.			
			\THM{SubsetOfUnion}(I,U,i)
			\Intro \FUNC{closure}
			\Intro \FUNC{interior} 
			\Intro V	
	}
	{
		\forall i \in \I \. 
		U_i \subset \bigcup_{i \in \I} U_i \subset 
		\intx \overline{\bigcup_{i \in \I} U_i} = V
	}
	\Say{[3]}
	{
		\Intro \cat(X)[1]	
	}
	{
		\forall i \in I \. u_i \le [V]
	}
	\AssumeIn{w}{\CAT(X)}
	\Assume{[4]}{\forall i \in I \. u_i \le w}
	\Say{(W,[5])}{\THM{OpenDomainTHM}(X,w)}
	{
		\sum W : \TYPE{OpenDomain}(X) \. w = [W]
	}
	\Say{[6]}{[1][4][5]}
	{
		\forall i \in \I \. U_i \setminus W  \in \MGR(X)
	}
	\Say{[7]}{\bigcup [6]}
	{
		\left(\bigcup_{i \in I} U_i\right) \setminus W \in \MGR(X)
	}
	\Say{[8]}{\THM{NowhereDenseReisdual}[7]\Intro V}
	{
		    V \setminus W \in \MGR(X)
	}
	\Conclude{[w.*]}{\Elim \CAT(X)\THM{LessByDifference}[8]}{[V] \le w}
	\DeriveConclude{[u.*]}{\Intro \vee [3]}
	{
		[V] = \bigvee_{i \in I} u_i
	}
	\DeriveConclude{[*]}{\THM{CompleteBySupremas}}
	{
		\TAlgebra\Big( \cat(X) \Big)
	}
	\EndProof
}
\newpage
\subsubsection{Banach-Mazur game} 
\Page{
	\DeclareFunc{gameOfBanachMazur}
	{
		\prod_{X \in \TOP} ?X \to \IIPG\Big( \T(X) \Big)
	}
	\DefineNamedFunc{gameOfBanachMazur}{A}
	{ \Game^{**}(A) }
	{
		\NewLine :\:=		
		\left( 
			\Big\{ U : [1,\ldots,n] \downarrow \T(X) \Big| 
				n \in \Int_+  , \forall i \in \{1,\ldots,n\} \. \exists U_i
			\Big\},
			\Lambda \U \in \T^\Nat(X) \. \bigcap^\infty_{n=1} U_n \subset A
		\right)
	}
	\\
	\Theorem{SecondPlayerBanachMazurTheorem}
	{
		\forall X \in \TOP \.
		\forall A \subset X \.
		\exists X 
		\Imply \NewLine \Imply
		\bigg(
		\Comeager(X,A) 
		\iff
		\exists \SPWS\Big( \Game^{**}(A) \Big)
		\bigg)
	}
	\Assume{[1]}{\Comeager(X,A)}
	\Say{\Big(U,[2]\Big)}{\Elim \Comeager(X,A)[1]}
	{
		\sum U : \Nat \to \Dense \And \TYPE{Open}(X) \.
		A = \bigcap^\infty_{n=1} U_n
	}
	\Assume{V}{\lp \; \Game^{**}(A)}
	\SayIn{n}{\len V}{\Int_+}
	\Assume{[3]}{\TYPE{Odd}(n)}
	\SayIn{k}{\frac{n+1}{2}}{\Nat}
	\Conclude{V_{n+1}}{V_n \cap U_k}{\TYPE{Open} \And \TYPE{NonEmpty}(X)}
	\Derive{V}{\LOGIC{Play}\Big(\Game^{**}(A)\Big)}{[\lp \Game^{**}(A) ]}
	\Say{[3]}{\Elim V [2]}
	{
		\bigcap^\infty_{n=1} V_n \subset \bigcap^\infty_{n=1} U_n = A
	}
	\Conclude{[1.*]}{\Intro \SPWS[3]}{\exists \SPWS\Big( \Game^{**}(A) \Big)}
	\EndProof 
}\Page{
	\Theorem{FirstPlayerBanachMazurTheorem}
	{
		\forall X : \CS \.
		\forall A \subset X \.
		\Big( \exists d : \TYPE{Metric}(X) \. 
		\T(X,d) \subset \T(X) \Big)
		\Imply \NewLine \Imply
		\bigg(
		\exists U \in \T(X) \.\exists U \And	 
		\Meager(U,U \cap A) 
		\iff
		\exists \FPWS\Big( \Game^{**}(A) \Big)
		\bigg)
	}
	\AssumeIn{U}{\T(X)}
	\Assume{[1]}{\exists U}
	\Assume{[2]}{\Meager(U,U\cap A)}
	\Say{\Big(W,[3]\Big)}{\Elim \Meager(U,U \cap A)}
	{
		\sum  W : \Nat \to \TYPE{Open} \And \TYPE{Dense}(U) \.	
		\bigcap_{n=1} W_n  = U \setminus A	
	}
	\Say{[4]}{\THM{OpenOpenSubset}(X,U,W)}{ \forall n \in \Nat \. W_n \in \T(X)}
	\Say{[5]}{\THM{ChoquetSpaceOpenSubset}(X,U)}
	{
		\CS(U)
	}
	\Say{\sigma}{\Elim \CS(U)}{\SPWS\Big( \Game_{\mathrm{Ch}}(U) \Big)}
	\Assume{V}{\lp\Big( \Game^{**}(A) \Big)}
	\SayIn{n}{\len V}{\Int_+}
	\Assume{[6]}{\TYPE{Even}(n)}
	\Assume{[7]}{n=0}
	\Conclude{V_1}{U}{\TYPE{Open} \And \TYPE{NonEmpty}(X)}
	\Derive{[7]}{\Intro \Imply}{ (n=0) \Imply \TYPE{Open} \And \TYPE{NonEmpty}(X) }
	\Assume{[8]}{n \neq 0}
	\Say{[9]}{\LOGIC{InGame}[7][8]}
	{
		\forall k \in [1, \ldots, n] \. V_k \subset U
	}
	\Say{V'}{
		\Lambda k \in [1,\ldots,n] \. 
		\If \TYPE{Odd}(k) \.
		\Then V_k \cap W_{(k+1)/2} 
		\Else V_k
	}{  [1,\ldots,n] \to \T(X)   }
	\Say{[10]}
	{[9] \Elim \lp\Big( \Game^{**}(A) \Big) \Intro \lp\Big( \Game_{\mathrm{Ch}}(U) \Big)}
	{
		V' \in \lp\Big( \Game_{\mathrm{Ch}}(U) \Big)
	}
	\Say{[11]}{\LOGIC{InGame}(n-1)[9]}
	{
		V' \in  \sigma
	}
	\Say{\Big( O, [12]\Big)}
	{\Elim \SPS( \Game_{\mathrm{Ch}}(U), \sigma, V'  \Big)}
	{
		\sum O \in \T(U)  \. \exists O \And V'O \in \sigma  
	}
	\Say{[13]}{\THM{OpenOpenSubset}(X,U,O)}{  O \in \T(X)}
	\Conclude{V_{n+1}}{O}{\TYPE{Open} \And \TYPE{NonEmpty}(X)}
	\Derive{V}{\LOGIC{Play}\Big(\Game^{**}(A)\Big)}{\Big[\lp \Game^{**}(A) \Big]}
	\Say{[6]}{\Elim V \Elim \SPWS(\sigma)}
	{  U \cap \bigcap^\infty_{n=1} V_n \neq \emptyset}
	\Conclude{[1.*]}{\Elim V [3]}{A^\c \cap \bigcap^\infty_{n=1} V_n}
	\Derive{[1]}{\Intro \Imply}
	{
		\exists U \in \T(X) \.\exists U \And	 
		\Meager(U,U \cap A) 
		\Imply
		\exists \FPWS\Big( \Game^{**}(A) \Big)
	}
}\Page{
	\Assume{\sigma}
	{
		 \FPWS\Big( \Game^{**}(A) \Big)
	}
	\Say{\Big(U,[2]\Big)}{\Elim \FPWS\Big( \Game^{**}(A),\sigma,1 \Big) }
	{
		\sum U \in \T(X) : \exists U \And \{  U \} =  \{ t_1 | t \in \sigma \} 
	}
	\Assume{V}{\lp\Big( \Game^{**}(A) \Big)}
	\SayIn{n}{\len V}{\Int_+}
	\Assume{[3]}{\TYPE{Even}(n)}
	\Assume{[4]}{n=0}
	\Conclude{V_1}{U}{\TYPE{Open} \And \TYPE{NonEmpty}(X)}
	\Derive{[4]}{\Intro \Imply}{ (n=0) \Imply \TYPE{Open} \And \TYPE{NonEmpty}(X) }
	\Assume{[5]}{n \neq 0}
	\SayIn{x}{\Elim \exists V_n}{V_n}
	\Say{W}{V_k \cap \mathbb{B}_d(x,2^{-n-1})}
	{
		\TYPE{Open} \And \TYPE{NonEmpty}(X)
	}
	\Say{V'}{V_{|[1,\ldots,1-n]}W}{[1,\ldots,n] \to \TYPE{Open} \And \TYPE{NonEmpty}(X) }
	\Say{[6]}{\Elim V' \LOGIC{InPlay}}{V' \in \sigma}
	\Say{\Big(O,[7]\Big)}{\Elim \FPS\Big( \Game^{**}(A),\sigma, V' \Big)}
	{
		\sum O : \TYPE{Open} \And \TYPE{NonEmpty}(X) \. V'O \in \sigma
	}
	\Say{[8]}{\Elim \Game^{**}(A) [7] }{\diam O \le 2^{-n}}
	\Conclude{V_{n+1}}{O}{  \TYPE{Open} \And \TYPE{NonEmpty}(X)} 
	\Derive{V}{\LOGIC{Play}\;\Game^{**}(A) }{\Big[\lp \Game^{**}(A) \Big]}
	\Say{[3]}{\Elim V \Elim  \FPWS\Big( \Game^{**}(A), \sigma \Big)}
	{
		A^\c \cap \bigcap^\infty_{n=1} V_n \neq \emptyset
	}
		\Say{[4]}{ \Elim V \Elim V.7}
	{
		\lim_{n \to \infty} \diam V_n = 0
	}
		\Conclude{\Big(x,[V.*]\Big)}{[3][4]}
	{
		\sum x \in A^\c \. \bigcap^\infty_{n=1} V_n = \{x\}
	}
	\Derive{\Big(\sigma',[3]\Big)}{\LOGIC{GameOver}}
	{
		\sum \sigma' :  \FPWS\Big( \Game^{**}(A) \Big) \.
		\NewLine \.
		\forall V \in [\sigma'] \. 
		\exists x \in A^\c \. 
		\bigcap^\infty_{n=1} V_n = \{x\}
	}
	\Say{[4]}{
		\Elim \FPWS\Big( \Game^{**}(A), \sigma' \Big)
		\Elim \sigma'  
	}
	{
		\Comeager(U, A^\c \cap U)
	}	
	\Conclude{[2.*]}{[4]^\c}{\Meager(U,A \cap U)}
	\DeriveConclude{[*]}{\Intro \iff}
	{
		\exists U \in \T(X) \.\exists U \And	 
		\Meager(U,U \cap A) 
		\iff
		\exists \FPWS\Big( \Game^{**}(A) \Big)
	}
	\EndProof
	\\
	\DeclareType{Determined}{\prod_{X \in \SET} ?\IIPG(X)}
	\DefineType{g}{Determined}{\exists \FPWS(g)\Big| \exists \SPWS(g)}
}
\Page{
	\Theorem{BairPropertyByDetermination}
	{
		\forall X : \CS \.
		\forall A \subset X \.
		\Big( \exists d : \TYPE{Metric}(X) \. 
		\T(X,d) \subset \T(X) \Big)
		\Imply \NewLine \Imply
		\bigg(
			A \in \BP(X) \iff
			\forall U \in \T(X) \.
			\exists U \Imply
			\TYPE{Determind}\Big(\Game^{**}(A \cup U)\Big)
		\bigg)
	}
	\Assume{[1]}{A \in \BP(X)}
	\AssumeIn{U}{\T(X)}
	\Assume{[3]}{\exists U}
	\Say{[4]}{\LOGIC{LEM}\Big(\TYPE{Comeager}(U,A \cap U) \Big)}
	{
		\TYPE{Comeager}(U,A \cap U)
		\Big|
		\neg \TYPE{Comeager}(U,A \cap U)
	}	
	\Assume{[5]}{\neg \TYPE{Comeager}(U,A \cap U)}
	\Say{\Big(W, [6]\Big)}{\Elim \TYPE{Comeager}[5]}
	{ 
		\sum W \in \T(X) \. W \subset U \And \exists W \And \Meager(W,A \cap W) 
	}
	\Say{[8]}{\THM{FirstPlayerBanachMazurTheorem}[6]}
	{
		\exists \FPWS\Big( \Game^{**}(A \cap U) \Big)
	}
	\Conclude{[5.*]}{\Intro \TYPE{Determined}[8]}
	{
		\TYPE{Determined}
		\Big(
				\Game^{**}(A \cap U)
		\Big)
	}
	\Derive{[5]}{\Intro \Imply}
	{
		V \du U \in \MGR(X)
		\Imply
		\TYPE{Determined}
		\Big(
				\Game^{**}(A \cap U)
		\Big)
	}
	\Say{[6]}{\THM{SecondPlayerBanachMazurTheorem} \Intro \TYPE{Determined}}
	{
			\TYPE{Comeager}(U,A \cap U) \Imply
			\TYPE{Determined}
			\Big(
				\Game^{**}(A \cap U)
			\Big)
	}
	\Conclude{[1.*]}{\Elim | [4][5][6]}
	{
		\TYPE{Determined}
		\Big(
				\Game^{**}(A \cap U)
		\Big)
	}
	\Derive{[1]}{\Intro \Imply}
	{
			A \in \BP(X) \Imply
			\forall U \in \T(X) \.
			\exists U \Imply
			\TYPE{Determined}\Big(\Game^{**}(A \cap U)\Big)
	}
	\Assume{[2]}
	{
		\forall U \in \T(X) \.
		\exists U \Imply
		\TYPE{Determined}\Big(\Game^{**}(A \cap U)\Big)
	}
	\Say{\U}
	{
		\bigg\{
			U \in \T(X) : \exists U \And \exists \SPWS\Big( \Game^{**}(A \cap U)\Big)
		\bigg\}
	}
	{
		?\T(X)
	}
	\SayIn{V}{\bigcup \U}{\T(X)}
	\Assume{[3]}{\neg \Comeager(V,V\cap A)}
	\Say{\Big(W,[4]\Big)}{[2][3]}
	{
		\sum W \in \T(V) \. \exists W \And W \cap A \in \MGR(V)
	}
	\Say{\Big( U,[5]\Big)}{\Elim V(W)}{\sum U \in \U \. U \cap W \neq \emptyset}
	\Say{[6]}{[5][4]\Elim \U \Elim \Game^{**}}
	{
		U \not \in \U	
	}
	\Conclude{[3.*]}{\Elim U [5]}{\bot}
	\Derive{[3]}{\Elim \bot}{\Comeager(V,V\cap A)}
	\Say{\U'}
	{
		\Big\{ U \in \T(X) : \exists \And \Meager(U,U\cap A)    \Big\}
	}
	{
		?\T(X)
	}
	\SayIn{V'}{\bigcup \U'}{\T(X)}
	\Say{[4]}{\Intro \TYPE{Meager}(X,A)\Elim V'}
	{
		\Meager(V',A \cap V')
	}
	\Say{[5]}{\Elim V' \Elim V [2]}
	{
		X = V' \cup \boundary V \cup V
	}
	\Conclude{[2.*]}{\Intro \BP \THM{MeagerResidual}(X)[3][4][5]}
	{
		A \in \BP(X)
	}
	\DeriveConclude{[*]}{}
	{
		A \in \BP(X) \iff
		\forall U \in \T(X) \.
		\exists U \Imply
		\TYPE{Determind}\Big(\Game^{**}(A \cup U)\Big)
	}
	\EndProof
}
\Page{
	\DeclareType{EquivalentGames}
	{
		\prod_{X,Y,Z \in \SET}
		 ?\Big( 
		 	\big(X \to \IIPG(Y)\big)
		 	\NewLine
		 	\big( X \to \IIPG(Z)\big)
		 \Big)
	}
	\DefineNamedType{(g,g')}{EquivalentGames}
	{
		g \cong g'
	}
	{
		\forall x \in X \. \NewLine \.
		\bigg((\exists \FPWS\Big(g(x)\Big) \iff  \exists \FPWS\Big(g'(x)\Big)
		\And \NewLine \And
		\exists \SPWS\Big(g(x)\Big) \iff  \exists \SPWS\Big(g'(x)\Big)\bigg)
	}
	\\
	\DeclareFunc{weakBasisBanachMazurGame}
	{
		\prod_{X \in \TOP} \prod 
		\mathcal{V} : \TYPE{WeakBasis}(X) \. 
		\NewLine \.		
		?X \to \IIPG(\mathcal{V})
	}
	\DefineNamedFunc{weakBasisBanachMazurGame}{A}
	{ \Game^{**}(A)_{\mathcal{V}} }
	{
		\NewLine :\:=		
		\left( 
			\Big\{ U : [1,\ldots,n] \downarrow \mathcal{V} \Big| 
				n \in \Int_+  , \forall i \in \{1,\ldots,n\} \. \exists U_i
			\Big\},
			\Lambda \U \in \mathcal{V}^\Nat(X) \. \bigcap^\infty_{n=1} U_n \subset A
		\right)
	}
	\\
	\Theorem{WeakBanachMazurGameEquivalence}
	{
		\forall X \in \TOP \.
		\forall \mathcal{V} : \TYPE{WeakBasis} \. 
		\Game^{**} \cong \Game^{**}_\mathcal{V}
	}
	\NoProof
	\\
}
\newpage
\subsubsection{Bair measurable functions}
\Page{
	\DeclareType{BairMeasurable}
	{
		\prod_{ X,Y \in \TOP}
		?\SET(X,Y)	
	}
	\DefineType{f}{\BM}
	{
		\forall U \in \T(Y) \. f^{-1}(U) \in \BP(X)
	}
	\\
	\Theorem{BairMeasurabilityContinuousPart}
	{
		\forall X \in \TOP \.
		\forall Y : \TYPE{SecondCoubtable} \.
		\forall f : \BM(X,Y) \. \NewLine \.
		\exists G \subset X \.
		\exists U : \Nat \to \Open \And \Dense(X) \.
		G = \bigcap^\infty_{n=1} U_n
		\And
		f_{|G} \in \TOP(G,Y)
	}
	\Say{\Big(\U, [1]\Big)}{\Elim \TYPE{SecondCountable}(Y)}
	{
		\sum \U : \TYPE{Base}(Y) \. |\U| \le \aleph_0
	}
	\Say{[2]}{\Elim \BM(X,Y)}
	{
		f^{-1}(\U) \subset \BP(X)
	}
	\AssumeIn{U}{\U}
	\Say{\Big(V,E,[3]\Big)}{\Elim \BP(X)}
	{
		\sum V \in \T(X) \.
		\sum E \in \MGR(X) \.
		f^{-1}(U)  = V \du E
	}
	\Say{\Big(N, [4]\Big)}{\Elim \MGR(X,V)}
	{
		\sum N : \Nat \to \TYPE{Closed} \. 
		E \subset \bigcap^\infty_{n=1} N_n
	}
	\SayIn{F}{\bigcap^\infty_{n=1} N_n}{\MGR(X)}
	\Say{[5]}{\Elim F [3][4] }{f^{-1}(U_n) \du V_n \subset F_n}
	\Conclude{G_U}{F^\c}{\TYPE{Comeager}(X)}
	\Derive{\Big( V, G, [2] \Big)}{\Intro \prod}
	{
		\prod_{U \in \U} \Big(V_U,G_U\Big) : \T(X) \times \TYPE{Comeager}(X) \.	
		f^{-1}(U) \du V_U \subset G_U^\c 
	}
	\Say{H}{\bigcap_{U \in \U} G_U}{\TYPE{Comeager}(X)}
	\Say{[3]}{[2]^\c \Elim \du}
	{
		\forall U \in \U \.
		f^{-1}(U) \cap H = V_U \cap H
	}
	\Say{[4]}{\Intro f^{-1}_{|H}[3]}
	{
		\forall U \in \U \.
		f^{-1}_{|H}(U) = V_U
	}
	\Conclude{[*]}{\Intro \TOP [5]}
	{
		f_{|H} \in \TOP(H,Y)
	}
	\EndProof
	\\
	\Theorem{BairMeasurableCantorImage}
	{
		\forall X : \Perfect \And \Polish \.
		\forall Y : \TYPE{SeconCountable} \. \NewLine \.
		\forall f : \BM \And \TYPE{Injective}(X,Y) \.
		\exists C \subset f(X) \.
		C \cong \C
	}
	\NoProof
}
\newpage
\subsubsection{Kuratowski-Ulam theorem}
\Page{
	\DeclareType{BairQuantificationForall}
	{
		\prod_{X \in \TOP} ??X	
	}
	\DefineNamedType{A}{BairQuantificationForall}
	{
		\forall^* x \. A(x)
	}
	{
		\TYPE{Comeager}(X,A)
	}
	\\
	\DeclareType{BairQuantificationExists}
	{
		\prod_{X \in \TOP} ??X	
	}
	\DefineNamedType{A}{BairQuantificationExists}
	{
		\exists^* x \. A(x)
	}
	{
		\neg\TYPE{Meager}(X,A)
	}
	\\
	\DeclareType{LocalBairQuantificationForall}
	{
		\prod_{X \in \TOP} \T(X) \to ??X	
	}
	\DefineNamedType{A}{BairQuantificationForall}
	{
		\Lambda U \in \T(U) \. \forall^* x \in U \. A(x)
	}
	{
		\Lambda U \in \T(U) \. \TYPE{Comeager}(U,A \cap U)
	}
	\\
	\DeclareType{LocalBairQuantificationExists}
	{
		\prod_{X \in \TOP} \T(X) \to ??X
	}
	\DefineNamedType{A}{LocalBairQuantificationExists}
	{
		\Lambda U \in \T(U) \. \exists^* x \in U \. A(x)
	}
	{
		\NewLine \iff		
		\Lambda U \in \T(U) \. \neg\TYPE{Meager}(U,A \cap U)
	}
	\\
	\Theorem{NowhereDenseSectionLemma}
	{
		\forall X \in  \TOP \.
		\forall Y : \TYPE{SecondCountable}(X) \. \NewLine \. 
		\forall F : \ND(X \times Y) \.
		\forall^* x \in X \. \ND(Y,F_x)
	}
	\Say{F'}{\overline{F}}{\TYPE{Closed}(X\times Y)}
	\Say{[1]}{
		\Elim \ND(X \times T, F) 
		\THM{ClosureIsRetraction}(X \times Y) 
		\Intro  \ND  \Intro F'
	}
	{
		\NewLine :		
		\ND(X\times T, F')
	}
	\Say{U}{(X \times Y) \setminus F'}
	{
		\Open \And \Dense(X \times Y)
	}
	\Say{\Big(\V,[2]\Big)}{\Elim \TYPE{SecondCountable}(Y)}
	{
		\sum \V : \TYPE{Base}(Y) \. |\V| \le \aleph_0 
	}
	\Say{\U}
	{
		\Big\{
				\pi_X \big(U \cap (X \times V)\big) 
		\Big|
			V \in \V, \exists V
		\Big\}
	}
	{
		?\T(X)
	}
	\AssumeIn{O}{\U}
	\Say{\Big(V,[4]\Big)}
	{
		\Elim \U(O)
	}
	{
		\sum V \in \V \. O = \pi_X\big(U \cap (X \times V)\big)
	}
	\AssumeIn{W}{\T(X)}
	\Assume{[3]}{\exists W}
	\Say{[5]}{\Elim \Dense( X \times Y,U, W \times V)}
	{
		\exists \big(U \cap (W \times V) \big)
	}
	\Say{[6]}{\THM{SubsetProduct}(X,Y,W,V) \THM{SubsetIntersection}\Big(U,X\Big)[4]}
	{
		U \cap (W \times V) \subset U \cap (X \times V) = O)
	}
	\Conclude{[O.*]}{\THM{ProjectionIntersection}(X,Y)[6][5]}
	{
		\exists(O \cap W)
	}
	\Derive{[3]}{\Intro \Dense \Intro \forall}
	{
		\forall O \in \U \. \Dense(X,O)
	}
	\Say{A}{\bigcap \U}{\TYPE{Comeager}(X)}
	\Say{[4]}{\Elim A \Elim \U}
	{
		\forall a \in A  \. 
		\forall V \in \V \.	
		U_a \cap V \neq \emptyset
	}
	\Say{[5]}{\THM{DenseByBase}[4]}
	{
		\forall a \in A \.
		\Dense(Y, U_a)
	}
	\Say{[6]}{\Intro \forall^*}
	{
		\forall^* x \in X \. \Dense(Y,U_x)
	}
	\Say{[7]}{\Elim U [6]}
	{
		\forall^* x \in X \.  \ND(Y,F'_x)
	}
	\Conclude{[8]}{\Elim F' \THM{NowhereDenseSubset}(Y)[7]}
	{
		\forall^* x \in X \.  \ND(Y,F_x)
	}
	\EndProof
}
\Page{
	\Theorem{MeagerSectionLemma}
	{
		\forall X \in  \TOP \.
		\forall Y : \TYPE{SecondCountable}(X) \. \NewLine \. 
		\forall F : \Meager(X \times Y) \.
		\forall^* x \in X \. \Meager(Y,F_x)
	}
	\NoProof
	\\
	\Theorem{KuratowskiUlamTHM1}
	{
		\forall X,Y : \TYPE{SecondCountable}(X) \.
		\forall A \in \BP(X \times Y) 
		\. \NewLine \.
		\forall^* x \in X \.  A_x \in \BP(Y)
		\And
		\forall^* y \in Y \. A^y \in \BP(X)
	}
	\NoProof
	\\
	\Theorem{KuratowskiUlamTHM2 }
	{
		\forall X,Y : \TYPE{SecondCountable}(X) \.
		\forall A \in \MGR(X \times Y) \. 
		\NewLine \.
		\forall^* x \in X \.  A_x \in \MGR(Y)
		\And
		\forall^* y \in Y \. A^y \in \MGR(X)
	}
	\NoProof
	\\
	\Theorem{MeagerProductLemma}
	{
		\forall X,Y : \TYPE{SecondCountable}(X) \.
		\forall A \subset X \.
		\forall B \subset Y \.
		\NewLine
		A \in \MGR(X) \Big| B \in \MGR(Y)
		\Imply
		A \times B \in \MGR(X \times Y)
	}
	\Assume{[1]}{A \in \MGR(X)}
	\Say{\Big(N,[2]\Big)}{\Elim \MGR(X,A)}
	{
		\sum N : \Nat \to \ND(X) \. 
		A = \bigcup^\infty_{n=1} N_n
	}
	\Say{[3]}{\THM{ProductUnion}(X,Y,A,B)[2]}
	{
		A \times B = \bigcup_{n=1}^\infty N_n \times B
	}
	\Say{[4]}{\Lambda n \in \Nat \.\THM{NowhereDenseProductProduct}(X,Y,N_n,B)}
	{
		\forall n \in \Nat \. \ND(X \times Y,N_n \times B)
	}
	\Conclude{[*]}{\Intro \MGR(X \times Y)[3][4]}
	{
		A \times B \in \MGR(X \times Y)
	}
	\Derive{[1]}{\Intro \Imply}{A \in \MGR(X) \Imply A \times B \in \MGR(X\times Y)}
	\Assume{[2]}{B \in \MGR(Y)}
	\Say{\Big(N,[3]\Big)}{\Elim \MGR(Y,B)}
	{
		\sum N : \Nat \to \ND(Y) \. 
		B = \bigcup^\infty_{n=1} N_n
	}
	\Say{[4]}{\THM{ProductUnion}(X,Y,A,B)[4]}
	{
		A \times B = A \times \bigcup_{n=1}^\infty N_n
	}
	\Say{[5]}{\Lambda n \in \Nat \. \THM{NowhereDenseProductProduct}(X,Y,A,N_n)}
	{
		\forall n \in \Nat \. \ND(X \times Y,A \times N_n)
	}
	\Conclude{[*]}{\Intro \MGR(X \times Y)[4][5]}
	{
		A \times B \in \MGR(X \times Y)
	}
	\Derive{[2]}{\Intro \Imply}{B \in \MGR(Y) \Imply A \times B \in \MGR(X\times Y)}
	\Conclude{[*]}{\Elim(|)[0][1][2]}
	{
		A \times B \in \MGR(X\times Y)
	}	
	\EndProof
	\\
	\Theorem{MeagerProductLemma2}
	{
		\forall X,Y : \TYPE{SecondCountable}(X) \.
		\forall A \subset X \.
		\forall B \subset Y \.
		\NewLine
		A \times B \in \MGR(X \times Y)
		\Imply
		A \in \MGR(X) \Big| B \in \MGR(Y)
	}
	\NoProof
}
\Page{
	\Theorem{InverseKuratowskiUlamTHM2}
	{
		\forall X,Y : \TYPE{SecondCountable}(X) \.
		\forall A \in \BP(X \times Y) \. 
		\NewLine \. \Big(
		\forall^* x \in X \.  A_x \in \MGR(Y)
		\Big|
		\forall^* y \in Y \. A^y \in \MGR(X) \Big)
		\Imply
		A \in \MGR(X \times Y)
	}
	\Say{\Big(U,E [-1]\Big)}{\Elim \BP(X \times Y,A)}
	{
		\sum U \in \T(X \times Y) \.
		\sum E \im \MGR(X \times Y) \.
		A = U \du E 
	}
	\Assume{[1]}{\forall^* x \in X \.  A_x \in \MGR(Y)}
	\Assume{[2]}{\exists^* A}
	\Say{[3]}{[2][-1]}{\exists^* U}
	\Say{\Big(V,W,[4]\Big)}{\THM{SCProducTopologyProperty}(X,Y,U)[3]}
	{
		\sum_{V \in \T(X) }
		\sum_{W \in \T(Y)} 
		V \times W \subset U
		\And
		\exists^* V \times W
	}
	\Say{[5]}{\THM{MeagerProductLemma}[4]}
	{
			\exists^* V  \exists^* W
	}
	\Say{\Big(x,[6]\Big)}
	{
		\Elim \forall^* [1](V)
	}
	{
		\sum x \in V \. A_x \in \MGR(Y)
	}
	\Say{[7]}{[6]\Elim \exists^*[5]}
	{
		E_x \in \MGR(Y)
	}
	\Say{[8]}{
		\THM{DifferenceSubset}(Y,W,E_x,U_x)[4.1]
		\THM{SymmetricIsMore}(Y)
		\THM{SectionSubset}(X,Y)[-1]	
	}
	{
		\NewLine :
		W \setminus E_x \subset U_x \setminus E_x \subset U_x \du E_x \subset A_x 
	}
	\Say{[9]}{\THM{MeagerSubset}(Y)[8][6]\THM{MeagerDifference}(Y)[7]}
	{
		W \in \MGR(Y)
	}
	\Conclude{[1.*]}{\Elim \exists^* [4.2][9]}{\bot}
	\Derive{[1]}{\Intro \Imply}
	{
		\Big(\forall^* x \in X \. A_x \in \MGR(Y) \Big)
		\Imply
		A \in \MGR(X \times Y)
	}
	\Assume{[2]}{\forall^* y \in Y \.  A^y \in \MGR(y)}
	\Assume{[3]}{\exists^* A}
	\Say{[4]}{[3][-1]}{\exists^* U}
	\Say{\Big(V,W,[5]\Big)}{\THM{SCProducTopologyProperty}(X,Y,U)[3]}
	{
		\sum_{V \in \T(X) }
		\sum_{W \in \T(Y)} 
		V \times W \subset U
		\And
		\exists^* V \times W
	}
	\Say{[6]}{\THM{MeagerProductLemma}[5]}
	{
			\exists^* V  \exists^* W
	}
	\Say{\Big(y,[7]\Big)}
	{
		\Elim \forall^* [2](W)
	}
	{
		\sum y \in W \. A^y \in \MGR(X)
	}
	\Say{[8]}{[7]\Elim \exists^*[6]}
	{
		E^y \in \MGR(X)
	}
	\Say{[9]}{
		\THM{DifferenceSubset}(X,V,E^y,U^y)[5.1]
		\THM{SymmetricIsMore}(X)
		\THM{SectionSubset}(X,Y)[-1]	
	}
	{
		\NewLine :		
		V \setminus E^y \subset U^y \setminus E^y \subset U^y \du E^y \subset A^y 
	}
	\Say{[10]}{\THM{MeagerSubset}(Y)[9][7]\THM{MeagerDifference}(Y)[6]}
	{
		V \in \MGR(Y)
	}
	\Conclude{[2.*]}{\Elim \exists^* [5.2][10]}{\bot}
	\Derive{[2]}{\Intro \Imply}
	{
		\Big(\forall^* x \in X \. A_x \in \MGR(Y) \Big)
		\Imply
		A \in \MGR(X \times Y)
	}
	\Conclude{[*]}{\Elim(|)[0][1][2]}{A \in \MGR(X \times Y)}
	\EndProof
	\\
	\Theorem{UlamKuratowskiTHM}
	{
		\forall X,Y: \TYPE{SecondCountable} \.
		\forall A \in \BP(X \times Y) \. \NewLine \.
		\forall^* x \in X \. \forall^* y \in Y \. A(x,y)
		\iff
		\forall^* y \in Y \. \forall^* x \in X \. A(x,y) 
		\iff
		\forall (x,y) \in X \times Y \. A(x,y)
	}
	\NoProof
}
\Page{
	\Theorem{BairProduct}
	{
		\forall X,Y : \Bair \And \TYPE{SecondCountable} \.
		\Bair(X \times Y)
	}
	\Assume{U}{\Nat \to \Open \And \Dense(X\times Y)}
	\Say{[1]}{\Lambda n \in \Nat \. \THM{UlamKuratowskiTHM}(U_n)}
	{
		\forall n \in \Nat \.
		\forall^*  x \in X \.
		\forall^* y \in Y \.
		U_{n}(x,y)
	}
	\Say{\Big(F,D,[2]\Big)}{\Elim \forall^*[1] \Elim \Bair(Y)}
	{
		\sum F : \Comeager(X) \.
		\sum D : X \to \TYPE{Dense}(Y) \.
		\forall x \in F \. D_x \subset \left(\bigcap U_n \right)_x
	}
	\Say{[3]}{\Elim \Bair(X,F)}{\TYPE{Dense}(X,F)}
	\AssumeIn{V}{\T(X \times Y)}
	\Assume{[5]}{\exists V}
	\Say{[6]}{\pi_x [5]}{\exists \pi_x(V)}
	\Say{\Big(f,[7]\Big)}{\Elim \TYPE{Dense}(X,F)[6]}
	{
		\sum f \in F \. f \in \pi_x(V)
	}
	\Say{[8]}{\Intro V_f[7]}{\exists V_f}
	\Conclude{[*]}{\Elim \TYPE{Dense}(Y,D_f)[7]}{\exists \left( \{f\} \times D_f \right)V}
	\Derive{[4]}{\Intro \TYPE{Dense}}
	{
		\Dense\left( X \times Y, \bigcup_{f \in F} \{f\} \times D_f \right)
	}
	\Say{[5]}{\bigcup_{x \in X} [2](x)}{
					\bigcup_{f \in F} \{f\} \times D_f \subset \bigcap^\infty_{n=1} U_n
	}
	\Conclude{[U.*]}{
		 	\THM{DenseSubset}[4][3]
	}
	{
		\Dense\left( X \times Y, \bigcap_{n =1} U_n \right)
	}
	\Derive{[*]}{\Intro \Bair}{\Bair(X \times Y)}
	\EndProof
	\\
	\Theorem{DensePreimageTheorem}
	{
		\forall X,Y \in \TOP \.
		\forall f \in \Open(X,Y) \.
		\forall D : \Dense(Y) \.
		\Dense\Big( X, f^{-1}(D) \Big) 
	}
	\AssumeIn{U}{\T(X)}
	\Assume{[1]}{\exists U}
	\Say{[2]}{f[1]}{\exists f(U)}
	\Say{[3]}{\Elim \TYPE{Open}}
	{
		f(U) \in \T(Y)	
	}
	\Say{[4]}{\Elim \TYPE{Dense}(Y,D)}{\exists D \cap f(U)}
	\Conclude{[U.*]}{\THM{ImagePreimage}[4]}{\exists f^{-1}(D) \cap U}
	\DeriveConclude{[*]}{\Intro \TYPE{Dense}}
	{
		\TYPE{Dense}(X,f^{-1}(D))	
	}
	\EndProof
	\\
	\Theorem{MeagerPreimageTheorem}
	{
		\forall X,Y \in \TOP \.
		\forall f \in \Open \And \TOP(X,Y) \.
		\forall M : \MGR(Y) \.
		f^{-1}(M) \in \MGR(X) 
	}
	\NoProof
	\AssumeIn{x}{D}	
}
\newpage
\subsubsection{Fun facts}
\Page{
	\DeclareType{TopTransitiveGroup}{\prod_{X \in \TOP}?_\GRP \Aut_\TOP(X)}
	\DefineNamedType{G}{TopTransitiveGroup}{G : \TOP\hyph\TYPE{Transitive}(X)}
	{
		\NewLine \iff		
		\forall U,V \in \T(X) \. \exists U \And \exists V \Imply 
		\exists g \in G \.  \exists g(U) \cap V
	}
	\\
	\Theorem{FirstTopological01Law}
	{
		\forall X : \Bair \.
		\forall G : \TOP\hyph\TYPE{Transitive}(X) \.
		\forall A : \TYPE{Invariant}(X,F) \.
		\NewLine \.
		A \in \BP(X)
		\Imply
		\forall^* A \Big| \neg \exists^* A
	}
	\Assume{[1]}{(\neg \forall^* A) \And \exists^* A}
	\Say{\Big(U,V,[2]\Big)}{\Elim \forall^* \Elim \exists^*}
	{
		U \Vdash A \And V \Vdash \neg A \And \exists U \And \exists V
	}
	\Say{\Big(g,[3]\Big)}{\Elim \TOP\hyph\TYPE{Transitive}(X,G,U,V)}
	{
		\sum g \in G \. \exists g(U) \cap V
	}
	\Say{[4]}{g[2.1]}
	{
			g(U) \vdash g(A)
	}
	\Say{[5]}{\Elim \TYPE{Invariant}(X,G,A)[4]}
	{
		g(U) \vdash A 	
	}
	\Say{[6]}{[5][3]}{g(U) \cap V \Vdash A}
	\Say{[7]}{[2.2][3]}{g(U) \cap V \Vdash A^\c}
	\Say{[8]}{
		\Elim \BOOL\Big( \cat(X) \Big)
		\Elim \Vdash [6][7]
		\Elim \BOOL\Big( \BP(X), \cat(X),   \pi_\cat \Big)
		\Elim \c	
	}{  
			\NewLine 			
			[g(U) \cap V]_\cat =
			[g(U) \cap V]_\cat^2 =  
			[A \cap g(U) \cap V]_\cat[A^\c \cap  g(U) \cap V]_\cat =  \NewLine =
			[A]_\cat [A]^\c_\cat [g(U) \cap V]_\cat = 0   
	}
	\Say{[9]}{\Elim \cat [8]}{g(U) \cap V \in \MGR(X)}
	\Conclude{[1.*]}{\Elim \Bair(X)[9][3]}{\bot}
	\DeriveConclude{[*]}{\Elim \bot \LOGIC{DeMorganaLaw}}
	{
		\forall^* A \Big| \neg \exists^* A
	}
	\EndProof
	\\
	\DeclareType{TailSet}
	{
		\prod_{I \in \SET} \prod_{X : I \to \SET} ??\prod_{i \in I} X_i
	}
	\DefineType{T}{TailSet}{
		\forall x \in T \. 
		\forall y \in \prod_{i=1} X_i\.
		\TYPE{Finite}\Big(\{ i \in I : x_i \neq y_i   \}\Big) \Imply
		y \in T
	}
}\Page{
	\Theorem{SecondTopolofical01Law}
	{
		\forall X : \Nat \to \Bair \And \TYPE{SecondCountable} \.
		\forall A : \TYPE{TailSet}(\Nat,X) \. 
		\NewLine \.
		A \in \BP\left( \prod^\infty_{n=1} X_n\right) 
		\Imply
		\forall^* A \Big| \neg \exists^* A
	}
	\Assume{[1]}{\neg \exists^* A}
	\Say{\Big( U, [2]  \Big)}
	{
		\Elim \exists^*
		\THM{ProductTopologyRepresentation}(\Nat,X)
	}
	{
		\NewLine :		
		\sum U : \prod^\infty_{n=1} \T(X_n) \.
		\exists \prod^\infty_{n=1} U_n
		\And  \prod^\infty_{n=1} U_n \Vdash A 
		\And \Finite\Big(\Nat,\{ n \in \Nat : U_n \neq X_n   \}\Big)
	}
	\Say{N}{\{ n \in \Nat : U_n \neq X_n   \}}{\Finite(\Nat)}
	\SayIn{Y}{\prod_{n \in N} X_i}{\TOP}
	\SayIn{Z}{\prod_{n \in N^\c}}{\TOP}
	\Say{[5]}{\Elim Y \THM{BaireProduct} \Intro Y}
	{
			\TYPE{SecondCountable} \And \Bair(Y) 
	}
	\Say{ [3]}{\Elim \Vdash \Intro \forall^* \Intro Y [2.2][5]}
	{
		\forall^* y \in \prod_{n \in \Nat} U_n \. \forall^* z \in Z \. A(y,z)
	}
	\Say{[4]}{\Elim \TYPE{TailSet}(\Nat,X,A)[3]}
	{
		\forall^* y \in Y \. \forall^* x \in X \. A(x,y)
	}
	\Conclude{[1.*]}{\THM{KuratowskiUlamTHM}[4]}
	{
		\forall^* A
	}
	\DeriveConclude{[*]}{\Intro \Imply \Intro |}
	{
		\forall^* A \Big| \neg \exists^* A
	}
	\EndProof
	\\
	\Theorem{WellOederingIsNotBP}
	{
		\forall X : \Perfect \And \Polish \.
		\forall (<) : \TYPE{WellOrdering}(X) \.
		(<) \not \in \BP(X^2)
	}
	\Assume{[1]}{(<) \in \BP(X^2)}
	\Assume{[2]}{(<) \in \MGR(X^2)}
	\Say{[3]}{\THM{KuratovskiUlamTHM2}[2]}
	{
		\forall^* x \in X \.  (<)_x \in \MGR(X)
		\And
		\forall^* x \in X \. (<)^x \in \MGR(X)
	}
	\Say{\Big(x,[4]\Big)}{\Elim \Bair(X) [3] }
	{
		\sum x \in X \. (<)_x, (<)^x \in \MGR(X)
	}
	\Say{[5]}{\Elim \TYPE{TotalOrder}(X,\le)}
	{
		X = \{x\} \cup (<)_x \cup (<)^x
	}
	\Say{[6]}{\THM{PerfectHasMeagerPoints}(X)\THM{MeagerUnion}(X)}
	{
		X \in \MGR(X)
	}
	\Conclude{[2.*]}{\Elim \Bair(X)[6]}{\bot}
	\Derive{[2]}{\Intro \exists^*}{\exists^* (<)}
	\Say{[3]}{\THM{InverseKuratovsliUlamTHM2}[2]}
	{
		\exists x \in X \. \exists^* <^x
	}
	\SayIn{x}{\min \{ x \in X \. \exists^* <^x   \}}{X}
	\Say{Y}{<^x}{?X}
	\Say{<'}{(<) \cap Y^2}{?Y^2}
	\Say{[4]}{\Elim (<')\THM{KuratovskiUlamTHM1}}
	{
		(<') \in \BP(Y^2)
	}
	\Say{[5]}{\Elim <'\Elim x}
	{
		\forall y \in Y \. (<')^y \in \MGR(Y)
	}
	\Say{[6]}
	{
		\THM{KuratovskiUlamTHM2}[5]
	}
	{
		(<') \in \MGR(Y)
	}
	\Say{[7]}
	{
		\THM{KuratovskiUlamTHM1}[6]
	}
	{
		\forall^* y \in Y \. (<')_* \in \MGR(Y)
	}
	\Say{[8]}
	{
		\Elim \Bair(Y) [7]
		\THM{PerfectHasMeagerPoints}(Y)\THM{MeagerUnion}(Y)
	}
	{
		Y \in \MGR(Y)
	}
	\Conclude{[1.*]}{\Elim \Bair(Y)[8]}
	{
		\bot
	}
	\DeriveConclude{[*]}{\Elim \bot}{(<) \not \in \BP(X^2) }
	\EndProof
}
\Page{
	\DeclareType{SeparatelyContinuous}
	{
		\prod X,Y,Z \in \TOP \.
		 ? (X\times Y \to Z) 
	}
	\DefineType{f}
	{
		SeparatlyContinuous	
	}
	{
		\forall x \in X \. f_x \in \TOP(X,Z)
		\And
		\forall  y \in Y \. f^y \in \TOP(Y,Z) 
	}
	\\
	\Theorem{SeparateJointContinuityTHM}
	{
			\forall X,Y,Z \in \MS \.
			\forall	f : \TYPE{SeparatlyContinuous}(X,Y,Z) \. \NewLine \.
			\forall^* \TYPE{ContiniutyPoint}(f)
	}
	\Say{F}
	{
		\Lambda n,k \in \Nat \. 
		\bigg\{ 
			(x,y) \in X \times T :
			\forall u,v \in \mathbb{B}\Big(y,2^{-k}\Big) \.
			d\Big( f(x,u), f(x,v) \Big) \le 2^{-n}      
		\bigg\}
	}
	{
		\Nat^2 \to ?(X \times Y)
	}
	\Say{[1]}{\Elim F \Elim \TYPE{SeparatelyContinuous}(X,Y,Z,f)}
	{
		X \times Y = \bigcap^\infty_{n=1} \bigcup_{k=1}^\infty F_{n,k}
	}
	\AssumeIn{n,k}{\Nat}
	\Assume{(x,y)}{\Nat \to F_{n,k}}
	\AssumeIn{(x',y')}{X \times Y}
	\Assume{[2]}{\lim_{n \to \infty} (x_n,y_n) = (x',y')}
	\AssumeIn{u,v}{\mathbb{B}(y',2^{-k})}
	\Say{\Big( N, [3] \Big)}{\THM{ContinuousMetric}(Y)[2](u,v)}
	{
		\sum N \in \Nat \. \forall n \ge N \. u,v \in \mathbb{B}(y',2^{-k})
	}
	\Say{[4]}{\Elim F_{n,k} [3]}
	{
		\forall n \ge N \. d\Big( f(x_n,u), f(x_n,v) \Big) \le 2^{-n} 
	}
	\Conclude{\Big[(u,v).*\Big]}
	{
		\THM{ContinuousMetric}(X)[4][2]
	}
	{
		 d\Big( f(x',u), f(x',v) \Big) \le 2^{-n} 
	}
	\DeriveConclude{\Big[(n,k).*\Big]}{\Elim F_{n,k}}
	{
		(x',y') \in F_{n,k}
	}
	\Derive{[3]}
	{
		\Intro \Imply
		\Intro \forall
		\Intro \TYPE{Closed}
		\Intro \forall
	}
	{
		\forall n,k \in \Nat \.
		\TYPE{Closed}(X\times Y, F_{n,k})
	}
	\SayIn{D}{ 
		\bigcup^\infty_{n=1} \bigcup^\infty_{k=1}
		\{
			(x,y) : x \in F^y_{n,k} \setminus \intx F^y_{n,k}
		\}     
	}{\MGR(X\times Y)}
	\Say{[4]}{
		\Elim D	
	}
	{
		\forall y \in Y \. D^y \in \MGR(X)
	}
	\Say{G}{(X \times Y) \setminus D}
	{
		\TYPE{Comeager}(X \times Y)
	}
	\AssumeIn{(x,y)}{G}
	\AssumeIn{\varepsilon}{\Reals_{++}}
	\Say{\Big(n,[5]\Big)}{\Elim \TYPE{Archimedian}(\Reals,\varepsilon)}
	{
		\sum n \in \Nat \.  2^{-n} \le \varepsilon
	}
	\Say{\Big(k,[6]\Big)}{[1](x,y)(n)}
	{
		\sum k \in \Nat \. (x,y) \in F_{n,k}
	}
	\Say{[7]}{\Elim G[6] \Elim D^y }
	{
		x \in F^y_{n,k} \setminus D^y \subset \intx F^y_{n,k}
	}
	\Say{[8]}{\Elim \TOP(Y,Z,f^y)}
	{
		\forall V \in \U_X(x) \. 
		V \subset F^y_{n,k} \Imply
		\exists s \in V \. 
		d\Big( f(x,y),f(s,y) \Big) \le \varepsilon 
	}
	\AssumeIn{V}{\U_X(x)}
	\Assume{[9]}{V \subset F^y_{n,k}}
	\Say{\Big(s,[10]\Big)}{[8](V)[9]}
	{
		\sum_{s \in V} d\Big( f(x,y),f(s,y) \Big) \le \varepsilon 
	}
	\Conclude{\Big[ (x,y).*\Big]}{\THM{TriangleIneq}(Z)\Elim F_{n,k}^y(s)[5]}
	{
		\NewLine :		
		\forall t \in \mathbb{B}(y,2^{-k}) \. 
		d\Big( f(x,y), f(s,t)\Big) \le
		d\Big( f(x,y), f(s,y)\Big) + 
		d\Big( f(s,y), f(s,t)\Big) \le 2\varepsilon
	}
	\DeriveConclude{[*]}{
		\Intro \TYPE{ContinuityPoint}
	}
	{
		\TYPE{ContinuityPoint}(f,G)
	}
	\EndProof	
}
\Page{
	\Theorem{NamiokaTHM}
	{
		\forall X,Y \in \MS \And \Compact \.
		\forall Z \in \MS \.
		\forall	f : \TYPE{SeparatlyContinuous}(X,Y,Z) \. \NewLine \.
		\forall^* x \in X \. \forall y \in Y \. \TYPE{ContinuityPoint}\Big(f,(x,y)\Big)
	}
	\NoProof
	\\
	\Theorem{UltrafilterBairProperty}
	{
		\forall A : \TYPE{Ultrafilter}(\Nat) \.
		\neg \TYPE{Principle}(\Nat,A)
		\Imply  
		A \not \in \BP(\C)
	}
	\Assume{[1]}{A \in \BP(\C)}
	\Say{[2]}
	{
		\Elim \TYPE{NonPrinciple}(\Nat,A)	
	}
	{
		\forall n \in \Nat \. \exists b \in A : b_n = 0
	}
	\Say{[3]}{\Elim \TYPE{Ultrafilter}(\Nat,A)[2]}
	{
		\forall n \in \Nat \.
		\forall x \in \Bool^n \.
		\exists a \in A \.
		\inits{a}{n} = x
	}
	\Say{[4]}{
		\THM{StandardBaseIsBase}(\C)[4]\Intro \TYPE{Dense}(\C)
	}
	{
		\TYPE{Dense}(A,\C)
	}
	\Say{[5]}
	{[3] \Elim \TYPE{Ultrafilter}(\Nat,A) \Intro \TYPE{Dense}(\C)}{\TYPE{Dense}(A,\C)}
	\Say{[4]}{\Elim \BOOL\Big(\cat(\C)\Big)[5][7] \Elim \c }
	{
		[\C]_\cat = [\C]^2_\cat =
		[A]_\cat[A^\c]_\cat = 0
	}
	\Conclude{[1.*]}{\Elim \Bair(\C)[4]}{\bot}
	\DeriveConclude{[*]}{\Elim \bot}{A \not \in \BP(\C)}
	\EndProof
}
\newpage
\section{Borel Topology}
\subsection{Measurability}
\subsubsection{Algebras of Sets}
\Page{
	\DeclareType{Algebra}{\prod_{X \in \SET} ?^3X}
	\DefineType{\F}{Algebra}{\F \subset_\BOOL ?X}
	\\
	\DeclareType{\SA}{\prod_{X \in \SET} ?^3X}
	\DefineType{\F}{\SA}{\F \subset_\BOOL^\sigma ?X}
	\\
	\DeclareFunc{generateSigmaAlgebra}{\prod_{X \in \SET}??X \to \SA(X) }
	\DefineNamedFunc{generateSigmaAlgebra}{S}{\sigma(S)}
	{
		\bigcap \Big\{ \A \Big| \SA(X,\A), S \subset \A   \Big\}
	}
	\\
	\DeclareType{\CGSA}
	{
		\prod_{X \in \SET} ?\SA(X)
	}
	\DefineType{\F}{\CGSA}
	{
		\exists S : \Countable(X) \. \F = \sigma(X)
	}
	\\
	\DeclareType{\MC}{\prod_{X \in \SET} ???X}
	\DefineType{\mathcal{M}}{\MC}
	{
		\left(
			\forall	A : \Nat \uparrow \mathcal{M} \. 
			\bigcup^\infty_{n=1} A_n \in \mathcal{M}
		\right)
		\And 
		\left(
			\forall	A : \Nat \downarrow \mathcal{M} \. 
			\bigcap^\infty_{n=1} A_n \in M
		\right)
	}
}\Page{
	\Theorem{MonotonicClassLemma}
	{
		\forall X : \SET \.
		\forall \A : \Alg(X) \.
		\sigma(\A)  = \bigcap\Big\{ \mathcal{M} :\MC(X,\mathcal{M}) 
		\Big| \A \subset \mathcal{M}\Big\}
	}
	\Say{\B}{ 
		\bigcap
		\Big\{ \mathcal{M} 
	   	\Big| \MC(X,\mathcal{M}), 
	   	\A \subset \mathcal{M}\Big\}
	   }{??X}
	\Say{[1]}{\Elim \B \Elim \sigma(\A)\Elim \SA\Big(X,\sigma(\A)\Big)\Intro \B}
	{
		\B \subset \sigma(\A)
	}
	\AssumeIn{A}{\A}
	\Say{\C_A}{\{ C \subset X :  C \setminus A, A \setminus C, A \cap C \in \B \}}
	{
		???X
	}
	\Say{[2]}{\Elim \C_A \Elim \B}{\MC(X,\C_A)}
	\Say{[3]}{\Elim \C_A \Elim \B \Elim \Alg(X,\A)}{\A \subset \C_A}
	\Conclude{[A.*]}{[2][3]\Elim \B}{\C_A \subset \B}
	\Derive{[2]}{\Intro \forall}
	{
		\forall A \in \A \. 
		\forall B \in \B \.
		A \setminus B, B \setminus A, B \cap A \in \B
	}
	\AssumeIn{B}{\B}
	\Say{\C_B}{\{ C \subset X :  C \setminus B, B \setminus C, B \cap C \in \B \}}
	{
		\MC(X)
	}
	\Say{[3]}{
			\Elim \C_B [2]
	}
	{
		\A \subset \C_B
	}
	\Conclude{[*.1]}{[3]\Elim \B}{\B \subset \C_B}
	\Derive{[3]}{\Intro \Alg }{\Alg(X,\B)}
	\Say{[4]}{\Elim \MC(\B)\Intro \SA}{\SA(X,\B)}
	\Say{[5]}{\Elim \sigma(\A)[4]}{\sigma(\A) \subset \B}
	\Conclude{[*]}{\Intro \TYPE{SetEq}[1][5]}{\sigma(\A) = \B}
	\EndProof
	\\
	\DeclareType{PiClass}{\prod_{X \in \SET} ???X}
	\DefineNamedType{\mathcal{P}}{PiClass}{\TYPE{\pi\hyph Class}(\mathcal{P})}
	{
		\forall A,B \in \mathcal{P} \. A \cap B \in \mathcal{P}
	}
	\\
	\DeclareType{DisjointSeq}{\prod_{X \in \SET} \prod_{\A : ??X} \Nat \to \A}
	\DefineType{A}{DisjointSeq}
	{
		\forall n ,m \in \Nat \. n \neq m \to A_n \cap A_m = \emptyset	
	}
	\\
	\DeclareType{LambdaClass}{\prod_{X \in \SET} ???X}
	\DefineNamedType{\mathcal{L}}{LambdaClass}{\TYPE{\lambda\hyph Class}(\mathcal{L})}
	{
		\forall A \in \mathcal{L} \. A^\c \in \mathcal{L} \And
		\forall  A : \TYPE{DisjointSeq}(\mathcal{L}) \. \bigcup^\infty_{n=1} A_n \in \L
	}
}\Page{
	\Theorem{PiLambdaClassLemma}
	{
		\forall X : \SET \.
		\forall \mathcal{P} : \TYPE{\pi\hyph Class}(X) \.
		\sigma(\mathcal{P})  = \bigcap\Big\{ \mathcal{L} : 
			 \TYPE{\lambda\hyph Class}(X,\mathcal{L})\Big|
			\mathcal{P} \subset \mathcal{L}\Big\}
	}
	\Say{\mathcal{L}}{ 
		\bigcap\Big\{ \mathcal{L} : 
			 \TYPE{\lambda\hyph Class}(X,\mathcal{L})\Big|
			\mathcal{P} \subset \mathcal{L}\Big\}
	   }{??X}
	\Say{[1]}{\Elim \L \Elim \sigma(\P)\Elim \SA\Big(X,\sigma(\P)\Big)\Intro \L}
	{
		\L \subset \sigma(\P)
	}
	\AssumeIn{A}{\P}
	\Say{\C_A}{\{ C \subset X :  A \cap C \in \L \}}
	{
		???X
	}
	\AssumeIn{C}{\C_A}
	\Say{[2]}{\Elim \C_A(C)}{C \cap A \in \L }
	\Say{[3]}{ 
		\THM{DeMorgannaLaw}(?X)
		\THM{UnionDisjoining}(?X) 
		[2]
		\Elim  \TYPE{\lambda\hyph Class}(X,\mathcal{L})
	}
	{
		\NewLine :		
		C^\c \cap  A = 
		(C \cup A^\c)^\c  =
		( (C \cap A) \cup A^\c )^\c \in \L
	}
	\Conclude{[C.*]}{\Elim \C_A [3]}{C^\c \in \C_A}
	\Derive{[2]}{\Intro \forall}{\forall C \in \C_A \. C^\c \in \C_A}
	\Say{[3]}{
		\Elim \C_A 
		\Elim \TYPE{AssociativeLattice}(?X)
		\Elim  \TYPE{\lambda\hyph Class}(X,\mathcal{L})
	}
	{
		\forall C : \TYPE{DisjointSeq}(\C_A) \.
		\bigcup^\infty_{n=1} C_n \in \C_A
	}
	\Say{[4]}{\Intro \TYPE{\lambda\hyph Class}}
	{
		\TYPE{\lambda\hyph Class}(X,\C_A)
	}
	\Say{[5]}{\Elim \C_A \Elim \TYPE{\pi\hyph Class}(X,\P)}
	{
		\P \subset \C_A
	}
	\Conclude{[A.*]}{[4][5]\Elim \L}{\C_A \subset \L}
	\Derive{[2]}{\Intro \forall}
	{
		\forall A \in \P \. 
		\forall B \in \L \.
		B \cap A \in \B
	}
	\AssumeIn{B}{\L}
	\Say{\C_B}{\{ C \subset X :   B \cap C \in \L \}}
	{
		\TYPE{\lambda \hyph Class}(X)
	}
	\Say{[3]}{
			\Elim \C_B [2]
	}
	{
		\P \subset \C_B
	}
	\Conclude{[*.1]}{[3]\Elim \L}{\L \subset \C_B}
	\Derive{[3]}{\Intro \Alg \THM{UnionDisjoinig}(?X) }{\Alg(X,\L)}
	\Say{[4]}{\Elim \TYPE{\lambda\hyph Algebra}(\L)\Intro \SA}{\SA(X,\L)}
	\Say{[5]}{\Elim \sigma(\P)[4]}{\sigma(\P) \subset \L}
	\Conclude{[*]}{\Intro \TYPE{SetEq}[1][5]}{\sigma(\P) = \L}
	\EndProof
	\\
	\Theorem{SigmaAlgebraGenerationWithDisjoinUinion}
	{
		\forall X \in \SET \.
		\forall  \A  : ??X \. \NewLine
		\sigma(\A) = \bigcap \left\{
			\B : ??X \Bigg| 
			\A \subset \B,
			\A^\c \subset \B,
			\forall B : \Nat \to \B \. \bigcap^\infty_{n=1} B_n \in \B,
			\forall B : \TYPE{DisjointSeq}(\B) \. \bigcup^\infty_{n=1} B_n \in \B
		\right\}
 	}
 	\NoProof
}
\newpage
\subsubsection{Measurable Category}
\Page{
	\Conclude{\TYPE{MeasurableSet}}{\sum X \in \SET \. \SA(X) }{\TYPE}
	\\
	\DeclareFunc{measurableSetAsSet}{\TYPE{MeasurableSet} \to \SET}
	\DefineNamedFunc{measurableSetAsSet}{(X,\S)}
	{(X,\S)}{X}
	\\
	\DeclareFunc{algebra}{\prod (X,\F) : \TYPE{MeasurableSet} \to \SA(X)}
	\DefineNamedFunc{algebra}{}
	{\S_{(X,\F)}}{\F}
	\\
	\DeclareType{MeasurableMap}
	{
		\prod X,Y : \TYPE{MeasurableSet} \. ?(X \to Y)
	}
	\DefineType{f}{MeasurableMap}{\forall A \in \S_Y \. f^{-1}(A) \in \S_X}
	\\
	\DeclareFunc{categoryOfBorel}{\CAT}
	\DefineNamedFunc{categoryOfBorel}{}{\BOR}
	{
		(\TYPE{MeasurableSet},\TYPE{MeasurableMap},  \circ,\id)
	}
	\\
	\DeclareFunc{forgetfulFunctorBor}{\Cov(\BOR,\SET)}
	\DefineNamedFunc{forgetfulFunctorBor}{X,\S}{\mathsf{U}_\BOR(X,\S)}
	{
		X
	}
	\DefineNamedFunc{forgetfulFunctorBor}{X,Y,f}{\mathsf{U}_{\BOR;X,Y}(f)}
	{
		f
	}
	\\
	\DeclareFunc{discreteMeasurableStructureFunctor}{\Cov(\SET,\BOR)}
	\DefineNamedFunc{discreteMeasurableStructureFunctor}{X}{\mathsf{F}_\BOR(X)}
	{
		(X,2^X)
	}
	\DefineNamedFunc{discreteMeasurableStructureFunctor}{X,Y,f}{\mathsf{F}_{\BOR;X,Y}(f)}
	{
		f
	}
	\\
	\DeclareFunc{codiscreteMeasurableStructureFunctor}{\Cov(\SET,\BOR)}
	\DefineNamedFunc{codiscreteMeasurableStructureFunctor}{X}{\mathsf{F}^\BOR(X)}
	{
		(X,\{\emptyset, X\})
	}
	\DefineNamedFunc{codiscreteMeasurableStructureFunctor}{X,Y,f}
	{\mathsf{F}^\BOR_{X,Y}(f)}
	{
		f
	}
	\\
	\Theorem{AdjointStructure}{
		\mathsf{F}_\BOR \dashv \mathsf{U}_\BOR \dashv  \mathsf{F}^\BOR}
	\NoProof
	\\
	\DeclareFunc{algebraFunctor}
	{
		\Contra(\BOR,\BOOL_\sigma)
	}
	\DefineNamedFunc{algebraFunctor}{X}{\mathsf{A}(X)}
	{
		\S_X
	}
	\DefineNamedFunc{algebraFunctor}{X,Y,f}
	{\mathsf{A}_{X,Y}(f)}
	{
		f_*
	}
	\\
	\DeclareFunc{embededStoneFunctor}
	{
		\Contra(\BOOL_\sigma,\BOR)
	}
	\DefineNamedFunc{embededStoneFunctor}{A}{\mathsf{Z}(A)}
	{
		\Big( Z_A, S_A(A)  \Big)
	}
	\DefineNamedFunc{embededStoneFunctor}{A,B,f}
	{\mathsf{Z}_{A,B}(f)}
	{
	     Z_A(f)
	}
}\Page{
	\DeclareFunc{initialMeasurableStructure}
	{
		\prod X,I \in \SET \.
		\prod Y : I \to \BOR \.
		\left(\prod_{i \in I} X \to Y_i\right) \to \SA(X)
	}
	\DefineNamedFunc{initialMeasurableStructure}{f}
	{
		\I_X(I,Y,f)
	}
	{
		\inf \Big\{ \A : \SA(X) 
			\Big| 
			\forall i \in I \. f_i \in \BOR\big((X,\A),Y_i\big) 
		\Big\}
	}
	\\
	\DeclareFunc{finalMeasurableStructure}
	{
		\prod Y,I \in \SET \.
		\prod X : I \to \BOR \.
		\left(\prod_{i \in I} Y_i \to X\right) \to \SA(X)
	}
	\DefineNamedFunc{initialMeasurableStructure}{f}
	{
		\F_Y(I,X,f)
	}
	{
		\sup \Big\{ \A : \SA(Y) 
			\Big| 
			\forall i \in I \. f_i \in \BOR\big(X_i,(Y,\A)\big) 
		\Big\}
	}
	\\
	\Theorem{BorIsBicomplete}
	{
		\TYPE{Bicomplete}(\BOR)
	}
	\Explain{Define all limits and colimits as in $\SET$}
	\Explain{Then equip them with initial or respectively final measurable structure.}
	\Explain{It is easy to see that this constructions have universal properties.}
	\Explain{This is analogues to what eas done with $\TOP$ 
	in some model theoretic sence.}
	\EndProof
	\\
	\Theorem{MeasurableSection}
	{
		\forall I \in \SET \.
		\forall X : I \to \BOR \.
		\forall i \in \I \.
		\forall x \in \prod_{i\neq j} X_j \.
		\forall A \in \alg\left( \prod_{i \in I} X_i \right) \.
		\sigma_{i,x}(A) \in \alg(X_i)
	}
	\Say{\C}{
		\left\{
			    \prod_{i\in I} A_i  \Bigg|
			    A \in \prod_{i\in I} \alg(X_i)	
		\right\}
	}
	{
			??\prod_{i \in I} X_i
	}
	\Say{[1]}{\Elim\left( \BOR, \prod_{i \in I} X_i \right)}
	{
			\alg\left( \prod_{i \in I} X_i \right) = \sigma(\C)
	}
	\Say{\B}{
		\left\{
			A \subset   \prod_{i \in I} X_i \Bigg|  
			\sigma_{i,x}(A) \in \alg(X_i)
		\right\}	
	}{?? \prod_{i \in I} X_i}
	\Say{[2]}{\Elim \C \Elim \B \Elim \sigma_{i,x}}
	{
		\C \subset \B	
	}
	\Say{[3]}{\Elim \B \Elim \SA\Big(X_i,\alg(X_i)\Big)\Elim \sigma_{i,x}\Intro \SA}
	{
		\SA\left(\prod_{i\in I} X_i, \B \right)
	}
	\Say{[4]}{\Elim \sigma [1][2][3]}{\alg\left(\prod_{i\in I} X_i\right) \subset \B}
	\Conclude{[*]}{\Elim \B [4]}
	{
		\forall A \in \alg\left( \prod_{i \in I} X_i \right) \.
		\sigma_{i,x}(A) \in \alg(X_i)
	}
	\EndProof
}\Page{
	\Theorem{MeasurablePartialComputation}
	{	
		\forall I \in \SET \.
		\forall X : I \to \BOR \.
		\forall Y \in \BOR \.
		\forall i \in I \.
		\forall x \in \prod_{j \neq i } X_j \. \NewLine \.
		\forall f \in \BOR\left( \prod_{i \in I} X_i, Y\right) \.
		f(x) \in \BOR(X_i,Y)
	}
	\Explain{Let $A$ be measurable in $Y$}
	\Explain{Then,  $(f(x))^{-1}(A) = \sigma_{i,x}\Big(f^{-1}(A)\Big)$}
	\Explain{This is measurable by the previous theorem, and so $f(x)$ is measurable}
	\EndProof
}
\newpage
\subsection{Borel Basics}
\subsubsection{Sets and the Functor}
\Page{
	\DeclareFunc{functorOfBorel}
	{
		\Cov(\TOP,\BOR)	
	}
	\DefineNamedFunc{functorOfBorel}{(X,\T)}{\bor(X,\T)}{\Big(X,\sigma(\T)\Big)}
	\DefineNamedFunc{functorOfBorel}{X,Y,f}{\bor_{X,Y}(f)}{f}
	\Say{\A}{\F_Y\Big(1,\bor(X),f\Big)}{\SA(Y)}
	\Say{[1]}{\Elim \A \Elim \bor(X)\Elim \TOP(X,Y,f)}{\T(Y) \subset \A}
	\Say{[2]}{\Elim \sigma [1]}{ \sigma\Big( \T(Y) \Big) \subset \A}
	\Conclude{[*]}{\Intro \bor [2] \Elim \A \Elim \F_Y}
	{
		f \in \BOR\Big( \bor(X),\bor(Y) \Big)
	}
	\EndProof
	\\
	\DeclareFunc{borelAlgebra}{\Contra(\TOP,\BOOL_\sigma)}
	\DefineNamedFunc{borelAlgebra}{}{\B}{\bor\alg}
	\\
	\Theorem{CountablyGeneratedBorel}
	{
		\NewLine ::		
		\forall X \in \TOP \.
		\forall \U : \TYPE{SubbaseOfTopology}(X) \.
		\forall |\U| \le \aleph_0 \.
		\CGSA\Big( \B(X) \Big)	
	}
	\NoProof
	\\
	\Theorem{CountablyGeneratedBorel2}
	{	
		\forall X  : \TYPE{SecondCountable} \.
		\CGSA\Big( \B(X) \Big)	
	}
	\NoProof
	\\
	\Theorem{BorelContainsOpen}
	{		
		\forall X \in \TOP \.
		\T(X) \subset \B(X)		
	}
	\NoProof
	\\
	\Theorem{BorelContainsClosed}
	{		
		\forall X \in \TOP \.
		\Closed(X) \subset \B(X)		
	}
	\NoProof
	\\
	\Theorem{BorelContainsGdelta}
	{		
		\forall X \in \TOP \.
		G_\delta(X) \subset \B(X)		
	}
	\NoProof
}\Page{
	\Theorem{BorelContainsFSigma}
	{		
		\forall X \in \TOP \.
		F_\sigma(X) \subset \B(X)		
	}
	\NoProof
	\\
	\Theorem{CountableBorelCommutesWithCountableProducts}
	{
		\NewLine ::		
		\forall n \in \sigma(\omega) \.
		\forall X : n \to \TYPE{SecondCountable} \.
		\prod^n_{i=0} \bor(X_i) = \bor\left( \prod^n_{i=0} X_i\right)
	}
	\NoProof
	\\
	\Conclude{\LOGIC{BorelMeasurable}}
	{
		\Lambda X \in \BOR \.
		\Lambda Y \in \TOP  \.
		\BOR(X,Y) \iff \BOR\Big( X, \bor(Y) \Big)	
	}
	{
		\LOGIC{Polymorphism}	
	}
	\\
	\Theorem{MeasurableBySubbase}
	{
		\forall X \in \BOR \.
		\forall Y \in \TOP \.
		\forall f : X \to Y \.
		\forall \U : \TYPE{SubbaseOfTopology}(Y) \. \NewLine \.
		\forall [0] : \forall U \in \U \. f^{-1}\; U  \in \S_X \.
		f \in \BOR(X,Y) 
	}
	\Say{[1]}{
		\Elim \BOR(X)\Elim \TYPE{SubbaseOfTopology}(Y,\U) [0]
		\THM{UnionPreimage}(X,Y,f) \NewLine
		\THM{IntersectionPreimage}(X,Y,f)
	}
	{
		\forall U \in \T(Y) \. f^{-1} \; U \in \S_X
	}
	\Say{\A}{\F_Y\Big(1,X,f\Big)}{\SA(Y)}
	\Say{[2]}{\Elim \A [1]}{\T(Y) \subset \A}
	\Say{[3]}{\Elim \sigma [2]}{ \sigma\Big( \T(Y) \Big) \subset \A}
	\Conclude{[*]}{\Intro \BOR [3] \Elim \A \Elim \F_Y}
	{
		f \in \BOR\Big(  X, Y \Big)
	}
	\EndProof
}
\newpage
\subsubsection{Hierarchi}
\Page{
	\DeclareFunc{hierarchiOfBorel}
	{
		\prod X : \TYPE{Metrizable} \. 
		\omega_1 \to (?X)^2
	}
	\DefineNamedFunc{hierchiOfBorel}{}{(\Sigma^0(X),\Pi^0(X))}
	{
		\FUNC{boundedCompleteTransfiniteRecursion} \NewLine
		\Bigg(
				\ORD,
				\Big( \Open(X),\Closed(X) \Big),
				\lambda \kappa \in (1, \omega_1) \.
				\lambda \Big(\Sigma,\Pi\Big):\kappa \to (?X)^2 \.
				\NewLine \.				
				\left( 
					\left\{ 
						\bigcup^\infty_{n=1} A_n    
						\Bigg| 
						\xi : \Nat \to \kappa, A : \prod^\infty_{n=1} \Pi_{\xi_n} 
					\right\},
					\left\{ 
						\bigcap^\infty_{n=1} A_n    
						\Bigg| 
						\xi : \Nat \to \kappa, A : \prod^\infty_{n=1} \Sigma_{\xi_n} 
					\right\}
				\right)  
		\Bigg)	
	}
	\\
	\DeclareFunc{ambigiousClass}
	{
		\prod X : \TYPE{Metrizable} \. 
		\ORD \to ?X
	}
	\DefineNamedFunc{ambigiousClass}{\kappa}{\Delta_\kappa^0(X)}
	{\Sigma_\kappa^0(X) \cap \Pi_\kappa^0(X)}
	\\
	\Theorem{DirectBorelHierarchi}
	{
		\NewLine ::		
		\forall X : \TYPE{Metrizable}(X) \.
		\forall \kappa \in \ORD \.
		\forall \xi \in \kappa \.
		\Sigma_{\xi}^0 \subset \Sigma_{\kappa}^0
		\And
		\Pi_{\xi}^0 \subset \Pi_{\kappa}^0
	}
	\Say{[1]}{\Elim \Sigma^0_1(X)}{\Sigma^0_1(X) = \T(X)}
	\Say{[2]}{\Elim \Sigma^0_2(X)}{\Sigma^0_2(X) = F_\sigma(X)}
	\Say{[3]}{\Elim \Pi^0_1(X)}{\Sigma^0_1(X) = \TYPE{Closed}(X)}
	\Say{[4]}{\Elim \Pi^0_2(X)}{\Sigma^0_2(X) = G_\delta(X)}
	\Say{[5]}{\THM{OpenIsFSigma}(X)[1][2]}{\Sigma^0_1(X) \subset \Sigma^0_2(X)}
	\Say{[6]}{\THM{ClosedIsGDelta}(X)[3][4]}{\Pi^0_1(X) \subset \Pi^0_2(X)}	
	\Conclude{[*]}{\Elim (\Sigma,\Pi)[5][6]}
	{
		\forall \kappa \in \ORD \.
		\forall \xi \in \kappa \.
		\Sigma_{\xi}^0 \subset \Sigma_{\kappa}^0
		\And
		\Pi_{\xi}^0 \subset \Pi_{\kappa}^0
	}
	\EndProof
	\\
	\Theorem{DirectAmbiguousClasses}
	{
		\forall X : \TYPE{Metrizable}(X) \.
		\forall \kappa \in \ORD \.
		\forall \xi \in \kappa \.
		\Delta_{\xi}^0 \subset \Delta_{\kappa}^0
	}
	\NoProof
	\\
	\Theorem{BorelHierarchiComplementation}
	{
		\forall X : \TYPE{Metrizable}(X) \.
		\forall A \subset X \.
		\forall \kappa \in \ORD \.
		A \in \Sigma^0_\kappa \iff A^\c \in \Pi^0_\kappa 
	}
	\NoProof
	\\
	\Theorem{BorelTransfiniteExpression}
	{
		\forall X : \TYPE{Metrizable}(X) \. 
		\exists \xi \in \ORD \.
		\bigcup_{\kappa < \xi} \Sigma^0_\kappa = 
		\bigcup_{\kappa < \xi} \Pi^0_\kappa = 
		\bigcup_{\kappa < \xi} \Delta^0_\kappa = 
		\B(X)
	}
	\Say{\alpha}{\min \Big\{ \kappa \in \ORD : |\kappa| = 2^{2^|X|} \Big\}}{\ORD}
	\Explain{
			By cardinality limitation $\Sigma^0,\Pi^0$ and $\Delta_0$ will stabilize 
			until reaching $\alpha$
		}
	\Explain{So, their unions are closed under finite unions and intersections}
	\Explain{Hence,This unions are sigma-algebras and contain $\B(X)$}
	\Explain{Also by simple transfinite induction they all consist of members of $\B(X)$}
	\Explain{The result follows}
	\EndProof
}
\newpage
\subsubsection{Examples}
\Page{
	\DeclareType{Simple}{\prod_{X \in \BOR} \BOR(X,\Reals)}
	\DefineType{\varsigma}{Simple}
	{
		\exists n \in \Nat \. 
		S : \{1,\ldots,n\} \to \S_X \.
		\alpha : \{1,\ldots,n\} \to  \Reals \.
		\varsigma = \sum^n_{i=1} \alpha_i \chi_{S_i}
	}
	\\
	\DeclareFunc{binaryDigits}{\Nat \to \BOR\Big([0,1], \Bool\Big)}
	\DefineNamedFunc{binaryDigits}{n,t}{\beta_n(t)}{
		\sum^{2^{n-1}}_{k=0}  \delta_t \left( \frac{2k+1}{2^n}, \frac{2k+2}{2^n} \right]
	} 
	\\
	\DeclareType{NormalNumbers}{\B[0,1]}
	\DefineNamedType{\alpha}{NormalNumbers}{\alpha \in \overline{\Nat}}
	{
		\lim_{n \to \infty} \frac{\sum^n_{i=1} \beta_i(\alpha)}{n} = \frac{1}{2}
	}
	& \overline{\Nat} = 
		\bigcap_{\varepsilon \in \Rats_{++}}
		\bigcup^\infty_{m=1}
		\bigcap_{n \ge m} \left\{ \alpha \in [0,1] : 
		\left| \frac{\sum^n_{i=1} \beta_i(\alpha)}{n} - \frac{1}{2} \right| < \varepsilon  
		\right\}  \\
	\EndProof
	\\
	\Theorem{ContDiffirientiableIsBorel}
	{
		C^1[0,1] \in \B\Big( C[0,1]\Big)
	}
	& \mathcal{I} = \Lambda n \in \Nat \. 
	\left\{
		I :  \{1,\ldots,n\} \to \TYPE{OpenInterval}\Big( \Rats \cap [0,1] \Big), 
		[0,1] = \bigcup^n_{i=1} I_i
	\right\} \\
	&  \forall n \in \Nat \. |\I_n| \le \aleph_0     \\
	&
		C^1[0,1] = 
		\bigcap_{\varepsilon \in \Rats_{++}}
		\bigcup^\infty_{n=1}
		\bigcup_{I \in \I_n}
		\bigcap^n_{k=1}
		\left\{                                                 
				f \in C[x,y] : 
				\forall a,b,c,d \in  I_k \.
				b > a, d > c \.
				\left|
					\frac{f(b) - f(a)}{b - a} - \frac{f(d) - f(c)}{d - c}
				\right|   \le \varepsilon
		\right\}
	\\
	\EndProof
	\\
	\Theorem{ZeroConvergentIsBorel}
	{
		l_0 \in \B(l_\infty)
	}
	&
		l_0 = 
		\bigcap_{\varepsilon \in \Rats_{++}}
		\bigcup^\infty_{m=1}
		\bigcap^\infty_{n=m}
		\Big\{                                                 
				x \in l_\infty \.  |x_n| < \varepsilon
		\Big\}
	\\
	\EndProof
	\\
	\Theorem{PointsOfDifferentiabilityIsBorel}
	{
		\forall f \in C[0,1] \. 
		D_f \in \B[0,1]
	}
	&
		D_f = 
		\bigcap_{\varepsilon \in \Rats_{++}}
		\bigcup_{\delta \in \Rats_{++}}
		\bigcap_{a,b \in \Rats \cap [0,1]}
		\left\{                                                 
				t \in [0,1]  :  
				0<  |a -t| < \delta, 
				0 < |b -t| < \delta,
				\left|
					\frac{f(t) - f(a)}{t - a}
					-
					\frac{f(t) - f(b)}{t - b}
				\right| 
				<
				\varepsilon 
		\right\}
	\\
	\EndProof
}
\newpage
\subsubsection{Functions}
\Page{
	\Theorem{BorelMeasurablePointwiseConvergence}
	{
		\NewLine		
		\forall X \in \BOR \.
		\forall Y : \TYPE{Metrizble} \.
		\forall \phi : \Nat \to \BOR(X,Y) \.
		\forall \varphi : X \to Y \.
		\forall [0] : (\mathrm{pt})\;\varphi = \lim_{n \to \infty} \phi_n \.
		\varphi \in \BOR(X,Y)
	}
	\Say{d}{\Elim \TYPE{Metrizable}(Y)}
	{
		\TYPE{Metrizes}(Y,d)
	}	
	\AssumeIn{K}{\Closed(Y)}
	\Conclude{[K.*]}{
		\Elim \FUNC{preimage}(X,Y,\varphi,K)
		\Elim \FUNC{pointwiseConvergence}\Big(X,(Y,d),f,\varphi\Big) \NewLine
		\Lambda n \in \Nat \. \Intro \FUNC{preimage}(X,Y,\varphi,K)	
		\Lambda n \in \Nat \.  \Elim \BOR(X,Y,f_n)
		\Elim \BOR(X)
	}
	{
		\NewLine :		
		\varphi^{-1}(K) =
		\{   x \in X \. \varphi(x) \in K \}  =
		\bigcap_{\varepsilon \in \Rats_{++}}
		\bigcup_{m=1}^\infty \bigcap_{n=m}^\infty
		\Big\{   x \in X \. d(f_n(x),K) < \varepsilon \Big\} = \NewLine =
		\bigcap_{\varepsilon \in \Rats_{++}}\bigcup_{m=1}^\infty \bigcap_{n=m}^\infty
		f^{-1}_n \bigcup_{y \in K} \mathbb{B}_d(y,\varepsilon) \in \S_X
	}
	\Derive{[*]}{\THM{MeasurableByGenerators}}{
		\varphi \in \BOR(X,Y)	
	}
	\EndProof
	\\
	\Theorem{DerivativeIsBorelMeasurable}
	{
		\forall f \in \mathsf{DIFF}\Big([0,1],\Reals\Big) \.
		\varphi' \in \BOR\Big([0,1],\Reals\Big)
	}
	\Say{\alpha}{ \Lambda t \in [0,1] \. \Lambda s \in [0,1] \. \min(t+s,1)   }
	{
		[0,1]^2 \to [0,1]
	}
	\Say{g}
	{
		\Lambda n \in \Nat \.
		\Lambda t \in [0,1] \.
		\If  t < 1
		\Then
		\frac{\varphi\Big(f(t,2^{-n})\Big) - f(t)}
		{
			\alpha(t,2^{-n}) - t
		}
		\Else
		f'(1)
	}
	{
		\Nat \to \BOR\Big( [0,1], \Reals \Big)
	}
	\Conclude{[*]}{\THM{BorelMeasurablePointwiseConvergence}}
	{
		f' = (\mathrm{pt}) \; \lim_{n \to \infty} g_n \in \B\Big([0,1],\Reals\Big)
	}
	\EndProof
	\\
	\Theorem{SemicontinuousIsMeasurable}
	{
		\forall X \in \TOP \.
		\forall f : \TYPE{Semicontinuous}(X) \.
		f \in \BOR(X,\Reals)
	}
	\Explain{ Half-intervals are expressible as intersections of open rays and their complements}
	\Explain{ Open intervals are expressible as countable unions or intersection of half-intervals}
	\Explain{ Open subsets of real line are expressible as countable disjoint unions of open intervals.}
	\Explain{ Preimages of open rays are open for semicontinuous functions are open.}
	\Explain{ This means that preimage of an open set is Borel}
	\Explain{ Hence, the semicontinuous functions are Borel-measurable}
	\EndProof
}\Page{
	\Theorem{BorelByPartialComputaions}
	{
		\NewLine ::		
		\forall X,Z : \TYPE{Metrizable} \.
		\forall Y \in \TOP \.
		\forall D : \Dense(X) \.
		\forall f : X \times Y \to Z \. 
		\forall [0.1] : |D| \le \aleph_0 \. \NewLine \.
		\forall [0.2] : \forall y \in Y \. f(\bullet,y) \in \TOP(X,Z) \.
		\forall [0.3] : \forall x \in D \.  f(x,\bullet) \in \BOR(Y,Z) \.
		f \in \BOR(X\times Y, Z)
	}
	\Say{\rho}{\Elim \TYPE{Metrizable}(X)}
	{
		\TYPE{Metrizes}(X,\rho)
	}
	\Say{\delta}{\Elim \TYPE{Metrizable}(Z)}
	{
		\TYPE{Metrizes}(Z,\sigma)
	}
	\Say{d}{\FUNC{enumerate}(D)}{\TYPE{Surjective}(\Nat,D)}
	\Say{K}{\Lambda n \in \Nat \.  \{d_1,\ldots,d_n\}}{\Nat \to \Finite(X)}
	\Say{\sigma}{ 
		\Lambda n \in \Nat \. \Lambda x \in X \. 
		d\Big( \min \{  m \in \argmin_{1 \le m \le n} \rho(d_m,x)\} \Big)    
	}
	{
		\prod_{n=1}^\infty  (X \to K_n)
	}
	\Say{g}{\Lambda n \in \Nat \. \Lambda (x,y) \in X\times Y \. f\Big( \sigma_n(x), y\Big)}
	{
		\Nat \to (X \times Y) \to Z
	}
	\AssumeIn{y}{Y}
	\AssumeIn{x}{X}
	\AssumeIn{U}{ \U\Big( f(x,y) \Big) }
	\SayIn{V}{f^{-1}(\bullet ,y)(U)}{\U(x)}
	\Say{\Big(r, [1] \Big)}{\THM{MetricTopology}(X,\rho,V)}
	{
		\sum R \in \Reals_{++} \. \Cell_\rho(x,r) \subset V
	}
	\Say{\Big(N,[2] \Big)}{\Elim d \Elim \Dense(X,D)(V)}
	{
		\sum N \in \Nat \. d_N \in \Cell_\rho\left(x, r\right)
	}
	\Assume{n}{\Nat}
	\Assume{[3]}{n \ge N}
	\Say{\Big(m,[4]\big)}{\Elim g_n [2][3]}
	{\sum^\infty_{m=N} g_n(x,y) = f(d_m,y) \And \rho(d_m,x) < r}
	\Say{[5]}{[4.2][1]}{ d_m \in V}
	\Conclude{[y.*]}{\Elim V [4.1][5]}{g_n(x,y) \in U}
	\Derive{[1]}{\Intro (\mathrm{pt})\;\lim}
	{
		(\mathrm{pt})\;\lim_{n \to \infty} g_n = f
	}
	\Say{B}{\Lambda n \in \Nat \. \Lambda k \in \{1,\ldots,n\} \sigma_n^{-1}(d_k)}
	{
		\prod^\infty_{n=1} \{1,\ldots,n\}  \to \B(X)
	}
	\Say{[2]}{
		\Lambda n \in \Nat
		\Lambda A \in \alg(Z)
		\Lambda k \in \{1,\ldots,n\} \.
		\Elim g_n
		[0.3](d_k,A)
		\NewLine
		\THM{CountableBorelCommutesWithCountableProducts}\Big(2,(X,Y)\Big)	
	}
	{
		\NewLine :		
		\forall n \in \Nat \.		
		\forall A \in \alg(Z) \.
		g_n^{-1}(A) =    \bigcup^n_{k=1} B_k \times f^{-1}(d_k,\bullet)(A) \in \B(X\times Y)
	}
	\Say{[3]}{\Intro \BOR [2]}{\forall n \in \Nat \. g_n \in \BOR(X\times Y, Z)}
	\Conclude{[*]}{
		\THM{BorelMeasurablePointwiseConvergence}(X\times Y, Z,g, f)[1][3]	
	}
	{
		f \in \BOR(X\times Y, Z)
	}
	\EndProof
}\Page{
	\Theorem{VietorisBorelSetsGeneration1}
	{
		\forall X : \Polish \.
		\B\Big( \K(X) \Big)  = \sigma\Big\{                     
			\big\{
				K \in \K(X)  : K \subset U   
			\big\}	
			\Big|   U \in \T(X)	
		\Big\}
	}
	\Explain{ 
		We need to express
		sets of form $\{ K \in \K(X) : \exists K \cap U   \}$ 
		by sets of form$\{ K \in \K(X)  : K \subset V    \}$	
	}
	\Explain{ Let $\rho$ be a metrization for $X$ and let $(d_n)^\infty_{n=1}$ be dense in it}
	\Explain{ First, note that $U$ is a $F_\sigma$ set}
	\Explain{ So, there is a sequence of closed sets $A$ such that $U = \bigcup^\infty_{n=1} A_n$}
	\Explain{ So, $\{ K \in \K(X) : \exists K \cap U   \} = 
	\bigcup^\infty_{n=1} \{ K \in \K(X) : \exists K \cap A_n \}  $}
	\Explain{ Now, let $\mathfrak{B}_{n,m}$ stay for a sets of 
		$m$-tuples of rational cells wich are disjoint from $A_n$}
	\Explain{ Each $\mathfrak{B}_{n,m}$ is countable}
	\Explain{ Every compact $K\subset X$ can be given a cover of open sets disjoined from $A_n$}
	\Explain{ This cover can be choosen to consist of rational cells 
		as they form the base of topology}
	\Explain{ As $K$ is compact we can find a finite subcover}
	\Explain{ So, $K$ would be contained in $\bigcup^m_{i=1} B_i$ 
	for some $B \in\mathfrak{B}_{n,m} $}
	\Explain{Thus,
			$\{ K \in \K(X) : \exists K \cap U   \} =
			\bigcup^\infty_{n=1}  \bigcap_{m=1}^\infty 
			\bigcap_{B \in \mathfrak{B}_{n,m}} 
			\left\{ K \in \K(X)  : K \subset \bigcup^m_{i=1} B_i    \right\}^\c$
	}	
	\EndProof
	\\
	\Theorem{VietorisBorelSetsGeneration2}
	{
		\forall X : \Polish \.
		\B\Big( \K(X) \Big)  = \sigma\Big\{                     
			\big\{
				K \in \K(X)  : \exists K \cap U   
			\big\}	
			\Big|   U \in \T(X)	
		\Big\}
	}
	\Explain{ 
		dually, we express
		sets of form $\{ K \in \K(X) :  K \subset U   \}$ 
		by sets of form$\{ K \in \K(X)  : \exists K \cap V    \}$	
	}
	\Explain{Closed set $U^\c$ is $G_\delta$}
	\Explain{So, there is a sequence of open sets $(W_n)^\infty_{n=1}$ such that 
		$U^\c = \bigcap^\infty_{n=1} W_n$}
	\Explain{Assume, that compact $K \subset U$ meets infinitly many $W_n$.}
	\Explain{Then, we can choose a sequence $(x_i)^{\infty}_{i=1}$ and the increasing 
		$n:\Nat \to \Nat$, such that $x_i \in K \cap W_{n_i}$}
	\Explain{From sequence-compactness $x$ will have a partial limit in $K$.}
	\Explain{And as $X$ is normal it is also in $U^\c$}
	\Exclaim{But $K \subset U$, a contradiction}
	\Explain{So, $\{ K \in \K(X) :  K \subset U   \} = 
		\bigcup^\infty_{m=1} \bigcap^\infty_{n=1}
		\{ K \in \K(X)  : \exists K \cap W_n    \}^\c		
		 $
	}
	\EndProof
}\Page{
	\DeclareFunc{projectionOfHausdorff}
	{
		\prod X  : \Polish \.  \Closed(X)  \to \BOR\Big(\K(X) , \K(X)\Big)
	}
	\DefineNamedFunc{projectionOfHausdorff}{A,K}{\varphi_{A\cap \bullet}(K)}{A \cap K}
	\Explain{
		Let $U$ be open in $A$.	
	}
	\Explain{
		define $\U = \Big\{ V \in \T(X) \. V \cap A = U  \Big\}$,
	}
	\Explain{ By definition of subset topology $\exists \U$}
	\Explain{ Fix some $V \in \U$}
	\Explain{
		Then, 
		$
			\varphi_{A\cap \bullet}^{-1}
			\Big\{
				K \in \K(A) \.
				K \subset U		  
			\Big\}
			=  \Big\{
				K \in \K(X) \.
				K \subset V \cup A^\c		  
			\Big\} \in \T(X)
		$
	}
	\Explain{
		Note that if $K = \varphi_{A\cap \bullet}(L)$,
		then $L = K \cup N$ with $N \subset A^\c$
	}
	\Explain{ If $K \subset U$, then  $L \subset V \cup A^\c \in \U$, so $L$ was counted}
	\Explain{ If $K$ has points outside $U$, then $L$ will also have them, so it was not counted}
	\EndProof
	\\
	\Theorem{CantorBendixsonDerivativeIsBorel}
	{
		\forall  X : \Polish \. \d \in \BOR\Big( K(X), K(X) \Big)
	}
	\Explain{
		Let $U$ be open in $X$.	
	}
	\Explain{  Then, $A = U^\c$ is closed}
	\Explain{  A compact can have only a finite number of isolated points.}
	\Explain{ $A$ is closed, so it is $G_\delta$ }
	\Explain{ It means that there is a decreasing sequence of
	 open sets $V$ such that $A = \bigcap^\infty_{n=1} V_n$}
	\Explain{ Each $V_n$ can be represented as countable unions of closed sets $D_{n,i}$}
	\Explain{ I want the number of points in each $D_{n,i}$ to be bounded by some $j$}
	\Explain{ Denote by $K_j(X)$ subsets of $X$ of cardinality atmost $j$.}
	\Explain{ Write
		$
			\d^{-1} \Big\{  K \subset \K(X) : K \subset U \Big\} = 
			\bigcup^\infty_{m=1} \bigcap^\infty_{n=1} 
			\bigcup^\infty_{j=1 } \bigcap^\infty_{i=1}
			\varphi^{-1}_{D_{n,i} \cap \bullet } \K_j(D_{n,i})		
		$
	}
	\Explain{ But projection is Borel measurable, so this set is also Borel}
	\Explain{ Sets as above genetate all Borel sets for $\K(X)$.}	
	\Explain{
		So, in Vietoris topology 
		the Cantor-Bendixson derivative is Borel measurable
	}
	\EndProof
}\Page{
	\Theorem{IntersectionIsBorel}
	{
		\forall X : \Polish \.
		(\cap) \in \BOR\Big( \K^{\times 2}(X), \K(X) \Big)
	}
	\Explain{
		Let $U$ be open in $X$.	
	}
	\Explain{ $U$ is open, so it is $F_\sigma$ }
	\Explain{ It means that there is an increasing sequence of closed sets $A$ such that 
	$U = \bigcup^\infty_{n=1} A_n$}
	\Explain{A pair of compacts $(K,L)$ have nonempty $K \cap L$ iff $K \times L$ intersects diagonal.}
	\Explain{ I want this intersection happen at $U$. }
	\Explain{ So there must be some $A_n$ with such intersection.}
	\Explain{ Note that set of compacts intersecting diagonal is closed.}
	\Explain{ Then write
		$
		   (\cap)^{-1}
		   \Big\{ K \in K(X) :  \exists K \cap U \Big\} =
		   \bigcap^\infty_{n=1} (\times)^{-1}
		   \Big\{ K \in K(X^2) : \exists K \cap \Delta(A_n) \Big\} 
		$	
	}
	\Explain{As $(\times)$ is a continuous functions the resulting set is closed.}
	\Explain{ Sets as above genetate all Borel sets for $\K(X)$.}	
	\Explain{
		So, in Vietoris topology 
		the intersection is Borel measurable
	}
	\EndProof
	\\
	\Theorem{SectionIsBorel}
	{
		\forall X : \Polish \.
		\forall Y : \TYPE{CompactMetrizable} \.
		\forall F : \Closed(X \times Y) \.
		\sigma_\bullet(F) \in \BOR\Big( X, \K(Y) \Big)
	}
	\Explain{ Let $A$ be a closed set in $Y$.}
	\Explain{ As $Y$ is compact, then $A$ also is compact}
	\Explain{ Approximate $F \cap X \times A$ by open cells 
		$U_n = \Cell\left(F \cap X \times A,\frac{1}{n}\right)$}
	\Explain{ Then $F \cap X \times A = \bigcap^\infty_{n=1} U_n$ }
	\Explain{ We claim that 
		$
			\sigma_\bullet^{-1}(F)\big\{  K \in \K(Y) : \exists K \cap A     \big\}  =
			\pi_X\Big( F \cap X \times A \Big) = 
			\pi_X\left( \bigcap^\infty_{n=1} U_n \right) 	 =
			\bigcap^\infty_{n=1} \pi_X(U_n)	
		$
	}
	\Explain{ The last equality us somethat questionable}
	\Explain{ It would be true iff 
		$
			\forall x \in X \. 
			\pi^{-1}_X(x) \cap \bigcap^\infty_{n=1} U_n = \emptyset 
			\Imply
			\exists m \in \Nat \.  
			\pi^{-1}_X(x) \cap \bigcap^m_{n=1} U_n = \emptyset 
		$
	}
	\Explain{ So there is an $x$ and a sequence of $y_n$ such that $(x,y_n) \in U_n$}
	\Explain{ But $Y$ is compact, so there must exists a partial limit $y$}
	\Explain{ But this means that $(x,y) \in   F \cap X \times A$ as  $F \cap X \times A$ is closed.}
	\Explain{ This means that the fiber of $x$ is non-empty}
	\Explain{ So, using the fact that $\pi_X$ is open, we see that 
		$
		\sigma_\bullet^{-1}(F)\big\{  K \in \K(Y) : \exists K \cap A     \big\}
		$	
		is $G_\delta$, and hence Borel
	}
	\Explain{
		As sets of this type generate Borel structure for
		 Vietoris topology, the section is Borel 	measurable
		}
	\EndProof
}
\newpage
\subsubsection{Lebesgue-Hausdorff Theorem}
\Page{
	\Theorem{BairHasBP}
	{
		\forall X \in \TOP \.
		\B(X) \subset \BP(X)
	}
	\NoProof
	\\
	\Theorem{EveryBorelIsBairMeasurable}
	{
		\forall X,Y \in \TOP \.
		\forall \varphi \in \BOR(X,Y) \.
		\BM(X,Y)
	}
	\NoProof
	\\
	\DeclareType{LebesgueClass}
	{
		\prod_{X \in \SET} \prod_{Y \in \TOP} 
		??(X \to Y)
	}
	\DefineType{\C}{LebesgueClass}
	{
		\forall f  : \Nat \to \C \.
		\forall g : X \to Y \.
		(\mathrm{pt}) \; \lim_{n \to \infty} f_n = g \Imply g \in \C
	}
	\\
	\Theorem{LebesgueClassIntersection}
	{
		\NewLine ::		
		\forall X,I \in \SET \.
		\forall Y \in \TOP \.
		\forall \C : I \to \TYPE{LebesgueClass}(X,Y) \. \NewLine \.
		\TYPE{LebesgueClass}\left( X,Y, \bigcap_{i \in I} \C_i \right) 
	}
	\NoProof
	\\
	\DeclareFunc{generateLebesgueClass}
	{
		\prod_{X \in \SET}
		\prod_{Y \in \TOP}
		?(X \to Y) \to \TYPE{LebesgueClass}(X,Y)
	}
	\DefineNamedFunc{generateLebesgueClass}{\A}
	{
		\mathrm{LC}(\A)	
	}
	{
		\bigcap \Big\{ \C : \TYPE{LebesgueClass}(X,Y), \A \subset \C \Big\}
	}
	\\
	\Theorem{LebesgueApproximationTHM}
	{
		\NewLine ::		
		\forall X \in \BOR \.
		\forall f  \in \BOR(X,\Reals) \.
		\exists B : \Nat\times \Int \to \B(X) \.
		\exists \alpha : \Nat \times \Int \to \Reals \.
		(\mathrm{pt}) \;		
		f = \lim_{n \to \infty} \sum^\infty_{m=-\infty} \alpha_{n,m}\chi_{B_{n,m}}
	}
	\Say{I}{
		\Lambda n \in \Nat  \.
		\Lambda m \in \Int  \.
		\left( \frac{m}{n}, \frac{m+1}{n}\right]
	}
	{
		\Nat \times \Int \to \B(\Reals)
	}
	\Say{B}{f^{-1}(B)}{\Nat \times \Int \to \B(X)}
	\Say{\alpha}{\Lambda n \in \Nat  \.
		\Lambda m \in \Int  \.
		\frac{2m+1}{2n}}
	{
		\Nat \times \Int \to \Reals	
	}
	\Say{[1]}{\Elim B \Elim \alpha}
	{
		\forall n \in \Nat \. \forall x \in X \.
		\left| f(x) - \sum^\infty_{m=-\infty} \alpha_{n,m}\chi_{B_{n,m}}(x) \right| \le 
		\frac{1}{2n}
	}
	\Conclude{[*]}{\Intro (\mathrm{pt}) [1] }{
		(\mathrm{pt}) \;		
		f = \lim_{n \to \infty} \sum^\infty_{m=-\infty} \alpha_{n,m}\chi_{B_{n,m}}
	}
	\EndProof
}\Page{
	\Theorem{LebesgueHausdorffTHM}
	{
		\forall X : \TYPE{Metrizable} \.
		\mathrm{LC}\Big( C(X) \Big) = \B(X,\Reals)
	}
	\Say{\C}{\mathrm{LC}\Big( C(X) \Big)}{ \TYPE{LebesgueClass}(X,Y) }
	\Say{[1]}{\Elim \; \mathrm{LC}\;\Big( C(X) \Big)\Elim \B(X,\Reals)}
	{
		\C \subset \B(X,\Reals)
	}
	\AssumeIn{\alpha}{\Reals}
	\AssumeIn{f}{C(X)}
	\Say{\C'}{ \{ g : X \to \Reals \. \alpha g + f  \in \C  \}  }
	{
		?(X \to \Reals)
	}
	\Say{[2]}{\Elim \C \Elim \C' \Elim \Reals\hyph\mathsf{VS}\Big(C(X)\Big) }{C(X) \subset \C'}
	\Conclude{[\alpha.*]}{\Elim \C \Elim \C' \Elim \mathrm{LC}}
	{
		\C \subset \C'
	}
	\Derive{[2]}{\Intro \forall}
	{
		\forall \alpha \in \C \.
		\forall f \in C(X) \.
		\forall \alpha \in \Reals \.
		\forall g \in \C \.
		 f +  \alpha g \in \C
	}
	\AssumeIn{\alpha,\beta}{\Reals}
	\AssumeIn{f}{\C}
	\Say{\C'}{ \{ g : X \to \Reals \. \alpha g + \beta f  \in \C  \}  }
	{
		?(X \to \Reals)
	}
	\Say{[3]}{
		 \Elim \C' [2]	
	}
	{
		C(X) \subset  \C'
	}
	\Conclude{[\alpha.*]}{\Elim \C \Elim \C' \Elim \mathrm{LC}}
	{
		\C \subset \C'
	}
	\Derive{[3]}{\Intro \Reals\hyph \mathsf{VS}}
	{
		\C \in \Reals\hyph \mathsf{VS}
	}
	\Say{[4]}{\Elim \chi \Elim \c}
	{
		\forall B \in \BOR(X,\Reals) \.  \chi_{B^\c} = 1-\chi_{B}
	}
	\Say{[5]}{\Elim \chi \Elim \bigcup}
	{
		\forall B : \TYPE{DisjointSequence}\Big( \BOR(X,\Reals) \Big) \.
		\chi_{\bigcup^\infty_{n=1} B_n} = \sum^\infty_{n=1} \chi_{B_n} 
	}
	\Say{\A}{\Big\{ B \in \BOR(X,\Reals) \Big| \chi_\B \in \C \Big\}}{?\BOR(X,\Reals)}
	\Say{[6]}{\Elim \A [4][5]}{ \TYPE{LambdaClass}(X,\A) }
	\AssumeIn{U}{\T(X)}
	\Say{\Big(A,[7]\Big)}{\Elim F_\sigma(U)}
	{
		\sum A : \Nat \to \Closed(X) \.  A \uparrow X
	}
	\Say{\Big(f,[8]\Big)}{\THM{UrysohnLemma}(X,A,U^\c)}
	{
		\sum f : \Nat \to \TOP\Big(X,[0,1]\Big) \.  
		f^{-1}(1) = A \And
		f^{-1}(0) = U^\c 
	}
	\Say{[9]}{[7][8]}{\lim_{n\to \infty} f_n = \chi_U}
	\Say{[10]}{\Elim \C [9]}{\chi_U \in \C}
	\Conclude{[*]}{\Elim \A [10]}{U \in \A}
	\Derive{[7]}{\Intro \subset}{\T(X) \subset \A}
	\Say{[8]}{\THM{PiLambdaLemma}[6][7]}
	{
			\B(X) = \A
	}
	\Say{[9]}{\THM{LebesgueApproximationTHM}[8][3]\Elim \A}
	{
		\B(X,\Reals) \subset \C
	}
	\Conclude{[*]}{\Elim \TYPE{TypeEq}[8][9]}
	{
			\B(X) = \C
	}
	\EndProof
}
\newpage
\subsubsection{Case of Separable Metrizable Space}
\Page{
	\DeclareType{\PSA}
	{
		\prod_{X \in \SET}   ?\Alg(X)
	}
	\DefineType{\A}{\PSA}{
		\forall x, y \in X \. 
		\forall U : x \neq y \.
		\exists A,B \in \A \.
		A \cap B = \emptyset \And x \in A \And y \in B
	}
	\\	
	\Theorem{BorelIsomorphismCondition}
	{
		\NewLine ::		
		\forall X : \BOR \.
		\forall [0] : \CGSA \And \PSA(X,\S_X) \. \NewLine \.
		\exists Y \subset \C \.
		X \cong_\BOR Y
	}
	\Say{\Big(A,[1]\Big)}{\Elim \CGSA [0.1]}
	{
		\sum A : \Nat \to ?X \. \S_X = \sigma(\im A)
	}
	\Say{\varphi}{\Lambda x \in X \. \Lambda n \in \Nat \. \delta_x(B_n) }
	{
		X \to \C
	}
	\Say{[2]}{\Elim \varphi \Elim \PSA [0.2]}{\Inj(X,\C,\varphi)}
	\Say{[3]}{
		\Lambda I : \TYPE{Finite}(\Nat) \.
		\Lambda b : I \to \BOOL \.
		\Elim \varphi
		\Intro \bigcap 	
	}
	{
		\forall I : \TYPE{Finite}(\Nat) \.
		\forall b : I \to \BOOL \.		
		\NewLine :		
		\varphi^{-1}\Big\{ c \in \C : \forall i \in I \.  c_i = b_i \Big\} = 
		\{
			x \in X \.  
			\forall i \in I \. 
			i = 1 \Imply x \in A_i \And
			i = 0 \Imply x \not \in A_i 
		\} = \NewLine =
		\bigcap_{b_i = 1 } A_i
		\cap 
		\bigcap_{b_i = 0 } A_i^\c
		\in 
		\S_X
	}
	\Say{[4]}{\Intro \BOR(X)[3]}{\varphi \in \BOR(X,\C)}
	\Say{Y}{\varphi(X)}{?\C}
	\Say{[5]}{
		\Lambda n \in \Nat \. \Elim \varphi \Intro Y
	}
	{
		 \varphi(A_n) =	
		 \{
			c \in \C :
			c_n = 1		 
		 \}
		\cap Y \in \B(\C)
	}
	\DeriveConclude{[*]}{}{
		X \cong_\BOR Y	
	}
	\EndProof
	\\
	\Theorem{RealIsomorphismTHM}
	{
		\forall X  : \Polish \.
		\exists A \subset \Reals \.
		A \cong_\BOR X
	}
	\NoProof
}\Page{
	\Theorem{KuratowskiMeasurableExtensionTHM}
	{
		\NewLine ::		
		\forall X  \in \BOR \.
		\forall Y : \Polish \. 
		\forall Z \subset X \.
		\forall f \in \BOR(X,Z) \.
		\exists F \in \BOR(X,Y) \.
		F_{|Z} = f
	}
	\Say{\Big(V,[1]\Big)}
	{
		\Elim \TYPE{SecondCountable}(Y)
	}
	{
		\sum V : \Nat \to \T(Y) \.
		\TYPE{BaseOfTopology}\Big(Y, \im V\Big)
	}
	\Say{\Big(B,[2]\Big)}{ \THM{SubsetMeasurableStructure} \Big(X,Z, f^{-1}(V)\Big)}
	{
			\sum B : \Nat \to \S_X \.
			\forall n \in \Nat \.  
			f^{-1}(V_n) = B_n \cap Z
	}
	\Say{Z'}{
		\Big\{
			x \in X : \exists y \in Y : 
			\forall n \in \Nat \. 
			x \in B_n
			\iff
			y \in V_n
		\Big\}	
	}
	{
		?X
	}
	\Say{f'}{\Lambda x \in Z' \. \Elim \TYPE{Singleton} \bigcap \{ V_n   | n \in \Nat, x \in B_n    \}}
	{
		Z' \to Y
	}
	\Say{[3]}{\Elim Z'(f)}{Z \subset Z'}
	\Say{[4]}{\Elim f'}{ \forall n \in \Nat \. {f'}^{-1} (V_n) = B_n \cap Z' }
	\Say{[5]}{\Intro \BOR [4]}
	{	
		f' \in \BOR(Z', Y)
	}
	\Say{\beta}{\{ (n,x) |  x \in B_n    \}}{?(\Nat \times X)}
	\Say{[6]}{\Elim Z' \Elim \beta }
	{
		\forall x \in X \.  x \in Z' \iff  
		\exists\sigma_x(\beta)
		\And \NewLine \And
		\forall k \in \Nat \.
		\forall n,m \in  \sigma_x(\beta) \.
		\exists l \in  \sigma_x(\beta) \.
		\overline{V_l} \subset V_n \cap V_m 
		\And
		\diam(V_l) < \frac{1}{k} 
		\And \NewLine \And
		\forall n \in \Nat \.
		\forall m \in \Nat \.
		m \in \sigma_x(\beta) 
		\And
		V_m \subset V_n
		\Imply
		n \in \sigma_x(\beta)
	}
	\Say{C}{
		\left\{ n,m,k,l \in \Nat :  \overline{V_l} \subset V_n \cap V_m, 
		\diam(V_l) < \frac{1}{k} \right\}	
	}
	{
		?\Nat^4
	}
	\Say{D}{
		\left\{ m,n \in \Nat :  V_m \subset V_n \right\}	
	}
	{
		?\Nat^2
	}
	\Conclude{[7]}{\Elim Z' \Elim C \Elim D}
	{
		Z' = \bigcup_{n,m,k,l \in C}   
		\Big((B_n \cap B_m)^\c \cup B_l\Big)
		\cap
		\bigcup_{n,m \in D} 
		B_m^\c \cup B_n 
		\in \B(X)
	}
	\SayIn{y}{\Elim \TYPE{NonEmpty}(Y)}{Y}
	\Say{F}{
		\Lambda x \in X \.
		\If  x \in Z' \Then f'(z) \Else Y
	}{\BOR(X,Y)}
	\Conclude{[*]}{\Elim F \Elim f'}
	{
		F_{|Z} = f
	}
	\EndProof
	\\
	\Theorem{MeasurableLavrentievTHM}
	{
		\forall X ,Y : \Polish \
		\forall A \subset X \.
		\forall B \subset Y \.
		\forall A  \ToIso{f} B : \BOR \. \NewLine
		\exists A' \in \BOR(X) \.
		\exists B' \in \BOR(Y) \.
		\exists A' \ToIso{F} B' : \BOR \.
		F_{\A} = f
	}
	\Explain{Proof by analogy with normal Lavrentiev Theorem.}
	\EndProof
	\\
	\Theorem{MeasurableGraphTHM}{
		\forall X \in \BOR \.
		\forall Y : \Metrizable \And \Separable \.
		\forall \varphi : \BOR(X,Y) \.
		G(\varphi) \in \S_X \otimes \B(Y) 	
	}
	\Say{\Big(V,[1]\Big)}
	{
		\Elim \TYPE{SecondCountable}(Y)
	}
	{
		\sum V : \Nat \to \T(Y) \.
		\TYPE{BaseOfTopology}\Big(Y, \im V\Big)
	}
	\Say{[*]}{\Elim G(\varphi)[1]}{
		G(\varphi) =
		\bigcap_{n=1}  (X \times V_n)^{\c} \cup \varphi^{-1}(X) \times Y \in \S_X \otimes \B(Y) 
	}
	\EndProof
}
\newpage
\subsubsection{Standard and Effros Spaces}
\Page{
	\DeclareType{\SBS}{?\BOR}
	\DefineType{X}{\SBS}{\exists P : \Polish \. P \cong_\BOR X }
	\\
	\Theorem{StandardBorelProduct}
	{
		\forall N \in \sigma(\omega) \.
		\forall X : N \to \SBS \.
		\SBS\left(\prod_{n=1}^N X_n\right)
	}
	\NoProof
	\\
	\Theorem{SdandardBorelSum}
	{
		\forall N \in \sigma(\omega) \.
		\forall X : N \to \SBS \.
		\SBS\left(\coprod_{n=1}^N X_n\right)
	}
	\NoProof
	\\
	\DeclareFunc{spaceOfEffros}
	{
		\Contra(\TOP, \BOR)
	}
	\DefineNamedFunc{spaceOfEffros}{X}{\Effros(X)}
	{
		\bigg( \Closed(X), 
		\sigma\Big\{ \big\{ K : \Closed(X) | \exists(K  \cap U) \big\} \Big| U \in \T(X) \Big\}
		\bigg) 
	}
	\DefineNamedFunc{spaceOfEffros}{X,Y}{\Effros_{X,Y}(f)}
	{
		f^{-1}
	}
	\\
	\Theorem{EffrosRegularityTHM}
	{
		\forall X : \Polish \.
		\SBS\big(\Effros(X)\big)  
	}
	\Say{[1]}{\THM{SubsetTopology}(\beta X,X)\Elim \beta X}
	{
		\Inj\Big( \Effros(X), \K(\beta X), \cl_{\beta X} \Big)
	}
	\Say{\Big(V,[2]\Big)}
	{
		\Elim \TYPE{SecondCountable}(\beta X)
	}
	{
		\sum V : \Nat \to \T(\beta X) \.
		\TYPE{BaseOfTopology}\Big(\beta X, \im V\Big)
	}
	\Say{\Big(U,[3]\Big)}
	{
		\Elim \beta X
		\THM{PolishIsGdelta}(\beta X, X)
	}
	{
		\sum U : \Nat \to \T(\beta X) \.
		X = \bigcap^\infty_{n=1} U_n
	}
	\Say{G}
	{
		\Big\{ 
			\cl_{\beta X}  F \Big| F \in \Effros(X) 
		\Big\}
	}{   
		?\Closed(\beta X)
	}
	\AssumeIn{K}{G}
	\Say{[4]}{\Elim G(K)}{\Dense(K,K \cap X)}
	\Say{[5]}{[4][3]}
	{
		\forall n \in \Nat \. \Dense(K, K \cap U_n)
	}
	\Conclude{[*]}{\Elim \Bair \beta X [2][5]}
	{
		\forall n \in \Nat \. \forall m \in \Nat \. 
		\exists (K \cap V_m) \Imply \exists (K \cap V_m \cap U_n)
	}
	\Derive{[4]}{\Intro \bigcap}
	{
		G = \bigcap^\infty_{n=1} \bigcap^\infty_{m=1}
		 \Big\{ K \in \K(\beta X)  :  \exists K \cap V_m  \Big\}^\c 
		 \cup
		 \Big\{ K \in \K(\beta X) : \exists K \cap V_m \cap U_n \Big\}
	}
	\Say{[5]}{\Intro G_\delta}
	{
		G \in G_\delta\Big( \K(\beta X) \Big)
	}
	\Say{[6]}{ \THM{GDeltaIsPolish}\Big( \K(\beta X), G \Big)   }
	{
		\Polish(G)
	}
	\Conclude{[*]}{\Elim \Effros(X) \Elim G [1][6]}
	{
		\Big( \BOR,\Effros(X) ,G, \cl_{\beta X} \Big)   
	}
	\EndProof
}
\Page{
	\DeclareFunc{spaceOfFell}
	{
		\TOP
		\to
		\TOP
	}
	\DefineNamedFunc{spaceOfFell}{X}{F(X)}
	{
		\Bigg( \Closed(X), 
		\bigg\langle
			\Big\{ 
			\big\{ A : \Closed(X) | \neg\exists(K  \cap A) \And 
			 	\forall i \in \{1,\ldots,n\} \. \exists(U_i \cap A) \big\} 
			\Big| \NewLine \Big|
			K \in \K(X), n \in \Int_+, U: \{1,\ldots,n\} \to \T(X) 
		\Big\}
		\bigg\rangle_\TOP
		\Bigg)
	}
	\\
	\Theorem{FellTopologyIsCompact}
	{
		\forall X :  \Polish \And \LC \.
		\TYPE{CompactMetrizable}\big(F(X)\big)
	}
	\Say{\iota_+}{
		\Lambda  A  \in F(X) \.
		A \cup \{\infty\}
	}
	{
		F(X) \to \K(X^+)
	}
	\Say{d}{\Elim \Polish(X^+)}
	{
		\sum d : \TYPE{Metric}(X^+) \.
		\TYPE{CompactlyMetrizes}(X^+)	
	}
	\Say{\rho}
	{
		\Lambda A,B : \Closed(X)	\.	
		\max\left(
		\inf_{x \in \iota_+(A)} \sup_{y \in \iota_+(B)} d(x,y),
		\inf_{x \in \iota_+(B)} \sup_{y \in \iota_+(A)} d(x,y)
		\right)
	}
	{
		\TYPE{Metric}\Big( F(X) \Big)
	}
	\Conclude{[*]}{\THM{HausdorffCompactIsCompact}(X^+)\THM{SubspaceTopology}(X^+,X)}
	{
		\Compact\Big( F(X), \rho \Big)
	}
	\Explain{
		The topology of Fell 
		corresponds 
		to subspace topology 
		for Hausdorff metric on ine point compactification
	}
	\Explain{
		The open sets of form
		$\Big\{ A \in F(X)  \Big| \neg \exists A \cap K  \Big\}$
		correspond to open sets of form
		$\Big\{ A \in \K(X^+)  \Big|  A \subset U  \Big\}$}
	\Explain{		
		Here $U \in \U(\infty)$  
	}
	\Explain{
		By the structure of the embedding  $\iota_+$ this is enough 
	}	
	\Explain{ Assume $(A_n)^\infty_{n=1}$ is a sequence of closed set in $F(X)$}
	\Explain{ Then $\iota_+(A)$ is a sequence in $\K(X)$}
	\Explain{ It will have a partial limit $B$}
	\Explain{ Then $B\cap X \in F(X) $ and is a partial limit of $(A_n)^\infty_{n=1}$}
	\EndProof
	\\
	\Theorem{FellBorelIsEffros}
	{
		\forall X : \Polish \And \LC \.
		\bor \Big( F(X)\Big) = \Effros(X)
	}
	\Explain{ Let $K$ be a compact in $X$}
	\Explain{ 
		We need to express sets of form 
		$\Big\{ A \in F(X)  \Big| \neg \exists A \cap K  \Big\}$
		by sets of form
		$\Big\{ A \in F(X) \Big| \exists A \cap U \Big\} $ for $U$ open 
	}
	\Explain{
		$K$ is a $G_\delta$, so where are open sets $(U_n)^\infty_{n=1}$ 
		such that $K = \bigcap^\infty_{n=1} U_n$
	}
	\Explain{
		Then,
		$
			\Big\{ A \in F(X)  \Big| \neg \exists A \cap K  \Big\} =
			\bigcup^\infty_{m=1}
			\bigcap^\infty_{n=m} 
			\Big\{
				A  \in F(X) \Big| \exists A \cap U_n
			\Big\}^\c
		$
	}
	\EndProof
}\Page{
	\Theorem{KSigmaEffrosSpaceIsStandard}
	{
		\forall  X : \TYPE{\sigma\hyph Compact} \And \Separable \And \Metrizable \.
		\NewLine \.
		\SBS\Big(\Effros(X) \Big) 
	}
	\Say{[1]}{\THM{SubsetTopology}(\beta X,X)\Elim \beta X}
	{
		\Inj\Big( \Effros(X), \K(\beta X), \cl_{\beta X} \Big)
	}
	\Say{\Big(K,[2]\Big)}
	{
		\Elim \TYPE{\sigma\hyph Compact}(X)
	}
	{
		\sum K : \Nat \to \K(X) \.
		X = \bigcup^\infty_{n=1} K_n
	}
	\Say{ \Big(U,[3]\Big)}
	{
		\THM{ClosedIsFSigma}(\beta X, \cl_{\beta X}\;K)
	}
	{
		\sum U : \Nat^2 \to \T(\beta X) \.
		\forall n \in \Nat \.
		\cl_{\beta X}\;K_n = \bigcap^\infty_{m=1} U_{n,m}
	}  
	\Say{\Big(V,[31]\Big)}
	{
		\Elim \TYPE{SecondCountable}(\beta X)
	}
	{
		\sum V : \Nat \to \T(\beta X) \.
		\TYPE{BaseOfTopology}\Big(\beta X, \im V\Big)
	}
	\Say{G}
	{
		\Big\{ 
			\cl_{\beta X}  F \Big| F \in \Effros(X) 
		\Big\}
	}{   
		?\Closed(\beta X)
	}
	\AssumeIn{K}{G}
	\Say{[4]}{\Elim G(K)}{\Dense(K,K \cap X)}
	\Say{[5]}{[4][3]}
	{
		\forall n \in \Nat \. \Dense(K, K \cap U_n)
	}
	\Conclude{[*]}{\Elim \Bair \beta X [2][5]}
	{
		\forall n,k,l \in \Nat \. 
		\exists (K \cap V_m) \Imply \exists (K \cap V_m \cap U_{k,l})
	}
	\Derive{[4]}{\Intro \bigcap}
	{
		G = \bigcap^\infty_{n=1} \bigcap^\infty_{k,l=1}
		 \Big\{ K \in \K(\beta X)  :  \exists K \cap V_m  \Big\}^\c 
		 \cup
		 \Big\{ K \in \K(\beta X) : \exists K \cap V_m \cap U_{k,l} \Big\}
	}
	\Say{[5]}{\Intro G_\delta}
	{
		G \in G_\delta\Big( \K(\beta X) \Big)
	}
	\Say{[6]}{ \THM{GDeltaIsPolish}\Big( \K(\beta X), G \Big)   }
	{
		\Polish(G)
	}
	\Conclude{[*]}{\Elim \Effros(X) \Elim G [1][6]}
	{
		\Big( \BOR,\Effros(X) ,G, \cl_{\beta X} \Big)   
	}
	\EndProof
	\\
	\Theorem{RaymondsTHM}
	{
		\forall X : \Separable \And \Metrizable \.
		\SBS(\Effros(X)) 
		\iff
		\NewLine
		\iff
		\exists P : \Polish \.
		\exists S : \TYPE{\sigma\hyph Compact} \.
		X = P \cap S	
	}
	\Explain{This is proof is out of the scope of this manuscript}
	\EndProof
	\\
	\Theorem{CompactsAreBorelForFell}
	{
		\forall X : \Polish \.
		\K(X) \in \B\Big(F(X)\Big)
	}
	\Explain{The closed set is compact iff it totally bounded}
	\Explain{Denote by $\Cell(r)$ the set of open rational cells of radius less then $r$}
	\Explain{For a finite sequence of open rational cells $(U_i)^n_{i=1}$
		define $V_{n,U,k}$ to be such open sets
		that  $V_{n,U,k} \downarrow \bigcap^n_{i=1} U_i^\c  $	
	}
	\Explain{ Then, Express
		$
			\K(X) = 
			\bigcap_{\varepsilon \in \Rats_{++}}	
			\bigcup^\infty_{n=1}
			\bigcup_{U:n \to \Cell(\varepsilon)}
			\bigcup^\infty_{k=1}
			\Big\{
				A \in F(X) : \exists A \cap V_{n,U,K} 
			\Big\}^\c
			\in \B\Big( F(X) \Big)
		$
	}
	\EndProof
}\Page{
	\Theorem{CompactsAreBorelSubspaceOfEffros}
	{
		\forall X : \Polish \.
		\K(X) \subset_\BOR \Effros(X)
	}
	\Explain{ Inspect generating sets for respective measurable algebras.}
	\EndProof
	\\
	\Theorem{CompactsAreEffros}
	{
		\forall X \in \HC \.
		\K(X) = \Effros(X) 
	}
	\Explain{ Obvious}
	\EndProof
	\\
	\Theorem{SubsetRelationIsBorel}
	{
		\forall X : \Polish \.
		\Big\{  (A,B) \in \Effros^2(X) \Big|  A \subset B      \Big\}
		\in  \alg\Big(\Effros^2(X)\Big)
	}
	\Explain{ Denote by $\Cell(X)$ the set of rational cells of $X$}
	\Explain{ Express
		$\Big\{  (A,B) \in \Effros^2(X) \Big|  A \subset B   \Big\} $}
	&
		\bigcap_{U \in \Cell(X)}   
		\bigg( 
			\Big\{
				A \in \Effros(X) \Big| \exists A \cap X    
			\Big\} 	
			\times
			\Effros(X)
		\bigg)^\c
		\cup
		\Effros(X)
		\times
		\Big\{
				B \in \Effros(X) \Big| \exists B \cap U    
		\Big\} \\
	\EndProof
	\\
	\Theorem{UnionIsBorel}
	{
		\forall X : \Polish \.
		(\cup) \in \BOR\Big( \Effros^2(X), \Effros(X) \Big)
	}
	&
		(\cup)^{-1}
		\Big\{ A \in \Effros(X)  \Big| \exists A \cap U  \Big\} = 
		\Big\{ A \in \Effros(X)  \Big| \exists A \cap U  \Big\} \times \Effros(X) 
		\cup
		\Effros(X) \times \Big\{ B \in \Effros(X)  \Big| \exists B \cap U  \Big\}
	\\
	\EndProof
	\\
	\Theorem{ProductIsBorel}{
		\forall X,Y : \Polish \.
		(\times) \in \BOR\Big( \Effros(X)\times \Effros(Y), \Effros(X \times Y) \Big)
	}
	\Explain{ Let $U$ be open In $X \times Y$}
	\Explain{
		Denote by $\U$ a set of pairs $(W,V)$ 
		of open rational cells such that $W \times V \subset  U$
	}
	&
		(\times)^{-1}
		\Big\{ A \in \Effros(X)  \Big| \exists A \cap U  \Big\} = 
		\bigcup_{(U,V) \in \U}
		\Big\{ A \in \Effros(X)  \Big| \exists A \cap V  \Big\} \times 
		\Big\{ B \in \Effros(X)  \Big| \exists B \cap W  \Big\}
	\EndProof
	\\
	\Theorem{EffrosPushforward}
	{
		\forall X,Y : \Polish \.
		\forall \varphi \in \TOP(X,Y) \.
		\Big( \Lambda A \in \Effros(X) \.  \overline{f(A)} \Big) 
		\in \BOR\Big( \Effros(X), \Effros(Y)  \Big) 
	}
	\Explain{ Open set $U$ intersects $\overline{f(A)}$ iff it intersects $f(A)$}
	\Explain{ So, $U$ intersects $\overline{f(A)}$ iff open set $f^{-1}(U)$ intersects $A$}
	\EndProof
	\\
	\Theorem{ClosedDomainsAreEffrosMeasurable}
	{
		\forall X : \Polish \.
		\TYPE{ClosedDomain}(X)  \in \alg\Big( \Effros(X) \Big)
	}
	&
	\bigcap_{U \in \Cell(X) }     
	\Big\{ A \in \Effros(X) \Big| \exists A \cap U   \Big\}^\c
	\cup
	\bigcup_{V \le_\Cell U} \bigcap_{W \le_\Cell V}	
	\Big\{ A \in \Effros(X) \Big| \exists A \cap W   \Big\}
	\\ 
	\EndProof
}\Page{
	\Theorem{CatAlgebraIsBorel}
	{
		\forall X : \Polish \.
		\cat(X) \in \B(X)
	}
	\NoProof
	\\
	\Theorem{SelectionTheorem}
	{
		\forall X : \Polish \.
		\exists \delta : \Nat \to \BOR\Big(\Effros(X), X \Big) \.
		\forall A \in \Effros(X) \. \exists A \Imply 
		\Dense\Big(  A  , \delta_\Nat(A) \Big)
	}
	\Assume{[1]}{X \neq \emptyset}
	\Say{\Big(\rho,[2]\Big)}{\Elim \Polish}
	{
		\sum \rho : \TYPE{Metrizes}(X) \.
		\Complete(X,\rho)
	}
	\Say{\Big(U,[3]\Big)}{\THM{SouslinSchemaExists}(X,\rho)}
	{
		\sum U : \FS{\Nat}  \to \T(X) \. \TYPE{SouslinSchema}(X,\rho,U)
	}
	\Say{[4]}{\Elim \TYPE{SouslinSchema}(X,\rho,U)}
	{
		\NewLine : 
			\forall w \in \FS{\Nat} \.  
			U_w \neq \emptyset
		\And \NewLine \And
		U_\emptyset = X 
		\And \NewLine \And
			\forall w \in \FS{\Nat} \. 
		  	\forall n \in \Nat \.
		  	\overline{U_{wn}} \subset U_{wn}
		\And \NewLine \And
			\forall w \in \FS{\Nat} \.
			U_w = \bigcup_{n \in \Nat} U_{wn}
		\And \NewLine \And
			\forall w \in \FS{\Nat} \.
			\forall b  : \len(w) > 0 \.
			\diam U_w \le 2^{-\len(w)}
	}
	\Say{\alpha}{\Lambda s \in \B \.  
		\Elim \TYPE{Singleton}\left( 
			\bigcap^\infty_{n=1} U_{\inits{s}{n}},
			\Elim \Complete(X,\rho)[2][4.2][4.5]
		\right)
	}
	{
		\B \to X
	}
	\Say{[5]}{\Elim \alpha [4.2][4.4][4.5]}{ \Surj(\B,X,\alpha) }
	\Say{[6]}{\Elim \alpha [4.3][4.4][4.5]\Elim \B}{\alpha \in \TOP(\B,X)}
	\Assume{A}{\Closed(X)}
	\Assume{[7]}{\exists A}
	\Say{T}{\{ w \in \FS{\Nat} | \exists A \cap U_w  \}}{?\FS{\Nat}}
	\Say{[8]}{\Elim T [4.3][4.4]}{\Pruned(\Nat,T)}
	\Say{[9]}{[9] [4.2]}{\exists T}
	\Say{d(F)}{\alpha(\lb T)}{X}
	\Conclude{a_F}{\lb T}{\B}
	\Derive{\Big(d , [7] \Big)}{\Intro \to \Elim \Complete(X,\rho)[4.5]}
	{
		\sum d : \Effros(X) \to X \. 
		\forall A \in \Effros(X) \.  A \neq \emptyset \Imply d(A) \in A
	}
	\Derive{\Big(a , [8] \Big)}{\Intro \to}
	{
		\sum a : \Effros(X) \setminus \{\emptyset\} \to \B \. 
		\forall A \in \Effros(X) \.  A \neq \emptyset \Imply d(A) \in \alpha(a_A)
	}
	\Say{[9]}{\Elim \Bair \Elim \Effros(X)}{\alpha \in \BOR\Big(\Effros(X) \setminus \emptyset,\B\Big)}
	\Say{[10]}{[9][8]}{d \in \BOR\Big(\Effros(X), X\Big)}
	\Say{\Big(V,[01]\Big)}{\THM{PolishIsSecondCountable}(X)}
	{
		\sum V : \Nat \to \T(X) \. \TYPE{BaseOfTopology}(X,\im V)
	}
	\AssumeIn{m}{\Nat}
	\Assume{A}{\Closed(X \cap V_n)}
	\Assume{[07]}{\exists A}
	\Say{T}{\{ w \in \FS{\Nat} | \exists A \cap U_w  \}}{?\FS{\Nat}}
	\Say{[08]}{\Elim T [4.3][4.4]}{\Pruned(\Nat,T)}
	\Say{[09]}{[09] [4.2]}{\exists T}
	\Say{e(F)}{\alpha(\lb T)}{X \cap V_n}
	\Conclude{b_F}{\lb T}{\B}
}\Page{
	\Derive{\Big(e , [11] \Big)}{\Intro \to \Elim \Complete(X,\rho)[4.5]}
	{
		\NewLine :		
		\sum e :\prod^\infty_{m=1} : \Effros(X) \to X \. 
		\forall A \in \Effros(X \cap V_n) \.  A \neq \emptyset \Imply e_m(A) \in A
	}
	\Derive{\Big(b , [12] \Big)}{\Intro \to}
	{
		\NewLine :		
		\sum b : \prod^\infty_{m=1} \Effros(X \cap V_m) \setminus \{\emptyset\} \to \B \. 
		\forall A \in \Effros(X \cap V_m) \.  A \neq \emptyset \Imply \alpha(b_m(A)) = e_m(A)  
	}
	\Say{\delta}{
		\Lambda n \in \Nat \. 
		\Lambda A \in \Effros(X) \.
		\If \exists A \cap V_n \Then  e_n(A) \Else d(A) 
	}
	{
		\Nat \to \BOR\Big(\Effros(X), X \Big) 
	}
	\Say{[13]}{\Elim \delta \Elim e \Elim d}
	{
		\forall n \in \Nat \.
		\forall  A \in \Effros(X) \. 
		A \neq \emptyset \Imply 
		\delta_n(A) \in A  
	}
	\Conclude{[14]}{\Elim \delta \Elim \TYPE{BaseofTopology}(X,\im V)}
	{
		\Dense\Big(  A  , \delta_\Nat(A) \Big)
	}
	\EndProof
	\\
	\Theorem{EffrosMeasurabilityCriterion}
	{
		\NewLine ::		
		\forall X \in \BOR \.
		\forall Y : \Polish \.
		\forall \varphi : X \to \Effros(Y) \. 
		\varphi \in \BOR\Big( X, \Effros(Y) \Big) \iff \NewLine \iff
		\varphi^{-1}(\emptyset) \in \alg(X) 
		\And
		\exists \phi : \Nat \to \BOR(X,Y) \.
		\forall x \in X \.
		\varphi(x) \neq \emptyset
		\Imply
		\Dense\Big( \varphi(x) , \phi_\Nat(x)\Big)
	}
	\Assume{[1]}{\phi \in \BOR\Big( X, \Effros(Y) \Big) }
	\Say{\Big(\delta,[2]\Big)}{\THM{SelectionTHM}(X)}
	{
		\sum \delta : \Nat \to \BOR\Big(\Effros(X), X \Big) \.
		\forall A \in \Effros(X) \. \exists A \Imply 
		\Dense\Big(  A  , \delta_\Nat(A) \Big)
	}
	\Say{\phi}{\varphi \delta}{\Nat \to \BOR(X,Y)}
	\Conclude{[1.*]}{\Elim \phi [2]}
	{
		\forall x \in X \.
		\varphi(x) \neq \emptyset
		\Imply
		\Dense\Big( \varphi(x) , \phi_\Nat(x)\Big)
	}
	\Derive{[1]}{\Intro \Imply}
	{
		\varphi \in \BOR\Big( X, \Effros(Y) \Big)  \Imply \NewLine \Imply
		\varphi^{-1}(\emptyset) \in \alg(X) 
		\And
		\exists \phi : \Nat \to \BOR(X,Y) \.
		\forall x \in X \.
		\varphi(x) \neq \emptyset
		\Imply
		\Dense\Big( \varphi(x) , \phi_\Nat(x)\Big)
	}
	\Assume{[2]}{\varphi^{-1}\{ \emptyset \} \in \alg(X)}
	\Assume{\phi}{\Nat \to \BOR(X,Y)} 
	\Assume{[3]}{ 
		\forall x \in X \.
		\varphi(x) \neq \emptyset
		\Imply
		\Dense\Big( \varphi(x) , \phi_\Nat(x)\Big)	
	}
	\AssumeIn{U}{\T(Y)}
	\Conclude{[U.*]}{[2][3]}
	{
		\varphi^{-1} \Big\{ A \in \Effros(Y) \Big|  \exists A \cap U \Big\} = 
		\Big(\varphi^{-1}\{\emptyset\}\Big)^{\c} \cap \bigcup^\infty_{n=1} \phi^{-1}_n(U)
	}
	\DeriveConclude{[2.*]}{\THM{BorelByGenerators}}
	{	
		\varphi \in \BOR\Big(X, \Effros(Y)\Big)
	}
	\DeriveConclude{[*]}{\Intro \iff [1]}
	{
		\varphi \in \BOR\Big( X, \Effros(Y) \Big)  \iff \NewLine \iff
		\varphi^{-1}(\emptyset) \in \alg(X) 
		\And
		\exists \phi : \Nat \to \BOR(X,Y) \.
		\forall x \in X \.
		\varphi(x) \neq \emptyset
		\Imply
		\Dense\Big( \varphi(x) , \phi_\Nat(x)\Big)
	}
	\EndProof
}
\newpage
\subsection{Representations and Transformatons}
\subsubsection{Clopen Set Representation}
\Page{
	\DeclareType{PolishTopology}{\prod_{X \in \SET} ?\TYPE{Topology}(X)}
	\DefineType{\T}{PolishTopology}{\Polish(X,\T)}
	\\
	\Theorem{ClosedSubsetTopologyEnrichment}
	{
		\NewLine ::		
		\forall X : \Polish \.
		\forall F : \Closed(X) \.
		\exists \T : \TYPE{PolishTopology}(X) \.
		\sigma(\T) = \B(X)
		\And
		\Clopen(X,\T, F)
	}
	\Explain{ Repesent $X = F \sqcup F^\c$}
	\Explain{ This topology is polish and has $F$ as a clopen set}
	\Explain{ As sigma-algebras contain intersections of both closed and open sets 
		the Borel structures coincide}
	\EndProof  
	\\
	\Theorem{SupTopologyIsPolish}
	{
		\NewLine ::
		\forall X : \Polish \.
		\forall \T : \Nat \to \TYPE{PolishTopology}(X) \. \NewLine \.
		\forall [0.1] : \forall n \in \Nat \.  \T(X) \subset \T_n \.
		\Polish( X, \sup_{n \in \Nat} \T_n  )
	}
	\Say{\varphi}
	{
		\Lambda x \in (X, \sup \T_n) \. \Lambda n \in \Nat \.  (\T_n) \; x	
	}
	{
		( X, \sup_{n \in \Nat} \T_n  ) \to \prod_{n \in \Nat} (X,\T_n)
	}
	\Say{[1]}{\Elim \varphi \Elim \TYPE{T_2}(X)[0.1] \THM{ProductTopologyBase}}
	{
		\Closed\left(\sup_{n \in \Nat} (X,\T_n), \im \varphi \right)
	}
	\Say{[2]}{\THM{GDeltaIsPolish}}
	{
		\Polish\left( \im \varphi \right)
	}
	\Conclude{[*]}{\Elim \TYPE{Homeomorphis}[2]}{
		\Polish(X, \sup_{n=1} \T_n)	
	}
	\EndProof
	\\
	\Theorem{SupTopologyBorelPreservation}
	{
		\NewLine ::
		\forall X : \Polish \.
		\forall \T : \Nat \to \TYPE{PolishTopology}(X) \. \NewLine \.
		\forall [0.1] : \forall n \in \Nat \.     \T(X) \subset  \T_n\.
		\forall [0.2] :  \forall n \in \Nat \.   \sigma(\T_n) \subset \B(X) \.
		\B\left( X, \sup_{n \in \Nat} \T_n  \right) = \B
	}
	\Explain{True by topology bases and definition of $\sigma$}
	\EndProof
}\Page{
	\Theorem{ClopenRepresentationTHM}
	{
		\NewLine :		
		\forall X : \Polish \.
		\forall B \in \B(X) \.
		\exists \T : \TYPE{PolishTopology}(X) \.
		\Clopen\big( (X,\T), B \big) 
		\And
		\B(X) = \sigma(\T)	
	}
	\Say{\A}
	{
		\Big\{
			A \subset X
			:
			\exists \T : \TYPE{PolishTopology}(X) \.
			\Clopen\big( (X,\T), B\big) 
			\And
			\B(X) = \sigma(\T)
		\Big\}
	}
	{
		??X
	}
	\Say{[1]}{\THM{ClosedSubsetTopologyEnrichment}(X)\Intro \A}{\T(X) \subset A}
	\Say{[2]}{\THM{ClopenIsAlgebra}(X)\Intro \A}{\forall A \in \A \. A^\c \in \A}
	\Assume{A}{\Nat \to \A}
	\Say{\Big(\T, [3] \Big)}{\Elim \A(\T)}
	{
		\sum \Nat \to \TYPE{PolishTopology}(X) \. 
		\forall n \in \Nat \.
		\Clopen\big( (X,\T), A_n\big) 
		\And
		\B(X) = \sigma(\T)
	}
	\Say{[4]}{\THM{SupTopologyIsPolish}(X,\T)}
	{
			\TYPE{PolishTopology}\Big(X,\sup_{n \in \Nat} T_n\Big)
	}
	\Say{[5]}{\THM{SupTopologyBorelPreservation}}
	{
		\sigma\Big(\sup_{n \in \Nat} T_n\Big) = \B(X)
	}
	\Say{[6]}{\THM{ClosedIntersection}\bigg( \Big(X, \sup_{n \in \Nat} \T_n\Big)[4], A \bigg)}
	{
		\Closed\left( \Big(X, \sup_{n \in \Nat} \T_n\Big) , \bigcap^\infty_{n=1} A_n \right)
	}
	\Say{[7]}{\THM{ClosedSubsetTopologyEnrichment}[5][6]}
	{
		\NewLine :		
		\exists \T : \TYPE{PolishTopology}(X) \.
		\sigma(\T) = \B(X)
		\And
		\Clopen\left(X,\T, \bigcap^\infty_{n=1} A_n \right)
	}
	\Conclude{[A.*]}{\Intro \A [7]}{\bigcap^\infty_{n=1} A_n \in \A}
	\Derive{[4]}{\Intro \SA[2]}{\SA(X,\A)}
	\Conclude{[*]}{[1][4]\Intro \B(X)}{ \B(X) \subset \A}
	\EndProof
	\\
	\Theorem{StandardBorelSubset}
	{
		\forall X : \SBS \.
		\forall Y \in \S_X \.
		\SBS(Y,\S_X|Y)
	}
	\Say{\Big(\T,[1]\Big)}{\Elim \SBS(X)\THM{ClopenRepresentationTHM}}
	{
		\NewLine :		
		\sum \T : \TYPE{PolishTopology}(X) \.
		\B(X,\T) = \S_X
		\And
		\Clopen\Big((X,\T),Y\Big)
	}
	\Say{[2]}{\THM{GDeltaIsPolish}(X,Y)[1.2]}
	{
		\Polish(Y,\T|Y)
	}
	\Conclude{[*]}{[1.1][2]}
	{
		\SBS(Y,\S_X|Y)
	}
	\EndProof
	\\
	\Theorem{MultipleClopenRepresentation}
	{
		\NewLine ::		
		\forall X : \Polish \.
		\forall A : \Nat \to  \S_X \.
		\exists : \T : \TYPE{PolishTopology}(X) \.
		\B(X) = \sigma(\T) \And
		\forall n \in \Nat \. \TYPE{Clopen}\Big( (X,\T), A_n \Big)	
	}
	\Explain{
		Construct separate topologies $\T_n$ for every set $A_n$ respectively
		by clopen representation  theorem 
	}
	\Explain{
		Then in $\sup_{n \in \Nat} \T_n$ all these sets are clopen 
		and the Borel structures coincide
	}
	\EndProof
	\\
	\Theorem{ZeroDimRepresentation}
	{
		\NewLine ::		
		\forall X : \Polish \.
		\forall A : \Nat \to  \S_X \.
		\exists : \T : \TYPE{PolishTopology}(X) \.
		\B(X) = \sigma(\T) \And
		\dim_\TOP (X,\T) = 0 
	}
	\Explain{
		Use base of rational cells as $A_n$ in the previous theorem 
	}
	\EndProof
}
\Page{
	\Theorem{PerfectSetTheoremForPerfectSets}
	{
		\forall X : \Polish \.
		\forall A \in \B(X) \.
		|A| \le \aleph_0 \Big| 
		\exists C \subset \C \. 
		C \cong_\TOP \C
	}
	\Say{\Big( \T,[1]\Big)}{\THM{StandardBorelSubset}(X)\Elim \SBS(X)}
	{
		\NewLine :		
		\sum \T : \TYPE{PolishToplogy}(A) \.
		\T(X)|A \subset \T
	}
	\Assume{[2]}{|A| > \aleph_0}
	\Say{\Big(C,[3]\Big)}{\THM{CantorSetSubsetTHM}(A,\T)[3]}
	{
		\sum C \subset A \. (\T) \; C \cong_\TOP \C 	
	}
	\Conclude{[2.*]}{\THM{CoarserHomeo}}
	{
			(X) \; C \cong_\TOP \C
	}
	\DeriveConclude{[*]}{\Intro |}
	{
		|A| \le \aleph_0 \Big| 
		\exists C \subset \C \. 
		C \cong_\TOP \C
	}
	\EndProof
	\\
	\Theorem{StandardBorelSpaceCardinality}
	{
		\forall X : \Polish \.
		\forall [0] : |X| > \aleph_0 \.
		|X| = 2^{\aleph_0}	
	}
	\Explain{ As $X$ contains a copy of $\C$ it at leas has cardinality $2^{\aleph_0}$}
	\Explain{ On the other hand $X$ is Polish and has a countable base $\U$}
	\Explain{ So, every element $x$ can be identified by a membership $x \in U, U \in \U$ }
	\Explain{ Then $X$ can be also embedded into $\C$, so $|X| \le 2^{\aleph_0}$}
	\Explain{ Overall, $|X| = 2^{\aleph_0}$}
	\EndProof
}
\newpage
\subsubsection{Further Representations}
\Page{
	\Theorem{LusinSouslinRepresentation}
	{
		\forall X : \Polish \.
		\forall A  \in \B(X) \.
		\exists F : \Closed(\B) \.
		\exists f : \Bij \And \TOP(F,A)
	}
	\Say{\Big( \T,[1]\Big)}{\THM{StandardBorelSubset}(X)\Elim \SBS(X)}
	{
		\NewLine :		
		\sum \T : \TYPE{PolishToplogy}(A) \.
		\T(X)|A \subset \T
	}
	\Say{\Big(F,f\Big)}{\THM{BaireSpaceUniversalProprty}(A,\T)}
	{
		\sum F : \Closed(\B) \.
		f : \Bij \And \TOP\Big((F,\T),A\Big)
	}
	\Conclude{[*]}{[1][2]}
	{
		\Bij \And \TOP\Big(F,A,f\Big)
	}
	\EndProof
	\\
	\Theorem{LusinSouslinExtension}
	{
		\forall X : \Polish \.
		\forall A  \in \B(X) \.
		\exists F : \Closed(\B) \.
		\exists f : \Surj \And \TOP(F,A)
	}
	\NoProof
	\\
	\Theorem{LusinBorelSchemaExists}
	{
		\forall X : \Polish \.
		\forall B \in \B(X) \.
		\exists A : \FS{\Nat}  \to \B(X) \. \NewLine \.
		A_\emptyset = B \And \NewLine \And 
		\forall w \in \FS{\Nat} \.
		A_w = \bigcup_{n \in \Nat} A_{w n}  \And \NewLine \And
		\forall b \in \B \.   
		\Big(\forall n \in \Nat \. A_{\inits{b}{n}} \neq \emptyset \Big)
		\Imply
		\exists L \in X \.
		\{ L \} = \bigcap^\infty_{n=1} A_{\inits{b}{n}} 
		\And
		\forall   x \in  \prod^\infty_{n=1} A_{\inits{b}{n}} \.
		\lim_{n \to \infty} x_n = L
	}
	\Explain{ Extend Topology for $B$ by clopen representation}
	\Explain{ Then construct Lusin schema for $B$ in this topology}
	\EndProof
	\\
	\Theorem{BaireBorelEncoding}
	{
		\forall X : \Polish \.
		\forall A \subset \B(X) \.
		\exists F : \Closed(X \times \B) \.
		x \in A
		\iff
		\exists! b \in \B \. (x,b) \in F
	}
	\Explain{ Construct Lusin schema for $A$ in $X$}
	\Explain{ Then there is a unique Baire encoding for each $x \in A$}
	\Explain{ The last convergence property shows that $F$ is closed}
	\EndProof
	\\
	\Theorem{CantorBorelEncoding}
	{
		\forall X : \Polish \.
		\forall A \subset \B(X) \.
		\exists G : G_\delta(X \times \C) \.
		x \in A
		\iff
		\exists! c \in \C \. (x,c) \in G
	}
	\Explain{ Construct encodding as in the previous problem}
	\Explain{ Then translate encodding to binary}
	\Explain{ $G$ can't be taken closed in general, as $\B$ is not compact}
	\EndProof
}
\Page{
	\Theorem{BorelMeasurableMapTopologization}
	{
		\forall X : \Polish \.
		\forall Y  : \TYPE{SecondCountableSpace} \.
		\forall \varphi \in \BOR(X,Y) \. \NewLine \.
		\exists \T : \TYPE{PolishTopology}(X) \.
		\forall [0] : \T(X) \subset \T
		\And
		\B(\T) = \B\Big( \T(X) \Big)
		\And	
		\varphi \in \TOP\Big( (X,\T), Y\Big)
	}
	\Explain{ Let $\U$ be a countable base for $Y$}
	\Explain{ Then enrich the topology to make $\varphi^{-1}(\U)$ clopen}
	\Explain{ Then $\varphi$ will be continuous}
	\EndProof
	\\
	\Theorem{BorelMeasurableIsomotphismTopologization}
	{
		\forall X : \Polish \.
		\forall Y  : \TYPE{SecondCountableSpace} \. \NewLine \.
		\forall \varphi \in \TYPE{Isomorphism}(\BOR,X,Y) \. 
		\exists \T : \TYPE{PolishTopology}(X) \.
		\forall [0] : \T(X) \subset \T  \NewLine \And
		\And
		\B(\T) = \B\Big( \T(X) \Big)
		\And	
		\varphi \in \TYPE{Isomorphism}\Big(\TOP, (X,\T), Y\Big)
	}
	\Explain{ First enrich topology of $Y$, so $\varphi$ is open}
	\Explain{ Sencondly, enrich $X$ so it is continuous}
	\Explain{ As $\varphi$ is bijection it will be an homeomorphism} 	
	\EndProof
	\\
	\Theorem{BorelMeasurableSequenceTopologization}
	{
		\forall X : \Polish \.
		\forall Y  : \TYPE{SecondCountableSpace} \. \NewLine \.
		\forall \varphi : \Nat \to \BOR(X,Y) \. 
		\exists \T : \TYPE{PolishTopology}(X) \.
		\forall [0] : \T(X) \subset \T  \NewLine \And
		\And
		\B(\T) = \B\Big( \T(X) \Big)
		\And	
		\forall n \in \Nat \.
		\varphi \in \TOP\Big( (X,\T), Y\Big)
	}
	\NoProof
}
\newpage
\subsubsection{Analytic Sets}
\Page{
	\DeclareType{Analytic}
	{
		\prod X : \Polish \. ??X
	}
	\DefineNamedType{A}{Analytic}{A \in \Sigma^1_1(X)}
	{
		\exists Y : \Polish \.
		\exists \varphi \in \TOP(X,Y) \.
		\varphi(Y) = A 
	}
	\\
	\DeclareType{BaireUniversalClass}{
		\prod T : \prod X : \Polish \.  ??X \. 
		\prod X : \Polish \. ??(\B \times X) 
	}
	\DefineType{A}{BaireUniversalClass}
	{
		T(\B \times X,A) \And T(X) = 
		\Big\{ \sigma_b(A) \Big| b \in \B  \Big\}
	}
	\\
	\Theorem{SouslinsCorrection}
	{
		\forall X : \Polish \.
		\forall [0] : |X| > \aleph_0 \.
		\B(X) \subsetneq \Sigma^1_1(X)
	}
	\Say{A}{\THM{LusinSchemaExists}(\B)}{
		\sum A : \FS{\Nat}  \to \T(\B) \. \NewLine \.
		A_\emptyset = \B \And \NewLine \And 
		\forall w \in \FS{\Nat} \.
		A_w = \bigcup_{n \in \Nat} A_{w n}  \And \NewLine \And
		\forall b \in \B \.   
		\Big(\forall n \in \Nat \. A_{\inits{b}{n}} \neq \emptyset \Big)
		\Imply
		\exists L \in \B \.
		\{ L \} = \bigcap^\infty_{n=1} A_{\inits{b}{n}} 
		\And
		\forall   x \in  \prod^\infty_{n=1} A_{\inits{b}{n}} \.
		\lim_{n \to \infty} x_n = L
	}
	\Say{w}{\FUNC{enumerate}(\FS{\Nat})}{\Surj(\Nat,\FS{\Nat})}
	\Say{\mathcal{N}}
	{
		\left\{ (b,x) \in \B \times \B : x \in \bigcup \{ A_{w_i} | i \in \Nat : b_i =0 \} \right\}
	}
	{
		?(\B \times \B)
	}
	\AssumeIn{(b,x)}{\mathcal{N}}
	\Say{[1]}{\Elim \mathcal{N}(b,x)}{
	 	x \in  \bigcup \{ A_{w_i} | i \in \Nat : b_i =0 \}
	 }
	 \Say{\Big(i,[2]\Big)}{
		 \Elim \FUNC{union} [1]
	 }{
	 	\sum_{i = 1}^\infty 	x \in A_{w_i} \And b_i = 0
	 }
	 \Say{U}{\prod_{j=1}^{i-1} \Nat \times \{0\} \times \prod_{j=i+1}^\infty \Nat }
	 {
	 	\T(\B)
	 }
	 \Say{V}{U \times A_{w_i}}{ \T(\B \times X) }
	 \Conclude{[*.3]}{\Elim V \Elim \mathcal{N}[2]}{(b,x) \in V \subset \mathcal{N} }
	 \Derive{[1]}{\THM{OpenByCover}(\B \times \B)}
	 {
	 	\mathcal{N} \in \T(\B \times \B)
	 }
	 \Say{[2]}{\Elim \mathcal{N}}
	 {
	 	\TYPE{BaireUniversalClass}(\T,\B,\mathcal{N})
	 }
	 \Say{[3]}{\THM{BaireSquerHomeomorphism}[2]}
	 {
	 	\exists   \TYPE{BaireUniversalClass}(\T,\B^2 )
	 }
	 \Say{\Big(\F,[4]\Big)}{[3]^\c}
	 {
	 	\exists   \TYPE{BaireUniversalClass}(\Pi^0_1,\B^2 )
	 }
	 \Say{\A}{\{ (x,y) \in \B^2 : \exists z \in \B \. (x,y,z) \in \F   \}}{?\B^2}
	 \Say{[5]}{\Intro \Sigma^1_1 \Elim \TOP(\B^2,\B,\pi) }
	 {
	 	 \A \in \Sigma^1_1,\B^2 )
	 }
	 \Say{[6]}{\Intro \Sigma^1_1 \Elim \TOP(\B^2,\B,\pi) }
	 {
	 	 \forall b \in \B \. \sigma_{1,b}(\A) \in \Sigma^1_1(\B)
	 }
	 \AssumeIn{A}{\Sigma^1_1(\B)}
	 \Say{\Big(F,\varphi,[7]\Big)}{\Elim \S^1_1(\B) \THM{BaireSpaceEmbedding}}
	 {
	 	\sum F : \Closed(\varphi) \.
	 	\sum \varphi : \TOP \And \Surj(F,A)
	 }
	 \Say{G}{\FUNC{swap} \; G(\varphi)}{ ? \B^2 }
	 \Say{[8]}{\THM{ClosedGraphTHM}(\B,\B,G)}
	 {
	 	\Closed(\B\times \B, G)
	 }
	 \Say{\Big( x, [9] \Big)}
	 {
		\Elim_2 \TYPE{BaireUniversalClass}(\Pi^0_1,\B^2,\F,G) 	
	 }
	{
		\sum x \in \B \. \sigma_{1,x}(\F) = G
	}
	\Conclude{[A.*]}{\Elim \A \Elim G [9]}
	{
		A = \sigma_{1,x}(A)
	}
	\Derive{[7]}{\Intro \TYPE{BaireUniversalClass}}
	{
		\TYPE{BaireUniversalClass}\Big(\Sigma^1_1, \B, \A \Big)
	}
}\Page{
	\Assume{[8]}{\A \in \B(\B^2)}
	\Say{[9]}{\Elim \Alg [8]}{\A^\c \in \B(\B)}
	\Say{A}{\{ x \in \B : (x,x) \not \in \A   \}}{??X}
	\Say{[10]}{\Elim \A [9]}{ A \in \B(\B)}
	\Say{[11]}{\Elim \Sigma_{1}^{1}[10]}{A \in \Sigma^1_1(\B)}
	\Say{\Big(x,[12]\Big)}{\Elim \TYPE{BaireUniversalClass}\Big(\Sigma^1_1, \B, \A, x \Big)}
	{
		\sum x \in \B \. A= \sigma_{1,x}(\A)	
	}
	\Say{[13]}{\Elim \A \Elim A [12]}
	{
		(x,x) \in A \iff (x,x) \not \in A
	}
	\Conclude{[*]}{\LOGIC{LEM}[13]}{\bot}
	\Derive{[8]}{\Elim \bot}{\A \not \in \B(\B^2)}
	\Conclude{[*]}{\THM{BaireUniversalProperty}[8]}
	{
		\forall X : \Polish \.
		\forall [0] : |X| > \aleph_0 \.
		\B(X) \subsetneq \Sigma^1_1(X)
	}
	\EndProof
	\\
	\Theorem{AnalyticUnion}
	{
		\forall X : \Polish \.
		\forall A : \Nat \to \sigma^1_1(X) \.
		\bigcup^\infty_{n=1} A_n \in \sigma^1_1(X)
	}
	\Explain{
		There are Polish spaces $(Y_n)^\infty_{n=1}$	 and continuous maps 
		$\phi_n : Y_n \to X$ such that $A_n = \phi_n(Y_n)$
	}
	\Explain{
		To get union as an image just use disjoint union $\bigsqcup_{n=1} Y_n$
	}
	\EndProof
	\\
	\Theorem{AnalyticIntersection}
	{
		\forall X : \Polish \.
		\forall A : \Nat \to \sigma^1_1(X) \.
		\bigcap^\infty_{n=1} A_n \in \sigma^1_1(X)
	}
	\Explain{
		There are Polish spaces $(Y_n)^\infty_{n=1}$	 and continuous maps 
		$\phi_n : Y_n \to X$ such that $A_n = \phi_n(Y_n)$
	}
	\Explain{
		Construct a pushout 
		$Z = \left\{ y \in \prod^\infty_{n=1} Y_n \Bigg| \forall n,m \in \Nat \. 
		\phi_n(y_n) = \phi_m(y_m)    
		\right\}$
	}
	\Explain{ Then the limit of $\phi$ will have intersection as its image}
	\EndProof
	\\
	\Theorem{AnalyticImage}
	{
		\forall X,Y : \Polish \.
		\forall \varphi \in \BOR(X,Y) \.
		\forall A \in \Sigma^1_1(X) \.
		f(A) \in \Sigma^1_1(Y)
	}
	\NoProof
	\\
	\Theorem{AnalyticPreimage}
	{
		\forall X,Y : \Polish \.
		\forall \varphi \in \BOR(X,Y) \.
		\forall A \in \Sigma^1_1(Y) \.
		f^{-1}(A) \in \Sigma^1_1(X)
	}
	\NoProof
	\\
	\DeclareType{BorelAnalyticSet}
	{
		\prod X : \SBS \. ??X
	}
	\DefineNamedType{A}{BorelAnalyticSet}{A \in \Sigma^1_1(X)}
	{
			\exists Y : \Polish \.
			\exists X \ToIso{\varphi} Y : \BOR \.
			\varphi(A) \in \Sigma^1_1(Y)
	}
}
\newpage
\subsubsection{Lusin Separation Theorem}
\Page{
	\DeclareType{BorelSeparated}
	{
		\prod X \in \BOR \.  ?\TYPE{DisjointPair}(X)
	}
	\DefineType{(A,B)}{BorelSeparated}
	{
		\exists S \in \S_X \.   A \subset S \And S \cap B = \emptyset
	}
	\\
	\Theorem{BorelSeparetedUnion}
	{
		\NewLine ::		
		\forall X \in \BOR \.
		\forall P,Q : \Nat \to ?X \.
		\forall [0] : \forall n,m \in \Nat \. \TYPE{BorelSeparated}(X,P_n,Q_m) 
		\. \NewLine \.
		\TYPE{BorelSepareted}\left( X, \bigcup_{n=1}^\infty P_n, \bigcup_{n=1}^\infty Q_n  \right) 	
	}
	\Explain{Let $B_{n,m}$ be separating sets for a pair $P_n,Q_m$ }
	\Explain{Then $A = \bigcup_{n=1}^\infty \bigcap_{m=1}^\infty B_{n,m}$ is Borel}
	\\
	\Theorem{LusinSeparationTheorem}
	{
		\forall X : \SBS \.
		\forall  A,B \in \Sigma^1_1(X) \. \NewLine \.
		\TYPE{DisjointPair}(X,A,B)
		\Imply
		\TYPE{BorelSepareted}(X,A,B)
	}
	\Say{\Big(\varphi,[1]\Big)}{\Elim \Sigma^1_1(X,A)}
	{
		\sum \varphi : \TOP(\B,X) \. \varphi(\B) = A
	}
	\Say{\Big(\psi,[2]\Big)}{\Elim \Sigma^1_1(X,B)}
	{
		\sum \psi : \TOP(\B,X) \. \psi(\B) = B
	}
	\Say{\Big(N,[3]\Big)}{\THM{LusinSchemaExists}(\B)}{
		\sum N : \FS{\Nat}  \to \T(\B) \. \NewLine \.
		N_\emptyset = \B \And \NewLine \And 
		\forall w \in \FS{\Nat} \.
		N_w = \bigcup_{n \in \Nat} N_{w n}  \And \NewLine \And
		\forall b \in \B \.   
		\Big(\forall n \in \Nat \. N_{\inits{b}{n}} \neq \emptyset \Big)
		\Imply
		\exists L \in \B \.
		\{ L \} = \bigcap^\infty_{n=1} N_{\inits{b}{n}} 
		\And
		\forall   x \in  \prod^\infty_{n=1} N_{\inits{b}{n}} \.
		\lim_{n \to \infty} x_n = L
	}
	\Say{a}{\Lambda w \in \FS{\Nat} \. \varphi(N_w)}{\FS{\Nat} \to ?X}
	\Say{b}{\Lambda w \in \FS{\Nat} \. \psi(N_w)}{\FS{\Nat} \to ?X}
	\Say{[4]}{\Elim a [3]}
	{
		a_\emptyset = A 
		\And
		\forall w \in \FS{\Nat} \.
		a_w = \bigcup^\infty_{n=1} a_{wn}	
	}
	\Say{[5]}{\Elim b [3]}
	{
		b_\emptyset = B 
		\And
		\forall w \in \FS{\Nat} \.
		b_w = \bigcup^\infty_{n=1} b_{wn}	
	}
	\Assume{[6]}{\neg \TYPE{BorelSeparated}(X,A,B)}
	\Say{\Big(x,y,[7]\Big)}{
		\THM{BorelSeparatedUnion}[4][5][6]
	}
	{
		\sum x,y \in \B \. 
		\forall n \in \Nat \.
		\neg \TYPE{BorelSeparated}(X,a_{\inits{x}{n}},b_{\inits{y}{n}})
	}
	\Say{[10]}{\Elim \TYPE{DisjointPair}(X,A,B)(x,y)}
	{
		\varphi(x) \neq \psi(y)
	}
	\Say{\Big(U,V,[11]\Big)}
	{
		\Elim \TYPE{T2}(X)[10]
	}
	{
		\sum U,V \in \T(X) \.
		\varphi(x) \in U \And \psi(y) \in V \And
		U \cap V = \emptyset
	}
	\Say{\Big(n,[12]\Big)}{\Elim \TOP(\B,X,\varphi \And \psi) [3]}
	{
		\sum n \in \Nat \.  a_{\inits{x}{n}} \subset U \And b_{\inits{y}{n}} \subset V
	}
	\Conclude{[6.*]}{[12][7](n)}{\bot}
	\DeriveConclude{[*]}{\Elim \bot}{\TYPE{BorelSeparated}(X,A,B)}
	\EndProof
}
\Page{
	\Theorem{LusinSequenceSeparationTheorem}
	{
		\NewLine :		
		\forall X : \SBS \.
		\forall  A : \TYPE{PairwiseDisjoint}\Big(\Sigma^1_1(X)\Big) \.
		\NewLine \. 
		\exists	B : \TYPE{PairwiseDisjoint}\Big(\B(X)\Big) \.
		\forall n \in \Nat \. A_n \subset B_n
	}
	\Explain{ 
		Iterate normal Lusin Separation Theorem, 
		using the fact that union of analytic sets is analytic
		}
	\EndProof
}
\newpage
\subsubsection{Souslin's Theorem}
\Page{
	\DeclareType{CoanalyticSet}{\prod X : \Polish \. ??X}
	\DefineNamedType{A}{CoanalyticSet}
	{
			A \in \Pi^1_1(X)
	}
	{
		A^\c \in \Sigma^1_1(X)
	}
	\\
	\DeclareType{BorelCoanalyticSet}{\prod X : \SBS \. ??X}
	\DefineNamedType{A}{BorelCoanalyticSet}
	{
			A \in \Pi^1_1(X)
	}
	{
		A^\c \in \Sigma^1_1(X)
	}
	\\
	\Conclude{\TYPE{BiAnalyticSet}}
	{
		\Lambda X : \Polish \.
		\Delta^1_1(X)	
		=		
		\Lambda X : \Polish \.
		\Sigma^1_1(X) \cap \Pi^1_1(X)	
	}
	{
		\Polish \to \Type
	}
	\\
	\Conclude{\TYPE{BorelBiAnalyticSet}}
	{
		\Lambda X : \SBS \.
		\Delta^1_1(X)	
		= \NewLine =		
		\Lambda X : \SBS \.
		\Sigma^1_1(X) \cap \Pi^1_1(X)	
	}
	{
		\Polish \to \Type
	}
	\\
	\Theorem{SouslinThm}
	{
		\forall X : \SBS \.
		\B(X) = \Delta^1_1(X)
	}
	\Explain{ Let $A$ ba a bi-analytic set in $X$ }
	\Explain{ Then by Souslin separation theorem there are borel set $B$ 
	which separates $A$ and $A^\c$}
	\Explain{ But, as it were complements, $A = B$ }
	\EndProof
	\\
	\Theorem{AnalyticGraphTHM}
	{
		\NewLine :		
		\forall X,Y : \SBS \.
		\forall \phi : X \to Y \.
		G(\phi) \in \Sigma^1_1(X \times Y) 
		\iff
		\phi \in \BOR(X,Y)
	}
	\Explain{ Proof by projections }
	\EndProof
	\\
	\Theorem{BorelIsomorphismTrivialityForStadardSpaces}
	{
		\NewLine ::		
		\forall X,Y : \SBS \.
		\forall \varphi : \BOR \And \Bij(X,Y) \.
		\TYPE{Isomorphism}(\BOR,X,Y,\varphi)
	}
	\Explain{The swapped graph is still analytic}
	\EndProof
	\\
	\Theorem{PerfectSetTheoremForAnalyticSets}
	{
		\NewLine ::		
		\forall X : \SBS \.
		\forall A \in \Sigma^1_1(X) \.
		|A| > \aleph_0 \Imply |A| = 2^{\aleph_0}
	}
	\Explain{ Assume Polish topology on $X$}
	\Explain{ There is a Polish space $Z$ and
		and a continuous map $\phi$ such that
		$\phi(Z) = A$}
	\Explain{
		Assuming $A$ is uncountable,
		construct a cantor schema on $Z$,
		so
		}
	\NoProof
}
\newpage
\subsubsection{Injective Images}
\Page{
	\Theorem{InjectiveImageTheorem}
	{
		\NewLine ::
		\forall X,Y : \Polish \.
		\forall f \in \TOP(X,Y) \.
		\forall B \in \B(X) \.
		\forall [0] : \Inj(B,Y,f_{|X}) \.
		f(B) \in \B(Y)
	}
	\Say{[1]}{\THM{LusinSouslinRepresentation}(X,B)}
	{
		X = \B
		\And
		\Closed(X,B)
	}
	\Say{\Big(N,[2]\Big)}{\THM{LusinSchemaExists}(\B)}{
		\sum N : \FS{\Nat}  \to \T(\B) \. \NewLine \.
		N_\emptyset = \B \And \NewLine \And 
		\forall w \in \FS{\Nat} \.
		N_w = \bigcup_{n \in \Nat} N_{w n}  \And \NewLine \And
		\forall b \in \B \.   
		\Big(\forall n \in \Nat \. N_{\inits{b}{n}} \neq \emptyset \Big)
		\Imply
		\exists L \in \B \.
		\{ L \} = \bigcap^\infty_{n=1} N_{\inits{b}{n}} 
		\And
		\forall   x \in  \prod^\infty_{n=1} N_{\inits{b}{n}} \.
		\lim_{n \to \infty} x_n = L
	}
	\Say{A}{
		\Lambda  w \in \FS{\Nat} \.
		f(N_w \cap B) 
	}
	{
		\FS{\Nat} \to \Sigma^1_1(Y)
	}
	\Say{[3]}{\Elim A [2.1]}{A_\emptyset =  f(B)}
	\Say{[4]}{\Elim A [2.2]}
	{
		\forall w \in \FS{\Nat} \.
		\forall n \in \Nat \.
		A_w = \bigcup^\infty_{n=1} A_{wn}
	}
	\Say{\Big(B',[6]\Big)}{\THM{LusinSequenceSeparatiomTHM}}
	{
		\sum B' : \TYPE{BorelLusinSchema}(Y,Y) \.
		\forall w \in \FS{\Nat} \. A_w \subset B'_w 
	}
	\Say{B^*}
	{
		\FUNC{lengthRec1}
		(
			Y,
			\Lambda n \in \Nat \.  B'_n \cap \overline{A_n},
			\Lambda w \in \FS{\Nat} \.
			\Lambda \beta \in \B(Y) \.
			B'_w \cap\beta \cap \overline{A_w}		
		)	
	}
	{
		\FS{\Nat} \to \B(Y)
	}
	\Say{[7]}{\Elim B^* \Intro \TYPE{BorelLusinSchema}}
	{
		\TYPE{BorelLusinSchema}(Y,Y,B^*)
	}
	\Say{[8]}{
		\THM{LengthInduction}
		\Big(
			[3]\Elim \B', [6][4]\Elim \B'
		\Big)
	}
	{
		\forall w \in \FS{\Nat} \. A_w \subset B^*_w \subset \overline{A_w}
	}
	\Say{C}{\bigcap^\infty_{n=1} \bigcup_{w \in \Nat^n} B^*_w}{\B(Y)}
	\AssumeIn{y}{f(B)}
	\Say{\Big( b, [9]\Big)}
	{
		\Elim \TYPE{Image}(X,Y,f,B,y)
	}
	{
		\sum b \in \B \. f(b) = y
	}
	\Say{[10]}{\Elim A [9]}
	{
		y \in \bigcap^\infty_{n=1} A_{\inits{b}{n}}
	}
	\Say{[11]}{[10][9]}
	{
		y \in \bigcap^\infty_{n=1} B_{\inits{b}{n}}^*
	}
	\Conclude{[y.*]}{\Elim C [11]}{y \in C}
	\Derive{[9]}{\Intro \subset }{ f(B) \subset C}
	\AssumeIn{y}{C}
	\Say{\Big(b,[10]\Big) }{\Elim C (y)[2.2][2.3] }
	{
		\sum b \in \B \.  y \in \bigcap^\infty_{n=1} B^*_{\inits{b}{n}}
	}
	\Say{ [11] }{\Elim C (y)[2.2] }
	{
		\sum b \in \B \.  y \in \bigcap^\infty_{n=1} \overline{A_{\inits{b}{n}}}
	}
	\Say{ [12] }{[11] \Elim \FUNC{closure} \Intro \exists }
	{
		\forall n \in \Nat \. \exists A_{\inits{b}{n}}
	}
	\Say{[13]}{[12]\Elim A_{\inits{b}{n}}}{
		\forall n \in \Nat \.  \exists B \cap N_{\inits{b}{n}}
	}
	\Say{[14]}{\Elim \Closed(\B,A)[13]}{b \in B}
	\Say{[15]}{\Intro A [14]}{f(b) \in \bigcap^\infty_{n=1} A_{\inits{b}{n}}}
	\Conclude{[16]}{[11][15][2.3][0]\Elim \TOP(X,Y,f)}{f(b) = y}
}\Page{
	\Derive{[10]}{\Intro \TYPE{SetEq}[9]}{f(B) = C}
	\Conclude{[*]}{[11]\Elim C}{f(B) \in \B(X)}
	\EndProof
	\\
	\Theorem{BorelInjectiveImageTheorem}
	{
		\NewLine ::
		\forall X,Y : \SBS \.
		\forall f \in \BOR(X,Y) \.
		\forall B \in \S_X \.
		\forall [0] : \Inj(B,Y,f_{|B}) \. \NewLine
		f(B) \in \B(Y) \And B \ToIso{f_{|B}} f(B) : \BOR
	}
	\NoProof
	\\
	\Theorem{BorelSetsInjectiveChar}
	{
		\NewLine :		
		\forall X : \Polish \.
		\B(X) = \Big\{ f(A) | A : \Closed(\B), f \in \TOP \And \Inj(A,X)  \Big\}
	}
	\Explain{ By previous theorem all such sets are Borel}	
	\Explain{ On the other hand if $B \in \B(X)$ ther exist an enriched topology on $X$ with $B$ closed}
	\Explain{ In this topology $B$ is itself Polish}
	\Explain{ By Baire space universal property an embedding of $B$ as a closed set into $\B$}
	\Explain{ By taking inverse of this embedding abd combining 
		it with continuous $\id$ we get an injective image}
	\EndProof
	\\
	\Theorem{BorelEquivalence}
	{
		\NewLine :		
		\forall X \in \SET \.
		\forall \T,\T' : \TYPE{PolishTopology}(X) \.
		\T \subset \B(X,\T')
		\Imply
		\B(X,\T) = \B(X,\T')
	}
	\Explain{
		As Borel sets are a minimal sigma-algebra,
		$\B(X,\T) \subset \B(X,\T') $}
	\Explain{ So assume without loss of generality that $\T \subset \T'$ }
	\Explain{ Then $\id$ is continuous as a mapping from $(X,\T')$ to $(X,\T)$}
	\Explain{ If $B$ is Borel in $(X,\T')$ then $\T'$ can be furtherly enriched to make $B$ closed}
	\Explain{ But by injective image theorem this means that $B$ is Borel in $(X,\T)$ }
	\EndProof
	\\
	\Theorem{MeasurableStructureEquivalence}
	{
		\NewLine :		
		\forall X : \SBS \.
		\forall \mathcal{E} \subset \S_X \.
		|\mathcal{E}| \le \aleph_0 \And \TYPE{SeparatesPoints}(X)
		\Imply
		\sigma(\mathcal{E}) = \S_X
	}
	\Explain{ Assume that $X$ is Polish}
	\Explain{ Generate another topology from $\mathcal{E}$}
	\Explain{ Then process as in the previous theorem}
	\EndProof
}
\newpage
\subsubsection{Isomorphism Theorem}
\Page{
	\Theorem{BorelSchroderBernsteinTheorem}
	{
		\NewLine 		
		\forall X,Y : \SBS \.
		\forall f : \BOR \And \Inj(X,Y) \.
		\forall g : \BOR \And \Inj(Y,X) \. \NewLine \.
		\exists A \in \BOR(X) \.
		\exists B \in \BOR(Y) \.
		f(A) =  Y \setminus B  \And
		g(B) = X \setminus A
	}
	\Say{A}{\FUNC{rec}\Big(X, \Lambda A \subset X \. fg(A) \Big)}{\Nat \to \B(X)}
	\Say{B}{\FUNC{rec}\Big(Y, \Lambda B \subset Y \. gf(B) \Big)}{\Nat \to \B(Y)}
	\Say{A'}{\bigcap^\infty_{n=1} A_n}{\B(X)}
	\Say{B'}{\bigcap^\infty_{n=1} B_n}{\B(Y)}
	\Say{[1]}{\Elim A' \Elim B'}{f(A') = B'}
	\Say{[2]}{\Elim A \Elim B \Elim \Inj(X,Y,f)}
	{
		\forall n \in \Nat \.
		f\Big( A_n \setminus g(B_n) \Big) =
		f(A_n) \setminus B_{n+1}
	}
	\Say{[3]}{\Elim B \Elim A \Elim \Inj(Y,X,g)}
	{
		\forall n \in \Nat \.
		g\Big( B_n \setminus f(A_n) \Big) =
		g(B_n) \setminus A_{n+1}
	}
	\Say{Q}{A' \cup \bigcup_{n=1}A_n \setminus g(B_n)}{\B(X)}
	\Say{E}{\bigcup^\infty_{n=1} B_n \setminus f(A_{n})}{\B(X)}
	\Say{[*.1]}{
		\Elim Q  [1][3] \Elim B' \Intro E	
	}
	{
		\NewLine :		
		f(Q) =
		f\left( A' \cup \bigcup_{n=1}(A_n \setminus g(B_n))\right) =
		B' \cup  \bigcup_{n=1} f\Big(A_n \setminus g(B_n)\Big) =
		B' \cup \bigcup_{n=1} f(A_{n}) \setminus B_{n+1} = \NewLine =
		Y \setminus E
	}
	\Conclude{[*.2]}{
		\Elim E  [2][4]  \Intro Q	
	}
	{
		\NewLine :		
		g(E) =
		g\left( \bigcup^\infty_{n=1} B_n \setminus f(A_{n})  \right) =
		\bigcup_{n=1} g\Big(B_n \setminus f(A_n)\Big) =
		\bigcup_{n=1} g(B_{n}) \setminus A_{n+1} = 
		X \setminus Q
	}		
	\EndProof
	\\
	\Theorem{BorelSchroderBernsteinIsomorphism}
	{
		\NewLine 		
		\forall X,Y : \SBS \.
		\forall f : \BOR \And \Inj(X,Y) \.
		\forall g : \BOR \And \Inj(Y,X) \. \NewLine \.
		X \cong_\BOR Y
	}
	\Explain{ Let $Q$ and $E$ be as in previous theorem}
	\Explain{ Then, represent $X = g(E) \sqcup Q$ and $Y = f(Q) \sqcup E$}
	\Explain{ But $g(E)$ is Borel isomorphic with $E$}
	\Explain{ And $f(Q)$ is Borel isomorphic with $Q$}
	\Explain{ So $X$ and $Y$ are indeed Borel isomorphic}
	\EndProof
}\Page{
	\Theorem{IsomorphismTHM}
	{
		\forall X,Y : \SBS \.
		|X| = |Y| \Imply X \cong_\BOR Y
	}
	\Explain{ By universal property of Hilbert cube $X$ can be embedded into $I^\Nat$}
	\Explain{ But $\C$ is Borel Isomorphic to $I^\Nat$}
	\Explain{ So, $\C$ can be embedded into $X$ and $X$ into $\C$ again}
	\Explain{ Then, use Borel-Schroder-Bernstein theorem so $X \cong_\BOR \C$ }
	\Explain{ Thus, all uncountable standard Borel spaces are isomorphic}
	\Explain{ For countable spaces the arguments are more all less trivial}
	\EndProof
	\\
	\Theorem{UncountableIsomorphismTHM}
	{
		\forall X,Y : \SBS \.
		|X| > \aleph_0  \And 
		|Y| > \aleph_0 
		\Imply X \cong_\BOR Y
	}
	\Explain{ See previous result}
	\EndProof
	\\
	\Theorem{DoubleBorelIsomorphismTHM}
	{
		\forall X,Y : \SBS \.
		\forall A \in \B(X) \.
		\forall B \in \B(Y) \. \NewLine \.
		|A| = |B| \And \Big|A^\c\Big| = \Big|B^\c\Big|
		\iff
		\exists X \ToIso{\varphi} B : \BOR \.
		\varphi(A) = B
	}
	\Explain{ Without loss of generality we can choose polish topologies
		such that both $A$ and $B$ is clopen	
	}
	\Explain{ Then  $X = A \sqcup A^\c$ and $Y = B \sqcup B^\c$ 
		as topological and measurable spaces}
	\Explain{ By isomorphism theorem there are Borel isomorphism $A \ToIso{\psi} B$ 
		and $A^\c \ToIso{\psi'} B^\c$}
	\Explain{ Then $\varphi = \psi \sqcup \psi'$ is an isomorphism with required property}
	\Explain{ To see the contrary it is always possible to limit $\varphi$ on $A$}
	\EndProof
}
\newpage
\subsubsection{Induced Homomorphism}
\Page{
	\DeclareType{\IH}
	{
		\prod X,Y \in \BOR \. 
		\prod I : \SIdeal(\S_X) \.
		\prod Y \Arrow{\Phi} \frac{X}{I} : \BOOL \.
		? \BOR(X,Y) 
	}
	\DefineType{\varphi}{\IH}
	{\forall B \in \B(X) \. \Phi(B) = \Big[\varphi^{-1}(B)\Big]}	
	\\
	\Theorem{SikorskiInducedHomomorphismTheorem}
	{
		\NewLine ::		
		\forall X \in \BOR \.
		\forall Y : \SBS \.
		\forall [0] : \exists Y \.
		\forall I : \SIdeal(\S_X) \. \ \NewLine \.
		\forall \Phi : \sC\left(\B(Y), \frac{\S_X}{I}  \right) \.
		\exists  \IH(X,Y,I,\Phi)
	}
	\Explain{By Isomorphism Theorem assume $Y=[0,1]$ }
	\Say{\Big(B,[1]\Big)}{
		\LOGIC{Choice}\Big(
			\Rats \cap [0,1], 
			\Lambda p \in \Rats \cap [0,1] \. 
			\Big)
	}
	{
		\sum \Rats \cap [0,1] \to \B(Y) \.
		\forall p \in \Rats \cap [0,1] \. \Phi[0,p] = \Big[ B_p \Big]_I
		\And \NewLine \And
		B_1 = X
	}
	\Say{\varphi}{\Lambda x \in X \. \inf\Big\{ p \in \Rats \cap [0,1] \Big| x \in B_p \Big\} }
	{
		X \to [0,1]
	}
	\Say{[2]}{[1]\Elim \varphi}
	{
		\forall t \in [0,1] \. \varphi^{-1}[0,t]= \bigcup_{p < t} B_p
	}
	\Say{[3]}{\THM{MeasurableByBase}[2]}
	{
		\varphi \in \BOR\Big(X,[0,1]\Big)
	}
	\SayIn{\Phi'}{\Lambda A \in \B(Y) \. [\varphi^{-1} A]_I}
	{
		\sC\left(\B(Y), \frac{\S_X}{I} \right)	
	}
	\Say{[4]}{\Elim \Phi' [2]}
	{
		\forall p \in \Rats \cap [0,1] \. \Phi[0,p] = \Phi'[0,p]
	}
	\Conclude{[*]}{\Elim \sC\left( \B(X), \frac{\S_X}{I}  \right)}
	{
		\Phi = \Phi'
	}
	\EndProof
	\\
	\Theorem{SikorskiInducedHomomorphismUniqueness}
	{
		\NewLine ::		
		\forall X \in \BOR \.
		\forall Y : \SBS \.
		\forall [0] : \exists Y \.
		\forall I : \SIdeal(\S_X) \. \ \NewLine \.
		\forall \Phi : \sC\left(\B(Y), \frac{\S_X}{I}  \right) \.
		\forall \varphi,\psi :  \IH(X,Y,I,\Phi) \.
		\Big\{ x \in X : \varphi(x) \neq \psi(x)   \Big\} \in I
	}
	\Explain{Again assume $Y=[0,1]$ }
	\Assume{[1]}{ \Big\{ x \in X : \varphi(x) < \psi(x)   \Big\} \not \in I  }
	\Say{\Big(q, [2]\Big)}{\THM{RationalDensity}}
	{
		\sum q \in \Rats  \. 
		 \Big\{ x \in X : \varphi(x)  \le q < \psi(x)   \Big\} \not \in I
	}
	\Say{[3]}{[2]\Intro \FUNC{preimage}\Intro \setminus}
	{
		\varphi^{-1}[0,q] \setminus  \psi^{-1}[0,q] \not \in I
	}
	\Say{[4]}{\Elim \IH(X,Y,I,\Phi,\varphi \And \Psi)}
	{
		\Big[\varphi^{-1}[0,q]\Big]_I = \Phi[0,1] = \Big[\psi^{-1}[0,q]\Big]_I
	}
	\Conclude{[1.*]}{\Elim [\bullet]_I [4][3]}{\bot}
	\Explain{Same reasoning  works for the case $\psi(x) < \varphi(x)$}
	\DeriveConclude{[*]}{\Intro I}{\Big\{ x \in X : \varphi(x) \neq \psi(x)   \Big\} \in I}
	\EndProof
}\Page{
	\Theorem{DoubleSikorskiInducedIsomorphisTheorem}
	{
		\NewLine ::		
		\forall X,Y : \SBS \.
		\forall I : \SIdeal\big(\B(X)\big) \. 
		\forall J : \SIdeal\big( \B(Y) \big) \.		
		\ \NewLine \.
		\forall \Phi : \sC\left(\frac{\B(X)}{I}, \frac{\B(Y)}{J}  \right) \.
		\frac{\B(X)}{I} \ToIso{\Phi}
		\frac{\B(Y)}{J} : \BOOL
		\iff 
		\exists A \in \B(X) \.
		\exists B \in \B(Y) \. \NewLine \.
		\exists! \varphi \in \BOR(B,A) \.
		A^\c \in I \And B^\c \in J \And
		\forall [C]_I \in \frac{X}{I} \.
		\Phi[C]_I = \Big[ \varphi^{-1}(C \cap A) \Big]_J 
	}
	\Explain{Apply Sikorski theorem two times in both directions, then combine}
	\EndProof
	\\
	\Theorem{SikorskiInducedAutomorphisTheorem}
	{
		\NewLine ::		
		\forall X, : \SBS \.
		\forall I : \SIdeal\big(\B(X)\big) \. 		
		\forall \Phi : \sC\left(\frac{\B(X)}{I}, \frac{\B(X)}{I}  \right) \. \NewLine \.
		\Phi \in \Aut_\BOOL\left(\frac{\B(X)}{I}\right) \.
		\iff  
		\exists!  \varphi \in \Aut_\BOR(X) \.
		\forall [C]_I \in \frac{X}{I} \.
		\Phi[C]_I = \Big[ \varphi^{-1}(C \cap A) \Big]_I 
	}
	\Explain{Apply Sikorski isomorphism theorem to automorphism}
	\Explain{It must be possible to choose $A,B = X$ as $|X| = |X|$}
	\EndProof
	\\
	\Theorem{CategoryAlgebraBorelExpression}
	{
		\forall X : \Polish \.
		\cat(X) =  \frac{\B(X)}{\B(X) \cap \MGR(X)}
	}
	\NoProof
	\\
	\Theorem{CategoryAlgebraInducedHomo}
	{
		\NewLine		
		\forall X : \TYPE{Perfect} \And \Polish \.
		\forall \Phi \in \Aut_\BOOL\Big(\cat(X)\Big) \.
		\exists A \in G_\delta(X) \.		
		\exists! \varphi \in \End_\TOP(A) \. \NewLine
		\forall [B] \in \cat(X) \. 
		\Phi[B] = \Big[\varphi^{-1}(A \cap B)\Big]
	}
	\NoProof
}
\newpage 
\subsubsection{Definability of Baire Sets}
\Page{
		\Theorem{NovikovMontgomeryNonmeagerTHM}
		{
			\NewLine :			
			\forall X \in \BOR \.
			\forall Y : \Polish \.
			\forall A \in \S_X \otimes \B(Y) \.
			\forall U \in \T(X) \. \NewLine \.
			\{ x \in X : \exists^* u \in U \. A(x,u)  \} \in \S_X		
		}
		\Assume{[1]}{\exists U}
		\Say{\Big(\V,[2]\Big)}{\Elim \TYPE{SecondCountable}(Y)}
		{
			\sum \V : \TYPE{BaseOfTopology}(X) \. |\V| \le \aleph_0
		}
		\Say{V}{\FUNC{enumerate}\Big(\V,[2]\Big)}{\Surj(\Nat,\V)}
		\Say{\A}
		{
			\Big\{
				A \in  \S_X \otimes \B(Y) \. 
				\forall U \in \T(Y) \.				
				\{ x \in X : \exists^* u \in U \. A(x,u)   \} \in \S_X
			\Big\}
		}
		{
			?\Big(\S_X \otimes \B(Y)\Big)
		}
		\Say{C}
		{
			\Lambda A \in \S_X \otimes \B(Y) \.
			\Lambda U \in \T(Y) \.
			\{ x \in X : \exists^* u \in U \. A(x,u)   \}
		}{
			\S_X \otimes \B(Y) \to \T(Y) \to ?X
		}
		\Say{[3]}
		{
		}
		{
			\forall S \in \S_X \.
			\forall U,V \in \T(X) \.
			C\Big( S \times V, U\Big) = \If \exists U \times V \Then S \Else \emptyset
		}
		\Say{[4]}{\Elim \A [3]}
		{
			\Big\{ S \times V | S \in \S_X, V \in \T(Y) \Big\} \subset \A
		}
		\Say{[5]}{\Elim C \THM{NonmeagerUnion}(Y)}
		{
			\forall A : \Nat \to \S_X \otimes \B(Y) \.
			\forall U \in \T(Y) \.
			C\left( \bigcup^\infty_{n=1} A_n, U \right) =
			\bigcup^\infty_{n=1} C(A_n,U)
		}
		\Say{[6]}{[5]\Elim \SA(X,\S_X) \Elim \A}
		{
			\forall A : \Nat \to \A \.
			\bigcup^\infty_{n=1} A_n \in \A
		}
		\AssumeIn{A}{\A}
		\AssumeIn{U}{\T(X)}
		\AssumeIn{x}{X}
		\Say{[7]}{
			\Elim \Big(x \in  C(A^\c,U)\Big)
			\THM{CategoryDeMorganaLaw}(Y)
			\Elim \BP\Big(Y,\sigma_{1,x}(S)\Big)
		}
		{
			\NewLine :			
			x \in  C(A^\c,U) 
			\iff
			\exists^* u \in U \.   A^\c(x,u) \iff
			\neg \forall^* u \in U \. A(x,u) \iff
			\neg \forall n \in \Nat \. \exists^* v \in V_n \. A(x,v)
		}
		\Conclude{[A.*]}{\Elim \A(A)[7]}{A^\c \in \A}
		\Derive{[7]}{\Intro \SA [6]}{\SA(X\times Y,\A)}
		\Conclude{[8]}{\Elim \TYPE{Product}[4][7] }
		{
			\S_X \times \B(Y) \subset \A
		}
		\EndProof
		\\
		\Theorem{NovikovMontgomeryMeagerTHM}
		{
			\NewLine :			
			\forall X \in \BOR \.
			\forall Y : \Polish \.
			\forall A \in \S_X \otimes \B(Y) \.
			\forall U \in \T(X) \. \NewLine \.
			\{ x \in X : \neg\exists^* u \in U \. A(x,u)  \} \in \S_X		
		}
		\NoProof
		\\
		\Theorem{NovikovMontgomeryComeagerTHM}
		{
			\NewLine :			
			\forall X \in \BOR \.
			\forall Y : \Polish \.
			\forall A \in \S_X \otimes \B(Y) \.
			\forall U \in \T(X) \. \NewLine \.
			\{ x \in X : \forall^* u \in U \. A(x,u)  \} \in \S_X		
		}
		\EndProof
}
\newpage
\subsection{Uniformization}
\subsection{Partitions}
\subsection{Games}
\subsection{Hierarchi}
\subsection{Applictions}
\subsection{Baire Hierarchi}
\section{Analytic and Projective Sets}
\newpage
\section*{Sources:}
\begin{enumerate}
\item CLASSICAL DESCRIPTIVE SET THEORY by Alexander S. Kechris  Springer Verlag
\end{enumerate}
\end{document}

