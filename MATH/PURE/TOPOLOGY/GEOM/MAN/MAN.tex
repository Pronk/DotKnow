\documentclass[12pt]{scrartcl}
\usepackage{mathtools}
\usepackage{amsmath}
\usepackage{amsfonts}
\usepackage{hyperref}
\usepackage{amssymb}
\usepackage{ wasysym }
\usepackage{accents}
\usepackage{extpfeil}
\usepackage{graphicx}
\usepackage{scalerel}
\usepackage{esvect}
\usepackage{upgreek}
\usepackage[dvipsnames]{xcolor}
\usepackage[a4paper,top=5mm, bottom=5mm, left=10mm, right=2mm]{geometry}
%Markup
\newcommand{\TYPE}[1]{\textcolor{NavyBlue}{\mathtt{#1}}}
\newcommand{\FUNC}[1]{\textcolor{Cerulean}{\mathtt{#1}}}
\newcommand{\LOGIC}[1]{\textcolor{Blue}{\mathtt{#1}}}
\newcommand{\THM}[1]{\textcolor{Maroon}{\mathtt{#1}}}
%META
\renewcommand{\.}{\; . \;}
\newcommand{\de}{: \kern 0.1pc =}
\newcommand{\extract}{\LOGIC{Extract}}
\newcommand{\where}{\LOGIC{where}}
\newcommand{\If}{\LOGIC{if} \;}
\newcommand{\Then}{ \; \LOGIC{then} \;}
\newcommand{\Else}{\; \LOGIC{else} \;}
\newcommand{\IsNot}{\; ! \;}
\newcommand{\Is}{ \; : \;}
\newcommand{\DefAs}{\; :: \;}
\newcommand{\Act}[1]{\left( #1 \right)}
\newcommand{\Example}{\LOGIC{Example} \; }
\newcommand{\Theorem}[2]{& \THM{#1} \, :: \, #2 \\ & \Proof = \\ } 
\newcommand{\DeclareType}[2]{& \TYPE{#1} \, :: \, #2 \\} 
\newcommand{\DefineType}[3]{& #1 : \TYPE{#2} \iff #3 \\} 
\newcommand{\DefineNamedType}[4]{& #1 : \TYPE{#2} \iff #3 \iff #4 \\} 
\newcommand{\DeclareFunc}[2]{& \FUNC{#1} \, :: \, #2 \\}  
\newcommand{\DefineFunc}[3]{&  \FUNC{#1}\Act{#2} \de #3 \\} 
\newcommand{\DefineNamedFunc}[4]{&  \FUNC{#1}\Act{#2} = #3 \de #4 \\} 
\newcommand{\NewLine}{\\ & \kern 1pc}
\newcommand{\Page}[1]{ \begin{align*} #1 \end{align*}   }
\newcommand{ \bd }{ \ByDef }
\newcommand{\NoProof}{ & \ldots \\ \EndProof}
%LOGIC
\renewcommand{\And}{\; \& \;}
\newcommand{\ForEach}[3]{\forall #1 : #2 \. #3 }
\newcommand{\Exist}[2]{\exists #1 : #2}
\newcommand{\Imply}{\Rightarrow} 
\newcommand{\Intro}{\LOGIC{I}}
\newcommand{\Elim}{\LOGIC{E}}
%TYPE THEORY
\newcommand{\DFunc}[3]{\prod #1 : #2 \. #3 }
\newcommand{\DPair}[3]{\sum #1 : #2 \. #3}
\newcommand{\Type}{\TYPE{Type}}
\newcommand{\Class}{\TYPE{Kind}}
%%STD
\newcommand{\Int}{\mathbb{Z} }
\newcommand{\NNInt}{\mathbb{Z}_{+} }
\newcommand{\Reals}{\mathbb{R} }
\newcommand{\Complex}{\mathbb{C}}
\newcommand{\Rats}{\mathbb{Q} }
\newcommand{\Sphere}{\mathbb{S}}
\newcommand{\Nat}{\mathbb{N} }
\newcommand{\EReals}{\stackrel{\mathclap{\infty}}{\mathbb{R}}}
\newcommand{\ERealsn}[1]{\stackrel{\mathclap{\infty}}{\mathbb{R}}^{#1}}
\DeclareMathOperator*{\centr}{center}
\DeclareMathOperator*{\argmin}{arg\,min}
\DeclareMathOperator*{\id}{id}
\DeclareMathOperator*{\im}{Im}
\DeclareMathOperator*{\supp}{supp}
\newcommand{\EqClass}[1]{\TYPE{EqClass}\left( #1 \right)}
\newcommand{\Cat}{\TYPE{Category}}
\newcommand{\Mor}{\mathcal{M}}
\newcommand{\Obj}{\mathcal{O}}
\newcommand{\End}{\mathrm{End}}
\newcommand{\Aut}{\mathrm{Aut}}
\newcommand{\Func}[2]{\TYPE{Functor}\left( #1, #2 \right)}
\mathchardef\hyph="2D
\newcommand{\Surj}[2]{\TYPE{Surjective}\left( #1, #2 \right)}
\newcommand{\ToInj}{\hookrightarrow}
\newcommand{\ToMono}{\xhookrightarrow}
\newcommand{\ToSurj}{\twoheadrightarrow}
\newcommand{\ToEpi}{\xtwoheadrightarrow}
\newcommand{\ToBij}{\leftrightarrow}
\newcommand{\ToIso}{\xleftrightarrow}
\newcommand{\Arrow}{\xrightarrow}
\newcommand{\Set}{\TYPE{Set}}
\newcommand{\du}{\; \triangle \;}
\renewcommand{\c}{\complement}
\renewcommand{\i}{\mathbf{i}}
\newcommand{\llbracket}{\left[\!\!\left[}
\newcommand{\rrbracket}{\right]\!\!\right]}
%%ProofWritting
\newcommand{\Say}[3]{& #1 \de #2 : #3, \\}
\newcommand{\SayIn}[3]{& #1 \de #2 \in #3, \\}
\newcommand{\Conclude}[3]{& #1 \de #2 : #3; \\}
\newcommand{\Derive}[3]{& \leadsto #1 \de #2 : #3, \\}
\newcommand{\DeriveConclude}[3]{& \leadsto #1 \de #2 : #3 ; \\} 
\newcommand{\Assume}[2]{& \LOGIC{Assume} \; #1 : #2, \\}
\newcommand{\AssumeIn}[2]{& \LOGIC{Assume} \; #1 \in #2, \\}
\newcommand{\As}{\; \LOGIC{as } \;} 
\newcommand{\QED}{\; \square}
\newcommand{\EndProof}{& \QED \\}
\newcommand{\ByDef}{\rotatebox[origin=c]{-180}{$D$}}%\text{\textthorn}}  %Extracts defining type statement from the type member, may be inverted  (T -> Type)
\newcommand{\ByConstr}{\rotatebox[origin=c]{-180}{$C$}}%\text{\textopeno}} %Extract the defining statement from the defined value 
\newcommand{\Alt}{\LOGIC{Alternative} \;}
\newcommand{\CL}{\LOGIC{Close} \;}
\newcommand{\More}{\LOGIC{Another} \;}
\newcommand{\Proof}{\LOGIC{Proof} \; }
%CategoryTheory
%Types
\newcommand{\Cov}{\TYPE{Covariant}}
\newcommand{\Contra}{\TYPE{Contravariant}}
\newcommand{\NT}{\TYPE{NaturalTransform}}
\newcommand{\UMP}{\TYPE{UnversalMappingProperty}}
\newcommand{\CMP}{\TYPE{CouniversalMappingProperty}}
\newcommand{\paral}{\rightrightarrows}
%functions
\newcommand{\op}{\mathrm{op}}
\newcommand{\obj}{\mathrm{obj}}
\DeclareMathOperator*{\dom}{dom}
\DeclareMathOperator*{\codom}{codom}
\DeclareMathOperator*{\colim}{colim}
%variable
\newcommand{\C}{\mathcal{C}}
\newcommand{\A}{\mathcal{A}}
\newcommand{\B}{\mathcal{B}}
\newcommand{\D}{\mathcal{D}}
\newcommand{\I}{\mathcal{I}}
\newcommand{\J}{\mathcal{J}}
\newcommand{\R}{\mathcal{R}}
%Cats
\newcommand{\CAT}{\mathsf{CAT}}
\newcommand{\SET}{\mathsf{SET}}
\newcommand{\PARALLEL}{\bullet \paral \bullet}
\newcommand{\WEDGE}{\bullet \to \bullet \leftarrow \bullet}
\newcommand{\VEE}{\bullet \leftarrow \bullet \to \bullet}
%Topology
%General Topology
%Types
\newcommand{\TS}{\TYPE{TopologicalSpace}} 
\newcommand{\LF}{\TYPE{LocallyFinite}}
\newcommand{\PN}{\TYPE{PerfectlyNormal}}
%FUNC
\DeclareMathOperator*{\intx}{int}
\DeclareMathOperator*{\cl}{cl} 
\DeclareMathOperator*{\boundary}{\partial} 
\DeclareMathOperator{\combo}{\triangledown} 
\DeclareMathOperator{\diag}{\mathrm{diag}} 
\DeclareMathOperator{\Diag}{\triangle} 
\DeclareMathOperator{\rem}{rem}
%CATS
\newcommand{\TOP}{\mathsf{TOP}}
\newcommand{\HC}{\mathsf{HC}}
\newcommand{\CG}{\mathsf{CG}}
%Symbols
\newcommand{\T}{\mathcal{T}}
\newcommand{\U}{\mathcal{U}}
\renewcommand{\O}{\mathcal{O}}
\renewcommand{\d}{\mathrm{d}}
\newcommand{\F}{\mathcal{F}}
\newcommand{\X}{\mathcal{X}}
%\newcommand{\d}{\mathrm{d}}
%Metric Topology
%FUNC
\DeclareMathOperator{\diam}{diam}
\newcommand{\Disk}{\mathbb{D}}
\newcommand{\Ball}{\mathbb{B}}
%CATS
\newcommand{\Semiiso}{\mathsf{SMS}_{\circ \to \cdot}}
\newcommand{\Iso}{\mathsf{MS}_{\circ \to \cdot}}
\newcommand{\SMS}{\mathsf{SMS}}
\newcommand{\MS}{\mathsf{MS}}
\newcommand{\UNI}{\mathsf{UNI}}
\newcommand{\UNIS}{\mathsf{UNIS}}
\newcommand{\TG}{\mathsf{TG}}
%Algebra
%Groups
%Types
\newcommand{\Group}{\TYPE{Group}}
\newcommand{\Abel}{\TYPE{Abelean}}
\newcommand{\Sgrp}{\subset_{\mathsf{GRP}}}
\newcommand{\Nrml}{\vartriangleleft}
\newcommand{\FG}{\TYPE{FiniteGroup}}
\newcommand{\Stab}{\mathrm{Stab}}
\newcommand{\FGA}{\TYPE{FinitelyGeneratedAbelean}}
\newcommand{\DN}{\TYPE{DirectedNormality}}
\newcommand{\action}{\curvearrowleft}
%Func
\DeclareMathOperator{\tor}{tor}
\DeclareMathOperator{\bool}{bool}
\DeclareMathOperator{\rank}{rank}
\DeclareMathOperator{\Fix}{Fix}
%Cats
\newcommand{\GRP}{\mathsf{GRP}}
\newcommand{\ABEL}{\mathsf{ABEL}}
%Ops
\newcommand{\SDP}{\rightthreetimes}
%LINEAR
%Linear Algebra
%Types
\newcommand{\Basis}{\TYPE{Basis}} % Basis of the linear space
\newcommand{\submod}[1]{\subset_{\LMOD{#1}}}% submodule as a subset
\newcommand{\subvec}[1]{\subset_{\VS{#1}}}% vector subspace as a subset
\newcommand{\FGM}{\TYPE{FinitelyGeneratedModule}}% Finitely generated module
\newcommand{\LI}{\TYPE{LinearlyIndependent}}
\newcommand{\LIS}{\TYPE{LinearlyIndependentSet}}
\newcommand{\FM}{\TYPE{FreeModule}}
\newcommand{\IBP}{\TYPE{InvariantBasisProperty}}
\newcommand{\UTM}{\TYPE{UpperTriangularMatrix}}
\newcommand{\LTM}{\TYPE{LowerTriangularMatrix}}
%\newcommand{\Diag}{\TYPE{DiagonalMatrix}}
\newcommand{\FP }{\TYPE{FinitelyPresented}}
\newcommand{\GL}{\mathbf{GL}}% General Linear Group
\newcommand{\SL}{\mathbf{SL}}% Special Linear group
\newcommand{\SO}{\mathbf{SO}}
\newcommand{\SU}{\mathbf{SU}}
\newcommand{\prsubvec}[1]{\subsetneq_{\VS{#1}}}	% poper vector subspace as a subset
\newcommand{\LC}{\TYPE{LinearComplement}} 
\newcommand{\IS}{\TYPE{InvariantSubspace}}
\newcommand{\RP}{\TYPE{ReducingPair}}
\newcommand{\RCF}{\TYPE{RationalCanonicalForm}}
\newcommand{\JCF}{\TYPE{JordanCanonicalForm}}
\newcommand{\Diagble}{\TYPE{Diagonalizable}}
\newcommand{\UT}{\TYPE{UpperTriangulizable}}
\newcommand{\LT}{\TYPE{LowerTriangulizable}}
\newcommand{\IPS}{\TYPE{InnerProductSpace}}
\newcommand{\OBasis}{\TYPE{OrthonormalBasis}}
\newcommand{\FDIPS}{\TYPE{FiniteDimensionalInnerProductSpace}}
\newcommand{\NO}{\TYPE{NormalOperator}}
\newcommand{\NM}{\TYPE{NormalMatrix}}
\newcommand{\SA}{\TYPE{SelfAdjoint}}
\newcommand{\SSA}{\TYPE{SkewSelfAdjoint}}
\newcommand{\PI}{\TYPE{Pseudoinverse}}
%\newcommand{\OVS}{\TYPE{OrthogonalVectorSpace}}
\newcommand{\SVS}{\TYPE{SymplecticVectorSpace}}
\newcommand{\MVS}{\TYPE{MetricVectorSpace}}
\newcommand{\FDMVS}{\TYPE{FiniteDimensionalMetricVectorSpace}}
\newcommand{\Sp}{\mathbf{Sp}}
%Func
\DeclareMathOperator{\Span}{span} % spann by subset
\DeclareMathOperator{\Ann}{Ann}   % annihilator
\DeclareMathOperator{\Ass}{Ass}   % associated primes:
\DeclareMathOperator{\adj}{adj}   % an adjoint matrix
\DeclareMathOperator{\tr}{tr}     % trace
\DeclareMathOperator{\codim}{codim} % codimension
\DeclareMathOperator{\Cell}{\mathbf{C}} % a componion matrix
\DeclareMathOperator{\JC}{\mathbf{J}}  % a Jordan cell
\DeclareMathOperator{\bigboxplus}{\scalerel*{\boxplus}{\sum}} % a direct sum of operators in the sence of the reducing a pair
\DeclareMathOperator{\Spec}{Spec} % Spectre
\DeclareMathOperator{\bigbot}{\scalerel*{\bot}{\sum}} % an othogonal direct sum
\DeclareMathOperator{\GS}{\mathbf{GS}} %Gramm-Smmidt process
\DeclareMathOperator{\NGS}{\mathbf{NGS}} %Normalized Gramm-Smmidt process
\DeclareMathOperator{\WI}{\mathrm{WI}} %Witt Index
%Cats
\newcommand{\VS}[1]{#1\hyph\mathsf{VS}} % a category of vector spaces (Field)
\newcommand{\FDVS}[1]{#1\hyph\mathsf{FDVS}} % a category of finite-dimensional vector spaces (Field)
\newcommand{\LALGE}[1]{#1\hyph\mathsf{ALGE}}
\newcommand{\LMOD}[1]{#1\hyph\mathsf{MOD}} % a category of the left modules (Ring)
\newcommand{\RMOD}[1]{\mathsf{MOD}\hyph#1} % a category of the right modules (Ring)
\newcommand{\LLMAP}[1]{#1\hyph\mathsf{LMAP}} % a cagory of based linear maps with the left scalar multiplication (Ring)
\newcommand{\LMAT}[1]{#1\hyph\mathsf{MAT}}  % a category of based matrices with the left scalar multiplication (Ring)
\newcommand{\NMAT}[1]{#1\hyph\mathbb{N}} % a category of finite matrices (Field)
%Symbols
\renewcommand{\L}{\mathcal{L}}
%\renewcommand{\O}{\mathbf{O}}
\renewcommand{\S}{\mathcal{S}}
%FIELDS
\newcommand{\Field}{\TYPE{Field}}
\newcommand{\ACF}{\TYPE{AlgebraicallyClosedField}}
%RINGS
%TYPE
\newcommand{\Ring}{\TYPE{Ring}}
\newcommand{\CR}{\TYPE{CommutativeRing}}
\newcommand{\Ideal}{\TYPE{Ideal}}
\newcommand{\ID}{\TYPE{IntegralDomain}}
\newcommand{\UFD}{\TYPE{UniqueFactorizationDomain}}
\newcommand{\PID}{\TYPE{PrincipleIdealDomain}}
\newcommand{\FGI}{\TYPE{FinitelyGeneratedIdeal}}
\newcommand{\ER}{\TYPE{EuclideanRing}}
\newcommand{\DVR}{\TYPE{DiscreteValuationRing}}
\newcommand{\MoFT}{\TYPE{MonoidOfFiniteType}}
%CATS
\newcommand{\RING}{\mathsf{RING}} % A category of Rings
\newcommand{\ANN}{\mathsf{ANN}} % A category of Commutative Rings
%FUNCS
\DeclareMathOperator{\lcd}{lcd} % least common devided 
\DeclareMathOperator{\lc}{lc} % leading coefficient of the polynomial
\DeclareMathOperator{\cont}{cont} % content of the polynomial
\DeclareMathOperator{\antideg}{antideg} % antideree if the foramal power series
%Symbolsqq
%ALGEBRA
\newcommand{\LALG}[1]{#1\hyph\mathsf{ALG}}% Left associative unital algebras (Ring)
\newcommand{\RALG}[1]{\mathsf{ALG}\hyph#1}% Right associative unital  algebras (Rings)
%Numbers
%Integers
%FUNCS
\DeclareMathOperator{\divi}{div} % devide withou reminder
\DeclareMathOperator{\remi}{rem} % reminder
\DeclareMathOperator{\Frac}{Frac} % Field of fractions
%Complex
%Symb
\newcommand{\Herm}{\mathbf{H}}
\newcommand{\p}{\mathbf{p}}
\newcommand{\Inv}{\mathrm{Inv}}
\newcommand{\Stg}{\mathrm{Stg}}
\newcommand{\M}{\mathcal{M}}
%Geometry
%Affine
%Type
\newcommand{\AS}{\TYPE{AffineSpace}}
\newcommand{\ASS}{\TYPE{AffineSubspace}}
\newcommand{\AI}{\TYPE{AffineIndepend}}
\newcommand{\WL}{\TYPE{WithLines}}
\newcommand{\SAFF}{\mathbf{SAFF}}
\newcommand{\AFF}{\mathbf{AFF}}
\newcommand{\SLI}{\mathcal{SL}}
\newcommand{\SGL}{\mathbf{SGL}}
\newcommand{\GVS}{\TYPE{GeometricVectorSpace}}
\newcommand{\TP}{\TYPE{TrigonometricPlane}}
\newcommand{\OrVS}{\TYPE{OrientatedVectorSpace}}
\newcommand{\OTP}{\TYPE{OrientatedTrigonometricPlane}}
\newcommand{\MAS}{\TYPE{MetricAffineSpace}}
%Func
\newcommand{\Gr}{\mathrm{Gr}}
\newcommand{\Di}{\mathrm{Di}}
\newcommand{\Sc}{\mathrm{Sc}}
\newcommand{\Tr}{\mathrm{Tr}}
\DeclareMathOperator{\Aff}{Aff}
\DeclareMathOperator{\rat}{rat}
%Symbol
\newcommand{\tri}{\triangle}
%Projective
%TYPE
\newcommand{\subproj}[1]{\subset_{\PROJ{#1}}}
%FUNC
\newcommand{\vs}{\mathsf{VS}}
\newcommand{\PGL}{\mathbf{PGL}}
%\CAT
\newcommand{\PROJ}[1]{#1\hyph\mathsf{PROJ}}
%Symbol
\renewcommand{\P}{\mathbb{P}} 
%TOPVS
\newcommand{\TOPVS}[1]{#1\hyph\mathsf{TOPVS}} % a category of topological vector spaces (Field)
%Convex
%
\newcommand{\Convex}{\TYPE{Convex}}
\newcommand{\CB}{\TYPE{ConvexBody}}
\newcommand{\CC}{\TYPE{ConvexCone}}
\newcommand{\CF}{\TYPE{ConvexFunction}}
\newcommand{\Mink}{\TYPE{MinkowskySpace}}
\newcommand{\Euc}{\TYPE{EucledeanSpace}}
\newcommand{\Cone}{\TYPE{Cone}}
\newcommand{\Wedge}{\TYPE{Wedge}}
\newcommand{\PVS}{\TYPE{PreorderedVectorSpace}}
\newcommand{\OVS}{\TYPE{OrderedVectorSpace}}
\newcommand{\AVS}{\TYPE{ArchemedeanVectorSpace}}
\newcommand{\RS}{\TYPE{RieszSpace}}
%FUNC
\newcommand{\rint}{{\mathrm{rel}  \intx}}
\DeclareMathOperator{\lina}{lina}
\DeclareMathOperator{\lin}{lin}
\DeclareMathOperator{\core}{core}
\DeclareMathOperator{\conv}{conv}
\DeclareMathOperator{\cconv}{\overline{conv}}
\DeclareMathOperator{\CCONV}{\mathsf{CCONV}}
\renewcommand{\H}{\mathrm{H}}
%%TOPOLOGICAL MANIFOLD
%TYPE
\newcommand{\SC}{\TYPE{SimplicialComplex}}
\newcommand{\ASC}{\TYPE{AbstractSimplicialComplex}}
\newcommand{\CS}{\TYPE{CompactSurface}}
\newcommand{\RH}{\TYPE{RelativelyHomotopic}}
\newcommand{\SVKD}{\TYPE{SeifertVanKampenDecomposition}}
%\FUNC
\newcommand{\Torus}{\mathbb{T}}
\DeclareMathOperator{\bigsum}{\scalerel*{\#}{\sum}} % a connected sum
\DeclareMathOperator{\ind}{ind}
\DeclareMathOperator{\Gal}{Gal}
%CATS
\newcommand{\TOPM}{\mathsf{TOPM}}
\newcommand{\CW}{\mathsf{CW}}
\newcommand{\CWC}{\mathsf{CWC}}
\newcommand{\CWR}{\mathsf{CWR}}
\newcommand{\HTOP}{\mathsf{HTOP}}
\newcommand{\COV}{\mathsf{COV}}
%Symbol
\newcommand{\VF}{\mathfrak{X}}
\newcommand{\E}{\mathcal{E}}
\DeclareFontFamily{U}{skulls}{}
\DeclareFontShape{U}{skulls}{m}{n}{ <-> skull }{}
\newcommand{\skull}{\text{\usefont{U}{skulls}{m}{n}\symbol{'101}}}
\renewcommand{\O}{\mathcal{O}}
\author{Uncultured Tramp} 
\title{Topological Manifolds}
\begin{document}
\maketitle
\newpage
\tableofcontents
\newpage
\section{Subject Matter}
\subsection{Paracompactness and Partition of Unity}
\Page{
	\DeclareType{LocallyFinite}
	{
		\prod_{ X \in \TOP } ?^3 X
	}
	\DefineType{\A}{LocallyFinite}
	{
		\forall x \in X \.
		\exists U \in \U(x) \.
		\Big|\Big\{ A \in \A : A \cap U \neq \emptyset   \Big\}\Big| < \infty
	}
	\\
	\DeclareType{OpenRefinement}
	{
		\prod X \in \TOP \. 
		\TYPE{Cover}(X) \to ?\TYPE{OpenCover}(X)
	}
	\DefineType{\U}{
		\TYPE{OpenRefinement}
	}
	{
		\Lambda \mathcal{O} : \TYPE{Cover}(X) \.
		\forall U \in \U \. \exists O \in \mathcal{O} : U \subset O
	}
	\\
	\DeclareType{Paracomapct}
	{
		?\TOP
	}
	\DefineType{X}{
		\TYPE{Paracompact}
	}
	{
		\forall \mathcal{O} : \TYPE{OpenCover}(X) \.
		\exists \mathcal{U} : 
		\TYPE{OpenRefinement}(X) \And \TYPE{LocallyFinite}
	}
	\\
	\DeclareType{Exhaustion}{\prod_{X \in \TOP} ?\Big(\Nat \to \TYPE{CompactSubset}(X)\Big)}
	\DefineType{K}{Exhaustion}{X = \bigcup^\infty_{n=1} K_n \And \forall n \in \Nat \. K_n \subset \intx K_{n+1}}
	\\
	\Theorem{ExhaustionExists}
	{
		\forall X : \TYPE{T2} \And \TYPE{SecondCountable} \And \TYPE{LocallyCompact} \.
		\exists \TYPE{Exhaustion}(X)
	}
	\Say{\Big(\B,[1]\Big)}{\bd \TYPE{LocallyCompact}}
	{
		\sum \B : \TYPE{Base}(X) \. \forall B \in \B \. \TYPE{Precompact}(X,B)
	}
	\Say{[2]}{ \bd \TYPE{SecondCountable}(B) \THM{BaseEquivalence}(X,\B)}
	{
		|\B| \le \aleph_0
	}
	\Say{B}{\FUNC{enumerate}(B,[2])}{\Nat \ToBij \B}
	\Say{C_1}{\overline{B}_1}{\TYPE{Compact}(X)}
	\Assume{n}{\Nat}
	\Assume{C}{n \to \TYPE{Compact}(X)}
	\Assume{[3]}{\forall i \in n \. B_i \subset C_i }
	\Assume{[4]}{\forall i \in [1,n-1]_\Nat \. C_i \subset \intx C_i }
	\Say{[5]}{\THM{FiniteCompactUnion}}{\TYPE{Compact}\left( X \right)}
	\Say{\Big(k,[6]\Big)}{\bd \TYPE{Compact}[5]\bd \TYPE{Base}(\B)\ByConstr B}
	{
		\sum k \in \Nat \.  \bigcup^n_{i=1} C_{i} \subset \bigcup^k_{i=1} B_i
	}
	\SayIn{k'}{\max(k,n+1)}{\Nat}
	\Say{C_{n+1}}{\bigcup^{k'}_{i=1} \overline{B}_i}{\TYPE{Compact}(X)}
	\Say{[n.*.1]}{\ByConstr C_{n+1} \bd k'}{B_{n+1} \subset C_{n+1}}
	\Conclude{[n.*.2]}{\ByConstr C_{n+1} \bd k' [6]}{C_n \subset \intx C_{n+1}}
	\Derive{\Big(C,[3]\Big)}{\Intro \Act{\sum}}
	{
		\sum C : \Nat \to \TYPE{Compact}(X) \. 
		\forall n \in \Nat \.  B_n \subset C_n \subset \intx C_{n+1}
	}
	\Say{[4]}{\bd \TYPE{Base}(\B) \ByConstr B [3]}{\bigcup^\infty_{n=1} C_n = X}
	\Conclude{[*]}{\bd^{-1} \TYPE{Exhuastiation}[4][3]}
	{ \TYPE{Exhaustiation}(X,C) }
	\EndProof
}
\Page{
	\Theorem{ParacompactnesCondition}
	{
		\forall X : \TYPE{T2} \And \TYPE{SecondCountable} \And \TYPE{LocallyCompact} \.
		\TYPE{Paracompact}
	}
	\Say{K}{\TYPE{ExhaustionExists}}
	{
		\TYPE{Exhaustion}(X)
	}
	\Assume{\mathcal{O}}{\TYPE{OpenCover}(X)}
	\Say{\U}{\Lambda n \in \Nat \. \bd \TYPE{Compact}(K_n)(\mathcal{O})}
	{
		\prod^\infty_{n=1} \TYPE{FiniteSubcover}(X,\mathcal{O},K_n) 
	}
	\Say{K_0}{\emptyset_X}{?X}
	\Say{\mathcal{O}'}{\bigcup^\infty_{n=1} \Big\{ U \setminus K_{n-1} \Big| U \in \U_n   \Big\}}{\TYPE{Refinement}(\mathcal{O})}
	\AssumeIn{x}{X}
	\Say{\Big(n,[1]\Big)}{\bd \TYPE{Exhaustion}(X)(x)}
	{
		\sum n \in \Nat \. x \in K_n
	}
	\Say{[2]}{\bd \TYPE{Exhaustion}(X)[1]}{x \in \intx K_{n+1}}
	\Conclude{[x.*]}{\ByConstr \mathcal{O}' [2] \THM{UnionCardinalityBound}(\U_{|n+1}) \THM{FiniteSumIsFinite}  }
	{
		\Big| \{ O \in \mathcal{O}' : \intx K_{n+1} \cap O \neq \emptyset \} \Big| \le \NewLine \le \sum^{n+1}_{i=1} |\U_i| < \infty
	}
	\DeriveConclude{[\mathcal{O}.*]}{\bd^{-1}\TYPE{LocallyFinite}}{\TYPE{LocallyFinite}(X,\mathcal{O}')}
	\DeriveConclude{[*]}{\bd^{-1}\TYPE{Paracompact}}
	{
		\TYPE{Paracompact}(X)
	}
	\EndProof
	\\
	\Theorem{ParacompactHausdorffIsNormal}
	{
		\forall X : \TYPE{T2} \And_{\TOP} \TYPE{Paracompact} \.
		\TYPE{T4}(X)
	}
	\Assume{A,B}{\TYPE{Closed}(X)}
	\Assume{[1]}{A \cap B = \emptyset}
	\AssumeIn{b}{B}
	\AssumeIn{a}{A}
	\Say{[2]}{\bd \TYPE{Disjoint}[1](b,a)}{b \neq a}
	\Say{\Big(U_a,V_a,\big[a.*\big]\Big)}
	{\bd \TYPE{T2}\Big(a,b,[2]\Big)}
	{
		\sum U_b \in \U(b) \. 
		\sum V_b \in \U(a)  \.
		U_n \cap V_b = \emptyset
	}
	\Derive{\Big( U,V,[2] \Big)}
	{
		\Intro \Act{\prod}
	}
	{
		\sum U,V : \prod_{a \in \A} \U(b) \times \U(a) \.
		\forall a \in A \. U(b) \cap V(a) = \emptyset
	}
	\Say{\mathcal{V}}{\{ V_a | a \in A \} \cup \Big\{ A^\c \Big\}}
	{ \TYPE{OpenCover}(X) }
	\Say{\mathcal{V}'}{\bd \TYPE{Paracompact}(X)(\mathcal{V})}
	{
		\TYPE{Refinement}(X,\mathcal{V})
		\And_{??X}
		\TYPE{LocallyFinite}(X)
	}
	\Say{\mathcal{V}''}{ \{ v \in \mathcal{V}' : \exists a \in A : v \subset V_a \}   }
	{
		\TYPE{OpenCover}(X,A) \And_{??X} \TYPE{LocallyFinite}(X)
	}
	\Assume{v}{\mathcal{V}''}
	\Say{\Big(a,[3]\Big)}{\ByConstr \mathcal{V}''}
	{
		\sum a \in A \. v \subset V_a
	}
	\Say{[4]}{[2][3]}{v \cap U_a = \emptyset}
	\Conclude{[v.*]}{\THM{ClosureAltDef}[4]}{b \not \in \overline{v} }
	\Derive{[3]}{\Intro(\forall)}{ \forall v \in \mathcal{V}'' \. b \not \in \overline{v}    }
	\Say{K_b}{\cl_X \bigcup_{v \in \mathcal{V}''} v}{\TYPE{Closed}(X)}
	\Say{[4]}{\bd^{-1} \TYPE{Union}\THM{LocallyFiniteUnionClusure}}{
		b \not \in \bigcup_{v \in \mathcal{V}''} \overline{v}  = K_b   }
	\Conclude{[b.*]}{\ByConstr K_b [2]}{A \subset \intx K_b}
	\Derive{\Big(K,[2]\Big)}{\Intro \Act{\sum}}{
		\sum K : B \to \TYPE{Closed}(X) \.
		\forall b \in B \.
		A \subset \intx K_b \And b \not \in K_b
	}
}\Page{
	\Say{\U}{ \Big\{ K_b^\c | b \in B   \Big\} \cup \{ B^\C \}}
	{
		\TYPE{OpenCover}(X)
	}
	\Say{\U'}{\bd \TYPE{Paracompact}(X)(\U)}
	{
		\TYPE{Refinement}(X,\U)
		\And_{??X}
		\TYPE{LocallyFinite}(X)
	}
	\Say{\U''}{ \{ u \in \U' : \exists b \in B : u \subset K_b^\C \}   }
	{
		\TYPE{OpenCover}(X,B) \And_{??X} \TYPE{LocallyFinite}(X)
	}
	\Say{[3]}{\THM{DualLocallyFiniteIntersection}\ByConstr \U''}
	{
		\bigcap_{u \in \U''} \intx u^\C \in \U(A)
	}
	\Conclude{\Big[(A,B).*\Big]}{ \ByConstr \U''  }
	{
		\bigcap_{u \in \U''} \intx u^\C \cap \bigcup_{u \in \U''} = \emptyset
	}
	\Derive{[*]}{\bd^{-1}\TYPE{T4}}{\TYPE{T4}(X)}
	\EndProof
	\\
	\DeclareType{PartitionOfUnity}
	{
		\prod_{X \in \TOP} 
		\prod \mathcal{O} : \TYPE{OpenCover}(X) \.
		\mathcal{O} \to X \Arrow{\TOP} [0,1]
	}
	\DefineType{f}{PartitionOfUnity}
	{
		\forall O \in \mathcal{O} \. 
		f_O\Big(O^\c\Big) = \{0\}
		\And
		\TYPE{LocallyFinite}(X,\supp f)
		\And
		\sum_{O \in \mathcal{O}} f_O = 1
	}
	\\
	\DeclareType{IndexedRefinement}{
		\prod_{X \in \TOP} 
		\prod \mathcal{I} : \TYPE{OpenCover}(X) \. 
		?(\O \to \T(X))
	}
	\DefineType{\U}{IndexedRefinement}{
		\TYPE{OpenCover}(X,\im \U) 
		\And 
		\forall O \in \mathcal{O} \. 
			\U_O \subset O
		}
	\\
	\Theorem{ParacompactOpenCoverRefiment}
	{
		\forall X : \TYPE{T2} \And \TYPE{Paracompact} \.
		\forall \mathcal{O} : \TYPE{OpenCover}(X) \. \NewLine \.
		\exists \mathcal{V} : 
		\TYPE{IndexedRefinement}(X,\mathcal{V}) :  
		\TYPE{LocallyFinite}(X, \im \mathcal{V} ) \And
		\forall O \in \mathcal{O} \.
		\overline{\mathcal{V}}_O \subset O
	}
	\AssumeIn{x}{X}
	\Say{\Big(O,[2]\Big)}{\bd \TYPE{OpenCover}(X)(x)}
	{
		\sum O \in \mathcal{O} \.
		\sum x \in O
	}
	\Derive{\Big(U_x,[x.*]\Big)}{\THM{HausdorffAltDef}[2]}
	{
		\sum U_x \in \U(x) :  
		\overline{U}_x \subset O
	}
	\Conclude{\Big(U,[1] \Big)}{\Intro \Act{\sum} \Intro \Act{\prod}}
	{
		\sum \prod_{x \in X} U_x \in \U(x) \.
		\exists O \in \mathcal{O} \.
		\overline{U}_x \subset O
	}
	\Say{\U}{\bd \TYPE{Paracompact}(X)}
	{
		\TYPE{Refinement}(X,\im \U) \And \TYPE{LocallyFinite}(X)
	}
	\Say{[2]}{\ByConstr \U[1]}
	{
		\forall u \in \U \. 
		\exists O \in \mathcal{O} \.
		\overline{u} \subset O
	}
	\AssumeIn{O}{\mathcal{O}}
	\Say{\mathcal{V}_O}{\bigcup \Big\{ u \in \mathcal{U} \Big| \overline{u} \subset O  \Big\}}
	{
		\TYPE{Open}(X)
	}
	\Conclude{[O.*]}{\THM{LocallyFiniteClosureUnion}[3]}
	{
		\overline{\mathcal{V}}_O \subset O
	}
	\Derive{\Big(\mathcal{V},[3]\Big)}{\Intro(\sum)}
	{
		\sum \mathcal{V} : \mathcal{O} \to \T(X) \.
		\forall O \in \mathcal{O} \. 
		\overline{\mathcal{V}}_O
	}
	\Say{[4]}{\ByConstr \mathcal{V}}{\TYPE{IndexedRefinement}(X,\mathcal{O},\mathcal{V})}
	\Conclude{[*]}{\bd \U \ByConstr \mathcal{V}}
	{
		\TYPE{LocallyFinite}\Big(X,\im \mathcal{V}\Big)
	}
	\EndProof
}
\Page{
	\Theorem{PartitionOfUnityExists}
	{
		\forall X : \TYPE{Paracompact} \And_{\TOP} \TYPE{T2} \.
		\forall \mathcal{O} : \TYPE{OpenCover}(X) \.
		\exists \TYPE{PartitionOfUnity}(X,\mathcal{O})
	}
	\Say{[1]}{\THM{ParacompactIsHausdorff}(X)}{\TYPE{T4}(X)}
	\Say{\Big(\mathcal{V},[2]\Big)}
	{
		\THM{ParacompactOpenCoverRefinement}(X,\mathcal{O})
	}
	{
		\sum \mathcal{V} : \TYPE{IndexedRefinement}(X,\mathcal{O}) 
		\And \NewLine \And
		\TYPE{LocallyFinite}(X)
		\.
		\forall O \in \mathcal{O} \. \overline{\mathcal{V}}_O \subset O
	}
	\Say{\Big(\mathcal{W}',[3]\Big)}
	{
		\THM{ParacompactOpenCoverRefinement}(X,\im \mathcal{V})
	}
	{
		\sum \mathcal{W}' : \TYPE{IndexedRefinement}(X,\im \mathcal{V}) 
		\And \NewLine \And
		\TYPE{LocallyFinite}(X)
		\.
		\forall V \in \im \mathcal{V} \. \overline{\mathcal{W}'}_V \subset V
	}
	\Say{\mathcal{W}}{\mathcal{W}'_{\mathcal{V}}}{\mathcal{O} \to \im \mathcal{W}'}
	\Say{\Big(f,[4]\Big)}{\THM{NormalAltDef}(X, \overline{\mathcal{W}} ,\mathcal{V})}
	{
		\sum f : \mathcal{O} \to X \Arrow{\TOP} [0,1] \. 
		\forall O \in \mathcal{O} \.
		f\Big(\overline{\mathcal{W}}_O\Big) = \{1\}
		\And
		f\Big(\mathcal{V}_O^\C\Big) = \{0\}
	}
	\Say{F}{\sum_{O \in \mathcal{O}} f_O}{X \Arrow{\TOP} [0,1]}
	\Say{[5]}{\bd \TYPE{OpenCover}(X,\im \mathcal{W})\ByConstr F}
	{
		\forall x \in X \. F(x) \neq 0	
	}
	\Say{\phi}{\Lambda O \in \mathcal{O} \. \frac{f_O}{F}}{\mathcal{O} \to X \Arrow{\TOP} [0,1]}
	\Say{[7]}{\ByConstr \phi \ByConstr^{-1} F \bd \TYPE{Inverse}(X \to \Reals)(F)}{
		\sum_{O \in \mathcal{O}} \phi_O = 
		\sum_{O \in \mathcal{O}} \frac{f_O}{F} =
		\frac{F}{F} = 1
	}
	\Say{[8]}{\ByConstr \phi [4] \ByConstr \mathcal{W} \bd^{-1} \supp}
	{ 
		\forall O \in \mathcal{O} \. \supp \phi_O \subset O
	}
	\Conclude{[*]}{\bd^{-1}\TYPE{PartitionOfUnity}[7][8]}
	{
		\TYPE{PartitionOfUnity}(X,\mathcal{O},\phi)
	}
	\EndProof
	\\
	\Theorem{ParacompactByPartitionOfUnity}
	{
		\forall X : \TYPE{T2} \.
		\forall [0] :
			\forall \mathcal{O} : \TYPE{OpenCover}(X) \. \NewLine \. 
			\exists \TYPE{PartitionOfUnity}(X,\mathcal{O}) \.
		\TYPE{Paracompact}(X)
	}
	\Assume{\mathcal{O}}{\TYPE{OpenCover}(X)}
	\Say{f}{[0](\mathcal{O})}{\TYPE{PartitionOfUnity}(X,\mathcal{O})}
	\Say{[1]}{\bd_3 \TYPE{PartitionOfUnity}(X,\mathcal{O},f)}{\sum_{O \in \mathcal{O}} f}
	\Say{\mathcal{V}}{\Big\{ f_O^{-1}(0,1] \Big| O \in \mathcal{O}  \}}{?\T(X)}
	\Say{[2]}{\ByConstr \mathcal{V}[1]\bd \FUNC{preimage}}{\TYPE{OpenCover}(X,\mathcal{V})}
	\Say{[3]}{\bd_2 \TYPE{PartitionOfUnity}(X,\mathcal{O},f)}
	{
		\TYPE{LocallyFinite}(X,\supp f)
	}
	\Say{[4]}{\ByConstr \mathcal{V} [3] \THM{ClosureIsSuper}}
	{
		\TYPE{LocallyFinite}(X,\mathcal{V})
	}
	\Say{[5]}{\ByConstr \mathcal{V}\TYPE{PartitionOfUnity}(X,\mathcal{O},f)}
	{
		\forall O \in \mathcal{O} \. \supp f_O \subset O
	}
	\Say{[6]}{\ByConstr \mathcal{V}[5] \THM{ClosureIsSuper}}
	{
		\forall V \in \mathcal{V} \. \exists O \in \mathcal{O} : V \subset O
	}
	\Conclude{[\mathcal{O}]}{\bd^{-1}\TYPE{Refinement}[6][2]}
	{
		\TYPE{Refinemnt}(X,\mathcal{O},\mathcal{V})
	}
	\DeriveConclude{[*]}{\bd^{-1}\TYPE{Paracompact}}{\TYPE{Paracompact}(X)}
	\EndProof
}
\Page{
	\Theorem{CompactPartitionOfUnityIsFinite}
	{
		\forall X : \TYPE{Compact} \.
		\forall \mathcal{O} : \TYPE{OpenCover}(X) \. \NewLine \. 
		\forall f : \TYPE{PartitionOfUnity}(X,\mathcal{O}) \.
		\Big|\Big\{ O \in \mathcal{O} | f_O \neq 0   \Big\}\Big| <\infty
	}
	\Say{[1]}{\bd \TYPE{PartitionOfUnity}(X,\mathcal{O},f)}
	{
		\TYPE{LocallyFinitr}(X,\supp f)
	}
	\Say{\Big(\mathcal{V},[2]\Big)}
	{
		\bd \TYPE{LocallyFinite}[1]
	}
	{
		\sum \mathcal{V}  : \TYPE{OpenCover}(X) \.
		\forall V \in \mathcal{V} \.
		\Big|\Big\{ O \in \mathcal{O} : V \cap \supp f_O \Big\}\Big| < \infty
	}
	\Say{\mathcal{V}'}{\bd \TYPE{Compact}(X)(\mathcal{V})}
	{
		\TYPE{FiniteSubcover}(X,\mathcal{V})
	}
	\Conclude{[*]}{\bd \TYPE{Finite}\Big(\mathcal{V}'\Big)[2]}
	{
		\Big|\big\{ O \in \mathcal{O} | f_0 \neq 0 \big\}\Big| < \infty
	}
	\EndProof
}
\newpage
\subsection{Proper Maps}
\Page{
	\DeclareType{ProperMap}{\prod_{X,Y \in \TOP} f : X \to Y}
	\DefineType{f}{ProperMap}{\forall K : \TYPE{CompactSubset}(Y) \.   \TYPE{CompactSubset}\Big(X,f^{-1}(K)\Big)}
	\\
	\DeclareType{DivergesToInfinity}{\prod_{X \in \TOP} ?(\Nat \to X)}
	\DefineNamedType{x}{DivergesToInfinity}{\lim_{n\to \infty} x_n = \infty}{\TYPE{ProperMap}(\Nat,X,x)}
	\\
	\Theorem{DivergenceToInfinityCriterion}
	{
		\forall X : \TYPE{T2} \And \TYPE{firstConountable} \.
		\forall x : \Nat \to X \.
		\lim_{n \to \infty} x_n 
		\iff
		\forall n : \Nat \uparrow \Nat \.
		\IsNot \TYPE{Convergent}(X,x_n)
	}
	\Assume{[1]}{\lim_{n \to \infty} x_n}
	\Assume{n}{\Nat \uparrow \Nat}
	\Assume{[2]}{\TYPE{Convergent}(X,x_n)}
	\SayIn{p}{\lim_{n\to\infty} x_n}{X}
	\SayIn{K}{\im x \cup \{p\}}
	{
		?X
	}
	\Assume{\O}{\TYPE{OpenCover}(K)}
	\Say{\Big(O,[3]\Big)}{ \Elim \TYPE{OpenCover}(K,\O)  }
	{
		\sum O \in \O \. p \in O 
	}
	\Say{\Big(M,[4]\Big)}{\Elim \TYPE{Limit}(X,x_n,p)(O)}
	{
		\sum M \in \Nat \. \forall m \in \Nat \. m \ge M \Imply x_{n_m} \in O 
	}
	\Say{\Big(\O' ,[5]\Big)}{\Elim \TYPE{Limit}(X,x_n,p)(x_{n_{[1,\ldots,M-1]}})}
	{
		\sum \O'  : \TYPE{Finite} \.
		\forall i \in [1,\ldots,M-1] \.
		\exists O' \in \O' \.
		x_{n_i} \in O'
	}
	\DeriveConclude{[\O.*]}{\Intro \TYPE{FiniteSubcober}[3][5]}
	{
		\sum \TYPE{FiniteSubcover}(X,\O,\O')
	}
	\Derive{[4]}{\Intro \TYPE{CompactSubset}}{\TYPE{CompactSubset}(X,K)}
	\Say{[5]}{\Elim K [4] }{x^{-1}(K) = \im m}
	\Say{[6]}{\Elim m [5]}{ \Big|x^{-1}(K)\Big| = \infty  }  
	\Say{[7]}{\Elim \Nat [6]}{\IsNot \TYPE{CompactSubset}\Big(\Nat,x^{-1}(K)\Big)}
	\Conclude{[1.*]}{\Intro \bot \Elim \TYPE{DevergentToInfinity}[7]}{\bot}
	\Derive{[1]}{\Intro(\Imply)\Intro(\forall)\Elim(\bot)}
	{
		\lim_{n \to \infty} x_n 
		\Imply
		\forall n : \Nat \uparrow \Nat \.
		\IsNot \TYPE{Convergent}(X,x_n)	
	}
	\Assume{[2]}
	{ 
		\forall n : \Nat \uparrow \Nat \.
		\IsNot \TYPE{Convergent}(X,x_n)		
	}
	\Assume{K}{\TYPE{CompactSubset}(X,x_n)}
	\Assume{[3]}{|K \cap \im x| = \infty}
	\Say{[4]}{
		\THM{T2CompactIsSequentiallyCompact}(K)\Elim\TYPE{SequentiallyCompact}[3]
	}
	{
		\exists n : \Nat \uparrow \Nat \.\TYPE{Convergent}(X,x_n)			
	}
	\Conclude{[3.*]}{\Intro(\bot)[3][4]}{\bot}
	\Derive{[4]}{\Elim(\bot)}{ |K \cap \im x| < \infty }
	\Conclude{[2.*]}{\Intro \TYPE{DivergesToInfinity}}
	{  
		\lim_{n \to \infty} x_n = \infty
	}
	\Derive{[2]}{\Intro(\Imply)}
	{
		\forall n : \Nat \uparrow \Nat \.
		\IsNot \TYPE{Convergent}(X,x_n)	
		\Imply
		\lim_{n \to \infty} x_n 
	}
	\Conclude{[*]}{\Elim(\iff)}
	{
		\lim_{n \to \infty} x_n 
		\iff
		\forall n : \Nat \uparrow \Nat \.
		\IsNot \TYPE{Convergent}(X,x_n)	
	}
	\EndProof
}\Page{
	\Theorem{CompositionOfProperMapsIsProper}
	{
		\forall X,Y,Z \in \TOP \.
		\forall f:\TYPE{ProperMap}(X,Y) \.
		\forall g : \TYPE{ProperMap}(Y,Z) \. \NewLine \. 
		\TYPE{ProperMap}(X,Z,fg)
	}
	\NoProof
	\\
	\Theorem{ProperMapsPreservesDivergenceToInfinity}
	{
		\NewLine ::
		\forall X,Y \in \TOP \.
		\forall f : \TYPE{ProperMap}(X,Y) \. \NewLine \. 
		\forall x : \TYPE{DivergesToInfinity}(X) \. \lim_{n \to \infty} f(x_n) 
	}
	\NoProof
	\\
	\Theorem{ProperByCompactDomain}
	{
		\forall X : \TYPE{Compact} \.
		\forall Y : \TYPE{T2} \.
		\forall X \Arrow{f} Y : \TOP \.
		\TYPE{ProperMap}(X,Y,f)
	}
	\Assume{K}{\TYPE{CompactSubset}(X)}
	\Say{[1]}{\THM{T2CompactIsClosed}(Y,K)}{ \TYPE{Closed}(Y,K)  }
	\Say{[2]}{\Elim \TOP(f)(K)[1]}{\TYPE{Closed}\Big( X, f^{-1}(K) \Big)}
	\Conclude{[*.K]}{\THM{ClosedCompactSubset}[2]}{\TYPE{CompactSubset}\Big( X, f^{-1}(K) \Big)}
	\EndProof
	\\
	\DeclareType{TotallyUnbounded}
	{
		\prod_{X,Y \in \TOP} ?\TOP(X,Y)
	}
	\DefineType{f}{TotallyUnbounded}{\forall x : \TYPE{DivergingToInfinity}(X) \. \TYPE{DivergingToInfinity}\big(f(x)\big)  }
	\\
	\Theorem{TotallyUnboundedIsProperByDomain}
	{
		\forall X : \TYPE{T2} \And \TYPE{SecondCountable} \.
		\forall Y \in \TOP \.
		\forall f : \TYPE{TotallyUnbounded} \. \NewLine
		\TYPE{ProperMap}(X,Y,f)
	}
	\Assume{K}{\TYPE{CompactSubset}(Y)}
	\Assume{[1]}{\TYPE{IsNot} \TYPE{CompactSubset}\Big(X,f^{-1}(K) \Big)}
	\Say{[2]}{\THM{SecondCountableCompactIffSequentillyCompact}(X)[1]}
	{
		\TYPE{IsNot} \TYPE{SequentiallyCompact}\Big( X, f^{-1}(K) \Big)
	}
	\Say{\Big(x,[3])}{\Elim[2]\TYPE{SequentiallyCompact}\THM{DivergenceToInfinityCriterion}}
	{\TYPE{DivergesToInfinity}\Big(f^{-1}(K)\Big)}
	\Say{[4]}{\Elim\TYPE{TotallyUnbounded}[3]\Elim \FUNC{preimage}}
	{\TYPE{DivergesToInfinity}\Big(K,f(x)\Big)}
	\Say{[5]}{\Intro \FUNC{Preimage}\big(f(x), K\big)}{ \big( f(x) \big)^{-1}(K) = \Nat}
	\Say{[6]}{\Elim \TYPE{ProperMap}(\Nat, K)[5]}{|\Nat| < \infty}
	\Conclude{[K.*]}{\Intro \bot\THM{InfiniteNaturalNumbers}[6]}{\bot}
	\DeriveConclude{[*]}{\Elim \bot \Intro \forall \Intro \TYPE{ProperMap} }
	{
		\TYPE{ProperMap}(X,Y,f)
	}
	\EndProof
}
\Page{
	\Theorem{ProperByCompactFibers}
	{
		\forall X,Y \in \TOP \.
		\forall f : \TYPE{ClosedMap}(X,Y) \. \NewLine \. 
		\forall [0] : \forall y \in Y \. 
		\TYPE{CompactSubset}\Big(X,f^{-1}(y)\Big) \. 
		\TYPE{ProperMap}(X,Y,f)
	}
	\Assume{K}{\TYPE{CompactSubset}(X)}
	\Assume{\O}{\TYPE{OpenCover}\Big(X,f^{-1}(K)\Big)}
	\AssumeIn{y}{K}
	\Say{[1]}{[0](y)}{\TYPE{ComapactSubset}\Big(X,f^{-1}(y)\Big)}
	\Say{[2]}{\THM{MonotonicPreimage}\Big(X,Y,f,K,\{y\}\Big)}
	{
		f^{-1}(y) \subset f^{-1}(K(
	}
	\Say{[3]}{\Elim \TYPE{OpenCover}[2]\Intro \TYPE{OpenCover}}
	{
		\TYPE{OpenCover}\Big( X, f^{-1}(y),/O \Big)
	}
	\Say{\Big(\O' \Big)}{\Elim \TYPE{CompactSubset}[1][3]}
	{
		\TYPE{FiniteSubCover}\Big(X,f^{-1}(y),\O)
	}
	\Say{C}{X \setminus \bigcup_{O \in \O'} O}
	{
		\TYPE{Closed}(X)	
	}
	\Say{[4]}{\Elim C \Elim \FUNC{setminus}\Elim \O' }
	{
		f^{-1}(y) \cap C = \emptyset
	}
	\Say{[5]}{\Elim \TYPE{ClosedMap}(X,Y,f)(K)}{\TYPE{Closed}(Y,f)}
	\Say{[6]}{
		\Elim \FUNC{image}\Big(f,f^{-1}(y)\Big) 
		\Elim \TYPE{Disjoint} [4] 
		\Intro \FUNC{complement}
	}
	{
		y \in f^\c(C)
	}
	\Conclude{U_y}{f^\c(C)}{\U(y)}
	\Derive{\Big( U, [1] \Big)}{\Intro\Act{\prod}}
	{
		\prod_{y \in K} \U(y) \. \TYPE{OpenCover}(Y,K,\im U) 
	}
	\Say{\mathcal{V}}{\Elim \TYPE{CompactSubset}(Y,K)(\im U)}
	{
		\TYPE{FiniteSubcover}(Y,K,\im U) 
	}
	\Say{\Big(\O', [2] \Big)}{\Elim U \Elim \mathcal{V}}
	{
		\sum \O' : \mathcal{V} \to \TYPE{Finite}(\O) \. 
		\forall V \in \mathcal{V} \. f^{-1}(V) = \bigcup_{O \in \O'} O
	}
	\Say{\O''}{\bigcup_{V \in \mathcal{V}} \O'}{?\O}
	\Say{[3]}{\THM{VeryFiniteUnion}(\mathcal{V},\O')\Intro \O''}
	{
		\TYPE{FiniteSubset}\Big(\O,\O''\Big)
	}
	\Conclude{[K.*]}{
		\Elim \TYPE{FiniteSubcover}(Y,K,\im U,\mathcal{V})
		[1][2][3]
		\Intro \TYPE{FiniteSubcover}\big(X,f^{-1}(K),\O\big)
	}
	{
		\NewLine ::
		\TYPE{FiniteSubcover}\big( X, f^{-1}(K),\O,\O'' \big)
	}
	\DeriveConclude{[*]}{
		\Intro \forall
		\Intro \TYPE{CompactSubset}
		\Intro \forall 
		\Intro \TYPE{ProperMap}
	}
	{
		\TYPE{ProperMap}(X,Y,f)
	}
	\EndProof
	\\
	\Theorem{ProperEmbedding}
	{
		\forall X,Y \in \TOP \.
		\forall f : \TYPE{TopologicalEmbedding}(X,Y,f) \.
		\forall [0] : \TYPE{Closed}\Big(Y,f(X)\Big) \.
		\NewLine \. 
		\TYPE{ProperMap}(X,Y,f)
	}
	\NoProof
}
\Page{
	\Theorem{ProperByLeftInverse}
	{
		\forall X \in \TOP \.
		\forall Y : \TYPE{T2} \.
		\forall X \Arrow{f} Y : \TOP \.
		\forall g : \TYPE{LeftInverse}{\TOP,X,Y,f} \.
		\TYPE{ProperMap}(X,Y,f)
	}
	\Say{[1]}{\Elim \TYPE{LeftInverse}(\TOP,X,Y,f,g)}
	{
		fg = \id
	}
	\Assume{K}{\TYPE{CompactSubset}(Y)}
	\Assume{x}{f^{-1}(K)}
	\Say{[3]}{\Elim \TYPE{preimage}}
	{
		f(x) \in K
	}
	\Say{[4]}{\Elim(=)\Big([1], fg(x) \Big) \Elim \id }{  fg(x) = x }
	\Conclude{[x.*]}{\Intro \FUNC{image}}{x \in g(K)}
	\Derive{[2]}{\Intro \TYPE{Subset}}{f^{-1}(K) \subset g(K)}
	\Say{[3]}{\THM{CompactImage}(K,g)}{\TYPE{CompactSubset}\Big( X, g(K) \Big)}
	\Say{[4]}{\THM{ProperByCompactDomain}[3]}
	{
		\TYPE{ProperMap}(g(K),Y,f_{g(K)})
	}
	\Say{[5]}{\Elim \TYPE{ProperMap}[4][2]}
	{
		\TYPE{CompactSubset}\Big( g(K), f^{-1}(K)  \Big)
	}
	\Derive{[K.*]}{\THM{ComapactSubsetTower}[2][5]}
	{
		\TYPE{CompactSubset}\Big( X, f^{-1}(K) \Big)
	}
	\DeriveConclude{[*]}{\Intro \TYPE{ProperMap}}
	{
		\TYPE{ProperMap}\Big( X, f^{-1}(K) \Big)
	}
	\EndProof
	\\
	\Theorem{ProperMapRestriction}
	{
		\forall X,Y \in \TOP \.
		\forall f : \TYPE{ProperMap}(X,Y) \.
		\forall A : \TYPE{Saturated}(X,Y,f) \. \NewLine \. 
		\TYPE{ProperMap}\Big(f_{|A}, A,f(A) \Big)
	}
	\NoProof
	\\
	\DeclareType{CompactlyGenerated}
	{?\TOP}
	\DefineType{X}{CompactlyGenerated}
	{
		\forall A \subset X \. 
		\Big(
			\forall K : \TYPE{CompactSubset}(X,A) \.
			\forall \TYPE{Closed}(K,A \cap K)
		\Big) 
		\Imply \NewLine
		\Imply \TYPE{Closed}(X,A)
	}
	\\
	\DeclareFunc{categoryOfCompactlyGenerated}{\CAT}
	\DefineNamedFunc{categoryOfCompactlyGenerated}{}{\CG}
	{
		(\TYPE{CompactlyGenerated}, \TOP \And \TYPE{ProperMap}, \circ, \id )
	}
}\Page{
	\Theorem{FirstCountableIsCG}
	{
		\forall X : \TYPE{FirstCountable} \. X \in \CG
	}
	\AssumeIn{A}{X}
	\Assume{[1]}{\forall K : \TYPE{CompactSubset}(X,A) \. \TYPE{Closed}(X, A \cap K) }
	\AssumeIn{x}{\overline{A}}
	\Say{\Big( a, [2] \Big)}{\THM{AltClosureDefinition}(a)}
	{
		\sum a : \Nat \to A \. \lim_{n \to \infty} a_n	= x
	}
	\Say{K}{\im a \cup \{x\}}{\TYPE{CompactSubset}(X)}
	\Say{[3]}{[1](A)}{ \TYPE{Closed}(K,A \cap K)  }
	\Conclude{[x.*]}{\THM{ClosedByLimits} [3][2] \Elim \FUNC{intersect}}
	{
		x \in A \cap K \subset A	
	}
	\DeriveConclude{[2]}{\Intro \TYPE{Subset}}
	{
		\bar A \subset A
	}
	\Say{[3]}{\Elim \FUNC{closure}(A) [2]\Intro \TYPE{SetEq}}
	{ A = \bar A }
	\Conclude{[*]}{\Elim(=)\Big( [3], \Elim \FUNC{closure}\Big)}
	{
		\TYPE{Closed}(X,A)
	}
	\DeriveConclude{[1]}{\Intro \CG}{X \in \CG}
	\EndProof
	\\
	\Theorem{LocallyCompactIsCG}
	{
		\forall X : \TYPE{LocallyCompact} \. X \in \CG
	}
	\AssumeIn{A}{X}
	\Assume{[1]}{\forall K : \TYPE{CompactSubset}(X,A) \. \TYPE{Closed}(X, A \cap K) }
	\AssumeIn{x}{\overline{A}}
	\Say{\Big(U,K,[2]\Big)}{\Elim \TYPE{LocallyCompact}(a)}
	{
		\sum U \in \U(x) \. \sum K : \TYPE{CompactSubset}(X,K) \. U \subset K
	}
	\Say{[3]}{[1](K)}{\TYPE{Closed}(K,K \cap A)}
	\Assume{V}{\U(x)}
	\Say{[4]}{\THM{ClosureAltDef}(A,x)}{U \And V \cap U \cap A \neq \emptyset   }
	\Conclude{[*.5]}{[4][2]}{K \cap  V \cap A \neq \emptyset}
	\Derive{[4]}{\THM{ClosureAltDef}[2][4][5]}{x \in A \cap K}
	\Conclude{[5]}{\Elim \FUNC{intersect}[1]}{x \in A}
	\Say{[3]}{\Elim \FUNC{closure}(A) [2]\Intro \TYPE{SetEq}}
	{ A = \bar A }
	\Conclude{[*]}{\Elim(=)\Big( [3], \Elim \FUNC{closure}\Big)}
	{
		\TYPE{Closed}(X,A)
	}
	\DeriveConclude{[1]}{\Intro \CG}{X \in \CG}
	\EndProof
}
\Page{
	\Theorem{ClosedMapLemma}
	{
		\forall X \in \TOP \.
		\forall Y \in \CG \And \TYPE{T2} \.
		\forall X \Arrow{f} Y : \TOP \. 
		\TYPE{ProperMap}(X,Y,f) \Imply \TYPE{ClosedMap}(X,Y,f)
	}
	\Assume{A}{\TYPE{Closed}(X)}
	\Assume{K}{\TYPE{CompactSubset}(Y)}
	\Say{[1]}{\THM{T2CompactIsClosed}}{\TYPE{Closed}(Y,K)}
	\Say{[2]}{\Elim \TYPE{ProperMap}(X,Y,f)(K) }
	{
		\TYPE{Comcpact}\Big( X,  f^{-1}(K) \Big) 
	}
	\Say{[2]}{\Elim \TOP(X,Y,f)(K) }
	{
		\TYPE{Closed}\Big( X,  f^{-1}(K) \Big) 
	}
	\Say{[4]}{\Elim \TOP(X)\Big( f^{-1}(K), A \Big)}
	{
		\TYPE{Closed}\Big(X, f^{-1}(K) \cap A \Big)
	}
	\Say{[5]}{\THM{ClosedSubset}\Big(X,f^{-1}(K) \Big)[4]}
	{
		\TYPE{Closed}\Big( f^{-1}(K), f^{-1}(K) \cap A \Big)
	}
	\Say{[6]}{\THM{CompactClosedSubset}[5]}
	{
		\TYPE{CompactSubset}\Big( f^{-1}(K), f^{-1}(K) \cap A \Big)
	}
	\Say{[7]}{
		\THM{ContinuousMapPreservesCompacts}
		\Big( f^{-1}(K),K,f_{|f^{-1}(K)}\Big)
		\Big( f^{-1}(K) \cap A  \Big)}
	{	
		\NewLine :
		\TYPE{CompactSubset}\Big(K, K \cap f(A) \Big)
	}
	\Conclude{[K,*]}
	{
		\THM{T2CompactIsClosed}[7]
	}
	{
		\TYPE{Closed}\Big( K, K \cap f(A) \Big)
	}
	\DeriveConclude{[A.*]}{\Elim \CG(Y)[8]}
	{
		\TYPE{Closed}\Big( Y, f(A) \Big)
	}
	\DeriveConclude{[*]}{\Intro \TYPE{ClosedMap}}
	{	
		\TYPE{ClosedMap}(X,Y,f)
	}
	\EndProof
	\\
	\Theorem{EmbeddingProperIffClosed}
	{
		\forall X \in \TOP \.
		\forall Y \in \CG \And \TYPE{T2} \.
		\forall f : \TYPE{TopologicalEmbedding}(X,Y) \. \NewLine \.
		\TYPE{ProperMap}(X,Y,f) \iff 
		\TYPE{Closed}\Big(Y,f(X)\Big)
	}
	\NoProof
	\\
	\Theorem{SurjectiveProperIsQuotientMap}
	{
		\forall X \in \TOP \.
		\forall Y \in \CG \And \TYPE{T2} \.
		\forall X \Arrow{f} Y : \TOP \. \NewLine \.
		\TYPE{ProperMap}(X,Y,f) \And \TYPE{Surjective}(X,Y,f) \iff 
		\TYPE{QuotientMap}\Big(X,Y,f\Big)
	}
	\NoProof
	\\
	\Theorem{InjectiveProperIsEmbedding}
	{
		\forall X \in \TOP \.
		\forall Y \in \CG \And \TYPE{T2} \.
		\forall X \Arrow{f} Y : \TOP \. \NewLine \. 
		\TYPE{ProperMap}(X,Y,f) \And \TYPE{Injective}(X,Y,f) \iff 
		\TYPE{TopologicalEmbedding}\Big(X,Y,f\Big)
	}
	\NoProof
	\\
	\Theorem{BijectiveProperIsHomeo}
	{
		\forall X \in \TOP \.
		\forall Y \in \CG \And \TYPE{T2} \.
		\forall X \Arrow{f} Y : \TOP \. \NewLine \.
		\TYPE{ProperMap}(X,Y,f) \And \TYPE{Bijective}(X,Y,f) \iff 
		\TYPE{Homeomorphism}\Big(X,Y,f\Big)
	}
	\NoProof
}
\newpage
\subsection{Topological Manifold}
\Page{
	\DeclareType{TopologicalManifold}
	{
		\Int_+ \times \TYPE{T2} \And \TYPE{SecondCountable} 
	}
	\DefineNamedType{(n,M)}{TopologicalManifold}{(n,M) \in \TOPM}{\forall x \in M \. \exists U \in \U(x) : U \cong_{\TOP} \Reals^n}
	\\
	\DeclareFunc{dimension}{\TOPM \to \Int_+}
	\DefineNamedFunc{dimension}{n,M}{\dim (n,M)}{n}
	\\
	\DeclareFunc{manifold}{\TOPM \to \TOP}
	\DefineNamedFunc{manifold}{n,M}{(n,M)}{M}
	\\
	\Conclude{\TYPE{CoordinateCharts} = \C\C_{(n,X)}(x)}{
		\prod n \in \Int_+ \. 
		\prod X \in \TOP \. 
		\prod x \in X \. 
		\sum U \in \U(x) \.  \NewLine \. 
		U \ToIso{\TOP}\Reals^n }{\Type}
	\\
	\Theorem{CoordinateChartsExists}{ 
		\forall M \in \TOPM \. \forall x \in M \. \C\C_M(x) \neq \emptyset 
	}
	\NoProof
	\\
	\Theorem{SubmanifoldAsOpenSubsets}
	{
		\forall M \in \TOPM \.
		\forall U \in \T(M) \.
		U \in \TOPM(\dim M) 
	}
	\NoProof
	\\
	\DeclareType{TopologicalManifoldWithBoundary}
	{
		\Int_+ \times \TYPE{T2} \And \TYPE{SecondCountable} 
	}
	\DefineNamedType{(n,M)}{TopologicalManifoldWithBoundary}
	{(n,M) \in \TOPM_\partial}{
		\NewLine |
		\forall x \in M \. 
		\Big(\exists U \in \U(x) : U \cong_{\TOP} \Reals^n\Big)
		\Big|
		\Big(\exists U \in \U(x) : U \cong_{\TOP} \Reals^n_+\Big)
	}
	\\
	\DeclareFunc{dimension}{\TOPM_\partial \to \Int_+}
	\DefineNamedFunc{dimension}{n,M}{\dim (n,M)}{n}
	\\
	\DeclareFunc{manifold}{\TOPM_\partial \to \TOP}
	\DefineNamedFunc{manifold}{n,M}{(n,M)}{M}
	\\
	\DeclareFunc{boundary}{ \prod M \in \TOPM_\partial \. \TYPE{Closed}(M)   }
	\DefineNamedFunc{boundary}{}{\boundary M}{ \{ m \in M : \C\C_M(m) = \emptyset \} }
	\\
	\DeclareFunc{interior}{ \prod M \in \TOPM_\partial \. \TOPM   }
	\DefineNamedFunc{interior}{}{\intx M}{ M \setminus \boundary M }
}
\Page{
	\Theorem{TopologicalManifoldsLocallyCompact}
	{
		\forall M \in \TOPM \cap \TOPM_\partial \.
		\TYPE{LocallyCompact}(M)
	}
	\NoProof
	\\
	\Theorem{TopologicalManifoldsParacompact}
	{
		\forall M \in \TOPM \cap \TOPM_\partial \.
		\TYPE{Paracompact}(M)
	}
	\NoProof
	\\
	\Theorem{ProductOfTopologicalManifolds}
	{
		\forall n \in \Nat \.
		\forall m : n \to \Nat \.
		\forall M : \prod^n_{i=1} \TOPM(m_i) \.
		\prod^n_{i=1} M_i \in \TOPM\left( \sum^n_{i=1} m_i \right)
	}
	\NoProof
	\\
	\Theorem{ProductOfTopologicalManifoldsWithBoundary}
	{
		\forall n \in \Nat \.
		\forall m : n \to \Nat \.
		\forall M : \prod^n_{i=1} \TOPM_\partial(m_i) \. \NewLine \. 
		\prod^n_{i=1} M_i \in \TOPM_\partial\left( \sum^n_{i=1} m_i \right)
	}
	\NoProof
	\\
	\Theorem{SumOfTopologicalManifolds}
	{
		\forall n,m \in \Nat \.
		\forall M : \prod^n_{i=1} \TOPM(m) \.
		\bigsqcup^n_{i=1} M_i \in \TOPM\left( m \right)
	}
	\NoProof
	\\
	\Theorem{ProductOfTopologicalManifoldsWithBoundary}
	{
		\forall n \in \Nat \.
		\forall m : n \to \Nat \.
		\forall M : \prod^n_{i=1} \TOPM_\partial(m_i) \.
		\NewLine \. 
		\prod^n_{i=1} M_i \in \TOPM_\partial\left( \sum^n_{i=1} m_i \right)
	}
	\NoProof
}
\Page{
	\Theorem{CompactManifoldCoordinateEmbedding}
	{
		\forall M \in \TOPM \And \HC \.
		\exists n \in \Nat :
		\exists \TYPE{HomeomorphicEmbedding}(M,\Reals^n)
	}
	\SayIn{d}{\dim M}{\Int_+}
	\Say{\Big(\mathcal{O},[1]\Big)}{\bd \TOPM(d,M)}
	{
		\sum  \mathcal{O} : \TYPE{OpenCover}(X) \. 
		\forall O \in \mathcal{O} \.
		O \cong  \Reals^d 
	}
	\Say{\mathcal{V}}{\bd \TYPE{Compact}(X)(\mathcal{O})}
	{\TYPE{FiniteSubcover}(X,\mathcal{O},\mathcal{V})}
	\SayIn{n}{d\Big(|\mathcal{V}| + 1\Big)}{\Nat}
	\Say{f}{\THM{PartitionOfUnityExist}(X,\mathcal{V})}
	{
		\TYPE{PartitionOfUnity}(X,\mathcal{V},f)
	}
	\Say{\varphi}{\bd \TYPE{Isomprphic}(\TOP,\mathcal{V},\Reals^{d})}
	{
		\prod_{V \in \mathcal{V}} V \ToIso{\TOP} \Reals^d
	}
	\Say{\mathbf{x}}{ \bigoplus_{U \in \mathcal{U}} f_U(\varphi_U \oplus 1) }
	{
		X \Arrow{\TOP} \Reals^n
	}
	\AssumeIn{p,q}{X}
	\Assume{[2]}{\mathbf{x}(p) = \mathbf{x}(q)}
	\Say{\Big(V,[3]\Big)}{\bd \TYPE{PartitionOfUnity}(X,\mathcal{V},f)(p)}
	{
		\sum V \in \mathcal{V} \. f_V(p) \neq 0
	}
	\Say{[4]}{\ByConstr \mathbf{x}[3][2]}{ f_V(q) \neq 0}
	\Say{[5]}{\bd \TYPE{PartionOfUnity}(X,\mathcal{V},f)[3][4]}
	{
		p,q \in V
	}
	\Say{[6]}{\ByConstr \mathbf{x}[2][5]}{ \varphi_V(p) = \varphi_V(q)}
	\Conclude{\Big[(p,q).*\Big]}{ \varphi^{-1}_V[6] }{p=q}
	\Derive{[2]}{\bd^{-1} \TYPE{Injective}}{\mathbf{x} : X \ToInj \Reals^n}
	\Conclude{[*]}{\THM{CompactInjectionTHM}[2]}{\TYPE{HomeomorphicEmbedding}(X,\Reals^n,\mathbf{x})}
	\EndProof
	\\
	\Theorem{FunctionByAZeroSet}
	{
		\forall M \in \TOPM \.
		\forall A : \TYPE{Closed}(M) \.
		\exists F : M \Arrow{\TOP} \Reals_{+} :
		A = F^{-1}\{0\}
	}
	\Say{d}{\dim M}{\Int_+}
	\Say{\Big(\mathcal{O},[1]\Big)}{\bd \TOPM(d,M)}
	{
		\sum  \mathcal{O} : \TYPE{OpenCover}(X) \. 
		\forall O \in \mathcal{O} \.
		O \cong  \Reals^d 
	}
	\Say{\varphi}{\bd \TYPE{Isomprphic}(\TOP,\mathcal{O},\Reals^{d})}
	{
		\prod_{O \in \mathcal{O}} O \ToIso{\TOP} \Reals^d
	}
	\Say{f}{\THM{PartitionOfUnityExist}(X,\mathcal{O})}
	{
		\TYPE{PartitionOfUnity}(X,\mathcal{O},f)
	}
	\Say{\Delta}{
		\Lambda O \in \mathcal{O} \. 
		\Lambda x \in X \. 
		\If x \in O 
		\Then \mathrm{dist}\Big(\varphi_O(x),\varphi_O(A \cap O) \Big)
		\Else 0 
	}
	{
		\mathcal{O} \to X \to  \Reals_+
	}
	\Say{F}{\sum_{O \in \mathcal{O}}  f_O \Delta_O}{X \Arrow{\TOP} \Reals_+}
	\Conclude{[*]}{\ByConstr F}{F^{-1}\{0\} = 0 }
	\EndProof
}\Page{
	\Theorem{ManifoldIsPerfectlyNormal}
	{
		\forall M \in \TOPM \.
		\forall A,B : \TYPE{Closed}(M) \.
		\forall [0] : A \cap B = \emptyset \. 
		\exists F : X \Arrow{\TOP} [0,1] \. \NewLine \.
		A = F^{-1}(0) \And B = F^{-1}(1)
	}
	\Say{\Big(f,[1]\Big)}{\THM{FunctionByZeroSet}(X,A)}
	{
		\sum f  : X \Arrow{\TOP} \Reals_+ \.  A = f^{-1}(0)
	}
	\Say{\Big(g,[2]\Big)}{\THM{FunctionByZeroSet}(X,B)}
	{
		\sum g : X \Arrow{\TOP} \Reals_+ \.  B = g^{-1}(0)
	}
	\Say{F}{\frac{f}{f + g}}{X \Arrow{\TOP} [0,1]}
	\Conclude{[*]}{\ByConstr F [0]}{F^{-1}(0) = A \And F^{-1}(1) = B}
	\EndProof
	\\
	\DeclareType{ExhaustionFunction}
	{
		\prod_{X \in \TOP} X \Arrow{\TOP} \Reals
	}
	\DefineType{f}{ExhaustionFunction}{\forall t \in \Reals \. \TYPE{Compact}(X,f^{-1}(-\infty,t))}
	\\
	\Theorem{TopologicalManifoldIsCompactlyGenerated}
	{
		\TOPM \subset \CG
	}
	\NoProof
	\\
	\Theorem{TopologicalManifoldWithBounaryIsCompactlyGenerated}
	{
		\TOPM_\partial \subset \CG
	}
	\NoProof
}\Page{
	\Theorem{ExhastionFunctionExists}{\forall M \in \TOPM \. \exists f : \TYPE{ExhastionFunction}(X) : f > 0}
	\Say{\Big(\mathcal{O},[1]\Big)}{\bd \TYPE{LocallyCompact}(M)}
	{
		\sum \mathcal{O} : \TYPE{OpenCover}(M) \.
		\forall O \in \mathcal{O} \. \TYPE{Precompact}(X,O)
	}
	\Say{\phi'}{\THM{PartitionOfUnityExist}(X,\mathcal{O})}
	{
		\TYPE{PartitionOfUnity}(X,\mathcal{O})
	}
	\Say{[2]}{\bd \TYPE{SecondCountable}(X) \THM{BaseEquivalence}(X,\mathcal{O})}
	{ |\mathcal{O}| \le \aleph_0 }
	\Say{O}{\Func{enumerate}(\mathcal{O})}{\Nat \ToBij \mathcal{O} }
	\Say{\phi}{\phi_O'}{\Nat \to X \Arrow{\TOP} [0,1]}
	\Say{f}{\sum^\infty_{n=1} n \phi_n}{X \Arrow{\TOP} \Reals_{++}}
	\Say{[3]}{\ByConstr f \bd \TYPE{PartitionOfUnity}(X,\mathcal{O},\varphi)}
	{
		f \ge 1
	}
	\AssumeIn{n}{\Nat}
	\AssumeIn{x}{X}
	\Assume{[4]}{f(x) \le n}
	\Assume{[5]}{\forall k \in n \. f(x) \not \in O_k}
	\Say{[6]}{\ByConstr \phi \bd^2 \TYPE{PartitionOfUinity}(X,\mathcal{O},\phi)[5]}
	{
		1 = \sum^n_{k=1} \phi_k(x) = \sum^n_{k=n+1} \phi_k(x) 
	}
	\Say{[7]}
	{
		\ByConstr f
		\ByConstr \phi \bd \TYPE{PartitionOfUinity}(X,\mathcal{O},\phi)[5]
		\THM{PosMultIneq}(\Reals)\THM{SumIneq}(\Reals)
		\bd \RING(\Reals)
		[6]
	}
	{
		f(x) = 
		\sum^\infty_{k=1} k\phi_k(x) =
		\NewLine =
		\sum^\infty_{k=n+1} k\phi_k(x) \ge
		\sum^\infty_{k=n+1}(n+1)\phi_k(x) = 
		(n+1) \sum^\infty_{k=n+1} \phi_k(x) =
		(n+1)
	}
	\Conclude{[5.*]}{[4][7]}{\bot}
	\Derive{[6]}{\Elim(\bot)\bd^{-1}\TYPE{Union}}{x \in \bigcup^n_{i=1} U_i}
	\Conclude{[n.*]}{\THM{ClosureIsSuper}}{x \in \bigcup^n_{i=1} \overline{O}_i}
	\Derive{[4]}{\Intro(\forall)\bd^{-1} \FUNC{preimage}(f) \Intro(\forall)\Intro(\Imply)}
	{
		\forall n \in \Nat \. f^{-1}(0,n] \subset \bigcup^n_{i=1} \overline{O}_i
	}
	\Assume{t}{\Reals_+}
	\Say{\Big(n,[5]\Big)}{\bd \TYPE{Archimedean}(\Reals,n)}
	{
		\sum n \in \Nat \. n \ge t
	}
	\Say{[6]}{\THM{monotonicPreimage}[4](t)}
	{
		f^{-1}(0,t] \subset f^{-1}(0,n] \subset \bigcup^n_{i=1} \overline{O}_i
	}
	\Conclude{[t.*]}{\THM{ClosedSubsetIsCompact}[6]}
	{
		\TYPE{Compact}\Big(X,f^{-1}(0,t]\Big)
	}
	\DeriveConclude{[*]}{\bd^{-1}\TYPE{ExhuastionFunction}}
	{
		\TYPE{ExhuastionFunction}(X,f)
	}
	\EndProof
}
\newpage
\section{Cell Complexes}
\subsection{Cell Structure}
\Page{
	\DeclareType{Cell}{\Int_+ \to ?\TOP}
	\DefineType{B}{Cell}{\Lambda n \in \Int_+ \. B \cong_{\TOP} \mathbb{B}^n}
	\\
	\DeclareType{ClosedCell}{\Int_+ \to ?\TOP}
	\DefineType{B}{ClosedCell}{\Lambda n \in \Int_+ \. B \cong_{\TOP} \mathbb{D}^n}
	\\
	\Theorem{CompactConvexBodyIsClosedCell}
	{
		\forall n \in \Nat \. 
		\forall C : \TYPE{Compact} \And \CB(\Reals^n) \.
		\TYPE{ClosedCell}(C)
	}
	\Say{[1]}{\bd \CB(\Reals^n,C)}{\intx C \neq \emptyset}
	\SayIn{O}{\bd \TYPE{Nonempty}[1]}{\intx C}
	\Say{[2]}{\THM{HeineBorelTHM}(\Reals^n,C)}{\TYPE{Bounded} \And \TYPE{Closed}(\Reals^n,C)}
	\Say{[2']}{\THM{ConvexBodyInteriorIsCore}(\Reals^n,C)(c)}{c \in \core C}
	\AssumeIn{c}{\boundary C}
	\SayIn{v}{c-O}{\Reals^n}
	\Say{[3]}{\ByConstr v \bd O \bd c \bd \boundary C}{v \neq 0}
	\Say{R}{\{ O + tv | t \in \Reals_+  \}}{?\Reals^n}
	\Say{[4]}{\ByConstr R}{\TYPE{LinearlyConnected}(\Reals^n,R)}
	\Say{[5]}{\bd \TYPE{Bounded}(\Reals^n,C)[2] \ByConstr R}{R \cap C^\c \neq \emptyset}
	\Say{[6]}{\THM{ClosedConectedIntersection}(\Reals^n,C,R)[3,5,6]}
	{
		R \cap \boundary C \neq \emptyset
	}
	\AssumeIn{x,y}{R \cap \boundary C}
	\Assume{[7]}{x \neq y}
	\Say{[8]}{\ByConstr R [7][3]}{ x \in (c,y) | y \in (c,x)}
	\Say{[9]}{\bd \TYPE{Closed} \And \TYPE{Convex}(\Reals^n,C) \bd (x,y)}{x,y \in \lin C}
	\Say{[10]}{\THM{ConvexInteriorInCore}(\Reals^n,C)[2',9][8]}{ x \in \core C | y \in \core C}
	\Say{[11]}{\THM{ConvexBodyInteriorIsCore}[10]}{ x \in \intx C | y \in \intx C  }
	\Conclude{\Big[(x,y).*\Big]}{\bd \boundary C \bd (x,y) [11]}{\bot}
	\Say{[7]}{\Elim(\bot)[6]}{\TYPE{Singleton}(R \cap \boundary C )}
	\Say{[8]}{\bd \TYPE{Singleton} [7] \ByConstr R \bd c }
	{
		 R \cap \boundary C = \{c\}
	}
	\Conclude{f(c)}{\frac{v}{\|v\|}}{\mathbb{D}^n}
	\Derive{f}{\Intro(\to)}{\boundary C \Arrow{\TOP} \mathbb{D}^n}
	\Say{[2]}{\ByConstr f \bd O}{ \Big(f : \boundary C \ToBij \boundary \mathbb{D}^n\Big)}
	\Say{[3]}{\THM{CompactOpenMappingTHM}[2]}{\Big( f : \boundary C \ToIso{\TOP} \boundary \mathbb{D}^n \Big)   }
	\Say{\varphi}{\Lambda v \in \mathbb{D}^n \. \If v == 0 \Then O \Else O + \|v\| f^{-1}\left( \frac{v}{\|v\|} \right)}
	{
		\mathbb{D}^n \Arrow{\TOP} C
	}
	\Say{[4]}{\ByConstr \varphi \bd O}{ \Big(\varphi :  C \ToBij \mathbb{D}^n\Big)}
	\Say{[5]}{\THM{CompactOpenMappingTHM}[4]}{\Big( \varphi : C \ToIso{\TOP}  \mathbb{D}^n \Big)   }
	\Conclude{[*]}{\bd^{-1} \TYPE{Isomorphic}(\TOP)[5]}{C \cong_{\TOP} \mathbb{D}^n}
	\EndProof
}
\Page{
	\DeclareType{CellDecomposition}
	{
		\prod X : \TOP \. 
		\sum \mathcal{E} : \prod^\infty_{n=0} ?\Big(?X \And \TYPE{Cell}(n) \Big) \.
		\prod^\infty_{n=1} \prod_{E \in \mathcal{E}_n} \mathbb{D}^n \Arrow{\TOP} X
	}
	\DefineType{(\mathcal{E},\varphi)}{CellDecomposition}
	{
			X = \bigsqcup^\infty_{n=0} \bigcup_{E \in \mathcal{E}} E 
		\And \NewLine \And
			\forall n \in \Nat \. 
			\forall E \in \mathcal{E}_n \. 
			\varphi_{n,E|\mathbb{B}^n} : \mathbb{B}^n \ToIso{\TOP} E 
		\And \NewLine \And
			\forall n \in \Nat \. 
			\forall E \in \mathcal{E}_n \. 
			\exists \C : \prod^{n-1}_{i=1} ? \mathcal{E}_i : 
			\boundary C = \bigcup_{i=0}^{n-1} \bigcup_{C \in \C_i } C
	}
	\\
	\Conclude{\TYPE{CellComplex}}{\sum X : \TYPE{T2} \. \TYPE{CellDecomposition}(X)}{\Type}
	\\
	\DeclareType{FiniteCellComplex}{?\TYPE{CellComplex}}
	\DefineType{(X,\E,\varphi)}{\TYPE{FiniteCellComplex}}{\left| \bigsqcup^\infty_{n=0} \E_n \right| < \infty}
	\\
	\DeclareType{LocallyFiniteCellComplex}{?\TYPE{CellComplex}}
	\DefineType{(X,\E,\varphi)}{\TYPE{FiniteCellComplex}}{\TYPE{LocallyFinite}\left( \bigsqcup^\infty_{n=0} \E_n \right)}
	\\
	\DeclareType{Coherent}{\prod_{X \in \TOP} ?\TYPE{Cover}(X)}
	\DefineType{\C}{Coherent}{
		\forall A \subset X \. 
		\Big( \forall C \in \C \. A \cap C \in \T(C) \Big)
		\Imply
		A \in \T(X) 
	}
	\\
	\Theorem{CoherentContinuity}{ 
		\forall X,Y \in \TOP \.
		\forall \C : \TYPE{Coherent}(X) \.
		\forall f : X \to Y \.
		f \in C(X,Y) 
		\iff \NewLine \iff
		\forall D \in \C \.
		f_{|D} \in C(D,Y) 
	}
	\NoProof
	\\
	\Theorem{CoherentQuotient}{ 
		\forall X \in \TOP \.
		\forall \C : \TYPE{Coherent}(X) \.
		\TYPE{QuotientMap}\left(\bigsqcup_{C \in \C} \iota_{C}\right)
	}
	\NoProof
	\\
	\DeclareType{CWComplex}
	{
		?\TYPE{CellComplex} 
	}
	\DefineType{(X,\E,\varphi)}{\TYPE{CWComplex}}
	{
		\TYPE{Coherent}(\overline{\E}) \And \NewLine
		\forall n \in \Int_+ \. 
		\forall E \in \E_n \.
		\exists F : \TYPE{Finite}\left( \bigcup^\infty_{n=1} \E_n \right) :
		\overline{E} \subset \bigcup_{f \in F} f
	}
}
\Page{
	\DeclareFunc{cellSet}{\TYPE{CellComplex} \to \SET}
	\DefineNamedFunc{cellSet}{(X,\E,\varphi)}{\E}{\bigsqcup^\infty_{n=0} \E_n}
	\\
	\DeclareFunc{cellDimension}
	{
		\prod (X,\E,\phi): \TYPE{CellComplex} \. \E \to \Int_+
	}
	\DefineNamedFunc{cellDimension}{E}{\dim E}{ \bd \TYPE{Singleton}\{ n \in \Nat : E \in \E_n \} }
	\\
	\Theorem{LocallyFiniteIsCW}{
		\forall (X,\E,\varphi) : \TYPE{LocallyFiniteCellComplex} \.
		\TYPE{CWComplex}(X,\E,\varphi)
	}
	\AssumeIn{A}{?X}
	\Say{\Big([U],[1] \Big)}{\bd \TYPE{LocallyFiniteCellComplex}(X,\E,\varphi)(A)}
	{
		\NewLine : 
		\sum U : \prod_{a \in A} \U(a) \.
		\forall a \in A \.
		\Big| \big\{ e \in \E : \overline{e} \cap U_a \neq \emptyset \big\} \Big| < \infty
	}
	\Assume{[2]}{\forall n \in \Nat \. \forall e \in E \. \overline{e} \cap A \in \T(\overline{e}) }
	\Say{[3]}{\bd U \bd \FUNC{subsetTopology}}{ 
		\forall a \in A \. 
		\forall e \in \E \.  
		U_a \cap \overline{e} \in \T(\overline{e})   }
	\AssumeIn{a}{A}
	\Say{\Big(n,e,[4]\Big)}{
		\FUNC{enumerate} \bd \TYPE{Finite} [2] U_a
	}
	{
		\sum n \in \Nat \.  \sum e : n \to \E \.
		e_n =  \{ e \in \E : \overline{e} \cap U_a \neq \emptyset   \}
	}
	\Say{[5]}{\bd^{-1} \TYPE{Closed}[3]\THM{ClosedSubetLemma}(X)}
	{
		\forall i \in n \. 
		\TYPE{Closed}(X,\overline{e}_i \cap A^\C) 
	}
	\Say{[6]}{\bd \TYPE{Intersection} \bd^{-1} \TYPE{SetMinus}[4]}{A \cap U_a = U_a \setminus \bigcup^n_{i=1} \overline{e_i} \cap A^\C  }
	\Conclude{[a.*]}{\THM{ClosedFiniteUnion} \And \THM{OpenClosedDiff}[6]}{ A \cap U_a \in \T(X)}
	\DeriveConclude{[A.*]}{\THM{OpenCoverLemma}}{A \in \T(X)}
	\Derive{[1]}{\bd^{-1} \TYPE{Coherent}}{\TYPE{Coherent}{X,\overline{E}}}
	\AssumeIn{E}{\E}
	\Say{\Big([U],[2] \Big)}{\bd \TYPE{LocallyFiniteCellComplex}(X,\E,\varphi)(\overline{E})}
	{
		\NewLine : 
		\sum U : \prod_{a \in A} \U(a) \.
		\forall a \in \overline{E} \.
		\Big| \big\{ e \in \E : \overline{e} \cap U_a \neq \emptyset \big\} \Big| < \infty
	}
	\SayIn{n}{\dim E}{\Int_+}
	\Say{[3]}{\THM{CompactMappingTHM}(\mathbb{D}^n,X,\varphi_{n,E})\bd \TYPE{CellComplex}(X,\E,\varphi_n)}
	{
		\TYPE{Compact}(X,\overline{E})
	}
	\Say{\Big(m,a,[4]\Big)}{\bd \TYPE{Compact}(X,\overline{E})(U)}
	{
		\sum m \in \Nat \. \sum a : m \to \overline{E} \. \overline{E} : \subset \bigcup^m_{i=1} U_{a_i}
	}
	\Conclude{[E.*]}{\THM{FiniteSumOfFinite}[2][4]}{\exists \F : \TYPE{Finite}(\E) : \overline{E} \subset \bigcup_{f \in F} f }
	\DeriveConclude{[*]}{[1]\bd^{-1} \TYPE{CWComplex}}{\TYPE{CWComplex}(X,\E,\varphi)}
	\EndProof
	\\
	\DeclareFunc{complexDimension}{\TYPE{CellComplex} \to \aleph_1}
	\DefineNamedFunc{complexDimension}{X,\E,\varphi}{\dim \E}{\sup \{ \dim e | e \in \E \}}
}\Page{
	\Theorem{OpenCellTHM}
	{
		\forall (X,\E,\varphi) : \TYPE{CWComplex} \.
		\forall n \in \Nat \.
		\forall [0] : \dim \E = n \.
		\forall e \in \E_n \.
		e \in \T(X)
	}
	\Say{[1]}{\bd \TYPE{CWComplex}(X,\E,\varphi)\bd \TYPE{CoherentIQuotientMap}(\varphi_{n,e})}
	{
		\TYPE{QuotientMap}(\mathbb{D}^n,\overline{e},\varphi_{n,e})
	}
	\Say{[2]}{\bd \THM{QuotientMapIsOpen}(\mathbb{D}^n,\overline{e},\varphi_{n,e})(\mathbb{B}^n) \bd \TYPE{CellComplex}(X,\E,\varphi)}
	{ 
		\varphi_{e,n}(\mathbb{B}^n) = e \in \T(\overline{e})
	}
	\AssumeIn{f}{\E}
	\Assume{[3]}{e \neq f}
	\Say{[4]}{\bd \TYPE{CellComplex}(X,\E,\varphi) \bd \TYPE{Partition}(X,\E)(e,f)[3]}
	{
		e \cap f = \emptyset
	}
	\Say{[5]}{\THM{MonotonicClosure}[4]\bd^{-1}\FUNC{boundary}}{e \cap \overline{f} \subset {\boundary}_{\overline{f}} f}
	\Say{\Big(\F, [6]\Big)}{\bd \TYPE{CWComplex}(X,\E,\varphi)(f)}
	{
		\sum \F : \prod^{(\dim f)-1}_{i=1} \TYPE{Finite}(\E_i) \. 
		{\boundary}_{\overline{f} } f = \bigsqcup^{(\dim f)-1 }_{i=1}\bigsqcup_{F \in \F_i} F
	}
	\Say{[7]}{\THM{NaturalNegation}\bd \dim \E [0]}{(\dim f) - 1 < \dim f \le n}
	\Say{[8]}{\bd \TYPE{CellComplex}[7]}{\forall F \in \F \. e \cap F = \emptyset}
	\Conclude{[f.*]}{[8][5]\bd \TYPE{Topology}\Big( \T \Big(\overline{f}\Big)\Big)}{e \cap \overline{f} = \emptyset \in \T(\overline{f})} 
	\DeriveConclude{[*]}{\bd \TYPE{CWComplex}(X,\E, \varphi)\bd \TYPE{Coherent}(X,\overline{\E})}
	{
		e \in \T(X)
	}
	\EndProof
	\\
	\DeclareType{Subcomplex}{\TYPE{CellComplex} \to ?\TYPE{CellComlex}}
	\DefineNamedType{(Y,\F,\psi)}{Subcomplex}{
		\Lambda (X,\E,\varphi) : \TYPE{CellComplex} \. (Y,\F,\psi) \subset (X,\E,\varphi)}
	{
		\NewLine \iff
		Y \subset X \And \F \subset \E \And \forall n \in \Int_+ \. \forall e \in \F_n \. 
		\psi_{n,e} = \varphi_{n,e}
	}
	\\
	\DeclareFunc{nSkeleton}{\prod (X,\E,\varphi) : \TYPE{CellComplex} \. \Int_+ \to \TYPE{Subcomplex}(X,\E,\varphi)}
	\DefineNamedFunc{nSkeleton}{n}{(X^{\skull n},\E^{\skull n},\varphi^{\skull n})}
	{\left( \bigsqcup^n_{i=1} \bigsqcup_{e \in \E_n} e, \E_{|\{0,\ldots,n\}}, \varphi_{|\{0,\ldots,n\}}  \right)} 
	\\
	\Theorem{ClosedSubcomplex}
	{
		\forall (X,\E,\varphi) : \TYPE{Subcomplex} \. 
		\forall (Y,\F,\psi) \subset (X,\E,\varphi) \.
		\TYPE{Closed}(Y,X) 
	}
	\AssumeIn{e}{\E}
	\Assume{[1]}{e \in \F}
	\Say{[2]}{\bd \TYPE{Subcomplex}\left( (X,\E,\varphi),(Y,\F,\psi) \right)[1]}
	{
		\overline{e} \cap Y = \overline{e}
	}
	\Conclude{[1.*]}{\bd \TYPE{Closed}(\overline{e})}
	{
		\TYPE{Closed}(\overline{e},\overline{e} \cap Y)
	}
	\Derive{[1]}{\Intro(\Imply)}
	{
		e  \in \F \Imply \TYPE{Closed}(\overline{e},\overline{e} \cap Y)
	}
	\Assume{[2]}{e \not \in \F}
	\Say{[3]}{\bd \TYPE{Subcomplex}\left( (X,\E,\varphi),(Y,\F,\psi)\right)}
	{
		\overline{e} \cap Y \subset {\boundary}_{\overline{e}} e
	}
	\Say{\Big(\E', [4]\Big)}{\bd \TYPE{CWComplex}(X,\E,\varphi)(e)}
	{
		\sum \E' : \prod^{(\dim e)-1}_{i=1} \TYPE{Finite}(\E_i) \. 
		{\boundary}_{\overline{e} } e = \bigsqcup^{(\dim e)-1 }_{i=1}\bigsqcup_{E \in \E_i'} E
	}
	\Say{\F'}{\F \cap \E}{\prod^{(\dim e) -1}_{i=1} \TYPE{Finite}(\F_i)}
	\Say{[5]}{[4][3]\bd \TYPE{CellComplex}(Y,\F,\psi)}
	{
		\overline{e} \cap Y = \bigsqcup^{(\dim e)-1}_{i=1} \bigcup_{F \in \F'_i} \overline{F}
	}
	\Conclude{[2.*]}{\THM{FiniteClosedUnion}}{\TYPE{Closed}\Big(\overline{e},\overline{e} \cap Y\Big)}
	\Derive{[2]}{\Intro(\Imply)}
	{
		e \not \in \F \Imply \TYPE{Closed}(\overline{e},\overline{e} \cap Y)
	}
}\Page{
	\Conclude{[e.*]}{\Elim(|)\LOGIC(LEM)(e \in \F)[2]}
	{
		\TYPE{Closed}(\overline{e},\overline{e \cap Y}) 
	}
	\DeriveConclude{[1]}{\bd \TYPE{CWComplex}(X,\E,\varphi)}
	{
		\TYPE{Closed}(X,Y)
	}
	\EndProof
	\\
	\Theorem{CWSubcomplex}
	{
		\forall (X,\E,\varphi) : \TYPE{Subcomplex} \. 
		\forall (Y,\F,\psi) \subset (X,\E,\varphi) \.
		\TYPE{CWComplex}(Y,\F,\psi) 
	}
	\\
	\Theorem{SkeletonCoherent}
	{
		\forall (X,\E,\varphi) : \TYPE{CWComplex} \.
		\TYPE{Coherent}\Big(X, \Big\{  X^{\skull n}  | n \in \Int_+    \Big\} \Big)
	}
	\Assume{A}{?X}
	\Assume{[1]}{\forall n \in \Int_+ \. X^{\skull n} \cap A \in \T\Big( X^{\skull n}\Big)}
	\Assume{e}{\E}
	\SayIn{n}{\dim e}{\Int_+}
	\Say{[2]}{[1](n)}
	{   
		X^{\skull n} \cap A \in \T\Big(X^{\skull n}\Big)
	}
	\Say{[3]}{\bd X^{\skull n} \bd \TYPE{SubComplex}\bigg( (X,\E,\varphi),\Big( X^{\skull n},\E^{\skull n},\varphi^{\skull n} \Big)\bigg)}
	{
		\overline{e} \cap X^{\skull n} \cap A = \overline{e} \cap A
	}
	\Conclude{[e.*]}{\bd \TYPE{SubsetTopology}[3]}{\overline{e} \cap A \in \T(\overline{e})}
	\DeriveConclude{[A.*]}{\bd \TYPE{CWComplex}(X,\E,\varphi)}
	{
		A \in \T(X)
	}
	\Conclude{{*}}{\bd^{-1} \TYPE{Coherent}}
	{ \LOGIC{This} }
	\EndProof
	\\
	\DeclareType{
		RegularCell
	}
	{
		\prod (X,\E,\varphi) : \TYPE{CellComplex} \.
		?\E
	}
	\DefineType{e}{RegularCell}{ \Big(\varphi_{n,e} : \mathbb{D}^n \ToIso{\TOP} \overline{e} \Big) \quad \where \quad n = \dim e  }
	\\
	\DeclareType{
		RegularComplex
	}
	{
		? \TYPE{CellComplex} 
	}
	\DefineType{(X,\E,\varphi)}{RegularComplex}{ \forall e \in \E \. \TYPE{RegularCell}(X,\E,\varphi,e) }
	\\
	\DeclareType{FiniteDimensionalComplex}
	{
		?\TYPE{CellComplex}
	}
	\DefineType{(X,\E,\varphi)}{FiniteDimensionalComplex}
	{
		\dim \E < \infty
	}
}
\newpage
\subsection{Topological Properties}
\Page{
	\Theorem{ConnectedHasConnectedSkeleton}
	{
		\forall (X,\E,\varphi) : \TYPE{CWComplex} \.
		\TYPE{Connected}(X) \Imply \TYPE{Connected}(X^\skull)
	}
	\Say{[1]}{\THM{ConnectedSpheres}}{\forall n \in \Nat \, \TYPE{Connected}\Big(\mathbb{S}^n\Big)}
	\Say{[2]}{\THM{ConnectedImage}[2]}{
		\forall n \in \Nat \. 
		n > 1 \Imply 
		\forall e \in \E_n \. 
		\TYPE{Connected}( \boundary e )
	}
	\AssumeIn{n}{\Nat}
	\Assume{[3]}{!\TYPE{Connected}(X^{\skull n})}
	\Say{\Big(A,B,[3.1]\Big)}{\bd \TYPE{Connected}[3]\bd^{-1} \TYPE{ConnectedComponents}}
	{\sum A,B \in \mathrm{CC}(X^{\skull n}) \. A \neq B}
	\Say{\A}{\bigcup \{ e \in E : \overline{e} \cap A \neq \emptyset \And \dim e \le n +1  \}}{?X}
	\Say{\B}{\bigcup \{ e \in E : \overline{e} \cap B \neq \emptyset \And \dim e \le n+1 \}}{?X}
	\Say{[4]}{\bd \mathrm{CC}(X^{\skull n })[3.1][2]}{\A \neq \B}
	\AssumeIn{e}{\E}
	\Say{[5]}{\ByConstr \A}{  \overline{e} \cap \A = \emptyset \Big| \overline{e} \cap \A = \overline{e} }
	\Say{[6]}{\ByConstr \B}{  \overline{e} \cap \B = \emptyset \Big| \overline{e} \cap \B = \overline{e} }
	\Conclude{[e.*]}{\bd \TYPE{Topology}[5][6]}
	{
		\TYPE{Clopen}(\overline{e},\A \cap \overline{e} \And \B \cap \overline{e} ) 
	}
	\Derive{[5]}{\bd \TYPE{CWComplex}(X,\E,\varphi)}
	{
		\TYPE{Clopen}(X,\A \And \B)
	}
	\Conclude{[3.*]}{ \bd \TYPE{Connected}[5] }{ \IsNot \TYPE{Connected}(X^{\skull (n+1)})  }
	\Derive{[3]}{\Elim(\bot) \mathrm{CC}(X^\skull)}
	{
		\forall n \in \Nat \.
		\IsNot \TYPE{Connected}(X^{\skull n}) 
		\Imply
		\IsNot \TYPE{Connected}(X^{\skull(n+1)})
	}
	\Assume{[4]}
	{ \IsNot \TYPE{Connected}(X^\skull)  }
	\Say{[5]}{\bd \Nat [4][3]}{\forall n \in \Nat \. \IsNot \TYPE{Connected}(X^{\skull n})}
	\Say{\Big( (A,\A,\psi), (B,\B,\psi') ,[6]\Big)}{\bd \TYPE{CellComplex}(X,\E,\varphi)[5]}
	{
		\sum (A,\A,\psi),(B,\B,\psi') : \TYPE{Subcomplex}(X,\E,\varphi) \.
		\NewLine \. 
		A \sqcup B = X \And  
		\forall n \in \Nat \. \TYPE{Clopen}\Big(X^{\skull n},A^{\skull n} \And B^{\skull n}\Big) 
	}
	\Say{[7]}{\bd^{-1} \TYPE{Clopen}[6]}{\TYPE{Clopen}(X,\A \And \B)}	
	\Conclude{[4.*]}{\bd \TYPE{Connected}(X)[7]}{\bot}
	\DeriveConclude{[*]}{\Elim(\bot)}
	{
		\TYPE{Connected}(X^{\skull  })
	}
	\EndProof
	\\
	\Theorem{PathConnectedComponentIsClopen}
	{
		\forall (X,\E,\varphi) : \TYPE{CellComplex} \. 
		\forall A \in \mathrm{PCC}(X) \.
		\TYPE{Clopen}(X,A)
	}
	\Say{[0]}{\THM{PathConnectedImage}}
	{
		\forall e \in \E \. \TYPE{PathConnected}(\overline{e})
	}
	\Assume{e}{\E}
	\Say{[1]}{[0](e) \bd \mathrm{PCC}(X)}
	{
		\overline{e} \cap A = \overline{e} 
		\Big|
		\overline{e} \cap A = \emptyset
	}
	\Conclude{[e.*]}
	{
		\bd \TYPE{Topology}(\overline{e})	
	}
	{
		\TYPE{Clopen}{\overline{e},\overline{e} \cap A}
	}
	\DeriveConclude{[*]}{\bd^{-1} \TYPE{CWComplex}(X,\E,\varphi)}
	{
		\TYPE{Clopen}(X,A)
	}
	\EndProof
}\Page{
	\Theorem{ConnectedIffPathConnected}
	{
		\forall (X,\E,\varphi) : \TYPE{CWComplex} \.
		\TYPE{Connected}(X)
		\iff
		\TYPE{PathConnected}(X)
	}
	\NoProof
	\\
	\Theorem{ConnectedSkeletonImplyPathConnected}
	{
		\forall (X,\E,\varphi) : \TYPE{CWComplex} \.
		\forall n \in \Int_+ \. \NewLine \. 
		\TYPE{Connected}(X^{\skull n}) \Imply
		\TYPE{PathConnected}(X)
	}
	\Say{[1]}{\THM{ConnectedHasConnectedSkeleton}(X^{\skull n},\E^{\skull n},\varphi^{\skull n})}
	{
		\TYPE{Connected}(X^\skull)
	}
	\Say{[2]}{\THM{ConnectedIffPathConnected}(X^\skull,\E^\skull,\varphi^\skull)}
	{
		\TYPE{PathConnected}(X^\skull)
	}
	\Assume{k}{\Nat}
	\Assume{[3]}{\TYPE{PathConnected}(X^{\skull k})}
	\AssumeIn{p,q}{X^{\skull (k+1)}}
	\Say{\Big(e,f,[4]\Big)}{\bd \TYPE{CellComplex}(X^{\skull(k+1)},\E^{\skull(K+1)},\varphi^{\skull(k+1)})}
	{ \sum e,f \in \E^{\skull(k+1)} \. p \in e \And q \in f }
	\Assume{[5]}{\dim e \le k \And \dim f \le k}
	\Say{[6]}{\bd \FUNC{nskeleton} [4]}{p,q \in X^{\skull k}}
	\Conclude{[5.*]}{\bd \TYPE{PathConnected}\Big(X^{\skull k}\Big)}
	{
		\exists \TYPE{Path}\Big(X^{\skull k}, p,q \Big)
	}
	\Derive{[5]}{\Imply}
	{
		(\dim e \le k \And \dim f \le k)
		\Imply
		\exists \TYPE{Path}\Big(X^{\skull(k+1)}, p,q \Big) 	
	}
	\Assume{[6]}{\dim f = (k+1) \Big| \dim e = (k+1)}
	\Say{\Big(f',e',[7]\Big)}{\bd \TYPE{CellComplex}\Big( X^{\skull (k+1)},\E^{\skull (k+1)},\varphi^{\skull (k+1)}\Big)}
	{
		\sum f',e' \in \E^{\skull k} \.
		f' \subset \boundary f \And
		e' \subset \boundary e'
	}
	\SayIn{p'}{\bd \TYPE{NonEmpty}(e')}{e'}
	\SayIn{q'}{\bd \TYPE{NonEmpty}(f')}{f'}
	\Say{\Big(\alpha,[8]\Big)}{\THM{DiskIsPathConnected}(\overline{e})(p,p')}{
		\sum \alpha \in C\Big( [0,1], \overline{e} \Big) \.
		\alpha(0) = p \And \alpha(1) = p' 
	}
	\Say{\Big(\beta,[9]\Big)}{\THM{DiskIsPathConnected}(\overline{f})(q',q)}{
		\sum \beta \in C\Big( [0,1], \overline{f} \Big) \.
		\alpha(0) = q' \And \alpha(1) = q
	}
	\Say{\Big(\omega,[10]\Big)}{\bd \TYPE{PathConnected}(X^{\skull k})(p',q')}
	{
	  \sum  \omega \in C\Big( [0,1], X^{\skull k} \Big) \.
	  \omega(0) = p' \And \omega(1) = q'
	}
	\Conclude{[6.*]}{\bd^{-1}\TYPE{Path}[8,9,10]}{
		\TYPE{Path}\Big( X^{\skull(k+1)}, p, q, \alpha \oplus \omega \oplus \beta \Big)
	}
	\Derive{[6]}{\Imply}
	{
		(\dim e = (k+1) \Big| \dim f \le k)
		\Imply
		\exists \TYPE{Path}\Big(X^{\skull(k+1)}, p,q \Big) 	
	}
	\Conclude{\Big[(p,q).*\Big]}{\Elim(|)\LOGIC{LEM}(\ldots)[5][6]}
	{
		\exists \TYPE{Path}\Big( X^{\skull(k+1)},p,q,\Big)
	}
	\DeriveConclude{[3.*]}{\bd^{-1} \TYPE{PathConnected}}
	{
		\TYPE{PathConnected}\Big(X^{\skull(k+1)} \Big)
	}
	\Derive{[3]}{\bd \Nat [2]}
	{
		\forall  k \in \Nat \. \TYPE{PathConnected}(X^{\skull k})
	}
	\AssumeIn{p,q}{X}
	\Say{\Big(e,f,[4]\Big)}{\bd \TYPE{CellComplex}(X,\E,\varphi)}
	{ \sum e,f \in \E \. p \in e \And q \in f }
	\SayIn{m}{\max(\dim e,\dim f,1)}{\Nat}
	\Conclude{\Big[(p,q).*\Big]}
	{ 
		[3](m)\bd \TYPE{PathConnected}(p,q)
	}
	{
		\exists \TYPE{Path}(X^{\skull m},p,q)
	}
	\DeriveConclude{[*]}{\bd^{-1} \TYPE{PathConnected}}
	{
		\TYPE{PathConnected}(X)
	}
	\EndProof
}\Page{
	\Theorem{FiniteSubcomplexLemma}
	{
		\forall (X,\E,\varphi) : \TYPE{CWComplex} \.
		\forall e \in \E \.
		\exists (Y,\F,\psi) \subset (X,\E,\varphi) \.
		\NewLine \.
		\TYPE{FiniteCellComplex}(Y,\F,\psi) \And
		\overline{e} \subset Y
	}
	\Say{\aries}
	{
		\Lambda n \in \Int_+ \.
		\forall e \in \E \.
		(\dim e \le n) \Imply
		\exists (Y,\F,\psi) \subset (X,\E,\varphi) \.
		\TYPE{FiniteCellComplex}(Y,\F,\psi) \And
		\overline{e} \subset Y
	}
	{
		\NewLine :
		\Int_+ \to \Type
	}
	\AssumeIn{e}{\E_0}
	\Say{\Big(x,[2]\Big)}{\bd \TYPE{CellComplex}(X,\E,\varphi)}
	{
		\sum_{x \in X} : e \subset \{x\}
	}
	\Conclude{[0]}{\bd^{-1}\TYPE{FiniteCellComplex}[2]}{\TYPE{FiniteCellComplex}\Big(e,0 \mapsto \{e\},0 \mapsto \varphi_{0,e}\Big)}
	\Derive{[1]}{\ByConstr^{-1}\aries}{\aries(0)}
	\Assume{n}{\Int_+}
	\Assume{[2]}{\aries(n)}
	\AssumeIn{e}{\E_{n+1}}
	\Say{\Big(\F,[3]\Big)}
	{
		\bd \TYPE{CWComplex}(X,\E,\varphi)(e)
	}
	{
		\sum \F : \prod^n_{i=1}\TYPE{Finite}(\E_i) \. 
		\boundary e \subset \bigsqcup^n_{i=1} \bigsqcup_{f \in \F} f
	}
	\Conclude{[n.*]}{\ByConstr \aries [2](\F)[3]}
	{
		\TYPE{FiniteCellComplex}( \overline{e} , e \sqcup \F, \varphi_{(n+1),e} \sqcup \varphi_{\F}    )
	}
	\DeriveConclude{[*]}{\THM{CompleteInduction}(\Int_+)[1] \ByConstr \aries}
	{
		\LOGIC{This}
	}
	\EndProof 
	\\
	\Theorem{DiscreteSubsetLemma}
	{
		\forall (X,\E,\varphi) : \TYPE{CWComplex} \.
		\forall A \subset X \.
		\TYPE{Closed}\And\TYPE{Discrete}(A) \iff
		\forall e \in \E \.  |e \cap A| < \infty
	}
	\Assume{[1]}{\TYPE{Discrete}(A)}
	\AssumeIn{e}{\E}
	\Say{n}{\dim e}{\Int_+}
	\Say{[2]}{\bd \TYPE{CellComlex}(X,\E,\varphi)(A)[1]}
	{
		\Type{Discrete}\Big( \varphi^{-1}_{n,e}(A \cap e) \Big)
		\And \varphi^{-1}_{n,e}(A \cap e) \subset \mathbb{B}^n
	}
	\Assume{[3]}{|A\cap e| = \infty}
	\Say{[4]}{\THM{SequenceCompactDisc}[2]\bd \TYPE{Discrete}[3]}
	{
		\overline{\varphi^{-1}_{n,e}(A \cap e) \cap \mathbb{S}^{n-1} \neq \emptyset}
	}
	\Say{[5]}{\THM{ContinuousImagePreservesConvergence}[4] }
	{
		\overline{e \cap A} \cap \partial e \neq \emptyset
	}
	\Conclude{[3.*]}{\bd \TYPE{Discrete}(A)[5]}{\bot}
	\DeriveConclude{[1.*]}{\Elim(\bot)}{|A\cap e| < \infty}
	\Derive{[1]}{\Intro(\Imply)\Intro(\forall)}
	{
		\TYPE{Discrete}(A)
		\Imply
		\forall e \in \E \.
		|e \cap A| < \infty
	}
	\Assume{[2]}{\forall e \in \E \. | e \cap A| < \infty}
	\Assume{B}{?A}
	\Say{[3]}{\bd \TYPE{CWComplex}(X,\E,\varphi)[2]}
	{
		\forall e \in \E \. |\overline{e} \cap B| < \infty
	}
	\Say{[4]}{\THM{FiniteIsClosed}[3]}
	{
		\forall e \in \E \.  \TYPE{Closed}\Big( \overline{e},\overline{e} \cap B \Big)
	}
	\Conclude{[B.*]}{\bd \TYPE{CWComplex}(X,\E,\varphi)[4]}{\TYPE{Closed}(X,B)}
	\DeriveConclude{[2.*]}{\bd^{-1} \TYPE{Discrete}}{\TYPE{Discrete}(A)}
	\DeriveConclude{[*]}{\Intro(\iff)[1]}{\LOGIC{This}}
	\EndProof
}
\Page{
	\Theorem{CompactSubsetOfCWComplex}
	{
		\forall (X,\E,\varphi) \.
		\forall A : \TYPE{Closed}(X) \. 
		\TYPE{Compact}(X,A)
		\iff
		\NewLine
		\iff
		\exists (Y,\F,\psi) \subset (X,\E,\varphi) \.
		A \subset Y \And
		\TYPE{FiniteComplex}(X,\E,\varphi)
	}
	\Say{\A}{\{ e \in \E :e \cap A \neq \emptyset   \}}{?\E}
	\Assume{[1]}{\TYPE{Compact}(X,A)}
	\Assume{[2]}{|\A| > \infty}
	\Say{D}{\{ \varphi_{n,e}(0) | e \in \A , n = \dim e \}}{?X}
	\Say{[3]}{\ByConstr e \bd \TYPE{CellComplex}(X,\E,\varphi)}
	{
		\forall e \in \E \. |D \cap e| \le 1
	}
	\Say{[4]}{\ByConstr e p[3] \bd \TYPE{CWComplex}(X,\E,\varphi)}
	{
		\forall e \in \in \E \. |D \cap \overline{e}| < \infty
	}
	\Say{[5]}{\THM{DiscreteSubsetLemma}[4]}
	{
		\TYPE{Closed} \And \TYPE{Discrete}(X,A)
	}
	\Say{[6]}{[2]\ByConstr D}
	{
		|D| = \infty
	}
	\Conclude{[2.*]}
	{
		\THM{CompactDiscreteSubset}[5][6]
	}
	{
		\bot
	}
	\DeriveConclude{[1.*]}{\Elim(\bot)}
	{
		|\A| < \infty
	}
	\Derive{[1]}{\Intro(\Imply)}
	{
		\TYPE{Compact}(A) 
		\Imply
		|\A| < \infty
	}
	\Assume{[2]}{|\A| < \infty}
	\Say{[3]}{\THM{CompactMappingTHM}(\varphi)}
	{
		\forall e \in \E \.
		\TYPE{Compact}(\overline{e})
	}
	\Say{[4]}{\TYPE{CompactFiniteUnion}[2][3]}
	{
		\TYPE{Compact}\left(X, \bigcup_{e \in \A} \overline{e} \right)
	}
	\Say{[5]}{\ByConstr \A \bd^{-1} \TYPE{Subset}}
	{
		A \subset \bigcup_{e \in \A} \overline{e}
	}
	\Conclude{[2.*]}{\THM{ClosedCompactSubset}[5][4]}
	{
		\TYPE{Compact}(X,A)
	}
	\DeriveConclude{[*]}{I(\iff)[1]}
	{
		\TYPE{Compact}(X,\A)
		\iff
		|\A| < \infty
	}
	\EndProof
	\\
	\Theorem{CWComplexFiniteIffCompact}
	{
		\forall (X,\E,\varphi) : \TYPE{CWComplex} \.
		\TYPE{Compact}(X)
		\iff
		\TYPE{FiniteComplex}(X,\E,\varphi)
	}
	\NoProof
	\\
	\Theorem{CWComplexLocallyFiniteIffLocallyCompact}
	{
		\forall (X,\E,\varphi) : \TYPE{CWComplex} \.
		\TYPE{LocallyCompact}(X)
		\iff
		\NewLine
		\iff
		\TYPE{LocallyFiniteComplex}(X,\E,\varphi)
	}
	\NoProof
}
\newpage
\subsection{Inductive Construction}
\Page{
	\DeclareType{ByAttachingNCells}
	{
		?(\TYPE{T2}^2 \times \Nat)
	}
	\DefineType{(X,Y,n)}{ByAttachingNCells}
	{
		\sum I \in \SET :
		\sum \varphi : I \to  \mathbb{S}^{n-1} \Arrow{\TOP} Y :
		X = Y \sqcup_{\bigsqcup_{i \in \I} \varphi_i} \bigsqcup_{i \in \I} \mathbb{D}^n
	}
	\\
	\Theorem{CharacteristicMapsCoproductIsQuotient}
	{
		\forall (X,\E,\varphi) : \TYPE{CWComplex} \. \NewLine \.
		\TYPE{QuotintMap}\left( \bigsqcup_{e \in \E} \mathbb{D}^{\dim e}, X, \bigsqcup_{e \in \E} \varphi_{\dim e,e} \right) 
	}
	\NoProof
	\\
	\Theorem{SkeletonByAttachingNCells}
	{
		\forall (X,\E,\varphi) : \TYPE{CWComplex} \. 
		\forall n \in \Nat \.
		\TYPE{ByAttachingNCells}(X^{\skull n},X^{\skull n - 1},n)
	}
	\SayIn{\I}{\E_n}{\SET}
	\Say{\psi}{\Lambda e \in \E_n \varphi_{n,e|\mathbb{S}^{n-1}}}
	{
		\I \to \mathbb{S}^{n-1} \Arrow{\TOP} X
	}
	\Say{[1]}{\ByConstr \psi \bd \TYPE{CellComplex}}
	{
		\forall i \in \I \.
		\im \psi_i \subset X^{\skull n}
	}
	\Assume{A}{\TYPE{SaturatedClosed}(\psi)}
	\Say{[2]}{\bd \TYPE{SaturatedClosed}(\psi,A)}
	{
		\TYPE{Closed}\left( A \sqcup \bigsqcup_{e \in \E_n} \mathbb{D}^n   ,A \cap X^{\skull(n-1)} \right) 
		\And \NewLine \And
		\forall e \in \E_n \.
		\TYPE{Closed}\left( A \sqcup \bigsqcup_{e \in \E_n} \mathbb{D}^n  ,A \cap \mathbb{D}_e^n \cap \right)
	}
	\Say{[3]}{\bd \TYPE{CWComplex}(X^{\skull(n-1)},\E,\varphi)[2]}
	{
		\forall k \in (n-1) \.
		\forall e \E_k \. 
		\TYPE{Closed}( \overline{e} \cap  A    )
	}
	\Say{[3]}{\bd \TYPE{CWComplex}(X^{\skull(n-1)},\E,\varphi)[2]}
	{
		\forall k \in (n-1) \.
		\forall e \E_k \. 
		\TYPE{Closed}(\overline{e} , \overline{e} \cap  A )
	}
	\Say{[4]}{\ByConstr \psi [3]}
	{
		\forall k \in [0,\ldots,n-1]_{\Int_+} \.
		\forall e \E_k \. 
		\TYPE{Closed}(\overline{e} , \overline{e} \cap  \widehat{\psi}(A) )
	}
	\Say{[5]}{ \THM{ClosedMappingTheorem}[2]}
	{
		\forall e \in \E_n \.
		\TYPE{Closed}\Big(\overline{e},\widehat(\psi)(A \cap \mathbb{D}^n_e) \Big)
	}
	\Say{[6]}{\ByConstr \psi [5]}{     
		\forall e \in \E_n \.
		\TYPE{Closed}\Big(
			\overline{e},
			\widehat(\psi)(A) \cap e
		\Big)
	}
	\Conclude{[A.*]}{\bd \TYPE{CWComplex}(X,\E,\varphi)[4][6]}
	{
		\TYPE{Closed}(X,\widehat{\psi}(A))
	}
	\Derive{[2]}{ \bd^{-1} \TYPE{QuotientMap} }
	{
		\TYPE{QuotientMap}\left( X^{\skull (n-1)} \sqcup \bigsqcup_{e \in \E_n} \mathbb{D}_n ,X^{\skull n}, \widehat{\psi}\right)
	}
	\Conclude{[*]}{\bd^{-1}\TYPE{ByAttachingNCells}}
	{
		\TYPE{ByAttachingNCells}\Big( X^{\skull (n-1)},X^{\skull n},n\Big)
	}
	\EndProof
}\Page{
	\Theorem{CellExtensionTHM}
	{
		\forall (X,\E,\varphi) : \TYPE{CellComplex} \.
		\forall n \in \Nat \.
		\forall f : X^{\skull n-1} \Arrow{\TOP} [0,1] \. 
		\NewLine \.
		\exists \bar f : X^{\skull n} \Arrow{\TOP} [0,1] :
		\hat f_{|X^{\skull (n-1)}} = f \And
		\forall e \in \E_n \hat f(e) < 1
	}
	\AssumeIn{e}{\E_n}
	\AssumeIn{x}{e}
	\SayIn{p}{\varphi_{n,e}^{-1}(x)}{\mathbb{B}^n}
	\Conclude{\tilde f(x) }{ \If p==0 \Then 0 \Else \|p\|\left(  \frac{p}{\|p\|}  \right) \varphi_{e,n} f}{\TYPE{In}[0,1)}
	\Derive{\tilde f}{\Intro(\iff)}{\bigcup \E_n \Arrow{\TOP} [0,1)}
	\Say{\bar f}{\Lambda x \in X^{\skull n} \If x \in \bigcup \E_n \Then \tilde f(x) \Else f(x)}
	{
		X \to [0,1]
	}
	\Say{[1]}{\ByConstr \bar f}
	{
		\forall e \in \E_n \. 
		\forall x : \Nat \to e \. 
		\lim_{n \to \infty} x_n \in \boundary e_n
		\Imply \NewLine \Imply
		\lim_{n \to \infty} \bar f(x_n) =
		\lim_{n \to \infty} 
		\Big\| \varphi^{-1}_{n,e}(x_n) \Big\|
		\left( \frac{\varphi^{-1}_{n,e}(x_n)}{\|\varphi^{-1}_{n,e}(x_n)\|} \right)
		\varphi_{n,e} f = 
		f\Big( \lim_{n \to \infty} x_n  \Big) = 
		\bar f\Big( \lim_{n \to \infty} x_n \Big)
	}
	\Conclude{[*]}{\THM{ContinuousByLimits}[1]}{\bar f : X^{\skull n} \Arrow{\TOP} [0,1]}
	\EndProof
}
\Page{
	\Theorem{InductiveConstructionTHM}
	{
		\forall X \in \SET \.
		\forall Y : \TYPE{Increasing} \Big( \Int_+, \TYPE{Subset}(X) \Big) 
		\. \NewLine \. 
		\forall [0] : \bigcup^\infty_{n=0} Y_n = X \.
		\forall \tau : \prod^\infty_{n=0} \TYPE{Topology}(Y_n) \.
		\forall [00] : \TYPE{Discrete}(Y_0,\tau_0) \. 
		\NewLine \.
		\forall [000] : \forall n \in \Nat \. \TYPE{ByAttachingNCells}\Big((Y_n,\tau_n),(Y_{n-1},\tau_{n-1}),n\Big) \.
		\exists! \mathcal{T} : \TYPE{Topology}(X) : 
		\NewLine :
		\exists! (\E,\varphi) : \TYPE{CellComplex}(X,\mathcal{T}) :
		\forall n \in \Int_+ \. 
		X^{\skull n} =  Y_n
	}
	\Say{\E_0}{\Big\{ \{ y  \} \Big| y \in Y_0   \Big\}}{??X}
	\Assume{e}{\E_0}
	\Say{\Big(y,[1]\Big)}{\ByConstr \E_0 (e) }{\sum y \in Y_0 \. e = \{y_0\}}
	\Conclude{\varphi_{0,e}}{y}{\TYPE{In}(Y_0)}
	\Derive{\Big(\varphi_0,[1]\Big)}{\Intro\Act{\prod}}
	{ \prod_{e \in \E_0} \varphi_{0,e} : \{0\} \to Y_0 \. e = \{ \varphi_{0,e}(0) \}}
	\AssumeIn{n}{\Nat}
	\Say{[2]}{[000](n)}{\TYPE{ByAttachingNCells}(Y_n,Y_{n-1},n)}
	\Say{\Big(\I,\psi,[3]\Big)}{\bd \TYPE{ByAttachingNCells}[2]}
	{
		\sum \I \in \SET \.
		\sum \psi : \I \to \mathbb{S}^{n-1} \Arrow{\TOP} Y_{n-1} \. \NewLine \. 
		Y_n = Y_{n-1} \sqcup_{\id \sqcup \bigsqcup_{i \in \I}\psi_i} \bigsqcup_{i \in \I} \mathbb{D}^n_i 
	}
	\Say{\E_n}{\Big\{ \widehat{\psi}\big( i, \mathbb{B}^n \big) \Big| i \in \I \Big\}}{??X}
	\AssumeIn{e}{\E_n}
	\Say{\Big(i,[4]\Big)}{\ByConstr \E_n (e) }{\sum i \in \I \. e = \widehat{\psi}\Big(i,\mathbb{B}^n\Big)}
	\Say{\varphi_{n,e}}{\Lambda x \in \mathbb{D}^n \. \widehat{\psi}(i,x)}{\mathbb{D}^n \to Y_n}
	\Conclude{[e.*]}{\ByConstr \psi \ByConstr \varphi_{n,e} \ByConstr \E_n }{
		\varphi_{n,e}\Big( \mathbb{S}^{n-1} \Big) 
		\subset
		Y_{n-1}
		\And
		\varphi_{n,e | \mathbb{B}^{n}} : \mathbb{B}^n \ToIso{\SET} e 
	}
	\DeriveConclude{\Big(\varphi_n,[4]\Big)}{\Intro\Act{\prod}}
	{ 
		\prod_{e \in \E_n} \varphi_{n,e} : \mathbb{D}^n \Arrow{\TOP} Y_n \. 
		\varphi_{n,e} ( \mathbb{S}^{n-1}) \subset Y_{n-1} 
		\And
		\varphi_{n,e | \mathbb{B}^{n}} : \mathbb{B}^n \ToIso{\SET} e 
	}
	\Derive{\Big(\E,\varphi,[2]\Big)}{\Intro \Act{\prod} }
	{
		\prod_{n=1}^\infty E_n : ??X \.
		\prod_{e \in \E_n} \varphi_{n,e} : \mathbb{D}^n \Arrow{\TOP} Y_n \. 
		\varphi_{n,e}(\mathbb{S}^{n-1}) \subset Y_{n-1} 
		\And
		\varphi_{n,e |\mathbb{B}^{n}} : \mathbb{B}^n \ToIso{\SET} e 
	}
	\Say{\mathcal{T}}{ \{ U \subset X : \forall n \in \Int_+ \. U \cap Y_n \in \tau_n  \}  }{??X}
	\Say{[3]}{\bd \TYPE{ByAttachingNCells}[000]}{
		\forall n \in \Nat \. 
		\TYPE{Closed}\Big( (Y_n,\tau_n), (Y_{n-1},\tau_{n-1}) \Big)}
	\Say{[4]}{\ByConstr \mathcal{T}\THM{FinalTopologyByInclusions}[3]}
	{
		\TYPE{Topology}\Big(X,\mathcal{T} \Big)
	}
	\Say{[5]}{\ByConstr \E \bd \TYPE{ByAttachingNCells}[000]}
	{
		X = \bigsqcup^\infty_{n=0} \E_n
	}
	\Say{[6]}{\ByConstr \E  [2]}
	{
		\forall n \in \Nat \. 
		\forall e \in \E_n \.
		\boundary e \subset Y_{n-1}
	}
	\Say{[7]}{\bd^{-1} \TYPE{CellComplex}[6][5]  }
	{
		\TYPE{CellComplex}(X,\E,\varphi)
	}
	\Say{[8]}{\ByConstr \varphi  [7][2]}
	{
		\forall n \in \Nat \. 
		X^{\skull n} = Y_n
	}
}\Page{
	\Assume{x}{X}
	\Say{\Big(e,[9]\Big)}
	{
		[5](x)
	}
	{
		\sum e \in \E \. x \in e 
	}
	\SayIn{n}{\dim e}{\Int_+}
	\Assume{[10]}{n = 0}
	\Say{f}{\Lambda y \in Y_0 \. x == y}{Y_0 \Arrow{\TOP} [0,1]}
	\Conclude{[10.*]}{\ByConstr f}{ f^{-1}(1) = \{x\}}
	\Derive{[10]}{\Intro(\Imply)\Intro\Act{\sum}}{n =0 \Imply \sum f : Y_0 \Arrow{\TOP} [0,1] \. f^{-1}(1) = \{x\} }
	\Assume{[11]}{n \neq 0}
	\SayIn{p}{\varphi_{n,e}^{-1}(x)}{\mathbb{B}^n}
	\Say{\Big(g,[12]\Big)}{\THM{RegularFunctionalProperty}(\mathbb{D}^n,p,\mathbb{S}^{n-1})[11]}
	{
		\sum g : \mathbb{D}^n \Arrow{\TOP} [0,1] \. g\Big(\mathbb{S}^{n-1}\Big) = 0 \And g^{-1}(1)=p
	}
	\Say{f}{\Lambda y \in \mathbb{D}^n \. \If y \in e \Then y \; \varphi^{-1}_{n,e} \; g \Else 0}
	{
		Y_{n} \Arrow{\TOP} [0,1]
	}
	\Conclude{[11.*]}{\ByConstr f [12]}{f^{-1}(1) = x}
	\Derive{[11]}{\Intro(\Imply)\Intro\Act{\sum}}{n \neq 0 \Imply \sum f : Y_n \Arrow{\TOP} [0,1] \. f^{-1}(1) = \{x\} }
	\Say{\Big(f,[12]\Big)}{\Elim(|)\LOGIC{LEM}(n=0)[10][11]}
	{
		\sum f : Y_n \Arrow{\TOP} [0,1] \. f^{-1}(1) = \{x\}
	}
	\Conclude{\Big(\bar f,[*]\Big)}
	{
		\bd \Nat \ByConstr \mathcal{T} \THM{CellExtensionTHM}\big(f,[12])
	}
	{
		\sum \bar f : X \Arrow{\TOP} [0,1] \. {\bar f}^{-1}(1) = \{x\}
	}
	\Derive{[10]}{\THM{HausdorffByMappings}}
	{
		\TYPE{T2}(X)
	}
	\Assume{n}{\Nat}
	\Assume{[11]}{\TYPE{CWComplex}(X^{\skull(n-1)},\E^{\skull(n-1)},\varphi^{\skull(n-1)})}
	\Assume{e}{\E_n}
	\Say{[12]}{\bd \TYPE{CellComplex}(X,\E,\varphi)}{\boundary e \subset X^{\skull(n-1)}}
	\Say{[13]}{\THM{CompactMappingTHM}(\mathbb{S}^{n-1},\varphi_n)}
	{
		\TYPE{Compact}(X^{\skull(n-1)},\boundary e)
	}
	\Conclude{[e.*]}{\THM{FiniteComplexIffCompact}[13]}{ 
		\exists \F   : \TYPE{Finite}{\E} \. \partial e \subset \bigcup \F
	}
	\Derive{[12]}{\Intro(\forall)}{ 
		\forall e \in \E_n \. \exists \F   : \TYPE{Finite}{\E} \. \partial e \subset \bigcup \F
	}
	\Assume{A}{?X^{\skull n}}
	\Assume{[13]}{\forall e \in \E^{\skull n} \. \TYPE{Closed}(\overline{e},\overline{e} \cap A) }
	\Say{[14]}{\bd \TYPE{CWComplex}(X^{\skull(n-1)},\E^{\skull(n-1)},\varphi^{\skull(n-1)})[13]}
	{
		\TYPE{Closed}\Big( X^{\skull(n-1)},A \cap X^{\skull(n-1)} \Big)
	}
	\Conclude{[A.*]}{\bd \TYPE{QuotientMap}\ByConstr \varphi [14][13]}
	{
		\TYPE{Closed}(X^{\skull n},A)
	}
	\DeriveConclude{[13]}{\bd^{-1}\TYPE{CWComplex}}
	{
		\TYPE{CWComplex}(X^{\skull n},\E^{\skull n},\varphi^{\skull n})
	}
	\Derive{[11]}{\bd \Nat [00]}
	{
		\forall n \in \Nat \.
		\TYPE{CWComplex}(X^{\skull n},\E^{\skull n},\varphi^{\skull n})
	}
	\Conclude{[*]}
	{
		\ByConstr \mathcal{T} [11]
	}
	{
		\TYPE{CWComplex}(X,\E,\varphi)
	}
	\EndProof
}
\Page{
	\Theorem{CWComplexIsParacompact}
	{
		\forall (X,\E,\varphi) : \TYPE{CWComplex} \.
		\TYPE{Paracompact}(X)
	}
	\Assume{\mathcal{O}}{\TYPE{OpenCover}(X)}
	\Say{\mathcal{O}^\skull}{
		\Lambda n \in \Int^+ \. 
		\Big\{ O \cap X^{\skull n} | O \in \mathcal{O}   \Big\}
	}
	{
		\prod^\infty_{n=0} \TYPE{OpenCover}\Big(X^{\skull n}\Big)    
	}
	\Say{[1]}{\bd \TYPE{Cover}\Big(X^{\skull 0},\mathcal{O}^{\skull 0})}
	{
		\forall x \in X^{\skull 0} \. \Big\{ O \in \mathcal{O}^{\skull 0}  \Big\} \neq \emptyset
	}
	\Say{\Big(O,[2]\Big)}{\LOGIC{Choice}[1]}{\prod_{x \in X^{\skull 0}} \sum_{O_x \in \mathcal{O}^{\skull 0}} x \in O_x  }
	\Say{\phi^0}{\Lambda U \in \mathcal{O}^{\skull 0} \. \Lambda x \in X^{\skull 0} \. \delta_{O_x,U}}
	{
		\mathcal{O}^{\skull 0} \to X^{\skull 0} \Arrow{\TOP} [0,1]
	}
	\Say{[3]}{\bd^{-1} \TYPE{PartitionOfUnity}\Big(X^{\skull 0},\mathcal{O}^{\skull 0}\Big)\ByConstr \phi^0 [2]}
	{
		\TYPE{PartitionOfUnity}\Big(X^{\skull n},\mathcal{O}^{\skull n},\phi^0\Big)
	}
	\Assume{n}{\Int_+}
	\Assume{\phi^{n}}{\TYPE{PartitionOfUnity}\Big( X^{\skull n}, \mathcal{O}^{\skull n}\Big)}
	\Assume{[4]}{\forall k \in [0,n)_{\Int_+} \. \phi^{n-1}_{|X^{\skull k}} = \phi^k }
	\Assume{[5]}{
		\forall k \in [0,n)_{\Int_+} \. 
		\forall U \in \T\Big( X^{\skull k} \Big) \. 
		\phi^k(U) = \{0\} \Imply \exists V \in \T\Big( X^{\skull n-1} \Big) :
		U \subset V \And \phi^{n-1}(V) = \{0\}
	}
	\Say{[6]}{\THM{SkeletonbyAttachingNCells}(X,\E,\varphi,n)}
	{
		\TYPE{ByAttachingNCells}\Big(X^{\skull n},X^{\skull(n+1)},n+1\Big)
	}
	\Say{I}
	{
		\Lambda A \subset \Sphere^n \.
		\Lambda t \in \Reals_{++} \.
		\left\{
			x \in \Disk^{n+1} : |x| > t 
			\And \frac{x}{\|x\|} \in A 
		\right\}
	}
	{
		?\Sphere^n \to \Reals_{++} \to \Disk^{n+1}
	}
	\Say{\Big(\I,\psi,[7] \Big)}
	{
		\bd \TYPE{ByAttachingNCells}[7]
	}
	{
		\sum \I \in \SET \.
		\sum \psi : \I \to \mathbb{S}^n \to X^{\skull n} \.
		X^{\skull n + 1}  = X^{\skull n} \sqcup_\psi \bigsqcup \mathbb{D}^n
	}
	\Assume{i}{\I}
	\Say{\mathcal{V}}{ \Big\{ \psi^{-1}_i(O) \Big| O \in \mathcal{O}^{\skull n} \Big\} }
	{
		\TYPE{OpenCover}(\Sphere^{n})
	}
	\Say{f^i}{ 
		\Lambda V \in \mathcal{V} \.
		\Lambda s \in \Sphere^n \.  
		\phi_{\psi_i(V)}\Big(\widehat \psi(i,s)\Big)}
	{
		\TYPE{PartitionOfUnity}(\Sphere^n,\mathcal{V} \cap \Sphere^n )
	}
	\Say{[8]}
	{
		\bd \THM{CompactPartitionOfUnity}(\Sphere^n,f^i)
	}
	{
		\Big|
			\Big\{
				V \in \mathcal{V} : 
				f^i_O \neq 0
			\Big\}
		\Big|<
		\infty
	}
	\SayIn{m}{\Big|\Big\{ V \in \mathcal{V} : f^i_0 \neq 0 \Big\} \Big|}{\Nat}
	\Say{V}{\FUNC{enumerate}\Big\{ V \in \mathcal{V} : f^i_0 \neq 0 \Big\} }{m \ToBij\Big\{ V \in \mathcal{V} : f^i_0 \neq 0 \Big\}  }
	\Say{g}{f_V}{m \to \mathbb{S}^n \Arrow{\TOP} [0,1]  }
	\AssumeIn{j}{[1,\ldots,m]_\Nat}
	\Say{[9]}{\ByConstr g_j}{\TYPE{Compact}( V_j \cap \Sphere^n, \supp g_j )}
	\Conclude{\Big(t_j,[j.*]\Big)}{\bd\TYPE{Compact}[9]\ByConstr^{-1} I}
	{
		\sum_{ t \in (0,1)}  I(\supp g_j,t) \subset V_j
	}
	\Derive{\Big(t,[9]\Big)}{\Intro \Act{\prod}}
	{
		\prod^m_{j=1} \sum_{t_j \in (0,1)} I(\supp g_j,t) \subset V_j
	}
	\SayIn{s}{\max t}{(0,1)}
	\SayIn{s'}{1 - \frac{(1-s)}{2}}{(0,1)}
	\Say{\sigma}{\FUNC{bump}\Big( \Disk^{n+1},\Disk^{n+1}\setminus I\Big(\Sphere^n,s\Big), I\Big(\Sphere^n) \Big)}
	{ \Disk^{n+1} \Arrow{\TOP} [0,1] }
	\Say{\mathcal{W}}{ \Big\{ \psi^{-1}_i(O) \Big| O \in \mathcal{O}^{\skull n+1} \Big\} }
	{
		\TYPE{OpenCover}(\Disk^{n+1})
	}
	\Say{[10]}{\ByConstr \mathcal{V} \mathcal{W}}{\mathcal{W} \cap \Sphere^n = \mathcal{V}}
	\Say{h}{\THM{PartitionOfUnityExists}(\Disk^{n+1},\mathcal{W})}
	{
		\TYPE{PartitionOfUnity}\Big( \Disk^{n+1},\mathcal{W} \Big)
	}
}\Page{
	\Say{F^i}{
		\Lambda W \in \mathcal{W} \. 
		\Lambda x \in \Disk^{n+1} \. 
		\sigma(x)h_W(x)
		\Big(1 - \sigma(x)\Big) f_{W \cap \Sphere^n}\left(\frac{x}{\|x\|}\right)	
	}
	{
		\mathcal{W} \to \Disk^{n+1} \Arrow{\TOP} [0,1]
	}
	\Say{[11]}{\ByConstr F^i}
	{
		\forall x \in \Disk^{n+1}
		\sum_{W \in \mathcal{W}}  F^i_W(x) =
		\sigma(x) \sum_{W \in \mathcal{W}} h_W(x) + 
		\Big(1 - \sigma(x)\Big) \sum_{W \in \mathcal{W}}f_{W \cap \Sphere^n}\left( \frac{x}{\|x\|} \right)
		\sigma(x) + 1 - \sigma(x) = 1
	}
	\Conclude{[i.*]}{\bd^{-1}\TYPE{PartitionOfUnity}[11]\bd \TYPE{Compact}(\mathcal{D}^n)}
	{
		\TYPE{PartitionOfUnity}\Big( \Disk^n,\mathcal{W},F \Big)
	}
	\Derive{\Big( F,[8]\Big)}
	{  
		\Intro \Act{\prod}
	}
	{
		\prod_{i \in \I} 
		\sum F^i 
			: \TYPE{PartitionOfUnity}\Big( \Disk^{n+1}, \widehat \psi^{-1} \mathcal{O}^{\skull n+1} \Big) \.
		\NewLine \.
		\forall O \in \mathcal{O}^{\skull n + 1} \.
		F^i_{\widehat \psi^{-1}O|\Sphere^n} = \psi_i \phi^n_{O \cap X^{\skull n}}
	}
	\Say{\phi^{n+1}}{
		\Lambda O \in \mathcal{O}^{\skull n + 1} \. 
		\Lambda x \in X^{\skull n + 1} \. 
		\If x \in X^{\skull} 
		\Then \phi^n_{O \cap X^{\skull n}}(x)
		\Else F^i_{\widehat \psi^{-1}O} \circ \widehat \psi^{-1}_2(x)
		\NewLine
		\quad
		\where
		\quad
		x \in \widehat \psi\Big( i , \Disk^n \Big)
	}
	{
		\mathcal{O}^{\skull n + 1} \to X^{\skull n + 1} \to [0,1]
	}
	\Say{[n.*.1]}{\ByConstr \phi^{n+1}}{
		\forall O \in \mathcal{O}^{\skull n + 1} \. 
		\phi^{n+1}_{O | X^{\skull n}} = \phi^n_{O \cap X^{\skull n}}
	}
	\Say{[n.*.2]}{\ByConstr F \ByConstr \phi^{n+1} \bd \TYPE{QuotientMap}(\hat \psi)}
	{
		\forall U \in \T\Big( X^{\skull n} \Big) \. 
		\forall O \in \mathcal{O} \.
		\phi^n(U)_{O \cap X^{\skull n}} = \{0\} \Imply 
		\NewLine \Imply
		\exists V \in \T\Big( X^{\skull n+1} \Big) :
		U \subset V \And \phi^{n+1}(V) = \{0\}
	}
	\Conclude{[n.*.3]}{
		\ByConstr F 
		\ByConstr \phi^{n+1} 
		\bd \TYPE{PartitionOfUnity}\Big(X^{\skull n},\mathcal{O}^{\skull n},\phi^n\Big) 
		[5]
	}
	{
		\TYPE{PartitionOfUnity}\Big( X^{\skull n + 1},\mathcal{O}^{\skull n +1},\phi^{n+1} \Big)	
	}
	\Derive{\Big(\phi,[4] \Big)}
	{
		\LOGIC{PrimitiveRecursion} \ByConstr \phi^0
	}
	{
		\prod^\infty_{n=0} 
		\sum \phi^n : \TYPE{PartitionOfUnity}\Big( X^{\skull n}, \mathcal{O}^{\skull n} \Big) \.
		\NewLine \. 
		\forall n \in \Nat \.  
		\forall O \in \mathcal{O}^{\skull n + 1} \. 
		\phi^{n+1}_{O | X^{\skull n}} = \phi^n_{O \cap X^{\skull n}}
		\And \NewLine \And
		\forall U \in \T\Big( X^{\skull n} \Big) \. 
		\forall O \in \mathcal{O} \.
		\phi^n(U)_{O \cap X^{\skull n}} = \{0\} \Imply \exists V \in \T\Big( X^{\skull n+1} \Big) :
		U \subset V \And \phi^{n+1}(V) = \{0\}
	}
	\Say{f}{\Lambda O \in \mathcal{O} \. {\varinjlim}_n \phi^n_{X^{\skull n}\cap O}}
	{
		\mathcal{O} \to X \Arrow{\TOP} [0,1]
	}
	\Conclude{[\mathcal{O}.*]}{\ByConstr f [4]\bd \TYPE{CWComplex}(X,\E,\varphi)}{
		\TYPE{PartitionOfUnity}(X,\mathcal{O},f)
	}
	\DeriveConclude{[*]}{\bd^{-1}\THM{ParacompactByPartitionOfUnity}}
	{
		\TYPE{Paracompact}(X)
	}
	\EndProof
	\\
	\Theorem{CWComplexIsNormal}
	{
		\forall (X,\E,\varphi) : \TYPE{CWComplex} \.
		\TYPE{T4}(X)
	}
	\NoProof
	\\
	\Theorem{countableLocallyEucleadeanCWComplexIsManifold}
	{
		\NewLine ::
		\forall (X,\E,\varphi) : \TYPE{CWComplex} \.
		\forall n \in \Nat \.
		|\E| \le \aleph_0 \And  \TYPE{LocallyEuclidean}(X,n)
		\Imply
		X \in \TOPM(n)
	}
	\NoProof
	\\
	\Theorem{DimensionAgrees}
	{
		\forall (X,\E, \varphi) : \TYPE{CWComplex} \. 
		\forall n \in \Nat \.
		X \in \TOPM(n) \Imply \dim \E = n
	}
	\NoProof
}
\newpage
\subsection{Embedding Theorems}
\Page{
	\DeclareType{EuclideanCWComplex}
	{
		?\Big(\TYPE{CWComplax} \And \TYPE{LocallyFiniteComplex} \And \NewLine \And
		\TYPE{FiniteDimensionalComplex} \Big)
	}
	\DefineType{(X,\E,\varphi)}{EuclideanCWComplex}{ |\E| \le \aleph_0  }
	\\
	\DeclareType{QuasiEuclideanCWComplex}
	{
		?\Big(\TYPE{CWComplax} \And \TYPE{LocallyFiniteComplex}  \Big)
	}
	\DefineType{(X,\E,\varphi)}{EuclideanCWComplex}{ |\E| \le \aleph_0  }
	\\	
	\Theorem{CWComplexMetrizationTHM}
	{
		\forall (X,\E,\varphi) : \TYPE{CWComplex} \.
		\TYPE{LocallyFinite}(X,\E,\varphi)
		\iff
		\TYPE{Metrizable}(X)
	}
	\NoProof
	\\
	\Theorem{ComplexEuclideanEmbeddingTheorem}
	{
		\forall (X,\E,\varphi) :
		\TYPE{EucleadinCWComplex} \. \NewLine \.
		\exists \TYPE{HomeomorphicEmbedding}\Big(X, \Reals^{1 + 2\dim \E }\Big)
	}
	\NoProof
	\\
	\Theorem{ComplexQuasiEuclideanEmbeddingTheorem}
	{
		\forall (X,\E,\varphi) :
	\TYPE{QuasiEucleadinCWComplex} \. \NewLine \.
		\exists \TYPE{HomeomorphicEmbedding}\Big(X, \Reals^{\oplus \Nat }\Big)
	}
	\NoProof
	\\
	\Theorem{ComplexQuasiEuclideanEmbeddingTheorem}
	{
		\forall (X,\E,\varphi) :\TYPE{CWComplex} \. \NewLine \.
		\exists V \in \Reals \hyph \mathsf{TOPVS} \.
		\exists \TYPE{HomeomorphicEmbedding}\Big(X, V\Big)
	}
	\NoProof
}
\newpage
\subsection{Classification of 1D manifolds}
\Page{
	\Theorem{ManifoldDimIsTopDim}
	{
		\forall M \in \TOPM \.
		\dim M = \mathrm{top}\;\dim  M
	}
	\SayIn{n}{\dim M}{\Int_+}
	\Assume{A}{\TYPE{Closed}(M,A)}
	\Assume{\Psi}{A \Arrow{\TOP} \Sphere^n}
	\Say{\Big( \O, [1] \Big)}{\bd \TOPM}
	{
		\sum \O : \TYPE{OpenCover}(M) \. 
		\forall O \in \O \.
		O \cong \Ball^n
	}
	\Say{[2]}{\THM{TopologicalDimInvariant}[1]\THM{BallDim}}
	{
		\forall O \in \O \. \mathrm{top}\; \dim O = n
	}
	\Say{\Big(\psi,[3]\Big)}{\THM{NormalTopologicalDim}[2]}
	{
		\prod O \in \O \. 
		\sum \psi_O : O \Arrow{\TOP} \Sphere^n \.
		\psi_{O|O \cap A} = \Psi_{|O \cap A}
	}
	\Say{f}{\THM{PartitionOfUnityexists}(M,\O)}
	{
		\TYPE{PartitionOfUnity}(M,\O)
	}
	\Say{\Psi'}{\sum_{O \in \O} f_O \psi_O}
	{
		M \Arrow{\TOP} \Sphere^n
	}
	\Conclude{[A.*]}{\bd \TYPE{PartitionOfUnity}[3]}
	{
		\Psi'_{|A} = \Psi
	}
	\DeriveConclude{[*]}{\THM{NormalTopologicalDim}}
	{
		\mathrm{top}\;\dim M = n
	}
	\EndProof
	\\
	\Theorem{OneManifoldAdmitsRegularCWStruct}
	{
		\forall M \in \TOPM(1)  \.
		\exists (X,\E,\varphi) : \TYPE{RegularCWComplex} \.
		M = X
	}
	\Say{\Big(\mathcal{V},[1]\Big)}{\THM{ManifoldDimIsTopDim}(M)\bd\mathrm{top}\;\dim}
	{
		\NewLine :
		\sum \mathcal{V} : \TYPE{OpenCover}(M) \.
		\forall V \in \mathcal{V} \. 
		V \cong_{\TOP} (0,1) \And 
		\bigg|\Big\{ U \in \mathcal{V} : U \neq V \And U \cap B \neq \emptyset \Big\}\bigg| \le 2
	}
	\Say{[2]}{\bd \TYPE{LocallyCompact}(X)\THM{BaseEq}(\mathcal{V}) }{\Big|\mathcal{V}\Big| \le \aleph_0}
	\Say{V}{\FUNC{emumerate}(\mathcal{V})}{\Nat \ToBij \mathcal{V}}
	\Say{N}{\Lambda n \in \Nat \. \bigcup^n_{i=1} \overline{V}_i}{\Nat \to \TYPE{Closed}(M)}
	\Say{[3]}{\bd \TYPE{OpenCover}(\mathcal{V}) \ByConstr N}{M = \bigcup^{\infty}_{n=1} N_n}
	\Say{\mathcal{E}^1}{\Big( 0 \to \boundary V_1,1 \to V_1 \Big)}{\{0,1\} \to ?M }
	\Say{\Big(\varphi^1,[4]\Big)}{\ByConstr \mathcal{E}^1}
	{
		\sum \varphi^1 \prod^1_{i=0} \prod_{e \in \E^1_i} \Disk^i \to e \.
		\TYPE{CWComplex}(N_1,\E^1,\varphi^1)
	}
	\AssumeIn{n}{\Nat}
	\Assume{(X,\E^n,\varphi^n)}{\TYPE{RegularCWComplex}}
	\Assume{[5]}{X = N_n}
	\Say{\E^{n+1}_0}{ E^n_0 \cup \boundary V_{n+1}}
	{
		?M
	}
	\Say{\E^{n+1}_1}{
		\Big\{ U \cap V_{n-1} | U \in \E^n_1 \Big\}
		\cup
		\Big\{ U \setminus V_{n-1} | U \in \E^n_1  \Big\}
		\cup
		\Big\{ V_{n-1} \setminus U | U \in \E^n_1  \Big\}
	}
	{
		?\T(M)
	}
	\Conclude{\Big(\varphi^{n+1},[n.*]\Big)}{[5]\ByConstr \mathcal{E}^{n+1}}
	{
		\sum \varphi^{n+1} \prod^1_{i=0} \prod_{e \in \E^{n+1}_i} \Disk^i \to e \.
		\TYPE{CWComplex}(N_{n+1},\E^{n+1},\varphi^{n+1})
	}
	\Derive{\Big(\E,[5]\Big)}{ \Intro \Act{\sum} \Intro \Act{\prod} }
	{
		\sum \E : \prod^\infty_{n=1} \{ 0,1 \} \to ? N_n \. 
		\forall n \in \Nat \. \exists (X,\E,\varphi) : \TYPE{RegularCWComplex} :
		X = N_n 
	}
}\Page{
	\Say{\E'}{\Lambda_{i=0}^1 \bigcup^\infty_{n=1} \bigcap_{k=n}^\infty \E^n_i }
	{
		\{0,1\} \to ?M
	}
	\Say{[6]}{\ByConstr \E'[1]\ByConstr \E}{ \forall i \in \{0,1\} \. \E'_i \neq \emptyset}
	\Say{\Big(\varphi, [7] \Big)}{[6][5]}
	{
		\prod^1_{i=0} \prod_{e \in \E'_i} \sum \varphi_{i,e} : \TYPE{TopologicalEmbedding}\Big( \Disk^i ,M\Big) \.
		\im \varphi_{i,e} = \overline{e}
	}
	\Say{[8]}{[3]\ByConstr \E'}{\bigcup^1_{i=1} \bigcup_{e \in \E_i} e = M}
	\Say{[9]}{\ByConstr \E' [7]}{\forall e \in \E_1 \. \Big|\FUNC{singleton} \boundary e \cap \E_0 \Big| = 2  }
	\Assume{A}{?M}
	\Assume{[10]}{\forall e \in \E'_1 \. \TYPE{Closed}\Big(\overline{e},\overline{e} \cap A\Big)}
	\Say{[11]}{\bd \TYPE{CWComplex}[5]}{\forall n \in \Nat \. \TYPE{Closed}\Big(N_n \cap A\Big) }
	\Say{[12]}{\THM{ClosedSubsetTHM}[11]}
	{
		\forall n \in \Nat \. \TYPE{Closed}\Big( M, N_n \cap A \Big)
	}
	\Say{[13]}{[3]\THM{UnionDistrivutibity}}{\bigcup^\infty_{m=1} N_n \cap A = 1}
	\Say{[14]}{[1]\ByConstr N}{\forall n \in \Nat \. \TYPE{LocallyFinite}(X,N_n)}
	\Conclude{[A.*]}{[13][14]\THM{LocalyFiniteClosureUnion}}
	{
		\TYPE{Closed}(M,A)
	}
	\DeriveConclude{[*]}{\bd^{-1}\TYPE{CWComplex}[7]}{ \TYPE{RegularCwComplex}\Big(M,\E',\varphi \Big)  }
	\EndProof
	\\
	\Theorem{RegularGraphManifoldEdge}
	{
		\forall (M,\E,\varphi) : \TYPE{RegularCWComplex} \.
		\forall [0] : M \in \TOPM \. \NewLine \. 
		\forall [00] : \dim M = 1 \.
		\forall e \in \E_1 \.
		\Big| \boundary e \Big| = 2
	}
	\NoProof
	\\
	\Theorem{RegularGraphManifoldVertex}
	{
		\forall (M,\E,\varphi) : \TYPE{RegularCWComplex} \.
		\forall [0] : M \in \TOPM \. \NewLine \. 
		\forall [00] : \dim M = 1 \.
		\forall v \in \E_0 \.
		\Big| \big\{ e \in \E_2 : v \in \bar e \big\} \Big| = 2
	}
	\NoProof
	\\
	\Theorem{1DManifoldClassificationTHM}
	{
		\forall M \in \TOPM \And \TYPE{Connected} \.
		\forall [0] : \dim M = 1 \.
		M \cong_{\TOP} \Sphere^1
		\Big|
		M \cong_{\TOP} \Reals
	}
	\Say{\Big( (X,\E,\varphi),[1] \Big)}
	{ 
		\THM{OneManifoldAdmitsRegularCWStructure}(M)[1]
	}
	{
		\sum (X,\E,\varphi) : \TYPE{RegularCWComplex} \. 
		\NewLine
		\. M = X
	}
	\Assume{[2]}{\TYPE{Compact}(X)}
	\Say{[3]}{\THM{CompactIffFinite}[1][2]}{|\E| < \infty}
	\Say{n}{|\E_0|}{\Nat}
	\Say{\Big(v,[3.5]\Big)}{\FUNC{cyclicEnumerate}(\E_0,\ldots)}
	{ 
		\sum v :
		\Int_n \ToBij \E_0  \. 
		\forall i \in \Int_+ \.
		\exists e \in \E_1 \. \NewLine \. 
		\lim_{t \to 0} \varphi^{-1}_{1,e}(t) = v_i \And
		\lim_{t \to 1} \varphi^{-1}_{1,e}(t) = v_{i+1} 
	}
}\Page{	
	\AssumeIn{x}{M}
	\Say{\Big(e,[4]\Big)}{\bd \TYPE{CellComplex}(M,\E,\varphi)(x)}{\sum_{e \in \E} x \in e}
	\Assume{[5]}{\dim e = 0}
	\Say{\Big(k,[6]\Big)}{\bd \TYPE{Bijection}(v,e)}
	{
		\sum_{k=0}^{n-1} e = v_k
	}
	\Conclude{f(x)}{\exp\left(\frac{2k\i\uppi}{n}\right)}{\Sphere^1}
	\Derive{[6]}{\Intro(\Imply)}{\dim e = 0 \Imply f(x) \in \Sphere^1}
	\Assume{[7]}{\dim e = 1}
	\Say{\Big( t, [9]\Big)}{\bd \TYPE{Bijection}(v)[3.5]}{\sum^{n-1}_{t,l=0} t\neq l \And \boundary e = \{v_{t},v_{t+1}\}}
	\Conclude{f(x)}{ \exp\left( \frac{2\i\uppi\Big( t + \varphi_{1,e}^{-1}(x) \Big)}{n} \right)  }
	{
		\Sphere^1
	}
	\Derive{[10]}{\Intro(\Imply)}
	{
		e \in \E_1 \Imply f(x) \in \Sphere^1
	}
	\Conclude{[2.*]}{\bd^{-1}\TYPE{CellComplex}(M,\E,\varphi)\ByConstr f [3.5]}
	{
		f :  M \ToIso{\TOP} \Sphere^1
	}
	\Derive{[2]}{\Intro(\Imply)}{\TYPE{Compact}(M) \Imply M \cong_{\TOP} \Sphere^1}
	\Assume{[3]}{\IsNot \TYPE{Compact}(M)}
	\Say{[4]}{\THM{CompactIffFinite}}{|\E| = \infty}
	\Say{[5]}{\THM{RegularManifoldVertex}(M)[0][1][4]}
	{
		|\E_0| = |\E_1|
	}
	\Say{[6]}{\bd \TOPM(M)}{\TYPE{SeconCountable}(M)}
	\Say{[7]}{\THM{OpenCellTHM}(M)[0]}
	{
		\forall e \in \E_1 \. e \in \T(X)
	}
	\Say{[8]}{\bd \TYPE{SecondCountable}[7][6]}
	{
		|\E_0| = \aleph_0
	}
	\Say{\Big(v,[8.5]\Big)}{\bd \TYPE{Cardinality}(\Int,\E_0)}
	{
		\sum v : \Int \ToBij \E_0 \.
		\forall i \in \Int \.
		\exists e \in \E_1 \.
		\lim_{t \to 0} \varphi_{1,e}(t) = v_i \And
		\lim_{t \to 1} \varphi_{1,e}(t) = v_{i+1} 
	}
	\Assume{x}{M}
	\Say{\Big(e,[4]\Big)}{\bd \TYPE{CellComplex}(M,\E,\varphi)(x)}{\sum_{e \in \E} x \in e}
	\Assume{[5]}{\dim e = 0}
	\Say{\Big(k,[6]\Big)}{\bd \TYPE{Bijection}(v,e)}
	{
		\sum_{k \in \Int_+} e = v_k
	}
	\Conclude{f(x)}{k}{\Reals}
	\Derive{[6]}{\Intro(\Imply)}{\dim e = 0 \Imply f(x) \in \Reals}
	\Assume{[7]}{\dim e = 1}
	\Say{\Big( t, [9]\Big)}{\bd \TYPE{Bijection}(v)[3.5]}{\sum_{t \in \Int_+} t\neq l \And \boundary e = \{v_{t},v_{t+1}\}}
	\Conclude{f(x)}{  t + \varphi_{1,e}^{-1}(x)  }
	{
		\Reals
	}
	\Derive{[10]}{\Intro(\Imply)}
	{
		e \in \E_1 \Imply f(x) \in \Reals
	}
	\Conclude{[3.*]}{\bd^{-1}\TYPE{CellComplex}(M,\E,\varphi)\ByConstr f [3.5]}
	{
		f :  M \ToIso{\TOP} \Reals
	}
	\DeriveConclude{[*]}{[2]\Intro(\Imply)\Elim(|)}
	{
		M \cong \Sphere^1 \Big| M \cong \Reals
	}
	\EndProof
}
\Page{
	\Theorem{1DManifoldWithBoundaryClassification}
	{
		\forall M \in \TOPM_\partial \And \TYPE{Connected} \. \NewLine \.
		\forall [0] : \dim M = 1 \.
		\forall [00] : \partial M \neq \emptyset \. 
		M \cong_\TOP [0,1] \Big|
		M \cong_\TOP [0,+\infty)
	}
	\SayIn{N}{M \setminus \boundary N}{\TOPM(1)}
	\Say{[1]}{\ByConstr N \bd \TYPE{Compact}[00]}{\IsNot \TYPE{Compact}(N)}
	\Say{[2]}{\THM{1DManifoldClassification}[1]}{N \cong_{\TOP} \Reals}
	\Conclude{[*]}{[00][2]\bd \TOPM_\partial(M)}{M \cong_\TOP [0,1] \Big| M \cong_\TOP [0,+\infty)}
	\EndProof
}
\newpage
\subsection{Category}
\Page{
	\DeclareType{CellularMap}{
		\prod (X,\E,\varphi),(Y,\F,\psi) : \TYPE{CellComplex} \.
		?(X \Arrow{\TOP} Y)
	}
	\DefineType{f}{CellularMap}{\forall n \in \Int_+ \. f\Big(X^{\skull n}\Big) \subset Y^{\skull n}}
	\\
	\DeclareType{RegularCellMap}{
		\prod (X,\E,\varphi),(Y,\F,\psi) : \TYPE{CellComplex} \.
		\TYPE{CellularMap}\Big( (X,\E,\varphi),(Y,\F,\psi) \Big)
	}
	\DefineType{f}{RegularCellMap}{\forall e \in \E  \.  f(e) \in \F }
	\\
	\DeclareFunc{CWCategory}
	{
		\CAT
	}
	\DefineNamedFunc{CWCategory}{}{\mathsf{CW}}
	{
		\Big(
			\TYPE{CWComplex},
			C,
			\circ,
			\id
		\Big)
	}
	\\
	\DeclareFunc{CWCellularCategory}
	{
		\CAT
	}
	\DefineNamedFunc{CWCelluralCategory}{}{\mathsf{CWC}}
	{
		\Big(
			\TYPE{CWComplex},
			\TYPE{CellularMap},
			\circ,
			\id
		\Big)
	}
	\\
	\DeclareFunc{CWRegularCategory}
	{
		\CAT
	}
	\DefineNamedFunc{CWRegularCategory}{}{\mathsf{CWR}}
	{
		\Big(
			\TYPE{CWComplex},
			\TYPE{RegularMap},
			\circ,
			\id
		\Big)
	}
	\\
	\Theorem{RegularImageOfClosedCell}
	{
		\forall (X,\E,\varphi),(Y,\F,\psi) \in \CW \. 
		\forall r : (X,\E,\varphi) \Arrow{\CWR} (Y,\F,\psi) \.
		\forall e \in \E \. \NewLine \.
		\exists f \in \F :
		r\Big(\overline{e}\Big) = \overline{f}
	}
	\SayIn{f}{r(e)}{\F}
	\Say{[1]}{\THM{LimitImage}(r,e)\ByConstr^{-1} f}{r\Big(\bar e\Big) \subset \bar f}
	\Say{[2]}{\THM{CompactMappingTHM}(r,\overline{e})}{\TYPE{Closed}\bigg(Y, r\Big(\overline{e}\Big)\bigg)}
	\Say{[3]}{\THM{MontonicImage}(e,\bar e,r)\THM{ClosureIsSuper}(e)\ByConstr^{-1} f}{f \subset r(\bar e)}
	\Say{[4]}{\bd \FUNC{closure}[2][3]}{ \bar f \subset r(\bar e)}
	\Conclude{[5]}{\bd^{-1}\TYPE{SetEq}}{\bar f = r(\bar e)}
	\EndProof
	\\
	\Theorem{RegularImageIsSubcomplex}
	{
		\forall (X,\E,\varphi),(Y,\F,\psi) \in \CW \. 
		\forall r : (X,\E,\varphi) \Arrow{\CWR} (Y,\F,\psi) \.
		\Big( r(X), r(\E), \varphi_{\bullet,r} \Big)	
		\subset
		\Big( Y, \F, \psi \Big)
	}
	\NoProof
}\Page{
	\Theorem{RegularIsQuotientMap}
	{
		\forall (X,\E,\varphi),(Y,\F,\psi) \in \CW \. 
		\forall r : (X,\E,\varphi) \Arrow{\CWR} (Y,\F,\psi) \.
		\TYPE{QuotientMap}\Big(X,r(X),r\Big)	
	}
	\Assume{A}{?r(X)}
	\Assume{[1]}{\TYPE{Closed}\Big(X,r^{-1}(A)\Big)}
	\Say{[2]}{\bd^{-1} \TYPE{subsetTopology}(X,\E)[1]}
	{
		\forall e \in \E \. \TYPE{Closed}(\overline{e},\overline{e} \cap r^{-1}(A))
	}
	\Say{[3]}{\THM{CompactMapTheorem}}
	{
		\forall e \in \E \. \TYPE{Closed}( \overline{r(e)}, \overline{r}(e) \cap A )
	}
	\Conclude{[*.A]}
	{
		\THM{RegularImageIsSubcomplex}(r)\bd \CW\Big(r(X),r(\E),\psi_{\bullet,r}\Big)
	}
	{
		\TYPE{Closed}\Big(r(X),A\Big)
	}
	\DeriveConclude{[*]}
	{
		\bd^{-1} \TYPE{QuotientMap} \. 
		\TYPE{QuotientMap}\Big(X,r(X),r\Big)	
	}
	\EndProof
	\\
	\Theorem{CWcomplexHasCoproducts}
	{
		\TYPE{WithCoproducts}(\CWR)
	}
	\NoProof
	\\
	\Theorem{CWcomplexHasFiniteProducts}
	{
		\TYPE{withFiniteProducts}(\CWR)
	}
	\NoProof
}
\newpage
\section{Simplicial Complexes}
\subsection{Simplices}
\Page{
	\DeclareType{KSimplex}{\prod^\infty_{n=0} \prod^n_{k=-1} ?\Big([0, \ldots, k]_\Int \to \Reals^n\Big)}
	\DefineType{v}{KSimplex}{\dim \Aff v = k}
	\\
	\DeclareFunc{body}{\prod^\infty_{n=1} \prod^n_{k=-1} \TYPE{KSimplex}(n,k) \to \Convex(\Reals^n)}
	\DefineNamedFunc{body}{v}{v}{\conv v}
	\\
	\DeclareType{KFace}{\prod^\infty_{n=0} \prod^n_{k=-1} \TYPE{KSimplex}(n,k) \to  \prod^k_{t=-1} ?\TYPE{KSimplex}(n,t) }
	\DefineNamedType {f}{KFace}{\Lambda v : \TYPE{KSimplex}(n,k) \. \Lambda t \in [-1,\ldots,k]_\Int \. f \in \mathrm{face}(v) }
	{
		\NewLine \iff
		\Lambda v : \TYPE{KSimplex}(n,k) \.\Lambda t \in [-1,\ldots,k]_\Int \. \exists i : t \ToInj k : f = v_i 
	}
	\\
	\DeclareFunc{simplexBoundary}{\prod^\infty_{n=0} \prod^n_{k=-1} \TYPE{KSimplex}(n,k) \to \TYPE{Compact}(\Reals^n) }
	\DefineNamedFunc{simplexBoundary}{v}{\partial v}{\bigcup \mathrm{face}(v,k-1)}
	\\
	\DeclareFunc{simplexInterior}{\prod^\infty_{n=0} \prod^n_{k=-1} \TYPE{KSimplex}(n,k) \to ?\Reals^n}
	\DefineNamedFunc{simplexInterior}{v}{\intx v}{v \setminus \partial v}
	\\
	\Theorem{SimplexIsACell}{\forall n \in \Nat \. \forall k \in [n,1]_{\Nat} \.  \TYPE{ClosedCell}(k,v)   }
	\AssumeIn{x}{v}
	\Say{\Big(t,[1]\Big)}{\bd v (x)\bd^{-1}\conv}
	{\sum t : [0,\ldots,k] \to \Reals_+ \. 1 = \sum^k_{i=0} t_i \And x = \sum^k_{i=0} t_i v_i}
	\Conclude{\varphi(x)}{(t_1,\ldots,t_k)}{[0,1]^k}
	\Derive{\varphi}{\Intro(\to)}{ v \to [0,1]^k}
	\Say{[1]}{\bd^{-1}\TYPE{KSimplex}\bd^{-1}\TYPE{Simplex}}{\Big(\varphi : v \ToIso{\TOP} [0,1]^k\Big) }
	\Conclude{[*]}{\bd^{-1}\TYPE{ClosedXell}}{\TYPE{ClosedCell}(k,c)}
	\EndProof
	\\
	\DeclareType{SimplicialRetract}
	{
		\prod v : \TYPE{KSimplex}(n,k) \.
		\prod u \in \mathrm{face}(v,t) \.
		?\AFF(\Reals^n,\Reals^n)
	}
	\DefineType{R}{SimplicialRetract}{R(\im v) = \im u \And \forall i \in t \. R(u_i) = u_i}
}
\Page{
	\Theorem{SimplicialRetractTheorem}
	{
		\forall v : \TYPE{KSimplex}(n,k) \.
		\forall u \in \mathrm{face}(v,t) \. \NewLine \.
		\forall w \in \mathrm{face}(u,s) \.
		\forall R : \TYPE{SimplicialRetract}(u,w) \.
		\forall [0] :  s < t < k \.
		\exists f : w \sqcup_{R} v \ToIso{\TOP} v:
		f_{|w} = \id
	}
	\NoProof
	\\
	\Theorem{SimplexDiameterTHM}
	{
		\forall v : \TYPE{Kimplex}(n,k) \.
		\diam v = \max_{0 \le i,j \le k} \| v_i - v_j\|
	}
	\AssumeIn{x,y}{v}
	\Say{\Big( t, [1] \Big)}{\bd x \bd v \bd \FUNC{convexCombination}}
	{
		\sum t : [0,\ldots,k] \to [0,1] \. x = \sum^k_{i=0} t_i v_i
		\And
		1 = \sum^k_{i=0} t_i 
	}
	\Say{\Big( s, [2] \Big)}{\bd y \bd v \bd \FUNC{convexCombination}}
	{
		\sum s : [0,\ldots,k] \to [0,1] \. y = \sum^k_{i=0} s_i v_i \And 1 = \sum^k_{i=0} s_i 
	}
	\Conclude{\Big[(x,y).*\Big]}{ 
		[1][2] 
		\THM{EucleadeanNormConvexity}^{k+1}(n) 
		\bd^{-1}\FUNC{max}\| v_\bullet - v_\bullet \|  
		[1][2]
	}
	{
		\NewLine :
		\| x - y\| =
		\left\| \sum^k_{i=1} t_i v_i  - \sum^k_{i=1} s_i v_i  \right\| \le
		\sum^k_{i=0} t_i \left\|  v_i - \sum^k_{j=1} s_i v_j \right\| \le
		\sum^k_{i,j = 0} t_i s_j \| v_i - v_j \| \le
		\sum^k_{i,j = 0} t_i s_j \max_{0 \le l,m \le n} \| v_l - v_m \| =
		\NewLine = 
		\max_{0 \le l,m, \le n} \| v_l - v_m \|
	}
	\Derive{[1]}{\bd^{-1} \diam v}{ \diam v \le \max_{0 \le i,j \le k} \| v_i - v_j\|}
	\Say{[2]}{ \max \bd \diam v \bd v}{\max_{0 \le i,j \le k} \| v_i - v_j \| \le \diam v}
	\Conclude{[*]}{\THM{DoubleIneq}[1][2]}{ \diam v = \max_{0 \le i,j \le k} \|v_i - v_j \|  }
	\EndProof
	\\
	\DeclareFunc{barycentre}{\TYPE{KSimplex}(n,k) \to \Reals^n}
	\DefineNamedFunc{barycentre}{v}{\bar v}{ \sum^k_{i=0} \frac{v_i}{k+1}}
	\\
	\Theorem{BarycentreDistanceIneq}
	{
		\forall u : \TYPE{KSimplex}(n,t) \.
		\forall w \in \mathrm{face}(u,s) \.
		\| \bar u - \bar w \|  \le \frac{t-s}{t} \diam v
	}
	\Say{\Big(f,[1]\Big)}{\bd \mathrm{face}(u,s)(w)}
	{
		\sum f \in \mathrm{face}(u,t-s) \. u = f \sqcup w
	}
	\Say{[2]}{\bd \bar u [1] \bd^{-1} \bar w \bd^{-1} \bar f}
	{
		\bar u =  \frac{s}{t} \bar w + \frac{t- s}{t} \bar f
	}
	\Say{[3]}{ [2] \THM{NormHomogen} \bd \diam v \THM{FractionalDiffIneq}  }
	{
		\NewLine : 
		\| \bar u - \bar w  \| = 
		\left\| \frac{t-s}{t} \bar f - \frac{t-s}{t} \bar w \right\| =
		\frac{t-s}{t} \left\| \bar f - \bar w \right\| \le  
		\frac{t-s}{t} \diam w 
	}
	\EndProof
}
\Page{
	\Theorem{SimplexIntersectionTHM}
	{
		\forall n \in \Nat \.
		\forall  (k,v) : \prod^\infty_{n=1} \sum^\infty_{k=0} \TYPE{KSimplex}(n,k) \.
		\forall [0] : \forall n \in \Nat \. v^{n+1} \subset v^n \. \NewLine \.
		\exists t \in \Int_+ :
		\exists u : \TYPE{KSimplex}(n,t) \.
		u = \bigcap^\infty_{n=1} v^n
	}
	\Say{[1]}{\THM{AffineDimSUbse}}{\TYPE{Decreasing}(\Nat,\Int_+,k)}
	\Say{[2]}{\THM{NonNegInegersBoundedFromBelow}[1]}
	{
		\TYPE{Stabilizes}(\Nat,\Int_+,k)
	}
	\Say{k'}{\lim_{i \to \infty} k_i}{\Int_+}
	\Say{i}{\FUNC{enumerate}\{ i \in \Nat : k_i = k' \}}{\Nat \uparrow \Nat}
	\Say{v}{v^i}{\Nat \downarrow \TYPE{KSimplex}(n,k')}
	\Say{\Big([3], j\Big)}
	{
		\THM{BolzanoWeierstrassTHM}(v)
	}
	{
		\sum j : \Nat \uparrow j \. \TYPE{Convergent}(\Nat, \Reals^{k'n},v_j)
	}
	\Say{v}{v^j}{\TYPE{Convergent}(\Nat,\Reals^{k'\times n})}
	\Say{w}{\lim_{i \to \infty} v^i}{\Reals^{k'\times n}}
	\Say{\Big( k'',w',[4]\Big)}{\THM{ConvexlyIndependenExists}}
	{
		\sum k'' \in [0,\ldots,k'] \.
		\sum w' \subset w \. \NewLine \.  
		 \TYPE{ConvexlyIndependent}(k'',\Reals^n,w')
	}
	\Say{\Big(f,\sigma,[5]\Big)}{\THM{NaturalSimplexIsoExists}(n,k)}
	{
		\sum f : \prod^\infty_{i=1} \Reals^n \Arrow{\AFF{\Reals}} \Reals^n \.
		\sum \sigma : \Nat \to S_{k'} \.
		\NewLine \. 
		\forall i \in \Nat \. 
		\forall j \in k'  \. 
		f(v^i_j) = \triangle_{\sigma(j)}
	}
	\NoProof
}
\newpage
\subsection{Euclidean Simplicial Complexes}
\Page{
	\DeclareType{\SC}{\prod^\infty_{n=0}?\prod^n_{k=-1} ?\TYPE{KSimplex}(n,k) }
	\DefineType{\triangle}{\SC}
	{
		\forall k \in [-1,\ldots,n] \.
		\forall s \in \triangle_k \.
		\forall t \in [-1,\ldots,k] \.
		\forall f \in \mathrm{face}(s,t) \.
		f \in \triangle_t  
		\And \NewLine \And
		\forall k,t \in [-1,\ldots,n] \.
		\forall s \in \triangle_k \.
		\forall s' \in \triangle_t \.
		\exists m \in \Big[ -1,\ldots,\min(k,t) \Big] :
		s \cap s' \in \mathrm{face}(s,m) \cap \mathrm{face}(s',m)
		\And \NewLine \And
		\TYPE{LocallyFinite}\left(\Reals^n ,\bigcap^\infty_{k=0} \triangle_k\right)
	}
	\\
	\DeclareFunc{simplexSet}{\SC(n) \to ??\Reals^n}
	\DefineNamedFunc{simplexSet}{\triangle}{\triangle}{\bigcup_{k=0}^n \triangle_k}
	\\
	\DeclareFunc{polyhedronOf}{\SC(n) \to ?\Reals^n}
	\DefineNamedFunc{[olyhedronOf}{\triangle}{\langle \triangle \rangle}
	{ \bigcup_{k=0}^n \bigcup \triangle_k   }
	\\
	\DeclareType{FiniteSimplicialComplex}
	{
		?\SC(n)
	}
	\DefineNamedType{\triangle}{FiniteSimplicialComplex}{\triangle < \infty}{|\triangle| < \infty}
	\\
	\DeclareFunc{simplitialComplexDimension}{\SC(n) \to [0,\ldots,n]}
	\DefineNamedFunc{simplicialComplexDimension}{\triangle}{\dim \triangle}
	{ \max \Big\{ k \in [0,\ldots,n] \Big| \triangle_k \neq \emptyset \Big\}}
	\\
	\DeclareType{SimplitialSubcomplex}{\SC(n) \to ?\SC(n)}
	\DefineNamedType{\triangle'}{SimplitialSubcomplex}
	{
		\Lambda \triangle  : \SC(n) \. 
		\triangle' \subset \triangle 
	}
	{
	        \forall k \in [-1,\ldots,n] \. 
		\triangle'_k \subset \triangle_k
	}
	\\
	\DeclareFunc{kSkeletn}{\prod \triangle \in \SC(n) \.  [-1,\ldots,n] \to \TYPE{SimplicialSubcomplex}(n)}
	\DefineNamedFunc{kSkeleton}{k}{\triangle^{\skull k}}
	{
		\Lambda t \in [-1,\ldots,n] \. \If t < k \Then \emptyset \Else \triangle_t
	}
	\\
	\DeclareFunc{asCellComplex}{ \SC(n) \to \TYPE{CellComplex} }
	\DefineNamedFunc{asCellComplex}{\triangle}{\C\C(\triangle)}
	{
		\Big( 
			\langle \triangle \rangle,
			\Lambda k \in [-1,\ldots,\infty] \. 
				\If k > n 
				\Then \emptyset
				\Else \intx \triangle_k, 
			\NewLine,
			k \mapsto \intx t \mapsto \THM{SimplexIsCell}(n,t)_{|\mathbb{B}^k}
		\Big)
	}
	\\
	\Conclude{\TYPE{Triangulation}}{
		\prod_{X \in \TOP} \prod \triangle : \SC(n) \. X \ToIso{\TOP} \langle \triangle \rangle
	}
	{
		\NewLine :
		\prod_{n=-1}^\infty \TOP \to \SC(n) \to \Type 
	}
	\\
	\DeclareType{Triangulable}{?\TOP}
	\DefineType{X}{Triangulable}{
		\exists n \in[-1,\ldots,+\infty) \.
		\exists \triangle \in \SC(n) \.
		\exists \TYPE{Triangulation}(X,\triangle)
	}
}
\Page{
	\Theorem{2ManifoldIsTriangulabe}
	{
		\forall M \in \TOPM(2) \.
		\TYPE{Triangulable}(M)
	}
	\NoProof
	\\
	\Theorem{3ManifoldIsTriangulabe}
	{
		\forall M \in \TOPM(3) \.
		\TYPE{Triangulable}(M)
	}
	\NoProof
}
\newpage
\subsection{Simplicial Maps}
\Page{
	\DeclareFunc{vertices}{\prod^\infty_{n=0} \prod^n_{k=-1} \TYPE{KSimplex}(n,k) \to \TYPE{Finite}(\Reals^n) }
	\DefineNamedFunc{vertices}{s}{\mathrm{vert}(s)}{\mathrm{face}(s,0)}
	\\
	\Conclude{\TYPE{VertexMap}}{
		\prod s : \TYPE{KSimplex}(n,m) \.
		\prod t :\TYPE{KSimplex}(n,k)  \.
		\mathrm{vert}(s) \to \mathrm{vert}(t)
	}
	{
		\NewLine : 
		\prod^\infty_{n=-1} 
		\prod^n_{m,k=-1} 
		\TYPE{KSimplex}(n,m)
		\times
		\TYPE{KSimplex}(n,k)
		\to
		\Type
	}
	\\
	\Theorem{VertexMapExtension}
	{
		\forall n,m \in [0,+\infty) \. 
		\forall l,k \in [-1,n) \. 
		\forall s \in \TYPE{KSimplex}(n,l) \.
		\forall t \in \TYPE{KSimplex}(m,k) \. \NewLine \.
		\forall \sigma : \TYPE{VertexMap}(s,t) \.
		\exists f : \Reals^n \Arrow{\Reals\hyph\AFF} \Reals^m \. 
		f_{|\mathrm{vert}(s)} = \sigma
	}
	\NoProof
	\\
	\DeclareType{SimplicialMap}
	{  
		\prod^\infty_{n,m=0} 
		\prod \triangle : \SC(n) \.
		\prod \triangle' : \SC(n) \.
		?(\langle \triangle \rangle
		\Arrow{\TOP}
		\langle \triangle' \rangle)
	}
	\DefineType{\sigma}{SimplicialMap}
	{
		\forall k \in [-1,\ldots,n] \.
		\forall s \in \triangle_k \.
		\sigma(s) \in \triangle' 
		\And
		\exists f : \Reals^n \Arrow{\Reals\hyph\AFF} \Reals^m \.
		f(s) = \sigma(s)
	}
	\\
	\DeclareFunc{vertexMap}
	{
		\TYPE{SimplicialMap}(n,m,\triangle,\triangle') 
		\to 
		\triangle^{\skull 0}
		\to
		{\triangle'}^{\skull 0}
	}
	\DefineNamedFunc{vertexMap}
	{
		\sigma
	}
	{
		\mathrm{vert}(\sigma)
	}
	{
		\sigma_{|\triangle^{\skull 0}}
	}
	\\
	\Theorem{SimplicialMapExtensionTHM}
	{
		\forall n,m \in \Int_+ \. 
		\forall \triangle : \SC(n) \. 
		\forall \triangle' : \SC(m) \.\NewLine \. 
		\forall f : \triangle^{\skull 0} \to {\triangle'}^{\skull 0} \.
		\exists! \sigma : \TYPE{SimplicialMap}(\triangle,\triangle') :
		\mathrm{vert}(\sigma) = f
	}
	\NoProof
}
\newpage
\subsection{Abstract Simplicial Complexes}
\Page{
	\DeclareType{\ASC}{\prod T : \Type \. ??\TYPE{Finite}(T)}
	\DefineType{C}{\ASC}{\forall A \in C \. \forall B \subset A \. B \in C}
	\\
	\DeclareType{FiniteAbstractComplex}{?\ASC(T)}
	\DefineType{C}{FiniteAbstractComplex}{|C| < \infty}
	\\
	\DeclareType{LocallyFiniteAbstractComplex}{?\ASC(T)}
	\DefineType{C}{LocallyFiniteAbstractComplex}{\forall A \in C \. \forall a \in A \. \Big| \{ B \in C : a \in A\} \Big|}
	\\
	\DeclareFunc{abstractSimplexDimension}{\prod C : \ASC(T) \. C \to [-1,\ldots,+\infty) }
	\DefineNamedFunc{abstractSimplexDimension}{A}{\dim_C A}{|A| - 1}
	\\	
	\DeclareFunc{abstractSimplecialDimension}{\ASC(T)  \to [-\infty,\ldots,+\infty] }
	\DefineNamedFunc{abstractSimplecialDimension}{A}{\dim C}{ \sup_{A \in C} \dim A }
	\\	
	\DeclareType{FiniteDimensionalAbstractComplex}{?\ASC(T)}
	\DefineType{C}{FiniteDimensionalAbstractComplex}{ |\dim C| < \infty }
	\\
	\DeclareFunc{vertexSet}{\ASC(T) \to ?T}
	\DefineNamedFunc{vertexSet}{C}{\langle C \rangle}{ \bigcup_{A \in C} A }
	\\
	\DeclareType{AbstractSimplicialMap}{
		\prod T,S : \Type \. 
		\prod C : \ASC(T) \. \NewLine \. 
		\prod C' : \ASC(S) \.  
		?(C \to C')
	}
	\DefineType{F}{AbstractSimplicialMap}{
		\exists f : \langle C \rangle \to \langle C' \rangle :
		\forall A \in C \. 
		F(A) = f(A)
	}
	\\
	\DeclareFunc{abstractVertexMap}{
		\prod T,S : \Type \. 
		\prod C : \ASC(T) \. \NewLine \. 
		\prod C' : \ASC(S) \.  
		\TYPE{AbstractSimplicialMap}(C,C') \to
		\langle C \rangle \to \langle C' \rangle 
	}
	\DefineNamedFunc{AbstractVertexMap}{F}{\langle F \rangle}
	{
		\bd \TYPE{AbstractSimplicialMap}(C,C',F)
	}
	\\
	\DeclareFunc{vertexSchema}{\prod^\infty_{n=0} \SC(n) \to \ASC(\Reals^n) }
	\DefineNamedFunc{vertexSchema}{\triangle}{\mathcal{VS}(\triangle)}
	{ 
		\Big\{ \im s \Big| k \in [0,\ldots,n], s \in \triangle_s    \Big\}
	}
}
\Page{
	\DeclareType{AbstractSimplicialIsomorphism}{
		\prod T,S : \Type \. 
		\prod C : \ASC(T) \. \NewLine \. 
		\prod C' : \ASC(S) \.  
		?\TYPE{AbstractSimplicalMap}
	}
	\DefineType{F}{AbstractSimpliciaIsomorphism}{
		\TYPE{Bijection}(C,C',F)
	}
	\\
	\DeclareType{IsomorphicASC}{
		\prod T,S : \Type \. 
		? \ASC(T) \times \ASC(S)
	}
	\DefineNamedType{C,C'}{IsomorphicASC}{C \cong C'}{
		\exists \TYPE{AbstractSimplicialIsomorphism}(C,C)
	}
	\\
	\DeclareType{GeometricRealization}{\ASC(T) \to ?\sum^\infty_{n=0} \SC(n) }
	\DefineType{(n,\triangle)}{GeometricRealization}{\Lambda C \in \ASC(T) \. \mathcal{VS}(\triangle) \cong C}
	\\
	\Theorem{GeometricRealizationTHM}
	{
		\forall C \in \TYPE{FiniteAbstractComplex}(T) \.
		\exists \TYPE{GeometricRealization}(C) 
	}
	\NoProof
}
\newpage
\section{Compact Surfaces}
\subsection{Polygones}
\Page{
	\Conclude{\CS}{\TOPM(2) \And \TYPE{Compact} \And \TYPE{Connected} \And \TYPE{NonEmpty} }{\Type}
	\\
	\DeclareType{Polygon}{??\Reals^2}
	\DefineType{P}{Polygon}{ 
		P \cong_{\TOP} \Sphere^1 
		\And 
		\exists \triangle : \SC(2) : P = \langle \triangle \rangle
	}
	\\
	\DeclareFunc{vertices}{\TYPE{Polygon} \to \TYPE{Finite}(\Reals^2)}
	\DefineNamedFunc{vertices}{P}{\mathbf{V}(P)}{\triangle_0}
	\\
	\DeclareFunc{edges}{\TYPE{Polygon} \to \TYPE{Finite}(\TYPE{KSimplex}(2,1))}
	\DefineNamedFunc{edges}{P}{\mathbf{E}(P)}{\triangle_1}
	\\
	\DeclareType{PolygonalRegion}{?\TYPE{Compact}(\Reals^2)}
	\DefineType{R}{PolygonalRegion}{\TYPE{Polygon}(\boundary R)}
	\\
	\DeclareType{PolygonalComplex}{ ?\TOP  }
	\DefineType{X}{PolygonalComplex}
	{
		\exists n \in \Nat :
		\exists P : [1,\ldots,n] \to \TYPE{PolygonalRegion} :
		\NewLine :
		\exists E : 
		?(\bigsqcup^n_{i,j=1}\mathbf{E}(P_i) 
			\times \bigsqcup^n_{i=1} \mathbf{E}(P_i)) :
		\exists A : \prod (e,f) \in E \. \TYPE{SymplecticMap}(e,f) :
		X = \frac{\bigsqcup^n_{i=1} P_i}{A}
		\And \NewLine \And
		\TYPE{DiogonalFree}(E)
	}
	\\
	\Theorem{PolygonalComplexIsCW}
	{
		\forall X : \TYPE{PolygonalComplex} \.
		\exists (Y,\E,\varphi) : \TYPE{FiniteCWComplex} :
		X = Y
	}
	\Say{\Big(n,P,E,A,[1] \Big)}
	{ \bd \TYPE{PolygonalComplex}(X) }
	{
		\sum^\infty_{n=1} 
		\sum P : [1,\ldots,n] \to \TYPE{Polygonal Region} \. \NewLine \. 
		\sum E \subset  \bigsqcup^n_{i,j=1} \mathbf{E}(P_i) \times \mathbf{E}(P_j) \.
		\sum A : \prod (e,f) \in E \. \TYPE{SymplecticMap}(e,f) \.
		X = \frac{\bigsqcup^n_{i=1} P_i}{A}
	}
	\Say{\E_2}{\left\{  P_i \setminus \bigcup \mathbf{E}(P_i)   \bigg| i \in [1,\ldots,n]\right\}}
	{
	      ?\TYPE{Cell}(2)
	}
	\Say{\E_1}{ \frac{\bigsqcup^n_{i=1} \intx \mathbf{E}(P_i)}{\intx E} }
	{
		?\TYPE{Cell}(1)
	}
	\Say{\E_0}{ \frac{\bigcup^n_{i=1} \mathbf{V}(P_i)}{A_{|}} }{ ?\TYPE{Cell}(0)}
	\AssumeIn{i}{\{0,1,2\}}
	\AssumeIn{e}{\E_i}
	\Say{\Big(j,p,[2]\Big)}{\ByConstr \E_i}{\sum j \in [1,\ldots,n] \. \sum p \subset P_j \. e = [p]  }
	\Say{[3]}{\bd \TYPE{Pushout}[1]\ByConstr \E_i(e)}{
			\forall Q,Q' \in [p] \. 
			\forall q \in \bar Q \. 
			\forall q' \in \bar Q' \.   
			\pi_e(q) = \pi_e(q) \Imply
			\pi_A(q) = \pi_A(q) 
		}
	\Conclude{\varphi_{i,e}}{  \pi_{A,[3]}   }
	{
		\bar e \to X
	}
	\Derive{\varphi}
	{
		\Intro\Act{\prod}
	}
	{
		\prod_{i=0}^3 \prod_{e \in \E_n} \bar e  \to X
	}
	\Conclude{[*]}{\bd^{-1} \TYPE{FiniteCWComplex}() }
	{
		\TYPE{FiniteCWComplex}(X,\E,\varphi)
	}
	\EndProof
}
\Page{	
	\DeclareFunc{polygonalDegree}{\TYPE{PolygonalComplex} \to \Nat}
	\DefineNamedFunc{polygonalDegree}{X}{\deg X}
	{n \quad \where \quad (n,P,E,A) = \bd \TYPE{PolygonalComplex}(X)}
	\\
	\DeclareFunc{polygons}{
		\prod X : \TYPE{PolygonalComplex} \. 
		\Big[1,\ldots,\deg X\Big] \to \TYPE{Polygone}
	}
	\DefineNamedFunc{polygonals}{}{\mathbf{P}(X)}
	{P \quad \where \quad (n,P,E,A) = \TYPE{PolygonalComplex}(X)}
	\\
	\DeclareFunc{edgeEquivalence}{
		\prod X : \TYPE{PolygonalComplex} \. 
		?\left( \bigsqcup_{i=1} \mathbf{E \; P}_i(X) \times \bigsqcup_{i=1} \mathbf{E \; P}_i(X) \right)
	}
	\DefineNamedFunc{edgeEquivalence}{}{\mathbf{E}(X)}
	{E \quad \where \quad (n,P,E,A) = \TYPE{PolygonalComplex}(X)}
	\\
	\Theorem{PolygonalComplexIsCompactSurfaceCondition}
	{
		\forall X : \TYPE{PolygonalComplex} \. \NewLine \. 
		\TYPE{Bijection} \Imply \CS(X)
	}
	\Say{\Big(  \E, \varphi, [1] \Big)}{\THM{PolygonalComplexIsCWComplex}(X)}
	{
		\ldots \TYPE{FiniteCWComplex}(X,\E,\varphi)
	}
	\Say{[2]}{\THM{FiniteCWComplexIsCompact}(X,\ldots)}
	{
		\TYPE{Compact}(X)
	}
	\AssumeIn{x}{X}
	\Say{\Big(i,e,[3]\Big)}{\bd \TYPE{Partition}(X,\E)}
	{
		\sum i \in \{0,1,2\} \. 
		\sum e \in \E_i \.
		x \in e
	}
	\Conclude{[x.*]}{\THM{FanTransform}\bd \TYPE{PolygonalComplex}(X)}
	{
		\exists U \in \U(x) \. \TYPE{Cell}(2,X)
	}
	\Derive{[3]}{\bd^{-1}\TYPE{LocallyEuclidean}}{\TYPE{LocallyEuclidean}(X)}
	\Conclude{[*]}{\bd^{-1}\CS[3]}{\CS(X)}
	\EndProof
	\\
	\DeclareFunc{covariantEdgeAssociation}{\prod a,b : \TYPE{KSimplex}(2,1) \. \TYPE{SimplecticMap}(a,b) }
	\DefineNamedFunc{covariantEdgeAssociation}{}{a \uparrow b}{\THM{AffineMapDetermination}(a_0,b_0) \where \det a \uparrow b > 0}
	\\
	\DeclareFunc{contravariantEdgeAssociation}{\prod a,b : \TYPE{KSimplex}(2,1) \. \TYPE{SimplecticMap}(a,b) }
	\DefineNamedFunc{contravariantEdgeAssociation}{}{a \downarrow b}{\THM{AffineMapDetermination}(a_0,b_0) \where \det a \downarrow b < 0}
	\\
	\DeclareFunc{standardSquare}{\TYPE{PolygonalRegion}}
	\DefineNamedFunc{standardSquare}{}{I^2}{[0,1]^2}
	\\
	\Conclude{A}{(0,0)}{?\Big(\boundary I^2\Big)_0}
	\Conclude{B}{(0,1)}{?\Big(\boundary I^2\Big)_0}
	\Conclude{C}{(1,1)}{?\Big( \boundary I^2\Big)_0}
	\Conclude{D}{(1,0)}{?\Big( \boundary I^2 \Big)_0 }
	\\
	\DeclareType{InjectivePair}{\prod_{ X,Y,Z \in \SET } \. ?\Big( (X \to Z) \times (X \to Z)\Big)}
	\DefineType{(f,g)}{InjectivePair}{\TYPE{Injective}\Big( X \times Y,Z^2,f \times g \Big)}
}
\Page{
	\DeclareFunc{simplePolygonalComplex}{
		\bigg(\TYPE{InjectivePair}\Big(\{1,2\},\{1,2\},\mathbf{V}(I^2)\Big) \bigg)
		\times
		\Big(\{1,2\} \to \{\uparrow,\downarrow \Big)
		\to \CS
	}
	\DefineNamedFunc{simplePolygonalComplex}{a,b,|}{\mathrm{spc}(a,b,|)}
	{
		\frac{I^2}{a|b}
	}
	\\
	\Theorem{SphereAsPolygonalComplex}
	{
		\mathrm{spc}( AB \uparrow BC, AD \uparrow DC  ) \cong_{\TOP} \Sphere^2
	}
	\NoProof
	\\
	\Theorem{SphereAsPolygonalComplex}
	{
		\mathrm{spc}( AB \uparrow BC, AD \uparrow DC  ) \cong_{\TOP} \Sphere^2
	}
	\NoProof
	\\
	\Theorem{TorusAsPolygonalComplex}
	{
		\mathrm{spc}( AB \uparrow CD, BC \uparrow AD  ) 
		\cong_{\TOP} 
		\Sphere^1 \times \Sphere^1 
	}
	\NoProof
	\\
	\Theorem{ProjectiveSpaceAsPolygonalComplex}
	{
		\mathrm{spc}( AB \downarrow CD, BC \downarrow AD  ) 
		\cong_{\TOP} 
		\mathbb{RP}^{2}  
	}
	\NoProof
	\\
	\Theorem{ProjectiveSpaceAsPolygonalComplex}
	{
		\mathrm{spc}( AB \downarrow CD, BC \downarrow AD  ) 
		\cong_{\TOP} 
		\mathbb{RP}^{2}  
	}
	\NoProof
	\\
	\DeclareFunc{bottelOfKlein}
	{
		\TYPE{PolygonalComplex}
	}
	\DefineNamedFunc{bottelOfKlein}{}{\mathbf{K}}
	{    
		\mathrm{spc}(AB \downarrow CD, BC \uparrow AD  )
	}
}
\newpage
\subsection{Connected Sums}
\Page{
	\DeclareFunc{CircledRegion}
	{
		\prod M \in \TOPM(n) \. ??M
	}
	\DefineType{A}{CircledRegion}{
		\TYPE{Cell}(n,A) 
		\And 
		\boundary A \cong_{\TOP} \Sphere^{n-1}
	}
	\\
	\Theorem{CicledRegionExists}
	{
		\forall M \in \TOPM(n) \. 
		\exists \TYPE{CircledRegion}(M)
	}
	\NoProof
	\\
	\Theorem{ConnectedSumsIsWellDefined}
	{
		\forall n \in \Nat \.
		\forall M,N \in \TOPM(n) \And \TYPE{Connected} \.
		\NewLine \. 
		\forall U,U' : \TYPE{CircledRegion}(M) \. 
		\forall V, V' : \TYPE{CircledRegion}(N) \. \NewLine \. 
		\frac{(M \setminus U) \bigsqcup (N \setminus V)}
		{
			\psi^{-1} \circ \varphi
		}
		\cong_{\TOP}
		\frac{(M \setminus U') \bigsqcup (N \setminus V')}
		{
			{\psi'}^{-1} \circ \varphi'
		}
		\NewLine
		\where \quad
		\varphi,\varphi' = \bd \TYPE{CircledRegion}(M,U \And U');
		\psi, \psi' = \bd \TYPE{CirculedRegion}(N,V \And V')
	}
	\NoProof
	\\
	\DeclareFunc{connectedSum}
	{
		\TOPM{n} \times \TOPM(n) \to \TOPM(n)
	}
	\DefineNamedFunc{connectedSum}{M,N}{A \# B}
	{   
		\frac{(M \setminus U) \bigsqcup (N \setminus V)}
		{
			\psi^{-1} \circ \varphi
		}
		\NewLine
		\where 
		\quad
		U,V = \THM{CircledRegionExists}(M \And N) ;
		\NewLine
		\varphi = \bd \TYPE{CircledRegion}(M,U) ;
		\NewLine
		\psi = \bd \TYPE{CircledRegion}(N,V) ;
	}
}
\newpage
\subsection{Polygonal Presentation}
\Page{
	\DeclareType{GeneralPolygonalPresentation}
	{
		\prod T : \Type \.
		\prod P : \TYPE{Finite}(T)
		?\sum^\infty_{n=0} 
		\sum k : [1,\ldots,n] \to \Nat \. \NewLine \. 
		\prod^n_{i=1} [1,\ldots,k_i] \to \Big(P \times \{1\}\Big) \sqcup \Big(P \times \{-1\} \Big) 
	}
	\DefineNamedType{(n,k,w)}{GeneralPolygonalPresentation}
	{
		(n,k,w) = \langle P | w_1, \ldots, w_n \rangle
	}
	{
		\NewLine \iff
		\forall i \in [1,\ldots,n] \. k_i \ge 3
		\And 
		\forall p \in P \.
		\exists i \in [1,\ldots,n] :
		\exists j \in [1,\ldots,k_i] :
		w_{i,j} = (p,1) | w_{i,j} = (p,-1)
	}
	\\
	\DeclareType{SpecialPolygonalPresentation}
	{
		\prod T : \Type \.
		\prod p : T \.
		?\Big(\{1,2\} \to \Big\{ (p,1), (p,-1) \Big\}\Big)
	}
	\DefineNamedType{w}{GeneralPolygonalPresentation}
	{
		w = \langle p | w_1w_2 \rangle
	}
	{
		\top
	}
	\\
	\Conclude{\TYPE{PolygonalPresentation}}{
		\prod T : \Type \. 
		\TYPE{GeneralPolygonalPresentation}\Big| 
		\NewLine \Big|
		\LOGIC{MaybeIf}\Big(\TYPE{Singleton},\TYPE{SpecialPolygonalPresentation}\Big) }
	{
		\prod T  : \Type \. \TYPE{Finite}(T) \to \SET
	}
	\\
	\DeclareType{PolygonalWord}{\prod_{ X \in \SET} \. ?\mathrm{Lang}(X) }
	\DefineType{w}{PolygonalWord}{|w| \ge 3}
	\\
	\DeclareFunc{wordPolygon}
	{
		\prod T : \Type \.
		\prod P : \TYPE{Finite}(T) \. \NewLine
		\TYPE{NonEmptyWord}\Big(P \times \{1\}\Big) \sqcup \Big(P \times \{-1\} \Big) \to \TYPE{Polygon}
	}
	\DefineNamedFunc{wordPolygon}{(k,w)}{\mathbf{P}(w)}
	{\THM{RegularPolyonDeterminationByCenterAndRay}\Big( 2,\mathrm{len}(w), \{0\} \times \Reals_+ \Big)}
	\\
	\DeclareFunc{wordPresentation}
	{
		\prod T : \Type \.
		\prod P : \TYPE{Finite}(T) \. \NewLine \. 
		\prod w : \TYPE{NonEmptyWord}\Big(P \times \{1\}\Big) \sqcup \Big(P \times \{-1\} \Big)  \.
		\Big[1, \ldots,\mathrm{len}(w)\Big] \to \mathbf{E\; P}(w)
	}
	\DefineNamedFunc{woedPresentation}{i}{\mathbf{E}_i(w)}
	{\FUNC{enumerateCounterClockwiseFromRay}\Big( \mathbf{E \; P}(w), \{0\} \times \Reals_+ \Big)(i)}
	\\
	\DeclareFunc{polygonalRealization}
	{
		\TYPE{PolygonalRepresentation} \to \TYPE{PolygonalComplex}
	}
	\DefineNamedFunc{polygonalRealization}{\langle x | xx \rangle}
	{ \mathrm{real} \langle x | xx \rangle }{\mathbb{RP}^2}
	\DefineNamedFunc{polygonalRealization}{\langle x | x^{-1}x \rangle}
	{ \mathrm{real} \langle x | x^{-1}x \rangle }{\Sphere^2}
	\DefineNamedFunc{polygonalRealization}{\langle x | xx^{-1} \rangle}
	{ \mathrm{real} \langle x | xx^{-1} \rangle }{\Sphere^2}
	\DefineNamedFunc{polygonalRealization}{\langle x | x^{-1}x^{-1} \rangle}
	{ \mathrm{real} \langle x | x^{-1}x^{-1} \rangle }{\mathbb{RP}^2}
	\DefineNamedFunc{polygonalRealization}{\langle X | w_1,\ldots, w_n \rangle}
	{ \mathrm{real} \langle X | w_1, \ldots, w_n \rangle }
	{
		\frac{\bigsqcup^n_{i=1} \mathbf{P}(w_i)}
		{
			A
		}
		\NewLine
		\where
		\quad
		A = 
			\bigg\{ \mathbf{E}_j(w_i) \downarrow \mathbf{E}_l(w_k) \bigg| i,k \in [1,\ldots,n],  
			j \in \Big[1,\ldots,|w_i|\Big], l \in \Big[1,\ldots,|w_k|\Big], w_{i,j} = w_{k,l} \bigg\}   
		\sqcup \NewLine \sqcup
			\bigg\{ \mathbf{E}_j(w_i) \uparrow \mathbf{E}_l(w_k) \bigg| i,k \in [1,\ldots,n],  
			j \in \Big[1,\ldots,|w_i|\Big], l \in \Big[1,\ldots,|w_k|\Big], 
			w_{i,j,1} = w_{k,l,1} \And w_{i,j,2} \neq w_{k,l,2} \bigg\}   
	}
}
\Page{
	\DeclareType{SurfacePresentation}{?\TYPE{PolygonalPresentation}}
	\DefineType{\langle X \Big| w_1, \ldots, w_n \rangle }{SurfacePresentation}
	{
		\forall x \in X \. 
			\Bigg| 
				\bigg\{ 
					(i,j) 
					\bigg| 
					i \in [1,\ldots,n ],
					j \in \Big[1, \ldots, |w_i| \Big],
					w_{i,j,1} = x
				\bigg\} 
			\Bigg|
	}
	\\
	\Theorem{SurfacePresentationProperty}
	{
		\forall X : \TYPE{SurfacePresentation} \.
		\CS\Big(\mathrm{real}(X) \Big)
	}
	\NoProof
	\\
	\DeclareFunc{relabling}{
		\prod \Big\langle X \Big|w_1,\ldots,w_n \Big\rangle : \TYPE{PolygonalPresentation} \.
		X \to X^\c \to \TYPE{PolygonalPresentation}
	}
	\DefineFunc{relabling}{a,b}
	{
		\bigg\langle \Big(X \setminus \{a\}\Big) \sqcup \{b\}\bigg| w'_1,\ldots w'_n  \bigg\rangle
		\NewLine
		\where
		\quad
		w' = \Lambda i \in [1,\ldots,n] \. \Lambda j \in \Big[1,\ldots, | w_i |\Big] \. 
		\If w_{i,j,1} == a \Then (b,w_{i,j,2}) \Else w_{i,j}
	}
	\\
	\Theorem{RelablingPreservesRealization}
	{
		\forall \Big\langle X | w_1,\ldots,w_n \Big\rangle : \TYPE{PolygonalPresentation} \. \NewLine \. 
		\forall a \in X \.
		\forall b \in X^\C \.
		\mathrm{real}\;\FUNC{relabling}\Big( \langle X | w_1,\ldots, w_n \rangle, a, b \Big) \cong_\TOP
		\mathrm{real} \;  \langle X | w_1, \ldots,w_n \rangle
	}
	\NoProof
	\\
	\DeclareFunc{subdividing}{
		\prod \Big\langle X \Big|w_1,\ldots,w_n \Big\rangle : \TYPE{PolygonalPresentation} \.
		X \to X^\c \to \TYPE{PolygonalPresentation}
	}
	\DefineFunc{subdividing}{a,b}
	{
		\bigg\langle  X \sqcup \{b\}\bigg| w'_1,\ldots w'_n  \bigg\rangle
		\NewLine
		\where
		\quad
		w' = \Lambda i \in [1,\ldots,n] \. \Big(\Lambda (a,1)  \. \Lambda (a,-1) \. w_i  \Big)(ab,b^{-1}a^{-1})
	}
	\\
	\Theorem{SubdivisionPreservesRealization}
	{
		\forall \Big\langle X | w_1,\ldots,w_n \Big\rangle : \TYPE{PolygonalPresentation} \. \NewLine \. 
		\forall a \in X \.
		\forall b \in X^\C \.
		\mathrm{real}\;\FUNC{subdivision}\Big( \langle X | w_1,\ldots, w_n \rangle, a, b \Big) \cong_\TOP
		\mathrm{real} \;  \langle X | w_1, \ldots,w_n \rangle
	}
	\NoProof
	\\
	\DeclareType{ConsalidatablePair}{
		\prod \langle X | w_1,\ldots,w_n \rangle \to ?(X \times X)
	}
	\DefineType{(a,b) }{ConsalidatablePair}
	{
		\forall i \in [1,\ldots,n] \. 
		\forall j \in \Big[1,\ldots,|w_{i}|\Big] \. \NewLine \. 
		\Big(w_{i,j} = (a,1) \Imply w_{i,j+1} = (b,1) \Big)
		\And
		\Big(w_{i,j} = (a,-1) \Imply w_{i,j-1} = (b,-1) \Big)
	}
	\\
	\DeclareFunc{consalidation}{
		\prod \langle X | w_1,\ldots,w_n \rangle : \TYPE{PolygonalPresentation} \.
		\TYPE{ConsalidatablePair}\langle X | w_1,\ldots,w_n \rangle \to \TYPE{PolygonalPresentation}
	}
	\DefineFunc{consalidating}{a,b}
	{
		\bigg\langle X \setminus \{b\}\bigg| w'_1,\ldots w'_n  \bigg\rangle
		\NewLine
		\where
		\quad
		w' = \Lambda i \in [1,\ldots,n] \. \Big(\Lambda ab  \. \Lambda b^{-1}a^{-1} \. w_i  \Big)(a,a^{-1})
	}
}\Page{
	\Theorem{ConsalidatingPreservesRealization}
	{
		\forall \langle X | w_1,\ldots,w_n \rangle : \TYPE{PolygonalPresentation} \. \NewLine \. 
		\forall (a,b) : \TYPE{ConsalidatablePair} \/
		\mathrm{real}\;\FUNC{consalidating}\Big( \langle X | w_1,\ldots, w_n \rangle, a, b \Big) \cong_\TOP
		\mathrm{real} \;  \langle X | w_1, \ldots,w_n \rangle
	}
	\NoProof
	\\
	\DeclareFunc{reflection}{
		\TYPE{PolygonalPresentation} \to \TYPE{PolygonalPresentation}
	}
	\DefineFunc{reflection}{\langle X | w_1, \ldots, w_n \rangle}
	{
		\langle X | w_1^{-1},\ldots,w_n^{-1}  \rangle
	}
	\\
	\Theorem{ReflectionPreservesRealization}
	{
		\forall \langle X | w_1,\ldots,w_n \rangle : \TYPE{PolygonalPresentation} \. \NewLine \. 
		\mathrm{real}\;\FUNC{reflection} \langle X | w_1,\ldots, w_n \rangle \cong_\TOP
		\mathrm{real} \;  \langle X | w_1, \ldots,w_n \rangle
	}
	\NoProof
	\\
	\DeclareFunc{rotation}{
		\prod \langle X | w_1,\ldots,w_n \rangle : \TYPE{PolygonalPresentation} \. 
		[1,\ldots,n] \to \TYPE{PolygonalPresentation}
	}
	\DefineFunc{rotation}{ k }
	{
		\langle X | w'_1, \ldots, w'_n  \rangle
		\NewLine
		\where
		w' = \Lambda i \in [1,\ldots,n] \If i == k \Then w_{i,|w|} w_{i,1}\ldots w_{i,|w|-1} \Else w_i 
	}
	\\
	\Theorem{RotationPreservesRealization}
	{
		\forall \langle X | w_1,\ldots,w_n \rangle : \TYPE{PolygonalPresentation} \. \NewLine \. 
		\forall i \in [1,\ldots,n] \.
		\mathrm{real}\;\FUNC{rotation}\Big( \langle X | w_1,\ldots, w_n \rangle, i\Big) \cong_\TOP
		\mathrm{real} \;  \langle X | w_1, \ldots,w_n \rangle
	}
	\NoProof
	\\
	\DeclareFunc{cutting}{
		\prod \langle X | w_1,\ldots,w_n \rangle : \TYPE{PolygonalPresentation} \. \NewLine \. 
		\prod^n_{k=1}
		[1,\ldots,|w_k|-1] \to X^\c \to 
		[1,\ldots,n] \to \TYPE{PolygonalPresentation}
	}
	\DefineFunc{cutting}{ j,z }
	{
		\langle X | w'_1, \ldots, w'_{n+1}  \rangle
		\NewLine
		\where \quad
		w' = \Lambda i \in [1,\ldots,n+1] \If i < k \Then w_i 
		\Else \If \NewLine \Else \If i == k \Then  w_{i,1} \ldots w_{i,j} z
		\Else \If i == k + 1 \Then  z w_{i,j+1}\ldots w_{i,|w_i|}
		\Else w_{i+1}
	}
	\\
	\Theorem{CuttingPreservesRealization}
	{
		\forall \langle X | w_1,\ldots,w_n \rangle : \TYPE{PolygonalPresentation} \. \NewLine \. 
		\forall i \in [1,\ldots,n] \. \forall j \in \Big[1,\ldots,|w_i|\Big] \. \forall z \in X^\c \.
		\mathrm{real}\;\FUNC{cutting}\Big( \langle X | w_1,\ldots, w_n \rangle, i,j,z\Big) \cong_\TOP
		\mathrm{real} \;  \langle X | w_1, \ldots,w_n \rangle
	}
	\NoProof
}
\Page{
	\DeclareType{PastableIndex}
	{
		\prod \langle X | w_1, \ldots, w_n \rangle : \TYPE{PolygonalPresentation} \.
		?[1,\ldots,n-1]
	}
	\DefineType{i}{PastableIndex}
	{
		\exists z \in X :  
		w_{i,-1,1} = z = w_{i+1,1,1} \And \NewLine \And
		\Bigg|\bigg\{  (i,j)  \bigg| i \in [1,\ldots,n], j \in \Big[1,\ldots,|w_i|\Big], w_{i,j,1} = z  \bigg\} \Bigg| = 2
	}
	\\
	\DeclareFunc{pasting}{
		\prod \langle X | w_1,\ldots,w_n \rangle : \TYPE{PolygonalPresentation} \. \NewLine \. 
		\TYPE{PastableIndex} \to \TYPE{PolygonalPresentation}
	}
	\DefineFunc{pasting}{ i }
	{
		\Big\langle X \setminus \{z\} \Big| w'_1, \ldots, w'_{n-1}  \Big\rangle
		\NewLine
		\where \quad
		w' = \Lambda i \in [1,\ldots,n+1] \If i < k \Then w_i 
		\Else \If \NewLine \Else \If i == k \Then  \widehat{w_i w_{i+1}}_{\{|w_i|,|w_i|+1\}}
		\Else w_{i-1}
		\NewLine
		\where
		\quad
		z = \bd \TYPE{PastableIndex}\Big( \langle X | w_1,\ldots, w_n \rangle, i \Big)
	}
	\\
	\Theorem{PastingPreservesRealization}
	{
		\forall \langle X | w_1,\ldots,w_n \rangle : \TYPE{PolygonalPresentatio} \. \NewLine \. 
		\forall i : \TYPE{PastableIndex}\langle X |w_1,\ldots,w_m \.
		\mathrm{real}\;\FUNC{pasting}\Big( \langle X | w_1,\ldots, w_n \rangle, i,j,z\Big) \cong_\TOP
		\mathrm{real} \;  \langle X | w_1, \ldots,w_n \rangle
	}
	\NoProof
	\\
	\DeclareType{FoldableIndex}
	{
		\prod \langle X | w_1, \ldots, w_n \rangle : \TYPE{PolygonalPresentation} \.
		?[1,\ldots,n-1]
	}
	\DefineType{i}{FoldableIndex}
	{
		\exists z \in X \sqcup X^{-1}  :
		\exists u : \TYPE{PolygonalWord}(X \sqcup X^{-1}) 
		w_{i} = uzz^{-1} \And \NewLine \And
		\Bigg|\bigg\{  (i,j)  \bigg| i \in [1,\ldots,n], j \in \Big[1,\ldots,|w_i|\Big], w_{i,j,1} = z  \bigg\} \Bigg| = 2
	}
	\\
	\DeclareFunc{folding}{
		\prod \langle X | w_1,\ldots,w_n \rangle : \TYPE{PolygonalPresentation} \. \NewLine \. 
		\TYPE{FoldableIndex} \to \TYPE{PolygonalPresentation}
	}
	\DefineFunc{folding}{ i }
	{
		\Big\langle X \setminus \{z\} \Big| w'_1, \ldots, w'_{n-1}  \Big\rangle
		\NewLine
		\where \quad
		w' = \Lambda i \in [1,\ldots,n+1] \If i < k \Then w_i 
		\Else \If \NewLine \Else \If i == k \Then  u
		\Else w_{i}
		\NewLine
		\where
		\quad
		(z,u) = \bd \TYPE{FoldableIndex}\Big( \langle X | w_1,\ldots, w_n \rangle, i \Big)
	}
	\\
	\Theorem{FoldingPreservesRealization}
	{
		\forall \langle X | w_1,\ldots,w_n \rangle : \TYPE{PolygonalPresentatio} \. \NewLine \. 
		\forall i : \TYPE{PastableIndex}\langle X |w_1,\ldots,w_m \.
		\mathrm{real}\;\FUNC{folding}\Big( \langle X | w_1,\ldots, w_n \rangle, i\Big) \cong_\TOP
		\mathrm{real} \;  \langle X | w_1, \ldots,w_n \rangle
	}
	\NoProof
}\Page{
	\DeclareFunc{unfolding}{
		\prod \langle X | w_1,\ldots,w_n \rangle : \TYPE{PolygonalPresentation} \. \NewLine \. 
		[1,\ldots,n] \to X^\c \to \TYPE{PolygonalPresentation}
	}
	\DefineFunc{unfolding}{ z,i }
	{
		\Big\langle X \setminus \{z\} \Big| w'_1, \ldots, w'_{n-1}  \Big\rangle
		\NewLine
		\where \quad
		w' = \Lambda i \in [1,\ldots,n+1] \If i < k \Then w_i 
		\Else \If \NewLine \Else \If i == k \Then  w_i z z^{-1}
		\Else w_{i}
	}
	\\
	\Theorem{UnfoldingPreservesRealization}
	{
		\forall \langle X | w_1,\ldots,w_n \rangle : \TYPE{PolygonalPresentatio} \. \NewLine \. 
		\forall i : [1,\ldots,n] \. \forall z \in X^\c \.
		\mathrm{real}\;\FUNC{unfolding}\Big( \langle X | w_1,\ldots, w_n \rangle,i, z\Big) \cong_\TOP
		\mathrm{real} \;  \langle X | w_1, \ldots,w_n \rangle
	}
	\NoProof
	\\
	\Theorem{ConnectedSumRealization}
	{
		\forall \langle X | w_1 \rangle, \langle Y | v_1 \rangle 
			:\TYPE{SurfacePresentation} \.
		\NewLine \.
		\mathrm{real} \langle X \sqcup Y | w_1 v_1 \rangle =
		\mathrm{real} \langle X | w_1 \rangle \# 
		\mathrm{real} \langle Y | w_2 \rangle 
	}
	\NoProof
	\\
	\Theorem{SpherePresentation}
	{
		\Sphere^2 \cong \mathrm{real}\langle a,b | abb^{-1}a^{-1}  \rangle
	}
	\NoProof
	\\
	\Theorem{TorusPresentation}
	{
		\Sphere^1 \times \Sphere^1 \cong \langle a,b | ab^{-1}ba^{-1}  \rangle
	}
	\NoProof
	\\
	\Theorem{TorusPresentation}
	{
		\Sphere^1 \times \Sphere^1 \cong_\TOP  \mathrm{real}\langle a,b | ab^{-1}a^{-1}b  \rangle
	}
	\NoProof
	\\
	\Theorem{ProjectiveSpacePresentation}
	{
		\mathbb{RP}^2 \cong_\TOP \mathrm{real}\langle a,b | abab  \rangle
	}
	\NoProof
	\\
	\Theorem{KleinPresentation}
	{
		\mathbf{K} \cong_\TOP \mathrm{real}\langle a,b | ab^{-1}ab  \rangle
	}
	\NoProof
}
\newpage
\subsection{Classification Theorem}
\Page{
	\Theorem{EveryCompactSurfaceAdmitsPresentation}
	{
		\forall M : \CS \. \NewLine \. 
		\exists P : \TYPE{SurfacePresentation} :
		M \cong \mathrm{real} \; P
	}
	\NoProof
	\\
	\DeclareFunc{torus}{\CS}
	\DefineNamedFunc{torus}{}{\mathbb{R}}{\mathbb{S}\times\mathbb{S}}
	\\
	\Theorem{ClassificationOfCompactSurfafacesI}
	{
		\forall M : \CS \. \NewLine \. 
		M \cong \Sphere^2 
		\Big| 
			\exists n \in \Nat :
			M \cong \bigsum^n_{i=1} \mathbb{T} |
			M \cong \bigsum^n_{i=1} \mathbb{RP}^2
	}
	\\
	\Theorem{KleinBottelAsSum}{\mathbf{k} =\mathbb{RP}^2 \# \mathbb{RP}^2}
	\NoProof
	\\
	\Theorem{SumProjectivization}{\mathbb{T} \# \mathbb{RP}^2 =\mathbb{RP}^2 \# \mathbb{RP}^2 \# \mathbb{RP}^2}
	\NoProof	
}
\newpage
\subsection{Euler Characteristic}
\Page{
	\DeclareFunc{characteristicOfEuler}
	{
		\TYPE{FiniteCWComplex} \to \Int
	}
	\DefineNamedFunc{characteristicOfEuler}{(X,\E,\varphi)}{\chi(X,\E,\varphi)}
	{
		\sum^{\infty}_{n=0} (-1)^n|\E_i|
	}
	\\
	\DeclareFunc{presentationEulerCharacteristic}
	{
		\TYPE{PolygonalPresentation} \to \Int
	}
	\DefineNamedFunc{characteristicOfEuler}{ P} {\chi(P)}
	{
		\chi(C) \NewLine 
		\where  \quad	
		C = \THM{SimplecticComplexIsCW} \; \THM{PolygonalComplexHasSimplecticStructure}(\mathrm{real}\;P)
	}
	\\
	\DeclareFunc{compactSurfacesEulerCharacteristic}
	{
		\CS \to \Int
	}
	\DefineNamedFunc{characteristicOfEuler}{ M} {\chi(M)}
	{
		\chi(P) \NewLine 
		\where  \quad	
		P = \THM{EveryCompactSurfaceAdmitsPresentation}(M)
	}
	\\
	\Theorem{SpheresEullerCharacteristic}{\chi\Big( \Sphere^2 \Big) = 2}
	\NoProof
	\\
	\Theorem{ConectedSumOfToriEullerCharacteristic}
	{
		\forall n \in \Nat \.
		\chi\Act{\bigsum^n_{i=1} \mathbb{T}} = 2 - 2n  
	}
	\NoProof
	\\
	\Theorem{ConectedSumOfToriEullerCharacteristic}
	{
		\forall n \in \Nat \.
		\chi\Act{\bigsum^n_{i=1} \Reals \P^2} = 2 - n  
	}
	\NoProof
	\\
	\Theorem{EulerCharacteristicIsPreservedByElementaryTransformation}{\ldots}
	\NoProof
}
\newpage
\subsection{Orientability}
\Page{
	\DeclareFunc{bandOfMobius}{\boundary \TOPM(2)}
	\DefineNamedFunc{bandOfMobius}{}{\mathbf{MB}}
	{ \langle a,b,c |  abcb \rangle  }
	\\
	\DeclareType{Oriented}{? \TYPE{PolygonalPresentation}}
	\DefineType{\langle X | w_1,\ldots,w_n \rangle}{Oriented}
	{
		\NewLine
		\iff
		\bigg\{
			\Big( (i,j),(k,l)\Big) 
			\bigg|	
			i,k \in [1,\ldots,n] ;
			j \in \Big[1, \ldots, |w_i| \Big] ,
			l \in \Big[1, \ldots, |w_l| \Big] ,
			(i,j) \neq (k,l), w_{i,j} = w_{k,l}
		\bigg\} = \emptyset
	}
	\\
	\DeclareType{Orientable}{?\CS}
	\DefineType{M}{Orientable}
	{
		\exists  P : \TYPE{Oriented} \. 
		M \cong_\TOP \mathrm{real}(P)
	}
	\\
	\Theorem{OrientableSurfacesClassification}
	{
		\forall M : \TYPE{Orientable} \.
		M \cong_{\TOPM} \Sphere^2 
		\Big|
		\exists n \in \Nat : 
		M \cong_{\TOPM} \bigsum^n_{i=1} \mathbb{T}
	}
	\NoProof
}
\newpage
\section{Basic Homotopy}
\subsection{Homotopy of Maps}
\Page{
	\DeclareType{Homotopy}
	{
		\prod X,Y \in \TOP \.
		(X \Arrow{\TOP} Y)^2 \to
		?\Big((I \times X) \Arrow{\TOP} Y \Big)
	}
	\DefineType{H}{Homotopy}
	{
		\Lambda f,g : X \Arrow{\TOP} Y 
		\.
		H(0, \bullet ) = f \And H(1,\bullet) = g
	}
	\\
	\DeclareType{Homotopic}
	{
		\prod X,Y \in \TOP \.
		?(X \Arrow{\TOP} Y)^2
	}
	\DefineNamedType{(f,g)}{Homotopic}{f \sim g}
	{
		\exists \TYPE{Homotopy}(X,Y,f,g)
	}	
	\\
	\DeclareType{NullHomotopic}
	{
		\prod X,Y \in \TOP \.
		?(X \Arrow{\TOP} Y)
	}
	\DefineType{f}{NullHomotopic}
	{
		\exists y \in Y \. \exists \TYPE{Homotopy}(X,Y,f,y)
	}	
	\\
	\Theorem{HomotopicIsEquivallence}
	{
		\forall X,Y \in \TOP \. 
		\TYPE{Equivalence}\Big( C(X,Y),\TYPE{Homotopic}(X,Y) \Big)
	}
	\AssumeIn{f}{C(X,Y)}
	\Say{H}{\Lambda t \in I \. \Lambda x \in X \. f(x) }{(I \times X) \Arrow{\TOP} Y}
	\Say{[1]}{\bd^{-1} \TYPE{Homotopy} \ByConstr H}{\TYPE{Homotopy}(X,Y,f,f,H)}
	\Conclude{[*]}{\bd^{-1}\TYPE{Homotopic}[1]}{f \sim f}
	\Derive{[1]}{\Intro(\forall)}{\forall f \in C(X,Y) \. f \sim f}
	\AssumeIn{f,g}{C(X,Y)}
	\Assume{[2]}{f \sim g}
	\Say{ H }{\bd \TYPE{Homotopic}[2]}{\TYPE{Homotopy}(X,Y,f,g)}
	\Say{ H' }{\Lambda t \in [0,1] \. H(1-t)}{\TYPE{Homotopy}(X,Y,g,f)}
	\Conclude{[*]}{\bd^{-1}\TYPE{Homotopic}(H')}{g \sim f}	
	\Derive{[2]}{\Intro(\forall)\Intro(\Imply)}{\forall f,g \in C(X,Y) \. f \sim g \Imply g \sim f}
	\AssumeIn{f,g,h}{C(X,Y)}
	\Assume{[3]}{f \sim g}
	\Assume{[4]}{g \sim h}
	\Say{ H }{\bd \TYPE{Homotopic}[3]}{\TYPE{Homotopy}(X,Y,f,g)}
	\Say{ H'}{\bd \TYPE{Homotopic}[4]}{\TYPE{Homotopy}(X,Y,g,h)}
	\Say{H''}{ \Lambda t \in I \. \If y \le \frac{1}{2} \Then H(2t) \Else H'(2t-1) }
	{ \TYPE{Homotopy}(X,Y,f,h)  }
	\Conclude{[*]}{\bd^{-1}\TYPE{Homotopic}(H'')}{f \sim h}	
	\Derive{[3]}{\Intro(\forall)\Intro(\Imply)}{\forall f,g,h \in C(X,Y) \. f \sim g \forall g \sim h \Imply f \sim h}
	\Conclude{[*]}{\bd^{-1} \TYPE{Equivalence}[1,2,3]}{\TYPE{Equivalence}\Big( C(X,Y),\TYPE{Homotopic}(X,Y) \Big)}
	\EndProof
}
\Page{
	\Theorem{HomotopicComposition}
	{
		\forall X,Y,Z \in \TOP \. 
		\forall f,g  :  X \Arrow{\TOP} Y \. 
		\forall f',g' : X \Arrow{\TOP} Y \.
		f \sim g \And f' \sim g' \Imply \NewLine \Imply
		f' \circ f \sim g' \circ g          
	}
	\NoProof
	\\
	\Theorem{LineHomotopy}
	{
		\forall X \in \TOP \.
		\forall C : \Convex \.
		\forall f,g : X \Arrow{\TOP} C \.
		f \sim g
	}
	\Say{H}{\Lambda t \in [0,1] \. t f + (1-t)g}{\TYPE{Homotopy}(X,C)}
	\Say{[1]}{\ByConstr H\Big( H(0)\Big)}{H(0) = f}
	\Say{[2]}{\ByConstr H\Big( H(1)\Big)}{H(1) = g}
	\Say{[3]}{\bd^{-1} \TYPE{Homotopy}}{\TYPE{Homotopy}(X,C,f,g,H)}
	\Conclude{[*]}{\bd^{-1} \TYPE{Homotopic}(f,g)}{f \sim g}
	\EndProof
}
\newpage
\subsection{Fundamental Group}
\Page{		
	\DeclareType{Stationary}{ 
		\prod X,Y \in \TOP \.
		\prod f,g : ?(X \Arrow{\TOP} Y)^2 \.
		?X \to ?\TYPE{Homotopy}(X,Y,f,g) 
	}
	\DefineType{H}{Stationary}{\prod A \subset X \. \forall t \in [0,1] \. \forall a \in A \. H(t,a) = f(a) }
	\\
	\DeclareType{RelativelyHomotopic}
	{
		\prod X,Y \in \TOP \.
		?X \to 
		?(X \Arrow{\TOP} Y)^2
	}
	\DefineNamedType{(f,g)}{Homotopic}{\Lambda A \subset X \. f \sim_A g}
	{
		\Lambda A \subset X \. \exists \TYPE{Stationary}(X,Y,f,g,A)
	}	
	\\
	\DeclareType{FreelyHomotopic}
	{
		\prod X,Y \in \TOP \.
		?\TYPE{Homotopic}(X,Y)
	}
	\DefineNamedType{(f,g)}{FreelyHomotopic}{f \sim_! g}
	{
		\forall A \subset X \.  (f,g) \IsNot \TYPE{RelativelyHomotopic}(X,Y,A)
	}
	\\
	\DeclareType{PathHomotopic}
	{
		\prod X \in \TOP \.
		?\TYPE{Homotopic}(I,X)
	}
	\DefineNamedType{(\alpha,\beta)}{PathHomotopic}{f \approx g}
	{
		\RH\Big(I,X,\alpha,\beta,\{0,1\}\Big)
	}
	\\
	\Theorem{HomotopicIsEquivallence}
	{
		\forall X \in \TOP \. 
		\forall x,y \in X \.
		\TYPE{Equivalence}\Big( \Omega(x,y),\TYPE{PathHomotopic}(X) \cap \Omega^2(x,y) \Big)
	}
	\AssumeIn{\gamma}{\Omega(x,y)}
	\Say{H}{\Lambda t \in I \. \Lambda x \in X \. \gamma(x) }{I^2 \Arrow{\TOP} X}
	\Say{[1]}{\bd^{-1} \TYPE{Homotopy} \ByConstr H}{\TYPE{PathHomotopy}(X, I,\gamma,\gamma,H)}
	\Conclude{[*]}{\bd^{-1}\TYPE{PathHomotopic}[1]}{\gamma \approx \gamma}
	\Derive{[1]}{\Intro(\forall)}{\forall \gamma \in \Omega(x,y) \. \gamma \approx \gamma}
	\AssumeIn{f,g}{\Omega(x,y)}
	\Assume{[2]}{f \sim g}
	\Say{ H }{\bd \TYPE{PathHomotopic}[2]}{\TYPE{Homotopy}(X,Y,\alpha,\beta)}
	\Say{ H' }{\Lambda t \in [0,1] \. H(1-t)}{\TYPE{Homotopy}(X,I,\alpha,\beta)}
	\Conclude{[*]}{\bd^{-1}\TYPE{PathHomotopic}(H')}{\alpha \approx \beta}	
	\Derive{[2]}{\Intro(\forall)\Intro(\Imply)}
	{
		\forall \alpha,\beta \in \Omega(x,y) \. 
		\alpha \approx \beta \Imply \beta \approx \alpha
	}
	\AssumeIn{\alpha,\beta,\gamma}{\Omega(x,y)}
	\Assume{[3]}{\alpha \approx \beta}
	\Assume{[4]}{\beta \approx \gamma}
	\Say{ H }{\bd \TYPE{PathHomotopic}[3]}{\TYPE{Homotopy}(X,I,\alpha,\beta)}
	\Say{ H'}{\bd \TYPE{PathHomotopic}[4]}{\TYPE{Homotopy}(X,I,\beta,\gamma)}
	\Say{H''}{ \Lambda t \in I \. \If y \le \frac{1}{2} \Then H(2t) \Else H'(2t-1) }
	{ \TYPE{Homotopy}(X,I,\alpha,\gamma)  }
	\Conclude{[*]}{\bd^{-1}\TYPE{PathHomotopic}(H'')}{\alpha \approx \gamma}	
	\Derive{[3]}{\Intro(\forall)\Intro(\Imply)}
	{
		\forall \alpha,\beta,\gamma \in C(X,Y) \. 
		\alpha \approx \gamma 
		\And
		\beta \approx \gamma  
		\Imply \alpha \approx \gamma
	}
	\Conclude{[*]}{\bd^{-1} \TYPE{Equivalence}[1,2,3]}{\TYPE{Equivalence}\Big( \Omega(x,y),\TYPE{PathHomotopic}(X) \Big)}
	\EndProof
}
\Page{
	\DeclareType{NullHomotopicPath}{\prod X \in \TOP \. \prod x \in X \. ?\Omega(x,x) }
	\DefineType{\gamma}{NullHomotopticPath}{\gamma \approx x}
	\\
	\DeclareType{Reparametrization}{\prod X \in \TOP \. (I \Arrow{\TOP} X) \to  ?(I \Arrow{\TOP} X) }
	\DefineType{\omega}{Reparametrization}
	{
		\Lambda \gamma : I \Arrow{\TOP} X \. 
		\exists \varphi : I \Arrow{\TOP} I : 
		\varphi(0) = 0 \And \varphi(1) = 1 \And \omega = \gamma \circ \varphi
	}
	\\
	\Theorem{PathHomotopicReparametrization}
	{
		\forall X \in \TOP \.
		\forall \alpha : I \Arrow{\TOP} X \.
		\forall \beta : \TYPE{Reparametrization}(X,\alpha) \.
		\alpha \approx \beta
	}
	\Say{\Big( \varphi,[1],[2],[3]\Big)}
	{
		\bd \TYPE{Reparametrization}(X,\alpha,\beta)
	}
	{
		\sum \varphi : (I \Arrow{\TOP} I) \.
		\beta = \alpha \circ \varphi \And \varphi(0) = 0 \And \varphi(1) = 1
	}
	\Say{[4]}{\THM{LineHomotopy}(I,I,\varphi,\id)}{\varphi \sim \id}
	\Say{[5]}{\THM{HomotopicComposition}(I,I,X,\id,\varphi,\alpha,\alpha)[1,4]}{\alpha \sim \beta}
	\Conclude{[*]}{\bd^{-1} \TYPE{PathHomotopic}[1,2,3,5]}{\alpha \approx \beta}
	\EndProof
	\\
	\DeclareFunc{basedFundamentalGroup}{\prod X \in \TOP \. X \to \SET}
	\DefineNamedFunc{basedFundamentalGroup}{x}{\pi(x)}{\frac{\Omega(x)}{\TYPE{PathHomotopic}}}
	\\
	\DeclareFunc{joinPaths}{
		\prod X \in \TOP \. 
		\prod x,y,z \in X \. 
		\Omega(x,y) \times \Omega(y,z) \to \Omega(x,z)
	}
	\DefineNamedFunc{joinPaths}{\alpha,\beta}{\alpha \circ \beta}
	{
		\Lambda t \in [0,1] \.
		\If t \le \frac{1}{2} \Then  \alpha(2t) \Else \beta(2t-1)
	}
	\\
	\Theorem{HomotopicLoopJoinIsWellDefine}
	{
		\forall X \in \TOP \. 
		\forall x \in \TOP \.
		\forall \alpha,\beta \in \pi(x) \.
		\forall a,a' \in \alpha \. 
		\forall b,b' \in \beta \. 
		[a \circ a'] = [b \circ b']
	}
	\Say{\Big(H,[1]\Big)}{\bd \pi(x)(\alpha)(a,a')}
	{
		\sum H : \TYPE{Homotopy}(I,X,a,a') \. 
		\forall t \in I \. H(t,0) = H(t,1) = x
	}
	\Say{\Big(H',[2]\Big)}{\bd \pi(x)(\alpha)(b,b')}
	{
		\sum H' : \TYPE{Homotopy}(I,X,b,b') \. 
		\forall t \in I \. H'(t,0) = H'(t,1) = x
	}
	\Say{H''}{\Lambda t \in I \. H(t) \circ H'(t)}{\TYPE{Homotopy}(I,X,ab,a'b')}
	\Conclude{[*]}{\bd \pi(x)(H'')}{[ab] = [a'b']}
	\EndProof
	\\
	\DeclareFunc{fundamentalGroupOperataion}{
		\prod X \in \TOP \. 
		\prod x \in X \. 
		\Omega(x) \times \Omega(y,z) \to \Omega(x,z)
	}
	\DefineNamedFunc{fundamentalGroupOperation}{[a],[b]}{[a][b]}
	{
		[ab]
	}
}\Page{
	\Theorem{FundamentalGroupIsAGroup}
	{
		\forall X \in \TOP \.
		\forall x \in X \.
		\Big(\pi(x), (\cdot) \Big) \in \GRP
	}
	\Say{[1]}{\THM{PathHomotopocReparametrizaton}\bd \Big(\pi(x),(\cdot)\Big)}
	{
		\forall \alpha, \beta, \gamma \in \pi(x) \. 
		(\alpha \beta) \gamma = \alpha (\beta \gamma)
	}
	\Say{[2]}{\bd \Big( \pi(x),(\cdot) \Big)}
	{
		\forall \alpha \in \pi(x) \. \alpha [x] = [x] \alpha = \alpha
	}
	\AssumeIn{[a]}{\pi(x)}
	\SayIn{b}{\Lambda s \in I \. a(1-s)}{\Omega(x)}
	\Say{H}{
		\Lambda t \in I \. 
		\Lambda s \in I \. 
		\If s  < \frac{1-t}{2} 
		\Then a( 2s )
		\Else \If  s \le  \frac{1 + t}{2}  
		\Then  a\left( \frac{1-t}{2} \right)
		\Else b\left(  2t - 1   \right)  
	}
	{
		\NewLine : \TYPE{Homotopy}(I,X,\alpha\beta,x)
	}
	\Say{[3.*]}{\bd \pi(x)(H)}{[ab] = [x]}
	\Say{H'}{
		\Lambda t \in I \. 
		\Lambda s \in I \. 
		\If s  < \frac{1-t}{2} 
		\Then b( 2s )
		\Else \If  s \le  \frac{1 + t}{2}  
		\Then  b\left( \frac{1-t}{2} \right)
		\Else a\left(  2t - 1   \right)  
	}
	{
		\NewLine : \TYPE{Homotopy}(I,X,\alpha\beta,x)
	}
	\Conclude{[4.*]}{\bd \pi(x)(H)}{[ba] = [x]}
	\Derive{[3]}{\Intro(\forall)\Intro(\exists)}
	{\forall \alpha \in \pi(x) \. \exists \beta \in \pi(x) : \alpha \beta = \beta \alpha = [x] }
	\Conclude{[*]}{\bd^{-1} \GRP}{\Big( \pi(x),(\cdot)\Big) \in \GRP}
	\EndProof
	\\
	\Theorem{ChangeOfBasePoint}{ 
		\forall X \in \TOP \.
		\forall x,y \in X \. 
		\forall [a] \in \pi(x) \. 
		\forall \gamma \in \Omega(y,x) \.
		\Big[\gamma a \gamma^{-1}\Big]  \in \pi(y)
	}
	\NoProof
	\\
	\Theorem{IsomorphicFundamentalGroup}{ 
		\forall X \in \TOP \.
		\forall C \in \mathrm{PCC}(X) \.
		\forall x,y \in C \. 
		\pi(x) \cong_{\GRP} \pi(y)
	}
	\NoProof
	\\
	\Theorem{FundamentalGroupsOfConnected}
	{
		\forall X : \TYPE{PathConnected} \.
		\forall x,y \in X \.
		\pi(x) \cong_{\GRP} \pi(y)
	}
	\\
	\DeclareFunc{generalFundamentalGroup}
	{
		\TYPE{PathConnected} \And \TYPE{NonEmpty} \to \GRP
	}
	\DefineNamedFunc{generalFundamentalGroup}{X}{\pi(X)}{\pi(x) \quad \where \quad x = \bd \TYPE{NonEmpty}(X)}
	\\
	\DeclareType{SimplyConnected}{?(\TYPE{PathConnected} \And \TYPE{NonEmpty})}
	\DefineType{X}{SimplyConnected}{\Big|\pi(X)\Big|=1}
	\\
	\Theorem{ConvexIsSimplyConnected}
	{
		\forall C : \TYPE{Convex} \.
		\TYPE{SimplyConnected}(C)
	}
	\NoProof
}\Page{
	\Theorem{RealVectorSpaceIsSimplyConnected}
	{
		\forall V : \TOPVS{\Reals} \.
		\TYPE{SimplyConnected}(V)
	}
	\NoProof
	\\
	\DeclareFunc{circleRepresentative}
	{
		\prod X \in \TOP \.
		\prod x \in X \. 
		\Omega(x) \to (\Sphere^1 \Arrow{\TOP} X)
	}
	\DefineNamedFunc{circleRepresentative}{\gamma}{\tilde \gamma}
	{
		\frac{\gamma}{\{0,1\}}
	}
	\\
	\Theorem{CircleRepresentativeOfNullHomotopic}
	{
		\forall X \in \TOP \.
		\forall x \in X \. 
		\forall \gamma : \TYPE{NullHomotopic}(x,X) \.
		\tilde \gamma \sim x
	}
	\Say{\Big(H,[1]\Big)}
	{
		\bd \TYPE{NullHomotopic}(X,x,\omega)
	}
	{
		\sum H : \TYPE{Homotopy}(I,X,\gamma,x) \. 
		\forall t \in I \.  H(t,1) = H(t,0) = x
	}
	\Say{H'}{\Lambda t \in I \. \widetilde{H(t)}}
	{
		\TYPE{Homotopy}(\Sphere^1,X,\tilde \gamma, x)
	}
	\Conclude{[*]}{\bd^{-1}\TYPE{Homotopic}}{\gamma \sim x}
	\EndProof
	\\
	\Theorem{NullHomotopicExtension}
	{
		\forall X \in \TOP \.
		\forall x \in X \. 
		\forall \gamma : \Sphere^1 \Arrow{\TOP} X  \.
		\gamma \sim x 
		\Imply \NewLine \Imply
		\exists \Gamma : \mathbb{D}^2 \Arrow{\TOP} X \.
		\Gamma_{|\Sphere^1} = \gamma
	}
	\Say{H}{\bd \TYPE{Homotopic}(x,\gamma)}
	{
		\TYPE{Homotopy}(\Sphere^1,X,x,\gamma)
	}
	\Say{\Gamma}
	{
		\Lambda v \in \Cell^2 \.
		\If v == 0 \Then x 
		\Else   H\left(\|v\|, \frac{v}{\|v\|} \right)
	}
	{
		\mathbb{D}^2 \Arrow{\TOP} X
	}
	\Conclude{[*]}{\bd \TYPE{Homotopy}(H)\ByConstr \Gamma}
	{
		\Gamma_{|\Sphere^1} = \gamma
	}
	\EndProof
	\\
	\Theorem{ExtensionImplyNullHomotopic}
	{
		\forall X \in \TOP \.
		\forall x \in X \. 
		\forall \gamma : \Omega(x)  \.
		\forall \Gamma :  \mathbb{D}^2 \Arrow{\TOP} X \.
		\Gamma_{|\Sphere^1} = \tilde \gamma
		\Imply \NewLine \Imply
		\TYPE{NullHomotopix}(\gamma)
	}
	\NoProof
	\\
	\Theorem{SquareLemma}
	{
		\forall X \in \TOP \.
		\forall F : I^2 \Arrow{\TOP} X \.
		fg = hk 
		\NewLine
		\where \NewLine
		f = \Lambda t \in [0,1] \. F(t,0) \NewLine
		g = \Lambda t \in [0,1] \. F(1,t) \NewLine
		h = \Lambda t \in [0,1] \. F(0,t) \NewLine
		k = \Lambda t \in [0,1] \. F(t,1) 
	}
	\NoProof
}
\Page{
	\DeclareType{LebesgueNumber}
	{
		\prod X \in \MS \. 
		\TYPE{OpenCover}(X) \to
		?\Reals_{++}
	}
	\DefineType{\lambda}{LebesgueNumber}
	{
		\Lambda \mathcal{O} : \TYPE{OpenCover}(X) \.
		\forall U \subset X \. 
		\diam U  < \lambda \Imply
		\exists O \in \mathcal{O} : 
		U \subset O
	}
	\\
	\Theorem{LebesgueNumberLemma}
	{
		\forall X \in \MS \And \TYPE{Compact} \.
		\forall \mathcal{O} : \TYPE{OpenCover}(X) \.
		\exists \TYPE{LebesgueNumber}(\mathcal{O})
	}
	\NoProof
	\\
	\Theorem{LoopTacklingTHM}
	{
		\forall M \in \TOPM \. 
		\forall [0] : \dim M \ge 2 \.
		\forall p, p' \in X \.
		\forall \gamma \in \Omega(p,p') \.
		\forall q \in X \setminus \{p,p'\} \. \NewLine \. 
		\exists \gamma' \in \Omega(p,p') :
		\gamma' \sim \gamma 
		\And
		q \not \in \im \gamma'
	}
	\SayIn{U}{\bd \TYPE{NonEmpty}\;\C\C(q)}{\C\C(q)}
	\SayIn{V}{M \setminus \{q\}}{\T(X)}
	\Say{\mathcal{O}}{\Big\{ \gamma^{-1}(U), \gamma^{-1}(V)  \Big\}}
	{
		\TYPE{OpenCover}(I)
	}
	\Say{\lambda}{\THM{LebesgueNumberLemma}(I,\mathcal{O})}{\TYPE{LebesgueNumber}(I,\mathcal{O})}
	\Say{\Big(m,[1]\Big)}{\THM{ReductioInfima}(\Reals,\lambda)}
	{
		\sum_{m=1}^\infty \frac{1}{m} < \lambda
	}
	\AssumeIn{k}{[1,\ldots,m-1]}
	\Assume{[2]}{\gamma\left( \frac{k}{m} \right)=q}
	\Conclude{[*]}{\bd \TYPE{LebesgueNumber}[1,2]}
	{
		\gamma\left[\frac{(k-1)}{m},\frac{k}{m}\right],
		\gamma\left[\frac{k}{m}, \frac{k+1}{m}\right] \subset U
	}
	\Derive{[2]}{\Intro(\forall)\Intro(\Imply)}
	{
		\forall k \in [1,\ldots,m-1] \.
		\gamma\left( \frac{k}{m} \right)=q
		\Imply
		\gamma\left[\frac{(k-1)}{m},\frac{k}{m}\right],
		\gamma\left[\frac{k}{m}, \frac{k+1}{m}\right] \subset U
	}
	\Say{A}{\left\{ k \in [0,\ldots,n] :  \gamma\left(\frac{k}{m}\right) \neq q  \right\}}
	{
		?[0,\ldots,n]
	}
	\SayIn{l}{|A|}{\Nat}
	\Say{a}{\frac{\FUNC{sort}(A)}{m}}{ \TYPE{increasing}\Big( [1,\ldots,l], I\Big)}
	\Say{[3]}{\bd \gamma \ByConstr a}{a_1 = 0 \And a_l = 1}
	\Say{[4]}{\THM{CellIsPathConnected}(\dim M)}{\TYPE{PathConnected}\Big(U \setminus \{q\}\Big)}
	\Say{[*]}{[2][4]}{\exists \gamma' \in \Omega(p,p') \. \gamma'\sim \gamma \And q \not \in \im \gamma}
	\EndProof
	\\
	\Theorem{HigherSphereIsSimplyConnected}
	{
		\forall n \in \Nat \.
		n \ge 2 
		\Imply
		\TYPE{SimplyConnected}\Big(\Sphere^n\Big)
	}
	\NoProof
}
\Page{
	\Theorem{FundamentalGroupIsCountable}
	{
		\forall M \in \TOPM \.
		\forall p \in M \.
		\Big|\pi(p)\Big| \le \aleph_0
	}
	\Say{\Big(\O,[1]\Big)}{\THM{ManifoldHasCoverByCoordinateCharts}(M)}
	{
		\sum \O : \TYPE{OpenCover}(M) \. \NewLine \.
		\Big|\O\Big| < \aleph_0 
		\And
		\forall O \in \O \.
		O \in \C\C(M)
	}
	\AssumeIn{O,O'}{\O}
	\Say{[2]}{
		\bd \TYPE{LocallyEuclidean}(M,O \And O')
		\bd \TYPE{SecondCountable}(M)
	}
	{
		\bigg|\mathrm{PCC}\Big( O \cap O' \Big)\bigg| \le \aleph_0
	}
	\Say{x}{\LOGIC{Choice}}{\prod_{C \in \mathrm{PCC}\Big( O \cap O' \Big)} C}
	\Conclude{X_{O,O'}}{\im x}{\TYPE{Countable}(M)}
	\Derive{\Big(X, [1]\Big)}{\Intro\Act{\prod}}
	{
		\prod_{O,O' \subset \O}  
		\sum X_{O,O'}  : \TYPE{Countable}(M) \.
		\forall C \in \mathrm{PCC}\Big( O \cap O' \Big) \.
		X_{O,O'} \cap C \neq \emptyset
	}
	\Conclude{\X}{\{p\} \cup \bigcup_{O,O' \subset \O} X_{O,O'}}{?M}
	\AssumeIn{O}{\O}
	\AssumeIn{x,y}{\O \cap \X}
	\Conclude{\gamma}{\bd \TYPE{LocallyEuCleadean}(O) \bd \TYPE{PathConnected}(O)}
	{
		\Omega_O(x,y)
	}
	\Derive{\gamma}{\Intro\Act{\prod}}{\prod_{O \in \O} \prod_{x,y \in O \cap \X} \Omega_O(x,y)}
	\Say{\Sigma}{\left\{ 
			\sigma \in \Omega(p) : 
				\exists n \in \Nat : 
				\exists O : n \to \O :
				\exists \prod^{n+1}_{i=1} \sum_{x_i \in \X}  x_i,x_{i+1} \in O_i :
				\sigma = \prod^n_{i=1} \gamma^{O_i}_{x_i,x_j} 
		\right\}
	}{?\Omega(x)}
	\Say{[2]}{\THM{CountableUnionOfFiniteProdCardBound}(X)\ByConstr \Sigma}
	{
		|\Sigma| \le \aleph_0
	}
	\AssumeIn{\alpha}{\Omega(p)}
	\Say{\Big(n,O,[3]\Big)}{\THM{LebesgueNumberLemma}(I,\alpha^{-1}\O)}
	{
		\sum_{n=1}^\infty \sum_{O : n \to \O} \forall k \in [1,\ldots n] \.
		\alpha\left[ \frac{k-1}{n}, \frac{k}{n} \right] \subset O_k
	}
	\Say{\beta}{\Lambda k \in [1,\ldots,n] \. \Lambda t \in [0,1] \alpha_{|[(k-1)/n,k/n]}(nt)}
	{
		[1,\ldots,n] \to I \Arrow{\TOP} M
	}
	\Say{[4]}{\ByConstr \beta}{\alpha = \prod^n_{k=1} \beta_k}
	\AssumeIn{k}{[1,\ldots,n-1]}
	\Say{[5]}{\THM{OpenHasNoBoundary}[3]}
	{
		\alpha\left(\frac{k}{n}\right) \in O_k \cap O_{k+1}
	}
	\Say{\Big(x, C, [6] \Big)}{\bd \X[5]}
	{ 
		\sum x \in \X \. \sum  C  \in \mathrm{PCC}( O_k \cap O_{k+1}) \. 
		x, \alpha\Act{\frac{k}{n}} \in C
	}
	\Conclude{\delta_k}{\bd \mathrm{PCC}(C)\bd \TYPE{PathConnected}(C)[6]}{\Omega_C\Act{x,\alpha\Act{\frac{k}{n}}}}
	\Derive{\Big(x,\delta\Big)}{\Intro\Act{\prod}}
	{
		\prod^{n-1}_{k=1} \sum_{x_k \in \X} \Omega_{O_k \cap O_{k+1}}\Act{x_k,\alpha\Act{\frac{k}{n}}}
	}
	\SayIn{\delta_0}{t \mapsto p}{\Omega(p)}
	\SayIn{\delta_n}{t \mapsto p}{\Omega(p)}
	\Say{\beta'}{\Lambda k \in [1,\ldots,n] \. \delta_{k-1}\beta_k \delta_k^{-1}}{[1,\ldots,n] \to I \Arrow{\TOP} M}
	\Say{[5]}{\ByConstr \beta' [4]}{\alpha \sim \prod^n_{k=1} \beta'_k}
}
\Page{
	\Conclude{[*]}{\ByConstr \Sigma [5] \bd \TYPE{SimplyConnected}(O)}
	{
		\exists \sigma \in \Sigma \. \sigma \approx \alpha
	}
	\Derive{[3]}{\Intro(\forall)}{\forall \alpha \in \Omega(p) \. \exists \sigma \in \Sigma : \alpha \approx \sigma}
	\Conclude{[*]}{\bd \pi(p)[2][3]}{\Big|\pi(p)\Big| \le \aleph_0}
	\EndProof
	\\
	\DeclareFunc{fundamentalGroupoid}
	{
		\TOP \to \mathsf{GROUPOID}
	}
	\DefineNamedFunc{fundamentalGroupoid}{X}{\Pi(X)}{ 
		\left( X, \frac{omega}{\TYPE{Homotopic}},\circ,x \mapsto (t \mapsto x) \right)  
	}
	\\
	\Theorem{FundamentalGroupoidIsGroupoid}
	{
		\forall X \in \TOP \. \TYPE{Groupoid}\Big( \Pi(X) \Big)
	}
	\NoProof
	\\
	\Theorem{manifolfdsFundamentalGroupoidHasCountableMorphisms}
	{
		\forall X \in \TOPM \. \forall x,y \in X \.
		\Big|\Omega(x,y)\Big| \le \aleph_0
	}
	\NoProof
}
\newpage
\subsection{Induced Functors}
\Page{
	\Theorem{PathHomotopyPreservedByC}
	{
		\forall X,Y \in \TOP \. 
		\forall f \in \TOP(X,Y) \. 
		\forall \alpha,\beta \in I \to X \.
		\alpha \sim \beta \Imply \alpha f \sim \beta f
	}
	\NoProof
	\\
	\DeclareFunc{inducedFunctor}{\prod X,Y \in \TOP \. \Pi(X) \Arrow{\CAT} \Pi(Y)}
	\DefineNamedFunc{inducedFunctor}{f}{f_*}{\Big(f,\Lambda \gamma \in \Pi(X)(p,q) \. f \circ \gamma\Big) }
	\\
	\Theorem{FundamentalGroupoidIsFunctor}
	{
		\Cov(\TOP,\mathsf{GROUPOID},\Pi) 
	}
	\NoProof
	\\
	\Theorem{FundamentalGroupIsomorphism}
	{
		\forall X,Y \in \TOP \. 
		\forall \varphi : X \ToIso{\TOP} Y \.
		\forall x \in X \.
		\pi(x) \cong_{\GRP} \pi\big( \varphi(x) \big)
	}
	\NoProof
	\\
	\DeclareType{Retraction}
	{
		\prod X \in \TOP \. 
		\prod R \subset X \.
		?(X \Arrow{\TOP} R) 
	}
	\DefineType{f}{Retraction}
	{
		\iota_R f = {\id}_R
	}
	\\
	\DeclareType{Retract}
	{
		\prod X \in \TOP \. 
		\prod R \subset X \.
		??X 
	}
	\DefineType{R}{Retraction}
	{
		\exists \TYPE{Retract}(X,R)
	}
	\\
	\Theorem{RetractOfCompactSpaceIsCompact}
	{
		\forall X : \TYPE{Compact} \. 
		\forall R : \TYPE{Retract}(X) \. 
		\TYPE{Compact}(R)
	}
	\NoProof
	\\
	\Theorem{RetractOfRetractIsRetract}
	{
		\forall X : \TYPE{Compact} \. 
		\forall R : \TYPE{Retract}(X) \.
		\forall S : \TYPE{Retract}(R) \. 
		\TYPE{Retract}(x,S)
	}
	\NoProof
	\\
	\Theorem{RetractOfConnectedSpaceIsConnected}
	{
		\forall X : \TYPE{Connected} \. 
		\forall R : \TYPE{Retract}(X) \. 
		\TYPE{Connected}(R)
	}
	\NoProof
}
\Page{
	\Theorem{InjectiveRetractFunctorProperty}
	{
		\forall X \in \TOP \.
		\forall R : \TYPE{Retract}(X) \.
		\forall p,q \in R \. \NewLine \.
		\TYPE{Injective}\Big( \Pi(R)(p,q),\Pi(X)(p,q), \iota_{R*} \Big)
	}
	\Say{r}{\bd \TYPE{Retract}(X)}{\TYPE{Retraction}(X,R)}
	\AssumeIn{\alpha,\beta}{\Pi(R)(p,q)}
	\Assume{[1]}{\iota_{R*}(\alpha) = \iota_{R*}(\beta)}
	\Say{[2]}{\bd \TYPE{Retraction}(X,R,r) \bd \iota_R [1]}
	{
		\iota_{R*} r_*(\alpha) = \alpha 
		\And
		\iota_{R*} r^*(\beta) = \beta
	}
	\Conclude{\Big[(\alpha,\beta).*\Big]}{\THM{PathHomotopyPreservedByC}}
	{
		\alpha = \beta
	}
	\DeriveConclude{[*]}{\bd^{-1} \TYPE{Injective}}{\TYPE{Injective}(\iota_*)}
	\EndProof
	\\
	\Theorem{SurjectiveRetractFunctorProperty}
	{
		\forall X \in \TOP \.
		\forall R : \TYPE{Retract}(X) \. \NewLine \. 
		\forall r : 
		\forall p,q \in R \.
		\TYPE{Surjective}\Big( \Pi(R)(p,q),\Pi(X)(p,q), r_{*} \Big)
	}
	\Assume{\alpha}{\Pi(R)(p,q)}
	\Conclude{[\alpha.*]}{ \bd \TYPE{Retraction}(X,R,r)}
	{
		\iota_{R*} r_*(\alpha) = \alpha
	}
	\DeriveConclude{[*]}{\bd^{-1} \TYPE{Surjectrve} }
	{
		\TYPE{Surjective}\Big( \Pi(R)(p,q),\Pi(X)(p,1), r_* \Big)
	}
	\EndProof
	\\
	\Theorem{RetractOfSimplyConnectedIsSimplyConnected}
	{
		\forall X : \TYPE{SimplyConnected} \.
		\forall R : \TYPE{Retract}(X) \. \NewLine \. 
		\TYPE{SimpltConnected}(R)
	}
	\NoProof
	\\
	\Theorem{FundamentalGroupoidPreservesProducts}
	{
		\forall X,Y \in \TOP \. 
		\Pi(X \times Y) = \Pi(X) \times \Pi(Y)
	}
	\NoProof
}
\newpage
\subsection{Homotopy Equivalence}
\Page{
	\DeclareFunc{HomotopyCategory}{\CAT}
	\DefineNamedFunc{HomotopyCategory}{}{\HTOP}
	{  \left( \TOP, \frac{\TOP}{\TYPE{Homotopic}},[\circ],[\id] \right)     }
	\\
	\Theorem{HomotopyEquivalence}
	{
		\forall X,Y \in \TOP \.
		X \cong_{\HTOP} Y \.
		\iff \NewLine \iff 
		\exists  \phi : X \Arrow{\TOP} Y :
		\exists \psi : Y \Arrow{\TOP} X
		\phi \psi \sim \id_X \And \psi \phi \sim \id_Y
	}
	\NoProof
	\\
	\DeclareType{DeformationRetraction}
	{
		\prod X \in \TOP \.
		\prod R \subset X \. 
		?\TYPE{Retraction}
	}
	\DefineType{r}{DeformationRetraction}
	{
		[\iota_R] =_{\HTOP} [r]^{-1} 
	}
	\\
	\DeclareType{DeformationRetract}
	{
		\prod X \in \TOP \. 
		?\TYPE{Subset}(X)
	}
	\DefineType{R}{DeformationRetract}
	{
		\exists \TYPE{DeformationRetraction}(X,R)
	}
	\\
	\DeclareType{StrongDeformationRetraction}
	{
		\prod X \in \TOP \.
		\prod R \subset X \. 
		?\TYPE{Retraction}
	}
	\DefineType{r}{StrongDeformationRetraction}
	{
		r \iota_R  \sim_R \id
	}
	\\
	\DeclareType{StrongDeformationRetract}
	{
		\prod X \in \TOP \. 
		?\TYPE{Subset}(X)
	}
	\DefineType{R}{StrongDeformationRetract}
	{
		\exists \TYPE{StrongDeformationRetraction}(X,R)
	}
	\\
	\DeclareType{Contractible}
	{
		?\TOP
	}
	\DefineType{X}{Contractible}
	{
		\forall x \in X \.
		\Big[\id_X\Big] =_{\HTOP} [x]
	}
	\\
	\Theorem{StarShapedIsContractible}
	{
		\forall V \in \TOPVS{\Reals} \.
		\forall A  : \TYPE{RelativelyStarshaped}(V) \.
		\TYPE{Contractible}(A)
	}
	\NoProof
	\\
	\Theorem{ContractibilityCondition}
	{
		\forall X \in \TOP \. 
		\TYPE{Contractible}(X)
		\iff
		X \cong_{\HTOP} \mathbf{pt}
	}
	\NoProof
}\Page{
	\Theorem{PathTranslationLemma}
	{
		\forall X,Y \in \TOP \.
		\forall \varphi,\psi : X \Arrow{\TOP} Y \.
		\forall H : \TYPE{Homotopy}(X,Y,\varphi,\psi) \.
		\varphi_* \Phi_h = \psi_* 
		\NewLine \where
		\NewLine
		h = \Lambda p \in X \. \Lambda t \in [0,1] \. H(p,t),
		\NewLine
		\Phi_h =  
			\Lambda p,q \in X \. 
			\Lambda \gamma \in \Pi(Y)\Big(\varphi(p),\varphi(q)\Big) \.
			h^{-1}_p \gamma h_q
	}
	\AssumeIn{p,q}{X}
	\AssumeIn{\gamma}{\Pi(X)(p,q)}
	\Say{[1]}{ \bd \varphi_* \bd \Phi_h   }
	{
		\varphi_* \Phi_h (\gamma) = 
		\Phi_h  \Big(  \gamma \varphi \Big) = 
		h^{-1}_p (\gamma \varphi) h_q
	}
	\Say{H'}
	{
		\Lambda t \in [0,1] \.
		h^{-1}_{p|[0,1-t]}  ( \gamma H(1-t,p) ) h_{p|[0,1-t]}
	}
	{
		\TYPE{Homotopy}(I,Y, \varphi_* \Phi_h(\gamma) , \psi_*(\gamma) )
	}
	\Conclude{\Big[(p,q).*\Big]}{\bd \TYPE{Homotopic}[1]}
	{
		\varphi_* \Phi_h(\gamma) = \psi_*(\gamma)
	}
	\DeriveConclude{[*]}{\Intro(=,\to)}
	{
		\varphi_* \Phi_h = \psi_*
	}
	\EndProof
	\\
	\Theorem{HomotopyInvariance}
	{
		\forall X,Y \in \TOP \.
		\forall [\varphi] : X \ToIso{\HTOP} Y \.
		\varphi_* : \Pi(X) \ToIso{\mathsf{SGRPD}} \Pi(Y)
	}
	\Say{(\psi,[1])}{\THM{HomotopyEquivalence}(X,Y,\varphi)}
	{
		\sum \psi : Y \Arrow{\TOP} X \. 
		\id_X \sim \varphi \psi \And \id_Y \sim \psi \varphi 
	}
	\Say{H}{\bd \TYPE{Homotopic}[1_1]}{\TYPE{Homotopy}(X,X,\varphi\psi,{\id}_X)}
	\Say{h}{\Lambda p \in X \. \Lambda t \in I \. H(t,p)}{X \to I \Arrow{\TOP} X}
	\Say{\Phi}{\Lambda p,q \in X \. \Lambda \gamma \in \Omega(p,q) \. h_p^{-1} \gamma h_q }
	{
		X^2 \to \Omega(p,q) \to \Omega( p,q)
	}
	\Say{[2]}{\THM{PathTranslationLemma}(X,X,\varphi \psi,{\id}_X,H)}
	{
			\varphi_* \psi_* = \Phi
	}
	\Say{[3]}{\THM{IsoAsComposition}[2]}
	{
		\TYPE{Injective}(\varphi_*)
		\And
		\TYPE{Surjectve}(\psi_*)
	}
	\Say{H'}{\bd \TYPE{Homotopic}[1_2]}{\TYPE{Homotopy}(Y,Y,\psi\varphi,{\id}_Y)}
	\Say{h'}{\Lambda p \in Y \. \Lambda t \in I \. H'(t,p)}{Y \to I \Arrow{\TOP} Y}
	\Say{\Phi'}{\Lambda p,q \in Y \. \Lambda \gamma \in \Omega(p,q) \. {h'}_p^{-1} \gamma h'_q }
	{
		Y^2 \to \Omega(p,q) \to \Omega( p,q)
	}
	\Say{[4]}{\THM{PathTranslationLemma}(Y,Y,\psi \varphi ,{\id}_X,H)}
	{
			 \psi_* \varphi_* = \Phi'
	}
	\Say{[5]}{\THM{IsoAsComposition}[2]}
	{
		\TYPE{Injective}(\psi_*)
		\And
		\TYPE{Surjectve}(\varphi_*)
	}
	\Conclude{[*]}{\THM{GrpIsomorphism}[3][5]}
	{
		\varphi_* : \Pi(X) \ToIso{\mathsf{SGRPD}} \Pi(Y)	
	}
	\EndProof
	\\
	\DeclareFunc{MappingCyllinder}
	{
		\prod_{X,Y \in \TOP} (X \Arrow{\TOP} Y) \to \TOP
	}
	\DefineNamedFunc{MappingCyllinder}{f}{Z_f}
	{
		(X \times I) \sqcup_\varphi Y 
		\NewLine
		\where 
		\NewLine
		\varphi = \Lambda (x,0) \in X \times \{0\} \. f(x)
	}
}
\Page{
	\Theorem{MappingCyllinderTHM}
	{
		\forall X,Y \in \HTOP \.
		\forall [f] : X \ToIso{\TOP} Y \.
		\forall \pi_X(Z_f),\pi_Y(Z_f) : \TYPE{DeformationRetract}(Z_f)
	}
	\Say{q}{\Lambda (p,t) \times \in X \times I \. \iota_{X \times I}(p,t) }
	{
		X \times I \Arrow{\TOP} Z_f
	}
	\Say{q'}{\Lambda y \in Y \. \iota_{Y}(y) }
	{
		Y  \Arrow{\TOP} Z_f
	}
	\Assume{z}{Z_f}
	\Say{[1]}{\bd Z_f(z)}{\exists x \in X : \exists t \in I : z = [x,t]  \Big| \exists y \in Y : z = [y]}
	\AssumeIn{x}{X}
	\AssumeIn{t}{I}
	\Assume{[2]}{z = [x,t]}
	\Say{H(z)}{\Lambda s \in [0,1] \. \Big[x,t(1-s)\Big]}{I \to Z_f}
	\Conclude{A(z)}{[x,0]}{ Z_f(z) }
	\Derive{[2]}{\Intro(\forall)}{\forall x \in X \. \forall t \in I \. z = [x,t] \Imply A(z) \in Z_f, H : I \to Z_f}
	\AssumeIn{y}{Y}
	\Assume{[3]}{z = [y]}
	\Say{H(z)}{\Lambda s \in [0,1] \. [y]}{I \to Z_f}
	\Conclude{A(z)}{[y]}{ Z_f }
	\Derive{[3]}{\Intro(\forall)}{\forall y \in Y \. z = [y] \Imply A(z) \in Z_f, H : I \to Z_f }
	\Say{A(z)}{\Elim(|)[1,2,3]\bd Z_f}{ Z_f  }
	\Conclude{H}{\Elim(|)[1,2,3]\bd Z_f}{I \to Z_f}
	\Derive{A}{\Intro(\to)}{Z_f \Arrow{\TOP} Z_f}
	\Derive{H}{\Intro(\to)}{I \Arrow{\TOP} Z_f \Arrow{\TOP} Z_f}
	\Say{[1]}{\ByConstr H}{\TYPE{Homotopy}(Z_f, Z_f, \id, A, H)}
	\Say{[2]}{\bd^{-1} \TYPE{StrongDeformationRetract}[1]}
	{
		\TYPE{StrongDeformationRetraction}\Big(Z_f,q'(Y)\Big)
	}
	\Say{\Big( g, [3] \Big)}
	{
		\THM{HomotopyEquivalence}(X,Y,f)
	}
	{
		\sum g : Y \Arrow{\TOP} X \.
		fg \sim {\id}_X \And gf \sim {\id}_Y
	}
	\Say{F}{\bd \HTOP [3_1]}{\TYPE{Homotopy}(X,X,fg,{\id}_X)}
	\Say{G}{\bd \HTOP [3_2]}{\TYPE{Homotopy}(Y,Y,gf,{\id}_Y)}
	\Assume{z}{Z_f}
	\Say{[1]}{\bd Z_f(z)}{\exists x \in X : \exists t \in I : z = [x,t]  \Big| \exists y \in Y : z = [y]}
	\AssumeIn{x}{X}
	\AssumeIn{t}{I}
	\Assume{[2]}{z = [x,t]}
	\Say{H'(z)}{\Lambda s \in [0,1] \. \Big[F\big( f(x),1-t\big)\Big]}{I \to Z_f}
	\Conclude{H''(z)}{\Lambda s \in [0,1] \. \Big[G(x,st),t\Big]}{ I \to Z_f }
	\Derive{[2]}{\Intro(\forall)}{\forall x \in X \. \forall t \in I \. z = [x,t] \Imply H',H'' : I \to Z_f}
	\AssumeIn{y}{Y}
	\Assume{[3]}{z = [y]}
	\Say{H'(z)}{\Lambda s \in [0,1] \. \Big[F(y,1-t)\Big]}{I \to Z_f}
	\Conclude{H''(z)}{\Lambda s \in [0,1] \. \Big(g(y),t\Big)}{ I \to Z_f }
	\Derive{[3]}{\Intro(\forall)}{\forall y \in Y \. z = [y] \Imply H', H'' : I \to Z_f }
	\Say{H'}{\Elim(|)[1,2,3]\bd Z_f}{ Z_f  }
}\Page{
	\Conclude{H''}{\Elim(|)[1,2,3]\bd Z_f}{I \to Z_f}
	\Derive{H'}{\Intro(\to)}{I \Arrow{\TOP} Z_f \Arrow{\TOP} Z_f}
	\Derive{H''}{\Intro(\to)}{I \Arrow{\TOP} Z_f \Arrow{\TOP} Z_f}
	\Say{B}{H'(0)}{Z_f \Arrow{\TOP} Z_f}
	\Say{C}{H'(1)}{Z_f \Arrow{\TOP} Z_f}
	\Say{[4]}{\ByConstr B}{ \TYPE{Homotopty}(Z_f,Z_f,{\id}_{Z_f},B)}
	\Say{[5]}{\ByConstr C}{ \TYPE{Homotopty}(Z_f,Z_f,{\id}_{Z_f},C)}
	\Say{[*]}{\bd^{-1} \TYPE{StrongDeformationRetract}[1]}
	{
		:\TYPE{StrongDeformationRetraction}\Big(Z_f,q(X),C\Big)
	}
	\EndProof
}
\newpage
\subsection{The Circle }
\Page{
	\DeclareFunc{circleParametrization}
	{
		\Reals \to \Sphere^1
	}
	\DefineNamedFunc{circleParametrisation}{t}{s(t)}
	{
		\exp(\mathrm{i}t)
	}
	\\
	\DeclareType{Lift}{\prod_{X \in \TOP} (X \Arrow{\TOP} \Sphere^1) \to ?(X \Arrow{\TOP} \Reals) }
	\DefineType{g}{Lift}{\Lambda f : X \Arrow{\TOP} \Sphere^1 \. g s = f }
	\\
	\DeclareType{Liftable}{\prod_{X \in \TOP} ?(X \Arrow{\TOP} \Sphere^1)}
	\DefineType{f}{Liftable}{\exists \TYPE{Lift}(X,f)}
	\\
	\Theorem{SpirallingAtlas}
	{
		\forall z \in \Sphere^1 \. 
		\exists U \in \U(z) :
		\exists U' : \Nat \to \TYPE{OpenInterval}(\Reals) :
		s^{-1}(U) = \bigsqcup^\infty_{n=1} U'_n \And \NewLine \And \forall n \in \Nat \.
		s_{|U'_n} : U'_n \ToIso{\TOP} U
	}
	\SayIn{U}{\Sphere^1 \setminus \{-z\}}{\U(z)}
	\Say{[1]}{\bd \TOP(\Reals,\Sphere^1,s) }{s^{-1}(U) \in \T(\Reals)}
	\Say{\Big( N, U', [3]\Big)}{\THM{RealOpenSubsetRepresentation}}
	{
		\sum N \in \aleph_0 \. \sum U' : N \to \TYPE{OpenInterval}(\Reals) \.
		\NewLine \. 
		s^{-1}(U) = \bigsqcup^N_{n=1} U'_n
	}
	\Say{\Big(t, [2]\Big)}{\bd s \ByConstr U}{
		\sum t \in \Reals : s^{-1\c}(U) = \bigsqcup^\infty_{n=-\infty} \{ t + 2\uppi n \}
	}
	\Say{[4]}{[3][2]}{N = \Nat}
	\Say{[5]}{\bd s [3][2]}{  \forall n \in \Nat \. |U'_n| < 2 \uppi }
	\Conclude{[*]}{\bd s [5]}
	{
		\forall n \in \Nat \. S_{|U'_n} : U'_n \ToIso{\TOP} U
	}
	\EndProof
	\\
	\DeclareType{EvenlyCovered}{?\T(\Sphere^1)}
	\DefineType{U}{EvenlyCovered}{
		\exists U' : \Nat \to \TYPE{OpenInterval}(\Reals) :
		s^{-1}(U) = \bigsqcup^\infty_{n=1} U'_n \And \NewLine \And \forall n \in \Nat \.
		s_{|U'_n} : U'_n \ToIso{\TOP} U	
	}
	\\
	\Theorem{CircleSectionLemma}
	{
		\forall U : \TYPE{EvenlyCovered} \.
		\forall z \in U \.
		\forall r \in s^{-1}(z)
		\exists \sigma : \TYPE{LocalSection}(U,\Reals,s) \.
		\sigma(z) = r
	}
	\NoProof
}
\Page{
	\Theorem{UniqueLiftingProperty}
	{
		\forall X : \TYPE{Connected} \.
		\forall f : X  \Arrow{\TOP} \Sphere^1 \.
		\forall g,g' : \TYPE{Lift}(X,f) \. 
		\forall x \in X \. \NewLine \.
		\forall [0] : g(x) = g'(x) \.
		g = g'
	}
	\Say{\X}{ \{ x \i x  : g(x) = g'(x)  \} }{?X}
	\Say{[1]}{[0]\ByConstr \X }{\X \neq \emptyset}
	\Say{[2]}{\bd \TYPE{Lift}(g)}{ g s = f }
	\Say{[3]}{\bd \TYPE{Lift}(g')}{ g' s = f }
	\Assume{p}{\X}
	\Say{\Big(U,[4]\Big)}{\THM{SpirallingAtlas}\big(f(x)\big)}
	{
		U : \TYPE{EvenlyCovered} \. f(p) \in U
	}
	\Say{U',[5]}{\bd \TYPE{EvenlyCovered}(U)}
	{
		\sum \Nat \to \T(\Reals) \. 
		\bigsqcup^\infty_{n=1} 
		s^{-1}(U) = \bigsqcup^\infty_{n=1} U' \And
		\forall n \in \Nat \. s_{|U'_n} U'_n \ToIso{\TOP} U
	}
	\Say{n,[6]}{\bd \TYPE{Preimage}[5_1][2]\ByConstr \X }
	{
		\sum n \in \Nat \.   g(p) \in U'
	}
	\Say{V}{g^{-1}(U'_n) \cap {g'}^{-1}(U'_n)}{\T(X)}
	\Say{[7]}{[6]\ByConstr V}{ p \in V}
	\Conclude{[p.*]}{[2][3]\ByConstr V [5_2][6]}
	{
		V \subset \X
	}
	\Derive{[4]}{\THM{OpenByOpenCover}}{\X \in \T(\X)}
	\Say{[5]}{\bd \TYPE{Continuous} (g - g')\ByConstr \X}{\TYPE{Closed}(X,\X)}
	\Say{[6]}{\bd \TYPE{Connected}(X)[1,4,5] }{\X = X}
	\Conclude{[*]}{[6]\ByConstr \X}{g = g'}
	\EndProof
	\\
	\Theorem{HomotopyLifitingProperty}
	{
		\forall X : \TYPE{LocallyConnected} \.
		\forall f,f' : X \Arrow{\TOP} \Sphere_1 \.
		\forall H : \TYPE{Homotopy}(X,\Sphere^1,f,f') \. \NewLine \. 
		\forall g : \TYPE{Lift}(X,f) \. 
		\exists! g' : \TYPE{Lift}(X,f') :
		\exists! \widetilde{H} : \TYPE{Homotopy}(X,\Reals,g,g')
	}
	\AssumeIn{x}{X}
	\AssumeIn{t}{I}
	\Say{\Big(U,[1]\Big)}{\THM{SpirallingAtlas}\big(H(t,x)\big)}
	{
		U : \TYPE{EvenlyCovered} \. H(t,x) \in U
	}
	\Say{W'}{H^{-1}(U)}{\U(t,x)}
	\Conclude{\Big( V_t,J_t,[2]\Big)}{\bd \TYPE{ProductTopology}}
	{
		\sum J \in \U(t) \.
		\sum V \in \U(x) \.
		J \times V \subset W'
	}
	\Derive{(V,J,[1])}{\Intro\Act{\prod}}
	{
		\prod_{t \in I} \sum V_t \in \U(x) \. \sum J_t \in \U(t) \. 
		\exists U : \TYPE{UniformlyCovered} :
		V_t \times J_t \subset H^{-1}(U)
	}
	\Say{[2]}{ \bd^{-1} \TYPE{OpencCover} \bd (V,J)  }
	{
		\TYPE{OpenCover}\Big(I\times\{x\},J \times V\Big)
	}
	\Say{\Big(n,t,[3]\Big)}{\bd \TYPE{Compact}(I \times \{x\},J \times V )}
	{
		\sum_{n=1}^\infty \sum t : [1,\ldots,n] \to I \.   \TYPE{OpenCover}\Big( I \times \{x\}, J_t \times V_t \Big)
	}	
	\Say{\Big(W,[4]\Big)}{\bd \TYPE{LocallyConncected}\left( \bigcap^n_{i=1} V_i , x \right)}
	{
		\sum W \in \U(x) \And \TYPE{Connected}  \.    
		W \subset \bigcup^n_{i=1} V_i 
	}
	\Say{\lambda}{\THM{LebesgueNumberExists}(J_t)}{\TYPE{LebesgueNumber}(J_t)}
	\Say{\Big(m,[5]\Big)}{\THM{ReductionInfima}(\lambda)}{\sum^\infty_{m=1} \frac{1}{m} < \lambda}
	\Say{[6]}{ \bd \TYPE{LebesgueNumber} [5][1]  }
	{
		\forall j \in [1,\ldots,m] \. 
		\exists U : \TYPE{UniformlyCovered} \.
		H\left( \left[ \frac{j-1}{m},\frac{j}{m} \right] \times W \right) \subset U
	}
}
\Page{
	\AssumeIn{j}{[1,\ldots,m]}
	\Say{\Big(U,[7]\Big)}{[6](j)}
	{
		\sum U : \TYPE{UniformlyCovered} \. 
		H\left( \left[ \frac{j-1}{m},\frac{j}{m} \right] \times W \right) \subset U
	}
	\Say{\Big(\sigma,[8]\Big)}
	{
		\THM{CircleSectionLemma}\Big(U,f(x),g(x)\Big)
	}
	{
		\sum \sigma : \TYPE{LocalSection}(U,s) \. f \sigma (x) = g(x)
	}
	\Assume{[9]}{j=1}
	\Assume{(t,x)}{\left[0,\frac{1}{m}\right] \times W}
	\Conclude{\widetilde{H}(t,x)}{H \sigma_1}{\Reals}
	\Derive{\widetilde{H}}{\Intro(\to)}{\left[0,\frac{1}{m}\right] \times W \Arrow{\TOP} \Reals}
	\Say{[10]}{\bd \TYPE{LocalSection}(\sigma)\ByConstr \widetilde{H}}
	{
		\forall x \in W \.  \widetilde{H} s (x,0) =  s \sigma H(x,0) = H(x,0) = f(x)
	}
	\Conclude{[9.*]}{\THM{UniqueLiftingProperty}[10]}
	{
		\forall x \in W \. \widetilde{H}(x,0) = g(x)
	}
	\Derive{(\widetilde{H},[9])}{\Intro\Act{\sum}}
	{
		\sum \widetilde{H} : \left[0,\frac{1}{m}\right] \times W \to \Reals \.
		\forall x \in W \. \widetilde{H}(x,0) = g(x)
	}
	\Assume{[10]}{j > 1}
	\Assume{\widetilde{H}}{\left[0,\frac{j-1}{m}\right] \times W \to \Reals}
	\Assume{[11]}{ \forall x \in W \. \widetilde{H}(0,x) = g(x)   }
	\Assume{(t,x)}{\left[\frac{k-1}{m},\frac{1}{m}\right] \times W}
	\Conclude{\widetilde{H}(t,x)}{H \sigma_1}{\Reals}
	\Derive{\widetilde{H}}{\Intro(\to)\THM{UniqueLiftProperty}()}{\left[0,\frac{k}{m}\right] \times W \Arrow{\TOP} \Reals}
	\Say{[12]}{\bd \TYPE{LocalSection}(\sigma)\ByConstr \widetilde{H}}
	{
		\forall x \in W \.  \widetilde{H} s (x,0) =  s \sigma H(x,0) = H(x,0) = f(x)
	}
	\Conclude{[10.*]}{\THM{UniqueLiftingProperty}[10]}
	{
		\forall x \in W \. \widetilde{H}(x,0) = g(x)
	}
	\DeriveConclude{(\widetilde{H},[9])}{\bd [1,\ldots,m]}
	{
		\sum \widetilde{H} : I \times W \to \Reals \.
		\forall x \in W \. \widetilde{H}(x,0) = g(x)
	}
	\Derive{(\widetilde{H},[1])}{\THM{UniqueLiftProperty}}
	{
		\sum \widetilde{H} : I \times X \to \Reals \.
		\forall x \in W \. \widetilde{H}(x,0) = g(x)
	}
	\Say{g'}{\Lambda x \in X \. \widetilde{H}(x,1)}{X \Arrow{\TOP} \Reals}
	\Conclude{[*]}{\bd^{-1}\TYPE{Homotopy}}
	{
		\TYPE{Homotopy}(X,\Reals,g,g',\widetilde{H})
	}
	\EndProof
	\\
	\Theorem{PathLifitingProperty1}
	{
		\forall \gamma : I \Arrow{\TOP} \Sphere^1 \.
		\exists \TYPE{Lift}(I,\gamma)
	}
	\NoProof
	\\
	\Theorem{PathLifitingProperty2}
	{
		\forall \gamma : I \Arrow{\TOP} \Sphere^1 \.
		\forall \alpha, \beta :  \TYPE{Lift}(I,\gamma) \.
		\exists n \in \Nat \.
		\alpha - \beta = 2\uppi n
	}
	\NoProof
}
\Page{
	\Theorem{PathHomotopyCriterion}
	{
		\forall f,f' : I \to \Sphere^1 \.
		\forall g : \TYPE{Lift}(I,f) \.
		\forall g' : \TYPE{Lift}(I,f') \.
		\forall [0] : f(0) = f'(0) \. \NewLine \. 
		\forall [00] : f(1) = f'(1) \. 
		\forall [000] : g(0) = g'(0) \.
		g(1) = g'(1) \iff f \approx f'
	}
	\NoProof
	\\
	\DeclareFunc{windingNumber}{\prod z \in \Sphere_1 \. \Omega(z) \to \Int}
	\DefineNamedFunc{windingNumber}{\gamma}{w(\gamma)}{ 
		\frac{\alpha(1) - \alpha(0)}{2\uppi} 
		\NewLine \where \NewLine
		\alpha = \THM{PathLifitingProperty1}(\gamma)
	}
	\\
	\Theorem{WindingNumberRotationInvariance}{
		\forall z,u \in \Sphere_1 \. 
		\forall \gamma \in \Omega(z) \.
		w(u\gamma) = w(\gamma)
	}
	\Say{\Big( x, [1] \Big)}{\THM{PolarRepresentation}(u)}
	{
		\sum x \in [0,2\uppi) \. u = e^{\mathrm{i}x}
	}
	\Say{\alpha}{\THM{PathLifitingProperty}(\gamma)}{\TYPE{Lift}(I,\gamma)}
	\Say{\beta}{ \alpha + x}{I \to \Reals}
	\Say{[2]}
	{
		\Lambda t \in I \. 
		[1]\bd \TYPE{Lift}(I,\gamma,\alpha)(t) 
		\bd \RING(\Complex)
		\THM{ExponentProduct}(\mathrm{i}t,\mathrm{i}\alpha(t)) 
		\ByConstr^{-1} \beta
	}
	{
		\NewLine : 
		\forall t \in I \.   u\gamma(t) =  
		e^{\mathrm{i}x} e^{\mathrm{i}\alpha(t)} = 
		\exp\Big(\mathrm{i}\big(\alpha(t) + x\big)\Big) = 
		e^{\mathrm{i}\beta(t)}
	}
	\Say{[3]}{\bd^{-1} \TYPE{Lift}[2]}
	{
		\TYPE{Lift}(I,u\gamma,\beta)
	}
	\Conclude{[*]}{\bd w(u\gamma)[3]\ByConstr \beta \bd^{-1} w(\gamma)}{w(u\gamma) = w(\gamma)}
	\EndProof
	\\
	\Theorem{WindingNumberLoopClassification}
	{
		\forall z \in \Sphere^1 \.
		\forall \gamma,\gamma' \in \Omega(z) \.
		w(\gamma) = w(\gamma') \iff 
		\gamma \approx \gamma'
	}
	\Say{\alpha}{\THM{PathLifitingProperty}(\gamma)}{\TYPE{Lift}(I,\gamma)}
	\Say{\alpha'}{\THM{PathLifitingProperty}(\gamma')}{\TYPE{Lift}(I,\gamma')}
	\Say{\beta}{\alpha' - \alpha'(0) + \alpha(0)}{I \to \Reals}
	\Say{[0]}{\bd \Omega(z) \ByConstr \beta \THM{PathLifitingProperty}}
	{
		\TYPE{Lift}(I,\omega',\beta)
	}
	\Assume{[1]}{w(\gamma) = w(\gamma')}
	\Say{[2]}{[1]\bd w \bd \alpha \bd \alpha'}{ \alpha(1) - \alpha(0) = \alpha'(1) - \alpha'(0)  }
	\Say{[3]}{\ByConstr \beta [2]}{\beta = \alpha' - \alpha'(1) + \alpha'(1) }
	\Say{[4]}{[3](1)}{\beta(1) = \alpha(1)}
	\Say{[5]}{\ByConstr \beta(0)}{\beta(0) = \alpha(0)}
	\Conclude{[1.*]}{\THM{PathHomotopyCriterion}[0][4][5]}{\gamma \approx \gamma'}
	\Derive{[1]}{\Intro(\Imply)}{ w(\gamma) = w(\gamma') \Imply \gamma \approx \gamma'  }
	\Assume{[2]}{\gamma \approx \gamma'}
	\Say{[3]}{\ByConstr \beta(0)}{\beta(0) = \alpha(0)}
	\Say{[4]}{\THM{PathHomotopyCriterion}(\gamma,\gamma',\alpha,\beta)[3]}
	{
		\alpha(1) = \beta(1)
	}
	\Conclude{[*]}{\bd 2 [3][4]}{w(\gamma) = w(\gamma')}
	\EndProof
}
\Page{
	\Theorem{FundamentlGroupOfThCircle}
	{
		\pi(\Sphere^1) \cong_{\GRP} \Int
	}
	\Say{F}{\Lambda n \in \Int \. [s]^{\circ n}}{\Int \to \pi(\Sphere^1)}
	\Say{G}{\Lambda [\gamma] \in \pi(\Sphere^1) \. w(\gamma) }{\pi(\Sphere^1) \to \Int}
	\Say{[1]}{\Lambda n \in \Int \. \ByConstr F \ByConstr G \bd s \bd w}
	{
		\forall n \in \Int \. 
		FG(n) = 
		G[s]^{\circ n} =  
		w\Big(s^{n} \Big) =
		n
	}
	\Say{[2]}{
		\Lambda [\gamma] \in \pi(\Sphere^1) \.
		\ByConstr G
		\ByConstr w
		\THM{WindingNumberLoopClassification}\Big(\gamma,s^{w(\gamma)}\Big)
	}
	{
		\NewLine : 
		\forall [\gamma] \in \pi(\Sphere^1) \.
		GF[\gamma] = 
		F\big( w(\gamma) \big) =
		[s]^{\circ w(\gamma)} = [\gamma]
	}
	\Say{[3]}{\bd^{-1} \TYPE{Bijection}[1][2]}
	{
		F : \Int \ToIso{\SET} \pi(\Sphere^1)
	}
	\Say{[4]}{ 
		\Lambda n,m \in \Int \. 
		\ByConstr F(n + m)
		\THM{GroupExponentiation}\Big( \pi\big(\Sphere^1\big)\Big)
		\ByConstr^{-1} F
	}
	{
		\NewLine : 
		\forall n,m  \in \Int \.
		F(n + m) = 
		[s]^{\circ(n + m)} = 
		[s]^{\circ n} [s]^{\circ m} = 
		F(n) F(m)
	}
	\Conclude{[*]}{\bd \GRP [4][3]}
	{
		F : \Int \ToIso{\GRP} \pi(\Sphere^1)
	}
	\EndProof
	\\
	\Theorem{FundamentalGroupOfThePuncturedPlane}
	{
		\pi\Big(\Complex \setminus \{0\}\Big) \cong_\GRP \Int
	}
	\NoProof
	\\
	\Theorem{FundamentalGroupOfTheNTorus}
	{
		\forall n \in \Nat \.
		\pi\Big(\mathbb{T}^n\Big) \cong_\GRP \Int^n
	}
	\NoProof
	\\
	\DeclareFunc{degreeOfTheMap}{ \End_{\TOP}(\Sphere_1) \to \Int}
	\DefineNamedFunc{degreeOfTheMap}{f}{\deg f}{ w(sf)}
	\\
	\Theorem{DegreeCharacterisation}
	{
		\forall f : \End_\TOP\big(\Sphere_1 \big) \.
		\deg f =   \deg \left( \frac{f}{f(0)} \right)_*
	}
	\NoProof
	\\
	\Theorem{HomotopicMapsHaveSameDegree}
	{
		\forall f,f' : \End_\TOP\big(\Sphere_1 \big) \. 
		f \sim f' \Imply \deg f = \deg f'
	}
	\Say{[1]}{\THM{RotatioIsHomotopicToId}\left(\frac{f}{f(0)},\frac{f'}{f'(0)}\right)}
	{
		\frac{f}{f(0)} \sim \frac{f'}{f'(0)}
	}
	\Say{\Big( h,[2]\Big)}{ \THM{PathTranslationLemma}[1]}
	{
		\sum h \in \Omega(1) \. \left(\frac{f}{f(0)} \right)_* \Phi_h = \left(\frac{f'}{f'(0)}\right)_* 
	}
	\Say{[3]}{\bd \ABEL(\Int) \THM{FundamentalGroupOfTheCircle}\bd \Phi_h}{\Phi_h = \id}
	\Say{[4]}{[2][3]}{  \left(\frac{f}{f(0)} \right)_* = \left(\frac{f'}{f'(0)}\right)_* }
	\Conclude{[*]}{\THM{DegreeCharacterisation}^2(f)(f')[4]}{\deg f = \deg f'}
	\EndProof
}
\Page{
	\Theorem{DegreeIsHomo}
	{
		\forall f,g \in \End_{\TOP}\big( \Sphere_1 \big) \.
		\deg fg = \deg f \deg g
	}
	\Conclude{[*]}{
		\THM{DegreeCharacterisation}(fg)
		\bd \Field(\Complex)
		\THM{HomoDegIsHomo}(\ldots)
		\THM{DegreeCharacterisation}^2(f)(\mu_{f(0)}g) \NewLine  
		\THM{HomotopicMapsHaveSameDegrees}(g,\mu_{f(0)}g)
	}
	{
		\NewLine :  
		\deg fg = 
		\deg \left( \frac{fg}{fg(1)} \right)_* = 
		\deg \left( \frac{1}{fg(1)} g\left( \frac{f(1)f}{f(1)} \right)\right)_*  =
		\deg \left( \frac{\mu_{f(1)} g}{fg(1)} \right)_* \deg \left( \frac{f}{f(1)} \right)_* =
		\deg g \deg f
	}
	\EndProof
	\\
	\Theorem{DegreeClassificationOfCircleEnd}
	{
		\forall f,f' \in \End_\TOP\big(\Sphere^1\big) \.
		\deg f = \deg f' \Imply f \sim f'
	}
	\Say{[1]}{\bd \deg}{w(sf) = w(sf')}
	\Say{[2]}{\THM{WindingNumberLoopClassification}}{sf \approx sf'}
	\Assume{[3]}{f(1) = f'(1)}
	\Say{H}{\bd \TYPE{Homotopic}}{\TYPE{Homotopy}(I,\Sphere^1,sf,sf')}
	\Say{\widetilde{H}}{\frac{H}{\id \times s}}{\TYPE{Homotopy}\Big(\Sphere^1,\Sphere^1,f,f'\Big)}
	\Conclude{[3.*]}{\bd^{-1} \TYPE{Homotopic}(\widetilde{H})}{f \sim f'}
	\DeriveConclude{[*]}{\THM{RotationIsHomotopicToId}}{f \sim f'}
	\EndProof
	\\
	\Theorem{SurjectiveByDegree}
	{
		\forall f \in \End_\TOP\big(\Sphere^1\big) \.
		\deg f \neq 0 \Imply \TYPE{Surjective}(\Sphere^1,\Sphere^1,f)
	}
	\NoProof
	\\
	\Theorem{HasFixedPointByDegree}
	{
		\forall f \in \End_\TOP\big(\Sphere^1\big) \.
		\deg f \neq 1 \Imply \Fix(f) \neq \emptyset	
	}
	\Assume{[1]}{\Fix(f) = \emptyset}
	\Say{H}{\Lambda t \in I \. \Lambda z \in \Sphere^1 \.  \frac{(1-t)f(z) - tz}{\Big\| (1-t)f(z) - tz \Big\|}}
	{
		\TYPE{Homotopy}\Big(\Sphere^1,\Sphere^1,f,\mathrm{inv}(\Complex,+)\Big)
	}
	\Say{[2]}{\THM{HomtopicMapsHaveSameDegree}(H)}
	{
		\deg f = \deg \mathrm{inv}(\Complex,+) = 1
	}
	\Conclude{[1.*]}{[2][0]}{\bot}
	\DeriveConclude{[*]}{\Elim(\bot)}{\Fix(f) \neq \emptyset}
	\EndProof
	\\
	\Theorem{DegreeZeroByExtensionToTheCell}
	{
		\forall f : \Sphere^1 \Arrow{\TOP} \Sphere^1 \.
		\forall F : \mathbb{D}^2 \Arrow{\TOP} \Sphere^1 \.
		F_{|\Sphere^1} = f \Imply \deg f = 0
	}
	\Say{[1]}{\bd \FUNC{CircleRepresentative}}
	{
		\widetilde{sf} = f
	}
	\Say{[2]}{\THM{ExtensionImplyNullHomotopic}[0][1]}
	{
		\TYPE{NullHomotopic}(sf)
	}
	\Say{[3]}{\bd \TYPE{NullHomotopic} \bd^{-1} w}{w(sf) = 0}
	\Conclude{[*]}{\bd \deg}{\deg f = 0}
	\EndProof
}
\Page{ 
	\Theorem{MainTheoremOfAlgebra}
	{
		\forall p \in \Complex[x] \.
		\deg p > 1 \Imply \rho(p) \neq \emptyset
	}
	\SayIn{n}{\deg p}{\Nat}
	\Say{\Big(a, [1] \Big)}{\bd p \ByConstr n}
	{
		\sum a : [1,\ldots,n] \to \Complex \.
		p(x) \sim  x^n + \sum^n_{i=1} a_i x^{i-1}
	}
	\Assume{[2]}{\rho(p) \neq \emptyset}
	\Say{f}{\Lambda z \in \mathbb{D}^2 \. \frac{p(z)}{\| p(z)\|}}
	{
		\mathbb{D}^2 \Arrow{\TOP} \Sphere^1
	}
	\Say{[3]}{\THM{DegreeZeroByExtensionToTheCell}}{\deg f_{|\Sphere^1} = 0}
	\Say{H}{
		\Lambda t \in I \. 
		\Lambda z \in \Sphere^1 \.  
		\frac{t^n p\left( \frac{z}{t} \right)}
		{
			\left\|
				t^n p\left( \frac{z}{t} \right)
			\right\|
		}
	}
	{
		\TYPE{Homotopy}\Big( p, z^n \Big)
	}
	\Say{[4]}{\THM{HomotopicMapsHaveSameDegree}(H)}
	{
		\deg p = n
	}
	\Conclude{[2.*]}{[4][3]}
	{
		\bot
	}
	\DeriveConclude{[*]}{\Elim(\bot)}
	{
		\rho(p) \neq \emptyset
	}
	\EndProof
	\\
	\Theorem{BrauwerFixedPointTHM}{
		\forall f : \mathbb{D}^2 \Arrow{\TOP} \mathbb{D}^2 \.
		\Fix(f) \neq \emptyset
	}
	\Assume{[1]}{\Fix(f) = \emptyset}
	\Say{F}{\frac{x-f(x)}{\Big\|x-f(x)\Big\|}}
	{
		\mathbb{D}^2 \Arrow{\TOP} \Sphere^1
	}
	\Say{[2]}{\THM{DegreeZeroByExtensionToTheCell}}{\deg F_{|\Sphere^1} = 0}
	\Say{H}{\Lambda t \in I \. \Lambda z \in \Sphere^1 \.  \frac{z - tf(z)}{\Big\| z - tf(z)\Big\|}}
	{
		\TYPE{Homotopy}\Big( \Sphere^1,\Sphere^1, F_{|\Sphere^1}, \id \Big)
	}
	\Say{[3]}{\THM{HomotopyPreservesDegree}(H)}{\deg F_{|\Sphere^1} = 1}
	\Conclude{[1.*]}{[2][3]}{\bot}
	\DeriveConclude{[*]}{\Elim(\bot)}{\Fix(f) \neq \emptyset}
	\EndProof
	\\
	\Theorem{InjectiveDegree}
	{
		\forall f : \Sphere^1 \Arrow{\TOP} \Sphere^1 \.
		\TYPE{Injective}(\Sphere_1,\Sphere_1,f) \Imply |\deg f| = 1
	}
	\NoProof
	\\
	\\
	\Theorem{DeifferentDegreesImplyAntipodalValuesExists}
	{
		\NewLine ::
		\forall f ,g : \Sphere^1 \Arrow{\TOP} \Sphere^1 \.
		\deg f \neq \deg g \Imply \exists z \in \Sphere^1 : f(z) = -g(z)
	}
	\NoProof
}
\newpage
\subsection{Index for Plane Vector Fields}
\Page{
	\Conclude{\TYPE{FlatVectroField} = \mathfrak{X}}{ \prod^\infty_{n=0} \Reals^n \Arrow{\TOP} \Reals^n }{\Int_+ \to \VS{\Reals} }
	\\
	\DeclareType{SingularPoint}{\prod_{n=0}^\infty \VF(n) \to ?\Reals^n}
	\DefineNamedType{s}{SingularPoint}{\Lambda V \in \VF(n) \. s \in \mathcal{S}_V}{V(s) = 0}
	\\
	\DeclareType{RegularPoint}{\prod_{n=0}^\infty \VF(n) \to ?\Reals^n}
	\DefineNamedType{r}{SingularPoint}{\Lambda V \in \VF(n) \. r \in \mathcal{R}_V}{V(r) \neq  0}
	\\
	\DeclareType{IsolatedSingularPoint}{\prod_{n=0}^\infty \prod V \in  \VF(n) \. ?\mathcal{S}_V}
	\DefineType{s}{IsolatedSingularPoint}{\exists U \in \U(s) \. U \setminus \{s\} \subset \mathcal{R}_V}
	\\
	\DeclareType{RegularLoop}{\prod_{V \in \VF(2)} ?\Big(I \Arrow{\TOP} \mathcal{R}_V\Big) }
	\DefineType{\gamma}{RegularLoop}{\gamma(0) = \gamma(1)}
	\\
	\DeclareFunc{windingNumberInThePuncturedSpace}{\Big(I \Arrow{\TOP} \Reals^2 \setminus\{0\}\Big) \to \Int}
	\DefineNamedFunc{windingNumberInThePuncturedSpace}{\gamma}{w(\gamma)}{w\left( \frac{\gamma}{\| \gamma \|}\right)}	
	\\
	\DeclareFunc{windingNumberRelativeToAVectorField}{\prod_{V \in \VF(2)} \TYPE{RegularLoop}(V) \to \Int}
	\DefineNamedFunc{windingNumberRelativeToAVectorField}{\gamma}{w_V(\gamma)}{w(\gamma V)} 
	\\
	\Theorem{HomotopyPreservesVectorFieldWindingNumber}{
		\forall V \in \VF(2) \. 
		\forall \gamma,\gamma' \.
		\gamma \sim \gamma' \Imply w_V(\gamma) = w_V(\gamma')
	}
	\Say{H}{\bd \TYPE{Homotopic}(\gamma,\gamma')}{\TYPE{Homotopy}(I,\mathcal{R}_V,\gamma,\gamma)}
	\Say{[1]}{\THM{HomotopyPreservedByC}(H)}{\TYPE{Homotopy}(I,\Reals^2 \setminus \{0\},\gamma V,\gamma' V, H V)}
	\Say{[2]}{\THM{HomotopyPreservesWindingNumber}[1]}{ w( \gamma V) = w( \gamma' V )  }
	\Conclude{[*]}{\bd^{-1} w_V [2]}{w_V(\gamma) = w_V(\gamma')}
	\EndProof
}
\Page{
	\Theorem{IndexIsWellDefined}
	{
		\forall V \in \VF(2) \.
		\forall p : \TYPE{IsolatedSingularPoint} \.
		\exists \varepsilon \in \Reals_{++} :
		\exists n \in \Int \. \NewLine \. 
		\forall t \in (0,\varepsilon] \.
		w_V(p + ts) = n
	}
	\Say{\Big(U,[1]\Big)}{\bd \TYPE{SingularPoint}(2,V,p)}
	{
		\sum U \in \U(p) \. 
		U \setminus \{p\} \subset \mathcal{R}_V
	}
	\Say{\Big( \varepsilon,[2] \Big)}{\THM{OpenInMetricSpace}(U,p)}
	{
		\sum \varepsilon \in \Reals_{++} \. 
		\mathbb{B}^2(p,\varepsilon) \subset U
	}
	\Conclude{[*]}{
		\THM{HomotopyPreservesVectorFieldsWindingNumber}[1][2]
	}
	{
		\forall t \in (0,\varepsilon] \.
		w_V(p + ts) = w_V(p + \varepsilon s) 
	}
	\EndProof
	\\
	\DeclareFunc{indexOfASingularPoint}{\prod_{V \in \VF(2)} \TYPE{IsolatedSingularPoint}(V) \to \Int}
	\DefineNamedFunc{indexOfASingularPoint}{ p }{ \ind_V p }
	{
		w_V(p + \varepsilon s) \quad \where \quad \varepsilon = \THM{IndexIsWellDefined}(V,p)
	}
	\\
	\Theorem{IndexOfManyPoints}
	{
		\forall V \in \VF(2) \.
		\forall \gamma : \TYPE{RegularLoop}(V) \.
		\forall U \in \T(\Reals^2) \.
		\forall [0] : \boundary U =  \im \gamma \. \NewLine \.
		\forall n \in \Nat \.
		\forall p : [1,\ldots,n] \ToSurj \mathcal{S}_n \cap U \.
		w_V(\gamma) = \sum^n_{i=1} \ind_V p
	}
	& \text{ 
		Note, that $\gamma$ is hamotopic to a flower with $n$ petels, each containg one singular point. }\\
	&\text{	Use some some complex analysis and compute the winding number as complex path integral. 
	} \\
	\EndProof
}
\newpage
\subsection{Degree Theory Of The Torus}
\Page{
	\DeclareFunc{torusDegree}
	{
		\Big(\mathbb{T}^2 \Arrow{\TOP} \mathbb{T}^2 \Big) \to \Int^{2 \times 2}
	}
	\DefineNamedType{f}{torusDegree}{  D(f)  }
	{
		\Lambda i,j \in\{1,2\} \.  \deg \iota_i f \pi_j
	}
	\\
	\Theorem{HomotopyPreservesTorusDegree}
	{
		\forall f,f' : \Big( \mathbb{T}^2 \Arrow{\TOP} \mathbb{T}^2 \Big) \.
		f \sim f' \iff D(f) = D(f')
	}
	& (\Imply) \text{  If $H$ is a homotopy of tori maps it evently constraints to the homotopy of circle maps 
		$ \iota_i f \pi_j $ and $\iota_i f' \pi_j$ } \\
	& \text{ By degree theory of the circle $D(f) = D(f')$.  } \\
	& (\Leftarrow) \text{ Use some lifting theory of $\mathbb{T}^2 = \frac{\Reals^2}{\Int^2}$  } \\
	& \text{ Note that $D(f) = D(f')$ imply that $f_* = f'_*$.} \\
	& \text{ Assume that $f[0] = f'[0]$ and define $g = f - f'$ with $D(g) = 0$ } \\
	& \text{There is a lift $\tilde{g} : \mathbb{T} \to \Reals^2$ with $g = \tilde g \pi $ } \\
	& \text{Then, there is a homotopy $H(t,x) = \pi( t \tilde g(x) )$ between $g$ and $0$} \\
	& \text{Hence $f \sim f'$ by topological groop theory}  \\
	\EndProof
	\\
	\Theorem{ToricDegreeComposition}
	{
		\forall f,g :  \End_{\TOP}(\mathbb{T}^2)
		D(fg) = D(g)D(f)
	}
	&\text{ use properties of homomrphisms $f_*,g_*$}\\
	\EndProof
	\\
	\Theorem{ToricDegreeisSurjective}
	{
		\TYPE{Surjective}(\mathbb{T}^2,\Int^{2\times 2}D)
	}
	&\text{try $f(u,v) = \binom{u^n v^m}{u^k v^l}$} \\
	\EndProof
	\\
	\Theorem{ToricHomeeIfInvertibleDegee}
	{
		\forall f \in  \End_{\TOP}(\mathbb{T}^2)
		[f] \in \Aut_{\HTOP}(\mathbb{T}^2)
		\iff
		\TYPE{Invertibe}(\Int^{2 \times 2},D(f))
	}
	&\text{ use properties of homomrphisms $f_*,g_*$}\\
	\EndProof
}
\newpage
\subsection{Seifert-van-Kampen Theorem}
\Page{
	\DeclareType{\SVKD}
	{
		?\left(\sum X : \TYPE{PathConnected} \And  \. \Big(\T(X) \And \TYPE{PathConnected} \Big)^2 \right)
	}
	\DefineType{(X,U,V)}{\SVKD}{X = U \cup V \And \TYPE{PathConnected} \And \TYPE{NonEmpty}(U \cap V)}
	\\
	\DeclareFunc{mapOfSeifertVanKampen}
	{
		\prod (X,U,V) : \SVKD \.  \pi(U) \sqcup_{\GRP} \pi(V) \Arrow{\GRP} \pi(X) 
	}
	\DefineNamedFunc{mapOfSeifertVanKampen}{\prod^n_{i=1} [\alpha_i]_U [\beta_i]_V}
	{\Phi\left( \prod^n_{i=1} [\alpha_i]_U [\beta_i]_V \right)}
	{
		\left[ \prod^n_{i=1} \alpha_i \beta_i \right]_X
	}
	\\
	\DeclareFunc{subgroupOfSeifertVanKampen}
	{ 
		\prod (X,U,V) : \SVKD \. \NewLine \.\TYPE{Subgroup}\Big( \pi(U) \sqcup_\GRP \pi(V) \Big)
	}
	\DefineNamedFunc{subgroupOfSeifertVanKampen}{}
	{ \bar C(X,U,V)}
	{
		N\Bigg(\Big\langle  [\gamma]_U [\gamma]_V^{-1}  \Big|  [\gamma]_{U \cap V} \in \pi(U \cap V) \Big\rangle \Bigg)
	}
	\\
	\Theorem{SeifertVanKampenLemma1}{ 
		\forall (X,U,V) : \SVKD \. 
		\NewLine \.
		\TYPE{Surjective}\Big( \pi(U) \sqcup_{\GRP} \pi(V), \pi(X),\Phi_{X,U,V}\Big) 
	}
	\SayIn{p}{\bd \TYPE{NonEmpty}(U \cap V)}{U \cap V}
	\AssumeIn{\gamma}{\Omega_X(p)}
	\Say{\lambda}{\THM{LebesgueNumberLemma}\Big(I, \big(\gamma^{-1}(U),\gamma^{-1}(V) \big) \Big) \bd \SVKD(X,U,V)}
	{
		\NewLine :
		\TYPE{LebesgueNumber}\Big(I, \big(\gamma^{-1}(U),\gamma^{-1}(V) \big) \Big) 
	}
	\Say{\Big(n,[1]\Big)}{\bd \TYPE{ReductioInfima}(\Reals,\lambda)}
	{
		\sum_{n=1}^\infty \frac{1}{n} < \lambda	
	}
	\Say{\alpha}{\Lambda k \in [1,\ldots,n] \. \Lambda t \in I \. \gamma\left( \frac{k-1}{n} + \frac{t}{n}\right)}
	{
		[1,\ldots,n] \to (I \Arrow{\TOP} X)
	}
	\Say{[2]}{\ByConstr \alpha \bd \TYPE{LebesgueNumber}\Big(I, \big(\gamma^{-1}(U),\gamma^{-1}(V)\big) \Big) }
	{
		\forall k \in [1,\ldots,n] \. \im \alpha_k \subset U | \im \alpha_k \subset V
	}
	\Say{W}{\Lambda k \in [1,\ldots,n] \. 
		\If \im \alpha_k \subset U \cap V 
		\Then U \cap V 
		\Else \If \im \alpha_k \subset U 
		\Then U
		\Else V
	}{ \NewLine : [1,\ldots,n] \to \{ U,V,U \cap V \}  }
	\AssumeIn{k}{[1,\ldots,n]}
	\Say{[3]}{\ByConstr W [2](k)}{\im \alpha_k \subset W_k}
	\Conclude{h_k}{\bd \SVKD(X,U,V) \bd \TYPE{PathConnected}(W_k,p,\alpha_i(1))[3] }
	{ 
		\Omega_{W_k}(p,\alpha_i(1) )	
	}
	\Derive{h}{\Intro\Act{\prod}}{\prod^n_{k=1} \Omega_{W_k}(p,\alpha_k(1))}
	\SayIn{h_0}{\Lambda t \in I \. p}{\Omega_{U \cap V}(p)}
	\Say{\beta}{\Lambda i \in [1,\ldots,n] \. h_{k-1}\alpha_k h_k^{-1}}
	{
		[1,\ldots,n] \to \Omega_X(p)
	}
	\Say{[3]}{\ByConstr \beta \bd \Pi(X)}{ [\gamma]_X = \prod^n_{i=1} [\beta_i]_X  }
	\Say{W'}{\Lambda k \in [1,\ldots,n] \. 
		\If \im \alpha_k \subset U 
		\Then U
		\Else V
	}{  : [1,\ldots,n] \to \{ U,V \}  }
	\Say{\omega}{\prod^n_{i=1} [\beta_i]_{W'_i}}{\pi(U) \sqcup_{\GRP} \pi(V)}
}\Page{
	\Conclude{[\gamma.*]}{\bd \Phi [3]}{ \Phi(\omega) = [\gamma]_X}
	\DeriveConclude{[*]}{\bd^{-1} \TYPE{Surjective}}
	{
		\TYPE{Surjective}\Big( \pi(U) \sqcup_{\GRP} \pi(V), \pi(X) \Big)
	}
	\EndProof
	\\
	\Theorem{SeifertVanKampenLemma2}
	{
		\forall (X,U,V) : \SVKD \. \bar C \subset \ker \Phi
	}
	\AssumeIn{[\gamma]_{U \cap V}}{\pi(U \cap V)}
	\Conclude{[\gamma.*]}{\bd \Phi \bd \TYPE{Inverse}}
	{ \Phi\left( [\gamma]_U [\gamma]^{-1}_V \right)  =  [\gamma]_X [\gamma]^{-1}_X = e   }
	\DeriveConclude{[*]}{\bd \bar C}{\bar C \subset \ker \Phi}
	\EndProof
	\\
	\Theorem{SeifertVanKampenLemma3}{ 
		\forall (X,U,V) : \SVKD \. 
		\NewLine \.
		\ker \Phi  \subset \bar C  
	}
	\SayIn{p}{\bd \TYPE{NonEmpty}(U \cap V)}{U \cap V}
	\AssumeIn{x}{\ker \Phi}
	\Say{ \Big( n, \alpha, \beta, [1] \Big)  }{
		\bd \pi(U) \sqcup_{\GRP} \pi(V)
	}
	{
		\sum^\infty_{n=1} 
		\sum \alpha : n \to \Omega_U(p) \. 
		\sum \beta : n \to \Omega_V(p) \. 
		x = \prod^n_{i=1} [\alpha_i]_V [\beta_i]_U
	}
	\Say{[2]}{\bd \ker \Phi [1]}
	{
		\prod^n_{i=1} \alpha_i \beta_i \approx_{X,p} p
	}
	\Say{H}{\bd \TYPE{RelativeHomotopic}[2]}{\TYPE{RelativeHomotopy}\left(I,X,p, \prod^n_{i=1} \alpha_i \beta_i  ,p\right)}
	\Say{\lambda}{\THM{LebesgueNumberLemma}\Big(I \times I , \big(H^{-1}(U),H^{-1}(V) \big) \Big) \bd \SVKD(X,U,V)}
	{
		\NewLine :
		\TYPE{LebesgueNumber}\Big(I\times I, \big(H^{-1}(U),\H^{-1}(V) \big) \Big) 
	}
	\Say{\Big(m,[1]\Big)}{\bd \TYPE{ReductioInfima}(\Reals,\lambda)}
	{
		\sum_{m=1}^\infty \frac{1}{m} < \lambda	
	}
	\Say{S}{
		\Lambda i,j \in [1,\ldots,m] \. 
		\left[ \frac{i-1}{m}, \frac{i}{m} \right] \times 
		\left[ \frac{j-1}{m}, \frac{j}{m} \right]
	}
	{
		[1,\ldots,m]^2 \to  ?[0,1]^2
	}
	\Say{[3]}{\ByConstr S \bd \TYPE{LebesgueNumber}\Big(I \times I, \big(H^{-1}(U),H^{-1}(V),\lambda \Big) }
	{
		\forall i,j \in [1,\ldots,m] \. H(S_{i,j}) \subset U | H(S_{i,j}) \subset V
	}
	\Say{v}{\Lambda i,j \in [1,\ldots,m] \. H\left( \frac{i}{m}, \frac{j}{m} \right)}
	{
		[1,\ldots,m]^2 \to X
	}
	\Say{\xi}{
		\Lambda i,j \in [1,\ldots,m] \. 
		\Lambda t \in I \.  
		H\left( \frac{i-1}{m} + \frac{t}{m}, \frac{j}{m} \right)
	}
	{
		[1,\ldots,m]^2 \to I \to X
	}
	\Say{\zeta}{
		\Lambda i,j \in [1,\ldots,m] \. 
		\Lambda t \in [1,\ldots,m] \.
		H\left( \frac{i}{m}, \frac{j-1}{m} + \frac{t}{m} \right)}
	{
		[1,\ldots,m]^2 \to I \to X 
	}
	\Say{W}{\Lambda i,j \in [1,\ldots,m] \. 
		\If H(S_{i,j}) \subset U \cap W  
		\Then U \cap W 
		\Else \If  H(S_{i,j}) \subset U 
		\Then U
		\Else V
	}{  \NewLine : [1,\ldots,m]^2 \to \{ U,V,U \cap V \}  }
	\Say{[4]}{\bd H \ByConstr \xi}{ \prod^n_{i=1} \alpha_i \beta_i = \prod^m_{i=1} \xi_{1,i}}
}\Page{
	\AssumeIn{i,j}{[1,\ldots,n]}
	\Say{[5]}{\ByConstr W [3](i,j)}{ H(S_{i,j})\subset W_{i,j}}
	\Conclude{h_k}{\bd \SVKD(X,U,V) \bd \TYPE{PathConnected}(W_{i,j},p,v_{i,j})[5] }
	{ 
		\Omega_{W_{i,j}}(p,v_{i,j} )	
	}
	\Derive{h}{\Intro\Act{\prod}}{\prod^m_{i,j=1} \Omega_{W_{i,j}}(p,v_{i,j}(1))}
	\SayIn{h_0}{\Lambda j \in [0,\ldots,m] \. \Lambda t \in I \. p}{ [0,\ldots,m]\Omega_{U \cap V}(p)}
	\Say{\mu}{\Lambda i \in [1,\ldots,n] \. h_{i-1,j}\xi_{i,j} h_{i,j}^{-1}}
	{
		[1,\ldots,m]^2 \to \Omega_X(p)
	}
	\Say{\nu}{\Lambda i \in [1,\ldots,n] \. h_{i-1,j}\zeta_{i,j} h_{i,j}^{-1}}
	{
		[1,\ldots,m]^2 \to \Omega_X(p)
	}
	\Say{W'}{\Lambda i,j \in [1,\ldots,m]^2 \. 
		\If  H(S_{i,j}) \subset U 
		\Then U
		\Else V
	}{  : [1,\ldots,m]^2 \to \{ U,V \}  }
	\Say{[5]}{[1][4]\ByConstr \mu}
	{
		x = \prod^m_{i=1} [\mu_{1,i}]_{W'_{1,i} }
	}
	\Assume{\gamma}{\Omega_{U \cap V}(p)}
	\Conclude{[\gamma.*]}{
		\bd \TYPE{Inverse}(\pi(U))
		\bd \GRP\Big( \pi(U) \sqcup_{\GRP} \pi(V) \Big)
		\bd \bar C
	}
	{
		\NewLine : 
		\Big[   [\gamma]_V \Big]_{\bar C} = 
		\bigg[ \Big( [\gamma]_U [\gamma]_U^{-1} \Big)  [\gamma]_V \bigg]_{\bar C} = 
		\bigg[  [\gamma]_U \Big( [\gamma]_U^{-1} \Big)  [\gamma]_V \big) \bigg]_{\bar C} = 
		\Big[   [\gamma]_U \Big]_{\bar C} 
	}
	\Derive{[6]}{\Intro(\forall)}
	{
		\forall \gamma \in \Omega_{U \cap V}(p) \. 
		\Big[ [\gamma]_U \Big]_{\bar C } =
		\Big[ [\gamma]_V \Big]_{\bar C }
	}
	\AssumeIn{k}{[1,\ldots,m-1]}
	\Assume{[7]}{ [x]_{\bar C} =  \left[\prod^n_{i=1} [\mu_{k,i}]_{W'_{k,i}} \right]_{\bar C} }
	\AssumeIn{i}{[1,\ldots,m-1]}
	\Say{[8]}{\THM{SquareLemma}\ByConstr \xi \ByConstr \zeta}
	{
		\xi_{k,i+1} \approx_{W_{i,j}} \zeta_{k+1,i} \xi_{k+1,i+1} \zeta^{-1}_{k+1,i+1}
	}
	\Conclude{[i.*]}{[8] \ByConstr \mu \ByConstr \nu}
	{
		\mu_{k,i+1} \approx_{W_{i,j}} \nu_{k+1,i} \mu_{k,i+1} \nu^{-1}_{k+1,i+1}	
	}
	\Conclude{[8]}{\Intro(\forall)}
	{
		\forall i \in [0, \ldots,i] \. 
		\mu_{k,i+1} \approx_{W_{i,j}} \nu_{k+1,i} \mu_{k+1,i+1} \nu^{-1}_{k+1,i+1}	
	}
	\Conclude{[k.*]}{[7][8][6] \bd \TYPE{Inverse}}
	{
		\NewLine :
		[x]_{\bar C} = 
		\left[\prod^n_{i=1} [\mu_{k,i}]_{W'_{k,i}} \right]_{\bar C} =
		\left[\prod^n_{i=0} [\mu_{k,i+1}]_{W'_{k,i+1}} \right]_{\bar C} =
		\left[\prod^n_{i=0} [\nu_{k+1,i}]_{W'_{k+1,i}}[\mu_{k+1,i+1}]_{W'_{k+1,i+1}} [\nu_{k+1,i+1}]_{W'_{k+1,i+1}} \right]_{\bar C} =
		\NewLine = 
		\left[\prod^n_{i=1} [\mu_{k+1,i}]_{W'_{k+1,i}}  \right]_{\bar C} 	
	}
	\DeriveConclude{[7]}{\bd \TYPE{Primitive}[1,\ldots, m]}
	{
		\forall k \in [1,\ldots,m] \. 
		[x]_{\bar C} = 
		\left[\prod^n_{i=1} [\mu_{k,i}]_{W'_{k,i}}  \right]_{\bar C} 	
	}
	\Say{[8]}{[7](m)}
	{
		[x]_{\bar C} = 
		\left[\prod^n_{i=1} [\mu_{m,i}]_{W'_{m,i}}  \right]_{\bar C} 		
	}
	\Say{[9]}{[8] \ByConstr \mu \bd H}{[x]_{\bar C} = e}
	\Conclude{[x.*]}{\bd \TYPE{Coset}[9]}{x \in \bar C}
	\DeriveConclude{[*]}{\bd^{-1} \TYPE{Subset}}{\ker \Phi \subset \bar C}
	\EndProof
}\Page{
	\Theorem{SeifertVanKampenTheorem}
	{
		\forall (X,U,V) : \SVKD \. 
		\pi(X) \cong_{\GRP} \frac{\pi(U) \sqcup_{\GRP} \pi(V)}{\bar C}
	}
	\Say{[1]}{\THM{SeifertVanKampenLemma2}}
	{
		\hat C \subset \ker \Phi
	}
	\Say{[2]}{\THM{SeifertVanKampenLemma3}}
	{
		\ker \Phi \subset \hat C
	}
	\Say{[3]}{\bd \TYPE{SetEq}[1][2]}{\ker \Phi \subset \hat C}
	\Say{[4]}{\THM{SeifertVanKampenLemma4}}{\TYPE{Surjective}\Big( \pi(U) \sqcup_{\GRP} \pi(V), \pi(X),\Phi \Big)}
	\Conclude{[*]}{\THM{IsomorphismTHM}[3][4]}
	{
		\pi(X) \cong_{\GRP} \frac{\pi(U) \sqcup_{\GRP} \pi(V)}{\bar C}
	}
	\EndProof
	\\
	\Theorem{SeifertVanKampenTheorem}
	{
		\forall (X,U,V) : \SVKD \. 
		\pi(X) \cong_{\GRP} \frac{\pi(U) \sqcup_{\GRP} \pi(V)}{\bar C}
	}
	\Say{[1]}{\THM{SeifertVanKampenLemma2}}
	{
		\hat C \subset \ker \Phi
	}
	\Say{[2]}{\THM{SeifertVanKampenLemma3}}
	{
		\ker \Phi \subset \hat C
	}
	\Say{[3]}{\bd \TYPE{SetEq}[1][2]}{\ker \Phi \subset \hat C}
	\Say{[4]}{\THM{SeifertVanKampenLemma4}}{\TYPE{Surjective}\Big( \pi(U) \sqcup_{\GRP} \pi(V), \pi(X),\Phi \Big)}
	\Conclude{[*]}{\THM{IsomorphismTHM}[3][4]}
	{
		\pi(X) \cong_{\GRP} \frac{\pi(U) \sqcup_{\GRP} \pi(V)}{\bar C}
	}
	\EndProof
	\\
	\Theorem{SpecialSeifertVanKampenTheorem1}
	{
		\forall (X,U,V) : \SVKD \. \NewLine \.  
		\TYPE{SimplyConnecte}(U \cap V) \Imply
		\pi(X) \cong_{\GRP} \pi(U) \sqcup_{\GRP} \pi(V)
	}
	\NoProof
	\\
	\Theorem{SpecialSeifertVanKampenTheorem2}
	{
		\forall (X,U,V) : \SVKD \. \NewLine \.  
		\TYPE{SimplyConnecte}( V) \Imply
		\pi(X) \cong_{\GRP} \frac{\pi(U)}{ N\Big(\iota_{U*} \pi(U \cap V) \Big)   } 
	}
	\NoProof
}
\newpage
\subsection{Applications to Geometric Topology}
\Page{
	\DeclareFunc{wedgeSum}
	{
		\prod_{\I \in \SET} (I \to \TOP_*) \to \TOP_*
	}
	\DefineNamedFunc{wedgeSum}{X}{\bigvee_{i \in \emptyset} X_i }{ \{p\}   } 
	\DefineNamedFunc{wedgeSum}{X}{\bigvee_{i \in \I} X_i}
	{
		\left(\frac{\bigsqcup_{i \in \I} X_i}
		{
			\Big\{ ( i, \mathrm{pt} \; X_i) \Big| i \in \I  \Big\}
		},
		[\mathrm{pt} \; X] 
		\right)
	}
	\\
	\DeclareType{NondegenarateBasepoint}
	{
		?\TOP^*
	}
	\DefineType{(X,p)}{NondegenerateBasepoint}{ 
		\exists U \in \U(p) \. \TYPE{StrongDeformationRetract}\Big(U,\{p\}\Big)
	}
	\\
	\Theorem{WedgeProductOpenUnion}
	{
		\forall \I \in \Set \. 
		\forall X : \I \to \TOP^* \. 
		\NewLine \.
		\forall U : \prod_{i \in \I} \U\Big(\mathrm{pt}(X_i)\Big) \.
		\bigcup_{i \in \I} \iota_i q (U_i) \in \T\left( \bigvee_{i \in I} X_i \right)
	}
	\Say{q}{\bd \FUNC{wedgeSum}(X) \bd \FUNC{quotientSpace}}
	{\TYPE{QuotienMap}\left( \bigsqcup_{i \in \I} X_i,\bigvee_{i \in \I}X_i \right)}
	\Say{\J}{\Lambda i 'in \I \. \I \setminus \{i\} }{\I \to ?\I}
	\Say{[1]}{\bd \FUNC{wedgeSum}(U) \bd U \ByConstr^{-1} q}
	{
		\forall i \in \I \. 
		q^{-1}(\iota_i q U_i) = \iota_i U_i \cup \bigcup_{j \in \J_i} \Big\{ \Big( j, \mathrm{pt}(X_j) \Big) \Big\}  
	}
	\Say{[2]}{\THM{PreimageUnion}(q,U)[1] \bd \TYPE{Union}}
	{
		= \NewLine = 
		q^{-1}\left( \bigcup_{i \in \I} \iota_i q  U_i \right) =
		\bigcup_{i \in \I}  q^{-1}(\iota_i q U_i) = 
		\bigcup_{i \in \I} \left(   \iota_i U_i \cup \bigcup_{j \in \J_i} \Big\{ \Big( j, \mathrm{pt}(X_j) \Big) \Big\} \right)
		= 
		\bigcup_{i \in \I} \iota_i U_i
	}
	\Conclude{[*]}{\bd \TYPE{QuotientMap}(q) \bd \TOP\left( \bigsqcup_{i \in \I} X_i \right)[2] }
	{
		\bigcup_{i \in \I} \iota_i q U_i \in \T\left( \bigwedge_{i \in \I} X_i \right)
	}
	\EndProof
}\Page{
	\Theorem{NondegenerateWedgeSum}
	{
		\forall \I \in \Set  \.
		\forall X : \I \to \TYPE{NondegenerateBasepoint} \. \NewLine \. 
		\TYPE{NondegenerateBasePoint}\left(\bigvee_{i \in \I} X_i\right) 
	}
	\Say{\Big(U,[1]\Big)}
	{
		\bd \TYPE{NondegenerateBasepoint}(X)
	}
	{
		\prod_{i \in \I} \sum U_i \in \U\Big(\mathrm{pt}(X_i)\Big) \. \NewLine \.  
		\TYPE{StrongDeformationRetract}\bigg( U_i,\Big\{ \mathrm{pt}(X_i) \Big\}  \bigg)   
	}
	\SayIn{\star}{\mathrm{pt}\left(\bigvee_{i \in \I} X_i \right)}{\bigvee_{i \in \I} X_i}
	\SayIn{V}{\bigvee_{i \in \I} \Big(U_i,\mathrm{pt}(X_i)\Big)}{\U(\star)}    
	\AssumeIn{v}{V}
	\Say{[2]}{\ByConstr V}{ v = \star | \exists i \in \I \. v \in U_i \setminus \Big\{ \mathrm{pt}(X_i) \Big\}  }
	\Assume{[3]}{v = \star}
	\Conclude{G(\bullet,v)}{\Lambda t \in I \. \star }
	{
		I \Arrow{\TOP} V
	}
	\DeriveConclude{[3]}{\Intro(\Imply)}{v = \star \Imply (I \Arrow{\TOP} V) }
	\AssumeIn{i}{\I}
	\Assume{[4]}{v \in U_i \setminus \Big\{ \mathrm{pt}(X_i) \Big\}  }
	\Say{H}{\bd \TYPE{StrongDeformationRetract}\bigg(U_i, \Big\{ \mathrm{pt}(X_i) \Big\} \bigg) }
	{
		\TYPE{RelativeHomotopy}\bigg(U_i, \Big\{ \mathrm{pt}(X_i)  \Big\}, \id, \mathrm{pt}(X_i) \bigg) 
	}
	\Conclude{G(\bullet,v)}{H(\bullet,v)\iota_i}{ I \Arrow{\TOP} V   }
	\DeriveConclude{[4]}{\Intro(\Imply)}{v \neq \star \Imply (I \Arrow{\TOP} V) }
	\Conclude{H(\bullet,v)}{\Elim(|)[2,3,4]}{I \Arrow{\TOP} V}
	\Derive{H}{\Intro(\to)}{ I \Arrow{\TOP} V \Arrow{\TOP} V  }
	\Say{[2]}{\bd \TYPE{QuotientSpace} \ByConstr H}
	{
		\TYPE{RelatriveHomotopy}\Big( V, \{\star\}, \id, \star \Big)
	}
	\Conclude{[*]}{\bd^{-1} \TYPE{NondegenerateBasePoint}}
	{
		\TYPE{NondegenerateBasePoint}\left(\bigvee_{i \in \I} X_i\right) 
	}
	\EndProof
}
\Page{
	\Theorem{FundamentalGroupOfWedgeSum}
	{
		\forall \I  : \TYPE{Finite}  \.
		\forall X : \I \to \TYPE{NondegenerateBasepoint} \. \NewLine \. 
		\pi \left(\bigvee_{i \in \I} X_i\right) = 
		\bigsqcup_{i \in \I}  \pi(X_i)   
	}
	\Assume{X,Y}{\TYPE{NondegenerateBasepoint}}
	\Say{\Big(U,[1]\Big)}
	{
		\bd \TYPE{NondegenerateBasepoint}(X)
	}
	{
		\sum U \in \U\Big(\mathrm{pt}(X)\Big) \. \NewLine \.  
		\TYPE{StrongDeformationRetract}\bigg( U,\Big\{ \mathrm{pt}(X) \Big\}  \bigg)   
	}
	\Say{\Big(V,[2]\Big)}
	{
		\bd \TYPE{NondegenerateBasepoint}(Y)
	}
	{
		\prod_{i \in \I} \sum U_i \in \U\Big(\mathrm{pt}(X_i)\Big) \. \NewLine \.  
		\TYPE{StrongDeformationRetract}\bigg( U_i,\Big\{ \mathrm{pt}(X_i) \Big\}  \bigg)   
	}
	\SayIn{U'}{Y \cup U}{\T(X \wedge Y) \And \TYPE{Connected}} 
	\SayIn{V'}{Y \cup U}{\T(X \wedge Y) \And \TYPE{Connected}} 
	\Say{[3]}{\bd^{-1} \TYPE{union}\ByConstr U' \ByConstr V'}{ U' \cup V' = X}
	\Say{[4]}{\bd^{-1}\TYPE{intersection}\ByConstr U' \ByConstr V'}{ U' \cap V' = U \cup V}
	\Say{[5]}{[1][2][4]}{\TYPE{Connected}(U' \cap V')}
	\Say{[6]}{\bd^{-1} \SVKD [5][3]}{\SVKD(X\vee Y,U',V')}
	\Say{[7]}{\ByConstr U' [1]}{\pi(U') \cong_\GRP \pi(Y)}
	\Say{[8]}{\ByConstr V' [2]}{\pi(V') \cong_\GRP \pi(X)}
	\Say{[9]}{[1][2][4]\bd^{-1} \bd^{-1} \TYPE{SimplyConnected}}{\TYPE{SimplyConnected}(U' \cap V' )}
	\Conclude{\big[(X,Y).*]}{\THM{SpecialSeifertVanKampenTHM1}[6,7,8,9]}
	{
		\pi(X \vee Y) = \pi(X) \sqcup \pi(Y)
	}
	\DeriveConclude{[*]}{\bd \Nat \THM{NonDegenerateWedgeSum}}
	{
		\NewLine : 
		\forall \I  : \TYPE{Finite}  \.
		\forall X : \I \to \TYPE{NondegenerateBasepoint} \. 
		\pi \left(\bigvee_{i \in \I} X_i\right) = 
		\bigsqcup_{i \in \I}  \pi(X_i)   
	}
	\EndProof
	\\
	\Theorem{FundamentalGroupOfBuquetOfCircles}
	{
		\forall n \in \Nat \. 
		\pi\Big( \Sphere^{1(\vee \! n)} \Big) = F_{\GRP}[1,\ldots,n] 
	}
	\NoProof
	\\
	\DeclareType{CWGraph}{?\TYPE{FiniteCommplex} \And \TYPE{Connected} }
	\DefineType{C}{CWGraph}{ \dim C = 1  }
	\\
	\DeclareFunc{cwGraphReprsentation}{\TYPE{CWgraph} \ToBij \sum X \in \TOP \. ?X \times \TYPE{Multiset}(X \times X)}
	\DefineNamedFunc{cwGraphRepresentation}{ (X,\E,\varphi)}{(X,V,E)}{(X,\E_0,\varphi_1(\E_1))}
	\\
	\DeclareType{SelfLoop}{\prod (X,V,E) : \TYPE{CWGraph} \. ?E}
	\DefineType{(x,y)}{SelfLoop}{x = y}
	\\
	\DeclareType{MultipleEdge}{\prod (X,V,E) : \TYPE{CWGraph} \. ?E}
	\DefineType{(e}{MultipleEdge}{|e|_E > 1}
}\Page{
	\DeclareType{SimpleGraph}{ ? \TYPE{CWGraph} }
	\DefineType{G}{SimpleGraph}{\TYPE{SelfLoop}(G) = \TYPE{MultipleEdge}(G) = \emptyset }
	\\
	\DeclareType{EdgePath}{\prod G : \TYPE{CWGraph} \ ? \sum^\infty_{n=1} [1,\ldots,n] \to E_G}
	\DefineType{p}{EdgePath}{\forall i \in [1,\ldots,n-1] \. p_{i,2} = p_{i+1},1}
	\\
	\DeclareType{EdgePath}{
		\prod G : \TYPE{CWGraph} \.  
		\sum^\infty_{n=0} 
		\Big([1,\ldots,n+1] \to V_G) \times
		\Big([1,\ldots,n] \to E_G \Big) 
	}
	\DefineType{(n,v,e)}{EdgePath}{\forall i \in [1,\ldots,n] \. e_{i,1} = v_i \And e_{i,2} = v_{i+1} }
	\\
	\DeclareFunc{length}{\prod G : \TYPE{CWGraph} \. \TYPE{EdgePath}(G) \to \Int_+}
	\DefineNamedFunc{length}{n,v,e}{\big|(n,v,e)\big|}{n}
	\\
	\DeclareFunc{verticePath}{\prod G : \TYPE{CWGraph} \. \prod \gamma :  \TYPE{EdgePath}(G) \. \Big[1,\ldots,|\gamma|+1\Big] \to V_G}
	\DefineNamedFunc{verticePath}{i}{v_\gamma^i}{\gamma_{2,i}} 
	\\
	\DeclareFunc{edgePath}{\prod G : \TYPE{CWGraph} \. \prod \gamma :  \TYPE{EdgePath}(G) \. \Big[1,\ldots,|\gamma|\Big] \to E_G}
	\DefineNamedFunc{edgePath}{i}{e_\gamma^i}{\gamma_{3,i}}
	\\
	\DeclareFunc{initialVertex}{\prod G : \TYPE{CWGraph} \. \TYPE{EdgePath}(G) \to V_G}
	\DefineNamedFunc{initialVertex}{\gamma}{\mathrm{init}\;\gamma}{v_\gamma^1}
	\\
	\DeclareFunc{terminalVertex}{\prod G : \TYPE{CWGraph} \. \TYPE{EdgePath}(G) \to V_G}
	\DefineNamedFunc{terminalVertex}{\gamma}{\mathrm{init}\;\gamma}{v_\gamma^{|\gamma| + 1}}
	\\
	\DeclareType{Closed}{\prod G : \TYPE{CWGraph} \. ?\TYPE{EdgePath}(G)}
	\DefineType{\gamma}{Closed}{\mathrm{init}\; \gamma = \mathrm{tetm} \; \gamma}
	\\
	\DeclareFunc{otherIndices}{\prod G : \TYPE{CWGraph} \. \prod \gamma : \TYPE{EdgePath}(G) \. 
		\big[1,\ldots |\gamma|\big] \to ??\big[1,\ldots,|\gamma| + 1\big] }
	\DefineNamedFunc{otherIndices}{1}{{\hat I }^1_\gamma}{\big[2,\ldots,|\gamma|\big]}
	\DefineNamedFunc{otherIndices}{i}{{\hat I}^i_\gamma}{ \big[1, \ldots, |\gamma| +1\big] \setminus \{i\}  }
	\\
	\DeclareType{Simple}{\prod G : \TYPE{CWGraph} \. ?G}
	\DefineType{\gamma}{Simple}{
		\forall i \in \big[1,\ldots,|\gamma|\big] \. 
		\forall j \in {\hat I}^i_\gamma \. 
		v_\gamma^i \neq v_\gamma^j
		\And
		\forall i,j \in \big[1,\ldots,|\gamma|\big] \.
		i \neq j \Imply
		e_\gamma^i \neq e_\gamma^j
	}
	\\
	\DeclareType{Cycle}{\prod G : \TYPE{CWGraph} \. \Big( \TYPE{Closed} \And \TYPE{Simple} \Big)(G)}
	\DefineType{\gamma}{Cycle}{|\gamma| \ge 1} 
	\\
	\DeclareType{CWTree}{? \TYPE{CWGraph}}
	\DefineType{T}{CWTree}{\TYPE{Cycle}(T) = \emptyset}
}\Page{
	\Theorem{CWTreeIsSimplyConnected}{ \forall (X,V,E) : \TYPE{CWTree} \. \TYPE{SimplyConnected}(X) }
	\Say{[1]}{\bd^{-1}\TYPE{SimplyConnected}}{\TYPE{SimplyConnected}(\star)}
	\AssumeIn{n}{\Nat}
	\Assume{[2]}{
		 \forall m \in [1,\ldots, n] \.
		 \forall \Gamma : \TYPE{CWTree}  \.
		 |E_\Gamma| = m 
		 \Imply
		 \TYPE{SimplyConnected}(\Gamma)
	}
	\Assume{(X,V,E)}{\TYPE{CWTree} \And \TYPE{FiniteComplex} \And \TYPE{Connected}}
	\Assume{[3]}{|V| = n + 1}
	\Assume{[4]}{\forall v \in V \. \deg v > 1}
	\Say{[5]}{\bd \deg [4]}
	{
		\forall n \in \Nat \. 
		\exists \gamma : \TYPE{EdgePath}(V,E) :
		|\gamma| = n
	}
	\Say{[6]}{[3][5]}
	{
		\exists \TYPE{Cycle}(G)
	}
	\Conclude{[4.*]}{\bd \TYPE{CWTree}(X,V,E)[6]}
	{
		\bot	
	}
	\Derive{\Big(v,[4]\Big)}{\Elim(\bot)}
	{
		\sum v \in V \. \deg v = 1
	}
	\Say{\big(e,[5]\Big)}{\bd \deg v [4]}
	{
		\sum e \in E \. e_2 = v
	}
	\Say{[6]}{\bd \deg v [4] [5]}
	{
		\forall f \in E \.
		f_1 = v | f_2 = v 
		\Imply
		f = e
	}
	\SayIn{w}{e_2}{V}
	\Say{\Gamma}{\Big(X \setminus \{v\},V \setminus \{v\},E \setminus \{e\}\Big)}{\TYPE{CWTree}}
	\Say{[7]}{\ByConstr \Gamma [2][6]}
	{
		\TYPE{SimpleConncected}(\Gamma)
	}
	\Conclude{[n.*]}{\THM{FundamentalGroupOfWedgeSum}\Big((\Gamma,w),(I,w)\Big)}
	{
		\TYPE{SimplyConnected}(X)
	}
	\DeriveConclude{[*]}{\bd \Nat [1]}
	{
		\TYPE{SimplyConnected}(X)
	} 
	\EndProof
	\\
	\DeclareType{SpanningTree}{\prod (X,V,E) : \TYPE{CWGraph} \. ?\TYPE{CWTree}}
	\DefineType{(Y,V',E')}{SpanningTree}{Y \subset X \And V = V' \And E' \subset E}
}\Page{
	\Theorem{SpanningTreeExists}{\forall (X,V,E) : \TYPE{CWGraph} \. \exists \TYPE{SpanningTree}(X,V,E)}
	\Say{[1]}{\bd^{-1}\TYPE{SpanningTree}}{\TYPE{SpanningTree}\Big( (\star,\star,\emptyset), (\star,\star,\emptyset)  \Big)}
	\AssumeIn{n}{\Nat}
	\Assume{[2]}{
		\forall (X,V,E) : \TYPE{CWGraph} \. 
		|E| < n 
		\Imply
		\exists 
		\TYPE{SpanningTree}(X,V,E)
	}
	\Assume{(X,V,E)}{\TYPE{CWGraph}}
	\Assume{[3]}{|E| = n}
	\Assume{[4]}{\TYPE{CWTree}(X,V,E)}
	\Conclude{[4.*]}{\bd^{-1} \TYPE{SpanningTree} \bd \TYPE{CWTree}}
	{
		\TYPE{SpanningTree}\Big( (X,V,E), (X,V,E) \Big)
	}
	\Derive{[4]}{\Intro(\Imply)}
	{
		\TYPE{CWTree}(X,V,E) 
		\Imply
		\exists \TYPE{SpanningTree}(X,V,E)
	}
	\Assume{[5]}{\IsNot \TYPE{CWTree}(X,V,E)}
	\Say{\gamma}{\bd \TYPE{CWTree}[5]}{\TYPE{Cycle}(X,V,E)}
	\Say{[6]}{\bd \TYPE{Cycle}(\gamma)}{|\gamma| \ge 1}
	\Say{\Gamma}{\big(X,V,E \setminus \{e_\gamma^1\}\big)}{\TYPE{CWGraph}}
	\Say{[7]}{ \ByConstr \Gamma  }{ |E_\Gamma| = n - 1 }
	\Say{T}{[2][7]}{\TYPE{SpanningTree}(\Gamma)}
	\Conclude{[5.*]}{\ByConstr T \ByConstr \Gamma \bd^{-1} \TYPE{SpanningTree} }
	{
		\TYPE{SpanningTree}\Big( (X,V,E) , T \Big)
	}
	\Derive{[5]}{\Intro(\Imply}
	{  
		\IsNot \TYPE{CWTree}(X,V,E) 
		\Imply
		\exists \TYPE{SpanningTree}(X,V,E)
	}
	\Conclude{[n.*]}{\Elim(|)\LOGIC{LEM}[4][5]}{   \exists \TYPE{SpanningTree}(X,V,E)}
	\DeriveConclude{[*]}{\bd \Nat}{\exists \TYPE{SpanningTree}(X,V,E)}
	\EndProof
	\\
	\Theorem{FundamentalGroupOfAGraph}
	{
		\forall (X,V,E) : \TYPE{CWGraph} \.
		\pi(X) =  F_{\GRP}[1,\ldots, n] 
		\NewLine
		\where \quad  
		(X',V,E') = \THM{SpanningTreeExists}(X,V,E), n =  | E \setminus E' |  
	}
	\Say{[1]}{\THM{CWTreeIsSimplyConnected}(X',V,E')}{ \TYPE{SimplyConnected}(X')  }
	\Say{[2]}{\bd \TYPE{SimplyConnected}[1]}{ X' \cong_{\HTOP} \star   }
	\Say{[3]}{[2]\bd X'}{X \cong_{\HTOP} \Sphere^{1 (\vee \! n)}}
	\Conclude{[*]}{\THM{FundamentalGroupOfBuquetOfCircles}[3]}
	{
		\pi(X) = F_\GRP[1,\ldots,n]
	}
	\EndProof
}
\Page{
	\Theorem{FundamentalGroupByAttachingADisk}
	{
		\forall X : \TYPE{Connected} \.
		\forall (X',\varphi) : \TYPE{ByAttachingNCell}(X,2) \. \NewLine \. 
		\pi(X') \cong_\GRP \frac{\pi(X)}{N(\tau)} 
		\quad \where \quad \tau =   [s \varphi]_X
	}
	\Say{q}{\FUNC{quotientMap}(X',\varphi)}{\TYPE{QuotientMap}\Big( X \sqcup \mathbb{D}^2, X' ) }
	\Say{U}{q\Big( \big(2,\mathbb{D}^2 \setminus \{0\}\big) \sqcup (1,X)\Big)}{\T(X')}
	\Say{[1]}{\THM{ConnectedImage}^2(q,\ldots)\THM{ConnectedByItersection}}{\TYPE{Connected}(U)}
	\Say{V}{ q(2,\mathbb{B}^2))}{\T(X')}
	\Say{[2]}{\THM{ConnectedImage}(q,\mathbb{B}^2)}{\TYPE{Connected}(V)}
	\Say{[3]}{\ByConstr U \ByConstr V  \bd^{-1} \TYPE{Union}}{U \cup V = X'}
	\Say{[4]}{\ByConstr U \ByConstr V \bd^{-1} \TYPE{Intersect}}
	{
		U \cap V = q(2,\mathbb{B} \setminus \{0\})
	}
	\Say{[5]}{\THM{ConnectedImage}\big(q,\mathbb{B}^2\setminus \{0\}\big)[4]}
	{
		\TYPE{Connected}(U \cap V) 
	}
	\Say{[6]}{\THM{CPreservesHomotopy}[4]}{ U \cap V \cong_{\HTOP} \varphi(\Sphere^1 )   }
	\Say{[7]}{\bd \TYPE{ByAttachingNCell}(X,2,X',\varphi)\THM{FundamentalGroupIsomorphism}}
	{ \varphi \pi \Sphere^1 = \Big\langle [s \varphi]  \Big\rangle } 
	\Say{[8]}{\bd \TYPE{BuAttacjongNCell}(X,2,X',\varphi) \ByConstr V }{\TYPE{SimplyConnected}(V)}
	\Say{[9]}{\bd^{-1} \SVKD [1][2][3][5]}{\SVKD(X',U,V)}
	\Conclude{[*]}{\THM{SpecialSeifertVanKampenTHM2}(X',U,V)[7][8]}
	{
		\pi(X') = \frac{\pi(X)}{N(\tau)}
	}
	\EndProof
	\\
	\Theorem{FundamentalGroupByAttachingHigherCell}
	{
		\forall n \in \Nat \.
		\forall [0] : n > 2 \.
		\forall X : \TYPE{Connected} \. \NewLine \. 
		\forall (X',\varphi) : \TYPE{ByAttachingNCell}(X,n) \. \NewLine \. 
		\pi(X') \cong_\GRP \pi(X)
	}
	\NoProof
	\\
	\Theorem{FundamentalGroupOfCWComplex}
	{
		\forall (X,\E,\varphi) : \TYPE{Connected} \And \TYPE{FiniteCWComplex} \.
		\pi(X) = \frac{\pi(X^\skull)}{N\Big\{ [s \varphi_{2,e}]_X  \Big| e \in \E_2  \Big\}}
	}
	\NoProof
	\\
	\Theorem{FundamentalGroupByPolygonalPresentation}
	{
		\forall X : \TYPE{CompactSurface} \.
		\forall [0] : X = \mathrm{real}\langle a_1,\ldots,a_n | w \rangle  \.
		\NewLine \. 
		\pi(X) = \langle a_1,\ldots, a_n | w \rangle_\GRP 
	}
	\NoProof
	\\
	\Theorem{ClassificationOfCompacttSurfaces2}
	{
		\forall n,m \in \Nat \. \Sphere^2 \not \cong_\TOP \bigsum^m_{i=1} \mathbb{T}^2 \not \cong_\TOP \bigsum^n_{i=1} \Reals\P^2
	}
	\NoProof
}
\Page{
	\Theorem{EulerCharacteristicIsTopologicalInvariant}
	{
		\forall X,Y : \CS \. 
		\forall [0] : X \cong_{\TOP} Y \.
		\chi(X) = \chi(Y)
	}
	\NoProof
	\\
	\Theorem{OrientabilityIsTopologicalInvariant}
	{
		\forall X,Y : \CS \. 
		\forall [0] : X \cong_{\TOP} Y \.
		\TYPE{Orientable}(X) \iff \TYPE{Orientable}(Y)
	}
	\NoProof
	\\
	\DeclareFunc{genus}{\CS \to \Int_+}
	\DefineNamedFunc{genus}{\Big[\Sphere^2\Big]}{\mathrm{gen}\;\Sphere^2}{0}
	\DefineNamedFunc{genus}{\left[\bigsum^n_{i=1}\mathbb{T}^2\right]}
	{\mathrm{gen}\left[\bigsum^n_{i=1}\mathbb{T}^2\right]}{n}
	\DefineNamedFunc{genus}{\left[\bigsum^n_{i=1}\Reals \P^2\right]}
	{\mathrm{gen}\left[\bigsum^n_{i=1}\Reals\P^2\right]}{n}
}
\newpage
\section{Covering Theory}
\subsection{Covering Map}
\Page{
	\DeclareType{EvenlyCovered}
	{
		\prod X,Y \in \TOP \.
		(X \Arrow{\TOP} Y)  \to ?\T(Y)
	}
	\DefineType{U}{EvenlyCovered}
	{
		\Lambda f : X \Arrow{\TOP} Y \.
		\exists n \in \Nat :
		\exists V : [1,\ldots,n] \to \T(X) \And \TYPE{Connected} :
		f^{-1}(U) = \bigsqcup^n_{i=1} V_i 
		\And \NewLine \And
		\forall i \in [1,\ldots,n] \. \TYPE{Homeo}\Big(f_{|V_i}, U\Big)
	}
	\\
	\Theorem{EvenlyCoveredIsConnected}
	{
		\forall X,Y \in \TOP \.
		\forall f : X \Arrow{\TOP} Y  \.
		\forall U : \TYPE{EvenlyCovered}(X,Y,f) \. 
		\TYPE{Connected}(U)
	}
	\NoProof
	\\
	\DeclareType{CoveringMap}
	{
		\prod X : \TYPE{Connected} \And \TYPE{Locally}\;\TYPE{PathConnected} \. 
		\prod B \in \TOP \. \NewLine \. 
		?\Big(\TYPE{Surjective} \And \TYPE{Continuous}\Big)(X,B)
	}
	\DefineType{f}{CoveringMap}{\forall p \in B \. \exists U \in \U(p) : \TYPE{EvenlyConvered}(X,Y,f,U)}
	\\
	\Theorem{CoveringMapisLocalHomeo}
	{
		\forall f : \TYPE{CoveringMap}(X,B) \. 
		f : \TYPE{Local} \; \TYPE{Homeo}(X,B)
	}
	\AssumeIn{x}{X}
	\Say{\Big(U,[1]\Big)}{\bd \TYPE{CoveringMap}(X,B,f)\big( f(U) \big)}
	{
		\sum U : \TYPE{EvenlyCovered}(X,B,f) \. f(x) \in U
	}
	\Conclude{\Big(V,[x.*]\Big)}{\bd \TYPE{EvenlyCovered}(X,B,f)}
	{
		\sum V \in \U(x) \.  \TYPE{Homeomorphism}(V,U, f_{|V})
	}
	\Derive{[*]}{\bd^{-1} \TYPE{Local}\;\TYPE{Homeomorphism}}
	{
		\TYPE{Local}\;\TYPE{Homeomorphism}(X,B,f)
	}
	\EndProof
}\Page{
	\Theorem{CoveringMapProduct}
	{
		\forall n \in \Nat \.
		\forall X : [1,\ldots,n] \to \TYPE{Connected} \And \TYPE{Locally}\; \TYPE{PathConnected} \. \NewLine \. 
		\forall B : [1,\ldots,n] \to X \.
		\forall f : [1,\ldots,n] \to \TYPE{CoveringMap}(X_i,B_i) \.
		\prod^n_{i=1} f_i : \TYPE{CoveringMap}\left( \prod^n_{i=1} X_i, \prod^n_{i=1} B_i \right) 
	}
	\Assume{p}{\prod^n_{i=1} B_i}
	\Say{U}{\lambda i \in [1,\ldots,n] \. \bd \TYPE{CoveringMap}(f)(p_i)}{\prod^n_{i=1} \TYPE{EvenlyCovered}(X_i,B_i,f_i)}
	\SayIn{U'}{\prod^n_{i=1} U'}{\T(U')}
	\Say{\Big(\I,V,[1],[2]\Big)}{\bd \TYPE{EvenlyCovered}(X,B,f,U)}
	{
		\prod^n_{i=1} \sum_{\I_i \in \SET} \sum V_{i,j} : \TYPE{Connected} \And \T(X_i) \. \NewLine \.  
		f^{-1}_i(U_i) = \bigsqcup_{j \in \I_i} V_{i,j} \And 
		\forall j \in \I_i \. 
		\TYPE{Homeomorphism}\Big( V_{i,j},U_i,f_{i|V_{i,j}}^{|U_i} \Big)
	}
	\Say{V'}{\Lambda j : \prod^n_{i=1}\I_i \. \prod^n_{i=1} V_{i,j_i}  }
	{
		\prod^n_{i=1} \I_i \to \T\left( \prod^n_{i=1} X_i \right)
	}
	\Say{[3]}{\ByConstr U' \THM{ProductPreImage}[1] \ByConstr^{-1} V'}
	{    
		\left( \prod^n_{i=1} f_i \right)^{-1}(U') =
		\left( \prod^n_{i=1} f_i \right)^{-1}\left( \prod^n_{i=1}  U_i \right) =
		\prod^n_{i=1}  f_i^{-1} ( U_i ) =
		\prod^n_{i=1} \bigsqcup_{j \in \I_i} V_{i,j} = \NewLine = 
		\bigsqcup j \in \prod^n_{i=1} \I_j \.  V'_j 
	}
	\Say{[4]}{
		\THM{HomeoProduct}[1]
	}
	{
		\forall j \in \prod^n_{i=1} \I_i
		\TYPE{Homeomorphism}\left( V'_j,U', \left(\prod^n_{i=1} f_i \right)^{|U'}_{|V'_j} \right)
	}
	\Conclude{[p.*]}{\bd^{-1} \TYPE{EvenlyCovered}[3][4]}
	{
		\TYPE{EvenlyCovered}\left( \prod^n_{i=1} X_i, \prod^n_{i=1} B_i, \prod^n_{i=1} f_i, U' \right)
	}
	\DeriveConclude{[*]}
	{
		\bd^{-1} \TYPE{CoveringMap}
	}
	{
		\TYPE{CoveringMap}\left( \prod^n_{i=1} X_i, \prod^n_{i=1} B_i, \prod^n_{i=1} f_i \right)	
	}
	\EndProof
	\\
	\Theorem{CoveringMapHasLocalSection}
	{
		\forall f : \TYPE{CoveringMap}(X,B) \.
		\forall U : \TYPE{EvenlyCovered}(X,B,f) \. 
		\NewLine \. 
		\exists \sigma : \TYPE{LocalSection}(U,X,f)
	}
	\NoProof
} 
\Page{
	\Theorem{CoveringMapHasNumber}
	{
		\forall f : \TYPE{CoveringMap}(X,B) \.
		\exists n \in \mathsf{CARD} \. 
		\forall p \in B \. 
		\big| f^{-1}(p) \big| = n
	}
	\Say{[1]}{\bd \TYPE{Surjective}(f) \THM{ConnectedImage}}{\TYPE{Connected}(B)}
	\AssumeIn{n}{\mathsf{CARD}}
	\AssumeIn{p}{B}
	\Assume{[2]}{\big| f^{-1}(p) \big| = n}
	\Say{U}{\bd \TYPE{CoveringMap}(f)(p)}
	{ 
		\TYPE{EvenlyCovered}(X,B,f)   
	}
	\Conclude{[p.*]}{\bd \TYPE{EvenlyCovered}(X,B,f,U)[2]\bd^{-1} \FUNC{card} [2]}
	{
		\forall u \in U \. \Big|f^{-1}(u)\Big| = n
	}
	\DeriveConclude{[n.*]}{\THM{OpenByOpenCover}}
	{
		\forall n \in \mathsf{CARD} \. 
		\bigg\{q \in B : \Big| f^{-1}(q) \Big| = 1  \bigg\} \in \T(B)
	}
	\DeriveConclude{[*]}{\bd \TYPE{Connected}[1]}{
		\exists n \in \mathsf{CARD} \. 
		\forall p \in B \. 
		\big| f^{-1}(p) \big| = n	
	}
	\EndProof
	\\
	\DeclareFunc{coveringNumber}{\TYPE{Covering}(X,B) \to \mathsf{CARD}}
	\DefineNamedFunc{coveringNumber}{f}{\mathrm{num}\; f}{\THM{CoveringMapHasNumber}}
	\\
	\Theorem{HausdorffByCovering}
	{
		\forall c : \TYPE{CoveringMap}(X,B) \. 
		\TYPE{Hausdorff}(B) \Imply \TYPE{Hausdorff}(X)
	}
	\AssumeIn{x,y}{X}
	\Assume{[1]}{ x \neq y }
	\Say{\Big(U,[2]\Big)}{\Elim \TYPE{CoveringMap}\big(X,B, c, c(x)  \big)}
	{
		\sum  U :  \TYPE{EvenlyCovered}(X,B,c) \.
		c(x) \in U
	}
	\Say{\Big(V,[3]\Big)}{\Elim \TYPE{CoveringMap}\big(X,B, c, c(y) \big)}
	{
		\sum  V :  \TYPE{EvenlyCovered}(X,B,c) \.
		c(y) \in V
	}
	\Say{\Big(U',[4]\Big)}{\Elim \TYPE{EvenlyCovered}\big(X,B, c, x \big)[2]}
	{
		\sum  U' \in \U(x) \And \TYPE{StronglyConnected} \. U \cong_{\TOP} U'
	}
	\Say{\Big(V',[5]\Big)}{\Elim \TYPE{EvenlyCovered}\big(X,B, c, y \big)[3]}
	{
		\sum  V' \in \U(y) \And \TYPE{StronglyConnected} \. V \cong_{\TOP} V'
	}
	\Assume{[6]}{c(y) = c(x)}
	\Conclude{[6.*]}{\Elim U' \Elim V' \Elim \TYPE{EvenlyCovered}}
	{
		U' \cap V' \neq \emptyset
	}
	\Derive{[6]}
	{  
		\Intro \Imply
	}
	{
		c(y) = c(x) \Imply 
		\exists u \in \U(x) : 
		\exists v \in \U(y) :
		u \cap v = \emptyset
	}
	\Assume{[7]}{c(x) \neq c(y)}
	\Say{(U'',V'',[8])}{\Elim \TYPE{Hausdorff}(U \cap V)\big(c(x),c(y)\big)}
	{
		\sum U'' \in \U(x) \. 
		\sum V'' \in \U(y) \. \NewLine 
		U'' \subset U \And
		V'' \subset V \And
		U'' \cap V'' = \emptyset
	}
	\Conclude{[*]}{\THM{DisjointPreimage}\Big(c,U'',V'',[8]\Big)}
	{
		c^{-1} U'' \cap c^{-1} V'' = \emptyset
	}
	\Derive{[7]}
	{  
		\Intro \Imply
	}
	{
		c(y) \neq c(x) \Imply 
		\exists u \in \U(x) : 
		\exists v \in \U(y) :
		u \cap v = \emptyset
	}
	\Conclude{\Big[(x,y).*\Big]}
	{
		\Elim(|)\LOGIC{LEM}\Big(c(x) = c(y)\Big)
	}
	{
		\exists u \in \U(x) :
		\exists v \in \U(y) :
		u \cap v = \emptyset
	}
	\DeriveConclude{{*}}{\Intro \TYPE{Hausdorff}}
	{
		\TYPE{Hausdorff}(X)
	}
	\EndProof
	\\
	\Theorem{ManifoldByCoveringBase}
	{
		\forall c : \TYPE{CoveringMap}(X,B) \. 
		B \in \TOPM \Imply X \in \TOPM
	}
	\NoProof
}\Page{
	\Theorem{ManifoldByCoveringSpace}
	{
		\forall c : \TYPE{CoveringMap}(X,B) \. 
		\TYPE{Haussdorff}(B) \And  X \in \TOPM 
		\Imply B \in \TOPM
	}
	\NoProof
	\\
	\Theorem{CoveringRestricion}
	{
		\forall c : \TYPE{CoveringMap}(X,B) \.
		\forall A \subset B \.
		\TYPE{Locally}\; \TYPE{PathConnected}(A) 
		\Imply \NewLine \Imply
		\forall C : \mathrm{PCC}\Big(c^{-1}(A)\Big) \.?
		\TYPE{CoveringMap}\Big( C, A  ,c_{|C}\Big)
	}
	\NoProof
	\\
	\Theorem{CoveringInducesCWStructure}
	{
		\forall (B,\E,\varphi) : \TYPE{CWComplex} \.
		\forall c : \TYPE{CoveringMap}(X,B) \. \NewLine \.
		\exists (Y,\F,\psi) : \TYPE{CWComplex} :		
		Y = X \And (X,\F,\psi) \Arrow{\CWR} (B,\E,\varphi)
	}
	\NoProof
}\Page{
	%!capm (Coverings and proper Maps)
	\Theorem{CoveringByRegularity}
	{
		\forall X,Y : \TYPE{StronglyConnected} \And \CG \And \TYPE{T2} \.
		\forall X \Arrow{f} Y : \CG \. \NewLine \. 
		\TYPE{Local}\;\TYPE{Homeomorphism}(X,Y,f) 
		\Imply
		\TYPE{Covering}(X,Y,f)
	}
	\Say{[1]}{\THM{ClosedMapLemma}(X,Y,f)\Elim \TYPE{ClosedMap}(X,Y,f)(X)}
	{
		\TYPE{Closed}\Big(Y,f(X) \Big) 
	}
	\Say{[2]}{\THM{LocalHomeoIsOpen}(X,Y,f)\Elim \TYPE{OpenMap}(X,Y,f)(X)}
	{
		\TYPE{Open}\Big(Y,f(X) \Big) 
	}
	\Say{[3]}{\Elim \TYPE{Connected}(X)\THM{NonEmptyImage}}{f(X) \neq \emptyset}
	\Say{[4]}{\Elim \TYPE{Connected}(Y)[1,2,3]}{f(X) = Y}
	\AssumeIn{y}{Y}
	\Say{[5]}{\Elim \TYPE{ProperMap}(X,y,f)\{y\}}{\TYPE{CompactSubset}\Big(X, f^{-1}(y) \Big)}
	\Say{[6]}{\Elim \FUNC{image}[4]}
	{
		f^{-1}(y) \neq \emptyset
	}
	\Say{\Big( U',[7]\Big)}{\Elim \TYPE{LocalHomeo}(X,Y,f)\Big( f^{-1}(y) \Big)}
	{
		\NewLine :
		\sum U' : \prod_{x \in f^{-1}(y)} \U(x) \. 
		\forall x \in f^{-1}(y)   \.  
		f(U'_x) \in \T(Y) \And 
		f_{|u'_x} : \TYPE{Homeomorphism}\Big( U', f(U') \Big)
	}
	\Assume{[8]}{\Big| f^{-1}(y) \Big| = \infty}
	\Say{\Big(\X,[9]\Big)}{\Elim \TYPE{CompactSubset}\Big(f^{-1}(y)\Big)(U')}
	{
		\sum \X : \TYPE{Finite}\Big( f^{-1}(y) \Big) \. \TYPE{OpenCover}\Big( f^{-1}(y), U'_\X \Big)
	}
	\Say{\Big(x,x',[10]\Big)}{\THM{DirichletPrinciple}[8][9]}{
		\sum x \in \X \.
		\sum x' \in f^{-1}(y) :
		x \neq x' \And x \in U'_x
	}
	\Say{[11]}{\Elim \FUNC{preimage}\Elim x \Elim x'}{ f(x) = y = f(x') }
	\Say{[12]}{[7](x) \Elim \TYPE{Hemeomorphisn}[10]}{f(x) \neq f(x')}
	\Conclude{[8.*]}{\Intro(\bot)[11][12]}{\bot }
	\Say{[8]}{\Elim(\bot)}{\Big| f^{-1}(y) \Big| < \infty}
	\Say{\Big(U,[9] \Big)}{\Elim \TYPE{StronglyConnected}(X)(U'}
	{
		\sum U : \prod_{x \in f^{-1}(y)} \U(x) \And \TYPE{StronglyConnected} \. 
		\forall x \in f^{-1}(y) \. U_x \subset U'_x
	}
	\SayIn{V}{\bigcap_{x \in f^{-1}(y)} f(U_x)}
	{
		\U(y)
	}
	\Conclude{[y.*]}{\Elim V \Intro \TYPE{EvenlyCovered}}{\TYPE{EvenlyCovered}(X,Y,f,V)}
	\DeriveConclude{[*]}{\Intro \TYPE{CoveringMap}}{\TYPE{CoveringMap}(X,Y,f)}
	\EndProof
}\Page{
	\Theorem{CoveringIsProperIffFinite}
	{
		\forall c : \TYPE{CoveringMap}(X,B) \.
		\TYPE{ProperMap}(X,B,c) \iff \mathrm{num}\; c < \infty
	}
	\Assume{[1]}{\TYPE{ProperMap}(X,B,c)}
	\AssumeIn{p}{B}
	\Say{[2]}{\Elim \TYPE{ProperMap}(X,B,c)\{p\}}
	{
		\TYPE{CompactSubset}\Big(X, f^{-1}(p)\Big)
	}
	\Say{[3]}{\Elim \TYPE{CoveringMap}\Big(X,B,c)\{p\}}
	{
		\TYPE{DiscreteSubset}\Big( X, c^{-1}(p) \Big)
	}
	\Conclude{[1.*]}{
		\THM{DiscreteCompactIsFinite}[2][3]
	}
	{
		\Big|c^{-1}(p)\Big| < \infty
	}
	\Derive{[1]}{\Intro(\Imply)}
	{
		\TYPE{Proper}(X,B,c) 
		\Imply
		\mathrm{num} \; c < \infty
	}
	\Assume{[2]}{\mathrm{num} \; c < \infty }
	\SayIn{n}{\mathrm{num}\;c}{\Nat}
	\Say{[3]}{\THM{DiscreteCompactIsFinite}\Elim (\mathrm{num} \; c)[2]}
	{
		\forall p \in B \. 
		\TYPE{CompactSubset}\Big( X   ,c^{-1}(p) \Big)
	}
	\Assume{A}{\TYPE{Closed}(X)}
	\Say{\Big( U,[4]\Big)}{\Elim (\mathrm{num}\; c)[2]}
	{
		\sum U : [1,\ldots,n] \to \T(X) \.
		A \subset \bigcup^{n}_{i=1} \And
		\forall i \in [1,\ldots,n] \. \TYPE{Homeomorphism}(U_i,c(U_i),c_{|U_i})
	}
	\Say{[5]}{\THM{ClosedSubset}(X,U,A)\Elim \TYPE{Homeomorphism}[4]}
	{  
		\forall i \in [1,\ldots,n] \. \TYPE{Closed}\Big( f(U_i), c(U_i \cap A) \Big)	
	}
	\Say{[6]}{[4]\THM{ImageUnion}}{c(A) = c\Act{\bigcup^n_{i=1} A \cap U_i} = \bigcup^n_{i=1} c(A \cap U_i)}
	\Conclude{[A.*]}{\THM{ClosedUnion}[5][6]}{\TYPE{Closed}\Big(B,c(A)\Big)}
	\Derive{[4]}{\Intro \TYPE{ClosedMap}}{\TYPE{ClosedMap}\Big(X,c(A)\Big)}
	\Conclude{[2.*]}{\THM{ProperByCompactFibers}[3][4]}{\TYPE{ProperMap}(X,B,c)}
	\Derive{[2]}{\Intro(\Imply)}
	{
		\mathrm{num} \; c < \infty
		\Imply
		\TYPE{Proper}(X,B,c) 
	}
	\Derive{[*]}{\Intro(\iff)[1][2]}
	{
		\TYPE{Proper}(X,B,c) 
		\iff
		\mathrm{num} \; c < \infty
	}
	\EndProof
	\\
	\Theorem{ProperMapCompact}
	{
		\forall c : \TYPE{CoveringMap}(X,B) \.
		\TYPE{Compact}(X) \iff
		\mathrm{num} \; c < \infty \And \TYPE{Compact}(B)
	}
	\NoProof
}
\newpage
\subsection{Lifting}
\Page{
	\DeclareType{Lift}{
		\prod c : \TYPE{CoveringMap}(X,B) \. 
		\prod Y \in \TOP \. 
		(Y \Arrow{\TOP} B) \to
		?(Y \Arrow{\TOP} X)
	}
	\DefineType{g}
	{
		Lift
	}
	{
		\Lambda f : Y \Arrow{\TOP} B \. f = g c
	}
	\\
	\Theorem{UniqueLiftingProperty}
	{
		\forall c : \TYPE{CoveringMap}(X,B) \. 
		\forall Y : \TYPE{Connected} \.
		\forall f : Y \Arrow{\TOP} X \. 
		\forall g,g' : \TYPE{Lift}(c,f) \. \NewLine \.
		\forall  y \in Y \.  
		g(y) =  g'(y) \Imply  g = g'
	}
	\NoProof
	\\
	\Theorem{HomotopyLiftingProperty}
	{
		\forall c : \TYPE{CoveringMap}(X,B) \. 
		\forall Y : \TYPE{Locally} \;\TYPE{Connected} \.
		\forall f,f' : Y \Arrow{\TOP} B \. \NewLine \. 
		\forall H : \TYPE{Homotopy}(Y,B,f,f') \. 
		\forall g : \TYPE{Lift}(c,f) \.
		\exists! g' : \TYPE{Lift}(c,g') :
		\exists! G : \TYPE{Homotopy}(Y,X,g,g') :
		G c = H 
	}
	\NoProof
	\\
	\Theorem{PathLifitngProperty}
	{
		\forall c : \TYPE{CoveringMap}(X,B) \.
		\forall \gamma : I \Arrow{\TOP} B \.
		\forall x \in c^{-1} \Big( f(0) \Big) \. \NewLine \.
		\exists! \xi : \TYPE{Lift}(c,\gamma) :
		\xi(0) = x
	}
	\NoProof
	\DeclareFunc{pathLift}
	{
		\prod c : \TYPE{CoveringMap}(X,B) \.
		\prod  I \Arrow{\gamma} B : \TOP  \. 
		c^{-1}\Big( \gamma(0) \Big) \to \TYPE{Lift}(c,\gamma)
	}
	\DefineNamedFunc{pathLift}{x}{{\tilde \gamma}_x}{ \THM{PathLifitingProperty}(c,\gamma,x)  }
	\\
	\Theorem{MonodromyTHM1}
	{
		\forall c : \TYPE{CoveringMap}(X,B) \.
		\forall p,q \in B \. 
		\forall \alpha,\beta \in \Omega(p,q) \.
		\forall x \in c^{-1}(p) \. 
		{\tilde \alpha}_x \sim {\tilde \beta}_x 
		\iff
		\alpha \sim \beta
	}
	\Say{[1]}{\bd \TYPE{Lift}(c,\alpha) \bd {\tilde \alpha}_x }
	{   {\tilde \alpha }_x c = \alpha}
	\Say{[2]}{\bd \TYPE{Lift}(c,\beta) \bd {\tilde \beta}_x }
	{   {\tilde \beta }_x c = \beta}
	\Assume{[3]}{{\tilde \alpha}_x \sim {\tilde \beta}_x}
	\Conclude{[1.*]}{\THM{CPreservesHomotopy}[1,2,3]}
	{
		\alpha \sim \beta
	}
	\Derive{[3]}
	{
		\Intro(\Imply)
	}
	{
		{\tilde \alpha}_x \sim {\tilde \beta}_x 
		\Imply
		\alpha \sim \beta
	}
	\Assume{[4]}
	{
		\alpha \sim \beta	
	}
	\Say{H}{\bd \TYPE{Homotopic}[4]}{\TYPE{Homotopy}(I,B,\alpha,\beta)}
	\Say{\Big(\gamma,G,[5]\Big)}{\THM{HomotopyLiftingProperty}(c,H,{\tilde \alpha}_x)}
	{
		\sum \gamma : \TYPE{Lift}(c,\gamma) \.
		\sum G : \TYPE{Homotopy}(I,X,{\tilde \alpha}_x,\gamma) \. \NewLine \. 
		H = G c
	}
	\Say{[6]}{\bd \TYPE{CoveringMap}[5]}{\gamma(0) = { \tilde \beta }_x(0) }
	\Say{[7]}{\THM{UniqueLifitingProperty}[6]}{\gamma = {\tilde \beta}_x}
	\Conclude{4.*}{\bd^{-1} \TYPE{Homotopic}(G)[7]}{{\tilde \alpha}_x \sim {\tilde \beta}_x }
	\Derive{[4]}
	{
		\Intro(\Imply)
	}
	{
		{\tilde \alpha}_x \sim {\tilde \beta}_x 
		\Imply
		\alpha \sim \beta
	}
	\Derive{[5]}
	{
		\Intro(\iff)[4][5]
	}
	{
		{\tilde \alpha}_x \sim {\tilde \beta}_x 
		\Imply
		\alpha \sim \beta
	}
	\EndProof
}\Page{
	\Theorem{MonodromyTHM2}
	{
		\forall c : \TYPE{CoveringMap}(X,B) \.
		\forall p,q \in B \. 
		\forall \alpha,\beta \in \Omega(p,q) \.
		\forall x \in c^{-1}(p) \. 
		\alpha \sim \beta
		\Imply 
		{\tilde \alpha }_x(1) = {\tilde \beta}_x(1)
	}
	\NoProof
	\\
	\Theorem{InjectivityTheorem }
	{
		\forall c : \TYPE{CoveringMap}(X,B) \.
		\forall x \in X \.
		\TYPE{Injective}\Big( \pi(X,x), \pi\big(B,c(x)\big), c_* \Big)
	}
	\AssumeIn{\alpha,\beta}{\Omega( x )}
	\Say{[1]}{\bd \TYPE{Lift}\big(c,c_*(\alpha)\big) \bd \widetilde{ c(\alpha)}_x }
	{   \widetilde{ c_*(\alpha) }_x c = c_*(\alpha) }
	\Say{[2]}{\bd \TYPE{Lift}\big(c,c_*\big(\beta\Big)) \bd \widetilde{ c(\beta)}_x }
	{   \widetilde{  c_* \beta }_x c = c_*(\beta)}
	\Say{[3]}{\bd c_* \bd \FUNC{pathLift}(\alpha)\THM{UniqueLiftingProperty}[1]}
	{
		\widetilde{ c_* \alpha  }_x = \alpha
	}
	\Say{[4]}{\bd c_* \bd \FUNC{pathLift}(\beta) \THM{UniqueLiftingProperty}[2]}
	{
		\widetilde{ c_* \beta  }_x = \beta
	}
	\Assume{[5]}{c_* \alpha \sim c_* \beta}
	\Say{[6]}{\THM{MonodromyTHM1}(c,\alpha,\beta,x)[3]}
	{
		\widetilde{ c_* \alpha}_x \sim \widetilde{ c_* \beta}_x
	}
	\Conclude{[5.*]}{\Elim^2(=)\Big([3],[4]\Big)[6] }
	{
		\alpha \sim 
	}
	\DeriveConclude{[*]}{\bd^{-1} \TYPE{Injective}}
	{
		\TYPE{Injective}\Big( \pi(X,x), \pi\big( B, c(x) \big), c_* \Big)
	}
	\EndProof
	\\
	\DeclareFunc{coveringInducedSubgroup}
	{
		\prod c : \TYPE{CoveringMap}(X,B) \. \TYPE{Subgroup}\Big( \pi(B) \Big)
	}
	\DefineNamedFunc{coveringInducedSubgroup}{}{\pi(c)}{c_* \pi(X)}
	\\
	\Theorem{LiftingCriterion}
	{
		\forall c : \TYPE{CoverinngMap}(X,B) \.
		\forall Y : \TYPE{StronglyConnected} \.
		\forall Y \Arrow{f} B : \TOP \.
		\forall y \in Y \. \NewLine \. 
		\forall x \in X \.  
		\forall c(x) = f(y) \. 
		\Big(
			\exists f' : \TYPE{Lift}(c,f) : f'(y) = x
		\Big)
		\iff
		f_* \pi(Y,y) \subset \pi(c)
	}
	\Assume{f'}{\TYPE{Lift}(c,f)}
	\Assume{[1]}{f'(y) = y}
	\Say{[2]}{\bd \TYPE{Lift}(c,f,f')}{f = f'c}
	\Say{[3]}{\bd \Cov(\pi)[2]}{f_* = f'_* c_*}
	\Conclude{[f'.*]}{\THM{ImageComposition}[3]\bd^{-1} \pi(c)}
	{
		f_* \pi(Y,y) \subset \pi(c)
	}
	\Derive{[1]}{\Intro(\Imply)}
	{
		\Big(
			\exists f' : \TYPE{Lift}(c,f) : f'(y) = x
		\Big)
		\Imply
		f_* \pi(Y,y) \subset \pi(c)
	}
	\Assume{[2]}{f_* \pi(Y,y) \subset \pi(c)}
	\AssumeIn{u}{Y}
	\SayIn{\gamma}{\bd \TYPE{PathConnected}(Y)(y,u)}{\Omega(y,u)}
	\SayIn{f'(y)}{\widetilde{ f_* \gamma  }_x(1)}{X} 
	\AssumeIn{\delta}{ \Omega(y,u) }
	\Say{[3]}{[2](\delta \gamma^{-1})}
	{ 
		f_* \big[ \delta \gamma^{-1}] \in \pi(c)
	}
	\Say{\Big(\xi,[4] \Big) }{ \bd \pi(c) [3]}
	{
		\sum \xi \in \Omega(x) \. 
		f_*\Big[( \delta \gamma^{-1}  ) ] = c_*\Big[ \xi \Big]
	}
	\Say{[5]}{\bd \GRP(\pi(Y)\Arrow{f_*}\pi(B)) \bd \pi }
	{
		f(\delta) f^{-1}(\gamma) \cong_x  c(\xi)
	}
	\Say{[6]}{[5]f(\gamma)}{ c(\xi) f(\gamma) = f(\delta)}
}\Page{
	\Conclude{[u.*]}{\THM{MonodromyTHM2}[6]\bd \FUNC{pathComposition}}
	{
		\widetilde{f_* \delta}_x(1) = 
		\widetilde{ (c_* \xi) (f_* \gamma)}_x(1) = 
		\widetilde{ f_* \gamma  }_x(1)
	}
	\Derive{Y \Arrow{f'} X}{\LOGIC{WellDefine}}{\SET}
	\Assume{V}{\TYPE{PathConnsectedSubset}(Y)}
	\AssumeIn{u,v}{V}
	\SayIn{\gamma}{\bd \TYPE{PathConnected}(Y)(y,u)}{\Omega(y,u)}
	\SayIn{\delta}{\bd \TYPE{PathConnected}(V)(u,v)}{\Omega(u,v)}
	\Say{[4]}{\ByConstr f}{  f'_* (\gamma \delta) = \widetilde{(f_* \gamma)(f_* \delta)}_x }
	\Conclude{[V.*]}{\bd \FUNC{pathComposition}\bd \FUNC{mapping} }
	{
		f'_* \delta \in \Omega(f'(u),f'(v))
	}
	\Derive{[4]}{\Intro(\forall)}
	{
		\forall  V : \TYPE{PathConnectedSubset}(Y) \.  
		\TYPE{PathConnectedSubset}\Big(X, f(V')\Big)
	}
	\AssumeIn{u}{Y}
	\Say{\Big(V,[5]\Big)}{\bd \TYPE{CoveringMap}(X,B,c)\Big(f(u)\Big)}
	{
		\sum V : \TYPE{EvenlyCovered}\big( X,B,c \big) \. 
		f(u) \in V
	}
	\Say{\Big( W,\sigma,[6])}{ \THM{EvenlyCoveredHasLoclaSection} }
	{
		\sum (W,\sigma) : \TYPE{LocalSection}(c) \. f'(u) \in U
	}
	\Say{\Big(U,[7]\Big)  }
	{
		\bd \TYPE{LocallyPathConnected}\Big(Y,f^{-1}(V)\Big)(u)           
	}
	{
		\sum U \in \U(u) \And \TYPE{PathConnected}(U) \. U \subset f^{V}
	}
	\Say{[8]}{[4](U)}{\TYPE{PathConnected}\Big(f'(U)\Big)}
	\Say{[9]}{[4](U)\ByConstr f' }{f'(U) \subset c^{-1}(W)}
	\Say{[10]}{[9] \bd U}{  f'(U) \subset W  }
	\Say{[11]}{\bd \TYPE{LoclalSection}(\sigma)[10]}
	{
		\forall v \in U \.
		f' s(u) = f'(u) = f \sigma s(u)
	}
	\Say{[12]}{\bd \TYPE{InjectionRightInversion}[11]}
	{
		\forall v \in U \.
		f' (u)  = f \sigma(s)
	}
	\Conclude{[u.*]}{\bd \CAT(\TOP)(f,\sigma)\Elim(=)[12]}
	{
		f'_{|U} \in  \TOP(U,X)
	}
	\DeriveConclude{[2.*]}{\THM{LocallyContinuousIsContinuous}}
	{
		f' \in \TOP(Y,X)
	}
	\Derive{[2]}{\Intro(\Imply)}
	{
		f_* \pi(Y,y) \subset \pi(c)
		\Imply
		\Big(
			\exists f' : \TYPE{Lift}(c,f) : f'(y) = x
		\Big)
	}
	\Conclude{[*]}{\Intro(\iff)[2]}
	{
		\Big(
			\exists f' : \TYPE{Lift}(c,f) : f'(y) = x
		\Big)
		\iff
		f_* \pi(Y,y) \subset \pi(c)	
	}
	\EndProof
	\\
	\Theorem{SimplyLifting}
	{
		\forall c : \TYPE{CoverinngMap}(X,B) \.
		\forall Y : \TYPE{StronglyConnected} \And \TYPE{SimplyConnected} \.
		\NewLine \.
		\forall Y \Arrow{f} B : \TOP \.
		\forall y \in Y \.
		\forall x \in X \.  
		\forall c(x) = f(y) \. 
		\exists f' : \TYPE{Lift}(c,f) : f'(y) = x
	}
	\NoProof
}
\newpage
\subsection{Transitive Group Action}
\Page{
	\DeclareType{Transitive}{
		\prod G \in \GRP \. 
		\prod X \in \SET \.
		?(X \action G)
	}
	\DefineType{\alpha}{Transitive}{\forall x,y \in X \. \exists g \in G \. \alpha(x,g) = }
	\\
	\Theorem{TransitiveGSetStabilizerAction}
	{
		\forall G \in \GRP \.
		\forall X \in \SET \.
		\forall \alpha : X \action G \. \NewLine
		\forall x \in X \. 
		\forall g \in G \. 
		\Stab(xg) = g^{-1}\Stab(x)g 
	}
	\Assume{h}{\Stab(xg)}
	\Say{[1]}{\bd \Stab(xg) h}{xgh = xg}
	\Say{[2]}{[1]g^{-1}}{xghg^{-1} = x}
	\Say{[3]}{\bd \Stab(x)[2]}{ ghg^{-1} \in \Stab(x)}
	\Conclude{[4]}{g^{-1}[3] g}{h \in g^{-1}\Stab(x)g}
	\Derive{[1]}{\bd^{-1} \TYPE{Subset}}{\Stab(xg) \subset g^{-1}\Stab(x)g}
	\Assume{h}{g^{-1}\Stab(x)g}
	\Say{\Big( f,[2]\Big)}{\bd \TYPE{Coset}\bd h}
	{
		\sum f \in \Stab(x) \. h = g^{-1} f g
	}
	\Say{[3]}{ 
		\Elim(=)
		\Big(xgh,[2]\Big) 
		\bd \TYPE{Inverse}(G,d)    
		\bd \Stab(x)(f)
	}
	{
		xgh = 
		xgg^{-1} f g = 
		x f g =
		x g
	}
	\Conclude{[4]}{\bd \Stab(xg) [3]}
	{
		g \in \Stab(xg)
	}
	\Derive{[2]}{\Intro \subset}{g^{-1} \Stab(x) g \subset \Stab(xg)}
	\Conclude{[*]}{\bd^{-1} \TYPE{SetEq}}
	{
		\Stab(xg) = g^{-1} \Stab(x) g
	}
	\EndProof
	\\
	\Theorem{StabIsAGMap}
	{
		\forall X : G\hyph\SET \.
		X  \Arrow{\Stab_X}  \Gamma_G : G\hyph\SET  
	}
	\NoProof
	\\
	\Theorem{StabAreOrbit}
	{
		\forall \alpha : \TYPE{Transitive}(G,X) \.
		\Big\{ \Stab_\alpha(x) \Big| x \in X \Big\} \in O_{\Gamma_G}
	}
	\NoProof
	\\
	\DeclareFunc{isotropyType}{ \TYPE{Transitive}(G,X) \to O_{\Gamma_G} }
	\DefineNamedFunc{isotropyType}{\alpha}{\mathrm{type}\;\alpha}
	{
		\Big\{ \Stab_\alpha(x) \Big| x \in X \Big\}
	}
}
\Page{
	\Theorem{GMapsBetweenTransitiveAreDeterminedByOnePoint}
	{
		\NewLine ::
		\forall G \in \GRP \.
		\forall X : \TYPE{Transitive}(G) \.
		\forall Y : G\hyph\SET \.
		\forall X \Arrow{\alpha,\beta} Y : G\hyph\SET \.
		\forall x \in X \. 
		\NewLine \. 
		\alpha(x) = \beta(x) \Imply f = g 
	}
	\AssumeIn{v}{X}
	\Say{\Big(g,[1] \Big)}{\bd \TYPE{Transitive}(G,X)(x,v)}
	{
		\sum g \in G \. xg=v
	}
	\Conclude{[v.*]}{
		\Elim(=)\Big([1],\alpha(v)\Big) 
		\bd X \Arrow{\alpha} Y : G\hyph\SET
		[0]
		\bd X \Arrow{\beta} Y : G\hyph\SET
		\Elim(=)([1])
	}
	{
		\NewLine = 
		\alpha(v) =
		\alpha(xg) =
		\alpha(x)g = 
		\beta(x)g =
		\beta(xg) =
		\beta(v)
	}	
	\DeriveConclude{[*]}{\Intro(=,\to)}{\alpha = \beta}
	\EndProof
	\\
	\Theorem{GMapsBetweenTransitiveAreSujective}
	{
		\NewLine ::
		\forall G \in \GRP \.
		\forall X \in G\hyph\SET \And \TYPE{Nonempty} \.
		\forall \beta : \TYPE{Transitive}(G,Y) \.
		\forall X \Arrow{f} \beta : G\hyph\SET \.
		\TYPE{Surjective}(X,Y,f)
	}
	\SayIn{y}{f(x)}{Y}
	\AssumeIn{u}{Y}
	\Say{\Big(g,[1] \Big)}{\bd \TYPE{Transitive}(G,Y)(y,u)}
	{
		\sum g \in G \. yg=u
	}
	\Conclude{[u.*]}{[1]\ByConstr y \bd X \Arrow{f} \beta : G\hyph\SET }
	{
		u = yg = f(x)g = f(xg)
	}
	\DeriveConclude{[*]}{\bd^{-1} \TYPE{Surjective}}
	{
		\TYPE{Surjective}(X,Y,f)
	}
	\EndProof
}\Page{
	\Theorem{ExistanceOfTransitiveGMap}
	{
		\forall G \in \GRP \.
		\forall X : \TYPE{Transitive}(G) \.
		\forall Y : \TYPE{Transitive}(G) \.
		\NewLine \.
		\forall x \in X \.
		\forall y \in Y \.
		\Big( \exists X  \Arrow{f} \beta : G\hyph\SET : f(x) = y \Big) 
		\iff
		\Stab(x) \subset \Stab(y)
	}
	\AssumeIn{f}{G\hyph \SET (X,Y)}
	\Assume{[1]}{f(x) = y}
	\AssumeIn{g}{\Stab(x)}
	\Say{[2]}{\bd \Stab(x)(g)}{xg = x}
	\Say{[3]}{[1]\Elim(=)[2] \bd G\hyph \SET (X,Y)\Elim(=)[1]}
	{
		y = 
		f(x) = 
		f(xg) = 
		f(x)g =
		yg
	}
	\Conclude{[g.*]}{\bd \Stab(y) [3]}{g \in \Stab(y)}
	\DeriveConclude{[f.*]}{\bd^{-1} \TYPE{Subset}}
	{
		\Stab(x) \subset \Stab(y)
	}
	\Derive{[1]}{\Intro(\Imply)}
	{
		\Big( \exists X  \Arrow{f} \beta : G\hyph\SET : f(x) = y \Big) 
		\Imply
		\Stab(x) \subset \Stab(y)
	}
	\Assume{[2]}{\Stab(x) \subset \Stab(y)}
	\AssumeIn{v}{X}
	\Say{\Big(g,[3] \Big)}{\bd \TYPE{Transitive}(G,X)(x,v)}
	{
		\sum g \in G \. xg= v
	}	
	\SayIn{f(v)}{yg}{Y}
	\AssumeIn{h}{G}
	\Assume{[4]}{xh = v}
	\Say{[5]}
	{
		[3]g^{-1} \Elim(=)[4]	
	}
	{ 
		x = vg^{-1} = xhg^{-1}
	}
	\Say{[6]}
	{
		\bd \TYPE{Stab}(x)[5]
	}
	{
		hg^{-1} \in \Stab(x)
	}
	\Say{[7]}{ \bd \TYPE{Subset}([2])[6] }{hg^{-1} \in \Stab(y)}
	\Conclude{[x.*]}{ \Elim(=)\Big( \bd \Stab(y)[7],yg \Big) \bd \TYPE{Inverse}(G)(g) }{ yg =yhg^{-1}g = yh  }
	\Derive{f}{\LOGIC{WellDefined}}
	{
		X \to Y
	}
	\AssumeIn{v}{X}
	\AssumeIn{g}{G}
	\Say{\Big(h,[3] \Big)}{\bd \TYPE{Transitive}(G,X)(x,vg)}
	{
		\sum h \in G \. xh= vg
	}	
	\Say{[4]}{[3]g^{-1}}
	{
		v = xhg^{-1}
	}
	\Say{[5]}{\ByConstr f}{f(v) = yhg^{-1} }
	\Conclude{[v.*]}{ \ByConstr f(vg) [3] \bd \TYPE{Inverse}(G)(g) \bd \TYPE{Identity}(G) \Elim(=)[5]   }
	{
		f(vg) =
		yh =
		yhg^{-1} g =
		f(v)g
	}
	\Derive{[3]}{\bd G\hyph\SET}{f \in G\hyph\SET(X,Y)}
	\Conclude{[2.*]}{\ByConstr f(x)}{f(x) = y}
	\Derive{[2]}{\Intro \Imply}
	{
		\Stab(x) \subset \Stab(y)
		\Imply
		\Big( \exists X  \Arrow{f} \beta : G\hyph\SET : f(x) = y \Big) 
	}
	\Conclude{[*]}{\Intro \iff [1][2]}
	{
		\Stab(x) \subset \Stab(y)
		\iff
		\Big( \exists X  \Arrow{f} \beta : G\hyph\SET : f(x) = y \Big) 	
	}
	\EndProof
}\Page{
	\Theorem{GSetInversion}
	{
		\forall G \in \GRP \. 
		\forall X \Arrow{f} Y : G\hyph\Set \.
		\forall [0] : \TYPE{Bijiection}(X,Y,f) \.
		Y \Arrow{f^{-1}} X : G\hyph\Set 
	}
	\AssumeIn{y}{Y}
	\Say{\Big(x,[1]\Big)}{ \bd \TYPE{Bijection} }
	{
		\sum x \in X \. y = f(x)
	}
	\AssumeIn{g}{G}
	\Conclude{ [y.*] }
	{  
		\Elim(=)\Big( [1], f^{-1}(yg) \Big)
		\bd \TYPE{Inverse}(f)
	}
	{
		f^{-1}(yg) = 
		f^{-1}\Big(f(x)g\Big) = 
		ff^{-1}(xg) =
		xg
	}
	\DeriveConclude{[*]}{\bd G\hyph\SET }
	{
		f^{-1} \in G\hyph\SET(Y,X)
	}
	\EndProof
	\\
	\Theorem{GSetIsomorphismExistance}
	{
		\forall G \in \GRP \. 
		\forall X,Y : \TYPE{Transitive}(G) \.
		\forall x \in X \.
		\forall y \in Y \.
		\NewLine \.
		\left( 
			\exists X \ToIso{f} Y : G\hyph\SET : f(x) = y
		\right)
		\iff
		\Stab(x) = \Stab(y)
	}
	\Assume{f}{\TYPE{Isomorphism}(G\hyph\SET,X,Y)}
	\Assume{[1]}{f(x) = y}
	\Say{[2]}{\THM{ExistanceOfTransitiveGMap}\Big(f,[1]\Big)}
	{
		\Stab(x) \subset \Stab(y)
	}
	\Say{[3]}{\THM{ExistanceOfTransitiveGMap}\Big(f^{-1},[1]\Big)}
	{
		\Stab(y) \subset \Stab(x)
	}
	\Conclude{[f.*]}{\bd^{-1} \TYPE{SetEq}}{\Stab(x) = \Stab(y)}
	\Derive{[1]}{\Intro \Imply}
	{
		\left( 
			\exists X \ToIso{f} Y : G\hyph\SET : f(x) = y
		\right)
		\Imply
		\Stab(x) = \Stab(y)
	}
	\Assume{[2]}{\Stab(x) = \Stab(y)}
	\Say{\Big( f,[3]\Big)}{ \THM{ExistanceOfTransitiveMap}[2]}
	{
		\sum f \in G\hyph \SET(X,Y) \. 
		f(x) = y
	}
	\Say{\Big( f',[4]\Big)}{ \THM{ExistanceOfTransitiveMap}[2]}
	{
		\sum f' \in G\hyph \SET(Y,X) \. 
		f'(y) = x
	}
	\Say{[5]}{[3][4]}{ff'(x) = x}
	\Say{[6]}{\THM{GMapsBetweenTransitiveAreDeterminedByOnePoint}[5]}{ff' = \id}
	\Say{[7]}{[3][4]}{f'f(y) = y}
	\Say{[8]}{\THM{GMapsBetweenTransitiveAreDeterminedByOnePoint}[7]}{f'f = \id}
	\Conclude{[2.*]}{\THM{GSetInversion}[6][8]}{\TYPE{Isomorphism}(G\hyph\SET,X,Y,f)}
	\Derive{[2]}{\Intro\Imply}{
		\Stab(x) = \Stab(y)
		\Imply
		\left( 
			\exists X \ToIso{f} Y : G\hyph\SET : f(x) = y
		\right)
	}
	\Conclude{[*]}{\Intro \iff [1][2]}
	{
		\left( 
			\exists X \ToIso{f} Y : G\hyph\SET : f(x) = y
		\right)
		\iff
		\Stab(x) = \Stab(y)
	}
	\EndProof
	\\
	\Theorem{TransitiveIsomorphismCriterion}
	{
		\forall G \in \GRP \. 
		\forall X,Y : \TYPE{Transitive}(G) \.
		X \cong_{G\hyph\SET} Y
		\iff
		\NewLine
		\iff
		\mathrm{type}(X) = \mathrm{type}(Y)	
	}
	\NoProof
}
\Page{
	\Theorem{TransitiveAutomrphismExists}
	{
		\NewLine ::
		\forall G \in \GRP \. 
		\forall X : \TYPE{Transitive}(G) \.
		\forall x \in X \. 
		\forall g \in N\big( \Stab(x) \big) \.
		\exists! f \in \Aut_{G\hyph \SET}(X) \. 
		f(x) = f(xg)
	}\NoProof
	\\
	\DeclareFunc{structutalAutomorphism}
	{
		\NewLine ::
		\prod G \in \GRP \.
		\prod X : \TYPE{Transitive}(G) \.
		\prod x \in X \.
		N\big( \Stab(x) \big) \Arrow{\GRP} \Aut_{G\hyph \SET}(X) 
	}
	\DefineNamedFunc{structuralAutomorphism}{g}{\varphi_{x,g}}
	{
		\THM{TransitiveutomorphismExists}(G,X,x,g)
	}
	\\
	\Theorem{StrucuralAutomorphismIsSurjective}
	{
		\NewLine ::  
		\forall G \in \GRP \. 
		\forall X : \TYPE{Transitive}(G) \.
		\forall x \in X \. 
		\TYPE{Surjective}\Big( N\big( \Stab(x) \big), \Aut_{G\hyph \SET}(X), \varphi_x \Big)
	}
	\Assume{f}{\Aut_{G\hyph\SET}(X)}
	\SayIn{y}{f(x)}{X}
	\Say{\Big(g,[1]\Big)}{\bd \TYPE{Transitive}(G,X)(x,y)}
	{
		\sum g \in G \. y = xg
	}
	\AssumeIn{h}{\Stab(x)}
	\Say{[2]}
	{
		\Elim(=)\Big( [1], x g h g^{-1} \Big)
		\ByConstr y
		\bd G\hyph\SET(X,X)(f)(x,h)
		\bd \Stab(x)(h) 
		\ByConstr^{-1} y
		\Elim(=)[1]
		\bd \TYPE{Inverse}(G,g)
	}
	{
		\NewLine :
		x g h g^{-1} = 
		y h g^{-1}  =
		f(x) h g^{-1} =
		f(xh) g^{-1} = 
		f(x) g^{-1} =
		y g^{-1} =
		x gg^{-1} =
		y
	}
	\Conclude{[h.*]}{\bd \Stab(x) [2]}{ ghg^{-1} \in \Stab(x)  }
	\Derive{[2]}{\bd N\Big( \Stab(x) \Big)}{g \in N\Big( \Stab(x) \Big)}
	\Say{[*]}{\THM{GMapsBetweenTransitiveAreDeterminedByOnePoint}\bd \varphi_{x,g}\ByConstr y [1]}
	{
		\varphi_{x,g} = f
	}
	\EndProof
	\\
	\Theorem{TransitiveAutomorphismStructure}
	{
		\NewLine :: 
		\forall G \in \GRP \. 
		\forall X : \TYPE{Transitive}(G,X) \.
		\forall x \in X \.  
		\End_{G\hyph\SET}(X) \cong_\GRP \frac{N\big(\Stab(x)\big)}{\Stab(x)}
	}
	\NoProof
}
\newpage
\subsection{Monodromy Action}
\Page{
	\DeclareFunc{monodromyAction}
	{
		\prod c : \TYPE{CoveringMap}(X,B) \.
		\prod p \in B \.
		 c^{-1}(p) \action \pi(B)
	}
	\DefineNamedFunc{monodromyAction}{x,[\gamma]}{x \action_{c,p} [\gamma]}{\widetilde{\gamma}_x(1)}
	\\
	\Theorem{MonodromyActionIsTransitive}
	{
		\forall c : \TYPE{CoveringMap}(X,B) \.
		\forall p \in B \.
		\TYPE{Transitive}(\action_{c,p})
	}
	\NoProof
	\\
	\Theorem{StabilizerOfMonodromyGroup}
	{
		\forall c : \TYPE{CoveringMap}(X,B) \. 
		\forall p \in B \.
		\forall x \in c^{-1}(p) \. 
		\Stab_{\action_{c,p}}(x) =  \pi(c) 
	}
	\AssumeIn{g}{\pi(c)}
	\Say{\Big(\gamma, [1] \Big)}
	{
		\bd \pi(c)(x)	
	}
	{
		\sum \gamma \in \Omega(x) =
		g = \big[c_* \gamma\big] 
	}
	\Conclude{[g.*]}
	{
		\Elim (\action_{c,p})(xg)[1]
		\THM{UniqueLifitingProperty}(c)
		\Elim \Omega(x)
	}
	{  
		xg  = \widetilde{[c_* \gamma]}_x(1)  = \gamma(1) = x  
	}
	\Derive{[1]}{\Intro \TYPE{\Subset}}{\pi(c) \subset \Stab_{\action_{c,p}}(x)}
	\AssumeIn{g}{\Stab_{\action_{c,p}}(x)}
	\Say{[2]}
	{
		\Elim \Stab_{\action_{c,p}}(x)(g) 
	}
	{
		x = xg = \widetilde{g}_x(1)
	}
	\Say{\Big( \gamma  ,[3] \Big)}{ \Elim \widetilde{g}_x(1)[2]  }
	{
		\sum \gamma \in \Omega(x) \. 
		g = [c_* \gamma] 
	}
	\Conclude{[2.*]}{\Elim \pi(c)[3]}
	{
		g \in \pi(c)
	}
	\Derive{[2]}{\Intro \TYPE{\Subset}}{ \Stab_{\action_{c,p}}(x) \subset \pi(c)}
	\Conclude{[*]}{\Intro \TYPE{SubsetEq}[1][2]}
	{
		\Stab_{\action_{c,p}}(x) = \pi(c) 
	}
	\EndProof
	\\
	\Theorem{FreeIsSimplyConnected}
	{
		\forall c : \TYPE{CoveringMap}(X,B) \.
		\forall p \in B \.
		\TYPE{Free}(\action_{c,p}) \iff
		\TYPE{SimplyConnected}(X)
	}
	\NoProof
	\\
	\Theorem{SimplyConnectedCoveringStructure}
	{
		\forall c : \TYPE{CoveringMap}(X,B) \.
		\TYPE{SimplyConnected}(X) \Imply 
		\forall p \in B \. \NewLine \.  
		\Big| c^{-1}(p) \Big| = \Big| \pi(B) \Big|
	}
	\NoProof
	\\
	\Theorem{SimplyConnectedCoveringBase}
	{
		:: \NewLine 
		\forall c : \TYPE{CoveringMap}(X,B) \.
		\TYPE{SimplyConnected}(B) \Imply 
		\TYPE{Homeomorphism}(X,B,c)
	}
	\NoProof
}
\Page{
	\Theorem{MonodromyConjugacyTHM}
	{
		\NewLine ::
		\forall c : \TYPE{CoveringMap}(X,B) \.
		\forall p \in B \. 
		\forall x' \in X \. 
		\TYPE{Orbit}\bigg(
			\Gamma_{\pi(B)},
			\Big\{ c_* \pi(X,x) \Big| x \in c^{-1}(p)  \Big\}
		\bigg)
	}
	\NoProof
	\\
	\DeclareType{NormalCovering}{? \TYPE{CoveringMap}(X,B)}
	\DefineType{c}{NormalCovering}{
		\forall x \in X \.
		c_* \pi(X,x) \Nrml \pi\Big(B, c(x)\Big)
	}
	\\
	\Theorem{NormalCoveringCharacterization}
	{
		\forall c : \TYPE{Covering}(X) \.
		\forall x \in X \.
		\forall [0] : c^* \pi(X,x) \Nrml \pi( B, c(x) ) \.
		\TYPE{NormalCovering}(c)
	}
	\SayIn{p}{c(x)}{B}
	\AssumeIn{y}{X}
	\SayIn{q}{c(y)}{B}
	\SayIn{\xi}{\bd \TYPE{PathConnected}(X)(x,y)}{\Omega(x,y)}
	\SayIn{\beta}{ c_* \xi }{\Omega(p,q)}
	\Say{[1]}{\THM{ChangeOfBasePoint}(\beta)}
	{
		\TYPE{Isomorphism}(\GRP, \pi(X,x),\pi(X,y), \gamma_{[\xi]} )
	}
	\Say{[2]}{\THM{ChangeOfBasePoint}(\xi)}
	{
		\TYPE{Isomorphism}(\GRP, \pi(B,p),\pi(B,q), \gamma_{[\beta]} )
	}
	\Say{[3]}
	{ 
		\Elim(\gamma)
		\bd \GRP\Big(\pi(X),\pi(B)\Big)(c_*) \Intro(\beta) \Intro(\gamma)
	}
	{
		c_* \gamma_{[\xi]} = \gamma_{[\beta]} c_*
	}
	\Say{[4]}{\Intro \FUNC{mapping}[3]}
	{
		\gamma_{\beta}\Big( c_* \pi(X,x) \Big) = c_* \pi(X,y)
	}
	\Conclude{[y.*]}{\THM{IsomorphismPreservesNormal}}
	{
		c_* \pi(X,y) \Nrml \pi\Big(B, c(y)\Big) 
	}
	\Derive{[*]}{\Intro \TYPE{NormalCove}}
	{
		\TYPE{NormalCover}(X,B,c)
	}
	\EndProof
}
\newpage
\subsection{Category of Coverings}
\Page{
	\DeclareType{CoveringMorphism}{
		\prod B \in \TOP \. \NewLine \. 
		\prod a : \TYPE{CoveringMapping}(X,B) \.
		\prod b : \TYPE{CoveringMapping}(Y,B) \.
		?\TOP(X,Y)
	}
	\DefineType{f}{CoveringMorphism}{f b = a}
	\\
	\DeclareFunc{coveringCategory}{\TOP \to \CAT}
	\DefineNamedFunc{coveringCategory}{B}{\COV(B)}
	{
		\Big( \sum X \in \TOP \.\TYPE{CoveringMap}(X,B), \TYPE{CoveringMorphism}(B), \circ, \id   \Big)
	}
	\\
	\Theorem{coveringMorphismUniqueness}{ 
		\forall (X,a),(Y,b) \in \COV(B) \.
		\forall f,g \in \COV(a,b) \. 
		\forall x \in X \.
		f(x) = g(x) \Imply f = g
	}
	\Say{[1]}{
		\Elim \Cov(a,b) \Intro \TYPE{Lift} 
	}
	{
		\TYPE{Lift}(a,b,f \And g)
	}
	\Conclude{[*]}{
		\THM{UniqueLiftingTHM}[0][1]
	}
	{
		f = g
	}
	\EndProof
	\\
	\Theorem{CoveringMorphismGSetInduction}{ 
		\forall (X,a),(y,b) \in \COV(B) \.
		\forall f \in \COV(a,b) \. 
		\forall p \in B \. \NewLine \. 
		f_{| a^{-1}(p)}  \in \pi(B,p)\hyph\SET\Big( a^{-1}(p), b^{-1}(p) \Big)
	}
	\AssumeIn{x}{a^{-1}(p) }
	\AssumeIn{g}{\pi(B,p)}
	\Say{\Big( \xi, [1] \Big)}
	{
		\bd \TYPE{PathLiftingProperty}(a,g) 
	}
	{    
		\sum \xi : I \to X \.
		g = [a_* \xi]                                           
	}
	\Say{[2]}{ [1] \Elim \COV(a,b)(f) \Elim \Cov(\TOP_*,\GRP ,\pi) \Elim(f_*)  }
	{
		g = [a_* \xi] =
		\Big[ (fb_*) (\xi)  \Big]= 
		[ f_* b_* \xi ] = 
		\Big[ b_*  f(\xi) \Big] 
	}
	\Conclude{[g.*]}{ \Elim \action_{a,p} \Intro \FUNC{compositon} \Intro \action_{b,p}    }
	{
		f( x g  ) =  
		f\big( \gamma(1) \big) =
		\gamma f (1) =
		f(x) g
	}
	\DeriveConclude{[*]}{\Intro \pi(B,p)\hyph\SET\Big( a^{-1}(p),b^{-1}(p)\Big)}
	{
		f_{| a^{-1}(p)}  \in \pi(B,p)\hyph\SET\Big( a^{-1}(p), b^{-1}(p) \Big)
	}
	\EndProof
}\Page{
	\Theorem{CoveringMorphismIsCoveringMap}
	{
		\forall (a,X), (b,Y) \in \COV(B) \.
		\forall f \in \COV(B)\Big( (a,X), (b,Y) \Big) \.
		f : \TYPE{CoveringMap}(X,Y)
	}
	\Say{[1]}{\Elim \TYPE{CoveringMorphism}(f)}
	{
		fb = a
	}
	\AssumeIn{y}{Y}
	\SayIn{p}{b(y)}{B}
	\Say{[2]}{\Elim \TYPE{Surjective}(a,p)}
	{
		a^{-1}(p) \neq \emptyset
	}
	\Say{[3]}{\THM{CoveringMorphismGSetInduction}(a,b,f)}
	{
		f_{|a^{-1}(p)} \in \pi(B,p)\hyph\SET\Big(a^{-1}(p),b^{-1}(p)\Big)
	}
	\Say{[4]}{\THM{TransitiveGMApIsSurjective}[3]}
	{
		\TYPE{Surjective}(a^{-1}(p),b^{-1}(p),f_{|a^{-1}(p)})
	}
	\Conclude{[y.*]}{\Elim \FUNC{specification}\Elim \TYPE{Surjection}[4]}
	{
		y \in \im f_{|a^{-1}(p)} \subset f_{|b^{-1}(p)}
	}
	\Derive{[2]}{\Intro \TYPE{Surjective}}{\TYPE{Surjective}(X,Y,a)}
	\AssumeIn{y}{Y}
	\SayIn{p}{b(y)}{B}
	\Say{\Big(U',[2]\Big)}{\Elim \TYPE{CoveringMap}(X,B,a)(p)}
	{
		\sum U : \TYPE{EvenlyCovered}(X,B,a) \. 
		p \in U
	}
	\Say{\Big(U'',[3]\Big)}{\Elim \TYPE{CoveringMap}(Y,B,b)(p)}
	{
		\sum U : \TYPE{EvenlyCovered}(Y,B,b) \. 
		p \in U
	}
	\SayIn{U'''}{U'\cap U''}{\U(p)}
	\SayIn{U}{\Elim \TYPE{LocallyPathConnected}(B) \THM{PathConnectedIsConnected}}{\mathrm{PCC}(U''') \cap \U(p)}
	\Say{\Big( V, [4] \Big)}
	{
		\Elim \TYPE{EvenlyCovered}(X,B,b,U) 
		\Elim \FUNC{disjointUnion} [1]
		\Elim p
	}
	{
		\sum V  \in \U(y) \And \TYPE{StronglyConnected} 
		\. \NewLine \.  
		\TYPE{Homeomprphism}(b_{|V},V, U)  
	}
	\Say{[5]}{ \Elim \TOP(X,Y)(f) } 
	{
		 \TYPE{Clopen}\Big(b^{-1}(U),f^{-1}(V)\Big)
	}
	\Say{[6]}{\THM{CompositionPreimage}[1][5] }
	{
		\TYPE{Clopen}\Big( a^{-1}(U), f^{-1}(V)\Big)
	}
	\Conclude{[\I,W,[y.*]]}{\Elim \TYPE{EvenlyCloded}(U'')\Elim U [6][1] }
	{
		\NewLine :
		\sum \I \in \SET \. 
		\sum W : \I \to \T(X) \And \TYPE{StronglyConnected} \. 
		f^{-1}(V) = \bigsqcup_{i \in \I} W_i 
		\forall i \in \I \. \NewLine \. 
		\TYPE{Homeomorphis}( W_i, V, f_{|W_i} )
	}
	\DeriveConclude{[[*]}{\Intro \TYPE{CoveringMap}}
	{
		\TYPE{CoveringMap}(X,Y,f)
	}
	\EndProof
	\\
	\Theorem{CoveringMorphisimCriterion}
	{
		\forall (a,X),(b,Y) \in \COV(B) \.
		\forall x \in X \. 
		\forall y \in Y \.
		\forall [0] : a(x) = b(y) \. \NewLine \. 
		\exists f \in \COV(B)(a,b) : f(x) = f(y) 
		\iff
		a_* \pi(X,x) \subset b_* \pi(Y,y)
	}
	\NoProof
	\\
	\Theorem{CoveringIsomorphisCriterion}
	{
		\forall (a,X),(b,Y) \in \COV(B) \.
		\forall x \in X \. 
		\forall y \in Y \.
		\forall [0] : a(x) = b(y) \. \NewLine \.
		\exists f:\TYPE{Isomorphism}\Big( \COV(B), a, b \Big) :  f(x) = f(y)
		\iff
		a_* \pi(X,x) = b_* \pi(Y,y)
	}
	\NoProof
}
\newpage
\subsection{The Universal Covering Space}
\Page{
	\DeclareType{UniversalCover}{\prod B \in \TOP \. ?\COV(B)  }
	\DefineType{(Z,z)}{UniversalCover}{
		\forall (X,c) \in \COV(B) \. 
		\exists (Z,z) \Arrow{f} (X,c) : \COV(B)
	}
	\\
	\Theorem{SimplyConnectedIsUniversalCover}
	{
		\forall B \in \TOP \.
		\forall ( Z, z) \in \COV(B) \. \NewLine \. 
		\TYPE{SimplyConnected}(Z) \Imply
		\TYPE{UniversalCover}(z,Z)
	}
	\Say{[1]}{\Elim \TYPE{SimplyConnected} \Intro \pi(Z)}
	{
		\pi(Z) = \star
	}
	\Say{[2]}{\Elim \FUNC{image}(z_*) [1] \Intro \pi(z)}
	{ 
		\pi(z) = z_* \pi(Z) = z_* \star = \star
	}
	\AssumeIn{(X,c)}{\COV(B)}
	\Say{[3]}{\THM{TrivialSubgroup}\Big( \pi(c) \Big)}{ \star \subset \pi(c) }
	\Conclude{\Big[(X,c).*\Big]}
	{
		\THM{CoveringMorphismCriterion}\Big((Z,z),(X,c)\Big)[2][3] 
	}
	{
		\exists f : (z,Z) \Arrow{f} (X,c) : \COV(B) 
	}
	\DeriveConclude{[*]}{\Intro \TYPE{UniversalCover}}
	{
		\TYPE{UniversalCover}\Big( B, (Z,z) \Big)
	}
	\\
	\Theorem{LocallyConnectedCoversAreIsomorphic}
	{
		\forall B \in \TOP \.
		\forall (X,x),(Y,y) \in \COV(B) \. \NewLine \. 
		\TYPE{SimplyConnected}(X \And Y) 
		\Imply
		(X,x) \cong_{\COV(B)} (Y,y)
	}
	\NoProof
	\\
	\Conclude{\TYPE{Reasonable}}{ \TYPE{Connected} \And \TYPE{Locally}\;\TYPE{SimplyConnected}}{\Type}
	\\
	\Theorem{UniversalCoverExists}
	{
		\forall B : \TYPE{Reasonable} \.  
		\exists (Z,z) \in \COV(B) :
		\TYPE{SimplyConnected}(Z)
	}
	\SayIn{p}{\Elim \TYPE{NonEmpty}}{B}
	\SayIn{Z}{\Big\{ \FUNC{pathClass}(\gamma)  \Big| I \Arrow{\gamma} B : \TOP, \gamma(0) = x  \Big\}}{\SET}
	\Say{z}{\Lambda [\gamma] \in Z \. \gamma(1) }{ Z \to B }
	\Say{V}{
		\Lambda [\gamma] \in Z \. 
		\Lambda U \in \U(z) \. 
		\Big\{
			[\gamma \circ \omega] 
			\Big|
			\I \Arrow{\gamma} B : \TOP,
			\omega(0) = \gamma(1)
		\Big\}
	}
	{
		\prod_{[\gamma] \in Z} \U(\gamma(1)) \to ?Z 
	}
	\Say{\B}{ \Big\{ V_{[\gamma],U} \Big| [\gamma] \in Z, U \in \U\big( \gamma(1) \big)  \Big\}  }
	{
		??Z
	}
	\Say{[1]}{\Elim \B \Elim \TYPE{PathComposition} \Intro \FUNC{Cover}}{\TYPE{Cover}(Z,\B)}
	\AssumeIn{v,v'}{\B}
	\Assume{[2]}{v \cap v' \neq \emptyset}
	\Say{[\alpha],[\beta],U,U',[3]}{\Elim \B(v,v')}
	{
		\sum [\alpha],[\beta] \in Z \.
		\sum U \in \U\big( \alpha(0) \big) \.
		\sum U' \in \U\big( \beta(0) \big) \.
		v = V_{[\alpha],U} \And v' = V_{[\beta],U'}
	}
	\Say{[4]}{
		\ByConstr V [2][3]
	}
	{
		[\alpha] = [\beta] 
	}
	\Conclude{\Big[(v,v').*\Big]}{ \Elim \FUNC{intersecion} [4]  \Intro  V \Elim \mathcal{B}   }
	{
		v \cap v' = V_{[\alpha], U \cap U'} \in \mathcal{B}
	}
	\Derive{[2]}{\Intro \TYPE{Base}(Z)}
	{
		\TYPE{Base}(Z,\B)
	}
	\SayIn{Z}{(Z,\FUNC{topology}(\B))}{\TOP}
	\Say{[3]}{\Elim \TYPE{SimplyConnected}(B) \Elim V \Intro \TYPE{SimplyConnected}(Z)   }
	{
		\forall [\gamma] \in Z \. 
		\forall U \in \U\big(\gamma(0)\big) \And \TYPE{SimplyConnected} \. \NewLine \. 
		\TYPE{SimplyConnected}( V_{[\gamma],U}   )
	}
	\Say{[4]}{\Elim \TYPE{Reasonable}(B)[2]}
	{
		\forall [\gamma] \in Z \.
		\exists V \in \U\big([\gamma]\big) :  \TYPE{SimplyConnected}(V) 
	}
}\Page{
	\Say{[5]}{
		\Elim \TYPE{Locally}\;\TYPE{SimplyConnected}[4] \Intro \TYPE{Locally} \; \TYPE{PathConnected}          
	}
	{
		\TYPE{Locally}\;\TYPE{PathConnected}(Z) 
	}
	\Say{[6]}{
		\Elim \TYPE{Connected}(B) \ByConstr Z
	}{
		\forall [\gamma] \in Z \. \Omega\Big([\gamma],p\Big) \neq \emptyset
	}
	\Say{[7]}{\Intro\TYPE{PathConnected}[6]}
	{
		\TYPE{PathConnected}(Z)
	}
	\Say{[8]}{\THM{PathConnectedIsConnected}[7]}{\TYPE{Connected}(Z)}
	\Assume{U}{\T(B)}
	\Say{E}{\Big\{ [\gamma] \in Z : \gamma(1) \in U \Big\}}{?Z	}
	\Conclude{[U.*]}
	{
		\Elim z \Intro V_{e,U} \Elim \T(Z)
	}
	{
		z^{-1}(U) = \bigcup_{e \in E} V_{e,U}  \in \T(Z)  
	}
	\Derive{[9]}{ \Intro \TOP  }
	{ 
		Z \Arrow{z} B : \TOP
	}
	\Assume{U}{\T(B) \And \TYPE{SimplyConnected}}
	\Say{E}{\Big\{ [\gamma] \in Z : \gamma(1) \in U \Big\}}{?Z	}
	\Say{[10]}
	{
		\Elim z \Intro V_{e,U} 
	}
	{
		z^{-1}(U) = \bigcup_{e \in E} V_{e,U} 
	}
	\Say{[11]}
	{
		\Elim  V_{e,U} [11] \Intro \TYPE{DisjointUnion}
	}
	{
		z^{-1}(U) = \bigsqcup_{e \in E} V_{e, U}
	}
	\AssumeIn{e}{E}
	\Say{[12]}{\Elim(z) \Intro \TYPE{Bijective}}{\TYPE{Surjective}(V_{e,U},U,z_{|V_{e,U}})}
	\AssumeIn{[\gamma],[\gamma']}{V_{e,U}}
	\Assume{[13]}{z[\gamma] = z[\gamma']}
	\Say{[14]}{\Elim z [13]}{ \gamma(1) = \gamma'(1)  }
	\Say{\Big[\alpha,[15]\Big]}{\Elim V_{e,U}[\gamma]}
	{
		\sum \alpha \in \Omega\big( e(1),\gamma(1) \big) \.
		\gamma \sim e \circ \alpha
	}
	\Say{\Big[\beta,[16]\Big]}{\Elim V_{e,U}[\gamma]}
	{
		\sum \beta \in \Omega\big(e(1),\gamma(1)\big) \.
		\gamma \sim e \circ \beta
	}
	\Conclude{\Big[\big([\gamma],[\gamma']\big).*\Big]}{\ByConstr \TYPE{SimplyConnected}[15][16]}
	{
		[\gamma] = [\gamma']
	}
	\Derive{[12]}{\Intro \TYPE{Injective}}
	{
		\TYPE{Injective}(V_{e,U},U,z_{|V_{e,U}})
	}
	\Assume{W}{\T(V_{e,U})}
	\Say{(\O,[13])}{\Elim W}
	{
		\sum \O \in \prod_{f \in V_{e,U}} ? \U(f(1)) \. 
		W = 
		\bigcup_{f \in W} 
		\bigcup_{O \in \O_f} 
		V_{f,O_f}
	}
	\Conclude{[W.*]}{\Elim z [13]}
	{
		z(W) = 
		\bigcup_{f \in W}
		\bigcup_{O \in \O_f} 
		O
	}
	\Derive{[14]}{ \Intro \TYPE{OpenMap}  }
	{
		\TYPE{OpenMap}\Big(V_{e,U},U,z_{V_{e,U}}\Big)
	}
	\Conclude{[U.*]}{\Intro \TYPE{Homeomophism}[12][13][14]}
	{
		\TYPE{Homeomorphism}\Big( V_{e,U},U,z_{V_{e,U}}\Big)
	}
	\Derive{[10]}{\Elim \TYPE{Reasonable}(B) \Intro \COV(B)}
	{
		(Z,z) \in \COV(B)
	}
	\AssumeIn{\Gamma}{\Omega_Z\big([p]\big)}
	\Say{\gamma}{\Gamma z}{\Omega_B(p)}
	\Say{H}{\Lambda t \in I \. \Lambda s \in \I \. \gamma(st)}{ \I^2 \to B  }
	\Say{[11]}{\Elim z \Elim H}
	{
		\forall t \in I \.
		[H_t] z = H_t(1) = \gamma(t)
	}
	\Say{[12]}{\Intro \TYPE{Lift} \Elim \gamma [11]}
	{
		\TYPE{Lift}(I,B,\gamma, \tilde \Gamma_x \And H)
	}
	\Say{[13]}{\THM{UniqueLiftProperty}[12]}
	{
		\Gamma = [H]
	}
	\Conclude{[\Gamma.*]}{\ByConstr \Omega_Z\Big([x]\Big)}
	{
		\Gamma = [p]
	}
	\DeriveConclude{[*]}{\bd^{-1}\TYPE{SimplyConnected}}
	{
		\TYPE{SimplyConnected}(Z)
	}
	\EndProof
}
\newpage
\subsection{Borsuk-Ulam Theory}
\Page{
	\DeclareType{OddFunction}{\End_{\TOP}(\Sphere^1)?}
	\DefineType{f}{OddFunction}{\forall s \in \Sphere_1 \. f(-s) = -f(s)}
	\\
	\DeclareType{EvenFunction}{\End_{\TOP}(\Sphere^1)?}
	\DefineType{f}{EvenFunction}{\forall s \in \Sphere_1 \. f(-s) = f(s)}
	\\
	\Theorem{OddSquareCommuter}{
		\forall f : \TYPE{OddFunction} \.
		\exists g : \End_{\TOP}(\Sphere^1) \.
		\deg f = \deg g \.
		f^2 = \Lambda z \in \Sphere^1 \. g(z^2)
	}
	\Say{ \Big( r, [1]\Big)  }{ \THM{ComplexRootCovers} }
	{
		\sum r : \Reals \Arrow{\TOP} \Reals \. 
		\forall t \in \Reals \. 
		rs^2(t)   = s(t)
	}
	\Say{g}{\Lambda z \in \Sphere_1 \. f^2\bigg(\exp\Big(\mathrm{i}r\Big( \mathrm{Arg}\; z\Big) \bigg) }
	{
		\Sphere^1 \to ?\Sphere^1
	}
	\Assume{z}{\Sphere^1}
	\Say{\Big( t,[2] \Big)}{\Elim \mathrm{Arg}(z)}
	{ \sum t \in [0, 2\uppi) \. \mathrm{Arg}\;z = \{ t + 2\uppi n | n \in \Int \} }
	\Say{[3]}
	{
		\Elim g 
		[2] \Elim r 
		\THM{ExpHomo}(\Complex)
		\Elim \TYPE{OddFunction}(f)
		\THM{SignSquare}
	}
	{
		g(z) = 	
		f^2\bigg(\exp\Big(\mathrm{i}r\Big( \mathrm{Arg}\; z\Big) \bigg) = \NewLine = 
		\left\{ f^2\left(\exp\left( \frac{\mathrm{i}t}{2} + n \uppi \right)\right)  |n \in \Int\right\} =
		\left\{ f^2\left(\pm\exp\left( \frac{\mathrm{i}t}{2}  \right)\right)  \right\}  =
		\left\{ (\pm)^2 f^2\left(\exp\left( \frac{\mathrm{i}t}{2}  \right)\right)  \right\}  =
		\left\{  f^2 \left(\exp\left( \frac{\mathrm{i}t}{t} \right) \right) \right\}
	}
	\Conclude{(z.*)}{\Intro \TYPE{Singleton}[3]}{\TYPE{Singleton}\Big(g(z)\Big)}
	\Derive{[2]}{\LOGIC{WellDefine}}{g \in \End_{\TOP}(\Sphere^1)}
	\Say{[3]}{\Elim g \Elim r}{ \forall z \in \Sphere^1 \.  f^2(z) =  g(z^2)  } 
	\Say{[4]}{\THM{DegreeCompostition}[3]\THM{DegreeComposition}}
	{
		2 \deg f = \deg f^2 = \deg g(\bullet^2) = 2 \deg g
	}
	\Conclude{[5]}{\Elim \IsNot \TYPE{ZeroDivisor}(\Int,2)}{\deg f = \deg g}
	\EndProof
	\\
	\DeclareType{SquareCommuter}
	{
		\End_{\TOP}(\Sphere^1) \to  ?\End_{\TOP}(\Sphere^1)
	}
	\DefineType{g}{SquareCommuter}
	{
		\Lambda f \in \End_{\TOP}(\Sphere^1) \.
		\deg f = \deg g \And f^2 = g(\bullet^2)
	}
	\\
	\DeclareFunc{oddSquareCommuter}
	{
		\prod f : \TYPE{OddFunction} \. \TYPE{SquareCommutor}(f)
	}
	\DefineNamedFunc{oddSquareCommuter}{}{g_f}
	{
		\THM{OddSquareCommuter}(f)
	}
	\\
	\Theorem{EvenDegreeLifting}
	{
		\forall f : \TYPE{OddFunction} \.
		\forall [0] : \TYPE{Even}(\deg f) \.
		\exists \tilde g : \TYPE{Lift}(\bullet^2,g_f)
	}
	\Say{ [1] }{\Elim \TYPE{Even} \Elim \deg f [0]}
	{
		f_* \pi(\Sphere_1) \subset 2 \Int
	}
	\Say{[2]}{\THM{PowerDegree}(2)[1]}
	{
		f^2_* \pi(\Sphere_1) \subset 4 \Int
	}
	\Say{[3]}{\Elim \TYPE{SquaeCommuter}(f,g_f)[2]}
	{
		(\bullet^2g_{f})_* \pi(\Sphere_1) \subset 4 \Int
	}
	\Say{[4]}{\THM{PowerDegree}(2)[3]}
	{ 
		g_{f*} \pi(\Sphere_1) \subset 2 \Int
	}
	\Conclude{[*]}{\THM{LiftingCriterion}(\Sphere^1,\Sphere^1,\bullet^2,g_f)[4]}
	{
		\exists \tilde g : \TYPE{Lift}(\bullet^2,g_f)
	}
	\EndProof
}
\Page{
	\Theorem{OddFunctionsHasOddDegree}
	{
		\forall f : \TYPE{OddFunction} \.
		\TYPE{Odd}(\deg f)
	}
	\Assume{[1]}{\TYPE{Even}(\deg f)}
	\Say{\tilde g}{\THM{EvenDegreeLifting}}
	{
		\TYPE{Lift}(\bullet^2,g_f)
	}
	\Say{[2]}{\Elim \TYPE{Lift}(\bullet^2,g_f)}
	{
		(\tilde g)^2 = g_f
	}
	\Say{[3]}{ \Elim(=)\Big( \bullet^2 g_f, [2] \Big)}
	{
		(\bullet^2 \tilde g)^2 = \bullet^2 g_f
	}
	\Say{[4]}{\Intro \TYPE{Lift}[3]}
	{
		\TYPE{Lift}(\bullet^2, \bullet^2 \tilde g, \bullet^2 g_f)
	}
	\Say{[5]}{\Elim \TYPE{SquareCommetor}(f,g_f)}
	{
		f^2 = \bullet^2 g_f
	}
	\Say{[6]}{\Intro \TYPE{Lift}[3]}
	{
		\TYPE{Lift}(\bullet^2, f, \bullet^2 g_f)
	}
	\Say{[7]}{\Elim \TYPE{OddFunction}(f)[5][3]}{f(-1) = \tilde g(1) | f(1) = \tilde g(1)}
	\Say{[8]}{\THM{UniqueLiftingProperty}[4][6][7]}{f = \bullet^2 \tilde g}
	\Say{[9]}{\Intro \TYPE{EvenFunction} \Elim \bullet^2 [8]}{\TYPE{EvenFunction}(f)}
	\Conclude{[1.*]}{\Elim \TYPE{OddFuncyion}(f) \Elim \TYPE{EvenFunction}(f)}{\bot}
	\DeriveConclude{[*]}{\Elim(\bot)\Elim (|) \THM{OddOrEven}}{\TYPE{Even}(\deg f)}
	\EndProof
	\\
	\Theorem{EvenFunctionsHaveEvenDegrees}
	{
		\forall f : \TYPE{EvenFunction} \.
		\TYPE{Even}(f)
	}
	\Say{ \Big( r, [1]\Big)  }{ \THM{ComplexRootCovers} }
	{
		\sum r : \Reals \Arrow{\TOP} \Reals \. 
		\forall t \in \Reals \. 
		rs^2(t)   = s(t)
	}
	\Say{g}{\Lambda z \in \Sphere_1 \. f\bigg(\exp\Big(\mathrm{i}r\Big( \mathrm{Arg}\; z\Big) \bigg) }
	{
		\Sphere^1 \to ?\Sphere^1
	}
	\Assume{z}{\Sphere^1}
	\Say{\Big( t,[2] \Big)}{\Elim \mathrm{Arg}(z)}
	{ \sum t \in [0, 2\uppi) \. \mathrm{Arg}\;z = \{ t + 2\uppi n | n \in \Int \} }
	\Say{[3]}
	{
		\Elim g 
		[2] \Elim r 
		\THM{ExpHomo}(\Complex)
		\Elim \TYPE{OddFunction}(f)
		\THM{SignSquare}
	}
	{
		g(z) = 	
		f\bigg(\exp\Big(\mathrm{i}r\Big( \mathrm{Arg}\; z\Big) \bigg) = \NewLine = 
		\left\{ f\left(\exp\left( \frac{\mathrm{i}t}{2} + n \uppi \right)\right)  |n \in \Int\right\} =
		\left\{ f\left(\pm\exp\left( \frac{\mathrm{i}t}{2}  \right)\right)  \right\}  =
		\left\{ f\left(\exp\left( \frac{\mathrm{i}t}{t} \right) \right) \right\}
	}
	\Conclude{(z.*)}{\Intro \TYPE{Singleton}[3]}{\TYPE{Singleton}\Big(g(z)\Big)}
	\Derive{[2]}{\LOGIC{WellDefine}}{g \in \End_{\TOP}(\Sphere^1)}
	\Say{[3]}{\Elim g \Elim r}{ \forall z \in \Sphere^1 \.  f(z) =  g(z^2)  } 
	\Say{[4]}{\THM{DegreeCompostition}[3]\THM{DegreeComposition}}
	{
		\deg f  = \deg g(\bullet^2) = 2 \deg g
	}
	\Conclude{[*]}{\Intro \TYPE{Even}[4]}
	{
		\TYPE{Even}( \deg f )
	}
	\EndProof
}
\Page{
	\Theorem{BorsukUlamTheorem}
	{
		\forall \Sphere^2 \Arrow{F} \Reals^2 \.
		\exists v \in \Sphere^2 \. F(v) = F(-v)
	}
	\Assume{[1]}{\forall v \in \Sphere^2 \. F(v) \neq F(-v)  }
	\Say{f'}{
		\Lambda v \in  \Sphere^2 \. \frac{F(v) - F(-v)}{\| F(v) - F(-v)\|}
	}
	{
		\TOP(\Sphere^2,\Sphere^1)
	}
	\Assume{L}{\TYPE{VectorPlane}(\Reals^3)}
	\Say{f_L}{f'_{|L \cap \Sphere^2}}{\End_{\TOP}(\Sphere^1)}
	\Say{f_L}{\Elim f \Intro \TYPE{OddFunction}}{\TYPE{OddFunction}(f)} 
	\Say{[2]}{\THM{OddFunctionHasOddDegree}(f)}{\TYPE{Odd}(\deg f) }
	\Say{H}{\THM{HigherSphereIsSimplyConnected}}{\TYPE{Homotopy}( L \cap \Sphere^2, \star  )}
	\Say{[3]}{\THM{CPreservesHomotopy}(f,H)}{\TYPE{Homotopy}(f,f(\star),Hf)}
	\Say{[4]}{\THM{HomotopyPreserrvesDegree}[3]}{\deg f = 0}
	\Conclude{[1.*]}{\Elim(\deg)[4][2]\Intro(\bot)}{\bot}
	\DeriveConclude{[*]}{\Elim \bot}{\exists v \in \Sphere^2 \. F(v) = F(-v)}
	\EndProof
	\\
	\Theorem{HamSandwichTHM}
	{
		\forall U_1,U_2,U_3 : \TYPE{Open} \And \TYPE{ConnectedSubset}(\Reals_3) \.
		\exists H : \TYPE{Hyperplane}(\Reals^3) :
		\forall i \in \{1,2,3\} \.
		\NewLine
		\mathrm{Vol}(U_i \cap H_-) = \mathrm{Vol}(U_i \cap H_+)
	}
	\AssumeIn{p}{\Sphere_2}
	\Assume{H}{\TYPE{HyperplenThrough}(p)}
	\Say{[1]}{\Elim \mathrm{Vol}}{ \mathrm{Vol}(U_3) = \mathrm{Vol}(U_3 \cap H_-) + \mathrm{Vol}(U_2 \cap H_+)  }
	\Say{[2]}{\Elim \FUNC{reverseHyperplane}}{ \mathrm{Vol}(U_3 \cap H_-) = \mathrm{Vol}(U_3 \cap -H_+)  }
	\Conclude{[H.*]}{[1][2]\Intro \exists}{ \NewLine 
		\mathrm{Vol}(U_3 \cap H_-) \le \frac{1}{2}\mathrm{Vol}(U_3) 
		\Imply 
		\exists H' : \TYPE{HyperplaneThrough}(p) :
		\mathrm{Vol}(U_3 \cap H'_-) \ge \frac{1}{2}\mathrm{Vol}(U_3) 
		\And \NewLine \And
		\mathrm{Vol}(U_3 \cap H_-) \ge \frac{1}{2}\mathrm{Vol}(U_3) 
		\Imply 
		\exists H' : \TYPE{HyperplaneThrough}(p) :
		\mathrm{Vol}(U_3 \cap H'_-) \le \frac{1}{2}\mathrm{Vol}(U_3) 	
	}
	\DeriveConclude{\Big(H_p,[1]\Big)}{\THM{IntermidiateVlaueTheirem}}{
		\NewLine
		\sum H^p : \TYPE{HyperplaneThrough}(p) \. 
		\mathrm{Vol}(U_3 \cap H_-^p) = \mathrm{Vol}(U_3 \cap H_+^p)
	}
	\Derive{H}{\Intro\Act{\prod}}
	{
		\prod p \in \Sphere^2 \. \sum H^p \TYPE{HyperplaneThrough}(p) \.
		\mathrm{Vol}(U_3 \cap H_-^p) = \mathrm{Vol}(U_3 \cap H_+^p)
	}
	\Say{F}{\Lambda p \in \Sphere^2 \. (\mathrm{Vol}(U_1 \cap H^p_+),\mathrm{Vol}(U_3\cap H^p_+))}
	{
		\TOP(\Sphere^2,\Reals^2)
	}
	\Say{\Big(p,[1]\Big)}{\THM{BorsukUlamTHM}}
	{
		\sum p \in \Sphere^2 \.
		F(p) = F(-p)
	}
	\Conclude{[*]}{\Elim F \Elim H \Elim \mathrm{Vol}}
	{
		\forall i \in \{1,2,3\} \. 
		\mathrm{Vol}(U_i \cap H^p_+) = \mathrm{Vol}(U_i \cap H^p_+)
	}
	\EndProof
}
\newpage
\subsection{Galois Covering Theory}
\Page{
	\DeclareFunc{deckTransformationGroup}
	{
		\prod B \in \TOP \. \COV(B) \to \GRP
	}
	\DefineNamedFunc{deckTransforamtionGroup}{X \Arrow{c} B}
	{
		\Gal(X \Arrow{c} B)
	}
	{
		\Aut_{\COV(B)}(X,B,c)
	}
	\\
	\Theorem{DeckTransformationMonodromy}
	{
		\forall (X \Arrow{c} B) \in \COV(B) \.
		\forall  f,g \in \Gal(X \Arrow{c} B ) \.
		\forall x \in X \. \NewLine \.
		f(x) = g(x) \Imply f = g
	}
	\NoProof
	\\
	\Theorem{DeckTransformationIsGSetIso}
	{
		\forall (X \Arrow{c} B) \in \COV(B) \. 
		\forall f \in \Gal(X \Arrow{c} B) \.
		\forall p \in B \. \NewLine \. 
		f_{|c^{-1}(x)}  \in \Aut_{\pi(B,p)\hyph\SET}\Big(c^{-1}(x)\Big)   
	}
	\NoProof
	\\
	\Theorem{DeckTransformationActsFreely}
	{
		\forall (X \Arrow{c} B) \in \COV(B) \.
		\TYPE{Free} \And\TYPE{HomeoAction}\Big(X, \Gal(X \Arrow{c} B) \Big)
	}
	\NoProof
	\\
	\Theorem{DeckTransformationOrbitCriterion}
	{
		\forall (X \Arrow{c} B) \in \COV(B) \.
		\forall p \in B \. 
		\forall x,y \in f^{-1}(p) \. \NewLine \. 
		\exists f : \Gal(X \Arrow{c} B) : f(x) = y 
		\iff
		c_* (X,x) = c_* (X, y)
	}
	\NoProof
	\\
	\Theorem{NormalCoveringHasTransitiveGal}
	{
		\NewLine ::
		\forall (X \Arrow{c} B) \in \COV(B) \.
		\bigg( \forall p \in B \. \TYPE{Transitive}\Big( c^{-1}(p) , \Gal(X \Arrow{c} B)  \Big)  \bigg)
		\iff
		\TYPE{NormalCovering}(X \Arrow{c} B)
	}
	\NoProof
}\Page{
	\Theorem{DeckTransformationIso}
	{
		\forall (X \Arrow{c} B) \in \COV(B) \.
		\forall p \in B \. 
		\Gal(X \Arrow{c} B) \cong_{\GRP} \Aut_{\pi(B,p)\hyph\SET} f^{-1}(p)
	}
	\SayIn{\varphi}{\Lambda f \in \Gal(X \Arrow{c} B) \. f_{|f^{-1}(p)}}
	{
		\Gal(X \Arrow{c} B) \to \Aut_{\pi(B,p)\hyph\SET} f^{-1}(p)
	}
	\Say{[1]}{\Elim \Gal(X \Arrow{c} B) \Elim \Aut_{\pi(B,p)\hyph\SET} \Elim \varphi \Intro \TYPE{Homo}}
	{
		\GRP\Big( \Gal(X \Arrow{c} B), \End_{\pi(B,p)\hyph\SET} f^{-1}(p), \varphi \Big)
	}
	\Say{[2]}{\THM{DeckTransformationIso}(X \Arrow{c} B) \Elim \varphi \Intro \TYPE{Injective}}
	{
		\TYPE{Injective}\Big( \Gal(X \Arrow{c} B), \End_{\pi(B,p)\hyph\SET} f^{-1}(p), \varphi \Big)
	}
	\AssumeIn{\sigma}{\End_{\pi(B,p)\hyph\SET} f^{-1}(p)}
	\Conclude{[4]}{\THM{GSetIsomorphismExistance}(\sigma)}
	{
		\forall x \in c^{-1}(p) \.
		\Stab(x) = \Stab \; \sigma(x)
	}
	\Say{[5]}{\THM{StabilizerOfMonodromyGroup}[4]}
	{
		\forall x \in c^{-1}(p) \.
		c_* \pi(X,x) = c_* \pi\Big(X, \sigma(x) \Big)
	}
	\AssumeIn{x}{X}
	\Say{\Big(f,[6] \Big)}
	{
		\THM{DeckTransformationOrbitCriterion}[5](x)
	}
	{
		\sum f \in \Gal(X \Arrow{c} B) \. f(x) = \sigma(x)
	}
	\Say{[7]}{\Elim(\varphi)[6]}{\varphi(f)(x) = \sigma(x)}
	\Conclude{[x.*]}{
		\THM{MonodromyActionIsTransitive}(X \Arrow{c} B) 
		\NewLine
		\THM{GMapsBetweenTransitiveAreDeterminedByOnePoint}(\pi(B,p)}
	{
		\varphi(f) = \sigma
	}
	\DeriveConclude{[\sigma.*]}{\Elim \TYPE{NonEmpty}(X)}{\varphi(f) = \sigma}
	\Say{[3]}{\Intro\TYPE{Surjective}}
	{
		\TYPE{Surjective}\Big( \Gal(X \Arrow{c} B), \End_{\pi(B,p)\hyph\SET} f^{-1}(p), \varphi \Big)
	}
	\Conclude{[*]}{\Intro \TYPE{Isomorphism}[1,2,3]}
	{
		\TYPE{Isomorphism}\Big( \GRP, \Gal(X \Arrow{c} B), \End_{\pi(B,p)\hyph\SET} f^{-1}(p), \varphi \Big)	
	}
	\EndProof
	\\
	\Theorem{DeckStructuralMorphismExists}
	{
		\forall (X\Arrow{c} B) : \COV(B) \.
		\forall p \in B \.
		\forall x \in c^{-1}(x) \.
		\forall \gamma \in N(c_* \pi(X,x)) \. 
		\exists! f \in \Gal(X \Arrow{c} B) :
		f(x) = x\gamma
	}
	\Say{[1]}{ \Elim \action_{c,p}(\gamma) \Elim N  \Intro \pi}
	{
		c_* \pi(X,x) = c_* \pi(X,x\gamma)
	}
	\Conclude{[*]}{\Elim \THM{DeckTransformationOrbitCriterion}[3]}
	{
		\exists f \in \Gal(X \Arrow{c} B) :
		f(x) = x\gamma	
	}
	\EndProof
	\\
	\DeclareFunc{deckStructuralMorphism}
	{
		\prod (X \Arrow{c} B) : \COV(B) \.
		\prod_{ p \in B}
		\prod_{x \in f^{-1}(x)}
		N(c_* \pi(X,x)) \Arrow{\GRP} \Gal(X \Arrow{c} B)
	}
	\DefineNamedFunc{deckStructuralMorphism}{\gamma}{\Delta_\gamma^{c,p,x}}{\THM{DeckStrucuralMorphismExists}}
	\\
	\Theorem{DeckStructuralMorphismIsSurjective}
	{
		\forall (X\Arrow{c} B) : \COV(B) \.
		\forall p \in B \.
		\forall x \in c^{-1}(x) \. \NewLine \. 
		\Delta^{c,p,x} : \TYPE{Surjective}\Big( N(c_* \pi(X,x), \Gal(X \Arrow{c} B)   \Big)	
	}
	\NoProof
	\\
	\Theorem{DeckGroupStructure}
	{
		\forall (X\Arrow{c} B) : \COV(B) \.
		\forall p \in B \.
		\forall x \in c^{-1}(x) \.
		\Gal(X \Arrow{c} B)  \cong_{\GRP} \frac{N\Big(c_* \pi(X,x)\Big)}{c_* \pi(X,x)}
	}
	\NoProof
}
\Page{
	\Theorem{NormalDeckGroupStructure}
	{
		\forall (X\Arrow{c} B) : \TYPE{NormalCover} \.
		\forall p \in B \.
		\forall x \in c^{-1}(x) \. \NewLine \.
		\Gal(X \Arrow{c} B)  \cong_{\GRP} \frac{\pi(B,p)}{c_* \pi(X,x)}
	}
	\NoProof
	\\	
	\Theorem{SimplyConnectedGroupStructure}
	{
		\forall (X\Arrow{c} B) : \COV(B) \.
		\forall p \in B \. \NewLine \. 
		\TYPE{SimplyConnected}(X) \Imply
		\Gal(X \Arrow{c} B)  \cong_{\GRP} \pi(B,p)
	}
	\NoProof
	\\
	\DeclareType{CoveringAction}
	{
		\prod X \in \TOP \.
		\prod G \in \GRP \.
		?(X \action_\TOP G)
	}
	\DefineType{(\cdot)}{CoveringAction}
	{
		\forall x \in X \.
		\exists U \in \U(x) : 
		\forall g \in G \.
		g \neq e \Imply
		U \cap Ug = \emptyset
	}
	\\
	\Theorem{CoveringActionCovering}
	{
		\forall X : \TYPE{StronglyConnected} \. 
		\forall G \in \GRP \. 
		\forall \alpha : \TYPE{CoveringAction}(X,G) \. \NewLine \.
		X \Arrow{\pi} \frac{X}{\alpha} : \COV(X)
	}
	\AssumeIn{[x]}{\frac{X}{\alpha}}
	\Say{\Big(U',[1]\Big)}{\Elim \TYPE{CoveringAction}(X,G,\alpha)(x) }
	{
		\sum U' \in \U(x) \. \forall g \in G \.
		g \neq e \Imply U' \cap U'g = \emptyset
	}
	\Say{\Big( U'',[2]\Big)}{\Elim \TYPE{StronglyConnected}(X)(x,U')}
	{
		\sum U'' \in \U(x) \. \TYPE{StronglyConnected}(X) \And U'' \subset U'
	}
	\Say{U}{\pi U''}{?\frac{X}{\alpha}}
	\Say{[2]}{\Elim U \Intro x}{[x] \in U} 
	\Say{[3]}{\Elim U \Elim \FUNC{quotientByGroupAction}(X,G,\alpha)}
	{
		\pi^{-1}(U) = \bigcup_{g \in G} U''g
	}
	\Say{[5]}{[1][3]}
	{
		\pi^{-1}(U) = \bigsqcup_{g \in G} U''g
	}
	\Say{[6]}{\Elim \TYPE{HomeAction}(X,G,\alpha)[3]\Intro \T(X)}
	{
		\pi^{-1}(U) \in \T(X)
	}
	\Say{[7]}{[2]\Elim \FUNC{quotientTopology}[6]}{U \in \U[x]}
	\Say{[8]}{\Elim \TYPE{HomeoAction}(X,G,\alpha)\THM{CPreservesStronglyConnect}(U'')}
	{
		\forall g \in G \. \TYPE{StronglyConnected}(U''x)
	}
	\Conclude{\Big[[x].*\Big]}
	{
		\Intro \TYPE{EvenlyCovered}[6,7,8]
	}
	{
		\TYPE{EvenlyCovered}\left(X,\frac{X}{\alpha},\pi,U\right)
	}
	\DeriveConclude{[*]}{\Intro \TYPE{CoveringMap}}
	{
		\TYPE{CoveringMap}\left(X,\frac{X}{\alpha},\pi,U \right) 
	}
	\EndProof
}\Page{
	\Theorem{ActionCoveringIsNormal}
	{
		\forall X : \TYPE{StronglyConnected} \. 
		\forall G \in \GRP \. 
		\forall \alpha : \TYPE{CoveringAction}(X,G) \. \NewLine \.
		\TYPE{NormalCovering}\left(X,\frac{X}{\alpha}, \pi \right)
	}
	\AssumeIn{g}{G}
	\AssumeIn{x}{X}
	\Conclude{[x.*]}{\Elim \frac{X}{\alpha}(x)}{ \pi(xg) = \pi(x)}
	\DeriveConclude{[g.*] }{\Elim \Gal\left(X \Arrow{\pi} \frac{X}{\alpha} \right)}
	{g \in \Gal\left(X \Arrow{\pi} \frac{X}{\alpha}\right) }
	\Derive{[1]}{\Intro \TYPE{Subset}}{G \subset \Gal\left(X \Arrow{\pi} \frac{X}{\alpha}\right)}
	\Say{[2]}{\Elim \TYPE{Orbit}(\alpha)[1]\Intro \TYPE{Transitive}}
	{
		\TYPE{Transitive}\left(G,\Gal\left(X \Arrow{\pi} \frac{X}{\alpha}\right)\right)
	}
	\Conclude{[*]}{\THM{NormalCoveringHasTransitiveGal}[2]}
	{\TYPE{NormalCovering}\left(G,\frac{G}{\alpha},\pi \right)}
	\EndProof
	\\
	\Theorem{ActionCoveringDeckTransformationGroup}
	{
		\forall X : \TYPE{StronglyConnected} \. 
		\forall G \in \GRP \. \NewLine \. 
		\forall \alpha : \TYPE{CoveringAction}(X,G) \. 
		\Gal\left(X \Arrow{\pi} \frac{X}{\alpha} \right) = G
	}
	\NoProof
	\\
	\Theorem{ActionCoveringByDiscreteSubgroup}
	{
		\NewLine ::
		\forall G : \TYPE{StronglyConnected} \And \TYPE{TopologicaGroup} \.
		\forall H : \TYPE{DiscreteSubgroup}(G) \. 
		\TYPE{CoveringAction}(G,H,\cdot) 
	}
	\Say{\Big(W,[1]\Big)}
	{
		\Elim \TYPE{DiscreteSubgroup}(G,H)(e)
		\Elim\mathsf{TOPGRP}(G)
	}
	{
		\NewLine :
		\sum W \in \U(e) \. W \cap H = \{e\} \And \TYPE{Balanced}\Big(G,(-1,1)W\Big)
	}
	\AssumeIn{g}{G}
	\SayIn{U}{gW}{\U(g)}
	\Assume{h}{H}
	\AssumeIn{[2]}{h \neq e}
	\AssumeIn{[3]}{U \cap Uh \neq \emptyset }
	\Say{\Big([4]\Big)}{\Elim U}
	{
		gW = gWh
	}
	\Say{\Big(a,b,[5]\Big)}{[1][4]}
	{
		\sum a,b \in W \.ga = gbh
	}
	\Say{[6]}{b^{-1}g^{-1}[5]}
	{
		b^{-1}a = h
	}
	\Say{[7]}{\Elim \TYPE{Balanced}(G,W)(a)}
	{
		h \in W
	}
	\Conclude{[3.*]}{[1][2][7]\Intro(\bot)}
	{
		\bot
	}
	\DeriveConclude{[h.*]}{\Elim(\bot)}
	{
		U \cap Uh = \emptyset
	}
	\DeriveConclude{[*]}{\Intro \TYPE{CoveringAction}}
	{
		\TYPE{CoveringAction}(G,H,\cdot)
	}
	\EndProof
}\Page{
	\Theorem{CoveringOfGRP}
	{
		\NewLine ::
		\forall G,H : \TYPE{StronglyConnected} \And \TYPE{TopologicaGroup} \.
		\forall G \Arrow{\varphi} H : \mathsf{TOPGRP} \.
		\NewLine \. 
		\TYPE{Discrete}\Big( \ker \varphi \Big) \And \TYPE{Closed}(G,H,\varphi) \And \TYPE{Open}(G,H,\varphi)   \Imply
		\TYPE{CoveringMap}(G,H,\varphi) 
	}
	\NoProof
	\\
	\Theorem{CoveringClassification}
	{
		\NewLine ::
		\forall B : \TYPE{Reasonable} \. \NewLine \. 
		\TYPE{Bijection}\left( 
			\FUNC{Isoclass}\;\big(\COV(B)\Big) ,
			\frac{\TYPE{Subgroup}\;\pi(X)}{\Gamma},	
			\Big( 
				\Lambda 
				\big[X \Arrow{c} B\big] : \FUNC{Isoclass}\;\COV(B) \.  \big[ \pi(c)\big]_\Gamma
			\Big)
		\right)
	}
	\Say{F}{ \Lambda X \Arrow{c} B : \COV(B) \. \big[ \pi(c) \big]_\Gamma }
	{
		\frac{\TYPE{Subgroup}\; \pi(X)}
		{ \Gamma }
	}
	\Say{\Big(\hat F,[1]\Big)}
	{
		\Elim F \THM{CoveringIsomorphismCriterion}
	}
	{
		\sum \hat F : \FUNC{Isoclass}\;\Big( \COV(B) \Big)  \ToInj \frac{\TYPE{Subgroup}\;\pi(X)}{\Gamma} \.
		\NewLine : 
		\forall (X,c) \in \COV(B) \.
		\hat F [X,c] = F(X,c)
	}
	\Say{(Z,z)}{\THM{UniversalCoverExists}(B)}{\TYPE{UniversalCover}(B)}
	\Say{[2]}{\Elim \TYPE{UniversalCover}(B,Z,z)\THM{SimplyConnectedGroupStructure}(B,Z,z)}
	{
		\Gal(Z \Arrow{z} B) \cong_\GRP \pi(B)
	}
	\Say{\varphi}{\Elim \TYPE{Isomorphic}[2]}
	{
		\TYPE{Isomorphism}\Big(\GRP, \Gal(Z \Arrow{z} B), \pi(B) \Big)
	}
	\Assume{[H]}{\frac{\TYPE{Subgroup}\; \pi(X)}{\Gamma}}
	\Say{H'}{\varphi(H)}{\TYPE{Subgroup} \Gal(Z \Arrow{z} B)}
	\Say{[3]}{\Intro \TYPE{CoveringAction} }{ \TYPE{CoveringAction}\Big(Z,H',\FUNC{application}\Big)  }
	\SayIn{Q}{\frac{Z}{H'}}{\TOP}
	\Say{[4]}{\THM{ActionCoveringIsNormal}(Z,H')}{\TYPE{NormalCovering}\left(Z, Q, \pi_Q \right)}
	\Say{[5]}{ \Elim H' \Elim \Gal(Z \Arrow{z} B  ) }
	{
		\forall q \in Q \.  \Big| z\big(\pi^{-1}_Q(q) \big)\Big| = 1
	}
	\Say{\Big( \hat z, [6] \Big)}{
		[5]\THM{FiberMap}
	}
	{
		\sum_{ \hat z \in \TOP(Q,B ) }
		\forall p \in Z \. 
		z(p) = \hat z [p]
	}
	\AssumeIn{p}{B}
	\Say{\Big(U,[6]\Big)}{\Elim \TYPE{CoveringMap}(Z,B,z)}
	{
		\sum_{U \in \U(p)} \TYPE{EvenlyCovered}(Z,B,z,U)
	}
	\Say{[7]}{\Elim \FUNC{preimage}[6]}{\pi_Q^{-1}{\hat z}^{-1}(U) = z^{-1}(U)}
	\Say{\Big(W,[7.1]\Big)}{ \Elim \TYPE{EvenlyCoveres}(Z,V,z,U)  }
	{
		\sum W : \T(Z) \And \TYPE{StronglyConnected} \. 
		\NewLine \. 
		z^{-1}(U) = \bigsqcup_{g \in \Gal(Z \Arrow{z} B) } Wg \. 
		\forall g \in \Gal(Z \Arrow{z} B)  \.
		\TYPE{Homeomorphism}(Wg,U,z_{|Wg})
	}
	\AssumeIn{V}{\mathrm{PCC}(  {\hat z}^{-1}(U)  )}
	\Say{[8]}{\Elim \TYPE{LocallyPathConnected}(Q) \Elim \mathrm{PCC}}
	{
		\TYPE{Clopen}\Big(V, {\hat z}^{-1}(U) \Big) 
	}
	\Say{[9]}{\Elim \TOP(\pi_Q)[7][8]}
	{
		\TYPE{Clopen}\Big( \pi^{-1}(V), z^{-1}(U)  \Big)
	}
	\Say{\Big(g,[10]\Big)}{\Elim \TYPE{Clopen}[9][7.1]}
	{
		\sum g \in \Gal(Z \Arrow{z} B) \. \pi^{-1}(V) = \bigsqcup_{h \in H'} Wgh
	}
	\Conclude{[p.*]}{[10][7.2][5]}{\TYPE{Homeomorphism}(V,U,{\hat z}_{|V})}
	\DeriveConclude{[6]}{\Intro \TYPE{CoveringMap}}
	{
		\TYPE{CoverigMap}(Q,B,\hat z)	
	}
}\Page{
	\AssumeIn{p}{B}
	\AssumeIn{q}{{\hat z}^{-1}(p)}
	\Say{[7]}{\THM{StabilizerOfMonodromyAction}(\hat z,p,q)}
	{
		\Stab_{\action_{\hat z,p}}(q) = \pi(\hat z)
	}
	\Say{[8]}{\Elim \hat z [7]}{H \subset \pi(\hat z) }
	\Say{\Big( u, [9] \Big)}{\Elim Q(q)}{\sum u \in Z \. q = [u]} 
	\AssumeIn{\gamma}{\pi(\hat z)}
	\Say{[10]}{[8](\gamma)[9]\Elim \COV(B)(Z,Q)(\pi)\Intro \varphi}{ 
		q =
		q \gamma =  
		[u] \gamma =  
		[u\gamma] = 
		\Big[ \varphi(\gamma)(u)\Big]
	}
	\Conclude{[\gamma.*]}{\Elim Q [10]}{ \gamma \in H}
	\Derive{[10]}{\Intro \TYPE{Subset}}{H \subset \pi(\hat z) }
	\Conclude{[p.*]}{\Intro \TYPE{SubsetEq}[8][10]}{ H = \pi(\hat z)  }
	\Derive{[7]}{\Elim \TYPE{NonEmpty}(B)}{H = \pi(\hat z)}
	\Conclude{[H.*]}{\Intro \hat F [7]}{H = \hat F( \hat z)}
	\Derive{[3]}{\Intro \TYPE{Surjective}}{\TYPE{Surjective}(\hat F)}
	\Conclude{[*]}{\Intro \TYPE{Bijective}[1][3]}{\TYPE{Bijective}(\hat F)}
	\EndProof
	\\
	\Theorem{HausdorffActionQuotientCriterion}
	{
		\forall X \in \TOP \.
		\forall G \in \GRP \.
		\forall \alpha : X \action_\TOP G \.
		\TYPE{T2}\left( \frac{X}{\alpha} \right) \iff \NewLine \iff
		\bigg(
			\forall x,y \in X \. y \not \in O_\alpha(x) \Imply
			\Big(	
				\exists U \in \U(x) :
				\exists V \in \U(y) :
				\forall g \in G
				U \cap Vg = \emptyset 
			\Big)
		\bigg)
	}
	\Assume{[1]}{\TYPE{T2}\left(\frac{X}{\alpha}\right)}
	\AssumeIn{x,y}{X}
	\Assume{[2]}{y \not \in O_\alpha(x)}
	\Say{[3]}{\Elim \frac{X}{\alpha}[2]}{[x]_\alpha \neq [y]_\alpha}
	\Say{\Big(U,V,[4]\Big)}{\Elim \TYPE{T3}\left(\frac{X}{\alpha}\right)\Big([x],[y]\Big)}
	{
		\sum U \in \U[x] \. 
		\sum V \in \U[x] \. 
		V \cap U = \emptyset
	}
	\Say{U'}{\pi^{-1}_\alpha(U)}{\U(x)}
	\Say{V'}{\pi^{-1}_\alpha(V)}{\U(x)}
	\Say{[5]}{\Elim U' \Elim V' \THM{DisjointPreimage}[4]}
	{
		U' \cap V' = \emptyset
	}
	\Say{[6]}{\Elim V' \Elim \pi_\alpha}{\forall g \in G \. V'g = V'}
	\Conclude{[1.*]}{\forall g \in G \. \Elim\Big(=,[6](g),[5]\Big)}
	{
		\forall g \in G \.
		U' \cap V'g = \emptyset
	}
	\Derive{[1]}{\Intro(\Imply)}
	{
		\TYPE{T2}\left( \frac{X}{\alpha} \right) \Imply
		\bigg(
			\forall x,y \in X \. y \not \in O_\alpha(x) \Imply
			\Big(	
				\exists U \in \U(x) :
				\exists V \in \U(y) :
				\forall g \in G
				U \cap Vg = \emptyset 
			\Big)
		\bigg)
	}
}\Page{
	\Assume{[2]}
	{
			\forall x,y \in X \. y \not \in O_\alpha(x) \Imply
			\Big(	
				\exists U \in \U(x) :
				\exists V \in \U(y) :
				\forall g \in G
				U \cap Vg = \emptyset 
			\Big)
	}
	\Assume{[x],[y]}{\frac{X}{\alpha}}
	\Assume{[3]}{\forall [x] \neq [y]}
	\Say{[4]}{\Elim \pi_\alpha [3] \Intro O_\alpha}
	{
		y \not \in O_\alpha(x)
	}
	\Say{\Big( U,V,[5]\Big)}{[2]\Big( x, y,[4]\Big)}
	{
		\sum_{U \in \U(x)} \sum_{V \in \U(y)} U \cap Vg = \emptyset
	}
	\SayIn{U'}{\bigcap_{g \in G} Ug}{\U(x)}
	\SayIn{V'}{\bigcap_{g \in G} Vg}{\U(y)}
	\Say{[6]}{\Elim U' \Elim \GRP(G)}{ \forall g \in G \. U'g = U'}
	\Say{[7]}{\Elim V' \Elim \GRP(G)}{\forall g \in G \. V'g = V'}
	\Say{[8]}{\Elim U' \Elim V' [5]}{U' \cap V'}
	\Say{[9]}{\Elim \pi_\alpha [6]}{\pi^{-1}\pi(U') = U'}
	\Say{[10]}{\Elim \TYPE{QuotinetMap}[9]}{\pi(U') \in \U[x]}
	\Say{[11]}{\Elim \pi_\alpha [7]}{\pi^{-1}\pi(V') = V'}
	\Say{[12]}{\Elim \TYPE{QuotinetMap}[9]}{\pi(V') \in \U[y]}
	\Conclude{\bigg[\Big([x],[y]\Big).*\bigg]}
	{
		\Elim \pi_\alpha [6][7][8] \Intro \pi_\alpha
	}
	{
		\pi(V') \cap \pi(U') = \emptyset
	}
	\DeriveConclude{[2.*]}{\Intro \TYPE{T2}}
	{
		\TYPE{T2}\left(\frac{X}{\alpha}\right) 
	}
	\Derive{[2]}{\Intro(\Imply)}
	{
		\bigg(
			\forall x,y \in X \. y \not \in O_\alpha(x) \Imply
			\Big(	
				\exists U \in \U(x) :
				\exists V \in \U(y) :
				\forall g \in G
				U \cap Vg = \emptyset 
			\Big)
		\bigg) \Imply
		\TYPE{T2}\left( \frac{X}{\alpha} \right)
	}
	\Conclude{[3]}{\Intro(\iff)[1][2]}
	{
		\bigg(
			\forall x,y \in X \. y \not \in O_\alpha(x) \Imply
			\Big(	
				\exists U \in \U(x) :
				\exists V \in \U(y) :
				\forall g \in G
				U \cap Vg = \emptyset 
			\Big)
		\bigg) \iff \NewLine \iff
		\TYPE{T2}\left( \frac{X}{\alpha} \right)
	}
	\EndProof
	\\
	\DeclareType{ProperAction}{
		\prod X \in \TOP \. 
		\prod G \in \mathsf{TOPGRP} \.
		?(X \action_\TOP G) 
	}
	\DefineType{\alpha}{ProperAction}
	{ \TYPE{ProperMap}(X \times G,X^2, \Lambda (x,g) \in X \times G \. (x,xg)) }
}\Page{
	\Theorem{ProperActionCriterion}
	{
		\forall X : \TYPE{T2} \.
		\forall G  \in \mathsf{TOPGR} \And \TYPE{Compact}  \.
		\forall \alpha : X \action_\TOP G
		\TYPE{ProperAction}(X,G,\alpha)
	}
	\Say{\theta}{\Lambda (g,x) \in G \times X \. (xg,x)}{\TOP(G \times X, X^2 )}
	\Assume{K}{\TYPE{CompactSubset}(X \times X)}
	\Say{[1]}{\THM{CompactImage}(K,\pi_2)}{\TYPE{CompactSubset}(X,\pi_2 K)}
	\Say{[2]}{\THM{T2CompactIsClosed}(X^2,K)}{\TYPE{Closed}(X^2,K)}
	\Say{[3]}{\Elim \TOP(G \times X, X^2)[2] }
	{
		\TYPE{Closed}\Big(G\times X, \theta^{-1}(K) \Big)
	}
	\Say{[4]}{\Elim \theta \Big( \theta^{-1}(K) \Big)}
	{
		\theta^{-1}(K) \subset G \times K
	}
	\Say{[5]}{\THM{TychonoffTHM}(G,K)}{\TYPE{Compact}(G \times K)}
	\Say{[6]}{\THM{ClosedSubset}\big(G\times X, G \times K,[3]\big)}
	{
		\TYPE{Closed}\Big(G \times K,\theta^{-1}(K)\Big)
	}
	\Say{[7]}{\THM{ClosedCompactSubset}[6]}
	{
		\TYPE{CompactSubset}\Big( G \times K, \theta^{-1}(K)\Big)
	}
	\Conclude{[K.*]}{\THM{ComapcCompactSubset}[5][7]}
	{
		\TYPE{CompactSubset}\big( G \times X, \theta^{-1}(K)\Big)
	}
	\DeriveConclude{[*]}{\Intro \TYPE{ProperAction}}
	{
		\TYPE{ProperAction}\big( G , X \big)
	}
	\EndProof
}\Page{
	\Theorem{ProperActionbyCompactOrbit}
	{
		\forall X : \TYPE{T2} \.
		\forall G  \in \mathsf{TOPGR}   \.
		\forall \alpha : X \action_\TOP G \.
		\TYPE{ProperAction}(X,G,\alpha)
		\iff
		\NewLine
		\iff
		\bigg(
			\forall K : \TYPE{CompactSubset}(X) \.
			\TYPE{CompactSubset}\Big( G, \{ g \in G : K \cap gK \neq \emptyset  \} \Big)
		\bigg)
	}
	\Say{\theta}{\Lambda (g,x) \in G \times X \. (xg,x)}{\TOP(G \times X, X^2 )}
	\Assume{[1]}{\TYPE{ProperAction}(X,G,\alpha)}
	\Assume{K}{\TYPE{CompactSubset}(X)}
	\Say{[2]}{\THM{TychonoffTHM}(K,K)}{\TYPE{CompactSubset}(X^2,K^2)}
	\Say{[3]}{\Elim \theta \Intro \FUNC{setBuilder}}
	{
		\theta^{-1}(K^2) = \Big\{ (g,x) \in G \times K : gx \in K \Big\}
	}
	\Say{[4]}{\Elim \TYPE{ProperAction}(X,G,\alpha)}{
		\TYPE{ComapactSubset}\Big(G \times X, \theta^{-1}(K^2)\Big)
	}
	\Conclude{[1.*]}{\THM{CompactImage}[4][3]}
	{
			\TYPE{CompactSubset}\Big( G, \{ g \in G : K \cap gK \neq \emptyset  \} \Big)	
	}
	\Derive{[1]}{\Intro \forall \Intro \Imply}
	{
		\TYPE{ProperAction}(X,G,\alpha)
		\Imply
		\NewLine
		\Imply
		\bigg(
			\forall K : \TYPE{CompactSubset}(X) \.
			\TYPE{CompactSubset}\Big( G, \{ g \in G : K \cap gK \neq \emptyset  \} \Big)
		\bigg)	
	}
	\Assume{[2]}
	{
			\forall K : \TYPE{CompactSubset}(X) \.
			\TYPE{CompactSubset}\Big( G, \{ g \in G : K \cap gK \ne	 \emptyset  \} \Big)	
	}
	\Assume{K}{\TYPE{CompactSubset}(X \times X)}
	\Say{[3]}{\THM{CompactImage}(K,\pi_2)}{\TYPE{CompactSubset}(X,\pi_2 K)}
	\Say{[4]}{\THM{CompactImage}(K,\pi_1)}{\TYPE{CompactSubset}(X,\pi_1 K)}
	\Say{L}{\pi_1 K \cap \pi_2 K }{\TYPE{CompactSubset}(X)}
	\Say{H}{ \{ g \in G : K \cap gK = \emptyset \}    }{\TYPE{CompacSubset}(G)}
	\Say{[5]}{\THM{PreimageSubset} \Elim \theta \Intro H}
	{
		\theta^{-1}(K) \subset \theta^{-1}(L \times L) = 
		\Big\{
			(g,x) \in G \times X \.
			gx \in L
		\Big\}
		\subset H \times L
	}
	\Say{[6]}{\THM{TychonoffTHM}(H,L)}{\TYPE{CompactSubset}(H\times L,G \times X)}
	\Say{[7]}{\THM{CompactClosedSubset}[5]}
	{	
		\TYPE{CompactSubset}\Big( H \times L  ,\theta^{-1}(K) \Big)
	}
	\Conclude{[K.*]}{\THM{CompactCompactSubset}[7][6]}
	{
		\TYPE{CompactSubset}\big( G \times X, \theta^{-1}(K) \Big)
	}
	\DeriveConclude{[2.*]}{\Intro \TYPE{ProperAction}}{ \TYPE{ProperAction}(X,G,\alpha)}
	\Derive{[2]}{ \Intro \Imply}
	{
		\bigg(
			\forall K : \TYPE{CompactSubset}(X) \.
			\TYPE{CompactSubset}\Big( G, \{ g \in G : K \cap gK \neq \emptyset  \} \Big)
		\bigg)	
		\Imply
		\NewLine
		\Imply
		\TYPE{ProperAction}(X,G,\alpha)
	}
	\Conclude{[*]}{\Intro \iff[1][2]}
	{
		\TYPE{ProperAction}(X,G,\alpha)
		\iff
		\NewLine
		\iff
		\bigg(
			\forall K : \TYPE{CompactSubset}(X) \.
			\TYPE{CompactSubset}\Big( G, \{ g \in G : K \cap gK \neq \emptyset  \} \Big)
		\bigg)	
	}
}\Page{
	\Theorem{HausdorffByProperAction}
	{
		\forall X : \TYPE{T2} \And \TYPE{LocallyCompact} \.
		\forall G  \in \mathsf{TOPGR}   \.
		\forall \alpha : \TYPE{ProperAction}(X,G) \. \NewLine 
		\TYPE{T2}\left( \frac{X}{\alpha}\right)
	}
	\Say{\theta}{\Lambda (g,x) \in G \times X \. (xg,x)}{\Type{ProperMap}(G \times X, X^2 )}
	\Say{[1]}{\THM{HausdorffProduct}(X,X) \And \TYPE{LocallyCompactProduct}(X,X)}
	{
		\TYPE{T2} \And \TYPE{LocallyCompact}(X^2)
	}
	\Say{[2]}{\THM{EmbeddingProperIffClosed}(\theta)[1]}
	{
		\TYPE{ClosedMap}(G\times X, X^2,\theta)
	}
	\Say{[3]}{\Elim \TYPE{ClosedMap}(G\times X,X^2,\theta)(G \times X)}
	{
		\TYPE{Closed}\Big( X^2, \theta(G \times X) \Big) 
	}
	\Conclude{[*]}{\THM{T2ByClosedOrbitRelation}[3]}
	{
		\TYPE{T2}\left( \frac{X}{\alpha} \right)
	}
	\EndProof
}\Page{
	\Theorem{ProperByDiscreteAction}
	{
		\NewLine ::
		\forall X \in \TOP \.
		\forall G \in \GRP \.  
		\forall \alpha : \TYPE{CoveringAction} \.
		\TYPE{T2}\left(\frac{X}{G}\right) \Imply \TYPE{ProperAction}(X,G,\alpha)
	}
	\SayIn{Q}{\frac{X}{G}}{\TOP}
	\Say{[1]}{\THM{ActionCoveringIsNormal}(X,G,\alpha)}{\TYPE{NormalCovering}(Q,X,\pi_Q)}
	\Say{\O}{\{ (x,xg) | x \in X, g \in G \}}{?(X \times X)}
	\Say{[2]}{\THM{HausdorffIfRelationIsClosed}(X,G)}{\TYPE{Closed}(X,G)}
	\Say{[3]}{\THM{HausdorffByCovering}[1]}{\TYPE{T2}(X)}
	\Assume{K}{\TYPE{CompactSubset}(X \times X)}
	\Say{H}{  \{ g \in G : K \cap gK \neq \emptyset  \} }{?G}
	\Assume{[4]}{\IsNot \TYPE{CompactSubset}\Big( G, H \Big)  }
	\Say{[5]}{\Elim \TYPE{DiscreteGroup}(G)[4]}{|H|=\infty}
	\AssumeIn{g}{H}
	\Say{\Big(x_g,[6]\Big)}{\Elim H(g)}{\sum x \in K \. xg \in K}
	\Conclude{F(g)}{(x_gg,x_g)}{K \times K}
	\Derive{F}{\Intro(\to)}{H \to K \times K}
	\Say{[6]}{\Elim \TYPE{Free}(X,G,\alpha)\Intro F}
	{
		\TYPE{Injective}( H,K \times K, F )	
	}
	\Say{[7]}
	{
		[5][6]
	}
	{
		\Big| F(H) \Big| = \infty	
	}
	\Say{ (x,y)}
	{
		\THM{LimitCompact}(K \times K)[7]
	}
	{
		\TYPE{LimitPoint}\Big( F(H) \Big)
	}
	\Say{[8]}{\Elim F(H) \Elim \O \Intro \TYPE{Subset} \Intro F(H) \Intro \O}{F(X) \subset \O}
	\Say{[9]}{\THM{ClosedLimit}[8](x,y)}{(x,y) \in \O}
	\Say{\Big(g, [10] \Big)}
	{  
		\Elim \O [9]
	}
	{ 
		\sum g \in G \. x = yg
	}
	\SayIn{\Big(U,[11]\Big)}{\THM{HausdorffByGroupActionQuotientCriterion}}{\sum_{U \in \U(y)} \forall g \in G \. gU \cap U = \emptyset}
	\SayIn{V}{Vg}{\U(x)}
	\Say{[12]}{\Elim \TYPE{LimitPoint}(x,y)(U \times V)}{\Big| U \times V \cap F(H) \Big| = \infty}
	\AssumeIn{h}{H}
	\Assume{[13]}{F(h) \in V \times U}
	\SayIn{p}{\pi_2 F(h)}{U}
	\Say{[14]}{(\Elim U \Elim F) (p)}{  p g = p h  }
	\Say{[15]}{ \Elim \TYPE{Free} \alpha [14] }{g = h}
	\DeriveConclude{[13]}{\Intro \mathsf{CARD}}{|H| = 1}
	\Conclude{[14]}{\Intro \bot [13][5]}{\bot}
	\DeriveConclude{[4]}{\Elim(\bot)}{\TYPE{CompactSubset}(G,H)}
	\Conclude{[*]}{\THM{ProperActioByCompactOrbit}[4]}{\TYPE{ProperAction}(X,G,\alpha)}
	\EndProof
}\Page{
	\Theorem{RegularCoveringAction}
	{
		\forall X : \TYPE{StronglyConnected} \And \TYPE{LocallyCompact} \And \TYPE{T2} \.
		\forall G : \TYPE{DiscreteGroup} \. \NewLine \.
		\forall \alpha : \TYPE{ProperAction} \And \TYPE{Free}(X,G) \.
		\TYPE{CoveringAction}(X,G,\alpha)
		\And
		\TYPE{T2}\left( \frac{X}{G} \right) 
		\And \NewLine \And
		\TYPE{NormalCovering}\left( X, \frac{X}{G}, \pi \right)
	}
	\AssumeIn{p}{X}
	\Say{\Big(V,[1]\Big) }{\Elim \TYPE{LocallyComapct}(X,p)}
	{
		\sum V \in \U(p) \. \TYPE{CompactSubset}(X,\overline{V})
	}
	\Say{K}{\overline{V}}{\TYPE{CompactSubset}(X)}
	\Say{H}{  \{ g \in G : K \cap gK \neq \emptyset  \} }{\TYPE{CompactSubset}(G)}
	\Say{[2]}{\Elim \TYPE{DiscreteGroup}(G)(H)}{|H| < \infty}
	\SayIn{m}{|H|}{\Nat}
	\Say{h}{\FUNC{enumerate}(H)}{[1,\ldots,m] \ToBij H }
	\Say{[3]}{\Elim \TYPE{Free}(X,G,\alpha)(p)}{\forall g \in G \. pg = p \iff g = e}
	\Say{\Big( W, W', [4] \Big)}{\Elim \TYPE{T2}[3](p,ph)}
	{
		\prod^m_{i=1} \sum_{W_i \in \U(p)} \sum_{W_i' \in \U(ph_i)} W_i \cap W'_i = \emptyset 
	}
	\SayIn{U}{V \cap \bigcap^m_{i=1} W_i \cap W'_i h_i^{-1}}{\U(p)}
	\AssumeIn{i}{[1,\ldots,m]}
	\AssumeIn{u}{U}
	\Say{[5]}{\Elim U(u) \Intro W'_i h_i^{-1}}{u \in W'_i h_i^{-1} }
	\Say{[6]}{[5]h_i}{uh_i \in W'_i}
	\Conclude{[i.*]}{[4][6]}{uh_i \not \in U}
	\Derive{[4]}{\Intro \TYPE{Disjoint} \Intro \forall }{\forall i \in [1,\ldots, m] \. Uh_i \cap U = \emptyset }
	\Assume{g}{H^\c}
	\Assume{[5]}{g \neq e}
	\AssumeIn{u}{U}
	\Say{[6]}{\Elim U(u) [1]}{ug \in Kg}
	\Say{[7]}{\Elim H^\c (g)}{Kg \cap K = \emptyset}
	\Conclude{[g.*]}{\Intro U [6][7]}{ ug \not \in U }
	\Derive{[5]}{\Intro \TYPE{Disjoint} \Intro(\Imply) \Intro(\forall)}
	{ \forall g \in H^\c \. g \neq e \Imply Ug \cap U = \emptyset }
	\Conclude{[p.*]}{  [4][5]   }{ \forall G \in G^\c \. g \neq e \Imply Ug \cap U = \emptyset }
	\Derive{[1]}{\THM{HausdorffByGroupQuotientCriterion}}
	{
		\TYPE{T2}\left( \frac{X}{\alpha} \right)
	}
	\Conclude{[*]}{\THM{ProperByDiscreteAction}[1]}
	{
		\TYPE{ProperAction}(X,G,\alpha)
	}
	\EndProof
	\\
	\Theorem{ManifoldByProperAction}
	{
		\forall X \in \TOPM \.
		\forall G : \TYPE{DiscreteGroup} \. \NewLine \.
		\forall \alpha : \TYPE{ProperAction} \And \TYPE{Free}(X,G) \.
		\frac{X}{G} \in \TOPM
	}
	\NoProof
}
\newpage
\subsection{Applications to Geometic Topology}
\Page{
	\Theorem{SphereCoversProjectiveSpace}
	{
		\forall n \in \Nat \. 
		(\Sphere^n,\pi) \in \COV(\Reals \P^n)
	}
	\AssumeIn{p}{\Reals \P^n}
	\Say{\Big(e,[1]\Big)}{\THM{ProjectiveCoordinatesExists}(n,p)}
	{
		\sum e : \TYPE{ProjectiveCooedinates}(\Reals,n) \. 
		p_e = [1,0,\ldots,0]
	}
	\SayIn{U}{ \Big\{ q \in \Reals\P^n \Big| p_e^1 \neq 0 \Big\}  }
	{
		\U(p)
	}
	\Say{V_+}{\Big\{ x \in \Sphere^n \Big| x_e^1 > 0 \Big\}}{\T(\Sphere^1) \And \TYPE{StronglyConnected}}
	\Say{V_-}{\Big\{ x \in \Sphere^n \Big| x_e^1 < 0 \Big\}}{\T(\Sphere^1) \And \TYPE{StronglyConnected}}
	\Say{[2]}{\Elim U\Intro V_+ \Intro V_-}{\pi^{-1}(U) = V_- \sqcup V_+}
	\Say{[*.1]}{\Elim \pi \Intro V_+ \Intro U}{\TYPE{Homeomorphism}\Big( U, V_+, \pi_{|V_+} \Big)}
	\Conclude{[*.2]}{\Elim \pi \Intro V_- \Intro U}{\TYPE{Homeomorphism}\Big( U, V_-, \pi_{|V_-} \Big)}
	\DeriveConclude{[*]}{\Intro \TYPE{CoveringMap}}{\TYPE{CoveringMap}(\Sphere^n,\Reals \P^n,\pi)} 
	\EndProof
	\\
	\DeclareFunc{complexSquareRootSpace}{ ?\Complex^2 }
	\DefineNamedFunc{complexSquareRootSpace}{}{\sqrt{\Complex}}
	{
		\Big\{ (z,w) \in \Complex^2 \Big| z \neq 0, z = w^2\Big\}
	}
	\\
	\Theorem{ComplexSquareRootCovers}
	{
		(\sqrt{\Complex},\pi_1) \in \COV(\Complex \setminus \{0\})
	}
	\Assume{p}{\Complex \setminus \{0\}}
	\SayIn{A}{\If p \in \Reals_{--} \Then \Im \Else \Re}{\Complex \to \Reals}
	\SayIn{U}
	{
		\If p \in \Reals_{--} 
		\Then
			\{
				z \in \Complex \setminus \{0\} : z \not \in \Reals_{++}
			\}
		\Else
			\{
				z \in \Complex \setminus \{0\} : z \not \in \Reals_{--}
			\}
	}
	{
		\U\Big(\Complex \setminus \{0\}\Big)
	}
	\Say{[1]}{\Elim U \Elim A}{ 
		\forall (u,v) \in \pi^{-1}_1 (U) \. 
		A(v) \neq 0
	}
	\Say{V_+}{\Big\{ (u,v) \in \sqrt{\Complex} \Big| A(v) > 0 \Big\}}{\T\sqrt{\Complex} \And \TYPE{StronglyConnected}}
	\Say{V_-}{\Big\{ x \in \sqrt{\Complex} \Big| A(v) < 0 \Big\}}{\T{\sqrt{\Complex}} \And \TYPE{StronglyConnected}}
	\Say{[2]}{\Elim U \Elim V_+ \Elim V_- \Elim A}
	{
			\pi^{-1}_1 U = V_+ \cup V_-
	}
	\Say{[*.1]}{\Elim \pi \Intro V_+ \Intro U}{\TYPE{Homeomorphism}\Big( U, V_+, \pi_{|V_+} \Big)}
	\Conclude{[*.2]}{\Elim \pi \Intro V_- \Intro U}{\TYPE{Homeomorphism}\Big( U, V_-, \pi_{|V_-} \Big)}
	\DeriveConclude{[*]}{\Intro \TYPE{CoveringMap}}{\TYPE{CoveringMap}\Big(\sqrt{\Complex},\Complex \setminus \{0\},\pi_1\Big)} 
	\EndProof
}\Page{
	\Theorem{TorusCoversKleinBottel}
	{
		\exists c : \TYPE{CoveringMap}(\mathbb{T}^2,\mathbf{KB}) : 
		\mathrm{num}\;c = 2
	}
	\Say{[1]}{\THM{TorusAsGroup}}
	{
		\mathbb{T}^2 = \frac{\Reals^2}{\Int^2}
	}
	\Say{[2]}{[1] \Intro \mathbf{KB}}{  \frac{\mathbb{T}^2}{ [s,t] \sim [s + 1/2,1-t]} \cong_\TOP \mathbf{KB}  }
	\AssumeIn{p}{\mathbf{KB}^2}
	\Assume{[3]}{\forall t \in I \. p \neq [0,t]}
	\SayIn{U}{(0,1/2) \times I }{\U(p)} 
	\Say{V_+}{\left\{ [a,b] \in \mathbb{T} \Big| a > \frac{1}{2} \right\}}{\T\sqrt{\Complex} \And \TYPE{StronglyConnected}}
	\Say{V_-}{\left\{ [a,b] \in \mathbb{T} \Big| a < \frac{1}{2} \right\}}{\T{\sqrt{\Complex}} \And \TYPE{StronglyConnected}}
	\Say{[3.*.1]}{\Elim \pi \Intro V_+ \Intro U}{\TYPE{Homeomorphism}\Big( U, V_+, \pi_{|V_+} \Big)}
	\Conclude{[3.*.2]}{\Elim \pi \Intro V_- \Intro U}{\TYPE{Homeomorphism}\Big( U, V_-, \pi_{|V_-} \Big)}
	\Derive{[3]}{\Intro(\Imply)}
	{
		\Big(\forall t \in I \. p \neq [0,t] \Big) 
		\Imply 
		\exists \TYPE{EvenlyCovered}(\mathbb{T}^2,\mathbf{KB},\pi) 
	}
	\AssumeIn{t}{I}
	\Assume{[4]}{p = [0,t]}
	\SayIn{U}{\Big([0,1/4)\sqcup (1/4,1/2) \Big) \times I }{\U(p)} 
	\Say{V_+}{\left\{ [a,b] \in \mathbb{T} \Big| \frac{1}{4} < a < \frac{3}{4} \right\}}
	{\T\sqrt{\Complex} \And \TYPE{StronglyConnected}}
	\Say{V_-}{\left\{ [a,b] \in \mathbb{T} \Big| a < \frac{1}{4} \vee a > \frac{3}{4} \right\}}
	{\T{\sqrt{\Complex}} \And \TYPE{StronglyConnected}}
	\Say{[4.*.1]}{\Elim \pi \Intro V_+ \Intro U}{\TYPE{Homeomorphism}\Big( U, V_+, \pi_{|V_+} \Big)}
	\Conclude{[4.*.2]}{\Elim \pi \Intro V_- \Intro U}{\TYPE{Homeomorphism}\Big( U, V_-, \pi_{|V_-} \Big)}
	\Derive{[4]}{\Intro(\Imply)}{
		\Big(\forall t \in I \. p \neq [0,t] \Big) \Imply 
		\exists \TYPE{EvenlyCovered}\Big(\mathbb{T}^2,\mathbf{KB},\pi\Big) 
	}
	\Conclude{[p.*]}{\Elim(|)\LOGIC{LEM}[3][4] }{\exists \TYPE{EvenlyCovered}\Big(\mathbb{T}^2,\mathbf{KB},\pi \Big)}
	\DeriveConclude{[3]}{\Intro \TYPE{CoveringMap}}{\TYPE{CoveringMap}\Big(\mathbb{T}^2,\mathbf{KB},\pi\Big)} 
	\Conclude{[4]}{\Intro\;\mathrm{num}[2][3]}
	{
		\mathrm{num}\;\pi = 2
	}
	\EndProof
	\\
	\Theorem{CoveringOfConnectedSum}
	{
		\forall B,D \in \TOPM(n) \And \TYPE{Connected} \.
		\forall c : \TYPE{CoveringMap}(X,B) \.
		\forall k \in \Nat \.\NewLine \.
		\forall [0] : \mathrm{num} a = k \.  
		\exists c' : \TYPE{Covering}\left( X \# \bigsum^k_{i=1} D , B \# D \right) \.
		\mathrm{num}\; c' = k
	}
	\NoProof
	\\
	\Theorem{OriantableCoversNonoriantable}
	{
		\forall M : \TYPE{Nonoriantable} \.
		\exists N : \TYPE{Oriantable} : \NewLine :
		\forall c : \TYPE{CoveringMap}(N,M) :
		\mathrm{num}\; c = 2
		\And
		\mathrm{gen}\; M = 1 + \mathrm{gen}\;N
	}
	\NoProof
}
\Page{
	\DeclareFunc{toriParametricCovering}{\Big(\Int^2 \setminus \{0\}\Big)^2 \to\TYPE{NormalCovering}(\mathbb{T}^2,\mathbb{T}^2)}
	\DefineNamedFunc{toriParametricCovering}{a,b}
	{
		\tau_{a,b}}{\Lambda (u,v) \in \mathbb{T}^2 \. \Big(u^{a_1}v^{a_2},u^{b_1}v^{b_2}\Big) 
	}
	\\
	\Theorem{ClassificationOfToriCoverings}
	{
		\forall (X,c) \in \COV(\Torus^2) \.
		X = \Reals^2 \And c = s \times s \Big| \NewLine \Big|
		X = \Reals \times \Sphere^1 \And 
		\exists a,b \in \Int^2 \. c = \tau_{a,b} \circ (s \times \id) \Big| \NewLine \Big|
		X = \Torus^2 \And \exists a,b \in \Int^2 \. c = \tau_{a,b}
	}
	\NoProof
	\\
	\DeclareFunc{spaceOfLens}{\TYPE{Coprime} \to \TOPM(3)}
	\DefineNamedFunc{spaceOfLence}{n,m}{\mathbb{L}_{n,m}}
	{
		\frac{\Sphere^3}{\alpha} 
		\quad
		\where
		\NewLine
		\quad
		\alpha = \Lambda (z_1,z_2) \in \Sphere^3 \. 
		\Lambda k \in \frac{\Int}{n\Int} \.
		\left( \exp\left( \frac{2\uppi\mathrm{i}k}{n} \right) z_1, 
			\exp\left( \frac{2\uppi\mathrm{i}km}{n} \right) z_2
		\right)
	}
}
\newpage
\subsection{Hyperbolic Cells and Presentation}
\end{document}

